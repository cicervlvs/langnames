% \iffalse meta-comment
%
% Copyright (C) 2022 by Alejandro García Matarredona
%
% This file may be distributed and/or modified under the
% conditions of the LaTeX Project Public License, either
% version 1.3 of this license or (at your option) any later
% version. The latest version of this license is in:
%
% http://www.latex-project.org/lppl.txt
%
% and version 1.3 or later is part of all distributions of
% LaTeX version 2005/12/01 or later.
%
% \fi
% \iffalse
%<package>\NeedsTeXFormat{LaTeX2e}[2005/12/01]
%<package>\ProvidesPackage{langnames}
%<package> [2022/08/25 v1.0 langnames package for naming and classification of languages]
%
%<*driver>
\documentclass{ltxdoc}
\usepackage{langnames}
\usepackage{expex}
\gathertags
\usepackage{xkeyval}
\usepackage{hyperref}
\EnableCrossrefs
\CodelineIndex
\RecordChanges
\begin{document}
\DocInput{langnames.dtx}
\end{document}
%</driver>
% \fi
% \CheckSum{30}
% \changes{v1.0}{2022/08/25}{Initial version}
% \GetFileInfo{langnames.sty}
% \DoNotIndex{\unskip}
% \title{The \textsf{langnames} package\thanks{This document
% corresponds to \textsf{langnames}~\fileversion,
% dated~\filedate.}}
% \author{Alejandro García Matarredona\\ \texttt{alejandrogarciaag41@gmail.com}}
% \maketitle
% \begin{abstract}
% \noindent The \textsf{langnames} package provides a set of macros for formatting names of languages, as well as their identification (in the form of ISO 639-3 codes) and their classification (in the form of its top-level family). The datasets from \href{https://wals.info}{WALS} and \href{https://glottolog.org}{Glottolog} are included in the package. The package also allows users to rename and add new languages.
% \end{abstract}
% \section{Introduction}
% The typing out of language names in academic papers, especially those in language typology or related fields where many names have to be typed many times, is often inconvenient and inconsistent. This package attempts to be a small help to writers, especially of large projects or of collaborative ones, to have a slightly easier time with names of languages. It does so by defining three main commands: |\lname|, |\liso|, and |\lfam|, which respectively print out the name, name and ISO 639-3 code, and name and family of the specified language. While the package comes with about 7500 pre-defined languages, with code, name, and family, the user may also define new ones through the |\newlang| command. The basic use of all four of these commands is explained below.
% \section{Usage}
% \subsection{Installation}
% Download the package from wherever it was found to a place where \LaTeX may see it, typically in \$TEXMFHOME/tex/latex. \textsf{langnames} should automatically load the \textsf{xkeyval} package.
% \subsection{Package options}
% When calling |\usepackage{langnames}|, the user must specify one of three options: \textsf{glottolog}, \textsf{wals}, or \textsf{none}.
% The first option, \textsf{glottolog}, selects the naming conventions from the \href{https://glottolog.org}{Glottolog} database. The second option, \textsf{wals}, predictably selects the naming conventions of the \href{https://wals.info}{WALS} database. The names and the genetic classification differ in some languages, so the user may choose what convention to follow.
% During the preparation of the dataset, there were instances of languages which appeared in WALS but not in Glottolog, and vice-versa. In such cases, the missing information was added from the other database. For more details on how I built the dataset, one may consult the Python script made for it in the \href{https://github.com/cicervlvs/langnames}{Github repository}.
% The third option, \textsf{none}, tells the package not to load either of the datasets, and instead start off from an empty canvas. If one specifies this option, one will have to fill in the details of each language with the macro |\newlang| (see explanation in Section 2.3 below).
% \subsection{Macros}\label{sec:mac}
% When referring to a language, the author may use one of three macros to print out different information about it. Languages are identified by their ISO 639-3 code.
% \DescribeMacro{\lname}
% The simplest macro is |\lname|, which prints out the name of the specified language according to the code provided. The basic syntax is thus|\newlang| \marg{ISO code}. This can be seen in example (\getfullref{lname}).\\
% \ex<lname>
% \textsf{My native language is} |\lname{cat}|.\\ \\
% My native language is Catalan. \\
% \xe
% \DescribeMacro{\liso}
% One may also use the |\liso| macro to print out both the name and the ISO 639-3 code of the language specified in the macro in parenthesis , again according to its ISO code(|\liso| \marg{ISO code}). Example (\getfullref{liso}) shows its behavior.\\
% \ex<liso>
% \textsf{I have recently taken up} |\liso{brg}|.\\ \\
% I have recently taken up Baure (Arawakan). \\
% \xe
% \DescribeMacro{\lfam}
% A third macro for use is the |\lfam| command, which prints the name of the language and its family in parenthesis. Once again, the language is identified by its ISO 639-3 code. Example (\getfullref{lfam}) shows how it works.\\
% \ex<lfam>
% \textsf{The tone system of} |\liso{ptk}| is fascinating.\\ \\
% The tone system of Maleng (Austroasiatic) is fascinating. \\
% \xe
% \DescribeMacro{\newlang}
% Finally, users may add their own languages (or change the code, name, or genetic affiliation of a language already in the dataset) via the use of the |\newlang| command, which takes the three arguments \marg{code}, \marg{name}, and \marg{family}. This command may be used in the preamble, before |\begin{document}|. Example (\getfullref{newlang}) shows its usage.\\
% \ex<newlang>
% |\newlang{boo}{Ameli}{Amelian}| \\
% |\begin{document}|\\
% \textsf{My new made up language is} |\lname{boo}|.\\
% \textsf{My new made up language is} |\liso{boo}|.\\
% \textsf{My new made up language is} |\lfam{boo}|.\\
% |\end{document}|\\ \\
% My new made up language is Ameli.\\
% My new made up language is Ameli (ISO 639-3: boo).\\
% My new made up language is Ameli (Amelian).\\
% \xe
% Be aware that setting a new language overwrites any other language with the same code, as the package only listens to the language that is defined last.
% \StopEventually{\PrintIndex\PrintChanges}
% \section{Implementation}
% \subsection{Option setting}
% Options are set for what dataset to use. \textsf{glottolog} use Glottolog data;  \textsf{wals} uses WALS data; \textsf{none} selects neither dataset and all languages are defined by the user. See \textsf{langnames.py} in the \href{https://github.com/cicervlvs/langnames}{Github repository} to see how I gathered and handled the data.
% \begin{macro}{Options}
%    \begin{macrocode}
\DeclareOption{glottolog}{\define@key{names}{knw}{!Xun (Ekoka)}
\define@key{names}{nmn}{!Xóõ}
\define@key{names}{alu}{'Are'are}
\define@key{names}{hnh}{//Ani}
\define@key{names}{xam}{/Xam}
\define@key{names}{huc}{=|Hoan}
\define@key{names}{apq}{A-Pucikwar}
\define@key{names}{aiw}{Aari}
\define@key{names}{aau}{Abau}
\define@key{names}{abq}{Abaza}
\define@key{names}{abe}{Abenaki (Western)}
\define@key{names}{abi}{Abidji}
\define@key{names}{axb}{Abipón}
\define@key{names}{abk}{Abkhaz}
\define@key{names}{abz}{Abui}
\define@key{names}{kgr}{Abun}
\define@key{names}{ace}{Acehnese}
\define@key{names}{aca}{Achagua}
\define@key{names}{acn}{Achang}
\define@key{names}{ach}{Acholi}
\define@key{names}{acu}{Achuar}
\define@key{names}{acv}{Achumawi}
\define@key{names}{guq}{Aché}
\define@key{names}{acr}{Achí}
\define@key{names}{kjq}{Acoma}
\define@key{names}{ads}{Adamorobe Sign Language}
\define@key{names}{adn}{Adang}
\define@key{names}{adj}{Adioukrou}
\define@key{names}{ady}{Adyghe (Abzakh)}
\define@key{names}{adt}{Adynyamathanha}
\define@key{names}{adz}{Adzera}
\define@key{names}{awi}{Aekyom}
\define@key{names}{afr}{Afrikaans}
\define@key{names}{agd}{Agarabi}
\define@key{names}{agq}{Aghem}
\define@key{names}{ahh}{Aghu}
\define@key{names}{agx}{Aghul}
\define@key{names}{agt}{Agta (Central)}
\define@key{names}{duo}{Agta (Dupaningan)}
\define@key{names}{agu}{Aguacatec}
\define@key{names}{agr}{Aguaruna}
\define@key{names}{aht}{Ahtna}
\define@key{names}{tba}{Aikaná}
\define@key{names}{ain}{Ainu}
\define@key{names}{ahp}{Aizi}
\define@key{names}{aja}{Aja}
\define@key{names}{ajg}{Ajagbe}
\define@key{names}{aji}{Ajië}
\define@key{names}{axk}{Aka}
\define@key{names}{abj}{Aka-Biada}
\define@key{names}{aci}{Aka-Cari}
\define@key{names}{akx}{Aka-Kede}
\define@key{names}{aka}{Akan}
\define@key{names}{ake}{Akawaio}
\define@key{names}{ahk}{Akha}
\define@key{names}{akv}{Akhvakh}
\define@key{names}{akl}{Aklanon}
\define@key{names}{akw}{Akwa}
\define@key{names}{nrz}{Ala'ala}
\define@key{names}{akz}{Alabama}
\define@key{names}{wbj}{Alagwa}
\define@key{names}{amp}{Alamblak}
\define@key{names}{btz}{Alas}
\define@key{names}{alh}{Alawa}
\define@key{names}{sqi}{Albanian}
\define@key{names}{ale}{Aleut}
\define@key{names}{alq}{Algonquin}
\define@key{names}{ald}{Alladian}
\define@key{names}{gsw}{Alsatian}
\define@key{names}{aes}{Alsea}
\define@key{names}{alt}{Altai (Southern)}
\define@key{names}{alp}{Alune}
\define@key{names}{ems}{Alutiiq}
\define@key{names}{alr}{Alutor}
\define@key{names}{aly}{Alyawarra}
\define@key{names}{amm}{Ama}
\define@key{names}{amc}{Amahuaca}
\define@key{names}{amn}{Amanab}
\define@key{names}{aie}{Amara}
\define@key{names}{amr}{Amarakaeri}
\define@key{names}{omb}{Ambae (Lolovoli Northeast)}
\define@key{names}{amk}{Ambai}
\define@key{names}{abt}{Ambulas}
\define@key{names}{adx}{Amdo}
\define@key{names}{aey}{Amele}
\define@key{names}{ase}{American Sign Language}
\define@key{names}{amh}{Amharic}
\define@key{names}{ami}{Amis}
\define@key{names}{amo}{Amo}
\define@key{names}{apz}{Ampeeli}
\define@key{names}{ame}{Amuesha}
\define@key{names}{amu}{Amuzgo}
\define@key{names}{imi}{Anamuxra}
\define@key{names}{ani}{Andi}
\define@key{names}{ano}{Andoke}
\define@key{names}{aty}{Anejom}
\define@key{names}{agm}{Angaataha}
\define@key{names}{njm}{Angami}
\define@key{names}{anc}{Angas}
\define@key{names}{agg}{Anggor}
\define@key{names}{aoa}{Angolar}
\define@key{names}{awg}{Anguthimri}
\define@key{names}{aoi}{Anindilyakwa}
\define@key{names}{nun}{Anong}
\define@key{names}{cko}{Anufo}
\define@key{names}{any}{Anyi}
\define@key{names}{anu}{Anywa}
\define@key{names}{anz}{Anêm}
\define@key{names}{njo}{Ao}
\define@key{names}{apm}{Apache (Chiricahua)}
\define@key{names}{apj}{Apache (Jicarilla)}
\define@key{names}{apw}{Apache (Western)}
\define@key{names}{apy}{Apalaí}
\define@key{names}{apt}{Apatani}
\define@key{names}{apn}{Apinayé}
\define@key{names}{apu}{Apurinã}
\define@key{names}{ard}{Arabana}
\define@key{names}{arl}{Arabela}
\define@key{names}{abv}{Arabic (Bahrain)}
\define@key{names}{mey}{Arabic (Bani-Hassan)}
\define@key{names}{shu}{Arabic (Chadian)}
\define@key{names}{ayl}{Arabic (Eastern Libyan)}
\define@key{names}{arz}{Arabic (Egyptian)}
\define@key{names}{afb}{Arabic (Gulf)}
\define@key{names}{acw}{Arabic (Hijazi)}
\define@key{names}{acm}{Arabic (Iraqi)}
\define@key{names}{acy}{Arabic (Kormakiti)}
\define@key{names}{arb}{Arabic (Modern Standard)}
\define@key{names}{ary}{Arabic (Moroccan)}
\define@key{names}{ajp}{Arabic (Negev)}
\define@key{names}{ayn}{Arabic (San'ani)}
\define@key{names}{apc}{Arabic (Syrian)}
\define@key{names}{aeb}{Arabic (Tunisian)}
\define@key{names}{rmz}{Arakanese (Marma)}
\define@key{names}{akr}{Araki}
\define@key{names}{atq}{Aralle-Tabulahan}
\define@key{names}{jbj}{Arandai}
\define@key{names}{aro}{Araona}
\define@key{names}{arp}{Arapaho}
\define@key{names}{aah}{Arapesh (Abu)}
\define@key{names}{ape}{Arapesh (Mountain)}
\define@key{names}{arv}{Arbore}
\define@key{names}{aqc}{Archi}
\define@key{names}{laz}{Aribwatsa}
\define@key{names}{ari}{Arikara}
\define@key{names}{hye}{Armenian (Iranian)}
\define@key{names}{hyw}{Armenian (Western)}
\define@key{names}{apr}{Arop-Lokep}
\define@key{names}{aia}{Arosi}
\define@key{names}{aer}{Arrernte (Mparntwe)}
\define@key{names}{are}{Arrernte (Western)}
\define@key{names}{cns}{Asmat}
\define@key{names}{asm}{Assamese}
\define@key{names}{ast}{Asturian}
\define@key{names}{asu}{Asuriní}
\define@key{names}{kuz}{Atacameño}
\define@key{names}{aqp}{Atakapa}
\define@key{names}{tay}{Atayal}
\define@key{names}{upv}{Atchin}
\define@key{names}{aph}{Athpare}
\define@key{names}{atj}{Atikamekw}
\define@key{names}{atw}{Atsugewi}
\define@key{names}{avt}{Au}
\define@key{names}{aul}{Aulua}
\define@key{names}{asf}{Auslan}
\define@key{names}{auy}{Auyana}
\define@key{names}{ava}{Avar}
\define@key{names}{avn}{Avatime}
\define@key{names}{avi}{Avikam}
\define@key{names}{avu}{Avokaya}
\define@key{names}{awb}{Awa}
\define@key{names}{kwi}{Awa Pit}
\define@key{names}{awa}{Awadhi}
\define@key{names}{awn}{Awngi}
\define@key{names}{kmn}{Awtuw}
\define@key{names}{auw}{Awyi}
\define@key{names}{nfl}{Ayiwo}
\define@key{names}{ayr}{Aymara (Central)}
\define@key{names}{aib}{Aynu}
\define@key{names}{ayo}{Ayoreo}
\define@key{names}{azb}{Azari (Iranian)}
\define@key{names}{koe}{Baale}
\define@key{names}{bvx}{Babole}
\define@key{names}{bav}{Babungo}
\define@key{names}{wdj}{Bachamal}
\define@key{names}{bfq}{Badaga}
\define@key{names}{bde}{Bade}
\define@key{names}{bia}{Badimaya}
\define@key{names}{ksf}{Bafia}
\define@key{names}{bfd}{Bafut}
\define@key{names}{bsp}{Baga Sitemu}
\define@key{names}{bmi}{Bagirmi}
\define@key{names}{fuu}{Bagiro}
\define@key{names}{bgq}{Bagri}
\define@key{names}{kva}{Bagvalal}
\define@key{names}{bdw}{Baham}
\define@key{names}{bjh}{Bahinemo}
\define@key{names}{bdq}{Bahnar}
\define@key{names}{bca}{Bai}
\define@key{names}{bdl}{Bajau (Sama)}
\define@key{names}{bdr}{Bajau (West Coast)}
\define@key{names}{bkc}{Baka (in Cameroon)}
\define@key{names}{bdh}{Baka (in South Sudan)}
\define@key{names}{bkq}{Bakairí}
\define@key{names}{bri}{Bakueri}
\define@key{names}{blw}{Balangao}
\define@key{names}{blz}{Balantak}
\define@key{names}{ban}{Balinese}
\define@key{names}{bft}{Balti}
\define@key{names}{bgn}{Baluchi}
\define@key{names}{ptu}{Bambam}
\define@key{names}{bam}{Bambara}
\define@key{names}{bax}{Bamun}
\define@key{names}{bcw}{Bana}
\define@key{names}{jaa}{Banawá}
\define@key{names}{bza}{Bandi}
\define@key{names}{bdy}{Bandjalang}
\define@key{names}{bgz}{Banggai}
\define@key{names}{bjb}{Banggarla}
\define@key{names}{bdg}{Banggi}
\define@key{names}{dba}{Bangime}
\define@key{names}{bvv}{Baniva}
\define@key{names}{bwi}{Baniwa}
\define@key{names}{abb}{Bankon}
\define@key{names}{bcm}{Banoni}
\define@key{names}{bnq}{Bantik}
\define@key{names}{peh}{Bao'an}
\define@key{names}{bci}{Baoulé}
\define@key{names}{loy}{Baragaunle}
\define@key{names}{bbb}{Barai}
\define@key{names}{brm}{Barambu}
\define@key{names}{bsn}{Barasano}
\define@key{names}{bcj}{Bardi}
\define@key{names}{mlp}{Bargam}
\define@key{names}{bfa}{Bari}
\define@key{names}{bba}{Bariba}
\define@key{names}{wra}{Barupu}
\define@key{names}{byr}{Baruya}
\define@key{names}{bae}{Baré}
\define@key{names}{mot}{Barí}
\define@key{names}{bsc}{Basari}
\define@key{names}{bas}{Basaá}
\define@key{names}{bak}{Bashkir}
\define@key{names}{eus}{Basque}
\define@key{names}{bya}{Batak}
\define@key{names}{btx}{Batak (Karo)}
\define@key{names}{bbc}{Batak (Toba)}
\define@key{names}{bhm}{Bathari}
\define@key{names}{bbd}{Bau}
\define@key{names}{brg}{Baure}
\define@key{names}{bvz}{Bauzi}
\define@key{names}{bgr}{Bawm}
\define@key{names}{bsw}{Bayso}
\define@key{names}{bxj}{Bayungu}
\define@key{names}{beq}{Beembe}
\define@key{names}{dbj}{Begak-Ida'an}
\define@key{names}{bej}{Beja}
\define@key{names}{byw}{Belhare}
\define@key{names}{blc}{Bella Coola}
\define@key{names}{bel}{Belorussian}
\define@key{names}{bem}{Bemba}
\define@key{names}{bef}{Benabena}
\define@key{names}{nhb}{Beng}
\define@key{names}{bng}{Benga}
\define@key{names}{ben}{Bengali}
\define@key{names}{ctg}{Bengali (Chittagong)}
\define@key{names}{bue}{Beothuk}
\define@key{names}{brf}{Bera}
\define@key{names}{shy}{Berber (Chaouia)}
\define@key{names}{grr}{Berber (Figuig)}
\define@key{names}{tzm}{Berber (Middle Atlas)}
\define@key{names}{mzb}{Berber (Mzab)}
\define@key{names}{rif}{Berber (Rif)}
\define@key{names}{siz}{Berber (Siwa)}
\define@key{names}{oua}{Berber (Wargla)}
\define@key{names}{brc}{Berbice Dutch Creole}
\define@key{names}{zag}{Beria}
\define@key{names}{bkl}{Berik}
\define@key{names}{wti}{Berta}
\define@key{names}{xub}{Betta Kurumba}
\define@key{names}{kap}{Bezhta}
\define@key{names}{bhb}{Bhili}
\define@key{names}{bho}{Bhojpuri}
\define@key{names}{unr}{Bhumij}
\define@key{names}{bif}{Biafada}
\define@key{names}{bhw}{Biak}
\define@key{names}{bth}{Biatah}
\define@key{names}{bid}{Bidiya}
\define@key{names}{bcl}{Bikol}
\define@key{names}{bip}{Bila}
\define@key{names}{bpr}{Bilaan (Koronadal)}
\define@key{names}{byn}{Bilin}
\define@key{names}{nbj}{Bilinarra}
\define@key{names}{bll}{Biloxi}
\define@key{names}{blb}{Bilua}
\define@key{names}{bhp}{Bima}
\define@key{names}{bim}{Bimoba}
\define@key{names}{bhg}{Binandere}
\define@key{names}{bin}{Bini}
\define@key{names}{gup}{Bininj Gun-Wok}
\define@key{names}{bkd}{Binukid}
\define@key{names}{bjr}{Binumarien}
\define@key{names}{bzr}{Biri}
\define@key{names}{bom}{Birom}
\define@key{names}{bvq}{Birri}
\define@key{names}{bib}{Bisa}
\define@key{names}{bis}{Bislama}
\define@key{names}{bla}{Blackfoot}
\define@key{names}{kvg}{Boazi (Kuni)}
\define@key{names}{bni}{Bobangi}
\define@key{names}{bbo}{Bobo Madaré (Northern)}
\define@key{names}{brx}{Bodo}
\define@key{names}{bzf}{Boiken}
\define@key{names}{bqc}{Boko}
\define@key{names}{bol}{Bole}
\define@key{names}{bli}{Bolia}
\define@key{names}{bot}{Bongo}
\define@key{names}{bpu}{Bongu}
\define@key{names}{lbk}{Bontok}
\define@key{names}{boa}{Bora}
\define@key{names}{adi}{Bori}
\define@key{names}{bor}{Bororo}
\define@key{names}{brn}{Boruca}
\define@key{names}{bos}{Bosnian}
\define@key{names}{boz}{Bozo (Tigemaxo)}
\define@key{names}{brh}{Brahui}
\define@key{names}{brb}{Brao}
\define@key{names}{bre}{Breton}
\define@key{names}{bzd}{Bribri}
\define@key{names}{bfi}{British Sign Language}
\define@key{names}{tcs}{Broken}
\define@key{names}{bkk}{Brokskat}
\define@key{names}{bru}{Bru (Eastern)}
\define@key{names}{brv}{Bru (Western)}
\define@key{names}{bvb}{Bubi}
\define@key{names}{buu}{Budu}
\define@key{names}{bdk}{Budukh}
\define@key{names}{bdm}{Buduma}
\define@key{names}{bug}{Bugis}
\define@key{names}{sab}{Buglere}
\define@key{names}{bgg}{Bugun}
\define@key{names}{buo}{Buin}
\define@key{names}{nmg}{Bujeba}
\define@key{names}{bxk}{Bukusu}
\define@key{names}{bul}{Bulgarian}
\define@key{names}{bwu}{Buli (in Ghana)}
\define@key{names}{bzq}{Buli (in Indonesia)}
\define@key{names}{bum}{Bulu}
\define@key{names}{tkw}{Buma}
\define@key{names}{bfu}{Bunan}
\define@key{names}{buh}{Bunu (Younuo)}
\define@key{names}{bck}{Bunuba}
\define@key{names}{bwr}{Bura-Pabir}
\define@key{names}{bvr}{Burarra}
\define@key{names}{bxm}{Buriat}
\define@key{names}{bji}{Burji}
\define@key{names}{mya}{Burmese}
\define@key{names}{mhs}{Buru}
\define@key{names}{bmu}{Burum}
\define@key{names}{bds}{Burunge}
\define@key{names}{bsk}{Burushaski}
\define@key{names}{bqp}{Busa}
\define@key{names}{buf}{Bushoong}
\define@key{names}{ngc}{Bwele}
\define@key{names}{bee}{Byansi}
\define@key{names}{bev}{Bété}
\define@key{names}{cjp}{Cabécar}
\define@key{names}{cbv}{Cacua}
\define@key{names}{cad}{Caddo}
\define@key{names}{chl}{Cahuilla}
\define@key{names}{cak}{Cakchiquel}
\define@key{names}{rab}{Camling}
\define@key{names}{cjo}{Campa Pajonal Asheninca}
\define@key{names}{kbh}{Camsá}
\define@key{names}{knm}{Canamarí}
\define@key{names}{cbu}{Candoshi}
\define@key{names}{ram}{Canela}
\define@key{names}{yue}{Cantonese}
\define@key{names}{kaq}{Capanahua}
\define@key{names}{cbc}{Carapana}
\define@key{names}{car}{Carib}
\define@key{names}{mch}{Carib (De'kwana)}
\define@key{names}{cal}{Carolinian}
\define@key{names}{crx}{Carrier}
\define@key{names}{cbr}{Cashibo}
\define@key{names}{cbs}{Cashinahua}
\define@key{names}{cat}{Catalan}
\define@key{names}{chc}{Catawba}
\define@key{names}{cto}{Catio}
\define@key{names}{cav}{Cavineña}
\define@key{names}{cbi}{Cayapa}
\define@key{names}{cay}{Cayuga}
\define@key{names}{cyb}{Cayuvava}
\define@key{names}{ceb}{Cebuano}
\define@key{names}{old}{Chaga}
\define@key{names}{suq}{Chai}
\define@key{names}{cld}{Chaldean (Modern)}
\define@key{names}{cjm}{Cham (Eastern)}
\define@key{names}{cja}{Cham (Western)}
\define@key{names}{cji}{Chamalal}
\define@key{names}{can}{Chambri}
\define@key{names}{cha}{Chamorro}
\define@key{names}{nbc}{Chang}
\define@key{names}{chx}{Chantyal}
\define@key{names}{tuu}{Chasta Costa}
\define@key{names}{cya}{Chatino (Nopala)}
\define@key{names}{cta}{Chatino (Tataltepec)}
\define@key{names}{ctp}{Chatino (Yaitepec)}
\define@key{names}{cdn}{Chaudangsi}
\define@key{names}{cbk}{Chavacano}
\define@key{names}{cbt}{Chayahuita}
\define@key{names}{che}{Chechen}
\define@key{names}{cjh}{Chehalis (Upper)}
\define@key{names}{mrn}{Cheke Holo}
\define@key{names}{xch}{Chemakum}
\define@key{names}{cdm}{Chepang}
\define@key{names}{chr}{Cherokee}
\define@key{names}{chy}{Cheyenne}
\define@key{names}{nya}{Chichewa}
\define@key{names}{pei}{Chichimeca-Jonaz}
\define@key{names}{cic}{Chickasaw}
\define@key{names}{cob}{Chicomuceltec}
\define@key{names}{cid}{Chimariko}
\define@key{names}{cbg}{Chimila}
\define@key{names}{mrh}{Chin (Mara)}
\define@key{names}{csy}{Chin (Siyin)}
\define@key{names}{ctd}{Chin (Tiddim)}
\define@key{names}{cco}{Chinantec (Comaltepec)}
\define@key{names}{cle}{Chinantec (Lealao)}
\define@key{names}{cpa}{Chinantec (Palantla)}
\define@key{names}{chq}{Chinantec (Quiotepec)}
\define@key{names}{cuc}{Chinantec (San Felipe Usila)}
\define@key{names}{cso}{Chinantec (Sochiapan)}
\define@key{names}{cnt}{Chinantec (Tepetotutla)}
\define@key{names}{csl}{Chinese Sign Language}
\define@key{names}{chh}{Chinook (Lower)}
\define@key{names}{wac}{Chinook (Upper)}
\define@key{names}{cap}{Chipaya}
\define@key{names}{chp}{Chipewyan}
\define@key{names}{cax}{Chiquitano}
\define@key{names}{gui}{Chiriguano}
\define@key{names}{ctm}{Chitimacha}
\define@key{names}{coz}{Chocho}
\define@key{names}{cho}{Choctaw}
\define@key{names}{ctu}{Chol}
\define@key{names}{cht}{Cholón}
\define@key{names}{chd}{Chontal (Highland)}
\define@key{names}{clo}{Chontal (Huamelultec Oaxaca)}
\define@key{names}{chf}{Chontal Maya}
\define@key{names}{caa}{Chortí}
\define@key{names}{crw}{Chrau}
\define@key{names}{cje}{Chru}
\define@key{names}{cjv}{Chuave}
\define@key{names}{cac}{Chuj}
\define@key{names}{ckt}{Chukchi}
\define@key{names}{clw}{Chulym}
\define@key{names}{boi}{Chumash (Barbareño)}
\define@key{names}{inz}{Chumash (Ineseño)}
\define@key{names}{ncu}{Chumburung}
\define@key{names}{chk}{Chuukese}
\define@key{names}{chv}{Chuvash}
\define@key{names}{cao}{Chácobo}
\define@key{names}{lua}{CiLuba}
\define@key{names}{clm}{Clallam}
\define@key{names}{xcw}{Coahuilteco}
\define@key{names}{cod}{Cocama}
\define@key{names}{coc}{Cocopa}
\define@key{names}{crd}{Coeur d'Alene}
\define@key{names}{con}{Cofán}
\define@key{names}{kog}{Cogui}
\define@key{names}{col}{Columbia-Wenatchi}
\define@key{names}{com}{Comanche}
\define@key{names}{xcm}{Comecrudo}
\define@key{names}{swb}{Comorian}
\define@key{names}{coo}{Comox}
\define@key{names}{csz}{Coos (Hanis)}
\define@key{names}{cop}{Coptic}
\define@key{names}{crn}{Cora}
\define@key{names}{cor}{Cornish}
\define@key{names}{crk}{Cree (Plains)}
\define@key{names}{csw}{Cree (Swampy)}
\define@key{names}{mus}{Creek}
\define@key{names}{crh}{Crimean Tatar}
\define@key{names}{cro}{Crow}
\define@key{names}{cua}{Cua}
\define@key{names}{cub}{Cubeo}
\define@key{names}{cui}{Cuiba}
\define@key{names}{cuy}{Cuitlatec}
\define@key{names}{cul}{Culina}
\define@key{names}{cup}{Cupeño}
\define@key{names}{kpc}{Curripaco}
\define@key{names}{ces}{Czech}
\define@key{names}{cam}{Cèmuhî}
\define@key{names}{kzf}{Da'a}
\define@key{names}{dbq}{Daba}
\define@key{names}{dav}{Dabida}
\define@key{names}{mps}{Dadibi}
\define@key{names}{dgz}{Daga}
\define@key{names}{dga}{Dagaare}
\define@key{names}{dag}{Dagbani}
\define@key{names}{dta}{Dagur}
\define@key{names}{dal}{Dahalo}
\define@key{names}{daj}{Daju (Dar Fur)}
\define@key{names}{dak}{Dakota}
\define@key{names}{mbp}{Damana}
\define@key{names}{dnj}{Dan}
\define@key{names}{daa}{Dangaléat (Western)}
\define@key{names}{dni}{Dani (Lower Grand Valley)}
\define@key{names}{dan}{Danish}
\define@key{names}{dry}{Darai}
\define@key{names}{dar}{Dargwa}
\define@key{names}{prs}{Dari}
\define@key{names}{drd}{Darma}
\define@key{names}{tcc}{Datooga}
\define@key{names}{dai}{Day}
\define@key{names}{afn}{Defaka}
\define@key{names}{deg}{Degema}
\define@key{names}{ing}{Degexit'an}
\define@key{names}{dny}{Dení}
\define@key{names}{des}{Desano}
\define@key{names}{shg}{Deti}
\define@key{names}{der}{Deuri}
\define@key{names}{gsg}{Deutsche Gebärdensprache}
\define@key{names}{dsh}{Dhaasanac}
\define@key{names}{dhl}{Dhalandji}
\define@key{names}{tbh}{Dharawal}
\define@key{names}{dhr}{Dhargari}
\define@key{names}{xgm}{Dharumbal}
\define@key{names}{dhi}{Dhimal}
\define@key{names}{div}{Dhivehi}
\define@key{names}{dhu}{Dhurga}
\define@key{names}{did}{Didinga}
\define@key{names}{mhu}{Digaro}
\define@key{names}{dur}{Dii}
\define@key{names}{dis}{Dimasa}
\define@key{names}{dim}{Dime}
\define@key{names}{diz}{Ding}
\define@key{names}{din}{Dinka}
\define@key{names}{dyo}{Diola-Fogny}
\define@key{names}{csk}{Diola-Kasa}
\define@key{names}{dif}{Diyari}
\define@key{names}{mdx}{Dizi}
\define@key{names}{dyy}{Djabugay}
\define@key{names}{djr}{Djambarrpuyngu}
\define@key{names}{duj}{Djapu}
\define@key{names}{ddj}{Djaru}
\define@key{names}{dji}{Djinang}
\define@key{names}{jig}{Djingili}
\define@key{names}{kbv}{Dla (Proper)}
\define@key{names}{kvo}{Dobel}
\define@key{names}{dgo}{Dogri}
\define@key{names}{dlg}{Dolgan}
\define@key{names}{dmk}{Domaaki}
\define@key{names}{rmt}{Domari}
\define@key{names}{kmc}{Dong (Southern)}
\define@key{names}{doo}{Dongo}
\define@key{names}{dds}{Donno So}
\define@key{names}{tds}{Doutai}
\define@key{names}{dow}{Doyayo}
\define@key{names}{dhv}{Drehu}
\define@key{names}{dua}{Duala}
\define@key{names}{dud}{Duka}
\define@key{names}{gwd}{Dullay (Gollango)}
\define@key{names}{duu}{Dulong}
\define@key{names}{dma}{Duma}
\define@key{names}{dgc}{Dumagat (Casiguran)}
\define@key{names}{dus}{Dumi}
\define@key{names}{vam}{Dumo}
\define@key{names}{duc}{Duna}
\define@key{names}{nld}{Dutch}
\define@key{names}{zea}{Dutch (Zeeuws)}
\define@key{names}{dyi}{Dyimini}
\define@key{names}{dbl}{Dyirbal}
\define@key{names}{dyu}{Dyula}
\define@key{names}{kwa}{Dâw}
\define@key{names}{igb}{Ebira}
\define@key{names}{etr}{Edolo}
\define@key{names}{erk}{Efate (South)}
\define@key{names}{efi}{Efik}
\define@key{names}{ega}{Ega}
\define@key{names}{eip}{Eipo}
\define@key{names}{etu}{Ejagham}
\define@key{names}{ekg}{Ekari}
\define@key{names}{eko}{Ekoti}
\define@key{names}{mrf}{Elseng}
\define@key{names}{ema}{Emai}
\define@key{names}{emb}{Embaloh}
\define@key{names}{cmi}{Embera Chami}
\define@key{names}{emp}{Emberá (Northern)}
\define@key{names}{amy}{Emmi}
\define@key{names}{enq}{Enga}
\define@key{names}{enn}{Engenni}
\define@key{names}{eno}{Enggano}
\define@key{names}{eng}{English}
\define@key{names}{gey}{Enya}
\define@key{names}{sja}{Epena Pedee}
\define@key{names}{erg}{Erromangan}
\define@key{names}{ese}{Ese Ejja}
\define@key{names}{esq}{Esselen}
\define@key{names}{ekk}{Estonian}
\define@key{names}{ets}{Etsako}
\define@key{names}{eve}{Even}
\define@key{names}{ewe}{Ewe}
\define@key{names}{ewo}{Ewondo}
\define@key{names}{eya}{Eyak}
\define@key{names}{fao}{Faroese}
\define@key{names}{faa}{Fasu}
\define@key{names}{fmp}{Fe'fe'}
\define@key{names}{fij}{Fijian}
\define@key{names}{fin}{Finnish}
\define@key{names}{fse}{Finnish Sign Language}
\define@key{names}{foi}{Foe}
\define@key{names}{ppo}{Folopa}
\define@key{names}{fon}{Fongbe}
\define@key{names}{frd}{Fordata}
\define@key{names}{for}{Fore}
\define@key{names}{sac}{Fox}
\define@key{names}{fra}{French}
\define@key{names}{fry}{Frisian}
\define@key{names}{frs}{Frisian (Eastern)}
\define@key{names}{frr}{Frisian (North)}
\define@key{names}{fuh}{Ful (Liptako)}
\define@key{names}{fuf}{Fula (Guinean)}
\define@key{names}{fub}{Fulfulde (Adamawa)}
\define@key{names}{ffm}{Fulfulde (Maasina)}
\define@key{names}{fuv}{Fulfulde (Nigerian)}
\define@key{names}{fun}{Fulniô}
\define@key{names}{fvr}{Fur}
\define@key{names}{fud}{Futuna (East)}
\define@key{names}{fut}{Futuna-Aniwa}
\define@key{names}{cdo}{Fuzhou}
\define@key{names}{pym}{Fyem}
\define@key{names}{gqa}{Ga'anda}
\define@key{names}{gbu}{Gaagudju}
\define@key{names}{dhg}{Gaalpu}
\define@key{names}{gdb}{Gadaba (Kondekor)}
\define@key{names}{ged}{Gade}
\define@key{names}{gaj}{Gadsup}
\define@key{names}{gla}{Gaelic (Scots)}
\define@key{names}{gag}{Gagauz}
\define@key{names}{gah}{Gahuku}
\define@key{names}{gbi}{Galela}
\define@key{names}{glg}{Galician}
\define@key{names}{adl}{Galo}
\define@key{names}{kld}{Gamilaraay}
\define@key{names}{gmv}{Gamo}
\define@key{names}{pwg}{Gapapaiwa}
\define@key{names}{grt}{Garo}
\define@key{names}{wrk}{Garrwa}
\define@key{names}{gyb}{Garus}
\define@key{names}{cab}{Garífuna}
\define@key{names}{gvo}{Gavião}
\define@key{names}{gay}{Gayo}
\define@key{names}{gya}{Gbaya (Northwest)}
\define@key{names}{gso}{Gbaya (Southwest)}
\define@key{names}{gbp}{Gbeya Bossangoa}
\define@key{names}{nlg}{Gela}
\define@key{names}{gqu}{Gelao}
\define@key{names}{kat}{Georgian}
\define@key{names}{deu}{German}
\define@key{names}{bar}{German (Bavarian)}
\define@key{names}{ksh}{German (Ripuarian)}
\define@key{names}{wep}{German (Westphalian)}
\define@key{names}{aaa}{Ghotuo}
\define@key{names}{ghl}{Ghulfan}
\define@key{names}{gih}{Gidabal}
\define@key{names}{gid}{Gidar}
\define@key{names}{glk}{Gilaki}
\define@key{names}{bcq}{Gimira}
\define@key{names}{git}{Gitksan}
\define@key{names}{gis}{Giziga}
\define@key{names}{guc}{Goajiro}
\define@key{names}{god}{Godié}
\define@key{names}{gdo}{Godoberi}
\define@key{names}{ank}{Goemai}
\define@key{names}{ggw}{Gogodala}
\define@key{names}{gju}{Gojri}
\define@key{names}{gkn}{Gokana}
\define@key{names}{gol}{Gola}
\define@key{names}{gvf}{Golin}
\define@key{names}{gno}{Gondi}
\define@key{names}{gni}{Gooniyandi}
\define@key{names}{gor}{Gorontalo}
\define@key{names}{gow}{Gorowa}
\define@key{names}{grj}{Grebo}
\define@key{names}{ell}{Greek (Modern)}
\define@key{names}{gss}{Greek Sign Language}
\define@key{names}{kal}{Greenlandic (West)}
\define@key{names}{guh}{Guahibo}
\define@key{names}{gub}{Guajajara}
\define@key{names}{gum}{Guambiano}
\define@key{names}{gva}{Guana}
\define@key{names}{gvc}{Guanano}
\define@key{names}{gug}{Guaraní}
\define@key{names}{var}{Guarijío}
\define@key{names}{gta}{Guató}
\define@key{names}{guo}{Guayabero}
\define@key{names}{gde}{Gude}
\define@key{names}{gdf}{Guduf}
\define@key{names}{ktd}{Gugada}
\define@key{names}{ggd}{Gugadj}
\define@key{names}{ghs}{Guhu-Samane}
\define@key{names}{gcr}{Guianese French Creole}
\define@key{names}{pov}{Guinea Bissau Crioulo}
\define@key{names}{guj}{Gujarati}
\define@key{names}{kcm}{Gula (in Central African Republic)}
\define@key{names}{glj}{Gula Iro}
\define@key{names}{gnn}{Gumatj}
\define@key{names}{gvs}{Gumawana}
\define@key{names}{kgs}{Gumbaynggir}
\define@key{names}{guk}{Gumuz}
\define@key{names}{wlg}{Gunbalang}
\define@key{names}{guw}{Gungbe}
\define@key{names}{gww}{Gunin}
\define@key{names}{yas}{Gunu}
\define@key{names}{gyy}{Gunya}
\define@key{names}{guf}{Gupapuyngu}
\define@key{names}{gnr}{Gureng Gureng}
\define@key{names}{gur}{Gurenne}
\define@key{names}{gue}{Gurindji}
\define@key{names}{gux}{Gurma}
\define@key{names}{goa}{Guro}
\define@key{names}{gge}{Gurr-goni}
\define@key{names}{guz}{Gusii}
\define@key{names}{gbj}{Gutob}
\define@key{names}{kky}{Guugu Yimidhirr}
\define@key{names}{gbr}{Gwari}
\define@key{names}{kcg}{Gworok}
\define@key{names}{gaa}{Gã}
\define@key{names}{pue}{Gününa Küne}
\define@key{names}{hts}{Hadza}
\define@key{names}{hai}{Haida}
\define@key{names}{hdn}{Haida (Northern)}
\define@key{names}{has}{Haisla}
\define@key{names}{hat}{Haitian Creole}
\define@key{names}{hak}{Hakka}
\define@key{names}{hal}{Halang}
\define@key{names}{hlb}{Halbi}
\define@key{names}{hla}{Halia}
\define@key{names}{amf}{Hamer}
\define@key{names}{hmt}{Hamtai}
\define@key{names}{wos}{Hanga Hundi}
\define@key{names}{hni}{Hani}
\define@key{names}{hnn}{Hanunóo}
\define@key{names}{har}{Harari}
\define@key{names}{hss}{Harsusi}
\define@key{names}{tmd}{Haruai}
\define@key{names}{had}{Hatam}
\define@key{names}{hau}{Hausa}
\define@key{names}{haw}{Hawaiian}
\define@key{names}{hwc}{Hawaiian Creole}
\define@key{names}{hac}{Hawrami}
\define@key{names}{hay}{Haya}
\define@key{names}{vay}{Hayu}
\define@key{names}{xed}{Hdi}
\define@key{names}{heb}{Hebrew (Modern)}
\define@key{names}{heh}{Hehe}
\define@key{names}{hei}{Heiltsuk}
\define@key{names}{hem}{Hemba}
\define@key{names}{her}{Herero}
\define@key{names}{hid}{Hidatsa}
\define@key{names}{hil}{Hiligaynon}
\define@key{names}{hin}{Hindi}
\define@key{names}{gin}{Hinuq}
\define@key{names}{hix}{Hixkaryana}
\define@key{names}{lic}{Hlai (Baoding)}
\define@key{names}{hmr}{Hmar}
\define@key{names}{mww}{Hmong Daw}
\define@key{names}{hnj}{Hmong Njua}
\define@key{names}{hoc}{Ho}
\define@key{names}{hoa}{Hoava}
\define@key{names}{hoo}{Holoholo}
\define@key{names}{hks}{Hong Kong Sign Language}
\define@key{names}{hop}{Hopi}
\define@key{names}{hre}{Hre}
\define@key{names}{ygr}{Hua}
\define@key{names}{hub}{Huambisa}
\define@key{names}{hus}{Huastec}
\define@key{names}{huv}{Huave (San Mateo del Mar)}
\define@key{names}{hch}{Huichol}
\define@key{names}{hto}{Huitoto (Minica)}
\define@key{names}{hux}{Huitoto (Muinane)}
\define@key{names}{huu}{Huitoto (Murui)}
\define@key{names}{hke}{Hunde}
\define@key{names}{hun}{Hungarian}
\define@key{names}{huz}{Hunzib}
\define@key{names}{jup}{Hup}
\define@key{names}{hup}{Hupa}
\define@key{names}{csh}{Hyow}
\define@key{names}{ksi}{I'saka}
\define@key{names}{iai}{Iaai}
\define@key{names}{ian}{Iatmul}
\define@key{names}{tmu}{Iau}
\define@key{names}{iba}{Iban}
\define@key{names}{ibg}{Ibanag}
\define@key{names}{ibb}{Ibibio}
\define@key{names}{isl}{Icelandic}
\define@key{names}{icl}{Icelandic Sign Language}
\define@key{names}{idu}{Idoma}
\define@key{names}{clk}{Idu}
\define@key{names}{viv}{Iduna}
\define@key{names}{mxe}{Ifira-Mele}
\define@key{names}{ifb}{Ifugao (Batad)}
\define@key{names}{ifm}{Ifumu}
\define@key{names}{ibo}{Igbo}
\define@key{names}{ige}{Igede}
\define@key{names}{ign}{Ignaciano}
\define@key{names}{ihp}{Iha}
\define@key{names}{ijc}{Ijo (Kolokuma)}
\define@key{names}{ikx}{Ik}
\define@key{names}{arh}{Ika}
\define@key{names}{ilb}{Ila}
\define@key{names}{mia}{Illinois}
\define@key{names}{ilo}{Ilocano}
\define@key{names}{imn}{Imonda}
\define@key{names}{szp}{Inanwatan}
\define@key{names}{ins}{Indo-Pakistani Sign Language (Indian dialects)}
\define@key{names}{pks}{Indo-Pakistani Sign Language (Karachi dialect)}
\define@key{names}{ind}{Indonesian}
\define@key{names}{pmy}{Indonesian (Papuan)}
\define@key{names}{inb}{Inga}
\define@key{names}{tbi}{Ingessana}
\define@key{names}{inh}{Ingush}
\define@key{names}{ynd}{Innamincka}
\define@key{names}{ils}{International Sign}
\define@key{names}{ike}{Inuktitut (Salluit)}
\define@key{names}{iqu}{Iquito}
\define@key{names}{irn}{Iranxe}
\define@key{names}{irk}{Iraqw}
\define@key{names}{irh}{Irarutu}
\define@key{names}{gle}{Irish}
\define@key{names}{isg}{Irish Sign Language}
\define@key{names}{its}{Isekiri}
\define@key{names}{isk}{Ishkashimi}
\define@key{names}{srl}{Isirawa}
\define@key{names}{isd}{Isnag}
\define@key{names}{iso}{Isoko}
\define@key{names}{isr}{Israeli Sign Language}
\define@key{names}{ita}{Italian}
\define@key{names}{egl}{Italian (Bologna)}
\define@key{names}{lij}{Italian (Genoa)}
\define@key{names}{nap}{Italian (Napolitanian)}
\define@key{names}{pms}{Italian (Turinese)}
\define@key{names}{itv}{Itawis}
\define@key{names}{itl}{Itelmen}
\define@key{names}{ito}{Itonama}
\define@key{names}{itz}{Itzaj}
\define@key{names}{ivb}{Ivatan}
\define@key{names}{ibd}{Iwaidja}
\define@key{names}{iwm}{Iwam}
\define@key{names}{yom}{Iwoyo}
\define@key{names}{ixc}{Ixcatec}
\define@key{names}{ixl}{Ixil}
\define@key{names}{izr}{Izere}
\define@key{names}{izh}{Izhor}
\define@key{names}{izz}{Izi}
\define@key{names}{esi}{Iñupiaq}
\define@key{names}{jbt}{Jabutí}
\define@key{names}{jae}{Jabêm}
\define@key{names}{jda}{Jad}
\define@key{names}{jhi}{Jahai}
\define@key{names}{jac}{Jakaltek}
\define@key{names}{jam}{Jamaican Creole}
\define@key{names}{djd}{Jaminjung}
\define@key{names}{djm}{Jamsay}
\define@key{names}{jpn}{Japanese}
\define@key{names}{jru}{Japreria}
\define@key{names}{jqr}{Jaqaru}
\define@key{names}{anq}{Jarawa (in Andamans)}
\define@key{names}{jav}{Javanese}
\define@key{names}{jeb}{Jebero}
\define@key{names}{jeh}{Jeh}
\define@key{names}{jek}{Jeli}
\define@key{names}{tow}{Jemez}
\define@key{names}{jya}{Jiarong}
\define@key{names}{shv}{Jibbali}
\define@key{names}{kac}{Jingpho}
\define@key{names}{jiu}{Jino}
\define@key{names}{jiv}{Jivaro}
\define@key{names}{rgk}{Johari}
\define@key{names}{tlo}{Jomang}
\define@key{names}{jun}{Juang}
\define@key{names}{nst}{Jugli}
\define@key{names}{jbu}{Jukun}
\define@key{names}{bex}{Jur Mödö}
\define@key{names}{juc}{Jurchen}
\define@key{names}{jur}{Juruna}
\define@key{names}{ktz}{Ju|'hoan}
\define@key{names}{jua}{Júma}
\define@key{names}{kek}{K'ekchí}
\define@key{names}{kbd}{Kabardian}
\define@key{names}{xkp}{Kabatei}
\define@key{names}{kbp}{Kabiyé}
\define@key{names}{nbu}{Kabui}
\define@key{names}{kab}{Kabyle}
\define@key{names}{xac}{Kachari}
\define@key{names}{kzj}{Kadazan}
\define@key{names}{kbc}{Kadiwéu}
\define@key{names}{kdm}{Kagoma}
\define@key{names}{kki}{Kagulu}
\define@key{names}{kct}{Kaian}
\define@key{names}{lew}{Kaili}
\define@key{names}{kgp}{Kaingang}
\define@key{names}{kxa}{Kairiru}
\define@key{names}{kgk}{Kaiwá}
\define@key{names}{tbd}{Kaki Ae}
\define@key{names}{mwp}{Kala Lagaw Ya}
\define@key{names}{kmh}{Kalam}
\define@key{names}{gwc}{Kalami}
\define@key{names}{kck}{Kalanga}
\define@key{names}{kyl}{Kalapuya}
\define@key{names}{kls}{Kalasha}
\define@key{names}{fla}{Kalispel}
\define@key{names}{ktg}{Kalkatungu}
\define@key{names}{bco}{Kaluli}
\define@key{names}{kay}{Kamaiurá}
\define@key{names}{kbq}{Kamano-Kafe}
\define@key{names}{kms}{Kamasau}
\define@key{names}{xas}{Kamass}
\define@key{names}{kam}{Kamba}
\define@key{names}{xbr}{Kambera}
\define@key{names}{kbx}{Kambot}
\define@key{names}{kcu}{Kami}
\define@key{names}{kgq}{Kamoro}
\define@key{names}{xmu}{Kamu}
\define@key{names}{ogo}{Kana}
\define@key{names}{kna}{Kanakuru}
\define@key{names}{xns}{Kanashi}
\define@key{names}{kbl}{Kanembu}
\define@key{names}{ikt}{Kangiryuarmiut}
\define@key{names}{kjb}{Kanjobal (Eastern)}
\define@key{names}{knj}{Kanjobal (Western)}
\define@key{names}{kne}{Kankanay}
\define@key{names}{kan}{Kannada}
\define@key{names}{kxo}{Kanoê}
\define@key{names}{khd}{Kanum (Bädi)}
\define@key{names}{kcd}{Kanum (Ngkâlmpw)}
\define@key{names}{knc}{Kanuri}
\define@key{names}{kny}{Kanyok}
\define@key{names}{pam}{Kapampangan}
\define@key{names}{kpg}{Kapingamarangi}
\define@key{names}{kah}{Kara (in Central African Republic)}
\define@key{names}{leu}{Kara (in Papua New Guinea)}
\define@key{names}{krc}{Karachay-Balkar}
\define@key{names}{gbd}{Karadjeri}
\define@key{names}{kdr}{Karaim}
\define@key{names}{kpj}{Karajá}
\define@key{names}{kaa}{Karakalpak}
\define@key{names}{zkk}{Karankawa}
\define@key{names}{kyj}{Karao}
\define@key{names}{kpt}{Karata}
\define@key{names}{krl}{Karelian}
\define@key{names}{bwe}{Karen (Bwe)}
\define@key{names}{kjp}{Karen (Pwo)}
\define@key{names}{ksw}{Karen (Sgaw)}
\define@key{names}{vka}{Kariera}
\define@key{names}{kdj}{Karimojong}
\define@key{names}{ktn}{Karitiâna}
\define@key{names}{yuj}{Karkar-Yuri}
\define@key{names}{kyh}{Karok}
\define@key{names}{arr}{Karó (Arára)}
\define@key{names}{xsm}{Kasem}
\define@key{names}{kju}{Kashaya}
\define@key{names}{kas}{Kashmiri}
\define@key{names}{csb}{Kashubian}
\define@key{names}{cog}{Kasong}
\define@key{names}{bqy}{Kata Kolok}
\define@key{names}{xtc}{Katcha}
\define@key{names}{bsh}{Kati (in Afghanistan)}
\define@key{names}{kts}{Kati (in West Papua, Indonesia)}
\define@key{names}{kcr}{Katla}
\define@key{names}{ktw}{Kato}
\define@key{names}{pss}{Kaulong}
\define@key{names}{bpp}{Kaure}
\define@key{names}{zku}{Kaurna}
\define@key{names}{xaw}{Kawaiisu}
\define@key{names}{kyz}{Kayabí}
\define@key{names}{eky}{Kayah Li (Eastern)}
\define@key{names}{kys}{Kayan (Baram)}
\define@key{names}{txu}{Kayapó}
\define@key{names}{gyd}{Kayardild}
\define@key{names}{gbb}{Kaytej}
\define@key{names}{kaz}{Kazakh}
\define@key{names}{ksx}{Kedang}
\define@key{names}{kbr}{Kefa}
\define@key{names}{kei}{Kei}
\define@key{names}{kcl}{Kela (Apoze)}
\define@key{names}{kzi}{Kelabit}
\define@key{names}{sbc}{Kele}
\define@key{names}{ahg}{Kemant}
\define@key{names}{kmt}{Kemtuik}
\define@key{names}{kyq}{Kenga}
\define@key{names}{keu}{Kenyah (Uma' Lung)}
\define@key{names}{xki}{Kenyan Sign Language}
\define@key{names}{ken}{Kenyang}
\define@key{names}{xxk}{Keo}
\define@key{names}{ker}{Kera}
\define@key{names}{krk}{Kerek}
\define@key{names}{kee}{Keresan (Santa Ana)}
\define@key{names}{ket}{Ket}
\define@key{names}{xdy}{Ketapang}
\define@key{names}{kcv}{Kete}
\define@key{names}{xte}{Ketengban}
\define@key{names}{kew}{Kewa}
\define@key{names}{kjh}{Khakas}
\define@key{names}{klj}{Khalaj}
\define@key{names}{klr}{Khaling}
\define@key{names}{khk}{Khalkha}
\define@key{names}{kjl}{Kham}
\define@key{names}{khg}{Kham (Dege)}
\define@key{names}{kca}{Khanty}
\define@key{names}{khr}{Kharia}
\define@key{names}{kha}{Khasi}
\define@key{names}{kjj}{Khinalug}
\define@key{names}{khm}{Khmer}
\define@key{names}{kjg}{Khmu'}
\define@key{names}{khw}{Khowar}
\define@key{names}{cnk}{Khumi}
\define@key{names}{khv}{Khwarshi}
\define@key{names}{kkh}{Khün}
\define@key{names}{kic}{Kickapoo}
\define@key{names}{kik}{Kikuyu}
\define@key{names}{hbb}{Kilba}
\define@key{names}{kij}{Kilivila}
\define@key{names}{klb}{Kiliwa}
\define@key{names}{lub}{Kiluba}
\define@key{names}{kig}{Kimaghama}
\define@key{names}{zga}{Kinga}
\define@key{names}{kfk}{Kinnauri}
\define@key{names}{kin}{Kinyarwanda}
\define@key{names}{kio}{Kiowa}
\define@key{names}{kzw}{Kipea}
\define@key{names}{geb}{Kire}
\define@key{names}{kir}{Kirghiz}
\define@key{names}{gil}{Kiribati}
\define@key{names}{kiy}{Kirikiri}
\define@key{names}{cme}{Kirma}
\define@key{names}{kje}{Kisar}
\define@key{names}{kss}{Kisi}
\define@key{names}{gia}{Kitja}
\define@key{names}{kii}{Kitsai}
\define@key{names}{ktu}{Kituba}
\define@key{names}{kjd}{Kiwai (Southern)}
\define@key{names}{kla}{Klamath}
\define@key{names}{klu}{Klao}
\define@key{names}{yak}{Klikitat}
\define@key{names}{kst}{Ko (Winye)}
\define@key{names}{cku}{Koasati}
\define@key{names}{kpw}{Kobon}
\define@key{names}{kfa}{Kodava}
\define@key{names}{xwg}{Koegu}
\define@key{names}{xuo}{Koh}
\define@key{names}{bcs}{Kohumono}
\define@key{names}{kpx}{Koiali (Mountain)}
\define@key{names}{kbk}{Koiari}
\define@key{names}{kqi}{Koita}
\define@key{names}{trp}{Kokborok}
\define@key{names}{kex}{Kokni}
\define@key{names}{kkk}{Kokota}
\define@key{names}{kvv}{Kola}
\define@key{names}{kfb}{Kolami}
\define@key{names}{kvw}{Kolana}
\define@key{names}{shm}{Koluri}
\define@key{names}{bkm}{Kom}
\define@key{names}{xbi}{Kombio}
\define@key{names}{kge}{Komering}
\define@key{names}{koi}{Komi-Permyak}
\define@key{names}{xom}{Komo}
\define@key{names}{kfc}{Konda}
\define@key{names}{kng}{Kongo}
\define@key{names}{kjc}{Konjo}
\define@key{names}{knn}{Konkani}
\define@key{names}{xon}{Konkomba}
\define@key{names}{mjd}{Konkow}
\define@key{names}{kma}{Konni}
\define@key{names}{kyx}{Konua}
\define@key{names}{cou}{Konyagi}
\define@key{names}{kqy}{Koorete}
\define@key{names}{kpr}{Korafe}
\define@key{names}{kqz}{Korana}
\define@key{names}{knk}{Koranko}
\define@key{names}{kor}{Korean}
\define@key{names}{coe}{Koreguaje}
\define@key{names}{kfq}{Korku}
\define@key{names}{kfz}{Koromfe}
\define@key{names}{khe}{Korowai}
\define@key{names}{kpy}{Koryak}
\define@key{names}{kia}{Kosop}
\define@key{names}{kos}{Kosraean}
\define@key{names}{kfe}{Kota}
\define@key{names}{aal}{Kotoko}
\define@key{names}{kff}{Koya}
\define@key{names}{khq}{Koyra Chiini}
\define@key{names}{ses}{Koyraboro Senni}
\define@key{names}{koy}{Koyukon}
\define@key{names}{kpk}{Kpan}
\define@key{names}{xpe}{Kpelle}
\define@key{names}{kpo}{Kposo}
\define@key{names}{xra}{Krahô}
\define@key{names}{kqq}{Krenak}
\define@key{names}{krs}{Kresh}
\define@key{names}{rop}{Kriol (Ngukurr)}
\define@key{names}{kgo}{Krongo}
\define@key{names}{jct}{Krymchak}
\define@key{names}{kry}{Kryz}
\define@key{names}{puo}{Ksingmul}
\define@key{names}{sdm}{Kualan}
\define@key{names}{uwa}{Kugu Nganhcara}
\define@key{names}{kxu}{Kui (in India)}
\define@key{names}{kvd}{Kui (in Indonesia)}
\define@key{names}{kui}{Kuikúro}
\define@key{names}{gvn}{Kuku-Yalanji}
\define@key{names}{mbt}{Kulamanen}
\define@key{names}{dwr}{Kullo}
\define@key{names}{kle}{Kulung}
\define@key{names}{kue}{Kuman}
\define@key{names}{kfy}{Kumauni}
\define@key{names}{kum}{Kumyk}
\define@key{names}{kvn}{Kuna}
\define@key{names}{kun}{Kunama}
\define@key{names}{kup}{Kunimaipa}
\define@key{names}{kjn}{Kunjen}
\define@key{names}{cmn}{Kunming}
\define@key{names}{kto}{Kuot}
\define@key{names}{ckb}{Kurdish (Central)}
\define@key{names}{kmr}{Kurmanji}
\define@key{names}{kru}{Kurukh}
\define@key{names}{kgg}{Kusunda}
\define@key{names}{vkt}{Kutai}
\define@key{names}{gwi}{Kutchin}
\define@key{names}{kut}{Kutenai}
\define@key{names}{thd}{Kuuk Thaayorre}
\define@key{names}{kuy}{Kuuku Ya'u}
\define@key{names}{kxv}{Kuvi}
\define@key{names}{kwd}{Kwaio}
\define@key{names}{kwk}{Kwakw'ala}
\define@key{names}{tnk}{Kwamera}
\define@key{names}{ksq}{Kwami}
\define@key{names}{kwn}{Kwangali}
\define@key{names}{xwa}{Kwaza}
\define@key{names}{kwe}{Kwerba}
\define@key{names}{kmo}{Kwoma}
\define@key{names}{kwo}{Kwomtari}
\define@key{names}{xuu}{Kxoe}
\define@key{names}{kyc}{Kyaka}
\define@key{names}{kgy}{Kyirong}
\define@key{names}{nuk}{Kyuquot}
\define@key{names}{kmg}{Kâte}
\define@key{names}{gdm}{Laal}
\define@key{names}{lbu}{Labu}
\define@key{names}{lac}{Lacandón}
\define@key{names}{lbt}{Lachi}
\define@key{names}{lbj}{Ladakhi}
\define@key{names}{lld}{Ladin}
\define@key{names}{lad}{Ladino}
\define@key{names}{laf}{Lafofa}
\define@key{names}{kot}{Lagwan}
\define@key{names}{lha}{Laha}
\define@key{names}{lhu}{Lahu}
\define@key{names}{cnh}{Lai}
\define@key{names}{lbe}{Lak}
\define@key{names}{lkt}{Lakhota}
\define@key{names}{lbc}{Lakkia}
\define@key{names}{ywt}{Lalo}
\define@key{names}{slp}{Lamaholot}
\define@key{names}{hia}{Lamang}
\define@key{names}{lmn}{Lamani}
\define@key{names}{lam}{Lamba}
\define@key{names}{lmu}{Lamen}
\define@key{names}{lns}{Lamnso'}
\define@key{names}{ljp}{Lampung}
\define@key{names}{lby}{Lamu-Lamu}
\define@key{names}{lme}{Lamé}
\define@key{names}{lag}{Langi}
\define@key{names}{laj}{Lango}
\define@key{names}{fsl}{Langue des Signes Française}
\define@key{names}{fcs}{Langue des Signes Québecoise}
\define@key{names}{lao}{Lao}
\define@key{names}{lrg}{Laragia}
\define@key{names}{lbz}{Lardil}
\define@key{names}{alo}{Larike}
\define@key{names}{lav}{Latvian}
\define@key{names}{llu}{Lau}
\define@key{names}{law}{Lauje}
\define@key{names}{lvk}{Lavukaleve}
\define@key{names}{lzz}{Laz}
\define@key{names}{agh}{Lebeo}
\define@key{names}{lea}{Lega}
\define@key{names}{agb}{Leggbó}
\define@key{names}{lec}{Leko}
\define@key{names}{lln}{Lele}
\define@key{names}{lef}{Lelemi}
\define@key{names}{tnl}{Lenakel}
\define@key{names}{led}{Lendu}
\define@key{names}{enx}{Lengua}
\define@key{names}{aed}{Lengua de Señas Argentina}
\define@key{names}{ssp}{Lengua de Señas Española}
\define@key{names}{lep}{Lepcha}
\define@key{names}{les}{Lese}
\define@key{names}{lti}{Leti}
\define@key{names}{lww}{Lewo}
\define@key{names}{lez}{Lezgian}
\define@key{names}{lhm}{Lhomi}
\define@key{names}{lil}{Lillooet}
\define@key{names}{lif}{Limbu}
\define@key{names}{lmc}{Limilngan}
\define@key{names}{liy}{Linda}
\define@key{names}{lin}{Lingala}
\define@key{names}{ise}{Lingua Italiana dei Segni}
\define@key{names}{lnj}{Linngithig}
\define@key{names}{lis}{Lisu}
\define@key{names}{lit}{Lithuanian}
\define@key{names}{liv}{Liv}
\define@key{names}{lob}{Lobi}
\define@key{names}{log}{Logoti}
\define@key{names}{lok}{Loko}
\define@key{names}{arw}{Lokono}
\define@key{names}{lom}{Loma}
\define@key{names}{bdu}{Londo}
\define@key{names}{lgu}{Longgu}
\define@key{names}{los}{Loniu}
\define@key{names}{crc}{Lonwolwol}
\define@key{names}{njh}{Lotha}
\define@key{names}{loj}{Lou}
\define@key{names}{lbo}{Loven}
\define@key{names}{nds}{Low German}
\define@key{names}{loz}{Lozi}
\define@key{names}{nie}{Lua}
\define@key{names}{ojv}{Luangiua}
\define@key{names}{lch}{Lucazi}
\define@key{names}{lug}{Luganda}
\define@key{names}{lgg}{Lugbara}
\define@key{names}{jos}{Lughat al-Isharat al-Lubnaniya}
\define@key{names}{lui}{Luiseño}
\define@key{names}{ule}{Lule}
\define@key{names}{str}{Lummi}
\define@key{names}{lnd}{Lun Dayeh}
\define@key{names}{lun}{Lunda}
\define@key{names}{luo}{Luo}
\define@key{names}{lrc}{Luri}
\define@key{names}{lut}{Lushootseed}
\define@key{names}{khl}{Lusi}
\define@key{names}{lue}{Luvale}
\define@key{names}{lwo}{Luwo}
\define@key{names}{ltz}{Luxemburgeois}
\define@key{names}{luy}{Luyia}
\define@key{names}{lee}{Lyele}
\define@key{names}{psr}{Língua Gestual Portuguesa}
\define@key{names}{bzs}{Língua de Sinais Brasileira}
\define@key{names}{khb}{Lü}
\define@key{names}{msj}{Ma}
\define@key{names}{mhy}{Ma'anyan}
\define@key{names}{mhi}{Ma'di}
\define@key{names}{slz}{Ma'ya}
\define@key{names}{mdy}{Maale}
\define@key{names}{mas}{Maasai}
\define@key{names}{mde}{Maba}
\define@key{names}{mca}{Maca}
\define@key{names}{mbn}{Macaguán}
\define@key{names}{mkd}{Macedonian}
\define@key{names}{mcb}{Machiguenga}
\define@key{names}{myy}{Macuna}
\define@key{names}{mbc}{Macushi}
\define@key{names}{mxu}{Mada (in Cameroon)}
\define@key{names}{mda}{Mada (in Nigeria)}
\define@key{names}{dmd}{Madimadi}
\define@key{names}{mad}{Madurese}
\define@key{names}{mmw}{Mae}
\define@key{names}{mag}{Magahi}
\define@key{names}{mgp}{Magar}
\define@key{names}{mrd}{Magar (Syangja)}
\define@key{names}{mgu}{Magi}
\define@key{names}{mdh}{Magindanao}
\define@key{names}{mhe}{Mah Meri}
\define@key{names}{xpq}{Mahican}
\define@key{names}{nmu}{Maidu (Northeast)}
\define@key{names}{zrs}{Mairasi}
\define@key{names}{mbq}{Maisin}
\define@key{names}{mai}{Maithili}
\define@key{names}{mpe}{Majang}
\define@key{names}{mcp}{Makaa}
\define@key{names}{myh}{Makah}
\define@key{names}{mkz}{Makasae}
\define@key{names}{mak}{Makassar}
\define@key{names}{mgf}{Maklew}
\define@key{names}{kde}{Makonde}
\define@key{names}{mgh}{Makua}
\define@key{names}{mcm}{Malacca Creole}
\define@key{names}{plt}{Malagasy}
\define@key{names}{mpb}{Malakmalak}
\define@key{names}{zsm}{Malay}
\define@key{names}{zlm}{Malay (Kuala Lumpur)}
\define@key{names}{zmi}{Malay (Ulu Muar)}
\define@key{names}{mal}{Malayalam}
\define@key{names}{mgl}{Maleu}
\define@key{names}{gcc}{Mali}
\define@key{names}{mlt}{Maltese}
\define@key{names}{kmj}{Malto}
\define@key{names}{mam}{Mam}
\define@key{names}{mmn}{Mamanwa}
\define@key{names}{mqj}{Mamasa}
\define@key{names}{mcs}{Mambai}
\define@key{names}{mgr}{Mambwe}
\define@key{names}{maw}{Mampruli}
\define@key{names}{mdi}{Mamvu}
\define@key{names}{xmm}{Manadonese}
\define@key{names}{mva}{Manam}
\define@key{names}{mle}{Manambu}
\define@key{names}{nmm}{Manange}
\define@key{names}{mnc}{Manchu}
\define@key{names}{mid}{Mandaic (Modern)}
\define@key{names}{mhq}{Mandan}
\define@key{names}{mdr}{Mandar}
\define@key{names}{mnk}{Mandinka}
\define@key{names}{jet}{Manem}
\define@key{names}{mna}{Mangap-Mbula}
\define@key{names}{mpc}{Mangarrayi}
\define@key{names}{mdj}{Mangbetu}
\define@key{names}{mqy}{Manggarai}
\define@key{names}{mjg}{Mangghuer}
\define@key{names}{mge}{Mango}
\define@key{names}{emk}{Maninka}
\define@key{names}{mlq}{Maninka (Western)}
\define@key{names}{mfv}{Manjaku}
\define@key{names}{knf}{Mankanya}
\define@key{names}{nge}{Mankon}
\define@key{names}{mev}{Mano}
\define@key{names}{mbb}{Manobo (Western Bukidnon)}
\define@key{names}{mns}{Mansi}
\define@key{names}{glv}{Manx}
\define@key{names}{mri}{Maori}
\define@key{names}{mcg}{Mapoyo}
\define@key{names}{arn}{Mapudungun}
\define@key{names}{mec}{Mara}
\define@key{names}{mrw}{Maranao}
\define@key{names}{zmr}{Maranungku}
\define@key{names}{mar}{Marathi}
\define@key{names}{rnp}{Marchha}
\define@key{names}{zmc}{Margany}
\define@key{names}{mrt}{Margi}
\define@key{names}{mrj}{Mari (Hill)}
\define@key{names}{mhr}{Mari (Meadow)}
\define@key{names}{mrc}{Maricopa}
\define@key{names}{mrz}{Marind}
\define@key{names}{mbw}{Maring}
\define@key{names}{zmt}{Maringarr}
\define@key{names}{mfr}{Marrithiyel}
\define@key{names}{mah}{Marshallese}
\define@key{names}{gcf}{Martinique Creole}
\define@key{names}{vma}{Martuthunira}
\define@key{names}{mhx}{Maru}
\define@key{names}{mcn}{Masa}
\define@key{names}{jle}{Masakin}
\define@key{names}{mls}{Masalit}
\define@key{names}{wam}{Massachusett}
\define@key{names}{mpq}{Matis}
\define@key{names}{zml}{Matngele}
\define@key{names}{mcf}{Matsés}
\define@key{names}{mvb}{Mattole}
\define@key{names}{mjk}{Matukar}
\define@key{names}{mgw}{Matuumbi}
\define@key{names}{mxx}{Mauka}
\define@key{names}{mph}{Maung}
\define@key{names}{mfe}{Mauritian Creole}
\define@key{names}{mke}{Mawchi}
\define@key{names}{mbl}{Maxakalí}
\define@key{names}{yan}{Mayangna}
\define@key{names}{ayz}{Maybrat}
\define@key{names}{xyj}{Mayi-Yapi}
\define@key{names}{mfy}{Mayo}
\define@key{names}{mdm}{Mayogo}
\define@key{names}{maz}{Mazahua}
\define@key{names}{mzn}{Mazanderani}
\define@key{names}{maq}{Mazatec (Chiquihuitlán)}
\define@key{names}{mau}{Mazatec (Huautla)}
\define@key{names}{mfc}{Mba}
\define@key{names}{vmb}{Mbabaram}
\define@key{names}{lnb}{Mbalanhu}
\define@key{names}{mpk}{Mbara}
\define@key{names}{myb}{Mbay}
\define@key{names}{mtk}{Mbe'}
\define@key{names}{mdt}{Mbere}
\define@key{names}{baw}{Mbili}
\define@key{names}{gmm}{Mbodomo}
\define@key{names}{mdq}{Mbole}
\define@key{names}{mdw}{Mbosi}
\define@key{names}{mhd}{Mbugu}
\define@key{names}{mdd}{Mbum}
\define@key{names}{mym}{Me'en}
\define@key{names}{nux}{Mehek}
\define@key{names}{gdq}{Mehri}
\define@key{names}{mni}{Meithei}
\define@key{names}{skf}{Mekens}
\define@key{names}{mek}{Mekeo}
\define@key{names}{mel}{Melanau}
\define@key{names}{bew}{Melayu Betawi}
\define@key{names}{men}{Mende}
\define@key{names}{mez}{Menomini}
\define@key{names}{mwv}{Mentawai}
\define@key{names}{sdo}{Mentuh Tapuh}
\define@key{names}{mcr}{Menya}
\define@key{names}{ulk}{Meryam Mir}
\define@key{names}{mej}{Meyah}
\define@key{names}{mpt}{Mian}
\define@key{names}{crg}{Michif}
\define@key{names}{mic}{Micmac}
\define@key{names}{mei}{Midob}
\define@key{names}{ium}{Mien}
\define@key{names}{mmy}{Migama}
\define@key{names}{mxj}{Miju}
\define@key{names}{msy}{Mikarew}
\define@key{names}{mik}{Mikasuki}
\define@key{names}{mjw}{Mikir}
\define@key{names}{hna}{Mina}
\define@key{names}{min}{Minangkabau}
\define@key{names}{mvn}{Minaveha}
\define@key{names}{xmf}{Mingrelian}
\define@key{names}{mep}{Miriwung}
\define@key{names}{nju}{Mirniny}
\define@key{names}{mrg}{Mising}
\define@key{names}{miq}{Miskito}
\define@key{names}{zmq}{Mituku}
\define@key{names}{csi}{Miwok (Bodega)}
\define@key{names}{csm}{Miwok (Central Sierra)}
\define@key{names}{lmw}{Miwok (Lake)}
\define@key{names}{nsq}{Miwok (Northern Sierra)}
\define@key{names}{pmw}{Miwok (Plains)}
\define@key{names}{skd}{Miwok (Southern Sierra)}
\define@key{names}{mxp}{Mixe (Ayutla)}
\define@key{names}{mco}{Mixe (Coatlán)}
\define@key{names}{mto}{Mixe (Totontepec)}
\define@key{names}{mim}{Mixtec (Alacatlatzala)}
\define@key{names}{mib}{Mixtec (Atatlahuca)}
\define@key{names}{miy}{Mixtec (Ayutla)}
\define@key{names}{mih}{Mixtec (Chayuco)}
\define@key{names}{miz}{Mixtec (Coatzospan)}
\define@key{names}{mxt}{Mixtec (Jamiltepec)}
\define@key{names}{mio}{Mixtec (Jicaltepec)}
\define@key{names}{mig}{Mixtec (Molinos)}
\define@key{names}{mie}{Mixtec (Ocotepec)}
\define@key{names}{mil}{Mixtec (Peñoles)}
\define@key{names}{mjc}{Mixtec (San Juan Colorado)}
\define@key{names}{mks}{Mixtec (Silacayoapan)}
\define@key{names}{mpm}{Mixtec (Yosondúa)}
\define@key{names}{mkf}{Miya}
\define@key{names}{lus}{Mizo}
\define@key{names}{mra}{Mlabri (Minor)}
\define@key{names}{moy}{Moca}
\define@key{names}{omc}{Mochica}
\define@key{names}{moc}{Mocoví}
\define@key{names}{mif}{Mofu-Gudur}
\define@key{names}{mhj}{Moghol}
\define@key{names}{moh}{Mohawk}
\define@key{names}{mov}{Mojave}
\define@key{names}{mkj}{Mokilese}
\define@key{names}{moz}{Mokilko}
\define@key{names}{mbe}{Molala}
\define@key{names}{mso}{Mombum}
\define@key{names}{fqs}{Momu}
\define@key{names}{mqf}{Momuna}
\define@key{names}{mnw}{Mon}
\define@key{names}{ndt}{Mondunga}
\define@key{names}{lol}{Mongo}
\define@key{names}{mog}{Mongondow}
\define@key{names}{mnz}{Moni}
\define@key{names}{mnr}{Mono (in United States)}
\define@key{names}{mte}{Mono-Alu}
\define@key{names}{moe}{Montagnais}
\define@key{names}{mxk}{Monumbo}
\define@key{names}{mos}{Mooré}
\define@key{names}{mop}{Mopan}
\define@key{names}{mhz}{Mor}
\define@key{names}{mok}{Moraori}
\define@key{names}{myv}{Mordvin (Erzya)}
\define@key{names}{mdf}{Mordvin (Moksha)}
\define@key{names}{mor}{Moro}
\define@key{names}{mgd}{Moru}
\define@key{names}{cas}{Mosetén}
\define@key{names}{meu}{Motu}
\define@key{names}{siw}{Motuna}
\define@key{names}{mzp}{Movima}
\define@key{names}{mye}{Mpongwe}
\define@key{names}{akc}{Mpur}
\define@key{names}{dmw}{Mudburra}
\define@key{names}{aoj}{Mufian}
\define@key{names}{sgw}{Muher}
\define@key{names}{bmr}{Muinane}
\define@key{names}{chb}{Muisca}
\define@key{names}{mlm}{Mulao}
\define@key{names}{mzm}{Mumuye}
\define@key{names}{mji}{Mun}
\define@key{names}{mnb}{Muna}
\define@key{names}{mua}{Mundang}
\define@key{names}{mnf}{Mundani}
\define@key{names}{myu}{Mundurukú}
\define@key{names}{mhk}{Mungaka}
\define@key{names}{umu}{Munsee}
\define@key{names}{moj}{Munzombo}
\define@key{names}{mtq}{Muong}
\define@key{names}{sur}{Mupun}
\define@key{names}{mtf}{Murik}
\define@key{names}{mur}{Murle}
\define@key{names}{mwf}{Murrinh-Patha}
\define@key{names}{muz}{Mursi}
\define@key{names}{zmu}{Muruwari}
\define@key{names}{mug}{Musgu}
\define@key{names}{msu}{Musom}
\define@key{names}{hur}{Musqueam}
\define@key{names}{emi}{Mussau}
\define@key{names}{css}{Mutsun}
\define@key{names}{myw}{Muyuw}
\define@key{names}{mwe}{Mwera}
\define@key{names}{mlv}{Mwotlap}
\define@key{names}{xak}{Máku}
\define@key{names}{bzk}{Mískito Coast English Creole}
\define@key{names}{muh}{Mündü}
\define@key{names}{naf}{Nabak}
\define@key{names}{wyy}{Nadroga}
\define@key{names}{mbj}{Nadëb}
\define@key{names}{nfr}{Nafaanra}
\define@key{names}{nbi}{Naga (Mao)}
\define@key{names}{nmf}{Naga (Tangkhul)}
\define@key{names}{nzm}{Naga (Zeme)}
\define@key{names}{nag}{Naga Pidgin}
\define@key{names}{nce}{Nagatman}
\define@key{names}{nll}{Nahali}
\define@key{names}{nhn}{Nahuatl (Central)}
\define@key{names}{ncj}{Nahuatl (Huauchinango)}
\define@key{names}{nhx}{Nahuatl (Mecayapan Isthmus)}
\define@key{names}{ncl}{Nahuatl (Michoacán)}
\define@key{names}{nhm}{Nahuatl (Milpa Alta)}
\define@key{names}{nhp}{Nahuatl (Pajapan)}
\define@key{names}{xpo}{Nahuatl (Pochutla)}
\define@key{names}{azz}{Nahuatl (Sierra de Zacapoaxtla)}
\define@key{names}{nhg}{Nahuatl (Tetelcingo)}
\define@key{names}{ngu}{Nahuatl (Xalitla)}
\define@key{names}{bio}{Nai}
\define@key{names}{nak}{Nakanai}
\define@key{names}{nck}{Nakkara}
\define@key{names}{nal}{Nalik}
\define@key{names}{naq}{Nama}
\define@key{names}{nmb}{Nambas (Big)}
\define@key{names}{nab}{Nambikuára (Southern)}
\define@key{names}{nnm}{Namia}
\define@key{names}{gld}{Nanai}
\define@key{names}{ncb}{Nancowry}
\define@key{names}{nnb}{Nande}
\define@key{names}{niq}{Nandi}
\define@key{names}{sen}{Nanerge}
\define@key{names}{nnk}{Nankina}
\define@key{names}{nnt}{Nanticoke}
\define@key{names}{tvl}{Nanumea}
\define@key{names}{npy}{Napu}
\define@key{names}{npa}{Nar-Phu}
\define@key{names}{nrb}{Nara (in Ethiopia)}
\define@key{names}{nrm}{Narom}
\define@key{names}{nas}{Nasioi}
\define@key{names}{nsk}{Naskapi}
\define@key{names}{ncz}{Natchez}
\define@key{names}{ntm}{Nateni}
\define@key{names}{ntu}{Natügu}
\define@key{names}{nau}{Nauruan}
\define@key{names}{nav}{Navajo}
\define@key{names}{nxq}{Naxi}
\define@key{names}{bud}{Ncàm}
\define@key{names}{nde}{Ndebele}
\define@key{names}{djj}{Ndjébbana}
\define@key{names}{ndz}{Ndogo}
\define@key{names}{ndo}{Ndonga}
\define@key{names}{nmd}{Ndumu}
\define@key{names}{ndv}{Ndut}
\define@key{names}{djk}{Ndyuka}
\define@key{names}{dse}{Nederlandse Gebarentaal}
\define@key{names}{neg}{Negidal}
\define@key{names}{nsn}{Nehan}
\define@key{names}{nee}{Nelemwa}
\define@key{names}{anh}{Nend}
\define@key{names}{yrk}{Nenets}
\define@key{names}{nen}{Nengone}
\define@key{names}{aij}{Neo-Aramaic (Arbel Jewish)}
\define@key{names}{aii}{Neo-Aramaic (Assyrian)}
\define@key{names}{trg}{Neo-Aramaic (Persian Azerbaijan)}
\define@key{names}{npi}{Nepali}
\define@key{names}{pia}{Nevome}
\define@key{names}{nzs}{New Zealand Sign Language}
\define@key{names}{new}{Newar (Dolakha)}
\define@key{names}{ney}{Neyo}
\define@key{names}{nez}{Nez Perce}
\define@key{names}{ntj}{Ngaanyatjarra}
\define@key{names}{nxg}{Ngad'a}
\define@key{names}{nig}{Ngalakan}
\define@key{names}{ngk}{Ngalkbun}
\define@key{names}{sba}{Ngambay}
\define@key{names}{nam}{Ngan'gityemerri}
\define@key{names}{nio}{Nganasan}
\define@key{names}{nid}{Ngandi}
\define@key{names}{nay}{Ngarinyeri}
\define@key{names}{nrk}{Ngarla}
\define@key{names}{nrl}{Ngarluma}
\define@key{names}{nxn}{Ngawun}
\define@key{names}{nbm}{Ngbaka (Ma'bo)}
\define@key{names}{nga}{Ngbaka (Minagende)}
\define@key{names}{ngb}{Ngbandi}
\define@key{names}{niy}{Ngiti}
\define@key{names}{wyb}{Ngiyambaa}
\define@key{names}{ngi}{Ngizim}
\define@key{names}{ngo}{Ngoni}
\define@key{names}{llp}{Nguna}
\define@key{names}{gym}{Ngäbere}
\define@key{names}{nha}{Nhanda}
\define@key{names}{nhr}{Nharo}
\define@key{names}{nia}{Nias}
\define@key{names}{caq}{Nicobarese (Car)}
\define@key{names}{pcm}{Nigerian Pidgin}
\define@key{names}{jsl}{Nihon Shuwa (Japanese Sign Language)}
\define@key{names}{nir}{Nimboran}
\define@key{names}{niz}{Ningil}
\define@key{names}{nsz}{Nisenan}
\define@key{names}{ncg}{Nisgha}
\define@key{names}{dtd}{Nitinaht}
\define@key{names}{num}{Niuafo'ou}
\define@key{names}{niu}{Niuean}
\define@key{names}{cag}{Nivacle}
\define@key{names}{niv}{Nivkh}
\define@key{names}{isi}{Nkem}
\define@key{names}{nko}{Nkonya}
\define@key{names}{cgg}{Nkore-Kiga}
\define@key{names}{fia}{Nobiin}
\define@key{names}{njb}{Nocte}
\define@key{names}{nog}{Noghay}
\define@key{names}{not}{Nomatsiguenga}
\define@key{names}{nhu}{Noni}
\define@key{names}{snf}{Noon}
\define@key{names}{nsl}{Norsk Tegnspråk}
\define@key{names}{nor}{Norwegian}
\define@key{names}{nse}{Nsenga}
\define@key{names}{nto}{Ntomba}
\define@key{names}{nxl}{Nuaulu}
\define@key{names}{kcn}{Nubi}
\define@key{names}{dgl}{Nubian (Dongolese)}
\define@key{names}{xnz}{Nubian (Kunuz)}
\define@key{names}{nus}{Nuer}
\define@key{names}{mbr}{Nukak}
\define@key{names}{nkr}{Nukuoro}
\define@key{names}{nut}{Nung (in Vietnam)}
\define@key{names}{nuy}{Nunggubuyu}
\define@key{names}{nuv}{Nuni (Northern)}
\define@key{names}{iii}{Nuosu}
\define@key{names}{nup}{Nupe}
\define@key{names}{nuf}{Nusu}
\define@key{names}{cbn}{Nyah Kur (Tha Pong)}
\define@key{names}{nly}{Nyamal}
\define@key{names}{now}{Nyambo}
\define@key{names}{tpq}{Nyamkad}
\define@key{names}{nym}{Nyamwezi}
\define@key{names}{nyj}{Nyanga}
\define@key{names}{nyp}{Nyangi}
\define@key{names}{nna}{Nyangumarda}
\define@key{names}{nyt}{Nyawaygi}
\define@key{names}{yly}{Nyelâyu}
\define@key{names}{nyh}{Nyigina}
\define@key{names}{nih}{Nyiha}
\define@key{names}{nyi}{Nyimang}
\define@key{names}{njz}{Nyishi}
\define@key{names}{nyv}{Nyulnyul}
\define@key{names}{nys}{Nyungar}
\define@key{names}{nzk}{Nzakara}
\define@key{names}{ood}{O'odham}
\define@key{names}{afz}{Obokuitai}
\define@key{names}{ann}{Obolo}
\define@key{names}{oca}{Ocaina}
\define@key{names}{oci}{Occitan}
\define@key{names}{ocu}{Ocuilteco}
\define@key{names}{ogb}{Ogbia}
\define@key{names}{ogu}{Ogbronuagum}
\define@key{names}{oyb}{Oi}
\define@key{names}{xal}{Oirat}
\define@key{names}{ojs}{Ojibwa (Severn)}
\define@key{names}{ciw}{Ojibwe (Minnesota)}
\define@key{names}{oka}{Okanagan}
\define@key{names}{opm}{Oksapmin}
\define@key{names}{oku}{Oku}
\define@key{names}{ong}{Olo}
\define@key{names}{plo}{Olutec}
\define@key{names}{omg}{Omagua}
\define@key{names}{oma}{Omaha}
\define@key{names}{aun}{One}
\define@key{names}{one}{Oneida}
\define@key{names}{oon}{Onge}
\define@key{names}{ons}{Ono}
\define@key{names}{ono}{Onondaga}
\define@key{names}{mvf}{Ordos}
\define@key{names}{ore}{Orejón}
\define@key{names}{tag}{Orig}
\define@key{names}{ory}{Oriya}
\define@key{names}{ort}{Oriya (Kotia)}
\define@key{names}{oru}{Ormuri}
\define@key{names}{oac}{Oroch}
\define@key{names}{oaa}{Orok}
\define@key{names}{okv}{Orokaiva}
\define@key{names}{oro}{Orokolo}
\define@key{names}{gax}{Oromo (Boraana)}
\define@key{names}{hae}{Oromo (Harar)}
\define@key{names}{ssn}{Oromo (Waata)}
\define@key{names}{gaz}{Oromo (West-Central)}
\define@key{names}{ury}{Orya}
\define@key{names}{osa}{Osage}
\define@key{names}{oss}{Ossetic}
\define@key{names}{iow}{Oto}
\define@key{names}{otz}{Otomí (Ixtenco)}
\define@key{names}{ote}{Otomí (Mezquital)}
\define@key{names}{otq}{Otomí (Santiago Mexquititlan)}
\define@key{names}{otm}{Otomí (Sierra)}
\define@key{names}{otr}{Otoro}
\define@key{names}{owi}{Owininga}
\define@key{names}{pqa}{Pa'a}
\define@key{names}{drl}{Paakantyi}
\define@key{names}{pma}{Paamese}
\define@key{names}{pac}{Pacoh}
\define@key{names}{pdo}{Padoe}
\define@key{names}{pgu}{Pagu}
\define@key{names}{duf}{Paita}
\define@key{names}{pck}{Paite}
\define@key{names}{pao}{Paiute (Northern)}
\define@key{names}{pwn}{Paiwan}
\define@key{names}{pkn}{Pakanha}
\define@key{names}{pau}{Palauan}
\define@key{names}{pll}{Palaung}
\define@key{names}{plu}{Palikur}
\define@key{names}{fap}{Palor}
\define@key{names}{nad}{Palyku}
\define@key{names}{pmz}{Pame}
\define@key{names}{pmf}{Pamona}
\define@key{names}{pbh}{Panare}
\define@key{names}{kre}{Panará}
\define@key{names}{pag}{Pangasinan}
\define@key{names}{pbr}{Pangwa}
\define@key{names}{pan}{Panjabi}
\define@key{names}{pnw}{Panyjima}
\define@key{names}{pap}{Papiamentu}
\define@key{names}{prk}{Parauk}
\define@key{names}{asa}{Pare}
\define@key{names}{pab}{Paresi}
\define@key{names}{pci}{Parji (Dravidian)}
\define@key{names}{pst}{Pashto}
\define@key{names}{pqm}{Passamaquoddy-Maliseet}
\define@key{names}{ptp}{Patep}
\define@key{names}{gfk}{Patpatar}
\define@key{names}{lae}{Pattani}
\define@key{names}{pwi}{Patwin}
\define@key{names}{plh}{Paulohi}
\define@key{names}{pad}{Paumarí}
\define@key{names}{pwa}{Pawaian}
\define@key{names}{paw}{Pawnee}
\define@key{names}{pay}{Pech}
\define@key{names}{aoc}{Pemon}
\define@key{names}{peg}{Pengo}
\define@key{names}{pip}{Pero}
\define@key{names}{pes}{Persian}
\define@key{names}{pww}{Phlong}
\define@key{names}{pio}{Piapoco}
\define@key{names}{pid}{Piaroa}
\define@key{names}{plg}{Pilagá}
\define@key{names}{piv}{Pileni}
\define@key{names}{pif}{Pingilapese}
\define@key{names}{piu}{Pintupi}
\define@key{names}{ppl}{Pipil}
\define@key{names}{myp}{Pirahã}
\define@key{names}{pir}{Piratapuyo}
\define@key{names}{pib}{Piro}
\define@key{names}{psa}{Pisa}
\define@key{names}{pjt}{Pitjantjatjara}
\define@key{names}{pit}{Pitta Pitta}
\define@key{names}{psd}{Plains-Indians Sign Language}
\define@key{names}{gob}{Playero}
\define@key{names}{fwa}{Po-Ai}
\define@key{names}{pbi}{Podoko}
\define@key{names}{poy}{Pogoro}
\define@key{names}{pon}{Pohnpeian}
\define@key{names}{rwa}{Poko-Rawo}
\define@key{names}{poh}{Pokomchí}
\define@key{names}{pko}{Pokot}
\define@key{names}{pox}{Polabian}
\define@key{names}{pol}{Polish}
\define@key{names}{poo}{Pomo (Central)}
\define@key{names}{peb}{Pomo (Eastern)}
\define@key{names}{pej}{Pomo (Northern)}
\define@key{names}{pom}{Pomo (Southeastern)}
\define@key{names}{pbe}{Popoloca (Metzontla)}
\define@key{names}{poe}{Popoloca (San Juan Atzingo)}
\define@key{names}{pbf}{Popoloca (San Vicente Coyotepec)}
\define@key{names}{poi}{Popoluca (Sierra)}
\define@key{names}{poc}{Poqomam}
\define@key{names}{psw}{Port Sandwich}
\define@key{names}{por}{Portuguese}
\define@key{names}{pot}{Potawatomi}
\define@key{names}{pim}{Powhatan}
\define@key{names}{prn}{Prasuni}
\define@key{names}{pre}{Príncipense}
\define@key{names}{pui}{Puinave}
\define@key{names}{fuc}{Pulaar}
\define@key{names}{nij}{Pulopetak}
\define@key{names}{puw}{Puluwat}
\define@key{names}{pmi}{Pumi}
\define@key{names}{puq}{Puquina}
\define@key{names}{prx}{Purki}
\define@key{names}{tsz}{Purépecha}
\define@key{names}{pbb}{Páez}
\define@key{names}{lkr}{Päri}
\define@key{names}{aar}{Qafar}
\define@key{names}{byx}{Qaqet}
\define@key{names}{alc}{Qawasqar}
\define@key{names}{yum}{Quechan}
\define@key{names}{qxa}{Quechua (Ancash)}
\define@key{names}{quy}{Quechua (Ayacucho)}
\define@key{names}{qvc}{Quechua (Cajamarca)}
\define@key{names}{quh}{Quechua (Cochabamba)}
\define@key{names}{quz}{Quechua (Cuzco)}
\define@key{names}{qug}{Quechua (Ecuadorean)}
\define@key{names}{qub}{Quechua (Huallaga)}
\define@key{names}{qvi}{Quechua (Imbabura)}
\define@key{names}{qvn}{Quechua (Tarma)}
\define@key{names}{quc}{Quiché}
\define@key{names}{qui}{Quileute}
\define@key{names}{rad}{Rade}
\define@key{names}{lml}{Raga}
\define@key{names}{rji}{Raji}
\define@key{names}{ral}{Ralte}
\define@key{names}{rma}{Rama}
\define@key{names}{bod}{Rang Pas}
\define@key{names}{rao}{Rao}
\define@key{names}{rap}{Rapanui}
\define@key{names}{ras}{Rashad}
\define@key{names}{rwo}{Rawa}
\define@key{names}{raw}{Rawang}
\define@key{names}{rej}{Rejang}
\define@key{names}{rmb}{Rembarnga}
\define@key{names}{bfw}{Remo}
\define@key{names}{rel}{Rendille}
\define@key{names}{ren}{Rengao}
\define@key{names}{mnv}{Rennellese}
\define@key{names}{rgr}{Resígaro}
\define@key{names}{tnc}{Retuarã}
\define@key{names}{ran}{Riantana}
\define@key{names}{rkb}{Rikbaktsa}
\define@key{names}{rim}{Rimi}
\define@key{names}{rit}{Ritharngu}
\define@key{names}{rog}{Roglai (Northern)}
\define@key{names}{rmn}{Romani (Bugurdzi)}
\define@key{names}{rmo}{Romani (Burgenland)}
\define@key{names}{rmy}{Romani (Lovari)}
\define@key{names}{rml}{Romani (North Russian)}
\define@key{names}{rmw}{Romani (Welsh)}
\define@key{names}{ron}{Romanian}
\define@key{names}{roh}{Romansch}
\define@key{names}{cla}{Ron}
\define@key{names}{rng}{Ronga}
\define@key{names}{rro}{Roro}
\define@key{names}{twu}{Roti}
\define@key{names}{roo}{Rotokas}
\define@key{names}{rtm}{Rotuman}
\define@key{names}{rug}{Roviana}
\define@key{names}{dru}{Rukai (Tanan)}
\define@key{names}{klq}{Rumu}
\define@key{names}{run}{Rundi}
\define@key{names}{rou}{Runga}
\define@key{names}{nyn}{Runyankore}
\define@key{names}{nyo}{Runyoro-Rutooro}
\define@key{names}{rus}{Russian}
\define@key{names}{rsl}{Russian Sign Language}
\define@key{names}{rut}{Rutul}
\define@key{names}{apb}{Sa'a}
\define@key{names}{snv}{Sa'ban}
\define@key{names}{sma}{Saami (Central-South)}
\define@key{names}{sjd}{Saami (Kildin)}
\define@key{names}{sme}{Saami (Northern)}
\define@key{names}{skb}{Saek}
\define@key{names}{uma}{Sahaptin (Umatilla)}
\define@key{names}{ssy}{Saho}
\define@key{names}{saj}{Sahu}
\define@key{names}{sku}{Sakao}
\define@key{names}{slr}{Salar}
\define@key{names}{sbe}{Saliba (in Papua New Guinea)}
\define@key{names}{sln}{Salinan}
\define@key{names}{slh}{Salish (Southern Puget Sound)}
\define@key{names}{sll}{Salt-Yui}
\define@key{names}{sse}{Sama (Balangingi)}
\define@key{names}{ssb}{Sama (Southern)}
\define@key{names}{ndi}{Samba Leko}
\define@key{names}{smq}{Samo}
\define@key{names}{smo}{Samoan}
\define@key{names}{sad}{Sandawe}
\define@key{names}{sxn}{Sangir}
\define@key{names}{sag}{Sango}
\define@key{names}{snq}{Sangu}
\define@key{names}{sce}{Santa}
\define@key{names}{sat}{Santali}
\define@key{names}{xsu}{Sanuma}
\define@key{names}{spu}{Sapuan}
\define@key{names}{srm}{Saramaccan}
\define@key{names}{srs}{Sarcee}
\define@key{names}{sro}{Sardinian}
\define@key{names}{dju}{Sare}
\define@key{names}{ybe}{Saryg Yughur}
\define@key{names}{sdg}{Savi}
\define@key{names}{svs}{Savosavo}
\define@key{names}{szw}{Sawai}
\define@key{names}{hvn}{Sawu}
\define@key{names}{pos}{Sayula Popoluca}
\define@key{names}{kpz}{Sebei}
\define@key{names}{sey}{Secoya}
\define@key{names}{sed}{Sedang}
\define@key{names}{trv}{Seediq}
\define@key{names}{slu}{Selaru}
\define@key{names}{sly}{Selayar}
\define@key{names}{spl}{Selepet}
\define@key{names}{ona}{Selknam}
\define@key{names}{sel}{Selkup}
\define@key{names}{nsm}{Sema}
\define@key{names}{sea}{Semai}
\define@key{names}{sif}{Seme}
\define@key{names}{sza}{Semelai}
\define@key{names}{seh}{Sena}
\define@key{names}{sef}{Senadi}
\define@key{names}{see}{Seneca}
\define@key{names}{szg}{Sengele}
\define@key{names}{set}{Sentani}
\define@key{names}{hbs}{Serbian-Croatian}
\define@key{names}{sei}{Seri}
\define@key{names}{ser}{Serrano}
\define@key{names}{sot}{Sesotho}
\define@key{names}{crs}{Seychelles Creole}
\define@key{names}{sbf}{Shabo}
\define@key{names}{ksb}{Shambala}
\define@key{names}{shn}{Shan}
\define@key{names}{mcd}{Sharanahua}
\define@key{names}{sht}{Shasta}
\define@key{names}{shj}{Shatt}
\define@key{names}{sjw}{Shawnee}
\define@key{names}{swv}{Shekhawati}
\define@key{names}{sdp}{Sherdukpen}
\define@key{names}{xsr}{Sherpa}
\define@key{names}{shk}{Shilluk}
\define@key{names}{scl}{Shina}
\define@key{names}{bwo}{Shinassha}
\define@key{names}{shp}{Shipibo-Konibo}
\define@key{names}{yuy}{Shira Yughur}
\define@key{names}{shb}{Shiriana}
\define@key{names}{sii}{Shompen}
\define@key{names}{sna}{Shona}
\define@key{names}{cjs}{Shor}
\define@key{names}{shh}{Shoshone}
\define@key{names}{sgh}{Shughni}
\define@key{names}{ryu}{Shuri}
\define@key{names}{shs}{Shuswap}
\define@key{names}{snp}{Siane}
\define@key{names}{sjr}{Siar}
\define@key{names}{sid}{Sidaama}
\define@key{names}{ski}{Sika}
\define@key{names}{tty}{Sikaritai}
\define@key{names}{sip}{Sikkimese}
\define@key{names}{skh}{Sikule}
\define@key{names}{dau}{Sila}
\define@key{names}{smr}{Simeulue}
\define@key{names}{snc}{Sinaugoro}
\define@key{names}{snd}{Sindhi}
\define@key{names}{sin}{Sinhala}
\define@key{names}{xsi}{Sio}
\define@key{names}{snn}{Siona}
\define@key{names}{qum}{Sipakapense}
\define@key{names}{fos}{Siraya}
\define@key{names}{sri}{Siriano}
\define@key{names}{srq}{Sirionó}
\define@key{names}{ssd}{Siroi}
\define@key{names}{sil}{Sisaala}
\define@key{names}{baa}{Sisiqa}
\define@key{names}{sis}{Siuslaw}
\define@key{names}{skv}{Skou}
\define@key{names}{den}{Slave}
\define@key{names}{xsl}{Slavey}
\define@key{names}{slk}{Slovak}
\define@key{names}{slv}{Slovene}
\define@key{names}{teu}{So}
\define@key{names}{sob}{Sobei}
\define@key{names}{gru}{Soddo}
\define@key{names}{evn}{Solon}
\define@key{names}{som}{Somali}
\define@key{names}{sop}{Songe}
\define@key{names}{snk}{Soninke}
\define@key{names}{sov}{Sonsorol-Tobi}
\define@key{names}{sqt}{Soqotri}
\define@key{names}{srb}{Sora}
\define@key{names}{dsb}{Sorbian (Lower)}
\define@key{names}{hsb}{Sorbian (Upper)}
\define@key{names}{nso}{Sotho (Northern)}
\define@key{names}{mnx}{Sougb}
\define@key{names}{kvk}{South Korean Sign Language}
\define@key{names}{tvk}{Southeast Ambrym}
\define@key{names}{wib}{Southern Toussian}
\define@key{names}{spa}{Spanish}
\define@key{names}{spt}{Spitian}
\define@key{names}{spo}{Spokane}
\define@key{names}{squ}{Squamish}
\define@key{names}{srn}{Sranan}
\define@key{names}{kpm}{Sre}
\define@key{names}{sto}{Stoney}
\define@key{names}{sbs}{Subiya}
\define@key{names}{tgo}{Sudest}
\define@key{names}{sue}{Suena}
\define@key{names}{swi}{Sui}
\define@key{names}{sui}{Suki}
\define@key{names}{sub}{Suku}
\define@key{names}{suk}{Sukuma}
\define@key{names}{sua}{Sulka}
\define@key{names}{suv}{Sulung}
\define@key{names}{sun}{Sundanese}
\define@key{names}{sjg}{Sungor}
\define@key{names}{spp}{Supyire}
\define@key{names}{sgz}{Sursurunga}
\define@key{names}{sus}{Susu}
\define@key{names}{sva}{Svan}
\define@key{names}{swl}{Svenska Teckenspråket}
\define@key{names}{swh}{Swahili}
\define@key{names}{ssw}{Swati}
\define@key{names}{swe}{Swedish}
\define@key{names}{slc}{Sáliba (in Colombia)}
\define@key{names}{mky}{Taba}
\define@key{names}{sst}{Tabare}
\define@key{names}{tby}{Tabaru}
\define@key{names}{tab}{Tabassaran}
\define@key{names}{tnm}{Tabla}
\define@key{names}{tap}{Tabwa}
\define@key{names}{tna}{Tacana}
\define@key{names}{tgl}{Tagalog}
\define@key{names}{tbw}{Tagbanwa (Aborlan)}
\define@key{names}{tah}{Tahitian}
\define@key{names}{gpn}{Taiap}
\define@key{names}{sps}{Taiof}
\define@key{names}{tbg}{Tairora}
\define@key{names}{tss}{Taiwanese Sign Language (Ziran Shouyu)}
\define@key{names}{tgk}{Tajik}
\define@key{names}{tkm}{Takelma}
\define@key{names}{tbc}{Takia}
\define@key{names}{tld}{Talaud}
\define@key{names}{tlj}{Talinga}
\define@key{names}{tly}{Talysh (Azerbaijan)}
\define@key{names}{tma}{Tama}
\define@key{names}{mla}{Tamabo}
\define@key{names}{tcg}{Tamagario}
\define@key{names}{taj}{Tamang (Eastern)}
\define@key{names}{taq}{Tamashek}
\define@key{names}{tam}{Tamil}
\define@key{names}{tpm}{Tampulma}
\define@key{names}{tcb}{Tanacross}
\define@key{names}{tfn}{Tanaina}
\define@key{names}{taa}{Tanana (Lower)}
\define@key{names}{tan}{Tangale}
\define@key{names}{skj}{Tangbe}
\define@key{names}{tgg}{Tangga}
\define@key{names}{tpg}{Tanglapui}
\define@key{names}{nwi}{Tanna (Southwest)}
\define@key{names}{tza}{Tanzania Sign Language}
\define@key{names}{tpj}{Tapieté}
\define@key{names}{tar}{Tarahumara (Central)}
\define@key{names}{tac}{Tarahumara (Western)}
\define@key{names}{txn}{Tarangan (West)}
\define@key{names}{tro}{Tarao}
\define@key{names}{tae}{Tariana}
\define@key{names}{yer}{Tarok}
\define@key{names}{shi}{Tashlhiyt}
\define@key{names}{ttt}{Tat (Muslim)}
\define@key{names}{txx}{Tatana'}
\define@key{names}{tat}{Tatar}
\define@key{names}{tks}{Tati (Southern)}
\define@key{names}{tav}{Tatuyo}
\define@key{names}{tuh}{Taulil}
\define@key{names}{trr}{Taushiro}
\define@key{names}{tsg}{Tausug}
\define@key{names}{tya}{Tauya}
\define@key{names}{tbo}{Tawala}
\define@key{names}{cks}{Tayo}
\define@key{names}{tbl}{Tboli}
\define@key{names}{ttc}{Tectiteco}
\define@key{names}{kps}{Tehit}
\define@key{names}{teh}{Tehuelche}
\define@key{names}{kkw}{Teke (Southern)}
\define@key{names}{tlf}{Telefol}
\define@key{names}{tel}{Telugu}
\define@key{names}{kdh}{Tem}
\define@key{names}{teq}{Temein}
\define@key{names}{tea}{Temiar}
\define@key{names}{tem}{Temne}
\define@key{names}{tex}{Tennet}
\define@key{names}{kza}{Tenyer}
\define@key{names}{tio}{Teop}
\define@key{names}{tep}{Tepecano}
\define@key{names}{tee}{Tepehua (Huehuetla)}
\define@key{names}{tpt}{Tepehua (Tlachichilco)}
\define@key{names}{ntp}{Tepehuan (Northern)}
\define@key{names}{stp}{Tepehuan (Southeastern)}
\define@key{names}{ttr}{Tera}
\define@key{names}{tfr}{Teribe}
\define@key{names}{tft}{Ternate}
\define@key{names}{ter}{Terêna}
\define@key{names}{teo}{Teso}
\define@key{names}{tll}{Tetela}
\define@key{names}{tet}{Tetun}
\define@key{names}{tew}{Tewa (Arizona)}
\define@key{names}{tcz}{Thadou}
\define@key{names}{tha}{Thai}
\define@key{names}{tsq}{Thai Sign Language}
\define@key{names}{ths}{Thakali}
\define@key{names}{thf}{Thangmi}
\define@key{names}{ssf}{Thao}
\define@key{names}{typ}{Thaypan}
\define@key{names}{thp}{Thompson}
\define@key{names}{tdh}{Thulung}
\define@key{names}{tca}{Ticuna}
\define@key{names}{tvo}{Tidore}
\define@key{names}{tif}{Tifal}
\define@key{names}{tgc}{Tigak}
\define@key{names}{tir}{Tigrinya}
\define@key{names}{tig}{Tigré}
\define@key{names}{dih}{Tiipay (Jamul)}
\define@key{names}{tik}{Tikar}
\define@key{names}{til}{Tillamook}
\define@key{names}{tms}{Tima}
\define@key{names}{aoz}{Timorese}
\define@key{names}{tjm}{Timucua}
\define@key{names}{tih}{Timugon}
\define@key{names}{lbf}{Tinani}
\define@key{names}{tin}{Tindi}
\define@key{names}{cir}{Tinrin}
\define@key{names}{tri}{Tiriyo}
\define@key{names}{tiy}{Tiruray}
\define@key{names}{tiv}{Tiv}
\define@key{names}{twf}{Tiwa (Northern)}
\define@key{names}{tix}{Tiwa (Southern)}
\define@key{names}{tiw}{Tiwi}
\define@key{names}{tcf}{Tlapanec}
\define@key{names}{tli}{Tlingit}
\define@key{names}{tqo}{Toaripi}
\define@key{names}{tob}{Toba}
\define@key{names}{tti}{Tobati}
\define@key{names}{tlb}{Tobelo}
\define@key{names}{sbu}{Tod}
\define@key{names}{tcx}{Toda}
\define@key{names}{kim}{Tofa}
\define@key{names}{toj}{Tojolabal}
\define@key{names}{tpi}{Tok Pisin}
\define@key{names}{tkl}{Tokelauan}
\define@key{names}{jic}{Tol}
\define@key{names}{ksd}{Tolai}
\define@key{names}{dto}{Tommo So}
\define@key{names}{tdn}{Tondano}
\define@key{names}{toi}{Tonga (in Zambia)}
\define@key{names}{ton}{Tongan}
\define@key{names}{tqw}{Tonkawa}
\define@key{names}{tnt}{Tontemboan}
\define@key{names}{mlu}{Toqabaqita}
\define@key{names}{sda}{Toraja}
\define@key{names}{rth}{Toratán}
\define@key{names}{dts}{Toro So}
\define@key{names}{trw}{Torwali}
\define@key{names}{tlc}{Totonac (Misantla)}
\define@key{names}{top}{Totonac (Papantla)}
\define@key{names}{tos}{Totonac (Sierra)}
\define@key{names}{too}{Totonac (Xicotepec de Juárez)}
\define@key{names}{trs}{Trique (Chicahuaxtla)}
\define@key{names}{trc}{Trique (Copala)}
\define@key{names}{tpy}{Trumai}
\define@key{names}{cof}{Tsafiki}
\define@key{names}{tkr}{Tsakhur}
\define@key{names}{huq}{Tsat}
\define@key{names}{ddo}{Tsez}
\define@key{names}{tsj}{Tshangla}
\define@key{names}{tsi}{Tsimshian (Coast)}
\define@key{names}{tsv}{Tsogo}
\define@key{names}{tso}{Tsonga}
\define@key{names}{tsu}{Tsou}
\define@key{names}{bbl}{Tsova-Tush}
\define@key{names}{tsn}{Tswana}
\define@key{names}{pmt}{Tuamotuan}
\define@key{names}{thz}{Tuareg (Air)}
\define@key{names}{thv}{Tuareg (Ghat)}
\define@key{names}{tbu}{Tubar}
\define@key{names}{tuo}{Tucano}
\define@key{names}{tzn}{Tugun}
\define@key{names}{bag}{Tuki}
\define@key{names}{tcy}{Tulu}
\define@key{names}{tmc}{Tumak}
\define@key{names}{tmq}{Tumleo}
\define@key{names}{tuf}{Tunebo}
\define@key{names}{tvu}{Tunen}
\define@key{names}{lcm}{Tungak}
\define@key{names}{tun}{Tunica}
\define@key{names}{tpn}{Tupi}
\define@key{names}{tui}{Tupuri}
\define@key{names}{tuv}{Turkana}
\define@key{names}{kmz}{Turkic (West Xorasan)}
\define@key{names}{tur}{Turkish}
\define@key{names}{tuk}{Turkmen}
\define@key{names}{tus}{Tuscarora}
\define@key{names}{ttm}{Tutchone (Northern)}
\define@key{names}{tta}{Tutelo}
\define@key{names}{tvt}{Tutsa}
\define@key{names}{tyv}{Tuvan}
\define@key{names}{tue}{Tuyuca}
\define@key{names}{twa}{Twana}
\define@key{names}{woa}{Tyeraity}
\define@key{names}{tzh}{Tzeltal}
\define@key{names}{tzo}{Tzotzil}
\define@key{names}{tzj}{Tzutujil}
\define@key{names}{tub}{Tübatulabal}
\define@key{names}{par}{Tümpisa Shoshone}
\define@key{names}{tsm}{Türk Isaret Dili}
\define@key{names}{umb}{UMbundu}
\define@key{names}{uby}{Ubykh}
\define@key{names}{udi}{Udi}
\define@key{names}{ude}{Udihe}
\define@key{names}{udm}{Udmurt}
\define@key{names}{ugn}{Ugandan Sign Language}
\define@key{names}{ukr}{Ukrainian}
\define@key{names}{ulc}{Ulcha}
\define@key{names}{udl}{Uldeme}
\define@key{names}{uli}{Ulithian}
\define@key{names}{ppk}{Uma}
\define@key{names}{cbd}{Umaua}
\define@key{names}{ubu}{Umbu Ungu}
\define@key{names}{ump}{Umpila}
\define@key{names}{mtg}{Una}
\define@key{names}{unm}{Unami}
\define@key{names}{ung}{Ungarinjin}
\define@key{names}{kuu}{Upper Kuskokwim}
\define@key{names}{uur}{Ura}
\define@key{names}{urf}{Uradhi}
\define@key{names}{urk}{Urak Lawoi'}
\define@key{names}{ura}{Urarina}
\define@key{names}{urt}{Urat}
\define@key{names}{urd}{Urdu}
\define@key{names}{urh}{Urhobo}
\define@key{names}{uri}{Urim}
\define@key{names}{ure}{Uru}
\define@key{names}{uks}{Urubú Sign Language}
\define@key{names}{urb}{Urubú-Kaapor}
\define@key{names}{uum}{Urum}
\define@key{names}{wnu}{Usan}
\define@key{names}{usa}{Usarufa}
\define@key{names}{ute}{Ute}
\define@key{names}{uig}{Uyghur}
\define@key{names}{uzn}{Uzbek (Northern)}
\define@key{names}{vaf}{Vafsi}
\define@key{names}{vag}{Vagla}
\define@key{names}{vai}{Vai}
\define@key{names}{vas}{Vasavi}
\define@key{names}{dic}{Vata}
\define@key{names}{ved}{Vedda}
\define@key{names}{ven}{Venda}
\define@key{names}{vep}{Veps}
\define@key{names}{vie}{Vietnamese}
\define@key{names}{vif}{Vili}
\define@key{names}{vnm}{Vinmavis}
\define@key{names}{vgt}{Vlaamse Gebarentaal}
\define@key{names}{vot}{Votic}
\define@key{names}{wwa}{Waama}
\define@key{names}{wkw}{Wagawaga}
\define@key{names}{waq}{Wagiman}
\define@key{names}{waw}{Wai Wai}
\define@key{names}{wbk}{Waigali}
\define@key{names}{bao}{Waimaha}
\define@key{names}{wbl}{Wakhi}
\define@key{names}{wls}{Wallisian}
\define@key{names}{van}{Walman}
\define@key{names}{wmt}{Walmatjari}
\define@key{names}{wmb}{Wambaya}
\define@key{names}{wms}{Wambon}
\define@key{names}{wme}{Wambule}
\define@key{names}{wan}{Wan}
\define@key{names}{wgg}{Wangkangurru}
\define@key{names}{xwk}{Wangkumara}
\define@key{names}{wbt}{Wanman}
\define@key{names}{wnc}{Wantoat}
\define@key{names}{auc}{Waorani}
\define@key{names}{wap}{Wapishana}
\define@key{names}{wao}{Wappo}
\define@key{names}{wba}{Warao}
\define@key{names}{wrz}{Waray (in Australia)}
\define@key{names}{war}{Waray-Waray}
\define@key{names}{wrr}{Wardaman}
\define@key{names}{gae}{Warekena}
\define@key{names}{wsa}{Warembori}
\define@key{names}{pav}{Wari'}
\define@key{names}{wrs}{Waris}
\define@key{names}{wbp}{Warlpiri}
\define@key{names}{wrb}{Warluwara}
\define@key{names}{wnd}{Warndarang}
\define@key{names}{wrp}{Waropen}
\define@key{names}{wgy}{Warrgamay}
\define@key{names}{gjm}{Warrnambool}
\define@key{names}{wrg}{Warrongo}
\define@key{names}{wwr}{Warrwa}
\define@key{names}{wrm}{Warumungu}
\define@key{names}{was}{Washo}
\define@key{names}{wsk}{Waskia}
\define@key{names}{wax}{Watam}
\define@key{names}{wth}{Wathawurrung}
\define@key{names}{wbv}{Watjarri}
\define@key{names}{noa}{Waunana}
\define@key{names}{wau}{Waurá}
\define@key{names}{oym}{Wayampi}
\define@key{names}{way}{Wayana}
\define@key{names}{wed}{Wedau}
\define@key{names}{cym}{Welsh}
\define@key{names}{xww}{Wembawemba}
\define@key{names}{wer}{Weri}
\define@key{names}{mqs}{West Makian}
\define@key{names}{lex}{Wetan}
\define@key{names}{wic}{Wichita}
\define@key{names}{mzh}{Wichí}
\define@key{names}{wim}{Wik Munkan}
\define@key{names}{wig}{Wik Ngathana}
\define@key{names}{yok}{Wikchamni}
\define@key{names}{win}{Winnebago}
\define@key{names}{wnw}{Wintu}
\define@key{names}{wgu}{Wirangu}
\define@key{names}{wiy}{Wiyot}
\define@key{names}{wob}{Wobe}
\define@key{names}{wog}{Wogamusin}
\define@key{names}{woi}{Woisika}
\define@key{names}{wyu}{Woiwurrung}
\define@key{names}{wal}{Wolaytta}
\define@key{names}{woe}{Woleaian}
\define@key{names}{wlo}{Wolio}
\define@key{names}{wol}{Wolof}
\define@key{names}{wmx}{Womo}
\define@key{names}{wro}{Worora}
\define@key{names}{wuu}{Wu}
\define@key{names}{wya}{Wyandot}
\define@key{names}{wem}{Wéménugbé}
\define@key{names}{kao}{Xasonga}
\define@key{names}{xav}{Xavánte}
\define@key{names}{xer}{Xerénte}
\define@key{names}{xho}{Xhosa}
\define@key{names}{xir}{Xiriana}
\define@key{names}{xok}{Xokleng}
\define@key{names}{ane}{Xârâcùù}
\define@key{names}{yai}{Yaghnobi}
\define@key{names}{yad}{Yagua}
\define@key{names}{yag}{Yahgan}
\define@key{names}{yaf}{Yaka}
\define@key{names}{yka}{Yakan}
\define@key{names}{yky}{Yakoma}
\define@key{names}{sah}{Yakut}
\define@key{names}{ylr}{Yalarnnga}
\define@key{names}{kkl}{Yale (Kosarek)}
\define@key{names}{yli}{Yali}
\define@key{names}{yam}{Yamba}
\define@key{names}{jmd}{Yamdena}
\define@key{names}{tao}{Yami}
\define@key{names}{yaa}{Yaminahua}
\define@key{names}{ybi}{Yamphu}
\define@key{names}{ynn}{Yana}
\define@key{names}{kdd}{Yankuntjatjara}
\define@key{names}{wca}{Yanomámi}
\define@key{names}{yns}{Yansi}
\define@key{names}{jao}{Yanyuwa}
\define@key{names}{yao}{Yao (in Malawi)}
\define@key{names}{yap}{Yapese}
\define@key{names}{jaq}{Yaqay}
\define@key{names}{yaq}{Yaqui}
\define@key{names}{yrb}{Yareba}
\define@key{names}{yae}{Yaruro}
\define@key{names}{yuf}{Yavapai}
\define@key{names}{yva}{Yawa}
\define@key{names}{ywr}{Yawuru}
\define@key{names}{pcc}{Yay}
\define@key{names}{xya}{Yaygir}
\define@key{names}{yah}{Yazgulyam}
\define@key{names}{kpv}{Yazva}
\define@key{names}{jei}{Yei}
\define@key{names}{jel}{Yelmek}
\define@key{names}{yle}{Yelî Dnye}
\define@key{names}{ybb}{Yemba}
\define@key{names}{jnj}{Yemsa}
\define@key{names}{yss}{Yessan-Mayo}
\define@key{names}{yey}{Yeyi}
\define@key{names}{ywq}{Yi (Wuding-Luquan)}
\define@key{names}{ydd}{Yiddish}
\define@key{names}{yii}{Yidiny}
\define@key{names}{yll}{Yil}
\define@key{names}{yee}{Yimas}
\define@key{names}{yij}{Yindjibarndi}
\define@key{names}{yia}{Yingkarta}
\define@key{names}{yyr}{Yir Yoront}
\define@key{names}{xyy}{Yorta Yorta}
\define@key{names}{yor}{Yoruba}
\define@key{names}{yua}{Yucatec}
\define@key{names}{yuc}{Yuchi}
\define@key{names}{ycn}{Yucuna}
\define@key{names}{yug}{Yugh}
\define@key{names}{yux}{Yukaghir (Kolyma)}
\define@key{names}{ykg}{Yukaghir (Tundra)}
\define@key{names}{yuk}{Yuki}
\define@key{names}{yup}{Yukpa}
\define@key{names}{gcd}{Yukulta}
\define@key{names}{mpj}{Yulparija}
\define@key{names}{yul}{Yulu}
\define@key{names}{esu}{Yup'ik (Chevak)}
\define@key{names}{ynk}{Yupik (Naukan)}
\define@key{names}{ess}{Yupik (Siberian)}
\define@key{names}{ysr}{Yupik (Sirenik)}
\define@key{names}{yuz}{Yuracare}
\define@key{names}{yur}{Yurok}
\define@key{names}{yui}{Yuruti}
\define@key{names}{zne}{Zande}
\define@key{names}{zro}{Zaparo}
\define@key{names}{zai}{Zapotec (Isthmus)}
\define@key{names}{zpd}{Zapotec (Ixtlan)}
\define@key{names}{zaa}{Zapotec (Juárez)}
\define@key{names}{zaw}{Zapotec (Mitla)}
\define@key{names}{zpm}{Zapotec (Mixtepec)}
\define@key{names}{zpi}{Zapotec (Quiegolani)}
\define@key{names}{zab}{Zapotec (San Lucas Quiaviní)}
\define@key{names}{zpz}{Zapotec (Texmelucan)}
\define@key{names}{zav}{Zapotec (Yatzachi)}
\define@key{names}{zpq}{Zapotec (Zoogocho)}
\define@key{names}{dje}{Zarma}
\define@key{names}{zay}{Zayse}
\define@key{names}{diq}{Zazaki}
\define@key{names}{zen}{Zenaga}
\define@key{names}{zgb}{Zhuang (Northern)}
\define@key{names}{zik}{Zimakani}
\define@key{names}{zoh}{Zoque (Chimalapa)}
\define@key{names}{zos}{Zoque (Francisco León)}
\define@key{names}{zoc}{Zoque (Ostuacan)}
\define@key{names}{zor}{Zoque (Rayon)}
\define@key{names}{zul}{Zulu}
\define@key{names}{zun}{Zuni}
\define@key{names}{eme}{Émérillon}
\define@key{names}{aom}{Ömie}
\define@key{names}{aas}{Aasax}
\define@key{names}{kbt}{Abadi}
\define@key{names}{abg}{Abaga}
\define@key{names}{abf}{Abai Sungai}
\define@key{names}{abm}{Abanyom}
\define@key{names}{mij}{Mungbam}
\define@key{names}{aba}{Abé}
\define@key{names}{abp}{Abenlen Ayta}
\define@key{names}{bsa}{Abinomn}
\define@key{names}{ash}{Aewa}
\define@key{names}{aob}{Abom}
\define@key{names}{abo}{Abon}
\define@key{names}{abr}{Abron}
\define@key{names}{abn}{Abua}
\define@key{names}{abu}{Abure}
\define@key{names}{mgj}{Abureni}
\define@key{names}{ado}{Abu}
\define@key{names}{tpx}{Acatepec Me'phaa}
\define@key{names}{yif}{Ache}
\define@key{names}{acz}{Acheron}
\define@key{names}{acs}{Acroá}
\define@key{names}{xad}{Adai}
\define@key{names}{ada}{Adangme}
\define@key{names}{adq}{Adangbe}
\define@key{names}{tiu}{Adasen}
\define@key{names}{ade}{Adele}
\define@key{names}{adh}{Adhola}
\define@key{names}{gas}{Adiwasi Garasia}
\define@key{names}{adr}{Adonara}
\define@key{names}{aez}{Aeka}
\define@key{names}{aeq}{Aer}
\define@key{names}{afg}{Afghan Sign Language}
\define@key{names}{aft}{Afitti}
\define@key{names}{afh}{Afrihili}
\define@key{names}{afs}{Afro-Seminole Creole}
\define@key{names}{agi}{Agariya}
\define@key{names}{agc}{Agatu}
\define@key{names}{avo}{Agavotaguerra}
\define@key{names}{ggr}{Aghu Tharnggalu}
\define@key{names}{xag}{Aghwan}
\define@key{names}{aif}{Agi}
\define@key{names}{kit}{Agob-Ende-Kawam}
\define@key{names}{ibm}{Agoi}
\define@key{names}{apf}{Agta-Pahanan}
\define@key{names}{aga}{Aguano}
\define@key{names}{aug}{Aguna}
\define@key{names}{msm}{Agusan Manobo}
\define@key{names}{agn}{Agutaynen}
\define@key{names}{yay}{Agwagwune}
\define@key{names}{aha}{Ahanta}
\define@key{names}{ahn}{Àhàn}
\define@key{names}{esg}{Aheri Gondi}
\define@key{names}{thm}{Thavung}
\define@key{names}{kak}{Ahin-Kayapa Kalanguya}
\define@key{names}{aho}{Ahom}
\define@key{names}{nfd}{Ndunic}
\define@key{names}{aih}{Ai-Cham}
\define@key{names}{aix}{Aighon}
\define@key{names}{mwg}{Aiklep}
\define@key{names}{aiq}{Aimaq}
\define@key{names}{ail}{Aimele}
\define@key{names}{aim}{Aimol}
\define@key{names}{aic}{Ainbai}
\define@key{names}{aki}{Aiome}
\define@key{names}{air}{Airoran}
\define@key{names}{aio}{Aiton}
\define@key{names}{ajw}{Ajawa}
\define@key{names}{cpc}{Ajyíninka Apurucayali}
\define@key{names}{soh}{Aka}
\define@key{names}{akm}{Akabo}
\define@key{names}{akj}{Akajeru}
\define@key{names}{ack}{Akakora}
\define@key{names}{aky}{Akakol}
\define@key{names}{acl}{Akarbale}
\define@key{names}{aks}{Akaselem}
\define@key{names}{aik}{Akye}
\define@key{names}{tsr}{Akei}
\define@key{names}{aeu}{Akeu}
\define@key{names}{sia}{Akkala Saami}
\define@key{names}{akk}{Akkadian}
\define@key{names}{akq}{Ak}
\define@key{names}{akt}{Akolet}
\define@key{names}{bss}{Akoose}
\define@key{names}{miw}{Akoye}
\define@key{names}{akf}{Akpa}
\define@key{names}{ibe}{Akpes}
\define@key{names}{afi}{Chini}
\define@key{names}{ayk}{Akuku}
\define@key{names}{aku}{Akum}
\define@key{names}{aqz}{Akuntsu}
\define@key{names}{ako}{Akurio}
\define@key{names}{dul}{Alabat Island Agta}
\define@key{names}{alw}{Alaba-K'abeena}
\define@key{names}{ala}{Alago}
\define@key{names}{alk}{Alak}
\define@key{names}{alj}{Alangan}
\define@key{names}{apv}{Alapmunte}
\define@key{names}{bhk}{Inland-Buhi-Daraga Bikol}
\define@key{names}{sqk}{Albanian Sign Language}
\define@key{names}{lsc}{Albarradas Sign Language}
\define@key{names}{xta}{Alcozauca Mixtec}
\define@key{names}{alf}{Alege}
\define@key{names}{asp}{Algerian Sign Language}
\define@key{names}{arq}{Algerian Arabic}
\define@key{names}{aao}{Algerian Saharan Arabic}
\define@key{names}{aiy}{Ali}
\define@key{names}{all}{Allar}
\define@key{names}{aid}{Alngith}
\define@key{names}{zaq}{Aloápam Zapotec}
\define@key{names}{ypo}{Alo Phola}
\define@key{names}{aol}{Alorese}
\define@key{names}{syy}{Al-Sayyid Bedouin Sign Language}
\define@key{names}{aub}{Alugu}
\define@key{names}{xua}{Alu Kurumba}
\define@key{names}{aab}{Arum}
\define@key{names}{yna}{Aluo}
\define@key{names}{alz}{Alur}
\define@key{names}{avd}{Alviri-Vidari}
\define@key{names}{amq}{Amahai}
\define@key{names}{ali}{Amaimon}
\define@key{names}{aad}{Amal}
\define@key{names}{jks}{Amami O Shima Sign Language}
\define@key{names}{ama}{Amanayé}
\define@key{names}{amg}{Amurdak}
\define@key{names}{aaz}{Amarasi}
\define@key{names}{zpo}{Amatlán Zapotec}
\define@key{names}{rwm}{Amba (Uganda)}
\define@key{names}{utp}{Amba (Solomon Islands)}
\define@key{names}{abc}{Ambala Ayta}
\define@key{names}{aew}{Ambakich}
\define@key{names}{ael}{Ambele}
\define@key{names}{amv}{Ambelau}
\define@key{names}{alm}{Amblong}
\define@key{names}{amb}{Ambo}
\define@key{names}{abs}{Ambonese Malay}
\define@key{names}{qva}{Ambo-Pasco Quechua}
\define@key{names}{aag}{Ambrak}
\define@key{names}{amj}{Amdang}
\define@key{names}{ifa}{Amganad Ifugao}
\define@key{names}{alx}{Mol}
\define@key{names}{mbz}{Amoltepec Mixtec}
\define@key{names}{aqd}{Ampari Dogon}
\define@key{names}{apg}{Ampanang}
\define@key{names}{ajz}{Amri Karbi}
\define@key{names}{amt}{Amto}
\define@key{names}{adw}{Amundava}
\define@key{names}{anw}{Anaang}
\define@key{names}{akg}{Anakalangu}
\define@key{names}{anm}{Anal}
\define@key{names}{pda}{Anam}
\define@key{names}{aan}{Anambé}
\define@key{names}{dti}{Ana Tinga Dogon}
\define@key{names}{grc}{Ancient Greek}
\define@key{names}{hbo}{Ancient Hebrew}
\define@key{names}{xna}{Ancient North Arabian}
\define@key{names}{xlg}{Ancient Ligurian}
\define@key{names}{hca}{Andaman Creole Hindi}
\define@key{names}{afd}{Andai}
\define@key{names}{aod}{Andarum}
\define@key{names}{ana}{Andaqui}
\define@key{names}{xaa}{Andalusian Arabic}
\define@key{names}{adg}{Andegerebinha}
\define@key{names}{bzb}{Andio}
\define@key{names}{anb}{Andoa}
\define@key{names}{anx}{Andra-Hus}
\define@key{names}{aby}{Aneme Wake}
\define@key{names}{myo}{Anfillo}
\define@key{names}{akh}{Angal Heneng}
\define@key{names}{age}{Angal}
\define@key{names}{aoe}{Angal Enen}
\define@key{names}{aqt}{Angaité}
\define@key{names}{avm}{Angkamuthi}
\define@key{names}{anp}{Angika}
\define@key{names}{rme}{Archaic Angloromani}
\define@key{names}{aog}{Angoram}
\define@key{names}{tnd}{Angosturas Tunebo}
\define@key{names}{blo}{Anii}
\define@key{names}{anf}{Animere}
\define@key{names}{aqk}{Aninka}
\define@key{names}{ypn}{Ani Phowa}
\define@key{names}{boj}{Anjam}
\define@key{names}{aak}{Ankave}
\define@key{names}{amx}{Anmatyerre}
\define@key{names}{anj}{Anor}
\define@key{names}{ans}{Anserma}
\define@key{names}{and}{Ansus}
\define@key{names}{ant}{Antakarinya}
\define@key{names}{xmv}{Antankarana Malagasy}
\define@key{names}{aig}{Antigua and Barbuda Creole English}
\define@key{names}{aui}{Anuki}
\define@key{names}{auq}{Anus}
\define@key{names}{aud}{Anuta}
\define@key{names}{anl}{Anu-Hkongso}
\define@key{names}{mtb}{Anyin Morofo}
\define@key{names}{pni}{Aoheng-Seputan}
\define@key{names}{aor}{Aore}
\define@key{names}{aou}{A'ou}
\define@key{names}{xap}{Apalachee}
\define@key{names}{apo}{Apalik}
\define@key{names}{ena}{Apali}
\define@key{names}{mip}{Apasco-Apoala Mixtec}
\define@key{names}{api}{Apiaká}
\define@key{names}{app}{Apma}
\define@key{names}{apx}{Aputai}
\define@key{names}{arg}{Aragonese}
\define@key{names}{stk}{Arammba}
\define@key{names}{aaf}{Aranadan}
\define@key{names}{xrt}{Aranama}
\define@key{names}{arj}{Arapaso}
\define@key{names}{awm}{Arawum}
\define@key{names}{awt}{Araweté}
\define@key{names}{aae}{Arbëreshë Albanian}
\define@key{names}{aea}{Areba}
\define@key{names}{mwc}{Are}
\define@key{names}{aem}{Arem}
\define@key{names}{qxu}{Arequipa-La Unión Quechua}
\define@key{names}{agj}{Argobba}
\define@key{names}{agf}{Arguni}
\define@key{names}{aqr}{Arhâ}
\define@key{names}{aok}{Arhö}
\define@key{names}{ylu}{Aribwaung}
\define@key{names}{aai}{Arifama-Miniafia}
\define@key{names}{aqg}{Arigidi}
\define@key{names}{aac}{Ari}
\define@key{names}{ait}{Arikem}
\define@key{names}{ark}{Arikapú}
\define@key{names}{xrn}{Arin}
\define@key{names}{luc}{Aringa}
\define@key{names}{dth}{Aritinngitigh}
\define@key{names}{aoh}{Arma}
\define@key{names}{aen}{Armenian Sign Language}
\define@key{names}{rup}{Aromanian}
\define@key{names}{aps}{Arop-Sissano}
\define@key{names}{atz}{Arta}
\define@key{names}{arx}{Aruá (Rondonia State)}
\define@key{names}{aru}{Aruá (Amazonas State)}
\define@key{names}{aur}{Aruek}
\define@key{names}{lsr}{Srenge}
\define@key{names}{atx}{Arutani}
\define@key{names}{aat}{Arvanitika Albanian}
\define@key{names}{mtv}{Asaro'o}
\define@key{names}{cni}{Asháninka}
\define@key{names}{ahs}{Ashe}
\define@key{names}{prq}{Ashéninka Perené}
\define@key{names}{ask}{Ashkun}
\define@key{names}{atn}{Ashtiani}
\define@key{names}{asl}{Asilulu}
\define@key{names}{eiv}{Askopan}
\define@key{names}{asv}{Asoa}
\define@key{names}{asb}{Assiniboine}
\define@key{names}{asz}{As}
\define@key{names}{aua}{Asumboa}
\define@key{names}{aum}{Asu (Nigeria)}
\define@key{names}{zoo}{Asunción Mixtepec Zapotec}
\define@key{names}{asr}{Asuri}
\define@key{names}{atm}{Ata}
\define@key{names}{amz}{Atampaya}
\define@key{names}{atd}{Ata Manobo}
\define@key{names}{ate}{Mand}
\define@key{names}{atk}{Ati}
\define@key{names}{aqm}{Atohwaim}
\define@key{names}{aot}{Atong (India)}
\define@key{names}{ato}{Atong}
\define@key{names}{aox}{Atorada}
\define@key{names}{cch}{Atsam}
\define@key{names}{atc}{Atsahuaca}
\define@key{names}{pkr}{Attapady Kurumba}
\define@key{names}{ati}{Attié}
\define@key{names}{kud}{'Auhelawa}
\define@key{names}{aux}{Aurê y Aurá}
\define@key{names}{auh}{Aushi}
\define@key{names}{avs}{Aushiri}
\define@key{names}{asq}{Austrian Sign Language}
\define@key{names}{asw}{Australian Aborigines Sign Language}
\define@key{names}{aut}{Austral}
\define@key{names}{smf}{Auwe}
\define@key{names}{auu}{Auye}
\define@key{names}{auo}{Auyokawa}
\define@key{names}{avv}{Avá-Canoeiro}
\define@key{names}{avb}{Avau}
\define@key{names}{ave}{Avestan}
\define@key{names}{awk}{Awabakal}
\define@key{names}{vwa}{Lavia-Awalai-Damangnuo Awa}
\define@key{names}{bcu}{Awad Bing}
\define@key{names}{awo}{Awak}
\define@key{names}{awx}{Awara}
\define@key{names}{aya}{Awar}
\define@key{names}{awh}{Awbono}
\define@key{names}{bob}{Aweer}
\define@key{names}{awr}{Awera}
\define@key{names}{awe}{Awetí}
\define@key{names}{azo}{Awing}
\define@key{names}{auj}{Awjilah}
\define@key{names}{aww}{Auwon}
\define@key{names}{afu}{Awutu}
\define@key{names}{yiu}{Southern Awu (Lope)}
\define@key{names}{ahb}{Axamb}
\define@key{names}{yix}{Axi Yi}
\define@key{names}{ayd}{Yintyinka-Ayabadhu}
\define@key{names}{vmy}{Ayautla Mazatec}
\define@key{names}{aye}{Ayere}
\define@key{names}{ayq}{Ayi (Papua New Guinea)}
\define@key{names}{yyz}{Ayizi}
\define@key{names}{ayb}{Ayizo Gbe}
\define@key{names}{zaf}{Ayoquesco Zapotec}
\define@key{names}{ayu}{Ayu}
\define@key{names}{aza}{Azha}
\define@key{names}{yiz}{Azhe}
\define@key{names}{tpc}{Azoyú Me'phaa}
\define@key{names}{bvj}{Baan}
\define@key{names}{bqx}{Baangi}
\define@key{names}{bbm}{Babango}
\define@key{names}{bbw}{Baba}
\define@key{names}{bbk}{Babanki}
\define@key{names}{mbf}{Baba Malay}
\define@key{names}{bcr}{Witsuwit'en-Babine}
\define@key{names}{bzg}{Babuza}
\define@key{names}{btj}{Bacanese Malay}
\define@key{names}{bcy}{Bacama}
\define@key{names}{xbc}{Bactrian}
\define@key{names}{bau}{Bada (Nigeria)}
\define@key{names}{bhz}{Bada (Indonesia)}
\define@key{names}{bdz}{Badeshi}
\define@key{names}{jbi}{Badjirri}
\define@key{names}{bac}{Badui}
\define@key{names}{pbp}{Jaad-Badyara}
\define@key{names}{bvd}{Baeggu}
\define@key{names}{bvc}{Baelelea}
\define@key{names}{btr}{Baetora}
\define@key{names}{bwt}{Bafaw-Balong}
\define@key{names}{bfj}{Bafanji}
\define@key{names}{bmd}{Baga Manduri}
\define@key{names}{bgo}{Baga Koga}
\define@key{names}{bcg}{Pukur}
\define@key{names}{bfy}{Bagheli}
\define@key{names}{fui}{Bagirmi Fulfulde}
\define@key{names}{bqg}{Bago-Kusuntu}
\define@key{names}{bqb}{Bagusa}
\define@key{names}{bpi}{Bagupi}
\define@key{names}{yha}{Baha Buyang}
\define@key{names}{bhv}{Bahau}
\define@key{names}{bah}{Bahamas Creole English}
\define@key{names}{bhj}{Bahing}
\define@key{names}{bsu}{Bahonsuai}
\define@key{names}{bbf}{Baibai}
\define@key{names}{bdj}{Bai}
\define@key{names}{bkx}{Baikeno}
\define@key{names}{bqh}{Baima}
\define@key{names}{bmx}{Baimak}
\define@key{names}{bab}{Bainounk-Gujaher}
\define@key{names}{bcz}{Bainouk-Gunyaamolo-Gutobor}
\define@key{names}{fah}{Baissa Fali}
\define@key{names}{bjs}{Bajan}
\define@key{names}{bjm}{Bajelani}
\define@key{names}{bqz}{Bakaka}
\define@key{names}{bqi}{Bakhtiari}
\define@key{names}{bki}{Baki}
\define@key{names}{bkh}{Bakoko}
\define@key{names}{kme}{Bakole}
\define@key{names}{bbs}{Bakpinka}
\define@key{names}{bkr}{Bakumpai}
\define@key{names}{bjw}{Bakwé}
\define@key{names}{ble}{Balanta-Kentohe}
\define@key{names}{bjt}{Balanta-Ganja}
\define@key{names}{bls}{Balaesang}
\define@key{names}{bdn}{Baldemu}
\define@key{names}{bcn}{Bali (Nigeria)}
\define@key{names}{bcp}{Bali (Democratic Republic of Congo)}
\define@key{names}{mhp}{Balinese Malay}
\define@key{names}{bgx}{Rumelian Turkish}
\define@key{names}{biz}{Loi-Likila}
\define@key{names}{bqo}{Balo}
\define@key{names}{blq}{Paluai}
\define@key{names}{bog}{Langue de Signes Malienne}
\define@key{names}{bbq}{Bamali}
\define@key{names}{myf}{Bambassi}
\define@key{names}{bmo}{Bambalang}
\define@key{names}{bce}{Bamenyam}
\define@key{names}{bqt}{Bamukumbit}
\define@key{names}{bvm}{Bamunka}
\define@key{names}{bcf}{Bamu}
\define@key{names}{bmg}{Bamwe}
\define@key{names}{bjx}{Banao Itneg}
\define@key{names}{byz}{Banaro}
\define@key{names}{bqj}{Bandial}
\define@key{names}{bqk}{Banda-Mbrès}
\define@key{names}{bpd}{Banda-Banda}
\define@key{names}{bfl}{Banda-Ndélé}
\define@key{names}{yaj}{Banda-Yangere}
\define@key{names}{bpq}{Banda Malay}
\define@key{names}{bnd}{Banda (Indonesia)}
\define@key{names}{bbe}{Bangba}
\define@key{names}{bgf}{Ngombe-Bangandu}
\define@key{names}{bsj}{Bangwinji}
\define@key{names}{bnx}{Bangubangu}
\define@key{names}{bxg}{Bangala}
\define@key{names}{bgj}{Bangolan}
\define@key{names}{mfb}{Bangka}
\define@key{names}{bjn}{Banjar}
\define@key{names}{bfk}{Ban Khor Sign Language}
\define@key{names}{bxw}{Bankagooma}
\define@key{names}{dbw}{Bankan Tey Dogon}
\define@key{names}{bap}{Bantawa}
\define@key{names}{bno}{Bantoanon}
\define@key{names}{bfx}{Bantayanon}
\define@key{names}{brd}{Baraamu}
\define@key{names}{bbg}{Barama}
\define@key{names}{baj}{Barakai}
\define@key{names}{bhr}{Bara Malagasy}
\define@key{names}{brs}{Baras}
\define@key{names}{brp}{Barapasi}
\define@key{names}{bmz}{Baramu}
\define@key{names}{bpb}{Barbacoas}
\define@key{names}{gry}{Barclayville Grebo}
\define@key{names}{bva}{Barain}
\define@key{names}{bxo}{Barikanchi}
\define@key{names}{bch}{Bariai}
\define@key{names}{bjc}{Bariji}
\define@key{names}{jbk}{Barikewa}
\define@key{names}{bbi}{Barombi}
\define@key{names}{bjk}{Barok}
\define@key{names}{bpt}{Barrow Point}
\define@key{names}{tbn}{Barro Negro Tunebo}
\define@key{names}{bjz}{Baruga}
\define@key{names}{bwg}{Barwe}
\define@key{names}{bjf}{Barzani Jewish Neo-Aramaic}
\define@key{names}{bsl}{Basa-Gumna}
\define@key{names}{buj}{Basa-Gurmana}
\define@key{names}{bzw}{Basa (Nigeria)}
\define@key{names}{bdb}{Basap}
\define@key{names}{byq}{Basay}
\define@key{names}{bsg}{Bashkardi}
\define@key{names}{bst}{Basketo}
\define@key{names}{bsr}{Bassa-Kontagora}
\define@key{names}{bsi}{Bassossi}
\define@key{names}{bnm}{Batanga}
\define@key{names}{bts}{Batak Simalungun}
\define@key{names}{akb}{Batak Angkola}
\define@key{names}{btm}{Batak Mandailing}
\define@key{names}{btd}{Batak Dairi}
\define@key{names}{ayt}{Bataan Ayta}
\define@key{names}{bta}{Bata}
\define@key{names}{btv}{Bateri}
\define@key{names}{btq}{Batek}
\define@key{names}{btc}{Bati (Cameroon)}
\define@key{names}{bvt}{Bati (Indonesia)}
\define@key{names}{btu}{Batu}
\define@key{names}{bay}{Batuley}
\define@key{names}{zbt}{Batui}
\define@key{names}{sne}{Bau-Jagoi Bidayuh}
\define@key{names}{bsf}{Bauchi}
\define@key{names}{bge}{Bauria}
\define@key{names}{bxa}{Bauro}
\define@key{names}{bwk}{Bauwaki}
\define@key{names}{bjy}{Bayali}
\define@key{names}{bvy}{Baybayanon}
\define@key{names}{byg}{Baygo}
\define@key{names}{mkq}{Bay Miwok}
\define@key{names}{bda}{Kugere-Kuxinge}
\define@key{names}{byl}{Bayono}
\define@key{names}{bfr}{Bazigar}
\define@key{names}{beo}{Beami}
\define@key{names}{bea}{Beaver}
\define@key{names}{bfp}{Beba}
\define@key{names}{beb}{Bebele}
\define@key{names}{bzv}{Bebe}
\define@key{names}{bek}{Bebeli}
\define@key{names}{bxp}{Bebil}
\define@key{names}{tnr}{Bedik}
\define@key{names}{bjv}{Nangnda}
\define@key{names}{bed}{Bedoanas}
\define@key{names}{bkf}{Beeke}
\define@key{names}{bxq}{Beele}
\define@key{names}{bnz}{Beezen}
\define@key{names}{bby}{Menchum}
\define@key{names}{bqv}{Begbere-Ejar}
\define@key{names}{bei}{Riuk Bekati'}
\define@key{names}{bkv}{Bekwarra}
\define@key{names}{bkw}{Bekwil}
\define@key{names}{bvi}{Belanda Viri}
\define@key{names}{bxb}{Belanda Bor}
\define@key{names}{beg}{Lemeting}
\define@key{names}{blm}{Beli (South Sudan)}
\define@key{names}{bey}{Beli (Papua New Guinea)}
\define@key{names}{bzj}{Belize Kriol English}
\define@key{names}{brw}{Bellari}
\define@key{names}{glb}{Belneng}
\define@key{names}{bmb}{Bembe}
\define@key{names}{yun}{Bena (Nigeria)}
\define@key{names}{bez}{Bena (Tanzania)}
\define@key{names}{bdp}{Bende}
\define@key{names}{bct}{Bendi}
\define@key{names}{bgy}{Benggoi}
\define@key{names}{bnu}{Bentong}
\define@key{names}{dbt}{Ben Tey Dogon}
\define@key{names}{byd}{Benyadu'}
\define@key{names}{bie}{Bepour}
\define@key{names}{bxv}{Berakou}
\define@key{names}{bve}{Berau Malay}
\define@key{names}{bit}{Berinomo}
\define@key{names}{byt}{Berti}
\define@key{names}{bes}{Besme}
\define@key{names}{bep}{Besoa}
\define@key{names}{bfe}{Betaf}
\define@key{names}{byf}{Bete (Yukubenic)}
\define@key{names}{btt}{Bete-Bendi}
\define@key{names}{eot}{Beti (Côte d'Ivoire)}
\define@key{names}{bhd}{Bhadrawahi}
\define@key{names}{bha}{Bharia}
\define@key{names}{bht}{Bhattiyali}
\define@key{names}{bgw}{Bhatri}
\define@key{names}{bhe}{Bhaya}
\define@key{names}{bhy}{Bhele}
\define@key{names}{bhi}{Bhilali}
\define@key{names}{nes}{Bhoti Kinnauri}
\define@key{names}{bhu}{Bhunjia}
\define@key{names}{bdf}{Biage}
\define@key{names}{beh}{Biali}
\define@key{names}{bpv}{Bian Marind}
\define@key{names}{big}{Biangai}
\define@key{names}{byk}{Shidong Biao}
\define@key{names}{bje}{Biao-Jiao Mien}
\define@key{names}{bmt}{Biao Mon}
\define@key{names}{bym}{Bidyara}
\define@key{names}{bjg}{Kanyaki-Kagbaaga-Kajoko Bidyogo}
\define@key{names}{bmc}{Biem}
\define@key{names}{bnk}{Bierebo}
\define@key{names}{brj}{Bieria}
\define@key{names}{biu}{Biete}
\define@key{names}{xbe}{Bigambal}
\define@key{names}{bhc}{Biga}
\define@key{names}{ibh}{Bih}
\define@key{names}{jbm}{Bijim}
\define@key{names}{bix}{Bijori}
\define@key{names}{byb}{Bikya}
\define@key{names}{kfs}{Bilaspuri}
\define@key{names}{bql}{Karen}
\define@key{names}{brz}{Bilibil}
\define@key{names}{bpz}{Bilba}
\define@key{names}{bil}{Bile}
\define@key{names}{bms}{Bilma Kanuri}
\define@key{names}{bxf}{Bilur}
\define@key{names}{bhl}{Bimin}
\define@key{names}{byj}{Bina (Nigeria)}
\define@key{names}{bmn}{Bina (Papua New Guinea)}
\define@key{names}{bxz}{Binahari-Neme}
\define@key{names}{bon}{Bine}
\define@key{names}{bpj}{Binji}
\define@key{names}{itb}{Binongan Itneg}
\define@key{names}{bne}{Bintauna}
\define@key{names}{bny}{Bintulu}
\define@key{names}{biq}{Bipi}
\define@key{names}{bxe}{Ongota}
\define@key{names}{brr}{Birao}
\define@key{names}{btf}{Birgit}
\define@key{names}{biy}{Birhor}
\define@key{names}{bqq}{Biritai}
\define@key{names}{brk}{Birked}
\define@key{names}{brl}{Birwa}
\define@key{names}{ije}{Biseni}
\define@key{names}{bpy}{Bishnupriya Manipuri}
\define@key{names}{bwh}{Bishuo}
\define@key{names}{bnw}{Bisis}
\define@key{names}{bir}{Bisorio}
\define@key{names}{bzi}{Bisu}
\define@key{names}{brt}{Bitare}
\define@key{names}{bgk}{Bit}
\define@key{names}{mcc}{Bitur}
\define@key{names}{bwm}{Biwat}
\define@key{names}{byo}{Biyo}
\define@key{names}{bpm}{Biyom}
\define@key{names}{blp}{Blablanga}
\define@key{names}{bfh}{Mblafe-Ránmo}
\define@key{names}{beu}{Blagar}
\define@key{names}{blr}{Blang}
\define@key{names}{zbl}{Blissymbols}
\define@key{names}{bzn}{Boano (Maluku)}
\define@key{names}{bzl}{Boano (Sulawesi)}
\define@key{names}{bty}{Bobot}
\define@key{names}{bgb}{Bobongko}
\define@key{names}{bdv}{Bodo Parja}
\define@key{names}{boy}{Bodo (Central African Republic)}
\define@key{names}{bff}{Bofi}
\define@key{names}{boq}{Bogaya}
\define@key{names}{bvw}{Boga}
\define@key{names}{bux}{Boghom}
\define@key{names}{bqu}{Boguru}
\define@key{names}{bhn}{Gardabani Bohtan Neo-Aramaic}
\define@key{names}{ybk}{Bokha}
\define@key{names}{bdt}{Bokoto}
\define@key{names}{bkp}{Boko (Democratic Republic of Congo)}
\define@key{names}{bus}{Bokobaru}
\define@key{names}{bky}{Bokyi}
\define@key{names}{bnp}{Bola}
\define@key{names}{bld}{Bolango}
\define@key{names}{xbo}{Bolgarian}
\define@key{names}{bvo}{Bolgo}
\define@key{names}{bvl}{Bolivian Sign Language}
\define@key{names}{smk}{Bolinao}
\define@key{names}{blv}{Kibala}
\define@key{names}{bkt}{Boloki}
\define@key{names}{bzm}{Bolondo}
\define@key{names}{bof}{Bolon}
\define@key{names}{blj}{Bolongan}
\define@key{names}{ply}{Bolyu}
\define@key{names}{boh}{Boma Yumu}
\define@key{names}{bml}{Bomboli-Bozaba}
\define@key{names}{bws}{Bomboma}
\define@key{names}{zmx}{Bomitaba}
\define@key{names}{bmf}{Bom-Kim}
\define@key{names}{bmq}{Bomu}
\define@key{names}{bmw}{Bomwali}
\define@key{names}{kzc}{Bondoukou Kulango}
\define@key{names}{bou}{Bondei}
\define@key{names}{dbu}{Najamba-Kindige}
\define@key{names}{bna}{Bonerate}
\define@key{names}{bnv}{Bonerif}
\define@key{names}{glc}{Bon Gula}
\define@key{names}{bui}{Bongili}
\define@key{names}{bpg}{Bonggo}
\define@key{names}{bok}{Impfondo}
\define@key{names}{bvg}{Bonkeng}
\define@key{names}{bop}{Bonkiman}
\define@key{names}{bnb}{Bookan}
\define@key{names}{bnl}{Boon}
\define@key{names}{bvf}{Boor}
\define@key{names}{bpw}{Bo (Papua New Guinea)}
\define@key{names}{gai}{Borei}
\define@key{names}{fue}{Borgu Fulfulde}
\define@key{names}{ksr}{Borong}
\define@key{names}{xxb}{Boro}
\define@key{names}{mae}{Bo-Rukul}
\define@key{names}{bwf}{Boselewa}
\define@key{names}{bqs}{Bosngun}
\define@key{names}{bmj}{Bote}
\define@key{names}{bph}{Botlikh}
\define@key{names}{sbl}{Botolan Sambal}
\define@key{names}{nku}{Bouna Kulango}
\define@key{names}{mux}{Bo-Ung}
\define@key{names}{suo}{Bouni-Bobe}
\define@key{names}{kxr}{Manus Koro}
\define@key{names}{aof}{Bragat}
\define@key{names}{bra}{Braj}
\define@key{names}{kvl}{Brek Karen}
\define@key{names}{buq}{Barem}
\define@key{names}{brq}{Breri}
\define@key{names}{rib}{Bribri Sign Language}
\define@key{names}{bzt}{Brithenig}
\define@key{names}{sgt}{Brokpake}
\define@key{names}{bro}{Dur Brokkat}
\define@key{names}{bpl}{Broome Pearling Lugger Pidgin}
\define@key{names}{plw}{Brooke's Point Palawano}
\define@key{names}{kxd}{Brunei}
\define@key{names}{bsb}{Brunei Bisaya-Dusun}
\define@key{names}{rnb}{Brunca Sign Language}
\define@key{names}{bub}{Bua}
\define@key{names}{cbl}{Bualkhaw Chin}
\define@key{names}{box}{Buamu}
\define@key{names}{buw}{Bubi}
\define@key{names}{stt}{Budeh Stieng}
\define@key{names}{btp}{Budibud}
\define@key{names}{bdx}{Budong-Budong}
\define@key{names}{bja}{Budza}
\define@key{names}{bbh}{Bugan}
\define@key{names}{buk}{Bugawac}
\define@key{names}{bgt}{Bughotu}
\define@key{names}{bku}{Buhid}
\define@key{names}{bxh}{Buhutu}
\define@key{names}{byh}{Bujhyal}
\define@key{names}{bvk}{Bukat}
\define@key{names}{bhh}{Bukharic}
\define@key{names}{bvu}{Bukit Malay}
\define@key{names}{bkn}{Bukitan}
\define@key{names}{tkb}{Buksa}
\define@key{names}{buz}{Bukwen}
\define@key{names}{bqn}{Bulgarian Sign Language}
\define@key{names}{bmp}{Bulgebi}
\define@key{names}{buy}{Bullom So}
\define@key{names}{sti}{Bulo Stieng}
\define@key{names}{bjl}{Bulu (Papua New Guinea)}
\define@key{names}{byp}{Bumaji}
\define@key{names}{aon}{Bumbita Arapesh}
\define@key{names}{bmv}{Bum}
\define@key{names}{kjz}{Bumthangkha}
\define@key{names}{bwx}{Bu-Nao Bunu}
\define@key{names}{bdd}{Bunama}
\define@key{names}{bvn}{Buna}
\define@key{names}{bfn}{Bunak}
\define@key{names}{bns}{Bundeli}
\define@key{names}{bqd}{Bung}
\define@key{names}{xbg}{Bunganditj}
\define@key{names}{wun}{Bungu}
\define@key{names}{bkz}{Bungku}
\define@key{names}{but}{Bungain}
\define@key{names}{buv}{Bun}
\define@key{names}{dgb}{Bunoge Dogon}
\define@key{names}{bnn}{Bunun}
\define@key{names}{blf}{Buol}
\define@key{names}{bys}{Burak}
\define@key{names}{bti}{Burate}
\define@key{names}{bxn}{Burduna}
\define@key{names}{bvh}{Bure}
\define@key{names}{pyx}{Burma Pyu}
\define@key{names}{vrt}{Burmbar}
\define@key{names}{bzu}{Burmeso}
\define@key{names}{bqw}{Buru-Angwe}
\define@key{names}{bdi}{Northern Burun}
\define@key{names}{bqr}{Burusu}
\define@key{names}{aip}{Burumakok}
\define@key{names}{asi}{Buruwai}
\define@key{names}{bry}{Burui}
\define@key{names}{bxs}{Busam}
\define@key{names}{bsm}{Busami}
\define@key{names}{bfg}{Busang Kayan}
\define@key{names}{buc}{Kibosy Kiantalaotsy-Majunga}
\define@key{names}{bup}{Busoa}
\define@key{names}{dox}{Bussa}
\define@key{names}{bju}{Busuu}
\define@key{names}{kyb}{Butbut Kalinga}
\define@key{names}{bnr}{Farafi}
\define@key{names}{btw}{Butuanon}
\define@key{names}{jid}{Bu}
\define@key{names}{bhs}{Buwal}
\define@key{names}{jiy}{Buyuan Jinuo}
\define@key{names}{byi}{Buyu}
\define@key{names}{bww}{Bwa}
\define@key{names}{bwd}{Bwaidoka}
\define@key{names}{tte}{Bwanabwana}
\define@key{names}{bwa}{Bwatoo}
\define@key{names}{bwl}{Bwela}
\define@key{names}{bwc}{Bwile}
\define@key{names}{bwz}{Bwisi}
\define@key{names}{mkk}{Byep-Besep}
\define@key{names}{msq}{Caac}
\define@key{names}{cbb}{Cabiyarí}
\define@key{names}{ccr}{Cacaopera}
\define@key{names}{miu}{Cacaloxtepec Mixtec}
\define@key{names}{roc}{Cacgia Roglai}
\define@key{names}{ccd}{Cafundo}
\define@key{names}{cah}{Cahuarano}
\define@key{names}{qvl}{Cajatambo North Lima Quechua}
\define@key{names}{zad}{Cajonos Zapotec}
\define@key{names}{frc}{Cajun French}
\define@key{names}{ckx}{Caka}
\define@key{names}{ckz}{Cakchiquel-Quiché Mixed Language}
\define@key{names}{cky}{Cakfem-Mushere-Jibyal}
\define@key{names}{tbk}{Calamian Tagbanwa}
\define@key{names}{qud}{Calderón Highland Quichua}
\define@key{names}{caw}{Callawalla}
\define@key{names}{rmq}{Caló}
\define@key{names}{clu}{Caluyanun}
\define@key{names}{abd}{Camarines Norte Agta}
\define@key{names}{csx}{Cambodian Sign Language}
\define@key{names}{mcu}{Donga Mambila}
\define@key{names}{wes}{Cameroon Pidgin}
\define@key{names}{cml}{Campalagian}
\define@key{names}{cmt}{Camtho}
\define@key{names}{xcc}{Camunic}
\define@key{names}{qxr}{Cañar-Azuay-South Chimborazo Highland Quichua}
\define@key{names}{caz}{Canichana}
\define@key{names}{mlc}{Cao Lan}
\define@key{names}{cov}{Cao Miao}
\define@key{names}{cps}{Capiznon}
\define@key{names}{cpg}{Cappadocian Greek}
\define@key{names}{cot}{Caquinte}
\define@key{names}{cby}{Carabayo}
\define@key{names}{cfd}{Cara}
\define@key{names}{crf}{Caramanta}
\define@key{names}{xcr}{Carian}
\define@key{names}{hns}{Caribbean Hindustani}
\define@key{names}{jvn}{Caribbean Javanese}
\define@key{names}{crr}{Carolina Algonquian}
\define@key{names}{rmc}{Central Romani}
\define@key{names}{asc}{Casuarina Coast Asmat}
\define@key{names}{csc}{Catalan Sign Language}
\define@key{names}{xcy}{Cayuse}
\define@key{names}{xce}{Celtiberian}
\define@key{names}{cen}{Cen}
\define@key{names}{hmm}{Central Mashan Hmong}
\define@key{names}{cmo}{Central Mnong}
\define@key{names}{zch}{Central Hongshuihe Zhuang}
\define@key{names}{hmc}{Central Huishui Hmong}
\define@key{names}{fuq}{Central-Eastern Niger Fulfulde}
\define@key{names}{grv}{Central Grebo}
\define@key{names}{cet}{Jalaa}
\define@key{names}{pse}{South Barisan Malay}
\define@key{names}{mwo}{Central Maewo}
\define@key{names}{mxz}{Central Masela}
\define@key{names}{syb}{Central Subanen}
\define@key{names}{tgt}{Central Tagbanwa}
\define@key{names}{plc}{Central Palawano}
\define@key{names}{sml}{Central Sama}
\define@key{names}{zbc}{Central Berawan}
\define@key{names}{dtp}{Kadazan Dusun}
\define@key{names}{awu}{Central Awyu}
\define@key{names}{ncx}{Central Puebla Nahuatl}
\define@key{names}{nch}{Central Huasteca Nahuatl}
\define@key{names}{ojc}{Central Ojibwa}
\define@key{names}{pbs}{Central Pame}
\define@key{names}{quk}{Chachapoyas Quechua}
\define@key{names}{cds}{Chadian Sign Language}
\define@key{names}{cdy}{Chadong}
\define@key{names}{chg}{Chagatai}
\define@key{names}{ciy}{Chaima}
\define@key{names}{ccp}{Chakma}
\define@key{names}{ckh}{Chak}
\define@key{names}{cli}{Chakali}
\define@key{names}{tgf}{Chalikha}
\define@key{names}{cll}{Chala}
\define@key{names}{cdh}{Chambeali}
\define@key{names}{ceg}{Chamacoco}
\define@key{names}{ccc}{Chamicuro}
\define@key{names}{cna}{Changthang}
\define@key{names}{cga}{Changriwa}
\define@key{names}{cra}{Chara}
\define@key{names}{crv}{Chaura}
\define@key{names}{xtb}{Chazumba Mixtec}
\define@key{names}{ruk}{Che}
\define@key{names}{cde}{Chenchu}
\define@key{names}{cjn}{Chenapian}
\define@key{names}{cnu}{Western Algerian Berber}
\define@key{names}{ycp}{Chepya}
\define@key{names}{cpn}{Cherepon}
\define@key{names}{ych}{Chesu}
\define@key{names}{cwg}{Chewong}
\define@key{names}{hne}{Chhattisgarhi}
\define@key{names}{ctn}{Chintang}
\define@key{names}{cur}{Chhulung}
\define@key{names}{csd}{Chiangmai Sign Language}
\define@key{names}{cip}{Chiapanec}
\define@key{names}{zpv}{Chichicapan Zapotec}
\define@key{names}{mii}{Chigmecatitlán Mixtec}
\define@key{names}{csg}{Chilean Sign Language}
\define@key{names}{clh}{Chilisso}
\define@key{names}{clc}{Chilcotin-Nicola}
\define@key{names}{csa}{Chiltepec Chinantec}
\define@key{names}{cpi}{Chinese Pidgin English}
\define@key{names}{chn}{Creolized Grand Ronde Chinook Jargon}
\define@key{names}{cih}{Chinali}
\define@key{names}{bxu}{China Buriat}
\define@key{names}{cnb}{Chinbon Chin}
\define@key{names}{qxc}{Chincha Quechua}
\define@key{names}{cdf}{Chiru}
\define@key{names}{nhd}{Chiripá}
\define@key{names}{the}{Chitwania Tharu}
\define@key{names}{cik}{Chhitkul-Rakchham}
\define@key{names}{zpc}{Choapan Zapotec}
\define@key{names}{cgk}{Chocangacakha}
\define@key{names}{cdi}{Chodri}
\define@key{names}{nri}{Chokri Naga}
\define@key{names}{cjk}{Chokwe}
\define@key{names}{cda}{Choni}
\define@key{names}{coh}{Chonyi-Dzihana-Kauma}
\define@key{names}{cce}{Chopi}
\define@key{names}{nct}{Chothe}
\define@key{names}{cvg}{Duhumbi}
\define@key{names}{cuw}{Chukwa}
\define@key{names}{cuh}{Chuka}
\define@key{names}{chu}{Church Slavic}
\define@key{names}{cdj}{Churahi}
\define@key{names}{scb}{Chut}
\define@key{names}{xcv}{Chuvantsy}
\define@key{names}{chw}{Chuwabu}
\define@key{names}{cia}{Cia-Cia}
\define@key{names}{ckl}{Cibak}
\define@key{names}{awc}{Cicipu}
\define@key{names}{cib}{Ci Gbe}
\define@key{names}{cim}{Cimbrian}
\define@key{names}{mkx}{Cinamiguin Manobo}
\define@key{names}{cdr}{Yara}
\define@key{names}{cie}{Cineni}
\define@key{names}{cin}{Cinta Larga}
\define@key{names}{xcg}{Cisalpine Gaulish}
\define@key{names}{asg}{Western-Kambari-Cishingini}
\define@key{names}{txt}{Citak}
\define@key{names}{tgd}{Ciwogai}
\define@key{names}{xcl}{Classical-Middle Armenian}
\define@key{names}{nci}{Classical Nahuatl}
\define@key{names}{qwc}{Classical Quechua}
\define@key{names}{syc}{Classical Syriac}
\define@key{names}{myz}{Classical Mandaic}
\define@key{names}{xct}{Classical Tibetan}
\define@key{names}{dri}{C'lela}
\define@key{names}{naz}{Coatepec Nahuatl}
\define@key{names}{zps}{Coatlán Zapotec}
\define@key{names}{zca}{Coatecas Altas Zapotec}
\define@key{names}{coj}{Cochimi}
\define@key{names}{coa}{Cocos Islands Malay}
\define@key{names}{liw}{Col}
\define@key{names}{csn}{Colombian Sign Language}
\define@key{names}{gct}{Colonia Tovar German}
\define@key{names}{cfg}{Como Karim}
\define@key{names}{swc}{Congo Swahili}
\define@key{names}{cnc}{Côông}
\define@key{names}{coq}{Coquille}
\define@key{names}{cry}{Kyoli}
\define@key{names}{qwa}{Corongo Ancash Quechua}
\define@key{names}{xxr}{Koropó}
\define@key{names}{cos}{Corsican}
\define@key{names}{csr}{Costa Rican Sign Language}
\define@key{names}{mta}{Cotabato Manobo}
\define@key{names}{xcn}{Cotoname}
\define@key{names}{cow}{Cowlitz}
\define@key{names}{toc}{Coyutla Totonac}
\define@key{names}{gyn}{Guyanese Creole English}
\define@key{names}{csq}{Croatian Sign Language}
\define@key{names}{mfn}{Cross River Mbembe}
\define@key{names}{crz}{Cruzeño}
\define@key{names}{csf}{Cuba Sign Language}
\define@key{names}{cbq}{Cuba}
\define@key{names}{cuo}{Cumanagoto}
\define@key{names}{xlu}{Cuneiform Luwian}
\define@key{names}{cnq}{Chung}
\define@key{names}{cuq}{Cun}
\define@key{names}{ccl}{Cutchi-Swahili}
\define@key{names}{cuv}{Cuvok}
\define@key{names}{xtu}{Cuyamecalco Mixtec}
\define@key{names}{cyo}{Cuyonon}
\define@key{names}{bwy}{Cwi Bwamu}
\define@key{names}{cse}{Czech Sign Language}
\define@key{names}{dao}{Daai Chin}
\define@key{names}{lni}{Daantanai'}
\define@key{names}{dtn}{Daats'iin}
\define@key{names}{dbr}{Dabarre}
\define@key{names}{dbe}{Dabe}
\define@key{names}{xdc}{Dacian}
\define@key{names}{dbd}{Dadiya}
\define@key{names}{dgd}{Dagaari Dioula}
\define@key{names}{dgk}{Dagba}
\define@key{names}{dec}{Dagik}
\define@key{names}{dgn}{Dagoman}
\define@key{names}{dlk}{Dahalik}
\define@key{names}{das}{Daho-Doo}
\define@key{names}{dij}{Dai}
\define@key{names}{drb}{Dair}
\define@key{names}{zhd}{Dai Zhuang}
\define@key{names}{bpa}{Dakaka}
\define@key{names}{dkk}{Dakka}
\define@key{names}{dka}{Dakpakha}
\define@key{names}{qer}{Dalecarlian}
\define@key{names}{dlm}{Dalmatian}
\define@key{names}{dmm}{Dama (Cameroon)}
\define@key{names}{dam}{Damakawa}
\define@key{names}{uhn}{Damal}
\define@key{names}{idb}{Daman-Diu Portuguese}
\define@key{names}{dac}{Dambi}
\define@key{names}{dml}{Dameli}
\define@key{names}{dms}{Dampelas}
\define@key{names}{dnu}{Danau}
\define@key{names}{dnr}{Danaru}
\define@key{names}{daq}{Dandami Maria}
\define@key{names}{thl}{Dangaura Tharu}
\define@key{names}{dsl}{Danish Sign Language}
\define@key{names}{daf}{Dan}
\define@key{names}{aso}{Dano}
\define@key{names}{gku}{Danster !Ui}
\define@key{names}{dnd}{Daonda}
\define@key{names}{daz}{Dao}
\define@key{names}{djc}{Dar Daju Daju}
\define@key{names}{dln}{Darlong}
\define@key{names}{dro}{Daro-Matu Melanau}
\define@key{names}{dot}{Dass}
\define@key{names}{daw}{Davawenyo}
\define@key{names}{dww}{Dawawa}
\define@key{names}{ddw}{Dawera-Daweloor}
\define@key{names}{dax}{Dayi}
\define@key{names}{dzg}{Dazaga}
\define@key{names}{dzd}{Daza}
\define@key{names}{ded}{Dedua}
\define@key{names}{gbh}{Defi Gbe}
\define@key{names}{dge}{Degenan}
\define@key{names}{mzw}{Deg}
\define@key{names}{deh}{Dehwari}
\define@key{names}{dek}{Dek}
\define@key{names}{row}{Dela-Oenale}
\define@key{names}{ntr}{Delo}
\define@key{names}{dmx}{Dema}
\define@key{names}{dei}{Demisa}
\define@key{names}{dem}{Dem}
\define@key{names}{dmy}{Demta}
\define@key{names}{deq}{Dendi (Central African Republic)}
\define@key{names}{ddn}{Dendi (Benin)}
\define@key{names}{dez}{Dengese}
\define@key{names}{dnk}{Dengka}
\define@key{names}{dbb}{Deno}
\define@key{names}{anv}{Denya}
\define@key{names}{dee}{Dewoin}
\define@key{names}{def}{Dezfuli-Shushtari}
\define@key{names}{dgh}{Dghwede}
\define@key{names}{dhs}{Dhaiso}
\define@key{names}{dhn}{Dhanki}
\define@key{names}{dwz}{Dewas-Done Danuwar}
\define@key{names}{nfa}{Dhao}
\define@key{names}{mki}{Dhatki}
\define@key{names}{dho}{Dhodia-Kukna}
\define@key{names}{adf}{Dhofari Arabic}
\define@key{names}{ddr}{Dhudhuroa}
\define@key{names}{dhd}{Dhundari}
\define@key{names}{dia}{Alu-Sinagen}
\define@key{names}{mbd}{Dibabawon Manobo}
\define@key{names}{dby}{Dibiyaso}
\define@key{names}{dio}{Dibo}
\define@key{names}{duy}{Dicamay Agta}
\define@key{names}{dig}{Digo}
\define@key{names}{cfa}{Dijim-Bwilim}
\define@key{names}{dil}{Dilling}
\define@key{names}{jma}{Dima}
\define@key{names}{dii}{Dimbong}
\define@key{names}{dmc}{Gavak}
\define@key{names}{ddi}{Diodio}
\define@key{names}{gdl}{Dirasha}
\define@key{names}{diu}{Diriku-Shambyu}
\define@key{names}{dir}{Dirim}
\define@key{names}{dwa}{Diri}
\define@key{names}{dsi}{Dissa-Canton Mufa}
\define@key{names}{tbz}{Ditammari}
\define@key{names}{diy}{Diuwe}
\define@key{names}{xtd}{Diuxi-Tilantongo Mixtec}
\define@key{names}{dix}{Dixon Reef}
\define@key{names}{djf}{Djangun}
\define@key{names}{djn}{Jawoyn}
\define@key{names}{djw}{Djawi}
\define@key{names}{djb}{Djinba}
\define@key{names}{dze}{Djiwarli}
\define@key{names}{dob}{Dobu}
\define@key{names}{doe}{Doe}
\define@key{names}{dgg}{Doga}
\define@key{names}{dgx}{Doghoro}
\define@key{names}{dgs}{Dogoso}
\define@key{names}{dos}{Dogosé}
\define@key{names}{dgr}{Dogrib}
\define@key{names}{dbg}{Dogul Dom Dogon}
\define@key{names}{dbi}{Doka}
\define@key{names}{uya}{Doko-Uyanga}
\define@key{names}{dre}{Dolpo}
\define@key{names}{dov}{Toka-Leya-Dombe}
\define@key{names}{doq}{Dominican Sign Language}
\define@key{names}{doa}{Dom}
\define@key{names}{doy}{Dompo}
\define@key{names}{dof}{Domu}
\define@key{names}{dev}{Domung}
\define@key{names}{dok}{Dondo}
\define@key{names}{yik}{Dongshanba Lalo}
\define@key{names}{doh}{Dong}
\define@key{names}{ddd}{Dongotono}
\define@key{names}{dde}{Doondo}
\define@key{names}{dor}{Dori'o}
\define@key{names}{kqc}{Doromu-Koki}
\define@key{names}{doz}{Dorze}
\define@key{names}{dol}{Doso}
\define@key{names}{dty}{Dotyali}
\define@key{names}{dup}{Duano}
\define@key{names}{dva}{Duau}
\define@key{names}{dub}{Dubli}
\define@key{names}{dmu}{Dubu}
\define@key{names}{duk}{Duduela}
\define@key{names}{ndu}{Dugun}
\define@key{names}{dbm}{Duguri}
\define@key{names}{dme}{Dugwor}
\define@key{names}{kbz}{Duhwa}
\define@key{names}{nke}{Duke}
\define@key{names}{dbo}{Dulbu}
\define@key{names}{duz}{Duli-Gewe}
\define@key{names}{dmv}{Dumpas}
\define@key{names}{wtf}{Dumpu}
\define@key{names}{dui}{Dumun}
\define@key{names}{duh}{Dungra Bhil}
\define@key{names}{raa}{Dungmali}
\define@key{names}{dng}{Dungan}
\define@key{names}{dbv}{Dungu}
\define@key{names}{drq}{Dura}
\define@key{names}{mvp}{Duri}
\define@key{names}{dbn}{Duriankere}
\define@key{names}{dug}{Duruma}
\define@key{names}{dsn}{Dusner}
\define@key{names}{duw}{Dusun Witu}
\define@key{names}{duq}{Dusun Malang}
\define@key{names}{dun}{Dusun Deyah}
\define@key{names}{dws}{Dutton World Speedwords}
\define@key{names}{dux}{Duungooma}
\define@key{names}{dae}{Duupa}
\define@key{names}{duv}{Duvle}
\define@key{names}{dbp}{Duwai}
\define@key{names}{gve}{Duwet}
\define@key{names}{nnu}{Dwang}
\define@key{names}{dyb}{Dyaberdyaber}
\define@key{names}{dyn}{Dyangadi}
\define@key{names}{dya}{Dyan}
\define@key{names}{dyd}{Dyugun}
\define@key{names}{jen}{Dza}
\define@key{names}{dzl}{Dzalakha}
\define@key{names}{dzn}{Dzando}
\define@key{names}{bpn}{Dzao Min}
\define@key{names}{add}{Dzodinka}
\define@key{names}{dzo}{Dzongkha}
\define@key{names}{dnn}{Dzùùngoo}
\define@key{names}{ktv}{Eastern Katu}
\define@key{names}{bgp}{Eastern Balochi}
\define@key{names}{lwl}{Eastern Lawa}
\define@key{names}{mng}{Eastern Mnong}
\define@key{names}{emu}{Eastern Muria}
\define@key{names}{tge}{Eastern Gorkha Tamang}
\define@key{names}{nos}{Eastern Nisu}
\define@key{names}{emq}{Eastern Muya}
\define@key{names}{kif}{Eastern Parbate Kham}
\define@key{names}{emg}{Eastern Meohang}
\define@key{names}{zeh}{Eastern Hongshuihe Zhuang}
\define@key{names}{hmq}{Eastern Qiandong Miao}
\define@key{names}{muq}{Eastern Xiangxi Miao}
\define@key{names}{hme}{Eastern Huishui Hmong}
\define@key{names}{lma}{East Limba}
\define@key{names}{gbx}{Eastern Xwla Gbe}
\define@key{names}{xrb}{Eastern Karaboro}
\define@key{names}{acp}{Eastern Acipa}
\define@key{names}{nle}{East Nyala}
\define@key{names}{kqo}{Konobo-Eastern Krahn}
\define@key{names}{vme}{East Masela}
\define@key{names}{tre}{East Tarangan}
\define@key{names}{dmr}{East Damar}
\define@key{names}{bnj}{Eastern Tawbuid}
\define@key{names}{pez}{Eastern Penan}
\define@key{names}{zbe}{East Berawan}
\define@key{names}{kjs}{East Kewa}
\define@key{names}{nhe}{Eastern Huasteca Nahuatl}
\define@key{names}{ojg}{Eastern Ojibwa}
\define@key{names}{aaq}{Eastern Abenaki}
\define@key{names}{qve}{Eastern Apurímac Quechua}
\define@key{names}{cly}{Eastern Highland Chatino}
\define@key{names}{avl}{Eastern Egyptian Bedawi Arabic}
\define@key{names}{sfe}{Eastern Subanen}
\define@key{names}{azd}{Eastern Durango Nahuatl}
\define@key{names}{yit}{Eastern Lalu}
\define@key{names}{cek}{Eastern Khumi Chin}
\define@key{names}{yol}{Irish Anglo-Norman}
\define@key{names}{xeb}{Eblaite}
\define@key{names}{ebr}{Ebrié}
\define@key{names}{ebg}{Ebughu}
\define@key{names}{ecs}{Ecuadorian Sign Language}
\define@key{names}{cbj}{Ede Cabe}
\define@key{names}{idd}{Ede Idaca}
\define@key{names}{ijj}{Ede Ije}
\define@key{names}{ica}{Ede Ica}
\define@key{names}{nqg}{Ede Nago}
\define@key{names}{awy}{Edera Awyu}
\define@key{names}{dbf}{Edopi}
\define@key{names}{eee}{E}
\define@key{names}{efa}{Efai}
\define@key{names}{efe}{Efe}
\define@key{names}{ofu}{Efutop}
\define@key{names}{ego}{Eggon}
\define@key{names}{esl}{Egypt Sign Language}
\define@key{names}{egy}{Egyptian (Ancient)}
\define@key{names}{ehu}{Ehueun}
\define@key{names}{eit}{Eitiep}
\define@key{names}{eja}{Ejamat}
\define@key{names}{eka}{Ekajuk}
\define@key{names}{eki}{Eki}
\define@key{names}{eke}{Ekit}
\define@key{names}{ekp}{Ekpeye}
\define@key{names}{zpp}{El Alto Zapotec}
\define@key{names}{elx}{Elamite}
\define@key{names}{elm}{Eleme}
\define@key{names}{ele}{Elepi}
\define@key{names}{elh}{El Hugeirat}
\define@key{names}{ekm}{Elip}
\define@key{names}{elk}{Elkei}
\define@key{names}{elo}{El Molo}
\define@key{names}{zte}{Elotepec Zapotec}
\define@key{names}{afo}{Ajiri}
\define@key{names}{elu}{Elu}
\define@key{names}{xly}{Elymian}
\define@key{names}{yzg}{E'ma Buyang}
\define@key{names}{emn}{Eman}
\define@key{names}{bdc}{Emberá-Baudó}
\define@key{names}{tdc}{Emberá-Tadó}
\define@key{names}{ebu}{Embu}
\define@key{names}{emw}{Emplawas}
\define@key{names}{enr}{Emumu}
\define@key{names}{unk}{Enawené-Nawé}
\define@key{names}{end}{Ende}
\define@key{names}{enc}{En}
\define@key{names}{ptt}{Enrekang}
\define@key{names}{enu}{Enu}
\define@key{names}{enw}{Enwan (Akwa Ibom State)}
\define@key{names}{env}{Enwan (Edo State)}
\define@key{names}{epi}{Epie}
\define@key{names}{emy}{Epigraphic Mayan}
\define@key{names}{era}{Eravallan}
\define@key{names}{kjy}{Erave}
\define@key{names}{twp}{Ere}
\define@key{names}{ert}{Eritai}
\define@key{names}{erw}{Erokwanas}
\define@key{names}{err}{Erre}
\define@key{names}{emx}{Erromintxela}
\define@key{names}{ers}{Ersu}
\define@key{names}{erh}{Eruwa}
\define@key{names}{ish}{Esan}
\define@key{names}{mcq}{Ese}
\define@key{names}{esh}{Eshtehardi}
\define@key{names}{ags}{Esimbi}
\define@key{names}{esy}{Eskayan}
\define@key{names}{epo}{Esperanto}
\define@key{names}{ots}{Estado de México Otomi}
\define@key{names}{eso}{Estonian Sign Language}
\define@key{names}{esm}{Esuma}
\define@key{names}{etb}{Etebi}
\define@key{names}{etx}{Eten}
\define@key{names}{ecr}{Eteocretan}
\define@key{names}{ecy}{Eteocypriot}
\define@key{names}{eth}{Ethiopian Sign Language}
\define@key{names}{ich}{Etkywan}
\define@key{names}{eto}{Eton-Mengisa}
\define@key{names}{etn}{Eton (Vanuatu)}
\define@key{names}{ett}{Etruscan}
\define@key{names}{utr}{Etulo}
\define@key{names}{bzz}{Evant}
\define@key{names}{gev}{Viya}
\define@key{names}{nou}{Ewage-Notu}
\define@key{names}{ext}{Extremaduran}
\define@key{names}{fab}{Annobonese}
\define@key{names}{faf}{Fagani}
\define@key{names}{fif}{Faifi}
\define@key{names}{azt}{Faire Atta}
\define@key{names}{faj}{Kulsab}
\define@key{names}{fai}{Faiwol}
\define@key{names}{fax}{Fala}
\define@key{names}{cfm}{Falam Chin}
\define@key{names}{fli}{Fali}
\define@key{names}{xfa}{Faliscan}
\define@key{names}{fam}{Fam}
\define@key{names}{fng}{Fanagalo}
\define@key{names}{fan}{Fang (Equatorial Guinea)}
\define@key{names}{fak}{Fang (Cameroon)}
\define@key{names}{fni}{Fania}
\define@key{names}{nsf}{Far Northwestern Nisu}
\define@key{names}{fmu}{Far Western Muria}
\define@key{names}{far}{Fataleka}
\define@key{names}{ddg}{Fataluku}
\define@key{names}{fau}{Fayu}
\define@key{names}{agl}{Fembe}
\define@key{names}{fpe}{Pichi}
\define@key{names}{fer}{Feroge}
\define@key{names}{hif}{Fiji Hindi}
\define@key{names}{fil}{Filipino}
\define@key{names}{tlp}{Filomeno Mata Totonac}
\define@key{names}{bkb}{Eastern-Southern Bontok}
\define@key{names}{fss}{Finland-Swedish Sign Language}
\define@key{names}{fag}{Finongan}
\define@key{names}{fip}{Fipa}
\define@key{names}{fir}{Firan}
\define@key{names}{fiw}{Fiwaga}
\define@key{names}{fln}{Flinders Island}
\define@key{names}{flh}{Abawiri}
\define@key{names}{fod}{Foodo}
\define@key{names}{frq}{Forak}
\define@key{names}{enf}{Forest Enets}
\define@key{names}{frt}{Kiai}
\define@key{names}{frp}{Arpitan}
\define@key{names}{fur}{Friulian}
\define@key{names}{flr}{Fuliiru}
\define@key{names}{ula}{Fungwa}
\define@key{names}{fuy}{Fuyug}
\define@key{names}{fwe}{Fwe}
\define@key{names}{fie}{Fyer}
\define@key{names}{ttb}{Gaa}
\define@key{names}{gie}{Gabogbo}
\define@key{names}{gab}{Gabri}
\define@key{names}{gdg}{Ga'dang}
\define@key{names}{gdk}{Gadang}
\define@key{names}{gbk}{Gaddi}
\define@key{names}{gad}{Gaddang}
\define@key{names}{gda}{Gade Lohar}
\define@key{names}{gdh}{Gajirrabeng}
\define@key{names}{gft}{Gafat}
\define@key{names}{btg}{Gagnoa Bété}
\define@key{names}{ggu}{Gban}
\define@key{names}{gbf}{Gaikundi}
\define@key{names}{gic}{Gail}
\define@key{names}{gcn}{Gaina}
\define@key{names}{xga}{Galatian}
\define@key{names}{glo}{Galambu}
\define@key{names}{gar}{Galeya}
\define@key{names}{gce}{Galice}
\define@key{names}{sdn}{Gallurese Sardinian}
\define@key{names}{gap}{Gal}
\define@key{names}{gal}{Galoli-Talur}
\define@key{names}{kgj}{Gamale Kham}
\define@key{names}{gma}{Gambera}
\define@key{names}{wof}{Gambian Wolof}
\define@key{names}{gbl}{Gamit}
\define@key{names}{gak}{Gamkonora}
\define@key{names}{bte}{Gamo-Ningi}
\define@key{names}{ihw}{Birrdhawal}
\define@key{names}{gne}{Ganang}
\define@key{names}{gnk}{//Gana}
\define@key{names}{gnq}{Gana}
\define@key{names}{unn}{Ganai}
\define@key{names}{gan}{Gan Chinese}
\define@key{names}{pgd}{Gandhari}
\define@key{names}{gzn}{Gane}
\define@key{names}{gnb}{Gangte}
\define@key{names}{gnl}{Gangulu}
\define@key{names}{ggl}{Ganglau}
\define@key{names}{gao}{Gants}
\define@key{names}{gza}{Ganza}
\define@key{names}{gnz}{Ganzi}
\define@key{names}{gga}{Gao}
\define@key{names}{gbm}{Garhwali}
\define@key{names}{ilg}{Garig-Ilgar}
\define@key{names}{gex}{Garre}
\define@key{names}{gaq}{Gata'}
\define@key{names}{gou}{Gavar}
\define@key{names}{gwt}{Gawar-Bati}
\define@key{names}{gyl}{Gayil}
\define@key{names}{gzi}{Gazic}
\define@key{names}{gbg}{Gbanziri-Boraka}
\define@key{names}{gbv}{Gbanu}
\define@key{names}{gby}{Gbari}
\define@key{names}{gyg}{Gbayi}
\define@key{names}{gbq}{Gbaya-Bozoum}
\define@key{names}{gbs}{Gbesi Gbe}
\define@key{names}{ggb}{Gbii}
\define@key{names}{xgb}{Gbin}
\define@key{names}{grh}{Gbiri-Niragu}
\define@key{names}{gec}{Gboloo Grebo}
\define@key{names}{kvq}{Geba Karen}
\define@key{names}{gei}{Gebe}
\define@key{names}{gdd}{Gedaged}
\define@key{names}{drs}{Gedeo}
\define@key{names}{hmj}{Ge}
\define@key{names}{gez}{Geez}
\define@key{names}{ghk}{Geko Karen}
\define@key{names}{giu}{Gelao Mulao}
\define@key{names}{geq}{Geme}
\define@key{names}{gaf}{Gende}
\define@key{names}{gej}{Gen}
\define@key{names}{ygp}{Gepo}
\define@key{names}{gew}{Gera}
\define@key{names}{gea}{Geruma}
\define@key{names}{ges}{Geser-Gorom}
\define@key{names}{gha}{Ghadames}
\define@key{names}{gse}{Ghanaian Sign Language}
\define@key{names}{ghn}{Ghanongga}
\define@key{names}{gpe}{Ghanaian Pidgin English}
\define@key{names}{gds}{Ghandruk Sign Language}
\define@key{names}{gri}{Ghari}
\define@key{names}{ajs}{Ghardaia Sign Language}
\define@key{names}{bmk}{Ghayavi}
\define@key{names}{aln}{Gheg Albanian}
\define@key{names}{ghr}{Ghera}
\define@key{names}{bbj}{Ghomálá'}
\define@key{names}{gho}{Ghomara}
\define@key{names}{bgi}{Giangan}
\define@key{names}{gib}{Gibanawa}
\define@key{names}{kks}{Giiwo}
\define@key{names}{acd}{Gikyode}
\define@key{names}{gix}{Gilima}
\define@key{names}{gip}{Gimi (West New Britain)}
\define@key{names}{gim}{Gimi (Eastern Highlands)}
\define@key{names}{kmp}{Gimme}
\define@key{names}{gmn}{Gimnime}
\define@key{names}{gnm}{Ginuman}
\define@key{names}{ayg}{Ginyanga}
\define@key{names}{bbr}{Girawa}
\define@key{names}{gii}{Girirra}
\define@key{names}{nyf}{Giryama}
\define@key{names}{toh}{Gitonga}
\define@key{names}{ggt}{Gitua}
\define@key{names}{giy}{Giyug}
\define@key{names}{tof}{Gizrra}
\define@key{names}{glr}{Glaro-Twabo}
\define@key{names}{glw}{Glavda}
\define@key{names}{oub}{Glio-Oubi}
\define@key{names}{gnu}{Gnau}
\define@key{names}{gom}{Goan Konkani}
\define@key{names}{gig}{Goaria}
\define@key{names}{goi}{Gobasi}
\define@key{names}{gox}{Gobu}
\define@key{names}{gdx}{Godwari}
\define@key{names}{gof}{Gofa}
\define@key{names}{gog}{Gogo}
\define@key{names}{goo}{Gone Dau}
\define@key{names}{goe}{Gongduk}
\define@key{names}{gjn}{Gonja}
\define@key{names}{gov}{Goo}
\define@key{names}{goq}{Gorap}
\define@key{names}{goc}{Gorakor}
\define@key{names}{grq}{Gorovu}
\define@key{names}{gqr}{Gor}
\define@key{names}{got}{Gothic}
\define@key{names}{goy}{Goundo}
\define@key{names}{gwf}{Gowro}
\define@key{names}{goz}{Alamuti}
\define@key{names}{nli}{Grangali-Ningalami}
\define@key{names}{giq}{Hagei Gelao}
\define@key{names}{gcl}{Grenadian Creole English}
\define@key{names}{grs}{Gresi}
\define@key{names}{gro}{Groma}
\define@key{names}{gos}{Gronings}
\define@key{names}{ats}{Gros Ventre}
\define@key{names}{gwx}{Gua}
\define@key{names}{gvj}{Guajá}
\define@key{names}{jiq}{Khroskyabs}
\define@key{names}{gnc}{Guanche}
\define@key{names}{gyr}{Guarayu}
\define@key{names}{gsm}{Guatemalan Sign Language}
\define@key{names}{xgd}{Gudang}
\define@key{names}{gdu}{Gudu}
\define@key{names}{zpg}{Guevea De Humboldt Zapotec}
\define@key{names}{gdc}{Gugu Badhun}
\define@key{names}{kkp}{Gugubera}
\define@key{names}{wrw}{Roth's Gugu Warra}
\define@key{names}{zgn}{Guibian Zhuang}
\define@key{names}{bet}{Guiberoua Béte}
\define@key{names}{ztu}{Güilá Zapotec}
\define@key{names}{gus}{Guinean Sign Language}
\define@key{names}{gkp}{Guinea Kpelle}
\define@key{names}{gqi}{Guiqiong}
\define@key{names}{gvl}{Gulay}
\define@key{names}{glu}{Gula (Chad)}
\define@key{names}{gmb}{Gula'alaa}
\define@key{names}{gly}{Gule}
\define@key{names}{gul}{Sea Island Creole English}
\define@key{names}{gmu}{Gumalu}
\define@key{names}{gdi}{Gundi}
\define@key{names}{gyf}{Gungabula}
\define@key{names}{rub}{Gungu}
\define@key{names}{gnt}{Warta Thuntai}
\define@key{names}{gpa}{Gupa-Abawa}
\define@key{names}{grz}{Guramalum}
\define@key{names}{gdj}{Gurdjar}
\define@key{names}{ggg}{Gurgula}
\define@key{names}{grx}{Guriaso}
\define@key{names}{gjr}{Gurindji Kriol}
\define@key{names}{gvm}{Gurmana}
\define@key{names}{gvr}{Gurung}
\define@key{names}{grd}{Guruntum-Mbaaru}
\define@key{names}{gsn}{Gusan}
\define@key{names}{gsl}{Gusilay}
\define@key{names}{xgw}{Guwa}
\define@key{names}{gwu}{Guwamu}
\define@key{names}{gvy}{Guyani}
\define@key{names}{gka}{Guya}
\define@key{names}{ngs}{Gvoko}
\define@key{names}{gwb}{Gwa}
\define@key{names}{dah}{Gwahatike}
\define@key{names}{bga}{Gwamhi-Wuri}
\define@key{names}{gwn}{Gwandara}
\define@key{names}{grw}{Gweda}
\define@key{names}{gwe}{Gweno}
\define@key{names}{gwr}{Gwere}
\define@key{names}{gwj}{/Gwi}
\define@key{names}{gyi}{Gyele}
\define@key{names}{gye}{Gyem}
\define@key{names}{haq}{Ha}
\define@key{names}{hbu}{Habu}
\define@key{names}{hdy}{Hadiyya}
\define@key{names}{hoj}{Hadothi}
\define@key{names}{xhd}{Hadrami}
\define@key{names}{ayh}{Hadrami Arabic}
\define@key{names}{aek}{Haeke}
\define@key{names}{hah}{Hahon}
\define@key{names}{hgw}{Haigwai}
\define@key{names}{bzx}{Hainyaxo Bozo}
\define@key{names}{hgm}{Hai//om-Akhoe}
\define@key{names}{haf}{Haiphong Sign Language}
\define@key{names}{hvc}{Haitian Vodoun Culture Language}
\define@key{names}{hji}{Haji}
\define@key{names}{haj}{Hajong}
\define@key{names}{hao}{Hakö}
\define@key{names}{hld}{Halang Doan}
\define@key{names}{hmu}{Hamap}
\define@key{names}{hba}{Hamba de Lomela}
\define@key{names}{hag}{Hanga}
\define@key{names}{han}{Hangaza}
\define@key{names}{haa}{Han}
\define@key{names}{hab}{Hanoi Sign Language}
\define@key{names}{xiv}{Harappan}
\define@key{names}{kjo}{Indo-Aryan Kinnauri}
\define@key{names}{hro}{Haroi}
\define@key{names}{hrk}{Haruku}
\define@key{names}{bgc}{Haryanvi}
\define@key{names}{hrz}{Harzani-Kilit}
\define@key{names}{ybj}{Hasha}
\define@key{names}{xht}{Hattic}
\define@key{names}{hsl}{Hausa Sign Language}
\define@key{names}{hvk}{Haveke}
\define@key{names}{hav}{Havu}
\define@key{names}{hps}{Hawai'i Pidgin Sign Language}
\define@key{names}{xda}{Hawkesbury}
\define@key{names}{haz}{Hazaragi}
\define@key{names}{hbn}{Ebang}
\define@key{names}{scp}{Lamjung-Melamchi Yolmo}
\define@key{names}{heg}{Helong}
\define@key{names}{nix}{Hema}
\define@key{names}{hed}{Herde}
\define@key{names}{llf}{Hermit}
\define@key{names}{hrt}{Hertevin}
\define@key{names}{ham}{Hewa}
\define@key{names}{auk}{Heyo}
\define@key{names}{hib}{Hibito}
\define@key{names}{hlu}{Hieroglyphic Luwian}
\define@key{names}{mba}{Higaonon}
\define@key{names}{kjk}{Highland Konjo}
\define@key{names}{hij}{Hijuk}
\define@key{names}{hir}{Himarimã}
\define@key{names}{hii}{Hinduri}
\define@key{names}{hmo}{Hiri Motu}
\define@key{names}{hit}{Hittite}
\define@key{names}{htu}{Hitu}
\define@key{names}{hiw}{Hiw}
\define@key{names}{yhl}{Hlepho Phowa}
\define@key{names}{hle}{Hlersu}
\define@key{names}{hmf}{Hmong Don}
\define@key{names}{hmz}{Sinicized Miao}
\define@key{names}{hmv}{Hmong Dô}
\define@key{names}{mrk}{Hmwaveke}
\define@key{names}{hoh}{Hobyót}
\define@key{names}{hos}{Ho Chi Minh City Sign Language}
\define@key{names}{hhi}{Hoia Hoia-Ukusi-Koperami}
\define@key{names}{hoy}{Holiya}
\define@key{names}{hoi}{Holikachuk}
\define@key{names}{hod}{Holma}
\define@key{names}{hol}{Holu}
\define@key{names}{hom}{Homa}
\define@key{names}{hds}{Honduras Sign Language}
\define@key{names}{juh}{Hõne}
\define@key{names}{how}{Honi}
\define@key{names}{hrm}{Horned Miao}
\define@key{names}{hoe}{Horom}
\define@key{names}{hor}{Horo}
\define@key{names}{ero}{Stau-Dgebshes}
\define@key{names}{hot}{Hote}
\define@key{names}{hti}{Hoti of East Seram}
\define@key{names}{hov}{Hobongan}
\define@key{names}{hhy}{Hoyahoya-Matakaia}
\define@key{names}{hoz}{Hozo}
\define@key{names}{hpo}{Hpon}
\define@key{names}{hra}{Hrangkhol}
\define@key{names}{hru}{Hruso}
\define@key{names}{hug}{Huachipaeri}
\define@key{names}{qvh}{Huamalíes-Dos de Mayo Huánuco Quechua}
\define@key{names}{hud}{Huaulu}
\define@key{names}{nhq}{Huaxcaleca Nahuatl}
\define@key{names}{qwh}{Huaylas Ancash Quechua}
\define@key{names}{qvw}{Huaylla Wanca Quechua}
\define@key{names}{huh}{Huilliche}
\define@key{names}{mxs}{Huitepec Mixtec}
\define@key{names}{czh}{Hui Chinese}
\define@key{names}{huw}{Hukumina}
\define@key{names}{hul}{Hula}
\define@key{names}{huy}{Hulaulá}
\define@key{names}{hui}{Huli}
\define@key{names}{huk}{Hulung}
\define@key{names}{hmb}{Humburi Senni Songhay}
\define@key{names}{huf}{Humene}
\define@key{names}{hut}{Humla}
\define@key{names}{hsh}{Hungarian Sign Language}
\define@key{names}{hnu}{Hung}
\define@key{names}{nat}{Hungworo}
\define@key{names}{hum}{Hungan}
\define@key{names}{hng}{Hungu-Pombo}
\define@key{names}{hkk}{Hunjara-Kaina Ke}
\define@key{names}{hap}{Hupla}
\define@key{names}{xhu}{Hurrian}
\define@key{names}{geh}{Hutterite German}
\define@key{names}{huo}{Hu}
\define@key{names}{hwo}{Hwana}
\define@key{names}{hya}{Hya}
\define@key{names}{jab}{Hyam}
\define@key{names}{yml}{Iamalele}
\define@key{names}{tek}{Kwa South}
\define@key{names}{ibl}{Ibaloi}
\define@key{names}{iby}{Ibani}
\define@key{names}{xib}{Iberian}
\define@key{names}{ibn}{Ibino}
\define@key{names}{ibr}{Ibuoro}
\define@key{names}{ibu}{Ibu}
\define@key{names}{bec}{Iceve-Maci}
\define@key{names}{ida}{Idakho-Isukha-Tiriki}
\define@key{names}{idt}{Idaté}
\define@key{names}{ide}{Idere}
\define@key{names}{idi}{Idi-Taeme}
\define@key{names}{idc}{Idon}
\define@key{names}{ido}{Ido}
\define@key{names}{ldb}{Dũya}
\define@key{names}{ife}{Ifè}
\define@key{names}{iff}{Ifo}
\define@key{names}{igl}{Igala}
\define@key{names}{igg}{Igana}
\define@key{names}{ahl}{Igo}
\define@key{names}{nar}{Iguta}
\define@key{names}{igw}{Igwe}
\define@key{names}{ihb}{Iha-based Pidgin}
\define@key{names}{ikk}{Ika}
\define@key{names}{ikr}{Ikaranggal}
\define@key{names}{ikz}{Ikizu}
\define@key{names}{meb}{Ikobi}
\define@key{names}{ntk}{Ikoma-Nata}
\define@key{names}{iki}{Iko}
\define@key{names}{ikp}{Ikpeshi}
\define@key{names}{txi}{Ikpeng}
\define@key{names}{ikv}{Iku-Gora-Ankwa}
\define@key{names}{ikl}{Ikulu}
\define@key{names}{ikw}{Ikwere}
\define@key{names}{ila}{Ile Ape}
\define@key{names}{mbi}{Ilianen Manobo}
\define@key{names}{ili}{Ili Turki}
\define@key{names}{ilu}{Ili'uun}
\define@key{names}{xil}{Illyrian}
\define@key{names}{ilk}{Ilongot}
\define@key{names}{ilv}{Ilue}
\define@key{names}{mlk}{Ilwana}
\define@key{names}{imo}{Imbongu}
\define@key{names}{arc}{Imperial Aramaic (700-300 BCE)}
\define@key{names}{imr}{Imroing}
\define@key{names}{abx}{Inabaknon}
\define@key{names}{mzu}{Itutang-Inapang}
\define@key{names}{inp}{Iñapari}
\define@key{names}{smn}{Inari Saami}
\define@key{names}{inl}{Jakartan Sign Language}
\define@key{names}{idr}{Indri}
\define@key{names}{mvy}{Indus Kohistani}
\define@key{names}{oin}{Inebu One}
\define@key{names}{iti}{Inlaod Itneg}
\define@key{names}{ino}{Inoke-Yate}
\define@key{names}{loc}{Inonhan}
\define@key{names}{ior}{Inoric}
\define@key{names}{ina}{Interlingua (International Auxiliary Language Association)}
\define@key{names}{ile}{Interlingue (Occidental)}
\define@key{names}{igs}{Interglossa}
\define@key{names}{int}{Intha-Danu}
\define@key{names}{iks}{Inuit Sign Language}
\define@key{names}{azm}{Ipalapa Amuzgo}
\define@key{names}{ipo}{Ipiko}
\define@key{names}{ipi}{Ipili}
\define@key{names}{ass}{Ipulo-Olulu}
\define@key{names}{ill}{Iranun}
\define@key{names}{iry}{Iraya}
\define@key{names}{ire}{Yerisiam}
\define@key{names}{iri}{Irigwe}
\define@key{names}{bto}{Iriga Bicolano}
\define@key{names}{iru}{Irula of the Nilgiri}
\define@key{names}{isa}{Isabi}
\define@key{names}{isn}{Isanzu}
\define@key{names}{agk}{Isarog Agta}
\define@key{names}{isc}{Isconahua}
\define@key{names}{igo}{Isebe}
\define@key{names}{inn}{Isinai}
\define@key{names}{crb}{Island Carib}
\define@key{names}{mir}{Isthmus Mixe}
\define@key{names}{nhk}{Isthmus-Cosoleacaque Nahuatl}
\define@key{names}{ist}{Istriot}
\define@key{names}{ruo}{Istro Romanian}
\define@key{names}{szv}{Isu (Fako Division)}
\define@key{names}{isu}{Isu (Menchum Division)}
\define@key{names}{ite}{Itene}
\define@key{names}{itr}{Iteri}
\define@key{names}{itx}{Itik}
\define@key{names}{itw}{Ito}
\define@key{names}{itm}{Itu Mbon Uzo}
\define@key{names}{mce}{Itundujia Mixtec}
\define@key{names}{ivv}{Itbayat}
\define@key{names}{atg}{Ivbie North-Okpela-Arhe}
\define@key{names}{iwk}{I-Wak}
\define@key{names}{kbm}{Iwal}
\define@key{names}{iwo}{Morop-Dintere}
\define@key{names}{mzi}{Ixcatlán Mazatec}
\define@key{names}{vmj}{Ixtayutla Mixtec}
\define@key{names}{iya}{Iyayu}
\define@key{names}{uiv}{Iyive}
\define@key{names}{crt}{Iyojwa'ja Chorote}
\define@key{names}{nca}{Iyo}
\define@key{names}{crq}{Iyo'wujwa Chorote}
\define@key{names}{izi}{Izi-Ezaa-Ikwo-Mgbo}
\define@key{names}{cbo}{Izora}
\define@key{names}{rzh}{Jabal Razih}
\define@key{names}{jdg}{Jadgali}
\define@key{names}{jad}{Jahanka}
\define@key{names}{jah}{Jah Hut}
\define@key{names}{awv}{Kia River Awyu}
\define@key{names}{jat}{Inku}
\define@key{names}{jak}{Jakun}
\define@key{names}{maj}{Jalapa De Díaz Mazatec}
\define@key{names}{bxl}{Jalkunan}
\define@key{names}{jcs}{Jamaican Country Sign Language}
\define@key{names}{jls}{Jamaican Sign Language}
\define@key{names}{jax}{Jambi Malay}
\define@key{names}{jnd}{Jandavra}
\define@key{names}{jna}{Jangshung}
\define@key{names}{djo}{Jangkang}
\define@key{names}{jni}{Janji}
\define@key{names}{jar}{Jarawa (Nigeria)}
\define@key{names}{jra}{Jarai}
\define@key{names}{jaf}{Jara}
\define@key{names}{qxw}{Jauja Wanca Quechua}
\define@key{names}{jns}{Jaunsari}
\define@key{names}{jvd}{Javindo}
\define@key{names}{jaz}{Jawe}
\define@key{names}{jyy}{Jaya}
\define@key{names}{jje}{Jejueo}
\define@key{names}{bze}{Jenaama Bozo}
\define@key{names}{xuj}{Jennu Kurumba}
\define@key{names}{jer}{Jere}
\define@key{names}{jee}{Jerung}
\define@key{names}{tmr}{Jewish Babylonian Aramaic (ca. 200-1200 CE)}
\define@key{names}{jhs}{Jhankot Sign Language}
\define@key{names}{jio}{Jiamao}
\define@key{names}{juo}{Jiba}
\define@key{names}{jib}{Jibu}
\define@key{names}{jii}{Jiiddu}
\define@key{names}{jie}{Jilbe}
\define@key{names}{jil}{Jilim}
\define@key{names}{jim}{Jimi (Cameroon)}
\define@key{names}{jmi}{Jimi (Nigeria)}
\define@key{names}{jia}{Jina}
\define@key{names}{cjy}{Jinyu Chinese}
\define@key{names}{pnu}{Jiongnai Bunu}
\define@key{names}{jul}{Jirel}
\define@key{names}{jrr}{Jiru}
\define@key{names}{jit}{Jita}
\define@key{names}{kaj}{Jju}
\define@key{names}{job}{Joba}
\define@key{names}{jbr}{Jofotek-Bromnya}
\define@key{names}{jeu}{Jonkor Bourmataguil}
\define@key{names}{jor}{Jorá}
\define@key{names}{jrt}{Jakattoe}
\define@key{names}{jow}{Jowulu}
\define@key{names}{itk}{Judeo-Italian}
\define@key{names}{jdt}{Judeo-Tat}
\define@key{names}{jpr}{Judeo-Persian}
\define@key{names}{yud}{Judeo-Tripolitanian Arabic}
\define@key{names}{aju}{Judeo-Moroccan Arabic}
\define@key{names}{yhd}{Judeo-Iraqi Arabic}
\define@key{names}{jye}{Judeo-Yemeni Arabic}
\define@key{names}{jum}{Jumjum}
\define@key{names}{jml}{Jumli}
\define@key{names}{jus}{Jumla Sign Language}
\define@key{names}{mxq}{Juquila Mixe}
\define@key{names}{juy}{Juray}
\define@key{names}{jut}{Jutish}
\define@key{names}{juu}{Ju}
\define@key{names}{mwb}{Juwal}
\define@key{names}{vmc}{Juxtlahuaca Mixtec}
\define@key{names}{jwi}{Jwira-Pepesa}
\define@key{names}{xku}{Kaamba}
\define@key{names}{gna}{Kaansa}
\define@key{names}{ldl}{Kaan}
\define@key{names}{ckn}{Kaang Chin}
\define@key{names}{ksp}{Kaba}
\define@key{names}{kvf}{Kabalai}
\define@key{names}{gbw}{Kabikabi}
\define@key{names}{klz}{Kabola}
\define@key{names}{onk}{Kabore One}
\define@key{names}{lkb}{Kabras}
\define@key{names}{uka}{Kaburi}
\define@key{names}{kbu}{Kabutra}
\define@key{names}{kea}{Kabuverdianu}
\define@key{names}{cwa}{Kabwa}
\define@key{names}{kcw}{Kabwari}
\define@key{names}{gjk}{Kachi Koli}
\define@key{names}{kfr}{Kachchi}
\define@key{names}{kcx}{Kachama-Ganjule-Haro}
\define@key{names}{xkk}{Kaco'}
\define@key{names}{kej}{Kadar}
\define@key{names}{kdu}{Kadaru}
\define@key{names}{kad}{Kadara}
\define@key{names}{kzd}{Kadai}
\define@key{names}{kdv}{Kado}
\define@key{names}{ktp}{Kaduo}
\define@key{names}{jka}{Kaera}
\define@key{names}{kpu}{Kafoa}
\define@key{names}{sqx}{Kafr Qasem Sign Language}
\define@key{names}{syw}{Kagate}
\define@key{names}{kll}{Kagan Kalagan}
\define@key{names}{cgc}{Kagayanen}
\define@key{names}{gel}{Ut-Main}
\define@key{names}{xkg}{Kagoro}
\define@key{names}{hka}{Kahe}
\define@key{names}{agw}{Kahua}
\define@key{names}{kzb}{Kaibobo}
\define@key{names}{kzp}{Kaidipang}
\define@key{names}{kbw}{Kaiep}
\define@key{names}{kep}{Kaikadi}
\define@key{names}{kzq}{Kaike}
\define@key{names}{kkq}{Kaiku}
\define@key{names}{xai}{Kaimbé}
\define@key{names}{zka}{Kaimbulawa}
\define@key{names}{krd}{Kairui-Midiki}
\define@key{names}{ckr}{Kairak}
\define@key{names}{kzm}{Kais}
\define@key{names}{kce}{Kaivi}
\define@key{names}{tcq}{Kaiy}
\define@key{names}{xkj}{Kajali}
\define@key{names}{kag}{Kajaman}
\define@key{names}{ckq}{Kajakse}
\define@key{names}{kjv}{Kajkavian}
\define@key{names}{xdq}{Kajtak}
\define@key{names}{kka}{Kakanda}
\define@key{names}{kke}{Kakabe}
\define@key{names}{kqf}{Kakabai}
\define@key{names}{kkj}{Kako}
\define@key{names}{keo}{Kakwa}
\define@key{names}{wkl}{Kalanadi}
\define@key{names}{kzz}{Kalabra}
\define@key{names}{kkf}{Kalaktang Monpa}
\define@key{names}{kba}{Kalarko-Mirniny}
\define@key{names}{gll}{Bulloo River}
\define@key{names}{ijn}{Kalabari}
\define@key{names}{knz}{Kalamsé}
\define@key{names}{kqe}{Kalagan}
\define@key{names}{kve}{Kalabakan}
\define@key{names}{kly}{Kalao}
\define@key{names}{lkm}{Kalaamaya}
\define@key{names}{xka}{Kalkoti}
\define@key{names}{rmf}{Kalo Finnish Romani}
\define@key{names}{ywa}{Kalou}
\define@key{names}{kli}{Kalumpang}
\define@key{names}{keq}{Kamar}
\define@key{names}{jmr}{Kamara}
\define@key{names}{kci}{Kamantan}
\define@key{names}{klp}{Kamasa}
\define@key{names}{kzx}{Kamarian}
\define@key{names}{kyk}{Kamayo}
\define@key{names}{kgx}{Kamaru}
\define@key{names}{vkm}{Kamakan}
\define@key{names}{xbw}{Kambiwá}
\define@key{names}{irx}{Kamberau}
\define@key{names}{kyy}{Kambaira}
\define@key{names}{ktb}{Kambaata}
\define@key{names}{kmi}{Kami (Nigeria)}
\define@key{names}{kdx}{Kam}
\define@key{names}{kcq}{Kamo}
\define@key{names}{xla}{Kamula}
\define@key{names}{hig}{Kamwe}
\define@key{names}{bjj}{Kanauji}
\define@key{names}{xnb}{Kanakanavu}
\define@key{names}{soq}{Kanasi}
\define@key{names}{kbs}{Kande}
\define@key{names}{kqw}{Kandas}
\define@key{names}{gam}{Kandawo}
\define@key{names}{xnr}{Kangri}
\define@key{names}{kxs}{Kangjia}
\define@key{names}{kzy}{Kango (Tshopo District)}
\define@key{names}{kty}{Kango (Bas-Uélé District)}
\define@key{names}{kcp}{Kanga}
\define@key{names}{kkv}{Kangean}
\define@key{names}{igm}{Kanggape}
\define@key{names}{kev}{Kanikkaran}
\define@key{names}{kdp}{Kaningdon-Nindem}
\define@key{names}{kzo}{Kaningi}
\define@key{names}{wat}{Kaninuwa}
\define@key{names}{ktk}{Kaniet}
\define@key{names}{knr}{Kaningra}
\define@key{names}{kmu}{Kanite}
\define@key{names}{kft}{Kanjari}
\define@key{names}{kbe}{Kanju}
\define@key{names}{kxn}{Kanowit-Tanjong Melanau}
\define@key{names}{ksk}{Kansa}
\define@key{names}{xkt}{Kantosi}
\define@key{names}{kni}{Kanufi}
\define@key{names}{khx}{Kanu}
\define@key{names}{kqn}{Kaonde}
\define@key{names}{kax}{Kao}
\define@key{names}{xpn}{Kapinawá}
\define@key{names}{tbx}{Kapin}
\define@key{names}{khp}{Kapori}
\define@key{names}{ykm}{Kap}
\define@key{names}{kbi}{Kaptiau}
\define@key{names}{klo}{Kapya}
\define@key{names}{xkh}{Karahawyana}
\define@key{names}{kzr}{Karang}
\define@key{names}{reg}{Kara (Tanzania)}
\define@key{names}{kth}{Karanga}
\define@key{names}{mry}{Mandaya}
\define@key{names}{xrw}{Karawa}
\define@key{names}{xar}{Karami}
\define@key{names}{kgv}{Kalamang}
\define@key{names}{kbn}{Kare (Central African Republic)}
\define@key{names}{kyd}{Karey}
\define@key{names}{kmf}{Kare (Papua New Guinea)}
\define@key{names}{kai}{Karekare}
\define@key{names}{kmv}{Uaçá Creole French}
\define@key{names}{kgn}{Karingani-Kalasuri-Khoynarudi}
\define@key{names}{kbj}{Kari}
\define@key{names}{kil}{Kariya}
\define@key{names}{kuq}{Karipúna}
\define@key{names}{kko}{Karko}
\define@key{names}{krb}{Karkin}
\define@key{names}{bbv}{Karnai}
\define@key{names}{krx}{Karon}
\define@key{names}{kxh}{Karo (Ethiopia)}
\define@key{names}{xkx}{Karore}
\define@key{names}{kyn}{Northern Binukidnon}
\define@key{names}{rxw}{Karruwali}
\define@key{names}{ccj}{Kasanga}
\define@key{names}{ksn}{Kasiguranin}
\define@key{names}{kkz}{Kaska}
\define@key{names}{khs}{Kasua}
\define@key{names}{ktq}{Katabaga}
\define@key{names}{xat}{Katawixi}
\define@key{names}{tmb}{Avava}
\define@key{names}{tkt}{Kathoriya Tharu}
\define@key{names}{ykt}{Thou-Kathu}
\define@key{names}{kfu}{Katkari}
\define@key{names}{kaf}{Katso}
\define@key{names}{kta}{Katua}
\define@key{names}{vkk}{Kaur}
\define@key{names}{xau}{Kauwera}
\define@key{names}{ckv}{Kavalan}
\define@key{names}{kcb}{Kawacha}
\define@key{names}{kgb}{Kawe}
\define@key{names}{kaw}{Kawi}
\define@key{names}{ktx}{Kaxararí}
\define@key{names}{kbb}{Kaxuiâna}
\define@key{names}{pdu}{Kayan Lahwi}
\define@key{names}{xay}{Kayan Mahakam}
\define@key{names}{xkn}{Kayan River Kayan}
\define@key{names}{kyt}{Kayagar}
\define@key{names}{kzl}{Kayeli}
\define@key{names}{kxy}{Kayong}
\define@key{names}{kzu}{Kayupulau}
\define@key{names}{kzk}{Kazukuru}
\define@key{names}{keh}{Keak}
\define@key{names}{khz}{Keapara}
\define@key{names}{meo}{Kedah-Perak Malay}
\define@key{names}{kdy}{Keder}
\define@key{names}{khh}{Kehu}
\define@key{names}{kec}{Keiga}
\define@key{names}{bmh}{Kein}
\define@key{names}{eyo}{Keiyo}
\define@key{names}{khy}{Kele-Foma}
\define@key{names}{keb}{Kélé}
\define@key{names}{ify}{Keley-i Kallahan}
\define@key{names}{kbo}{Keliko}
\define@key{names}{xel}{Kelo}
\define@key{names}{kyo}{Klon}
\define@key{names}{kem}{Kemak}
\define@key{names}{bzp}{Kemberano}
\define@key{names}{xem}{Mateq}
\define@key{names}{xkw}{Kembra}
\define@key{names}{dmo}{Kemezung}
\define@key{names}{sjk}{Kemi Saami}
\define@key{names}{xbn}{Kenaboi}
\define@key{names}{gat}{Kenati}
\define@key{names}{kvm}{Kendem}
\define@key{names}{klf}{Kendeje}
\define@key{names}{knx}{Kendayan-Belangin}
\define@key{names}{knl}{Keninjal}
\define@key{names}{kxi}{Keningau Murut}
\define@key{names}{kns}{Kensiu}
\define@key{names}{ndb}{Kenswei Nsei}
\define@key{names}{kzh}{Kenuzi-Dongola}
\define@key{names}{lke}{Kenyi}
\define@key{names}{xeu}{Keoru-Ahia}
\define@key{names}{kpn}{Kepkiriwát}
\define@key{names}{kuk}{Kepo'}
\define@key{names}{hhr}{Keerak}
\define@key{names}{ked}{Kerewe}
\define@key{names}{xke}{Kereho}
\define@key{names}{kxz}{Kerewo}
\define@key{names}{kvr}{Kerinci}
\define@key{names}{xes}{Kesawai}
\define@key{names}{kae}{Ketangalan}
\define@key{names}{ktt}{Ketum}
\define@key{names}{kyg}{Keyagana}
\define@key{names}{xkv}{Kgalagadi}
\define@key{names}{hkh}{Khah}
\define@key{names}{kbg}{Khamba}
\define@key{names}{kht}{Khamti}
\define@key{names}{ksu}{Khamyang}
\define@key{names}{khn}{Khandesi}
\define@key{names}{kjm}{Kháng}
\define@key{names}{ksy}{Kharia Thar}
\define@key{names}{kfw}{Kharam Naga}
\define@key{names}{lko}{Khayo}
\define@key{names}{kqg}{Khe}
\define@key{names}{tlx}{Khehek}
\define@key{names}{xkf}{Khengkha}
\define@key{names}{xhe}{Khetrani}
\define@key{names}{nkh}{Khezha Naga}
\define@key{names}{kix}{Khiamniungan Naga}
\define@key{names}{kwx}{Khirwar}
\define@key{names}{kqm}{Khisa}
\define@key{names}{ykl}{Khlula}
\define@key{names}{xkc}{Kho'ini}
\define@key{names}{nkb}{Khoibu}
\define@key{names}{ktc}{Kholok}
\define@key{names}{kho}{Khotanese}
\define@key{names}{khf}{Khuen}
\define@key{names}{kfm}{Khunsaric}
\define@key{names}{xco}{Khwarezmian}
\define@key{names}{kie}{Kibet}
\define@key{names}{prm}{Kibiri}
\define@key{names}{kzg}{Kikai}
\define@key{names}{kih}{Kilmeri}
\define@key{names}{kqr}{Kimaragang}
\define@key{names}{kmb}{Kimbundu}
\define@key{names}{kiv}{Kimbu}
\define@key{names}{sbt}{Kimki}
\define@key{names}{kqp}{Kimre}
\define@key{names}{krj}{Kinaray-a}
\define@key{names}{kco}{Kinalakna}
\define@key{names}{cbw}{Kinabalian}
\define@key{names}{knq}{Kintaq}
\define@key{names}{kkd}{Kinuku}
\define@key{names}{ues}{Kioko}
\define@key{names}{kkm}{Kiong}
\define@key{names}{apk}{Kiowa Apache}
\define@key{names}{sgc}{Kipsigis}
\define@key{names}{kyi}{Kiput}
\define@key{names}{kkr}{Kir-Balar}
\define@key{names}{okr}{Kirike}
\define@key{names}{kiu}{Kirmanjki}
\define@key{names}{fkk}{Kirya-Konzel}
\define@key{names}{lks}{Kisa}
\define@key{names}{kiz}{Kisi}
\define@key{names}{kis}{Kis}
\define@key{names}{zkt}{Kitan}
\define@key{names}{mwk}{Kita Maninkakan}
\define@key{names}{mkw}{Kituba (Congo)}
\define@key{names}{kqt}{Klias River Kadazan}
\define@key{names}{tlh}{Klingon}
\define@key{names}{kib}{Koalib-Rere}
\define@key{names}{kpd}{Koba}
\define@key{names}{kcj}{Kobiana}
\define@key{names}{kgu}{Kobol}
\define@key{names}{thq}{Kochila Tharu}
\define@key{names}{kdq}{Koch}
\define@key{names}{dhw}{Kochariya-East Danuwar}
\define@key{names}{cdz}{Koda}
\define@key{names}{ksz}{Kodaku}
\define@key{names}{vko}{Kodeoha}
\define@key{names}{kwp}{Kodia}
\define@key{names}{kod}{Kodi-Gaura}
\define@key{names}{kcs}{Koenoem}
\define@key{names}{kpi}{Kofei}
\define@key{names}{kwl}{Pan}
\define@key{names}{zkg}{Koguryo}
\define@key{names}{plk}{Kohistani Shina}
\define@key{names}{kkx}{Kohin}
\define@key{names}{kkt}{Koi}
\define@key{names}{nkd}{Koireng}
\define@key{names}{kxt}{Koiwat}
\define@key{names}{kou}{Koke}
\define@key{names}{gko}{Kok-Nar}
\define@key{names}{xod}{Kokoda}
\define@key{names}{kzn}{Kokola}
\define@key{names}{klc}{Kolbila}
\define@key{names}{ekl}{Kol (Bangladesh)}
\define@key{names}{biw}{Kol (Cameroon)}
\define@key{names}{skn}{Kolibugan Subanon}
\define@key{names}{klm}{Kolom}
\define@key{names}{kol}{Kol (Papua New Guinea)}
\define@key{names}{klx}{Koluwawa}
\define@key{names}{kmy}{Koma Ndera}
\define@key{names}{kpf}{Komba}
\define@key{names}{tyn}{Kombai}
\define@key{names}{kmm}{Kom (India)}
\define@key{names}{xoi}{Kominimung}
\define@key{names}{kmw}{Komo (Democratic Republic of Congo)}
\define@key{names}{kvh}{Komodo}
\define@key{names}{kvp}{Kompane}
\define@key{names}{kzv}{Komyandaret}
\define@key{names}{kxw}{Konai}
\define@key{names}{knd}{Yaben (Konda)}
\define@key{names}{kdw}{Koneraw}
\define@key{names}{klk}{Kono (Nigeria)}
\define@key{names}{kcz}{Konongo-Ruwila}
\define@key{names}{knu}{Kono (Guinea)}
\define@key{names}{kno}{Kono (Sierra Leone)}
\define@key{names}{koa}{Konomala}
\define@key{names}{kxc}{Konso}
\define@key{names}{nbe}{Konyak Naga}
\define@key{names}{mku}{Konyanka Maninka}
\define@key{names}{koo}{Konzo}
\define@key{names}{ozm}{Koonzime}
\define@key{names}{fuj}{Ko}
\define@key{names}{xop}{Kopar}
\define@key{names}{opk}{Kopkaka}
\define@key{names}{kcy}{Korandje}
\define@key{names}{koz}{Korak}
\define@key{names}{okh}{Karanic}
\define@key{names}{vkp}{Korlai Portuguese}
\define@key{names}{ktl}{Koroshi}
\define@key{names}{krp}{Korop}
\define@key{names}{kfo}{Koro (Côte d'Ivoire)}
\define@key{names}{krf}{Koro-Olrat}
\define@key{names}{xkq}{Koroni}
\define@key{names}{kqj}{Koromira}
\define@key{names}{jkr}{Koro}
\define@key{names}{vkn}{Koro Nulu}
\define@key{names}{vkz}{Koro Zuba}
\define@key{names}{kfd}{Korra Koraga}
\define@key{names}{kpq}{Korupun-Sela}
\define@key{names}{xor}{Korubo}
\define@key{names}{kfp}{Korwa}
\define@key{names}{kiq}{Kosadle}
\define@key{names}{kid}{Koshin}
\define@key{names}{kqk}{Kotafon Gbe}
\define@key{names}{koq}{Kota (Gabon)}
\define@key{names}{mqg}{Kota Bangun Kutai Malay}
\define@key{names}{grm}{Kota Marudu Talantang}
\define@key{names}{avk}{Kotava}
\define@key{names}{zko}{Kott-Assan}
\define@key{names}{kyf}{Kouya}
\define@key{names}{kqb}{Kovai}
\define@key{names}{kvc}{Kove}
\define@key{names}{xow}{Kowaki}
\define@key{names}{kwh}{Kowiai}
\define@key{names}{kga}{Koyaga}
\define@key{names}{koh}{Koyo}
\define@key{names}{kqd}{Koy Sanjaq Jewish Neo-Aramaic}
\define@key{names}{kuw}{Kpagua}
\define@key{names}{kpl}{Kpala}
\define@key{names}{pbn}{Kpasam}
\define@key{names}{koc}{Kpati}
\define@key{names}{cpo}{Kpeego}
\define@key{names}{kef}{Kpessi}
\define@key{names}{kph}{Kplang}
\define@key{names}{kye}{Krache}
\define@key{names}{rka}{Kraol}
\define@key{names}{xre}{Northeastern Timbira}
\define@key{names}{kri}{Krio}
\define@key{names}{kxb}{Krobu}
\define@key{names}{tyu}{Southern Tshwa}
\define@key{names}{yku}{Kuamasi}
\define@key{names}{uan}{Kuan}
\define@key{names}{kua}{Kuanyama}
\define@key{names}{ykn}{Kua-nsi}
\define@key{names}{ugh}{Kubachi}
\define@key{names}{kgf}{Kulungtfu-Yuanggeng-Tobo}
\define@key{names}{kof}{Kubi}
\define@key{names}{jko}{Kubo}
\define@key{names}{kvb}{Kubu}
\define@key{names}{lkc}{Kucong}
\define@key{names}{kfg}{Kudiya}
\define@key{names}{kyw}{Kudmali}
\define@key{names}{kov}{Kudu-Camo}
\define@key{names}{kow}{Gengle-Kugama}
\define@key{names}{kes}{Kugbo}
\define@key{names}{dkr}{Kuijau}
\define@key{names}{vkj}{Kujarge}
\define@key{names}{kux}{Kukatja}
\define@key{names}{kez}{Kukele}
\define@key{names}{kfn}{Kuk}
\define@key{names}{ugb}{Kuku-Ugbanh}
\define@key{names}{xmp}{Kuku-Mu'inh}
\define@key{names}{xmh}{Kuku-Muminh}
\define@key{names}{ukv}{Kuku}
\define@key{names}{kul}{Kulere}
\define@key{names}{kxj}{Kulfa}
\define@key{names}{vkl}{Kulisusu}
\define@key{names}{xpk}{Kulina Pano}
\define@key{names}{kfx}{Kullu Pahari}
\define@key{names}{pzh}{Pazeh-Kahabu}
\define@key{names}{uon}{Kulon}
\define@key{names}{bbu}{Kulung (Nigeria)}
\define@key{names}{kdi}{Kumam}
\define@key{names}{ksl}{Kumalu}
\define@key{names}{ksm}{Kumba}
\define@key{names}{xks}{Kumbewaha}
\define@key{names}{kra}{Kumhali}
\define@key{names}{kuo}{Kumukio}
\define@key{names}{zum}{Kumzari}
\define@key{names}{wku}{Kunduvadi}
\define@key{names}{kdn}{Chikunda}
\define@key{names}{shd}{Kundal Shahi}
\define@key{names}{kgl}{Kunggari}
\define@key{names}{ggk}{Kungarakany}
\define@key{names}{kfl}{Kung}
\define@key{names}{kse}{Kuni}
\define@key{names}{xug}{Kunigami}
\define@key{names}{pep}{Kánchá}
\define@key{names}{njx}{Kunyi}
\define@key{names}{kug}{Kupa}
\define@key{names}{mkn}{Kupang Malay}
\define@key{names}{key}{Kupia}
\define@key{names}{nqk}{Kura Ede Nago}
\define@key{names}{krh}{Kurama}
\define@key{names}{kfh}{Kurichiya}
\define@key{names}{kuj}{Kuria}
\define@key{names}{nbn}{Nabi}
\define@key{names}{kfv}{Kurmukar}
\define@key{names}{vku}{Kurrama}
\define@key{names}{kuv}{Kur}
\define@key{names}{xkz}{Kurtokha}
\define@key{names}{ktm}{Kurti}
\define@key{names}{kjr}{Kurudu}
\define@key{names}{kyr}{Kuruáya}
\define@key{names}{kus}{Kusaal}
\define@key{names}{ksg}{Kusaghe-Njela}
\define@key{names}{kuh}{Kushi}
\define@key{names}{ksv}{Kusu}
\define@key{names}{ght}{Kutang Ghale}
\define@key{names}{kub}{Kutep}
\define@key{names}{xut}{Kuthant}
\define@key{names}{kpa}{Kutto}
\define@key{names}{khj}{Kuturmi}
\define@key{names}{kdc}{Kutu}
\define@key{names}{uky}{Kuuk-Yak}
\define@key{names}{lku}{Kuungkari of Barcoo River}
\define@key{names}{olu}{Kuvale}
\define@key{names}{cwt}{Kuwaataay}
\define@key{names}{blh}{Kuwaa}
\define@key{names}{kdt}{Kuy}
\define@key{names}{fkv}{Kven Finnish}
\define@key{names}{kwb}{Baa}
\define@key{names}{bko}{Kwa'}
\define@key{names}{kwz}{Kwadi}
\define@key{names}{wka}{Kw'adza}
\define@key{names}{kdz}{Kwaja-Ndaktup}
\define@key{names}{kwu}{Kwakum}
\define@key{names}{qwt}{Kwalhioqua-Clatskanie}
\define@key{names}{kmq}{Gwama}
\define@key{names}{ktf}{Kwami}
\define@key{names}{kwm}{Kwambi}
\define@key{names}{okk}{Kwamtim One}
\define@key{names}{knp}{Kwanja}
\define@key{names}{kwj}{Kwanga}
\define@key{names}{kvi}{Kwang}
\define@key{names}{xdo}{Kwandu}
\define@key{names}{kwf}{Kwara'ae}
\define@key{names}{kop}{Kwato}
\define@key{names}{kya}{Kwaya}
\define@key{names}{cwe}{Kwere}
\define@key{names}{xwr}{Kwerba Mamberamo}
\define@key{names}{kkb}{Kwerisa}
\define@key{names}{kwr}{Kwer}
\define@key{names}{kws}{Kwese}
\define@key{names}{kwt}{Kwesten}
\define@key{names}{kuc}{Kwinsu}
\define@key{names}{kww}{Kwinti}
\define@key{names}{bka}{Kyak}
\define@key{names}{tye}{Kyenga}
\define@key{names}{kql}{Kyenele}
\define@key{names}{ldn}{Láadan}
\define@key{names}{bwj}{Láá Láá Bwamu}
\define@key{names}{ldi}{Laari}
\define@key{names}{lbb}{Label}
\define@key{names}{lbi}{La'bi}
\define@key{names}{jku}{Labir}
\define@key{names}{ypb}{Labo Phowa}
\define@key{names}{mwi}{Ninde}
\define@key{names}{dtb}{Labuk-Kinabatangan Kadazan}
\define@key{names}{zpl}{Lachixío Zapotec}
\define@key{names}{zpa}{Lachiguiri Zapotec}
\define@key{names}{lkl}{Laeko-Libuat}
\define@key{names}{lgh}{Laghuu}
\define@key{names}{lgb}{Laghu}
\define@key{names}{lhh}{Laha (Indonesia)}
\define@key{names}{lhn}{Lahanan}
\define@key{names}{lhl}{Lahul Lohar}
\define@key{names}{lhi}{Lahu Shi}
\define@key{names}{lmx}{Laimbue}
\define@key{names}{lji}{Laiyolo}
\define@key{names}{lap}{Laka (Chad)}
\define@key{names}{lka}{Lakalei}
\define@key{names}{lkh}{Lakha}
\define@key{names}{lki}{Laki}
\define@key{names}{lkn}{Lakon}
\define@key{names}{lkd}{Lakondê}
\define@key{names}{lxm}{Lakuramau}
\define@key{names}{lla}{Lala-Roba}
\define@key{names}{leb}{Lala-Bisa}
\define@key{names}{cnl}{Lalana Chinantec}
\define@key{names}{las}{Lama (Togo)}
\define@key{names}{lmr}{Peripheral Lembata}
\define@key{names}{lmq}{Lamatuka}
\define@key{names}{lai}{Lambya}
\define@key{names}{lmy}{Lamboya}
\define@key{names}{quf}{Lambayeque Quechua}
\define@key{names}{lbn}{Lamet}
\define@key{names}{bma}{Lame}
\define@key{names}{ldh}{Lamja-Dengsa-Tola}
\define@key{names}{lmk}{Lamkang}
\define@key{names}{lev}{Western Pantar}
\define@key{names}{lmg}{Lamogai}
\define@key{names}{abl}{Lampung Nyo}
\define@key{names}{llh}{Lamu}
\define@key{names}{ruu}{Lanas Lobu}
\define@key{names}{ldm}{Landoma}
\define@key{names}{sfb}{Langue des signes de Belgique Francophone}
\define@key{names}{yln}{Langnian Buyang}
\define@key{names}{lna}{Langbashe}
\define@key{names}{lno}{Lango-Logire-Logir}
\define@key{names}{lnm}{Pondi}
\define@key{names}{lnh}{Lanoh}
\define@key{names}{lwm}{Laomian}
\define@key{names}{ztl}{Lapaguía-Guivini Zapotec}
\define@key{names}{laa}{Lapuyan Subanun}
\define@key{names}{lrt}{Larantuka Malay}
\define@key{names}{lrv}{Larevat}
\define@key{names}{hmd}{Diandongbei-Large Flowery Miao}
\define@key{names}{lrl}{Larestani}
\define@key{names}{lro}{Laru (North Sudan)}
\define@key{names}{lar}{Larteh}
\define@key{names}{lan}{Laru (Nigeria)}
\define@key{names}{llm}{Lasalimu}
\define@key{names}{lsa}{Lasgerdi}
\define@key{names}{lsi}{Lashi}
\define@key{names}{lss}{Lasi}
\define@key{names}{lat}{Latin}
\define@key{names}{ltu}{Latu}
\define@key{names}{ltn}{Latundê}
\define@key{names}{lsl}{Latvian Sign Language}
\define@key{names}{llx}{Lauan}
\define@key{names}{luf}{Laua}
\define@key{names}{lre}{Laurentian}
\define@key{names}{clt}{Lautu}
\define@key{names}{lbv}{Lavatbura-Lamusong}
\define@key{names}{lbx}{Lawangan}
\define@key{names}{lvi}{Lawi}
\define@key{names}{tgi}{Lawunuia}
\define@key{names}{lwu}{Lawu}
\define@key{names}{lya}{Layakha}
\define@key{names}{ldk}{Leelau}
\define@key{names}{lfa}{Lefa}
\define@key{names}{lgm}{Lega-Mwenga}
\define@key{names}{lcc}{Legenyem}
\define@key{names}{cae}{Lehar}
\define@key{names}{tql}{Lehali}
\define@key{names}{urr}{Lehalurup}
\define@key{names}{lzn}{Leinong Naga}
\define@key{names}{lek}{Leipon}
\define@key{names}{llk}{Lelak}
\define@key{names}{lel}{Lele (Democratic Republic of Congo)}
\define@key{names}{llc}{Lele (Guinea)}
\define@key{names}{lpa}{Lelepa}
\define@key{names}{lle}{Lele (Papua New Guinea)}
\define@key{names}{leq}{Lembena}
\define@key{names}{lrz}{Lemerig}
\define@key{names}{lei}{Lemio}
\define@key{names}{xle}{Lemnian}
\define@key{names}{ldj}{Lemoro}
\define@key{names}{ley}{Lemolang}
\define@key{names}{lej}{Lengola}
\define@key{names}{lgr}{Lengo}
\define@key{names}{lgi}{Lengilu}
\define@key{names}{leh}{Lenje}
\define@key{names}{ler}{Lenkau}
\define@key{names}{ldg}{Lenyima}
\define@key{names}{lpe}{Lepki}
\define@key{names}{xlp}{Lepontic}
\define@key{names}{gnh}{Lere}
\define@key{names}{let}{Lesing-Gelimi}
\define@key{names}{nms}{Letemboi-Repanbitip}
\define@key{names}{leo}{Leti (Cameroon)}
\define@key{names}{lvu}{Central Lembata-Lewokukun}
\define@key{names}{lwe}{Lewo Eleng}
\define@key{names}{lwt}{Lewotobi}
\define@key{names}{ayi}{Leyigha}
\define@key{names}{lhp}{Lhokpu}
\define@key{names}{lix}{Liabuku}
\define@key{names}{njn}{Liangmai Naga}
\define@key{names}{zln}{Lianshan Zhuang}
\define@key{names}{ste}{Liana-Seti}
\define@key{names}{lir}{Kru Pidgin English}
\define@key{names}{liz}{Libinza}
\define@key{names}{liq}{Libido}
\define@key{names}{lbs}{Libyan Sign Language}
\define@key{names}{lig}{Ligbi}
\define@key{names}{lgz}{Ligenza}
\define@key{names}{lih}{Lihir}
\define@key{names}{mgi}{Lijili}
\define@key{names}{lik}{Liko}
\define@key{names}{lie}{Balobo}
\define@key{names}{lio}{Liki}
\define@key{names}{kxx}{Likuba}
\define@key{names}{lib}{Likum}
\define@key{names}{kwc}{Likwala}
\define@key{names}{lll}{Lilau}
\define@key{names}{bme}{Limassa}
\define@key{names}{lim}{Limburgan}
\define@key{names}{lmp}{Limbum}
\define@key{names}{ylm}{Limi}
\define@key{names}{kmk}{Limos Kalinga}
\define@key{names}{qlm}{Limonese Creole}
\define@key{names}{klw}{Tado-Lindu}
\define@key{names}{pml}{Mediterranean Lingua Franca}
\define@key{names}{onb}{Western Ong-Be}
\define@key{names}{lgk}{Neverver}
\define@key{names}{lfn}{Lingua Franca Nova}
\define@key{names}{ljl}{Li'o}
\define@key{names}{apl}{Lipan Apache}
\define@key{names}{lpo}{Lipo}
\define@key{names}{lcs}{Lisabata-Nuniali}
\define@key{names}{lcl}{Lisela}
\define@key{names}{lsh}{Khispi}
\define@key{names}{lsd}{Lishana Deni}
\define@key{names}{lzh}{Literary Chinese}
\define@key{names}{lls}{Lithuanian Sign Language}
\define@key{names}{lzl}{Naman}
\define@key{names}{zlj}{Liujiang Zhuang}
\define@key{names}{zlq}{Liuqian Zhuang}
\define@key{names}{olo}{Livvi}
\define@key{names}{loq}{Lobala}
\define@key{names}{lbm}{Lodhi}
\define@key{names}{lgq}{Ikpana}
\define@key{names}{rag}{Logooli}
\define@key{names}{liu}{Logorik}
\define@key{names}{lof}{Logol}
\define@key{names}{src}{Logudorese Sardinian}
\define@key{names}{qvj}{Loja Highland Quichua}
\define@key{names}{jbo}{Lojban}
\define@key{names}{yaz}{Lokaa}
\define@key{names}{lky}{Lokoya}
\define@key{names}{lcd}{Lola}
\define@key{names}{llq}{Lolak}
\define@key{names}{llg}{Lole}
\define@key{names}{ycl}{Lolopo}
\define@key{names}{llb}{Lolo}
\define@key{names}{loa}{Loloda-Laba}
\define@key{names}{rmi}{Lomavren}
\define@key{names}{loi}{Loma (Côte d'Ivoire)}
\define@key{names}{lmv}{Lomaiviti}
\define@key{names}{lmi}{Lombi}
\define@key{names}{lmo}{Lombard}
\define@key{names}{loo}{Lombo}
\define@key{names}{ngl}{Mozambique Lomwe}
\define@key{names}{lce}{Loncong}
\define@key{names}{lpn}{Long Phuri Naga}
\define@key{names}{wok}{Longto}
\define@key{names}{lnu}{Longuda}
\define@key{names}{ttw}{Western Lowland Kenyah}
\define@key{names}{ldo}{Loo}
\define@key{names}{lop}{Lopa}
\define@key{names}{lpx}{Lopit}
\define@key{names}{lrn}{Lorang}
\define@key{names}{spq}{Peruvian Amazonian Spanish}
\define@key{names}{lnn}{Nethalp}
\define@key{names}{uvl}{Lote}
\define@key{names}{lht}{Lo-Toga}
\define@key{names}{dtr}{Lotud}
\define@key{names}{lou}{Louisiana Creole French}
\define@key{names}{lox}{Loun}
\define@key{names}{xlo}{Loup A}
\define@key{names}{sli}{Lower Silesian}
\define@key{names}{tto}{Lower Ta'oih}
\define@key{names}{nsb}{Lower-Nosop}
\define@key{names}{kml}{Tanudan Kalinga}
\define@key{names}{cea}{Lower Chehalis}
\define@key{names}{axl}{Lower Southern Aranda}
\define@key{names}{ztp}{Loxicha Zapotec}
\define@key{names}{kcc}{Lubila}
\define@key{names}{lcf}{Lubu}
\define@key{names}{knb}{Lubuagan Kalinga}
\define@key{names}{luq}{Lucumi}
\define@key{names}{lud}{Ludian}
\define@key{names}{ldq}{Lufu}
\define@key{names}{ruf}{Luguru}
\define@key{names}{lcq}{Luhu-Piru}
\define@key{names}{lum}{Luimbi}
\define@key{names}{dop}{Lukpa}
\define@key{names}{smj}{Lule Saami}
\define@key{names}{lmz}{Lumbee}
\define@key{names}{lup}{Lumbu}
\define@key{names}{lmd}{Lumun}
\define@key{names}{luk}{Lunanakha}
\define@key{names}{luj}{Luna}
\define@key{names}{lga}{Lungga}
\define@key{names}{luw}{Luo (Cameroon)}
\define@key{names}{hml}{Luopohe Hmong}
\define@key{names}{ldd}{Luri}
\define@key{names}{lse}{Lusengo}
\define@key{names}{xls}{Lusitanian}
\define@key{names}{ndy}{Lutos}
\define@key{names}{luv}{Luwati}
\define@key{names}{lyn}{Luyi}
\define@key{names}{lwa}{Lwalu}
\define@key{names}{xlc}{Lycian A}
\define@key{names}{xld}{Lydian}
\define@key{names}{lyg}{India Lyngam}
\define@key{names}{cma}{Maa}
\define@key{names}{mew}{Maaka}
\define@key{names}{ymm}{Maay}
\define@key{names}{mmz}{Mabaale}
\define@key{names}{mfz}{Mabaan}
\define@key{names}{mqa}{Maba (Indonesia)}
\define@key{names}{kkg}{Mabaka Valley Kalinga}
\define@key{names}{muj}{Mabire}
\define@key{names}{mcl}{Macaguaje}
\define@key{names}{mzs}{Macanese}
\define@key{names}{mvw}{Machinga}
\define@key{names}{jmc}{Machame}
\define@key{names}{mpd}{Machinere}
\define@key{names}{wpc}{Maco}
\define@key{names}{mzc}{Madagascar Sign Language}
\define@key{names}{mmx}{Madak}
\define@key{names}{xmx}{Salawati}
\define@key{names}{grg}{Madi (Papua New Guinea)}
\define@key{names}{kmd}{Madukayang Kalinga}
\define@key{names}{mme}{Tirax}
\define@key{names}{itt}{Maeng Itneg}
\define@key{names}{maf}{Mafa}
\define@key{names}{mkv}{Mafea}
\define@key{names}{sgb}{Mag-Anchi Ayta}
\define@key{names}{mtw}{Southern Binukidnon}
\define@key{names}{xtm}{Magdalena Peñasco Mixtec}
\define@key{names}{gmd}{Mághdì}
\define@key{names}{blx}{Mag-Indi Ayta}
\define@key{names}{gkd}{Magi}
\define@key{names}{gmg}{Magiyi}
\define@key{names}{gmx}{Magoma}
\define@key{names}{zgr}{Magori}
\define@key{names}{bfz}{Mahasu Pahari}
\define@key{names}{mjx}{Mahali}
\define@key{names}{pmh}{Maharastri Prakrit}
\define@key{names}{mjy}{Mohican}
\define@key{names}{mhb}{Mahongwe}
\define@key{names}{mzz}{Maiadomu}
\define@key{names}{tnh}{Maiani}
\define@key{names}{sks}{Maia}
\define@key{names}{mmm}{Maii}
\define@key{names}{vmf}{Ostfränkisch}
\define@key{names}{cwb}{Maindo}
\define@key{names}{xkl}{Usun Apau Kenyah}
\define@key{names}{mum}{Maiwala}
\define@key{names}{wmm}{Maiwa (Indonesia)}
\define@key{names}{mti}{Maiwa (Papua New Guinea)}
\define@key{names}{xmj}{Majera}
\define@key{names}{mmj}{Majhwar}
\define@key{names}{mjz}{Majhi}
\define@key{names}{mfp}{Makassar Malay}
\define@key{names}{aup}{Makayam}
\define@key{names}{mkg}{Mak (China)}
\define@key{names}{vmk}{Makhuwa-Shirima}
\define@key{names}{xmc}{Makhuwa-Marrevone}
\define@key{names}{vmw}{Makhuwa}
\define@key{names}{mhm}{Makhuwa-Moniga}
\define@key{names}{xsq}{Makhuwa-Saka}
\define@key{names}{pbl}{Mak (Nigeria)}
\define@key{names}{zmh}{Makolkol}
\define@key{names}{jmn}{Makuri Naga}
\define@key{names}{lva}{Maku'a}
\define@key{names}{mpu}{Makuráp}
\define@key{names}{ymk}{Makwe}
\define@key{names}{umn}{Makyan Naga}
\define@key{names}{lon}{Malawi Lomwe}
\define@key{names}{xml}{Malaysian Sign Language}
\define@key{names}{ima}{Mala Malasar}
\define@key{names}{ymr}{Malasar}
\define@key{names}{mjo}{Malankuravan}
\define@key{names}{mjr}{Malavedan}
\define@key{names}{mjq}{Malaryan}
\define@key{names}{mjp}{Malapandaram}
\define@key{names}{ruy}{Mala (Nigeria)}
\define@key{names}{swk}{Malawi Sena}
\define@key{names}{ccm}{Malaccan Creole Malay}
\define@key{names}{mln}{Malango}
\define@key{names}{mqz}{Malasanga}
\define@key{names}{mmt}{Malalamai}
\define@key{names}{ped}{Mala (Papua New Guinea)}
\define@key{names}{mkr}{Manep}
\define@key{names}{lws}{Malawian Sign Language}
\define@key{names}{bfo}{Malba Birifor}
\define@key{names}{pkt}{Maleng}
\define@key{names}{mdc}{Male (Papua New Guinea)}
\define@key{names}{gut}{Maléku Jaíka}
\define@key{names}{mlx}{Na'ahai}
\define@key{names}{vml}{Malgana}
\define@key{names}{mxf}{Malgbe}
\define@key{names}{mgq}{Malila}
\define@key{names}{mzd}{Malimba}
\define@key{names}{mli}{Malimpung}
\define@key{names}{mlf}{Mal}
\define@key{names}{mbk}{Malol}
\define@key{names}{mkb}{Mar Paharia of Dumka}
\define@key{names}{mdl}{Maltese Sign Language}
\define@key{names}{mll}{Malua Bay}
\define@key{names}{mup}{Malvi}
\define@key{names}{myk}{Mamara Senoufo}
\define@key{names}{mma}{Mama}
\define@key{names}{mhf}{Mamaa}
\define@key{names}{wmd}{Mamaindé}
\define@key{names}{mvd}{Mamboru}
\define@key{names}{mgm}{Mambae}
\define@key{names}{kdf}{Mamusi}
\define@key{names}{mqx}{Mamuju}
\define@key{names}{znk}{Manangkari}
\define@key{names}{mjl}{Mandeali}
\define@key{names}{mha}{Manda (India)}
\define@key{names}{zma}{Manda (Australia)}
\define@key{names}{zmk}{Mandandanyi}
\define@key{names}{mgs}{Manda-Matumba}
\define@key{names}{mqu}{Mandari}
\define@key{names}{tbf}{Mandara}
\define@key{names}{mqr}{Mander}
\define@key{names}{aax}{Mandobo Atas}
\define@key{names}{bwp}{Mandobo Bawah}
\define@key{names}{mht}{Mandahuaca}
\define@key{names}{zng}{Mang}
\define@key{names}{zme}{Mangerr}
\define@key{names}{mem}{Mangala}
\define@key{names}{myj}{Mangayat}
\define@key{names}{mdk}{Mangbutu}
\define@key{names}{kby}{Manga Kanuri}
\define@key{names}{mrv}{Mangareva}
\define@key{names}{mbh}{Mangseng}
\define@key{names}{mmo}{Mangga Buang}
\define@key{names}{zns}{Mangas}
\define@key{names}{xkb}{Manigri-Kambolé Ede Nago}
\define@key{names}{mqp}{Manipa}
\define@key{names}{nlm}{Mankiyali}
\define@key{names}{mml}{Man Met}
\define@key{names}{mjv}{Mannan}
\define@key{names}{woo}{Manombai}
\define@key{names}{msw}{Mansoanka}
\define@key{names}{msk}{Mansaka}
\define@key{names}{nty}{Mantsi}
\define@key{names}{myg}{Manta}
\define@key{names}{kxf}{Manumanaw Karen}
\define@key{names}{wha}{Manusela}
\define@key{names}{mxc}{Manyika}
\define@key{names}{mny}{Manyawa}
\define@key{names}{mzj}{Manya}
\define@key{names}{mzv}{Manza}
\define@key{names}{mmd}{Maonan}
\define@key{names}{mjn}{Ma (Papua New Guinea)}
\define@key{names}{mlh}{Mape}
\define@key{names}{mnm}{Mapena}
\define@key{names}{mpy}{Mapia}
\define@key{names}{mpw}{Mapidian-Mawayana}
\define@key{names}{bzh}{Mapos Buang}
\define@key{names}{sjm}{Mapun}
\define@key{names}{vmh}{Maraghei}
\define@key{names}{nma}{Maram Naga}
\define@key{names}{lrm}{Marama}
\define@key{names}{lri}{Marachi}
\define@key{names}{mgb}{Mararit}
\define@key{names}{mvr}{Marau}
\define@key{names}{mrs}{Maragus}
\define@key{names}{mpg}{Marba}
\define@key{names}{dsz}{Mardin Sign Language}
\define@key{names}{vmr}{Marenje}
\define@key{names}{mrx}{Maremgi}
\define@key{names}{mvu}{Marfa}
\define@key{names}{mhg}{Margu}
\define@key{names}{qvm}{Margos-Yarowilca-Lauricocha Quechua}
\define@key{names}{mfm}{Marghi South}
\define@key{names}{nsr}{Maritime Sign Language}
\define@key{names}{mrr}{Maria (India)}
\define@key{names}{nng}{Maring Naga}
\define@key{names}{zmm}{Marimanindji}
\define@key{names}{zmj}{Maridjabin}
\define@key{names}{zmd}{Maridan}
\define@key{names}{zmy}{Mariyedi}
\define@key{names}{mrb}{Sunwadia}
\define@key{names}{dad}{Marik}
\define@key{names}{hob}{Mari (Madang Province)}
\define@key{names}{mqi}{Mariri}
\define@key{names}{mbx}{Mari (East Sepik Province)}
\define@key{names}{mds}{Maria (Papua New Guinea)}
\define@key{names}{msp}{Maritsauá}
\define@key{names}{enb}{Markweeta}
\define@key{names}{rkm}{Marka}
\define@key{names}{mvo}{Marovo}
\define@key{names}{xru}{Marriammu}
\define@key{names}{mre}{Martha's Vineyard Sign Language}
\define@key{names}{zmg}{Marti Ke}
\define@key{names}{mzr}{Marúbo}
\define@key{names}{mve}{Marwari (Pakistan)}
\define@key{names}{rwr}{Marwari (India)}
\define@key{names}{myx}{Masaaba}
\define@key{names}{tis}{Masadiit Itneg}
\define@key{names}{bks}{Masbate Sorsogon}
\define@key{names}{msb}{Masbatenyo}
\define@key{names}{mho}{Mashi (Zambia)}
\define@key{names}{jms}{Mashi (Nigeria)}
\define@key{names}{cuj}{Mashco Piro}
\define@key{names}{ism}{Masimasi}
\define@key{names}{bnf}{Masiwang}
\define@key{names}{msh}{Masikoro Malagasy}
\define@key{names}{klv}{Maskelynes}
\define@key{names}{msv}{Maslam}
\define@key{names}{mes}{Masmaje}
\define@key{names}{mdg}{Massalat}
\define@key{names}{mvs}{Massep}
\define@key{names}{mtn}{Matagalpa}
\define@key{names}{mfh}{Matal}
\define@key{names}{xmt}{Matbat}
\define@key{names}{mgv}{Matengo}
\define@key{names}{mqe}{Matepi}
\define@key{names}{mzo}{Matipuhy}
\define@key{names}{mtm}{Mator-Taigi-Karagas}
\define@key{names}{met}{Mato}
\define@key{names}{axg}{Mato Grosso Arára}
\define@key{names}{stj}{Matya Samo}
\define@key{names}{cty}{Maundadan Chetti}
\define@key{names}{lsy}{Mauritian Sign Language}
\define@key{names}{mhl}{Mauwake}
\define@key{names}{wma}{Mawa (Nigeria)}
\define@key{names}{mjj}{Mawak}
\define@key{names}{mcz}{Mawan}
\define@key{names}{mcw}{Mawa (Chad)}
\define@key{names}{mgk}{Mawes}
\define@key{names}{mxl}{Maxi Gbe}
\define@key{names}{xmy}{Mayaguduna}
\define@key{names}{sym}{Maya Samo}
\define@key{names}{mnt}{Maykulan}
\define@key{names}{ifu}{Mayoyao Ifugao}
\define@key{names}{mzl}{Mazatlán Mixe}
\define@key{names}{zpy}{Mazaltepec Zapotec}
\define@key{names}{vmz}{Mazatlán Mazatec}
\define@key{names}{dkx}{Mazagway}
\define@key{names}{mdp}{Mbala}
\define@key{names}{mgn}{Mbangi}
\define@key{names}{zmz}{Mbandja}
\define@key{names}{mxg}{Mbangala}
\define@key{names}{zmn}{Mbangwe}
\define@key{names}{zmv}{Rimanggudhinma}
\define@key{names}{mvl}{Mbara-Yanga}
\define@key{names}{gwa}{Mbato}
\define@key{names}{mdn}{Mbati}
\define@key{names}{xmd}{Mbedam}
\define@key{names}{mfo}{Mbe}
\define@key{names}{mql}{Mbelime}
\define@key{names}{zms}{Mbesa}
\define@key{names}{emz}{Mbessa}
\define@key{names}{mbo}{Mbo (Cameroon)}
\define@key{names}{zmw}{Mbo (Democratic Republic of Congo)}
\define@key{names}{moi}{Mboi}
\define@key{names}{mdu}{Mboko}
\define@key{names}{xmb}{Mbonga}
\define@key{names}{bgu}{Mbongno}
\define@key{names}{mxo}{Mbowe}
\define@key{names}{mka}{Mbre}
\define@key{names}{mgz}{Mbugwe}
\define@key{names}{mhw}{Mbukushu}
\define@key{names}{mqb}{Mbuko}
\define@key{names}{bpc}{Mbuk}
\define@key{names}{mbv}{Mbulungish}
\define@key{names}{mbu}{Mbula-Bwazza}
\define@key{names}{mlb}{Mbule}
\define@key{names}{mgy}{Mbunga}
\define@key{names}{mck}{Mbunda}
\define@key{names}{bbt}{Mburku}
\define@key{names}{muc}{Ajumbu}
\define@key{names}{mfu}{Mbwela}
\define@key{names}{gun}{Mbyá Guaraní}
\define@key{names}{mjm}{Medebur}
\define@key{names}{dmf}{Medefidrin}
\define@key{names}{mue}{Media Lengua}
\define@key{names}{mud}{Mednyj Aleut}
\define@key{names}{byv}{Medumba}
\define@key{names}{mfj}{Mefele}
\define@key{names}{mef}{Bangladesh Lyngam}
\define@key{names}{ruq}{Megleno Romanian}
\define@key{names}{mmh}{Mehináku}
\define@key{names}{mvk}{Mekmek}
\define@key{names}{msf}{Mekwei}
\define@key{names}{hkn}{Mel-Khaonh}
\define@key{names}{mfx}{Melo}
\define@key{names}{med}{Melpa}
\define@key{names}{mby}{Memoni}
\define@key{names}{mfd}{Mendankwe-Nkwen}
\define@key{names}{xkd}{Mendalam Kayan}
\define@key{names}{sim}{Mende (Papua New Guinea)}
\define@key{names}{xmg}{Mengaka}
\define@key{names}{mee}{Mengen}
\define@key{names}{mea}{Menka}
\define@key{names}{mvx}{Meoswar}
\define@key{names}{mxm}{Meramera}
\define@key{names}{lmb}{Merei}
\define@key{names}{meq}{Merey}
\define@key{names}{mrm}{Merlav}
\define@key{names}{xmr}{Meroitic}
\define@key{names}{mnu}{Mer}
\define@key{names}{mer}{Meru}
\define@key{names}{wry}{Merwari}
\define@key{names}{iyo}{Mesaka}
\define@key{names}{mci}{Mese}
\define@key{names}{zim}{Mesme}
\define@key{names}{mys}{Mesmes}
\define@key{names}{mvz}{Mesqan}
\define@key{names}{cms}{Messapic}
\define@key{names}{mgo}{Meta'}
\define@key{names}{mxv}{Metlatónoc Mixtec}
\define@key{names}{mtr}{Mewari}
\define@key{names}{wtm}{Mewati}
\define@key{names}{mfs}{Mexican Sign Language}
\define@key{names}{zmf}{Mfinu}
\define@key{names}{nfu}{Southern Mfumte}
\define@key{names}{zam}{Cuixtla-Xitla Zapotec}
\define@key{names}{pla}{Miani}
\define@key{names}{xmi}{Miarrã}
\define@key{names}{mmc}{Michoacán Mazahua}
\define@key{names}{enm}{Middle English}
\define@key{names}{gml}{Middle Low German}
\define@key{names}{dum}{Middle Dutch}
\define@key{names}{mpl}{Middle Watut}
\define@key{names}{gmh}{Middle High German}
\define@key{names}{ltc}{Middle Chinese}
\define@key{names}{xng}{Middle Mongol}
\define@key{names}{dnt}{Mid Grand Valley Dani}
\define@key{names}{bjo}{Mid-Southern Banda}
\define@key{names}{mpp}{Migabac}
\define@key{names}{ymh}{Mili}
\define@key{names}{mlj}{Miltu}
\define@key{names}{iml}{Miluk}
\define@key{names}{imy}{Milyan}
\define@key{names}{mcv}{Minanibai-Foia Foia}
\define@key{names}{inm}{Minaean}
\define@key{names}{mnp}{Min Bei Chinese}
\define@key{names}{mpn}{Mindiri}
\define@key{names}{drc}{Minderico}
\define@key{names}{mko}{Mingang Doso}
\define@key{names}{vmg}{Minigir}
\define@key{names}{wii}{Minidien}
\define@key{names}{xxm}{Minkin}
\define@key{names}{omn}{Minoan}
\define@key{names}{mqq}{Minokok}
\define@key{names}{mnq}{Minriq}
\define@key{names}{mzt}{Mintil}
\define@key{names}{czo}{Min Zhong Chinese}
\define@key{names}{zgm}{Minz Zhuang}
\define@key{names}{yiq}{Miqie}
\define@key{names}{mwl}{Mirandese}
\define@key{names}{mvh}{Mire}
\define@key{names}{mmv}{Miriti}
\define@key{names}{rsm}{Miriwoong Sign Language}
\define@key{names}{mjs}{Miship}
\define@key{names}{mpx}{Misima-Paneati}
\define@key{names}{vmm}{Mitlatongo Mixtec}
\define@key{names}{mwu}{Mittu}
\define@key{names}{mpo}{Miu}
\define@key{names}{vmi}{Miwa}
\define@key{names}{mfg}{Mixifore}
\define@key{names}{mix}{Mixtepec Mixtec}
\define@key{names}{mvi}{Miyako}
\define@key{names}{ehs}{Miyakubo Sign Language}
\define@key{names}{soy}{Miyobe}
\define@key{names}{lhs}{Mlahsô}
\define@key{names}{kja}{Mlap}
\define@key{names}{mlo}{Mlomp}
\define@key{names}{mmu}{Mmaala}
\define@key{names}{bfm}{Mmen}
\define@key{names}{mfq}{Moba}
\define@key{names}{mod}{Mobilian}
\define@key{names}{ahm}{Mobumrin Aizi}
\define@key{names}{jkm}{Mobwa Karen}
\define@key{names}{mhn}{Mòcheno}
\define@key{names}{mhc}{Mocho}
\define@key{names}{gbn}{Mo'da}
\define@key{names}{mxd}{Modang}
\define@key{names}{mqo}{Modole}
\define@key{names}{mvq}{Moere}
\define@key{names}{mou}{Mogum}
\define@key{names}{mof}{Mohegan-Montauk-Narragansett}
\define@key{names}{mow}{Moi (Congo)}
\define@key{names}{mxn}{Moi (Indonesia)}
\define@key{names}{mkp}{Moikodi}
\define@key{names}{mwz}{Moingi}
\define@key{names}{ymi}{Moji}
\define@key{names}{mft}{Mokerang}
\define@key{names}{mwt}{Moken}
\define@key{names}{mqt}{Mok}
\define@key{names}{mkm}{Moklen}
\define@key{names}{mkl}{Mokole}
\define@key{names}{vms}{Moksela}
\define@key{names}{pwm}{Molbog}
\define@key{names}{vsi}{Moldova Sign Language}
\define@key{names}{bxc}{Molengue}
\define@key{names}{mox}{Molima}
\define@key{names}{zmo}{Molo}
\define@key{names}{msl}{Molof}
\define@key{names}{mlw}{Moloko}
\define@key{names}{myl}{Moma}
\define@key{names}{msz}{Momare}
\define@key{names}{dmb}{Mombo Dogon}
\define@key{names}{mmb}{Momina}
\define@key{names}{ver}{Mom Jango}
\define@key{names}{mzg}{Monastic Sign Language}
\define@key{names}{npn}{Mondropolon}
\define@key{names}{msr}{Mongolian Sign Language}
\define@key{names}{mgt}{Mwakai}
\define@key{names}{mom}{Mangue}
\define@key{names}{moo}{Monom}
\define@key{names}{mru}{Mono (Cameroon)}
\define@key{names}{mnh}{Mono (Democratic Republic of Congo)}
\define@key{names}{nmh}{Monsang Naga}
\define@key{names}{mtl}{Montol}
\define@key{names}{gwg}{Moo}
\define@key{names}{crm}{Moose Cree}
\define@key{names}{msg}{Moraid}
\define@key{names}{mze}{Morawa}
\define@key{names}{moq}{Mor (Bomberai Peninsula)}
\define@key{names}{msx}{Moresada}
\define@key{names}{xmo}{Morerebi}
\define@key{names}{xmz}{Mori Bawah}
\define@key{names}{mzq}{Mori Atas}
\define@key{names}{mdb}{Morigi}
\define@key{names}{xms}{Moroccan Sign Language}
\define@key{names}{bdo}{Morom}
\define@key{names}{mgc}{Morokodo}
\define@key{names}{mrp}{Morouas}
\define@key{names}{mqn}{Moronene}
\define@key{names}{mrl}{Mortlockese}
\define@key{names}{mwy}{Akie}
\define@key{names}{mqv}{Mosimo}
\define@key{names}{mtj}{Moskona}
\define@key{names}{mtt}{Mota}
\define@key{names}{mwh}{Mouk-Aria}
\define@key{names}{jmw}{Mouwase}
\define@key{names}{ity}{Moyadan Itneg}
\define@key{names}{nmo}{Moyon}
\define@key{names}{mzy}{Mozambican Sign Language}
\define@key{names}{mxi}{Mozarabic}
\define@key{names}{xnq}{Mozambican Ngoni}
\define@key{names}{mpi}{Mpade}
\define@key{names}{mcx}{Mpiemo}
\define@key{names}{mpz}{Mpi}
\define@key{names}{pnd}{Mpinda}
\define@key{names}{mgg}{Mpongmpong}
\define@key{names}{mpa}{Mpoto}
\define@key{names}{mvt}{Mpotovoro}
\define@key{names}{zmp}{Mbuun}
\define@key{names}{cmr}{Mro Chin}
\define@key{names}{mro}{Mru}
\define@key{names}{kqx}{Mser}
\define@key{names}{agz}{Mt. Iriga Agta}
\define@key{names}{atl}{Mt. Iraya Agta}
\define@key{names}{mtd}{Mualang}
\define@key{names}{tsx}{Mubami}
\define@key{names}{mub}{Mubi}
\define@key{names}{ymd}{Muda}
\define@key{names}{gau}{Mudhili Gadaba}
\define@key{names}{udg}{Muduga}
\define@key{names}{vmd}{Mudu Koraga}
\define@key{names}{wiv}{Muduapa}
\define@key{names}{muk}{Mugom}
\define@key{names}{mmk}{Mukha-Dora}
\define@key{names}{mfw}{Mulaha}
\define@key{names}{kpb}{Mullu Kurumba}
\define@key{names}{vmu}{Muluridyi}
\define@key{names}{kqa}{Mum}
\define@key{names}{mwq}{Mün Chin}
\define@key{names}{boe}{Mundabli-Mufu}
\define@key{names}{mmf}{Mindat}
\define@key{names}{mth}{Munggui}
\define@key{names}{mpv}{Mungkip}
\define@key{names}{mtc}{Munit}
\define@key{names}{myr}{Muniche}
\define@key{names}{mnj}{Munji}
\define@key{names}{asx}{Muratayak}
\define@key{names}{mxr}{Murik (Malaysia)}
\define@key{names}{rmh}{Murkim}
\define@key{names}{tkv}{Mur Pano}
\define@key{names}{mqw}{Murupi}
\define@key{names}{smm}{Musasa}
\define@key{names}{mmi}{Hember Avu}
\define@key{names}{mmq}{Aisi}
\define@key{names}{mse}{Musey}
\define@key{names}{mui}{Musi}
\define@key{names}{mje}{Muskum}
\define@key{names}{muv}{Muthuvan}
\define@key{names}{tuc}{Mutu}
\define@key{names}{muy}{Muyang}
\define@key{names}{ymz}{Muzi}
\define@key{names}{mcj}{Mvano}
\define@key{names}{mxh}{Mvuba}
\define@key{names}{wlc}{Mwali Comorian}
\define@key{names}{wmw}{Mwani}
\define@key{names}{moa}{Mwan}
\define@key{names}{mwa}{Mwatebu}
\define@key{names}{mjh}{Mwera (Nyasa)}
\define@key{names}{mws}{Mwimbi-Muthambi}
\define@key{names}{gmy}{Mycenaean Greek}
\define@key{names}{nme}{Mzieme Naga}
\define@key{names}{nbt}{Na}
\define@key{names}{nao}{Naaba}
\define@key{names}{mne}{Naba}
\define@key{names}{mty}{Nabi-Metan}
\define@key{names}{ncd}{Nachering}
\define@key{names}{srf}{Nafi}
\define@key{names}{nxx}{Nafri}
\define@key{names}{jbn}{Nafusi}
\define@key{names}{nbg}{Nagarchal}
\define@key{names}{nxe}{Nage}
\define@key{names}{ngv}{Nagumi}
\define@key{names}{nlx}{Nahali-Baglani}
\define@key{names}{nhh}{Nahari}
\define@key{names}{ars}{Najdi Arabic}
\define@key{names}{nae}{Naka'ela}
\define@key{names}{nib}{Nakama}
\define@key{names}{nkj}{Nakai}
\define@key{names}{nbk}{Nake}
\define@key{names}{mff}{Naki}
\define@key{names}{nax}{Nakwi}
\define@key{names}{nlc}{Nalca}
\define@key{names}{nss}{Nali}
\define@key{names}{nlz}{Nalögo}
\define@key{names}{ylo}{Naluo Yi}
\define@key{names}{naj}{Nalu}
\define@key{names}{nmx}{Nama (Papua New Guinea)}
\define@key{names}{nkm}{Namat}
\define@key{names}{nmk}{Namakura}
\define@key{names}{nmq}{Nambya}
\define@key{names}{ncm}{Nambo}
\define@key{names}{neo}{Ná-Meo}
\define@key{names}{nbs}{Namibian Sign Language}
\define@key{names}{nvm}{Namiae}
\define@key{names}{naa}{Namla}
\define@key{names}{mxw}{Namo}
\define@key{names}{nmt}{Namonuito}
\define@key{names}{bwb}{Namosi-Naitasiri-Serua}
\define@key{names}{nmy}{Namuyi}
\define@key{names}{nnc}{Nancere}
\define@key{names}{nzz}{Nanga}
\define@key{names}{ngr}{Nanggu}
\define@key{names}{cox}{Nanti}
\define@key{names}{afk}{Nanubae-Imangae}
\define@key{names}{qvo}{Napo Lowland Quechua}
\define@key{names}{nrg}{Narango}
\define@key{names}{nac}{Narak}
\define@key{names}{loh}{Narim}
\define@key{names}{nnr}{Narungga}
\define@key{names}{nsy}{Nasal}
\define@key{names}{nvh}{Nasarian}
\define@key{names}{ntz}{Natanzic}
\define@key{names}{nte}{Nathembo}
\define@key{names}{nti}{Natioro}
\define@key{names}{nxa}{Nauete}
\define@key{names}{ncn}{Nauna}
\define@key{names}{nwo}{Nauo}
\define@key{names}{nsw}{Navut}
\define@key{names}{nwr}{Nawaru}
\define@key{names}{nwa}{Nawathinehena}
\define@key{names}{nmz}{Nawdm}
\define@key{names}{naw}{Nawuri}
\define@key{names}{nyq}{Nayinic}
\define@key{names}{noz}{Nayi}
\define@key{names}{ncr}{Ncane-Mungong}
\define@key{names}{nlu}{Nchumbulu}
\define@key{names}{gke}{Ndai}
\define@key{names}{ndk}{Ndaka}
\define@key{names}{ndh}{Ndali}
\define@key{names}{ndj}{Ndamba}
\define@key{names}{ndm}{Ndam}
\define@key{names}{nxo}{Ndambomo}
\define@key{names}{nnz}{Nda'nda'}
\define@key{names}{nda}{Ndasa}
\define@key{names}{ndc}{Ndau}
\define@key{names}{nml}{Ndemli}
\define@key{names}{ndg}{Ndengereko}
\define@key{names}{dne}{Ndendeule}
\define@key{names}{ndd}{Nde-Nsele-Nta}
\define@key{names}{eli}{Nding}
\define@key{names}{ndw}{Ndobo}
\define@key{names}{nbb}{Ndoe}
\define@key{names}{ndl}{Ndolo}
\define@key{names}{ndq}{Ndombe}
\define@key{names}{nqm}{Ndom}
\define@key{names}{ndr}{Ndoola}
\define@key{names}{ndp}{Ndo}
\define@key{names}{dno}{Ndrulo}
\define@key{names}{ndx}{Nduga}
\define@key{names}{nuh}{Ndunda}
\define@key{names}{nww}{Ndwewe}
\define@key{names}{njt}{Ndyuka-Trio Pidgin}
\define@key{names}{wni}{Ndzwani Comorian}
\define@key{names}{nec}{Klamu}
\define@key{names}{nef}{Nefamese}
\define@key{names}{dcr}{Negerhollands}
\define@key{names}{nkg}{Nekgini}
\define@key{names}{nif}{Nek}
\define@key{names}{nej}{Neko}
\define@key{names}{nek}{Neku}
\define@key{names}{nex}{Neme}
\define@key{names}{nem}{Nemi}
\define@key{names}{nqn}{Nen}
\define@key{names}{neu}{Neo (Artificial Language)}
\define@key{names}{nsp}{Nepalese Sign Language}
\define@key{names}{net}{Nete}
\define@key{names}{jas}{New Caledonian Javanese}
\define@key{names}{jui}{Ngadjuri}
\define@key{names}{nnf}{Ngaing}
\define@key{names}{hlt}{Nga La}
\define@key{names}{szb}{Ngalum}
\define@key{names}{nud}{Ngala}
\define@key{names}{nmv}{Ngamini-Yarluyandi-Karangura}
\define@key{names}{nbv}{Ngamambo}
\define@key{names}{nmc}{Ngam}
\define@key{names}{nbh}{Ngamo}
\define@key{names}{nyx}{Nganyaywana}
\define@key{names}{gng}{Ngangam}
\define@key{names}{nne}{Ngandyera}
\define@key{names}{nxd}{Ngando-Lalia}
\define@key{names}{ngd}{Ngando (Central African Republic)}
\define@key{names}{nji}{Ngarnka}
\define@key{names}{rxd}{Ngardi}
\define@key{names}{nsg}{Ngasa}
\define@key{names}{ngm}{Ngatik Men's Creole}
\define@key{names}{cnw}{Ngawn Chin}
\define@key{names}{zdj}{Ngazidja Comorian}
\define@key{names}{ngg}{Ngbaka Manza}
\define@key{names}{jgb}{Ngbee}
\define@key{names}{nbd}{Ngbinda-Mayeka}
\define@key{names}{nuu}{Ngbundu}
\define@key{names}{gnj}{Ngen of Djonkro}
\define@key{names}{nql}{Ngendelengo}
\define@key{names}{ngt}{Kriang-Khlor}
\define@key{names}{nnn}{Ngete}
\define@key{names}{nbq}{Nggem}
\define@key{names}{ngx}{Nggwahyi}
\define@key{names}{nnh}{Ngiemboon}
\define@key{names}{ngj}{Ngie}
\define@key{names}{nnq}{Ngindo}
\define@key{names}{nra}{Ngom}
\define@key{names}{nla}{Ngombale}
\define@key{names}{jgo}{Ngomba}
\define@key{names}{noq}{Ngongo}
\define@key{names}{nsh}{Ngoshie}
\define@key{names}{nuw}{Nguluwan}
\define@key{names}{ngp}{Ngulu}
\define@key{names}{nlo}{Ngwii}
\define@key{names}{xnm}{Ngumbarl}
\define@key{names}{nui}{Ngumbi}
\define@key{names}{nue}{Ngundu}
\define@key{names}{ndn}{Ngundi}
\define@key{names}{ngz}{Ngungwel}
\define@key{names}{nuo}{Nguôn}
\define@key{names}{nrx}{Ngurmbur}
\define@key{names}{nbx}{Wilson River (Grey Range)}
\define@key{names}{ngq}{Ngoreme}
\define@key{names}{ngw}{Ngwaba}
\define@key{names}{nwe}{Ngwe}
\define@key{names}{ngn}{Ngwo}
\define@key{names}{yrl}{Nhengatu}
\define@key{names}{nhf}{Nhuwala}
\define@key{names}{ncs}{Nicaraguan Sign Language}
\define@key{names}{nsi}{Nigerian Sign Language}
\define@key{names}{mzk}{Western Mambila}
\define@key{names}{nii}{Nii}
\define@key{names}{xny}{Nyiyaparli-Palyku}
\define@key{names}{gbe}{Niksek}
\define@key{names}{nim}{Nilamba}
\define@key{names}{nil}{Nila}
\define@key{names}{noe}{Nimadi}
\define@key{names}{nmp}{Nimanbur}
\define@key{names}{nmr}{Nimbari}
\define@key{names}{nis}{Nimi}
\define@key{names}{nmw}{Nimoa}
\define@key{names}{niw}{Nimo}
\define@key{names}{nxi}{Nindi}
\define@key{names}{nxr}{Ninggerum}
\define@key{names}{nby}{Ningera}
\define@key{names}{nlk}{Ninia Yali}
\define@key{names}{nin}{Ninzo}
\define@key{names}{nps}{Nipsan}
\define@key{names}{njs}{Nisa-Anasi}
\define@key{names}{yso}{Nisi (China)}
\define@key{names}{nkp}{Niuatoputapu}
\define@key{names}{njl}{Njalgulgule}
\define@key{names}{nzb}{Njebi}
\define@key{names}{njj}{Njen}
\define@key{names}{njr}{Njerep}
\define@key{names}{njy}{Njyem}
\define@key{names}{nkq}{Nkami}
\define@key{names}{nkn}{Nkangala}
\define@key{names}{nkz}{Nkari}
\define@key{names}{khu}{Nkhumbi}
\define@key{names}{nqo}{N'Ko}
\define@key{names}{nkc}{Nkongho}
\define@key{names}{nkx}{Nkoroo}
\define@key{names}{nka}{Nkoya}
\define@key{names}{nbo}{Nkukoli}
\define@key{names}{nkw}{Nkutu}
\define@key{names}{nbp}{Nnam}
\define@key{names}{ngh}{N||ng}
\define@key{names}{gaw}{Nobonob}
\define@key{names}{noi}{Noiri}
\define@key{names}{nkk}{Nokuku}
\define@key{names}{lem}{Nomaande}
\define@key{names}{nof}{Nomane}
\define@key{names}{noh}{Nomu}
\define@key{names}{zhn}{Nong Zhuang}
\define@key{names}{noj}{Nonuya}
\define@key{names}{nok}{Nooksack}
\define@key{names}{nrc}{Noric}
\define@key{names}{nrp}{North Picene}
\define@key{names}{huj}{Northern Guiyang Hmong}
\define@key{names}{hmp}{Northern Mashan Hmong}
\define@key{names}{crl}{Northern East Cree}
\define@key{names}{pbu}{Northern Pashto}
\define@key{names}{hno}{Northern Hindko}
\define@key{names}{glh}{Northwest Pashayi}
\define@key{names}{aee}{Northeast Pashayi}
\define@key{names}{kxm}{Northern Khmer}
\define@key{names}{atv}{Northern Altai}
\define@key{names}{azj}{North Azerbaijani}
\define@key{names}{ghh}{Northern Ghale}
\define@key{names}{ymx}{Northern Muji}
\define@key{names}{yiv}{Northern Nisu}
\define@key{names}{cng}{Northern Qiang}
\define@key{names}{bfc}{Northern Bai}
\define@key{names}{nnl}{Northern Rengma Naga}
\define@key{names}{lbr}{Lohorung}
\define@key{names}{tji}{Northern Tujia}
\define@key{names}{doc}{Northern Dong}
\define@key{names}{nod}{Northern Thai}
\define@key{names}{tts}{Northeastern Thai}
\define@key{names}{hea}{Northern Qiandong Miao}
\define@key{names}{hmi}{Northern Huishui Hmong}
\define@key{names}{kqs}{Northern Kissi}
\define@key{names}{fll}{North Fali}
\define@key{names}{dgi}{Northern Dagara}
\define@key{names}{tsp}{Northern Toussian}
\define@key{names}{gbo}{Northern Grebo}
\define@key{names}{dip}{Northeastern Dinka}
\define@key{names}{diw}{Northwestern Dinka}
\define@key{names}{max}{North Moluccan Malay}
\define@key{names}{mmg}{North Ambrym}
\define@key{names}{mrq}{North Marquesan}
\define@key{names}{tnn}{North Tanna}
\define@key{names}{una}{North Watut}
\define@key{names}{bcd}{North Babar}
\define@key{names}{weo}{Wemale}
\define@key{names}{nni}{North Nuaulu}
\define@key{names}{aqn}{Northern Alta}
\define@key{names}{xnn}{Northern Kankanay}
\define@key{names}{cts}{Northern Catanduanes Bicolano}
\define@key{names}{stb}{Northern Subanen}
\define@key{names}{bmm}{Northern Betsimisaraka Malagasy}
\define@key{names}{onr}{Northern One}
\define@key{names}{kti}{North Muyu}
\define@key{names}{nks}{Momogo-Pupis-Irogo}
\define@key{names}{yir}{North Awyu}
\define@key{names}{whg}{North Wahgi}
\define@key{names}{kiw}{Northeast Kiwai}
\define@key{names}{ryn}{Northern Amami-Oshima}
\define@key{names}{neq}{North Central Mixe}
\define@key{names}{scs}{North Slavey}
\define@key{names}{esk}{Seward Alaska Inupiatun}
\define@key{names}{thh}{Northern Tarahumara}
\define@key{names}{nhy}{Northern Oaxaca Nahuatl}
\define@key{names}{ojb}{Northwestern Ojibwa}
\define@key{names}{pef}{Northeastern Russian River Pomo}
\define@key{names}{cst}{San Francisco Bay Ohlone}
\define@key{names}{enl}{Enlhet Norte}
\define@key{names}{qvz}{Northern Pastaza Quichua}
\define@key{names}{qul}{North Bolivian Quechua}
\define@key{names}{qxn}{Northern Conchucos Ancash Quechua}
\define@key{names}{pmq}{Northern Pame}
\define@key{names}{xtn}{Northern Tlaxiaco Mixtec}
\define@key{names}{mxa}{Northwest Oaxaca Mixtec}
\define@key{names}{mfk}{North Mofu}
\define@key{names}{ayp}{North Mesopotamian Arabic}
\define@key{names}{ntd}{Northern Tidung}
\define@key{names}{cnp}{Northern Pinghua}
\define@key{names}{ncq}{Northern Katang}
\define@key{names}{bly}{Notre}
\define@key{names}{ncf}{Notsi}
\define@key{names}{ntw}{Nottoway}
\define@key{names}{nov}{Novial}
\define@key{names}{noy}{Noy}
\define@key{names}{asj}{Nsari}
\define@key{names}{nsc}{Nshi}
\define@key{names}{nsx}{Nsongo}
\define@key{names}{baf}{Nubaca}
\define@key{names}{kte}{Gyalsumdo-Nubri}
\define@key{names}{wbm}{Zhenkang Wa}
\define@key{names}{bsq}{Bassa}
\define@key{names}{wla}{Walio}
\define@key{names}{wgi}{Wahgi}
\define@key{names}{gyz}{Gyaazi}
\define@key{names}{nqt}{Nteng}
\define@key{names}{nnv}{Nugunu (Australia)}
\define@key{names}{noc}{Nuk}
\define@key{names}{klt}{Nukna}
\define@key{names}{nuq}{Nukumanu}
\define@key{names}{nur}{Nukuria}
\define@key{names}{nuc}{Nukuini}
\define@key{names}{nbr}{Numana}
\define@key{names}{nop}{Numanggang}
\define@key{names}{sij}{Numbami}
\define@key{names}{tgs}{Nume}
\define@key{names}{kdk}{Numee}
\define@key{names}{nxm}{Numidian}
\define@key{names}{nug}{Nungali}
\define@key{names}{rin}{Nungu}
\define@key{names}{nul}{Nusa Laut}
\define@key{names}{nwb}{Nyabwa}
\define@key{names}{nev}{Nyaheun}
\define@key{names}{nyy}{Nyakyusa-Ngonde}
\define@key{names}{nlj}{Nyali}
\define@key{names}{mwn}{Nyamwanga}
\define@key{names}{nwm}{Nyamusa-Molo}
\define@key{names}{nmi}{Nyam}
\define@key{names}{nny}{Yangkaal}
\define@key{names}{nyb}{Nyangbo}
\define@key{names}{nyc}{Nyanga-li}
\define@key{names}{nyk}{Nyaneka}
\define@key{names}{nnj}{Nyangatom}
\define@key{names}{sev}{Nyarafolo Senoufo}
\define@key{names}{nba}{Nyemba}
\define@key{names}{neh}{Upper Mangdep}
\define@key{names}{nye}{Nyengo}
\define@key{names}{nyl}{Nyeu}
\define@key{names}{nyr}{Nyiha (Malawi)}
\define@key{names}{nkv}{Nyika (Malawi and Zambia)}
\define@key{names}{nkt}{Nyika (Tanzania)}
\define@key{names}{nyg}{Nyindu}
\define@key{names}{lid}{Nyindrou}
\define@key{names}{nvo}{Nyokon}
\define@key{names}{nuj}{Nyole}
\define@key{names}{muo}{Nyong}
\define@key{names}{nyd}{Nyore}
\define@key{names}{nyu}{Nyungwe}
\define@key{names}{nzd}{Nzadi}
\define@key{names}{nzy}{Nzakambay}
\define@key{names}{nja}{Nzanyi}
\define@key{names}{nzi}{Nzima}
\define@key{names}{bzy}{Obanliku}
\define@key{names}{obi}{Obispeño}
\define@key{names}{obl}{Oblo}
\define@key{names}{obo}{Obo Manobo}
\define@key{names}{obu}{Obulom-Ochichi}
\define@key{names}{zac}{Ocotlán Zapotec}
\define@key{names}{odk}{Od}
\define@key{names}{bhf}{Odiai}
\define@key{names}{kkc}{Odoodee}
\define@key{names}{odu}{Odual}
\define@key{names}{tyh}{O'du}
\define@key{names}{opy}{Ofayé}
\define@key{names}{ofo}{Ofo}
\define@key{names}{ogc}{Ogbah}
\define@key{names}{ogg}{Ogbogolo}
\define@key{names}{eri}{Ogea}
\define@key{names}{oia}{Oirata}
\define@key{names}{chj}{Ojitlán Chinantec}
\define@key{names}{oki}{Okiek}
\define@key{names}{okn}{Oki-No-Erabu}
\define@key{names}{okb}{Okobo}
\define@key{names}{okd}{Okodia}
\define@key{names}{oks}{Oko-Eni-Osayen}
\define@key{names}{okj}{Okojuwoi}
\define@key{names}{kqv}{Okolod}
\define@key{names}{oie}{Okolie}
\define@key{names}{opa}{Okpamheri}
\define@key{names}{okx}{Okpe (Northwestern Edo)}
\define@key{names}{oke}{Okpe (Southwestern Edo)}
\define@key{names}{oar}{Old Aramaic-Sam'alian}
\define@key{names}{obr}{Old Burmese}
\define@key{names}{och}{Old Chinese}
\define@key{names}{odt}{Old Dutch-Old Frankish}
\define@key{names}{ang}{Old English (ca. 450-1100)}
\define@key{names}{fro}{Old French (842-ca. 1400)}
\define@key{names}{ofs}{Old Frisian}
\define@key{names}{oge}{Old Georgian}
\define@key{names}{goh}{Old High German (ca. 750-1050)}
\define@key{names}{sga}{Early Irish}
\define@key{names}{ojp}{Old Japanese}
\define@key{names}{okl}{Old Kentish Sign Language}
\define@key{names}{qok}{Old Khmer}
\define@key{names}{qkn}{Old Kannada}
\define@key{names}{qbb}{Old Latin}
\define@key{names}{omx}{Old Mon}
\define@key{names}{omr}{Old Marathi}
\define@key{names}{non}{Old Norse}
\define@key{names}{onw}{Old Nubian}
\define@key{names}{oos}{Old Ossetic}
\define@key{names}{pro}{Old Provençal}
\define@key{names}{peo}{Old Persian (ca. 600-400 B.C.)}
\define@key{names}{orv}{Old Russian}
\define@key{names}{osp}{Old Spanish}
\define@key{names}{osx}{Old Saxon}
\define@key{names}{oty}{Old Tamil}
\define@key{names}{oui}{Old Turkic}
\define@key{names}{owl}{Old-Middle Welsh}
\define@key{names}{ole}{Olekha}
\define@key{names}{olm}{Oloma}
\define@key{names}{lul}{Olu'bo}
\define@key{names}{iko}{Olulumo-Ikom}
\define@key{names}{acx}{Omani Arabic}
\define@key{names}{oml}{Ombo}
\define@key{names}{nht}{Ometepec Nahuatl}
\define@key{names}{omi}{Omi}
\define@key{names}{omt}{Omotik}
\define@key{names}{omu}{Omurano}
\define@key{names}{oog}{Ong-Ir}
\define@key{names}{onx}{Onin Pidgin}
\define@key{names}{oni}{Onin}
\define@key{names}{onj}{Onjob}
\define@key{names}{onn}{Onobasulu}
\define@key{names}{oor}{Oorlams}
\define@key{names}{opo}{Opao}
\define@key{names}{opt}{Teguima}
\define@key{names}{lgn}{Opo}
\define@key{names}{orn}{Orang Kanaq}
\define@key{names}{ors}{Orang Seletar}
\define@key{names}{sdr}{Oraon Sadri}
\define@key{names}{org}{Oring}
\define@key{names}{nlv}{Orizaba Nahuatl}
\define@key{names}{fnb}{Orkon-Fanbak}
\define@key{names}{orc}{Orma}
\define@key{names}{orz}{Ormu}
\define@key{names}{ora}{Oroha}
\define@key{names}{orx}{Oro}
\define@key{names}{orh}{Oroqen}
\define@key{names}{bpk}{Orowe}
\define@key{names}{orw}{Oro Win}
\define@key{names}{orr}{Oruma}
\define@key{names}{syx}{Osamayi}
\define@key{names}{ost}{Osatu}
\define@key{names}{osc}{Oscan}
\define@key{names}{osi}{Osing}
\define@key{names}{oso}{Ososo}
\define@key{names}{uta}{Otank}
\define@key{names}{otd}{Ot Danum}
\define@key{names}{oti}{Oti}
\define@key{names}{otw}{Ottawa}
\define@key{names}{lot}{Otuho}
\define@key{names}{otu}{Otuke}
\define@key{names}{oum}{Ouma}
\define@key{names}{oue}{Ounge}
\define@key{names}{stn}{Owa}
\define@key{names}{wsr}{Oweina}
\define@key{names}{oyy}{Oya'oya}
\define@key{names}{oyd}{Oyda}
\define@key{names}{zao}{Ozolotepec Zapotec}
\define@key{names}{chz}{Ozumacín Chinantec}
\define@key{names}{pfa}{Pááfang}
\define@key{names}{sig}{Paasaal}
\define@key{names}{qvp}{Pacaraos Quechua}
\define@key{names}{pcp}{Pacahuara}
\define@key{names}{pdi}{Pa Di}
\define@key{names}{pkc}{Paekche}
\define@key{names}{pae}{Pagibete}
\define@key{names}{pgi}{Pagi}
\define@key{names}{phr}{Pahari Potwari}
\define@key{names}{phj}{Pahari Newari}
\define@key{names}{lgt}{Pahi}
\define@key{names}{phv}{Pahlavani}
\define@key{names}{pal}{Pahlavi}
\define@key{names}{pha}{Pa-Hng}
\define@key{names}{pri}{Paicî}
\define@key{names}{ppi}{Paipai}
\define@key{names}{qpp}{Paisaci Prakrit}
\define@key{names}{pta}{Pai Tavytera}
\define@key{names}{pkg}{Pak-Tong}
\define@key{names}{jkp}{Paku Karen}
\define@key{names}{pku}{Paku}
\define@key{names}{pfl}{Pfaelzisch-Lothringisch}
\define@key{names}{plq}{Palaic}
\define@key{names}{plr}{Palaka Senoufo}
\define@key{names}{pln}{Palenquero}
\define@key{names}{pnl}{Palen}
\define@key{names}{pli}{Pali}
\define@key{names}{pcf}{Paliyan}
\define@key{names}{pmd}{Pallanganmiddang}
\define@key{names}{abw}{Pal}
\define@key{names}{pmc}{Palumata}
\define@key{names}{ple}{Palu'e}
\define@key{names}{plz}{Paluan}
\define@key{names}{bpx}{Palya Bareli}
\define@key{names}{pmb}{Pambia}
\define@key{names}{pmn}{Pam (Cameroon)}
\define@key{names}{hih}{Pamosu}
\define@key{names}{att}{Pamplona Atta}
\define@key{names}{pnz}{Pana (Central African Republic)}
\define@key{names}{pnq}{Pana (Burkina Faso)}
\define@key{names}{pwb}{Panawa}
\define@key{names}{psn}{Panasuan}
\define@key{names}{qxh}{Panao Huánuco Quechua}
\define@key{names}{lsp}{Panamanian Sign Language}
\define@key{names}{tdb}{Panchpargania}
\define@key{names}{pnp}{Pancana}
\define@key{names}{bkj}{Pande}
\define@key{names}{pgg}{Pangwali}
\define@key{names}{pgs}{Pangseng}
\define@key{names}{slm}{Pangutaran Sama}
\define@key{names}{pcg}{Paniya}
\define@key{names}{pnr}{Panim}
\define@key{names}{pax}{Pankararé}
\define@key{names}{pkh}{Pangkhua}
\define@key{names}{paz}{Pankararú}
\define@key{names}{pnc}{Pannei}
\define@key{names}{knt}{Panoan Katukína}
\define@key{names}{pno}{Panobo}
\define@key{names}{blk}{Pa'o Karen}
\define@key{names}{ppv}{Papavô}
\define@key{names}{ppn}{Papapana}
\define@key{names}{dpp}{Papar}
\define@key{names}{pas}{Papasena}
\define@key{names}{pbo}{Papel}
\define@key{names}{ppe}{Papi}
\define@key{names}{ppu}{Papora-Hoanya}
\define@key{names}{ppm}{Papuma}
\define@key{names}{pgz}{Papua New Guinean Sign Language}
\define@key{names}{prc}{Parachi}
\define@key{names}{pzn}{Jejara Naga}
\define@key{names}{prf}{Paranan}
\define@key{names}{prw}{Parawen}
\define@key{names}{aap}{Pará Arára}
\define@key{names}{pak}{Parakanã}
\define@key{names}{paf}{Paranawát}
\define@key{names}{gvp}{Pará-Maranhão Gavião}
\define@key{names}{pbg}{Paraujano}
\define@key{names}{pys}{Paraguayan Sign Language}
\define@key{names}{pcl}{Pardhi}
\define@key{names}{pch}{Pardhan}
\define@key{names}{pcj}{Gorum-Parenga}
\define@key{names}{ppt}{Pare}
\define@key{names}{kvx}{Parkari Koli}
\define@key{names}{xpr}{Parthian}
\define@key{names}{paq}{Parya}
\define@key{names}{psq}{Pasi}
\define@key{names}{yac}{Pass Valley Yali}
\define@key{names}{ptn}{Patani}
\define@key{names}{pth}{Pataxó Hã-Ha-Hãe}
\define@key{names}{pbc}{Patamona}
\define@key{names}{pty}{Pathiya}
\define@key{names}{ptq}{Pattapu}
\define@key{names}{mfa}{Kelantan-Pattani Malay}
\define@key{names}{pnk}{Paunaka}
\define@key{names}{bfb}{Pauri Bareli}
\define@key{names}{psm}{Warázu}
\define@key{names}{pmr}{Manat}
\define@key{names}{pcb}{Pear}
\define@key{names}{xpc}{Pecheneg}
\define@key{names}{pai}{Pye}
\define@key{names}{pfe}{Peere}
\define@key{names}{ppq}{Pei}
\define@key{names}{pel}{Pekal}
\define@key{names}{bxd}{Pela}
\define@key{names}{ata}{Pele-Ata}
\define@key{names}{pev}{Pémono}
\define@key{names}{psg}{Penang Sign Language}
\define@key{names}{pek}{Penchal}
\define@key{names}{ums}{Pendau}
\define@key{names}{pdc}{Pennsylvania German}
\define@key{names}{pnh}{Māngarongaro}
\define@key{names}{ptw}{Pentlatch}
\define@key{names}{pea}{Peranakan Indonesian}
\define@key{names}{wet}{Perai}
\define@key{names}{psc}{Zaban Eshareh Irani}
\define@key{names}{prl}{Peruvian Sign Language}
\define@key{names}{pex}{Petats}
\define@key{names}{zpe}{Petapa Zapotec}
\define@key{names}{pey}{Petjo}
\define@key{names}{prt}{Prai}
\define@key{names}{phk}{Phake}
\define@key{names}{phl}{Palula}
\define@key{names}{ypa}{Phala}
\define@key{names}{phq}{Phana'}
\define@key{names}{pem}{Phende}
\define@key{names}{psp}{Philippine Sign Language}
\define@key{names}{phm}{Phimbi}
\define@key{names}{phn}{Phoenician}
\define@key{names}{yip}{Pholo}
\define@key{names}{ypg}{Phola}
\define@key{names}{nph}{Phom Naga}
\define@key{names}{pnx}{Phong-Kniang}
\define@key{names}{kjt}{Phrae Pwo Karen}
\define@key{names}{xpg}{Phrygian}
\define@key{names}{phu}{Phuan}
\define@key{names}{phd}{Phudagi}
\define@key{names}{pug}{Phuie}
\define@key{names}{phh}{Phukha}
\define@key{names}{ypm}{Phuma}
\define@key{names}{pho}{Phunoi}
\define@key{names}{phg}{Phuong}
\define@key{names}{yph}{Phupha}
\define@key{names}{ypp}{Phupa}
\define@key{names}{pht}{Phu Thai}
\define@key{names}{ypz}{Phuza}
\define@key{names}{ptr}{Piamatsina}
\define@key{names}{pin}{Piame}
\define@key{names}{pcd}{Picard}
\define@key{names}{cpu}{Pichis Ashéninka}
\define@key{names}{xpi}{Pictish}
\define@key{names}{dep}{Pidgin Delaware}
\define@key{names}{pij}{Pijao}
\define@key{names}{piz}{Pije}
\define@key{names}{pis}{Pijin}
\define@key{names}{piw}{Pimbwe}
\define@key{names}{pnn}{Pinai-Hagahai}
\define@key{names}{pnv}{Pinigura}
\define@key{names}{tjp}{Lake Carnegie Western Desert}
\define@key{names}{pic}{Pinji}
\define@key{names}{pti}{Pintiini}
\define@key{names}{pny}{Pinyin}
\define@key{names}{bxi}{Pirlatapa}
\define@key{names}{pie}{Piro}
\define@key{names}{xpa}{Pirriya}
\define@key{names}{tpp}{Pisaflores Tepehua}
\define@key{names}{pig}{Pisabo}
\define@key{names}{psy}{Piscataway}
\define@key{names}{xps}{Pisidian}
\define@key{names}{pih}{Pitcairn-Norfolk}
\define@key{names}{sje}{Pite Saami}
\define@key{names}{pcn}{Piti}
\define@key{names}{pix}{Piu}
\define@key{names}{piy}{Piya-Kwonci}
\define@key{names}{ktj}{Plapo Krumen}
\define@key{names}{pdt}{Plautdietsch}
\define@key{names}{pbv}{Pnar}
\define@key{names}{npo}{Pochuri Naga}
\define@key{names}{pdn}{Podena}
\define@key{names}{pof}{Poke}
\define@key{names}{pkb}{Pokomo}
\define@key{names}{pld}{Polari}
\define@key{names}{plj}{Pesse}
\define@key{names}{pso}{Polish Sign Language}
\define@key{names}{plb}{Polonombauk}
\define@key{names}{pmo}{Pom}
\define@key{names}{pmm}{Pol}
\define@key{names}{ncc}{Ponam}
\define@key{names}{png}{Pongu}
\define@key{names}{pns}{Ponosakan}
\define@key{names}{pnt}{Pontic}
\define@key{names}{prh}{Porohanon}
\define@key{names}{ptv}{Daakie}
\define@key{names}{pmx}{Poumei Naga}
\define@key{names}{bye}{Pouye}
\define@key{names}{pwr}{Powari}
\define@key{names}{pyn}{Poyanáwa}
\define@key{names}{prz}{Providencia Sign Language}
\define@key{names}{prg}{Old Prussian}
\define@key{names}{kvj}{Psikye}
\define@key{names}{pux}{Puare}
\define@key{names}{atp}{Pudtol Atta}
\define@key{names}{pbm}{Puebla and Northeastern Mazatec}
\define@key{names}{psl}{Puerto Rican Sign Language}
\define@key{names}{pkp}{Pukapuka}
\define@key{names}{pup}{Pulabu}
\define@key{names}{pum}{Puma}
\define@key{names}{xpm}{Pumpokol}
\define@key{names}{puj}{Punan Tubu}
\define@key{names}{pud}{Punan Aput}
\define@key{names}{puf}{Punan Merah}
\define@key{names}{pna}{Punan Bah-Biau}
\define@key{names}{pnm}{Punan Batu 1}
\define@key{names}{xpu}{Punic}
\define@key{names}{qxp}{Puno Quechua}
\define@key{names}{puu}{Punu}
\define@key{names}{pru}{Puragi}
\define@key{names}{iar}{Purari}
\define@key{names}{puy}{Purisimeño}
\define@key{names}{prr}{Puri}
\define@key{names}{pur}{Puruborá}
\define@key{names}{pub}{Purum}
\define@key{names}{mfl}{Putai}
\define@key{names}{afe}{Utugwang-Irungene-Afrike}
\define@key{names}{cpx}{Pu-Xian Chinese}
\define@key{names}{pyu}{Puyuma}
\define@key{names}{pme}{Pwaamei}
\define@key{names}{pop}{Pwapwa}
\define@key{names}{pwo}{Pwo Western Karen}
\define@key{names}{pcw}{Pyapun}
\define@key{names}{pye}{Pye Krumen}
\define@key{names}{pyy}{Pyen}
\define@key{names}{pby}{Pyu}
\define@key{names}{laq}{Pubiao-Qabiao}
\define@key{names}{qxq}{Qashqa'i}
\define@key{names}{xqt}{Qatabanian}
\define@key{names}{ymq}{Qila Muji}
\define@key{names}{zqe}{Qiubei Zhuang}
\define@key{names}{qua}{Quapaw}
\define@key{names}{qya}{Quenya}
\define@key{names}{qvy}{Queyu}
\define@key{names}{zpj}{Quiavicuzas Zapotec}
\define@key{names}{quq}{Quinqui}
\define@key{names}{qun}{Quinault}
\define@key{names}{ztq}{Quioquitani-Quieri Zapotec}
\define@key{names}{rah}{Rabha}
\define@key{names}{xrr}{Raetic}
\define@key{names}{raz}{Rahambuu}
\define@key{names}{mqk}{Rajah Kabunsuwan Manobo}
\define@key{names}{rjs}{Rajbanshi}
\define@key{names}{rjg}{Rajong}
\define@key{names}{gra}{Rajput Garasia}
\define@key{names}{rkh}{Rakahanga-Manihiki}
\define@key{names}{rki}{Rakhine}
\define@key{names}{rai}{Ramoaaina}
\define@key{names}{kjx}{Ramopa}
\define@key{names}{lje}{Rampi}
\define@key{names}{thr}{Rana Tharu}
\define@key{names}{rkt}{Central-Eastern Kamta}
\define@key{names}{rnl}{Halam}
\define@key{names}{rax}{Rang}
\define@key{names}{ray}{Mangaia-Old Rapa}
\define@key{names}{rpt}{Rapting}
\define@key{names}{lra}{Rara Bakati'}
\define@key{names}{rar}{Southern Cook Island Maori}
\define@key{names}{rac}{Rasawa}
\define@key{names}{btn}{Ratagnon}
\define@key{names}{bgd}{Rathwi Bareli}
\define@key{names}{rtw}{Rathawi}
\define@key{names}{rau}{Raute}
\define@key{names}{yea}{Ravula}
\define@key{names}{jnl}{Rawat}
\define@key{names}{rat}{Razajerdi}
\define@key{names}{gir}{Red Gelao}
\define@key{names}{atu}{Reel}
\define@key{names}{ree}{Rejang Kayan}
\define@key{names}{rei}{Reli}
\define@key{names}{bow}{Rema}
\define@key{names}{reb}{Rembong-Wangka}
\define@key{names}{agv}{Hatang Kayi}
\define@key{names}{rem}{Remo of the Moa river}
\define@key{names}{rmp}{Rempi}
\define@key{names}{lkj}{Remun}
\define@key{names}{rsi}{Rennellese Sign Language}
\define@key{names}{rea}{Rerau}
\define@key{names}{rer}{Rer Bare}
\define@key{names}{pgk}{Rerep}
\define@key{names}{res}{Reshe}
\define@key{names}{ret}{Reta}
\define@key{names}{rcf}{Réunion Creole French}
\define@key{names}{rey}{Reyesano}
\define@key{names}{ril}{Riang (Myanmar)}
\define@key{names}{ria}{Riang (India)}
\define@key{names}{rir}{Ribun}
\define@key{names}{zar}{Rincón Zapotec}
\define@key{names}{rgu}{Ringgou}
\define@key{names}{hrx}{Hunsrik}
\define@key{names}{rri}{Ririo}
\define@key{names}{riu}{Riung}
\define@key{names}{snj}{Riverain Sango}
\define@key{names}{rod}{Rogo}
\define@key{names}{rhg}{Rohingya}
\define@key{names}{rge}{Romano-Greek}
\define@key{names}{rms}{Romanian Sign Language}
\define@key{names}{rgn}{Romagnol}
\define@key{names}{rmx}{Romam}
\define@key{names}{rmm}{Roma}
\define@key{names}{rmv}{Romanova}
\define@key{names}{rof}{Rombo}
\define@key{names}{rol}{Romblomanon}
\define@key{names}{rmk}{Romkun}
\define@key{names}{ror}{Rongga}
\define@key{names}{roe}{Ronji}
\define@key{names}{rnn}{Roon}
\define@key{names}{rga}{Mores}
\define@key{names}{pce}{Ruching Palaung}
\define@key{names}{rdb}{Rudbari}
\define@key{names}{ruh}{Ruga}
\define@key{names}{rbb}{Rumai Palaung}
\define@key{names}{ruz}{Ruma}
\define@key{names}{rna}{Runa}
\define@key{names}{rnw}{Rungwa}
\define@key{names}{drg}{Rungus}
\define@key{names}{bxr}{Russia Buriat}
\define@key{names}{rue}{Rusyn}
\define@key{names}{ruc}{Ruuli}
\define@key{names}{rnd}{Ruund}
\define@key{names}{rwk}{Rwa}
\define@key{names}{rsn}{Rwandan Sign Language}
\define@key{names}{sax}{Sa}
\define@key{names}{sav}{Saafi-Saafi}
\define@key{names}{raq}{Saam}
\define@key{names}{lsm}{Saamia}
\define@key{names}{sxr}{Saaroa}
\define@key{names}{spy}{Sabaot}
\define@key{names}{msi}{Sabah Malay}
\define@key{names}{bsy}{Sabah Bisaya}
\define@key{names}{sae}{Sabanê}
\define@key{names}{saa}{Saba}
\define@key{names}{xsa}{Sabaic}
\define@key{names}{qhr}{Old Sabellic}
\define@key{names}{sbo}{Sabüm}
\define@key{names}{quv}{Sacapulteco}
\define@key{names}{sck}{Sadri}
\define@key{names}{spd}{Saep}
\define@key{names}{saf}{Safaliba}
\define@key{names}{sbk}{Safwa}
\define@key{names}{sbm}{Sagala}
\define@key{names}{tga}{Sagalla}
\define@key{names}{aec}{Saidi Arabic}
\define@key{names}{acf}{Saint Lucian Creole French}
\define@key{names}{xsy}{Saisiyat}
\define@key{names}{sjl}{Sajolang}
\define@key{names}{sjb}{Sajau-Latti}
\define@key{names}{sch}{Sakachep-Chorei}
\define@key{names}{skt}{Sakata}
\define@key{names}{skg}{West Malagasy Sakalava}
\define@key{names}{skm}{Sakam}
\define@key{names}{sak}{Sake}
\define@key{names}{szy}{Sakizaya}
\define@key{names}{shq}{Sala}
\define@key{names}{slx}{Salampasu}
\define@key{names}{sgu}{Salas}
\define@key{names}{qxl}{Tungurahua Highland Quichua}
\define@key{names}{mnd}{Salamãi}
\define@key{names}{slq}{Salchuq}
\define@key{names}{sau}{Saleman}
\define@key{names}{loe}{Saluan}
\define@key{names}{esn}{Salvadoran Sign Language}
\define@key{names}{tmj}{Samarokena}
\define@key{names}{ysd}{Samatao}
\define@key{names}{smp}{Samaritan}
\define@key{names}{xab}{Sambe}
\define@key{names}{smx}{Samba}
\define@key{names}{ccg}{Samba Daka}
\define@key{names}{saq}{Samburu}
\define@key{names}{ssx}{Samberigi}
\define@key{names}{spv}{Sambalpuri}
\define@key{names}{smh}{Samei}
\define@key{names}{snx}{Sam}
\define@key{names}{swm}{Samosa}
\define@key{names}{rav}{Sampang}
\define@key{names}{stu}{Samtao}
\define@key{names}{smv}{Samvedi}
\define@key{names}{ztm}{San Agustín Mixtepec Zapotec}
\define@key{names}{icr}{San Andres Creole English}
\define@key{names}{spn}{Sanapaná}
\define@key{names}{zpx}{San Baltazar Loxicha Zapotec}
\define@key{names}{cuk}{San Blas Kuna}
\define@key{names}{hve}{San Dionisio del Mar Huave}
\define@key{names}{hue}{San Francisco del Mar Huave}
\define@key{names}{mat}{San Francisco Matlatzinca}
\define@key{names}{pow}{San Felipe Otlaltepec Popoloca}
\define@key{names}{xso}{San Francisco Solano}
\define@key{names}{sgr}{Sangisari}
\define@key{names}{sgk}{Sangkong}
\define@key{names}{nsa}{Sangtam Naga}
\define@key{names}{xsn}{Sanga (Nigeria)}
\define@key{names}{sbp}{Sangu (Tanzania)}
\define@key{names}{sng}{Sanga (Democratic Republic of Congo)}
\define@key{names}{snl}{Sangil}
\define@key{names}{scg}{Sanggau}
\define@key{names}{sgy}{Sanglechi}
\define@key{names}{ysy}{Sanie}
\define@key{names}{ysn}{Sani}
\define@key{names}{sny}{Saniyo-Hiyewe}
\define@key{names}{xtj}{San Juan Teita Mixtec}
\define@key{names}{maa}{San Jerónimo Tecóatl Mazatec}
\define@key{names}{msc}{Sankaran Maninka}
\define@key{names}{pps}{San Luís Temalacayuca Popoloca}
\define@key{names}{qvs}{San Martín Quechua}
\define@key{names}{xtp}{San Miguel Piedras Mixtec}
\define@key{names}{trq}{San Martín Itunyoso Triqui}
\define@key{names}{pls}{San Marcos Tlalcoyalco Popoloca}
\define@key{names}{azg}{San Pedro Amuzgos Amuzgo}
\define@key{names}{zpf}{San Pedro Quiatoni Zapotec}
\define@key{names}{san}{Sanskrit}
\define@key{names}{ssi}{Sansi}
\define@key{names}{kwy}{San Salvador Kongo}
\define@key{names}{hvv}{Santa María del Mar Huave}
\define@key{names}{nhz}{Santa María La Alta Nahuatl}
\define@key{names}{cok}{Santa Teresa Cora}
\define@key{names}{qus}{Santiago del Estero Quichua}
\define@key{names}{mza}{Santa María Zacatepec Mixtec}
\define@key{names}{mdv}{Santa Lucía Monteverde Mixtec}
\define@key{names}{zpn}{Santa Inés Yatzechi Zapotec}
\define@key{names}{ztn}{Santa Catarina Albarradas Zapotec}
\define@key{names}{zas}{Santo Domingo Albarradas Zapotec}
\define@key{names}{zpr}{Santiago Xanica Zapotec}
\define@key{names}{pca}{Santa Inés Ahuatempan Popoloca}
\define@key{names}{zpt}{San Vicente Coatlán Zapotec}
\define@key{names}{scq}{Sa'och}
\define@key{names}{zkp}{São Paulo Kaingáng}
\define@key{names}{cri}{Sãotomense}
\define@key{names}{spr}{Saparua}
\define@key{names}{spc}{Sapé}
\define@key{names}{krn}{Sapo}
\define@key{names}{spi}{Saponi}
\define@key{names}{sbz}{Sara Kaba}
\define@key{names}{kwv}{Sara Kaba Náà}
\define@key{names}{kwg}{Sara Kaba Deme}
\define@key{names}{zsa}{Sarasira}
\define@key{names}{bps}{Sarangani Blaan}
\define@key{names}{mbs}{Sarangani Manobo}
\define@key{names}{sre}{Sara Bakati'}
\define@key{names}{sar}{Saraveca}
\define@key{names}{srh}{Sarikoli}
\define@key{names}{mwm}{Sar}
\define@key{names}{onp}{Sartang}
\define@key{names}{sdu}{Sarudu}
\define@key{names}{sra}{Saruga}
\define@key{names}{swy}{Sarua}
\define@key{names}{sxs}{Sasaru}
\define@key{names}{sas}{Sasak}
\define@key{names}{sdc}{Sassarese Sardinian}
\define@key{names}{stw}{Satawalese}
\define@key{names}{stq}{Ems-Weser Frisian}
\define@key{names}{mav}{Sateré-Mawé}
\define@key{names}{sdl}{Saudi Arabian Sign Language}
\define@key{names}{skc}{Ma Manda}
\define@key{names}{saz}{Saurashtra}
\define@key{names}{mjt}{Sauria Paharia}
\define@key{names}{srt}{Sauri}
\define@key{names}{psu}{Sauraseni Prakrit}
\define@key{names}{ssj}{Sausi}
\define@key{names}{sao}{Sause}
\define@key{names}{swr}{Saweru}
\define@key{names}{swt}{Sawila}
\define@key{names}{saw}{Sawi}
\define@key{names}{swn}{Sawknah-Fogaha}
\define@key{names}{sxw}{Saxwe Gbe}
\define@key{names}{say}{Saya}
\define@key{names}{sco}{Scots}
\define@key{names}{kdg}{Seba}
\define@key{names}{sbx}{Seberuang}
\define@key{names}{sib}{Sebop}
\define@key{names}{sec}{Sechelt}
\define@key{names}{tvw}{Sedoa}
\define@key{names}{sos}{Seeku}
\define@key{names}{sge}{Segai}
\define@key{names}{sbg}{Seget}
\define@key{names}{seg}{Segeju}
\define@key{names}{sfw}{Sehwi}
\define@key{names}{ssg}{Seimat}
\define@key{names}{hik}{Seit-Kaitetu}
\define@key{names}{skz}{Sekar}
\define@key{names}{skp}{Sekapan}
\define@key{names}{sek}{Sekani}
\define@key{names}{ske}{Seke (Vanuatu)}
\define@key{names}{syi}{Seki}
\define@key{names}{sko}{Seko Tengah}
\define@key{names}{skx}{Seko Padang}
\define@key{names}{lip}{Sekpele}
\define@key{names}{kgi}{Selangor Sign Language}
\define@key{names}{snw}{Selee}
\define@key{names}{sws}{Seluwasan}
\define@key{names}{slg}{Selungai Murut}
\define@key{names}{szc}{Semaq Beri}
\define@key{names}{sbr}{Sembakung Murut}
\define@key{names}{etz}{Semimi}
\define@key{names}{smy}{Semnani-Biyabuneki}
\define@key{names}{ssm}{Semnam}
\define@key{names}{xse}{Sempan}
\define@key{names}{seq}{Senar de Kankalaba}
\define@key{names}{sej}{Sene}
\define@key{names}{sds}{Sened}
\define@key{names}{ssz}{Sengseng}
\define@key{names}{spk}{Sengo}
\define@key{names}{snu}{Senggi}
\define@key{names}{sjs}{Senhaja De Srair}
\define@key{names}{sni}{Sensi}
\define@key{names}{std}{Sentinel}
\define@key{names}{sez}{Senthang Chin}
\define@key{names}{spe}{Sepa (Papua New Guinea)}
\define@key{names}{spb}{Sepa (Indonesia)}
\define@key{names}{spm}{Sepen}
\define@key{names}{iws}{Sepik Iwam}
\define@key{names}{skr}{Saraiki}
\define@key{names}{sry}{Sera}
\define@key{names}{srr}{Sereer}
\define@key{names}{swf}{Sere}
\define@key{names}{sve}{Serili}
\define@key{names}{seu}{Serui-Laut}
\define@key{names}{srw}{Serua}
\define@key{names}{srk}{Serudung Murut}
\define@key{names}{stf}{Seta}
\define@key{names}{stm}{Setaman}
\define@key{names}{sbi}{Seti}
\define@key{names}{sta}{KiSetla}
\define@key{names}{sew}{Sewa Bay}
\define@key{names}{lsw}{Seychelles Sign Language}
\define@key{names}{sze}{Seze}
\define@key{names}{scw}{Sya}
\define@key{names}{sdb}{Shabaki}
\define@key{names}{srz}{Shahmirzadi}
\define@key{names}{sha}{Shall-Zwall}
\define@key{names}{xsh}{Shamang}
\define@key{names}{sqa}{Shama-Sambuga}
\define@key{names}{jih}{Stodsde}
\define@key{names}{sho}{Shanga}
\define@key{names}{swo}{Shanenawa}
\define@key{names}{ssv}{Ngen}
\define@key{names}{swq}{Sharwa}
\define@key{names}{sqh}{Shau}
\define@key{names}{shx}{She}
\define@key{names}{she}{Sheko}
\define@key{names}{sth}{Shelta}
\define@key{names}{shl}{Shendu}
\define@key{names}{scv}{Sheni-Ziriya}
\define@key{names}{bun}{Sherbro}
\define@key{names}{kip}{Sheshi Kham}
\define@key{names}{ssh}{Shihhi Arabic}
\define@key{names}{shr}{Shi}
\define@key{names}{gua}{Shiki}
\define@key{names}{snh}{Shinabo}
\define@key{names}{sxg}{Shixing}
\define@key{names}{sle}{Sholaga}
\define@key{names}{bcv}{Shoo-Minda-Nye}
\define@key{names}{suj}{Shubi}
\define@key{names}{sts}{Shumashti}
\define@key{names}{scu}{Shumcho}
\define@key{names}{ksa}{Shuwa-Zamani}
\define@key{names}{shw}{Shwai}
\define@key{names}{slw}{Sialum}
\define@key{names}{sya}{Siang}
\define@key{names}{spg}{Sihan}
\define@key{names}{mmp}{Siawi}
\define@key{names}{nco}{Sibe (Nasioi)}
\define@key{names}{sty}{Siberian Tatar}
\define@key{names}{sdx}{Sibu Melanau}
\define@key{names}{sxc}{Sicana}
\define@key{names}{scn}{Sicilian}
\define@key{names}{sep}{Sìcìté Sénoufo}
\define@key{names}{scx}{Sicula}
\define@key{names}{xsd}{Sidetic}
\define@key{names}{sgx}{Sierra Leone Sign Language}
\define@key{names}{nsu}{Sierra Negra Nahuatl}
\define@key{names}{sxe}{Sighu}
\define@key{names}{snr}{Sihan (Gum)}
\define@key{names}{qws}{Sihuas Ancash Quechua}
\define@key{names}{sky}{Sikaiana}
\define@key{names}{slt}{Sila}
\define@key{names}{szl}{Silesian}
\define@key{names}{sbq}{Sirva}
\define@key{names}{mkc}{Siliput}
\define@key{names}{wul}{Silimo}
\define@key{names}{xsp}{Silopi}
\define@key{names}{stv}{Silt'e}
\define@key{names}{sie}{Simaa}
\define@key{names}{sbw}{Simba}
\define@key{names}{smb}{Simbari}
\define@key{names}{sbb}{Simbo}
\define@key{names}{smg}{Simbali}
\define@key{names}{smz}{Simeku}
\define@key{names}{smt}{Simte}
\define@key{names}{siu}{Galu}
\define@key{names}{sbn}{Sindhi Bhil}
\define@key{names}{xts}{Sindihui Mixtec}
\define@key{names}{sjn}{Sindarin}
\define@key{names}{sgp}{Northern Jinghpaw}
\define@key{names}{sgm}{Singa}
\define@key{names}{skq}{Sininkere}
\define@key{names}{xti}{Sinicahua Mixtec}
\define@key{names}{snz}{Kou}
\define@key{names}{sys}{Sinyar}
\define@key{names}{swj}{Sira}
\define@key{names}{sir}{Siri}
\define@key{names}{srx}{Sirmauri}
\define@key{names}{sld}{Sissala of Burkina Faso}
\define@key{names}{sso}{Sissano}
\define@key{names}{siy}{Sivandi}
\define@key{names}{lsv}{Sivia Sign Language}
\define@key{names}{akp}{Siwu}
\define@key{names}{skw}{Skepi Creole Dutch}
\define@key{names}{sms}{Skolt Saami}
\define@key{names}{svm}{Slavomolisano}
\define@key{names}{svk}{Slovakian Sign Language}
\define@key{names}{sfm}{Gha-mu}
\define@key{names}{kxq}{Smärky Kanum}
\define@key{names}{sox}{So (Cameroon)}
\define@key{names}{soc}{So (Democratic Republic of Congo)}
\define@key{names}{xog}{Soga}
\define@key{names}{sog}{Sogdian}
\define@key{names}{soj}{Soic}
\define@key{names}{sok}{Sokoro}
\define@key{names}{sby}{Soli}
\define@key{names}{sol}{Solos}
\define@key{names}{aaw}{Solong}
\define@key{names}{szs}{Solomon Islands Sign Language}
\define@key{names}{smc}{Som}
\define@key{names}{smu}{Somray of Battambang-Somre of Siem Reap}
\define@key{names}{sor}{Somrai}
\define@key{names}{kgt}{Somyev}
\define@key{names}{ysg}{Sonaga}
\define@key{names}{shc}{Sonde}
\define@key{names}{soo}{Nsong-Mpiin}
\define@key{names}{sod}{Songoora}
\define@key{names}{soe}{Ohendo}
\define@key{names}{soi}{Sonha}
\define@key{names}{siq}{Sonia}
\define@key{names}{sss}{Sô}
\define@key{names}{urw}{Sop}
\define@key{names}{sbh}{Sori-Harengan}
\define@key{names}{sqo}{Sorkhei-Aftari}
\define@key{names}{ays}{Sorsogon Ayta}
\define@key{names}{sdk}{Sos Kundi}
\define@key{names}{krz}{Sota Kanum}
\define@key{names}{sfs}{South African Sign Language}
\define@key{names}{nit}{Southeastern Kolami}
\define@key{names}{hmy}{Southern Guiyang Hmong}
\define@key{names}{hma}{Southern Mashan Hmong}
\define@key{names}{sdh}{Southern Kurdish}
\define@key{names}{bcc}{Southern Balochi}
\define@key{names}{fay}{Fars Dialects}
\define@key{names}{luz}{Southern Luri}
\define@key{names}{pbt}{Southern Pashto}
\define@key{names}{hnd}{Southern Hindko}
\define@key{names}{psh}{Southwest Pashayi}
\define@key{names}{psi}{Southeast Pashayi}
\define@key{names}{vro}{South Estonian}
\define@key{names}{nik}{Southern Nicobarese}
\define@key{names}{mnn}{Southern Mnong}
\define@key{names}{uzs}{Southern Uzbek}
\define@key{names}{ghe}{Southern Ghale}
\define@key{names}{ymc}{Southern Muji}
\define@key{names}{nsd}{Southern Nisu}
\define@key{names}{qxs}{Southern Qiang}
\define@key{names}{pmj}{Southern Pumi}
\define@key{names}{bfs}{Southern Bai}
\define@key{names}{nre}{Southern Rengma Naga}
\define@key{names}{lrr}{Southern Yamphu}
\define@key{names}{tjs}{Southern Tujia}
\define@key{names}{sou}{Southern Thai}
\define@key{names}{hms}{Southern Qiandong Miao}
\define@key{names}{hmh}{Southwestern Huishui Hmong}
\define@key{names}{hmg}{Southwestern Guiyang Hmong}
\define@key{names}{xtv}{Southern Coastal Yuin}
\define@key{names}{ijs}{Southeast Ijo}
\define@key{names}{fal}{South Fali}
\define@key{names}{nbw}{Southern Ngbandi}
\define@key{names}{lnl}{South Central Banda}
\define@key{names}{biv}{Southern Birifor}
\define@key{names}{nnw}{Southern Nuni}
\define@key{names}{snm}{Southern Ma'di}
\define@key{names}{dik}{Southwestern Dinka}
\define@key{names}{dib}{South Central Dinka}
\define@key{names}{dks}{Southeastern Dinka}
\define@key{names}{bwq}{Southern Bobo Madaré}
\define@key{names}{sbd}{Southern Samo}
\define@key{names}{sns}{Nahavaq}
\define@key{names}{mqm}{South Marquesan}
\define@key{names}{mcy}{South Watut}
\define@key{names}{vbb}{Southeast Babar}
\define@key{names}{lmf}{Eastern Atadei}
\define@key{names}{agy}{Southern Alta}
\define@key{names}{ksc}{Bangad}
\define@key{names}{bln}{Coastal-Virac Bikol}
\define@key{names}{plv}{Southwest Palawano}
\define@key{names}{bzc}{Southern Betsimisaraka Malagasy}
\define@key{names}{osu}{Southern One}
\define@key{names}{aws}{South Awyu}
\define@key{names}{omw}{South Tairora}
\define@key{names}{ams}{Southern Amami-Oshima}
\define@key{names}{hax}{Southern Haida}
\define@key{names}{tce}{Southern Tutchone}
\define@key{names}{caf}{Southern Carrier}
\define@key{names}{twr}{Southwestern Tarahumara}
\define@key{names}{tcu}{Southeastern Tarahumara}
\define@key{names}{npl}{Nahuatl, Southeastern Puebla}
\define@key{names}{tla}{Southwestern Tepehuan}
\define@key{names}{crj}{Southern East Cree}
\define@key{names}{peq}{Southern Pomo}
\define@key{names}{qup}{Southern Pastaza Quechua}
\define@key{names}{qxo}{Southern Conchucos Ancash Quechua}
\define@key{names}{ayc}{Southern Aymara}
\define@key{names}{meh}{Southwestern Tlaxiaco Mixtec}
\define@key{names}{mit}{Southern Puebla Mixtec}
\define@key{names}{mxy}{Southeastern Nochixtlán Mixtec}
\define@key{names}{rgs}{Southern Roglai}
\define@key{names}{giz}{South Giziga}
\define@key{names}{cpy}{South Ucayali Ashéninka}
\define@key{names}{itd}{Southern Tidung}
\define@key{names}{csp}{Southern Pinghua}
\define@key{names}{sct}{Southern Katang}
\define@key{names}{sqq}{Sou}
\define@key{names}{sww}{Sowa}
\define@key{names}{sow}{Sowanda}
\define@key{names}{vmq}{Soyaltepec Mixtec}
\define@key{names}{vmp}{Soyaltepec Mazatec}
\define@key{names}{sqs}{Sri Lankan Sign Language}
\define@key{names}{sci}{Sri Lanka Malay}
\define@key{names}{seo}{Asabano}
\define@key{names}{swp}{Suau}
\define@key{names}{sxb}{Suba}
\define@key{names}{ssc}{Suba-Simbiti}
\define@key{names}{sut}{Subtiaba}
\define@key{names}{apd}{Sudanese Arabic}
\define@key{names}{pga}{South Sudanese Creole Arabic}
\define@key{names}{sgi}{Nizaa}
\define@key{names}{sug}{Suganga}
\define@key{names}{kzs}{Sugut Dusun}
\define@key{names}{zsu}{Sukurum}
\define@key{names}{syk}{Sukur}
\define@key{names}{szn}{Sula}
\define@key{names}{srg}{Sulod}
\define@key{names}{sqm}{Suma}
\define@key{names}{siv}{Sumariup}
\define@key{names}{six}{Sumau}
\define@key{names}{suw}{Sumbwa}
\define@key{names}{smw}{Sumbawa}
\define@key{names}{sux}{Sumerian}
\define@key{names}{csv}{Sumtu Chin}
\define@key{names}{ssk}{Sunam}
\define@key{names}{suz}{Sunwar}
\define@key{names}{syo}{Suoy}
\define@key{names}{sbj}{Surbakhal}
\define@key{names}{sgd}{Surigaonon}
\define@key{names}{sjp}{Surjapuri}
\define@key{names}{tdl}{Sur}
\define@key{names}{sde}{Vori}
\define@key{names}{mdz}{Suruí Do Pará}
\define@key{names}{sru}{Suruí}
\define@key{names}{swx}{Suruahá}
\define@key{names}{sqn}{Susquehannock}
\define@key{names}{ssu}{Susuami}
\define@key{names}{sdj}{Suundi}
\define@key{names}{swu}{Suwawa}
\define@key{names}{suy}{Suyá}
\define@key{names}{swg}{Swabian}
\define@key{names}{slf}{Swiss-Italian Sign Language}
\define@key{names}{sgg}{Swiss-German Sign Language}
\define@key{names}{ssr}{Swiss-French Sign Language}
\define@key{names}{xdk}{Sydney}
\define@key{names}{syl}{Sylheti}
\define@key{names}{zoq}{Tabasco Zoque}
\define@key{names}{nhc}{Tabasco Nahuatl}
\define@key{names}{zat}{Tabaa Zapotec}
\define@key{names}{knv}{Tabo}
\define@key{names}{tzx}{Tabriak}
\define@key{names}{xtt}{Tacahua-Yolotepec Mixtec}
\define@key{names}{lts}{Tachoni}
\define@key{names}{dsq}{Tadaksahak}
\define@key{names}{tdy}{Tadyawan}
\define@key{names}{rob}{Tae'}
\define@key{names}{tcd}{Tafi}
\define@key{names}{klg}{Tagakaulu Kalagan}
\define@key{names}{bgs}{Tagabawa}
\define@key{names}{mvv}{Tagal Murut}
\define@key{names}{tgz}{Tagalaka}
\define@key{names}{tbm}{Tagbu}
\define@key{names}{tda}{Tagdal}
\define@key{names}{tgx}{Tagish}
\define@key{names}{tgj}{Tagin}
\define@key{names}{tgw}{Tagwana Senoufo}
\define@key{names}{tht}{Tahltan}
\define@key{names}{blt}{Tai Dam}
\define@key{names}{tyj}{Tai Do-Mene-Yo}
\define@key{names}{tyr}{Tai Daeng-Meuay}
\define@key{names}{twh}{Tai Dón}
\define@key{names}{tiz}{Tai Hongjin}
\define@key{names}{taw}{Tai}
\define@key{names}{aos}{Taikat}
\define@key{names}{tlq}{Muak}
\define@key{names}{thi}{Tai Long}
\define@key{names}{tjl}{Tai Laing}
\define@key{names}{tdd}{Tai Nüa}
\define@key{names}{ago}{Tainae}
\define@key{names}{tnq}{Taino}
\define@key{names}{tpo}{Tai Pao}
\define@key{names}{uar}{Tairuma}
\define@key{names}{tmm}{Tai Thanh}
\define@key{names}{cuu}{Tai Ya}
\define@key{names}{acq}{Ta'izzi-Adeni Arabic}
\define@key{names}{pee}{Taje}
\define@key{names}{tdj}{Tajio}
\define@key{names}{abh}{Tajiki Arabic}
\define@key{names}{tja}{Tajuasohn}
\define@key{names}{tkz}{Takua}
\define@key{names}{nho}{Takuu}
\define@key{names}{tke}{Takwane}
\define@key{names}{tak}{Tala}
\define@key{names}{tdf}{Talieng}
\define@key{names}{tlr}{Talise}
\define@key{names}{tlv}{Taliabu}
\define@key{names}{tal}{Tal}
\define@key{names}{tln}{Talondo'}
\define@key{names}{tlk}{Taloki}
\define@key{names}{tzl}{Talossan}
\define@key{names}{yta}{Lavu-Yongsheng-Talu}
\define@key{names}{tcl}{Taman (Myanmar)}
\define@key{names}{tmn}{Taman (Indonesia)}
\define@key{names}{tmz}{Tamanaku}
\define@key{names}{vmx}{Tamazola Mixtec}
\define@key{names}{ten}{Tama (Colombia)}
\define@key{names}{tls}{Tambotalo}
\define@key{names}{xxt}{Tambora}
\define@key{names}{tdk}{Tambas}
\define@key{names}{tmy}{Tami}
\define@key{names}{tax}{Tamki}
\define@key{names}{tml}{Tamnim Citak}
\define@key{names}{tpu}{Tampuan}
\define@key{names}{low}{Tampias Lobu}
\define@key{names}{tpv}{Tanapag}
\define@key{names}{tcm}{Tanahmerah}
\define@key{names}{tni}{Tandia}
\define@key{names}{tdx}{Tandroy Malagasy}
\define@key{names}{tgn}{Tandaganon}
\define@key{names}{tnx}{Tanema}
\define@key{names}{tnv}{Tangchangya}
\define@key{names}{txg}{Tangut}
\define@key{names}{tgp}{Movono}
\define@key{names}{tkx}{Tangko}
\define@key{names}{tgu}{Tanggu}
\define@key{names}{tbs}{Tanguat}
\define@key{names}{ytl}{Tanglang-Toloza}
\define@key{names}{tbe}{Tanimbili}
\define@key{names}{uji}{Rjili}
\define@key{names}{txy}{Tanosy Malagasy}
\define@key{names}{xnj}{Tanzanian Ngoni}
\define@key{names}{qcs}{Tapachultec}
\define@key{names}{afp}{Tapei}
\define@key{names}{taf}{Tapirapé}
\define@key{names}{txj}{Tarjumo}
\define@key{names}{tpf}{Tarpia}
\define@key{names}{txr}{Tartessian}
\define@key{names}{tdm}{Taruma}
\define@key{names}{twq}{Tasawaq}
\define@key{names}{tmt}{Tasmate}
\define@key{names}{ttd}{Tauade}
\define@key{names}{tco}{Taungyo}
\define@key{names}{tpa}{Taupota}
\define@key{names}{tad}{Tause}
\define@key{names}{tvs}{Taveta}
\define@key{names}{tvn}{Tavoyan}
\define@key{names}{rmu}{Tavringer Romani}
\define@key{names}{twl}{Tawara}
\define@key{names}{xtw}{Tawandê}
\define@key{names}{ttq}{Tawallammat Tamajaq}
\define@key{names}{twy}{Tawoyan}
\define@key{names}{tbp}{Taworta}
\define@key{names}{tcp}{Laamtuk Thet}
\define@key{names}{ayy}{Tayabas Ayta near Lucena City in Western Quezon}
\define@key{names}{tas}{Tay Boi}
\define@key{names}{tnu}{Tay Khang}
\define@key{names}{tys}{Tày Sa Pa}
\define@key{names}{tyt}{Tày Tac}
\define@key{names}{tyz}{Tày}
\define@key{names}{tck}{Tchitchege}
\define@key{names}{bqa}{Tchumbuli}
\define@key{names}{dtu}{Tebul Ure Dogon}
\define@key{names}{tsy}{Tebul Sign Language}
\define@key{names}{tcw}{Tecpatlán Totonac}
\define@key{names}{tuq}{Tedaga}
\define@key{names}{tkq}{Tee}
\define@key{names}{lor}{Téén}
\define@key{names}{tfo}{Tefaro}
\define@key{names}{twe}{Teiwa}
\define@key{names}{ztt}{Tejalapan Zapotec}
\define@key{names}{teg}{Latege}
\define@key{names}{tyx}{Teke-Tyee}
\define@key{names}{lli}{Teke-Laali}
\define@key{names}{ebo}{Teke-Eboo-Nzikou}
\define@key{names}{tyi}{Teke-Tsaayi}
\define@key{names}{tvm}{Tela-Masbuar}
\define@key{names}{tlt}{Teluti}
\define@key{names}{nhv}{Temascaltepec Nahuatl}
\define@key{names}{tjo}{Oued Righ}
\define@key{names}{tbt}{Tembo (Kitembo)}
\define@key{names}{tmv}{Motembo-Kunda}
\define@key{names}{tqb}{Tenetehara}
\define@key{names}{tdo}{Teme}
\define@key{names}{soz}{Temi}
\define@key{names}{tmo}{Temoq}
\define@key{names}{ott}{Temoaya Otomi}
\define@key{names}{tmw}{Temuan}
\define@key{names}{quw}{Tena Lowland Quichua}
\define@key{names}{otn}{Tenango Otomi}
\define@key{names}{dtk}{Tengou-Togo Dogon}
\define@key{names}{tes}{Tengger}
\define@key{names}{pah}{Tenharim-Parintintin-Diahoi}
\define@key{names}{tqn}{Tenino}
\define@key{names}{tns}{Tenis}
\define@key{names}{tct}{T'en}
\define@key{names}{tev}{Teor}
\define@key{names}{cux}{Tepeuxila Cuicatec}
\define@key{names}{cte}{Tepinapa Chinantec}
\define@key{names}{ted}{Tepo Krumen}
\define@key{names}{tef}{Teressa}
\define@key{names}{trb}{Terebu}
\define@key{names}{twg}{Tereweng}
\define@key{names}{tec}{Terik}
\define@key{names}{tmg}{Ternateño}
\define@key{names}{sjt}{Ter Saami}
\define@key{names}{tkg}{Tesaka Malagasy}
\define@key{names}{keg}{Tese}
\define@key{names}{twc}{Teshenawa}
\define@key{names}{tez}{Tetserret}
\define@key{names}{tdt}{Tetun Dili}
\define@key{names}{tve}{Te'un}
\define@key{names}{cut}{Teutila Cuicatec}
\define@key{names}{twx}{Tewe}
\define@key{names}{otx}{Texcatepec Otomi}
\define@key{names}{poq}{Texistepec Popoluca}
\define@key{names}{mxb}{Tezoatlán Mixtec}
\define@key{names}{thy}{Tha}
\define@key{names}{thn}{Thachanadan}
\define@key{names}{soa}{Thai Song}
\define@key{names}{nki}{Thangal Naga}
\define@key{names}{thk}{Tharaka}
\define@key{names}{iin}{Thiin}
\define@key{names}{tou}{Tho}
\define@key{names}{ytp}{Thopho}
\define@key{names}{txh}{Thracian}
\define@key{names}{thu}{Thuri}
\define@key{names}{ahi}{Tiagbamrin Aizi}
\define@key{names}{mnl}{Tiale}
\define@key{names}{tbj}{Tiang}
\define@key{names}{ngy}{Tibea}
\define@key{names}{lsn}{Tibetan Sign Language}
\define@key{names}{tcn}{Tichurong}
\define@key{names}{mtx}{Tidaá Mixtec}
\define@key{names}{tia}{Tidikelt-Tuat Tamazight}
\define@key{names}{tiq}{Tiefo-Daramandugu}
\define@key{names}{boo}{Tiemacèwè Bozo}
\define@key{names}{tii}{Tiene}
\define@key{names}{nza}{Tigon Mbembe}
\define@key{names}{txq}{Tii}
\define@key{names}{xtl}{Tijaltepec Mixtec}
\define@key{names}{tkp}{Tikopia}
\define@key{names}{otl}{Tilapa Otomi}
\define@key{names}{zts}{Tilquiapan Zapotec}
\define@key{names}{tij}{Tilung}
\define@key{names}{tim}{Timbe}
\define@key{names}{tvy}{Timor Pidgin}
\define@key{names}{xsb}{Tinà Sambal}
\define@key{names}{tit}{Tinigua}
\define@key{names}{tpz}{Tinputz}
\define@key{names}{tpe}{Tippera}
\define@key{names}{tra}{Tirahi}
\define@key{names}{tic}{Tira}
\define@key{names}{tde}{Tiranige Diga Dogon}
\define@key{names}{tdq}{Tita}
\define@key{names}{ttv}{Titan}
\define@key{names}{lax}{Tiwa (India)}
\define@key{names}{tju}{Tjurruru}
\define@key{names}{tpl}{Tlacoapa Me'phaa}
\define@key{names}{ctl}{Tlacoatzintepec Chinantec}
\define@key{names}{zpk}{Tlacolulita Zapotec}
\define@key{names}{nuz}{Tlamacazapa Nahuatl}
\define@key{names}{mqh}{Tlazoyaltepec Mixtec}
\define@key{names}{tmf}{Toba-Enenlhet}
\define@key{names}{tng}{Tobanga}
\define@key{names}{tgh}{Tobagonian Creole English}
\define@key{names}{tox}{Tobian}
\define@key{names}{tgb}{Tobilung}
\define@key{names}{taz}{Tocho}
\define@key{names}{tdr}{Todrah}
\define@key{names}{tlg}{Tofanma}
\define@key{names}{tfi}{Tofin Gbe}
\define@key{names}{tor}{Togbo-Vara Banda}
\define@key{names}{tgy}{Togoyo}
\define@key{names}{zuh}{Tokano}
\define@key{names}{xto}{Tokharian A}
\define@key{names}{txb}{Tokharian B}
\define@key{names}{tok}{Toki Pona}
\define@key{names}{tkn}{Toku-No-Shima}
\define@key{names}{lbw}{Tolaki}
\define@key{names}{tlm}{Tolomako}
\define@key{names}{tol}{Tolowa-Chetco}
\define@key{names}{tod}{Toma}
\define@key{names}{tdi}{Tomadino}
\define@key{names}{tom}{Tombulu}
\define@key{names}{txa}{Tombonuo}
\define@key{names}{ttp}{Tombelala}
\define@key{names}{txm}{Tomini}
\define@key{names}{dtm}{Tomo Kan Dogon}
\define@key{names}{tqp}{Tomoip}
\define@key{names}{tst}{Tondi Songway Kiini}
\define@key{names}{tnz}{Maniq}
\define@key{names}{tny}{Tongwe}
\define@key{names}{tog}{Tonga (Nyasa)}
\define@key{names}{xgf}{Tongva}
\define@key{names}{tjn}{Tonjon}
\define@key{names}{tnw}{Tonsawang}
\define@key{names}{txs}{Tonsea}
\define@key{names}{toz}{To}
\define@key{names}{ttj}{Tooro}
\define@key{names}{toq}{Toposa}
\define@key{names}{toy}{Topoiyo}
\define@key{names}{ttu}{Torau}
\define@key{names}{trz}{Torá}
\define@key{names}{trj}{Toram}
\define@key{names}{fit}{Meänkieli}
\define@key{names}{tdv}{Toro}
\define@key{names}{tqr}{Torona}
\define@key{names}{dtt}{Toro Tegu Dogon}
\define@key{names}{tno}{Toromono}
\define@key{names}{tei}{Aro}
\define@key{names}{als}{Northern Tosk Albanian}
\define@key{names}{ttl}{Totela}
\define@key{names}{txo}{Toto}
\define@key{names}{txe}{Totoli}
\define@key{names}{ttk}{Totoro}
\define@key{names}{zph}{Totomachapan Zapotec}
\define@key{names}{tqu}{Touo}
\define@key{names}{neb}{Toura (Côte d'Ivoire)}
\define@key{names}{don}{Toura (Papua New Guinea)}
\define@key{names}{ttn}{Towei}
\define@key{names}{xtg}{Transalpine Gaulish}
\define@key{names}{trl}{Traveller Scottish}
\define@key{names}{rmg}{Traveller Norwegian}
\define@key{names}{rmd}{Traveller Danish}
\define@key{names}{trm}{Tregami}
\define@key{names}{tme}{Tremembé}
\define@key{names}{stg}{Trieng}
\define@key{names}{tip}{Trimuris}
\define@key{names}{trx}{Tringgus-Sembaan Bidayuh}
\define@key{names}{tgq}{Tring}
\define@key{names}{trn}{Trinitario-Javeriano-Loretano}
\define@key{names}{trf}{Trinidadian Creole English}
\define@key{names}{lst}{Trinidad and Tobago Sign Language}
\define@key{names}{tka}{Truká}
\define@key{names}{tsa}{Tsaangi}
\define@key{names}{tsd}{Tsakonian}
\define@key{names}{kvz}{Tsaukambo}
\define@key{names}{tsb}{Tsamai}
\define@key{names}{tsk}{Tseku}
\define@key{names}{txc}{Tsetsaut}
\define@key{names}{kdl}{Tsikimba}
\define@key{names}{xmw}{Tsimihety Malagasy}
\define@key{names}{tsw}{Salka-Tsishingini}
\define@key{names}{hio}{Northern Tshwa}
\define@key{names}{ldp}{Tso}
\define@key{names}{lto}{Tsotso}
\define@key{names}{fly}{Tsotsitaal}
\define@key{names}{ttz}{Tsum}
\define@key{names}{tsl}{Ts'ün-Lao}
\define@key{names}{tvd}{Tsuvadi}
\define@key{names}{tsh}{Tsuvan}
\define@key{names}{two}{Tswapong}
\define@key{names}{tsc}{Tswa}
\define@key{names}{nrt}{Tualatin-Yamhill}
\define@key{names}{tuy}{Tugen}
\define@key{names}{tuj}{Tugutil}
\define@key{names}{khc}{Tukang Besi North}
\define@key{names}{bhq}{Tukang Besi South}
\define@key{names}{tkf}{Tukumanféd}
\define@key{names}{tkd}{Tukudede}
\define@key{names}{tul}{Tula}
\define@key{names}{tlu}{Tulehu}
\define@key{names}{tey}{Tulishi}
\define@key{names}{rak}{Tulu-Bohuai}
\define@key{names}{krt}{Tumari Kanuri}
\define@key{names}{iou}{Tuma-Irumu}
\define@key{names}{tum}{Tumbuka}
\define@key{names}{kku}{Tumi}
\define@key{names}{xtq}{Tumshuqese}
\define@key{names}{tbr}{Tumtum}
\define@key{names}{enh}{Tundra Enets}
\define@key{names}{trt}{Tunggare}
\define@key{names}{tse}{Tunisian Sign Language}
\define@key{names}{tug}{Tunia}
\define@key{names}{tjg}{Tunjung}
\define@key{names}{tqq}{Tunni}
\define@key{names}{dza}{Tunzu}
\define@key{names}{ttf}{Tuotomb}
\define@key{names}{tpr}{Tuparí}
\define@key{names}{tpw}{Lingua Geral Paulista}
\define@key{names}{trh}{Turaka}
\define@key{names}{trd}{Turi}
\define@key{names}{twt}{Turiwára}
\define@key{names}{tuz}{Turka}
\define@key{names}{tch}{Turks And Caicos Creole English}
\define@key{names}{tru}{Turoyo}
\define@key{names}{try}{Turung}
\define@key{names}{tqm}{Turumsa}
\define@key{names}{ttg}{Tutong}
\define@key{names}{tmi}{Tutuba}
\define@key{names}{mtu}{Tututepec Mixtec}
\define@key{names}{tww}{Tuwari}
\define@key{names}{ifk}{Tuwali Ifugao}
\define@key{names}{bov}{Tuwuli}
\define@key{names}{tud}{Tuxá}
\define@key{names}{tux}{Tuxináwa}
\define@key{names}{xjb}{Tweed-Albert}
\define@key{names}{twn}{Cambap-Langa}
\define@key{names}{uam}{Uamué}
\define@key{names}{ksj}{Uare}
\define@key{names}{byc}{Ubaghara}
\define@key{names}{uba}{Ubang}
\define@key{names}{ubi}{Ubi}
\define@key{names}{ubr}{Ubir}
\define@key{names}{cpb}{Ucayali-Yurúa Ashéninka}
\define@key{names}{uda}{Uda}
\define@key{names}{udu}{Uduk}
\define@key{names}{ufi}{Ufim}
\define@key{names}{uga}{Ugaritic}
\define@key{names}{uge}{Ughele}
\define@key{names}{ugo}{Ugong}
\define@key{names}{uha}{Uhami}
\define@key{names}{uis}{Uisai}
\define@key{names}{udj}{Ujir}
\define@key{names}{kcf}{Ukaan}
\define@key{names}{ukh}{Ukhwejo}
\define@key{names}{umi}{Ukit}
\define@key{names}{ukp}{Ukpe-Bayobiri}
\define@key{names}{akd}{Ukpet-Ehom}
\define@key{names}{ukl}{Ukrainian Sign Language}
\define@key{names}{uku}{Ukue}
\define@key{names}{ukg}{Ukuriguma}
\define@key{names}{ukq}{Ukwa}
\define@key{names}{ukw}{Ukwuani-Aboh-Ndoni}
\define@key{names}{svb}{Ulau-Suain}
\define@key{names}{ull}{Ullatan}
\define@key{names}{ulb}{Ulukwumi}
\define@key{names}{ulm}{Ulumanda'}
\define@key{names}{ulw}{Ulwa}
\define@key{names}{ulu}{Uma' Lung}
\define@key{names}{xky}{Uma' Lasan}
\define@key{names}{gdn}{Umanakaina}
\define@key{names}{umd}{Umbindhamu}
\define@key{names}{xum}{Umbrian}
\define@key{names}{umr}{Umbugarla}
\define@key{names}{umg}{Umbuygamu}
\define@key{names}{upi}{Umeda-Punda}
\define@key{names}{sju}{Ume Saami}
\define@key{names}{due}{Umiray Dumaget Agta}
\define@key{names}{umm}{Umon}
\define@key{names}{umo}{Umotína}
\define@key{names}{unz}{Unde Kaili}
\define@key{names}{bbn}{Uneapa}
\define@key{names}{une}{Uneme}
\define@key{names}{xgu}{Unggumi}
\define@key{names}{uni}{Uni}
\define@key{names}{uln}{Unserdeutsch}
\define@key{names}{onu}{Unua}
\define@key{names}{unu}{Unubahe}
\define@key{names}{tov}{Upper Taromi}
\define@key{names}{tku}{Upper Necaxa Totonac}
\define@key{names}{sxu}{Central East Middle German}
\define@key{names}{tth}{Upper Ta'oih}
\define@key{names}{dmg}{Upper Kinabatangan}
\define@key{names}{dna}{Upper Grand Valley Dani}
\define@key{names}{xup}{Upper Umpqua}
\define@key{names}{tau}{Upper Tanana}
\define@key{names}{url}{Urali of Idukki}
\define@key{names}{urm}{Urapmin}
\define@key{names}{uro}{Ura (Papua New Guinea)}
\define@key{names}{xur}{Urartian}
\define@key{names}{urg}{Urigina}
\define@key{names}{uvh}{Uri}
\define@key{names}{urx}{Urimo}
\define@key{names}{urc}{Urningangg}
\define@key{names}{urv}{Uruava}
\define@key{names}{urn}{Uruangnirin}
\define@key{names}{urz}{Uru-Eu-Wau-Wau}
\define@key{names}{ugy}{Uruguayan Sign Language}
\define@key{names}{uru}{Urumi}
\define@key{names}{urp}{Uru-Pa-In}
\define@key{names}{usk}{Usaghade}
\define@key{names}{ush}{Ushojo}
\define@key{names}{ulf}{Usku}
\define@key{names}{usp}{Uspanteco}
\define@key{names}{usi}{Usui}
\define@key{names}{omo}{Utarmbung}
\define@key{names}{wsg}{Adilabad Gondi}
\define@key{names}{utu}{Utu}
\define@key{names}{uuu}{U}
\define@key{names}{evh}{Uvbie}
\define@key{names}{usu}{Uya}
\define@key{names}{auz}{Uzbeki Arabic}
\define@key{names}{eze}{Uzekwe}
\define@key{names}{vaa}{Vaagri Booli}
\define@key{names}{kqu}{Vaal-Orange}
\define@key{names}{vgr}{Vaghri}
\define@key{names}{dkg}{Kadung}
\define@key{names}{tva}{Vaghua}
\define@key{names}{vap}{Vaiphei}
\define@key{names}{vae}{Vale}
\define@key{names}{vsv}{Valencian Sign Language}
\define@key{names}{vmv}{Valley Maidu}
\define@key{names}{cvn}{Valle Nacional Chinantec}
\define@key{names}{vlp}{Valpei}
\define@key{names}{mkt}{Vamale}
\define@key{names}{mlr}{Vame}
\define@key{names}{mpr}{Vangunu}
\define@key{names}{vnk}{Lovono}
\define@key{names}{vau}{Vanuma}
\define@key{names}{vao}{Vao}
\define@key{names}{vah}{Varhadi-Nagpuri}
\define@key{names}{vrs}{Varisi}
\define@key{names}{vav}{Dungar Varli}
\define@key{names}{vaj}{Northern Ju}
\define@key{names}{val}{Vehes}
\define@key{names}{vem}{Vemgo-Mabas}
\define@key{names}{vsl}{Venezuelan Sign Language}
\define@key{names}{xve}{Venetic}
\define@key{names}{vec}{Venetian}
\define@key{names}{veo}{Ventureño}
\define@key{names}{vra}{Vera'a}
\define@key{names}{vid}{Vidunda}
\define@key{names}{vig}{Viemo}
\define@key{names}{vil}{Vilela}
\define@key{names}{dyg}{Villa Viciosa Agta}
\define@key{names}{svc}{Vincentian Creole English}
\define@key{names}{vin}{Vinza}
\define@key{names}{vic}{Virgin Islands Creole English}
\define@key{names}{vis}{Vishavan}
\define@key{names}{vit}{Viti}
\define@key{names}{vto}{Vitou}
\define@key{names}{vls}{Western Flemish}
\define@key{names}{vol}{Volapük}
\define@key{names}{kch}{Vono}
\define@key{names}{vor}{Voro}
\define@key{names}{vum}{Vumbu}
\define@key{names}{vnp}{Vunapu}
\define@key{names}{vun}{Vunjo}
\define@key{names}{msn}{Vurës}
\define@key{names}{vut}{Vute}
\define@key{names}{wbi}{Vwanji}
\define@key{names}{wmn}{Waamwang}
\define@key{names}{wab}{Wab}
\define@key{names}{wbb}{Wabo}
\define@key{names}{kmx}{Waboda}
\define@key{names}{wci}{Waci Gbe}
\define@key{names}{wdg}{Wadaginam}
\define@key{names}{wbq}{Waddar}
\define@key{names}{kxp}{Wadiyara Koli}
\define@key{names}{wdu}{Wadjigu}
\define@key{names}{wag}{Wa'ema}
\define@key{names}{wrx}{Kolor}
\define@key{names}{waj}{Waffa}
\define@key{names}{wga}{Wagaya}
\define@key{names}{wgb}{Wagawaga}
\define@key{names}{wbr}{Wagdi}
\define@key{names}{fad}{Wagi (Papua New Guinea)}
\define@key{names}{whk}{Eastern Lowland Kenyah}
\define@key{names}{wgo}{Waigeo}
\define@key{names}{wlr}{Ale}
\define@key{names}{wlk}{Eel River Athabaskan}
\define@key{names}{wmh}{Waima'a}
\define@key{names}{atr}{Waimiri-Atroari}
\define@key{names}{wli}{Waioli}
\define@key{names}{wja}{Waja}
\define@key{names}{wav}{Waka}
\define@key{names}{wwb}{Wakabunga}
\define@key{names}{wkd}{Wakde}
\define@key{names}{waf}{Wakoná}
\define@key{names}{lgl}{Wala}
\define@key{names}{wlw}{Walak}
\define@key{names}{wly}{Waling}
\define@key{names}{wll}{Wali (Sudan)}
\define@key{names}{wlx}{Wali (Ghana)}
\define@key{names}{waa}{Northeast Sahaptin}
\define@key{names}{wln}{Walloon}
\define@key{names}{wae}{Walser}
\define@key{names}{ola}{Walungge}
\define@key{names}{wmc}{Wamas}
\define@key{names}{wmi}{Wamin}
\define@key{names}{lbq}{Wampar}
\define@key{names}{waz}{Wampur}
\define@key{names}{qyp}{Wampano}
\define@key{names}{wnp}{Wanap}
\define@key{names}{wnb}{Mokati}
\define@key{names}{nnp}{Wancho Naga}
\define@key{names}{wbh}{Wanda}
\define@key{names}{wdd}{Wandji}
\define@key{names}{wad}{Wandamen}
\define@key{names}{mfi}{Wandala}
\define@key{names}{wne}{Waneci}
\define@key{names}{hwa}{Wané}
\define@key{names}{wnm}{Wanggamala}
\define@key{names}{lwg}{Wanga}
\define@key{names}{wng}{Wanggom}
\define@key{names}{jub}{Wannu}
\define@key{names}{wno}{Wano}
\define@key{names}{wnk}{Wanukaka}
\define@key{names}{wny}{Wanyi}
\define@key{names}{juk}{Wapan}
\define@key{names}{juw}{Wãpha}
\define@key{names}{wbf}{Samue}
\define@key{names}{tci}{Anta-Komnzo-Wára-Wérè-Kémä}
\define@key{names}{srv}{Waray Sorsogon}
\define@key{names}{bpe}{Bauni}
\define@key{names}{wre}{Ware}
\define@key{names}{wai}{Wares}
\define@key{names}{wri}{Wariyangga}
\define@key{names}{wbe}{Waritai}
\define@key{names}{aml}{War-Jaintia}
\define@key{names}{wji}{Warji}
\define@key{names}{bgv}{Warkay-Bipim}
\define@key{names}{wrl}{Warlmanpa}
\define@key{names}{wrn}{Warnang}
\define@key{names}{wru}{Waru}
\define@key{names}{wrv}{Waruna}
\define@key{names}{wss}{Wasa}
\define@key{names}{gsp}{Wasembo}
\define@key{names}{wsu}{Wasu}
\define@key{names}{wtk}{Watakataui}
\define@key{names}{wah}{Watubela}
\define@key{names}{wuy}{Wauyai}
\define@key{names}{www}{Wawa}
\define@key{names}{wow}{Wawonii}
\define@key{names}{wxa}{Waxianghua}
\define@key{names}{ctt}{Wayanad Chetti}
\define@key{names}{wyr}{Wayoró}
\define@key{names}{weh}{Weh}
\define@key{names}{wew}{Wewewa}
\define@key{names}{wlh}{Welaun}
\define@key{names}{klh}{Weliki}
\define@key{names}{wei}{Were}
\define@key{names}{gxx}{Wè Southern}
\define@key{names}{ywl}{Western Lalu}
\define@key{names}{hmw}{Western Mashan Hmong}
\define@key{names}{ojw}{Western Ojibwa}
\define@key{names}{tqt}{Ozumatlán Totonac}
\define@key{names}{yih}{Western Yiddish}
\define@key{names}{pnb}{Western Panjabi}
\define@key{names}{lcp}{Western Lawa}
\define@key{names}{kuf}{Western Katu}
\define@key{names}{mut}{Western Muria}
\define@key{names}{kyu}{Western Kayah}
\define@key{names}{tdg}{Western Tamang}
\define@key{names}{wmg}{Western Muya}
\define@key{names}{raf}{Western Meohang}
\define@key{names}{mmr}{Western Xiangxi Miao}
\define@key{names}{lia}{West-Central Limba}
\define@key{names}{xwl}{Western Xwla Gbe}
\define@key{names}{bbp}{West Central Banda}
\define@key{names}{ssl}{Western Sisaala}
\define@key{names}{krw}{Western Krahn}
\define@key{names}{nnd}{West Ambae}
\define@key{names}{uve}{West Uvean}
\define@key{names}{mss}{West Masela}
\define@key{names}{lmj}{West Lembata}
\define@key{names}{drn}{West Damar}
\define@key{names}{suc}{Western Subanon}
\define@key{names}{twb}{Western Tawbuid}
\define@key{names}{pne}{Western Penan}
\define@key{names}{zbw}{West Berawan}
\define@key{names}{dnw}{Western Dani}
\define@key{names}{nhw}{Western Huasteca Nahuatl}
\define@key{names}{pua}{Western Highland Purepecha}
\define@key{names}{gnw}{Western Bolivian Guaraní}
\define@key{names}{jmx}{Western Juxtlahuaca Mixtec}
\define@key{names}{tnb}{Western Tunebo}
\define@key{names}{amw}{Western Neo-Aramaic}
\define@key{names}{azn}{Western Durango Nahuatl}
\define@key{names}{wwo}{Dorig}
\define@key{names}{wea}{Wewaw}
\define@key{names}{wec}{Wè Western}
\define@key{names}{woy}{Weyto}
\define@key{names}{lwh}{White Lachi}
\define@key{names}{giw}{Duoluo Gelao}
\define@key{names}{tnp}{Whitesands}
\define@key{names}{tua}{Wiarumus}
\define@key{names}{mtp}{Wichí Lhamtés Nocten}
\define@key{names}{wlv}{Wichí Lhamtés Vejoz}
\define@key{names}{wik}{Wikalkan}
\define@key{names}{wie}{Wik-Epa}
\define@key{names}{wij}{Wik-Iiyanh}
\define@key{names}{wif}{Wik-Keyangan}
\define@key{names}{wih}{Wik-Me'anha}
\define@key{names}{wua}{Wikngenchera}
\define@key{names}{wil}{Wilawila}
\define@key{names}{wit}{Wintu}
\define@key{names}{gdr}{Wipi}
\define@key{names}{wrh}{Wiradhuri}
\define@key{names}{wir}{Wiraféd}
\define@key{names}{wiu}{Wiru}
\define@key{names}{xwc}{Woccon}
\define@key{names}{woc}{Wogeo}
\define@key{names}{wbw}{Woi}
\define@key{names}{wyi}{Woiwurrung-Thagungwurrung}
\define@key{names}{jod}{Wojenaka}
\define@key{names}{wod}{Wolani}
\define@key{names}{wle}{Wolane}
\define@key{names}{wom}{Wom (Nigeria)}
\define@key{names}{wmo}{Wom (Papua New Guinea)}
\define@key{names}{won}{Wongo}
\define@key{names}{cwd}{Woods Cree}
\define@key{names}{kda}{Worimi}
\define@key{names}{wor}{Woria}
\define@key{names}{jud}{Worodougou}
\define@key{names}{wsv}{Wotapuri-Katarqalai}
\define@key{names}{wtw}{Wotu}
\define@key{names}{wud}{Wudu}
\define@key{names}{qgu}{Wulguru}
\define@key{names}{wlu}{Wuliwuli}
\define@key{names}{wux}{Wulna}
\define@key{names}{bqm}{Wumboko-Bubia}
\define@key{names}{wum}{Wumbvu}
\define@key{names}{ywu}{Wumeng Nasu}
\define@key{names}{bwn}{Wunai Bunu}
\define@key{names}{wub}{Wunambal}
\define@key{names}{wur}{Wurrugu}
\define@key{names}{yig}{Wusa Nasu}
\define@key{names}{bse}{Wushi}
\define@key{names}{wsi}{Kula (Vanuatu)}
\define@key{names}{wuh}{Wutunhua}
\define@key{names}{wut}{Wutung}
\define@key{names}{wuv}{Wuvulu-Aua}
\define@key{names}{wym}{Wymysorys}
\define@key{names}{zax}{Xadani Zapotec}
\define@key{names}{xkr}{Xakriabá}
\define@key{names}{xan}{Xamtanga}
\define@key{names}{ztg}{Xanaguía Zapotec}
\define@key{names}{axx}{Xaragure}
\define@key{names}{xeg}{//Xegwi}
\define@key{names}{xet}{Xetá}
\define@key{names}{hsn}{Xiang Chinese}
\define@key{names}{sjo}{Xibe}
\define@key{names}{asn}{Xingú Asuriní}
\define@key{names}{xiy}{Xipaya}
\define@key{names}{xip}{Xipináwa}
\define@key{names}{xii}{Xiri}
\define@key{names}{xoo}{Xukurú}
\define@key{names}{xwe}{Xwela Gbe}
\define@key{names}{tyy}{Tiyaa}
\define@key{names}{muu}{Yaaku}
\define@key{names}{yar}{Yabarana}
\define@key{names}{ybn}{Yabaâna-Mainatari}
\define@key{names}{ybm}{Yaben}
\define@key{names}{ybo}{Yabong}
\define@key{names}{ekr}{Yace}
\define@key{names}{rys}{Yaeyama}
\define@key{names}{wfg}{Yafi}
\define@key{names}{ygm}{Yagomi}
\define@key{names}{ygw}{Yagwoia}
\define@key{names}{rhp}{Yahang}
\define@key{names}{ner}{Yahadian}
\define@key{names}{ynu}{Yahuna}
\define@key{names}{iyx}{Yaka (Congo)}
\define@key{names}{ykk}{Yakaikeke}
\define@key{names}{ybh}{Yakkha}
\define@key{names}{xyl}{Yalakalore}
\define@key{names}{yba}{Yala}
\define@key{names}{jal}{Yalahatan-Haruru-Awaiya}
\define@key{names}{zpu}{Yalálag Zapotec}
\define@key{names}{yal}{Yalunka}
\define@key{names}{ymp}{Yamap}
\define@key{names}{yat}{Yambeta}
\define@key{names}{ymb}{Yambes}
\define@key{names}{yme}{Yameo}
\define@key{names}{ymn}{Yamna}
\define@key{names}{qur}{Chaupihuaranga Quechua}
\define@key{names}{yda}{Yanda}
\define@key{names}{dym}{Yanda Dom Dogon}
\define@key{names}{xyb}{Yandjibara}
\define@key{names}{zyg}{Yang Zhuang}
\define@key{names}{jng}{Yangman}
\define@key{names}{yng}{Yango}
\define@key{names}{bsx}{Yangkam}
\define@key{names}{yav}{Yangben}
\define@key{names}{ygl}{Yangum Gel}
\define@key{names}{ymo}{Yangum Mon}
\define@key{names}{yde}{Yangum Dey}
\define@key{names}{ynl}{Yangulam}
\define@key{names}{tjj}{Yangathimri}
\define@key{names}{ysm}{Yangon Myanmar Sign Language}
\define@key{names}{jay}{Nhangu}
\define@key{names}{guu}{Yanomamö}
\define@key{names}{asy}{Yaosakor Asmat}
\define@key{names}{yre}{Yaouré}
\define@key{names}{yev}{Yeri}
\define@key{names}{yrw}{Yarawata}
\define@key{names}{zae}{Yareni Zapotec}
\define@key{names}{yro}{Yaroame}
\define@key{names}{yko}{Yasa}
\define@key{names}{zty}{Yatee Zapotec}
\define@key{names}{yla}{Ulwa (Papua New Guinea)}
\define@key{names}{yuw}{Yau-Nungon}
\define@key{names}{jau}{Yaur}
\define@key{names}{yyu}{Yau (Sandaun Province)}
\define@key{names}{zpb}{Yautepec Zapotec}
\define@key{names}{qux}{Yauyos Quechua}
\define@key{names}{yvt}{Yavitero-Pareni}
\define@key{names}{yww}{Yawarawarga}
\define@key{names}{ywn}{Yawanawa}
\define@key{names}{yaw}{Yawalapití}
\define@key{names}{yby}{Yaweyuha}
\define@key{names}{ybx}{Yawiyo}
\define@key{names}{ykr}{Yekora}
\define@key{names}{yel}{Yela-Kela}
\define@key{names}{ylg}{Yalaku}
\define@key{names}{ynq}{Yendang}
\define@key{names}{yec}{Yeniche}
\define@key{names}{yei}{Yeni}
\define@key{names}{yra}{Yerakai}
\define@key{names}{gop}{Yeretuar}
\define@key{names}{yrn}{Yerong-Southern Buyang}
\define@key{names}{yeu}{Yerukula}
\define@key{names}{yes}{Yeskwa}
\define@key{names}{yet}{Yetfa}
\define@key{names}{yej}{Yevanic}
\define@key{names}{ydg}{Yidgha}
\define@key{names}{yim}{Yimchungru Naga}
\define@key{names}{kvu}{Yinbaw Karen}
\define@key{names}{yin}{Yinchia}
\define@key{names}{yil}{Yindjilandji}
\define@key{names}{ywg}{Yinhawangka}
\define@key{names}{kvy}{Yintale Karen}
\define@key{names}{yxm}{Yinwum}
\define@key{names}{ljw}{Yirandhali}
\define@key{names}{yiy}{Yir-Yoront}
\define@key{names}{yis}{Yis}
\define@key{names}{gek}{Yiwom}
\define@key{names}{yob}{Yoba}
\define@key{names}{gud}{Yocoboué Dida}
\define@key{names}{yog}{Yogad}
\define@key{names}{ydk}{Yoidik}
\define@key{names}{yki}{Yoke}
\define@key{names}{ygs}{Yolngu Sign Language}
\define@key{names}{xty}{Yoloxochitl Mixtec}
\define@key{names}{pil}{Yom}
\define@key{names}{yoi}{Yonaguni}
\define@key{names}{sxk}{Yoncalla}
\define@key{names}{nru}{Narua}
\define@key{names}{zyn}{Yongnan Zhuang}
\define@key{names}{zyb}{Yongbei Zhuang}
\define@key{names}{yno}{Yong}
\define@key{names}{yon}{Yonggom}
\define@key{names}{yut}{Yopno}
\define@key{names}{mts}{Yora}
\define@key{names}{yox}{Yoron}
\define@key{names}{yot}{Yotti}
\define@key{names}{zyj}{Youjiang Zhuang}
\define@key{names}{ytw}{Yout Wam}
\define@key{names}{yoy}{Yoy}
\define@key{names}{nua}{Yuaga}
\define@key{names}{msd}{Yucatec Maya Sign Language}
\define@key{names}{mvg}{Yucuañe Mixtec}
\define@key{names}{yub}{Yugambal}
\define@key{names}{ysl}{Yugoslavian Sign Language}
\define@key{names}{ygu}{Yugul}
\define@key{names}{yab}{Yuhup}
\define@key{names}{omk}{Malyj Anjuj Omok}
\define@key{names}{ybl}{Yukuben}
\define@key{names}{yuq}{Yuqui}
\define@key{names}{ljx}{Yuru}
\define@key{names}{mab}{Yutanduchi Mixtec}
\define@key{names}{yau}{Hoti}
\define@key{names}{ztx}{Zaachila Zapotec}
\define@key{names}{kji}{Zabana}
\define@key{names}{nhi}{Zacatlán-Ahuacatlán-Tepetzintla Nahuatl}
\define@key{names}{ctz}{Zacatepec Chatino}
\define@key{names}{atb}{Zaiwa}
\define@key{names}{zkr}{Zakhring}
\define@key{names}{zsl}{Zambian Sign Language}
\define@key{names}{zak}{Zanaki}
\define@key{names}{zau}{Zangskari}
\define@key{names}{zna}{Zan Gula}
\define@key{names}{zah}{Zangwal}
\define@key{names}{zpw}{Zaniza Zapotec}
\define@key{names}{zaj}{Zaramo}
\define@key{names}{zbu}{Bu (Zaranda)}
\define@key{names}{zaz}{Zari}
\define@key{names}{zal}{Zauzou}
\define@key{names}{kxk}{Lahta-Zayein Karen}
\define@key{names}{zwa}{Zay}
\define@key{names}{jaj}{Zazao}
\define@key{names}{zua}{Zeem}
\define@key{names}{dhm}{Zemba}
\define@key{names}{zeg}{Zenag}
\define@key{names}{czn}{Zenzontepec Chatino}
\define@key{names}{zhb}{Zhaba}
\define@key{names}{xzh}{Zhangzhung}
\define@key{names}{zhi}{Zhire}
\define@key{names}{zhw}{Zhoa}
\define@key{names}{zia}{Zia}
\define@key{names}{zil}{Zialo}
\define@key{names}{ziw}{Zigula-Mushungulu}
\define@key{names}{zib}{Zimbabwe Sign Language}
\define@key{names}{zmb}{Zimba}
\define@key{names}{zin}{Zinza}
\define@key{names}{sih}{Zire}
\define@key{names}{zrn}{Zirenkel}
\define@key{names}{ziz}{Zizilivakan}
\define@key{names}{pto}{Zo'é}
\define@key{names}{yzk}{Zokhuo}
\define@key{names}{gbz}{Zoroastrian Yazdi}
\define@key{names}{czt}{Zotung Chin}
\define@key{names}{zom}{Zou}
\define@key{names}{zla}{Zula}
\define@key{names}{gnd}{Zulgo-Gemzek}
\define@key{names}{zuy}{Zumaya}
\define@key{names}{jmb}{Zumbun}
\define@key{names}{zzj}{Zuojiang Zhuang}
\define@key{names}{zyp}{Zyphe}

                          \define@key{fams}{knw}{Kxa}
\define@key{fams}{nmn}{Tu}
\define@key{fams}{alu}{Austronesian}
\define@key{fams}{hnh}{Khoe-Kwadi}
\define@key{fams}{xam}{Tu}
\define@key{fams}{huc}{Kxa}
\define@key{fams}{apq}{Great Andamanese}
\define@key{fams}{aiw}{Afro-Asiatic}
\define@key{fams}{aau}{Sepik}
\define@key{fams}{abq}{Northwest Caucasian}
\define@key{fams}{abe}{Algic}
\define@key{fams}{abi}{Niger-Congo}
\define@key{fams}{axb}{Guaicuruan}
\define@key{fams}{abk}{Northwest Caucasian}
\define@key{fams}{abz}{Greater West Bomberai}
\define@key{fams}{kgr}{Isolate}
\define@key{fams}{ace}{Austronesian}
\define@key{fams}{aca}{Arawakan}
\define@key{fams}{acn}{Sino-Tibetan}
\define@key{fams}{ach}{Eastern Sudanic}
\define@key{fams}{acu}{Jivaroan}
\define@key{fams}{acv}{Hokan}
\define@key{fams}{guq}{Tupian}
\define@key{fams}{acr}{Mayan}
\define@key{fams}{kjq}{Keresan}
\define@key{fams}{ads}{other}
\define@key{fams}{adn}{Greater West Bomberai}
\define@key{fams}{adj}{Niger-Congo}
\define@key{fams}{ady}{Northwest Caucasian}
\define@key{fams}{adt}{Pama-Nyungan}
\define@key{fams}{adz}{Austronesian}
\define@key{fams}{awi}{Kamula-Elevala}
\define@key{fams}{afr}{Indo-European}
\define@key{fams}{agd}{Trans-New Guinea}
\define@key{fams}{agq}{Niger-Congo}
\define@key{fams}{ahh}{Trans-New Guinea}
\define@key{fams}{agx}{Nakh-Daghestanian}
\define@key{fams}{agt}{Austronesian}
\define@key{fams}{duo}{Austronesian}
\define@key{fams}{agu}{Mayan}
\define@key{fams}{agr}{Jivaroan}
\define@key{fams}{aht}{Na-Dene}
\define@key{fams}{tba}{Isolate}
\define@key{fams}{ain}{Isolate}
\define@key{fams}{ahp}{Niger-Congo}
\define@key{fams}{aja}{Central Sudanic}
\define@key{fams}{ajg}{Niger-Congo}
\define@key{fams}{aji}{Austronesian}
\define@key{fams}{axk}{Niger-Congo}
\define@key{fams}{abj}{Great Andamanese}
\define@key{fams}{aci}{Great Andamanese}
\define@key{fams}{akx}{Great Andamanese}
\define@key{fams}{aka}{Niger-Congo}
\define@key{fams}{ake}{Cariban}
\define@key{fams}{ahk}{Sino-Tibetan}
\define@key{fams}{akv}{Nakh-Daghestanian}
\define@key{fams}{akl}{Austronesian}
\define@key{fams}{akw}{Niger-Congo}
\define@key{fams}{nrz}{Austronesian}
\define@key{fams}{akz}{Muskogean}
\define@key{fams}{wbj}{Afro-Asiatic}
\define@key{fams}{amp}{Sepik}
\define@key{fams}{btz}{Austronesian}
\define@key{fams}{alh}{Mangarrayi-Maran}
\define@key{fams}{sqi}{Indo-European}
\define@key{fams}{ale}{Eskimo-Aleut}
\define@key{fams}{alq}{Algic}
\define@key{fams}{ald}{Niger-Congo}
\define@key{fams}{gsw}{Indo-European}
\define@key{fams}{aes}{Oregon Coast}
\define@key{fams}{alt}{Altaic}
\define@key{fams}{alp}{Austronesian}
\define@key{fams}{ems}{Eskimo-Aleut}
\define@key{fams}{alr}{Chukotko-Kamchatkan}
\define@key{fams}{aly}{Pama-Nyungan}
\define@key{fams}{amm}{Left May}
\define@key{fams}{amc}{Pano-Tacanan}
\define@key{fams}{amn}{Border}
\define@key{fams}{aie}{Austronesian}
\define@key{fams}{amr}{Harakmbet}
\define@key{fams}{omb}{Austronesian}
\define@key{fams}{amk}{Austronesian}
\define@key{fams}{abt}{Sepik}
\define@key{fams}{adx}{Sino-Tibetan}
\define@key{fams}{aey}{Trans-New Guinea}
\define@key{fams}{ase}{other}
\define@key{fams}{amh}{Afro-Asiatic}
\define@key{fams}{ami}{Austronesian}
\define@key{fams}{amo}{Niger-Congo}
\define@key{fams}{apz}{Trans-New Guinea}
\define@key{fams}{ame}{Arawakan}
\define@key{fams}{amu}{Oto-Manguean}
\define@key{fams}{imi}{Trans-New Guinea}
\define@key{fams}{ani}{Nakh-Daghestanian}
\define@key{fams}{ano}{Isolate}
\define@key{fams}{aty}{Austronesian}
\define@key{fams}{agm}{Trans-New Guinea}
\define@key{fams}{njm}{Sino-Tibetan}
\define@key{fams}{anc}{Afro-Asiatic}
\define@key{fams}{agg}{Senagi}
\define@key{fams}{aoa}{other}
\define@key{fams}{awg}{Pama-Nyungan}
\define@key{fams}{aoi}{Gunwinyguan}
\define@key{fams}{nun}{Sino-Tibetan}
\define@key{fams}{cko}{Niger-Congo}
\define@key{fams}{any}{Niger-Congo}
\define@key{fams}{anu}{Eastern Sudanic}
\define@key{fams}{anz}{Isolate}
\define@key{fams}{njo}{Sino-Tibetan}
\define@key{fams}{apm}{Na-Dene}
\define@key{fams}{apj}{Na-Dene}
\define@key{fams}{apw}{Na-Dene}
\define@key{fams}{apy}{Cariban}
\define@key{fams}{apt}{Sino-Tibetan}
\define@key{fams}{apn}{Macro-Ge}
\define@key{fams}{apu}{Arawakan}
\define@key{fams}{ard}{Pama-Nyungan}
\define@key{fams}{arl}{Zaparoan}
\define@key{fams}{abv}{Afro-Asiatic}
\define@key{fams}{mey}{Afro-Asiatic}
\define@key{fams}{shu}{Afro-Asiatic}
\define@key{fams}{ayl}{Afro-Asiatic}
\define@key{fams}{arz}{Afro-Asiatic}
\define@key{fams}{afb}{Afro-Asiatic}
\define@key{fams}{acw}{Afro-Asiatic}
\define@key{fams}{acm}{Afro-Asiatic}
\define@key{fams}{acy}{Afro-Asiatic}
\define@key{fams}{arb}{Afro-Asiatic}
\define@key{fams}{ary}{Afro-Asiatic}
\define@key{fams}{ajp}{Afro-Asiatic}
\define@key{fams}{ayn}{Afro-Asiatic}
\define@key{fams}{apc}{Afro-Asiatic}
\define@key{fams}{aeb}{Afro-Asiatic}
\define@key{fams}{rmz}{Sino-Tibetan}
\define@key{fams}{akr}{Austronesian}
\define@key{fams}{atq}{Austronesian}
\define@key{fams}{jbj}{South Bird's Head}
\define@key{fams}{aro}{Pano-Tacanan}
\define@key{fams}{arp}{Algic}
\define@key{fams}{aah}{Torricelli}
\define@key{fams}{ape}{Torricelli}
\define@key{fams}{arv}{Afro-Asiatic}
\define@key{fams}{aqc}{Nakh-Daghestanian}
\define@key{fams}{laz}{Austronesian}
\define@key{fams}{ari}{Caddoan}
\define@key{fams}{hye}{Indo-European}
\define@key{fams}{hyw}{Indo-European}
\define@key{fams}{apr}{Austronesian}
\define@key{fams}{aia}{Austronesian}
\define@key{fams}{aer}{Pama-Nyungan}
\define@key{fams}{are}{Pama-Nyungan}
\define@key{fams}{cns}{Asmat-Kamrau Bay}
\define@key{fams}{asm}{Indo-European}
\define@key{fams}{ast}{Indo-European}
\define@key{fams}{asu}{Tupian}
\define@key{fams}{kuz}{Kunza}
\define@key{fams}{aqp}{Isolate}
\define@key{fams}{tay}{Austronesian}
\define@key{fams}{upv}{Austronesian}
\define@key{fams}{aph}{Sino-Tibetan}
\define@key{fams}{atj}{Algic}
\define@key{fams}{atw}{Hokan}
\define@key{fams}{avt}{Torricelli}
\define@key{fams}{aul}{Austronesian}
\define@key{fams}{asf}{other}
\define@key{fams}{auy}{Trans-New Guinea}
\define@key{fams}{ava}{Nakh-Daghestanian}
\define@key{fams}{avn}{Niger-Congo}
\define@key{fams}{avi}{Niger-Congo}
\define@key{fams}{avu}{Central Sudanic}
\define@key{fams}{awb}{Trans-New Guinea}
\define@key{fams}{kwi}{Barbacoan}
\define@key{fams}{awa}{Indo-European}
\define@key{fams}{awn}{Afro-Asiatic}
\define@key{fams}{kmn}{Sepik}
\define@key{fams}{auw}{Border}
\define@key{fams}{nfl}{Austronesian}
\define@key{fams}{ayr}{Aymaran}
\define@key{fams}{aib}{Altaic}
\define@key{fams}{ayo}{Zamucoan}
\define@key{fams}{azb}{Altaic}
\define@key{fams}{koe}{Eastern Sudanic}
\define@key{fams}{bvx}{Niger-Congo}
\define@key{fams}{bav}{Niger-Congo}
\define@key{fams}{wdj}{Wandjiginy}
\define@key{fams}{bfq}{Dravidian}
\define@key{fams}{bde}{Afro-Asiatic}
\define@key{fams}{bia}{Pama-Nyungan}
\define@key{fams}{ksf}{Niger-Congo}
\define@key{fams}{bfd}{Niger-Congo}
\define@key{fams}{bsp}{Niger-Congo}
\define@key{fams}{bmi}{Central Sudanic}
\define@key{fams}{fuu}{Central Sudanic}
\define@key{fams}{bgq}{Indo-European}
\define@key{fams}{kva}{Nakh-Daghestanian}
\define@key{fams}{bdw}{Greater West Bomberai}
\define@key{fams}{bjh}{Sepik}
\define@key{fams}{bdq}{Austro-Asiatic}
\define@key{fams}{bca}{Sino-Tibetan}
\define@key{fams}{bdl}{Austronesian}
\define@key{fams}{bdr}{Austronesian}
\define@key{fams}{bkc}{Niger-Congo}
\define@key{fams}{bdh}{Central Sudanic}
\define@key{fams}{bkq}{Cariban}
\define@key{fams}{bri}{Niger-Congo}
\define@key{fams}{blw}{Austronesian}
\define@key{fams}{blz}{Austronesian}
\define@key{fams}{ban}{Austronesian}
\define@key{fams}{bft}{Sino-Tibetan}
\define@key{fams}{bgn}{Indo-European}
\define@key{fams}{ptu}{Austronesian}
\define@key{fams}{bam}{Mande}
\define@key{fams}{bax}{Niger-Congo}
\define@key{fams}{bcw}{Afro-Asiatic}
\define@key{fams}{jaa}{Arauan}
\define@key{fams}{bza}{Mande}
\define@key{fams}{bdy}{Pama-Nyungan}
\define@key{fams}{bgz}{Austronesian}
\define@key{fams}{bjb}{Pama-Nyungan}
\define@key{fams}{bdg}{Austronesian}
\define@key{fams}{dba}{Isolate}
\define@key{fams}{bvv}{Arawakan}
\define@key{fams}{bwi}{Arawakan}
\define@key{fams}{abb}{Niger-Congo}
\define@key{fams}{bcm}{Austronesian}
\define@key{fams}{bnq}{Austronesian}
\define@key{fams}{peh}{Altaic}
\define@key{fams}{bci}{Niger-Congo}
\define@key{fams}{loy}{Sino-Tibetan}
\define@key{fams}{bbb}{Trans-New Guinea}
\define@key{fams}{brm}{Niger-Congo}
\define@key{fams}{bsn}{Tucanoan}
\define@key{fams}{bcj}{Nyulnyulan}
\define@key{fams}{mlp}{Trans-New Guinea}
\define@key{fams}{bfa}{Eastern Sudanic}
\define@key{fams}{bba}{Niger-Congo}
\define@key{fams}{wra}{Skou}
\define@key{fams}{byr}{Trans-New Guinea}
\define@key{fams}{bae}{Arawakan}
\define@key{fams}{mot}{Chibchan}
\define@key{fams}{bsc}{Niger-Congo}
\define@key{fams}{bas}{Niger-Congo}
\define@key{fams}{bak}{Altaic}
\define@key{fams}{eus}{Isolate}
\define@key{fams}{bya}{Austronesian}
\define@key{fams}{btx}{Austronesian}
\define@key{fams}{bbc}{Austronesian}
\define@key{fams}{bhm}{Afro-Asiatic}
\define@key{fams}{bbd}{Trans-New Guinea}
\define@key{fams}{brg}{Arawakan}
\define@key{fams}{bvz}{Geelvink Bay}
\define@key{fams}{bgr}{Sino-Tibetan}
\define@key{fams}{bsw}{Afro-Asiatic}
\define@key{fams}{bxj}{Pama-Nyungan}
\define@key{fams}{beq}{Niger-Congo}
\define@key{fams}{dbj}{Austronesian}
\define@key{fams}{bej}{Afro-Asiatic}
\define@key{fams}{byw}{Sino-Tibetan}
\define@key{fams}{blc}{Salishan}
\define@key{fams}{bel}{Indo-European}
\define@key{fams}{bem}{Niger-Congo}
\define@key{fams}{bef}{Trans-New Guinea}
\define@key{fams}{nhb}{Mande}
\define@key{fams}{bng}{Niger-Congo}
\define@key{fams}{ben}{Indo-European}
\define@key{fams}{ctg}{Indo-European}
\define@key{fams}{bue}{Isolate}
\define@key{fams}{brf}{Niger-Congo}
\define@key{fams}{shy}{Afro-Asiatic}
\define@key{fams}{grr}{Afro-Asiatic}
\define@key{fams}{tzm}{Afro-Asiatic}
\define@key{fams}{mzb}{Afro-Asiatic}
\define@key{fams}{rif}{Afro-Asiatic}
\define@key{fams}{siz}{Afro-Asiatic}
\define@key{fams}{oua}{Afro-Asiatic}
\define@key{fams}{brc}{other}
\define@key{fams}{zag}{Saharan}
\define@key{fams}{bkl}{Tor-Kwerba}
\define@key{fams}{wti}{Isolate}
\define@key{fams}{xub}{Dravidian}
\define@key{fams}{kap}{Nakh-Daghestanian}
\define@key{fams}{bhb}{Indo-European}
\define@key{fams}{bho}{Indo-European}
\define@key{fams}{unr}{Austro-Asiatic}
\define@key{fams}{bif}{Niger-Congo}
\define@key{fams}{bhw}{Austronesian}
\define@key{fams}{bth}{Austronesian}
\define@key{fams}{bid}{Afro-Asiatic}
\define@key{fams}{bcl}{Austronesian}
\define@key{fams}{bip}{Niger-Congo}
\define@key{fams}{bpr}{Austronesian}
\define@key{fams}{byn}{Afro-Asiatic}
\define@key{fams}{nbj}{Pama-Nyungan}
\define@key{fams}{bll}{Siouan}
\define@key{fams}{blb}{Solomons East Papuan}
\define@key{fams}{bhp}{Austronesian}
\define@key{fams}{bim}{Niger-Congo}
\define@key{fams}{bhg}{Trans-New Guinea}
\define@key{fams}{bin}{Niger-Congo}
\define@key{fams}{gup}{Gunwinyguan}
\define@key{fams}{bkd}{Austronesian}
\define@key{fams}{bjr}{Trans-New Guinea}
\define@key{fams}{bzr}{Pama-Nyungan}
\define@key{fams}{bom}{Niger-Congo}
\define@key{fams}{bvq}{Central Sudanic}
\define@key{fams}{bib}{Mande}
\define@key{fams}{bis}{other}
\define@key{fams}{bla}{Algic}
\define@key{fams}{kvg}{Trans-New Guinea}
\define@key{fams}{bni}{Niger-Congo}
\define@key{fams}{bbo}{Mande}
\define@key{fams}{brx}{Sino-Tibetan}
\define@key{fams}{bzf}{Sepik}
\define@key{fams}{bqc}{Mande}
\define@key{fams}{bol}{Afro-Asiatic}
\define@key{fams}{bli}{Niger-Congo}
\define@key{fams}{bot}{Central Sudanic}
\define@key{fams}{bpu}{Trans-New Guinea}
\define@key{fams}{lbk}{Austronesian}
\define@key{fams}{boa}{Boran}
\define@key{fams}{adi}{Sino-Tibetan}
\define@key{fams}{bor}{Bororoan}
\define@key{fams}{brn}{Chibchan}
\define@key{fams}{bos}{Indo-European}
\define@key{fams}{boz}{Mande}
\define@key{fams}{brh}{Dravidian}
\define@key{fams}{brb}{Austro-Asiatic}
\define@key{fams}{bre}{Indo-European}
\define@key{fams}{bzd}{Chibchan}
\define@key{fams}{bfi}{other}
\define@key{fams}{tcs}{other}
\define@key{fams}{bkk}{Indo-European}
\define@key{fams}{bru}{Austro-Asiatic}
\define@key{fams}{brv}{Austro-Asiatic}
\define@key{fams}{bvb}{Niger-Congo}
\define@key{fams}{buu}{Niger-Congo}
\define@key{fams}{bdk}{Nakh-Daghestanian}
\define@key{fams}{bdm}{Afro-Asiatic}
\define@key{fams}{bug}{Austronesian}
\define@key{fams}{sab}{Chibchan}
\define@key{fams}{bgg}{Sino-Tibetan}
\define@key{fams}{buo}{South Bougainville}
\define@key{fams}{nmg}{Niger-Congo}
\define@key{fams}{bxk}{Niger-Congo}
\define@key{fams}{bul}{Indo-European}
\define@key{fams}{bwu}{Niger-Congo}
\define@key{fams}{bzq}{Austronesian}
\define@key{fams}{bum}{Niger-Congo}
\define@key{fams}{tkw}{Austronesian}
\define@key{fams}{bfu}{Sino-Tibetan}
\define@key{fams}{buh}{Hmong-Mien}
\define@key{fams}{bck}{Bunuban}
\define@key{fams}{bwr}{Afro-Asiatic}
\define@key{fams}{bvr}{Mangrida}
\define@key{fams}{bxm}{Altaic}
\define@key{fams}{bji}{Afro-Asiatic}
\define@key{fams}{mya}{Sino-Tibetan}
\define@key{fams}{mhs}{Austronesian}
\define@key{fams}{bmu}{Trans-New Guinea}
\define@key{fams}{bds}{Afro-Asiatic}
\define@key{fams}{bsk}{Isolate}
\define@key{fams}{bqp}{Mande}
\define@key{fams}{buf}{Niger-Congo}
\define@key{fams}{ngc}{Niger-Congo}
\define@key{fams}{bee}{Sino-Tibetan}
\define@key{fams}{bev}{Niger-Congo}
\define@key{fams}{cjp}{Chibchan}
\define@key{fams}{cbv}{Cacua-Nukak}
\define@key{fams}{cad}{Caddoan}
\define@key{fams}{chl}{Uto-Aztecan}
\define@key{fams}{cak}{Mayan}
\define@key{fams}{rab}{Sino-Tibetan}
\define@key{fams}{cjo}{Arawakan}
\define@key{fams}{kbh}{Isolate}
\define@key{fams}{knm}{Katukinan}
\define@key{fams}{cbu}{Isolate}
\define@key{fams}{ram}{Macro-Ge}
\define@key{fams}{yue}{Sino-Tibetan}
\define@key{fams}{kaq}{Pano-Tacanan}
\define@key{fams}{cbc}{Tucanoan}
\define@key{fams}{car}{Cariban}
\define@key{fams}{mch}{Cariban}
\define@key{fams}{cal}{Austronesian}
\define@key{fams}{crx}{Na-Dene}
\define@key{fams}{cbr}{Pano-Tacanan}
\define@key{fams}{cbs}{Pano-Tacanan}
\define@key{fams}{cat}{Indo-European}
\define@key{fams}{chc}{Siouan}
\define@key{fams}{cto}{Choco}
\define@key{fams}{cav}{Pano-Tacanan}
\define@key{fams}{cbi}{Barbacoan}
\define@key{fams}{cay}{Iroquoian}
\define@key{fams}{cyb}{Isolate}
\define@key{fams}{ceb}{Austronesian}
\define@key{fams}{old}{Niger-Congo}
\define@key{fams}{suq}{Eastern Sudanic}
\define@key{fams}{cld}{Afro-Asiatic}
\define@key{fams}{cjm}{Austronesian}
\define@key{fams}{cja}{Austronesian}
\define@key{fams}{cji}{Nakh-Daghestanian}
\define@key{fams}{can}{Lower Sepik-Ramu}
\define@key{fams}{cha}{Austronesian}
\define@key{fams}{nbc}{Sino-Tibetan}
\define@key{fams}{chx}{Sino-Tibetan}
\define@key{fams}{tuu}{Na-Dene}
\define@key{fams}{cya}{Oto-Manguean}
\define@key{fams}{cta}{Oto-Manguean}
\define@key{fams}{ctp}{Oto-Manguean}
\define@key{fams}{cdn}{Sino-Tibetan}
\define@key{fams}{cbk}{other}
\define@key{fams}{cbt}{Cahuapanan}
\define@key{fams}{che}{Nakh-Daghestanian}
\define@key{fams}{cjh}{Salishan}
\define@key{fams}{mrn}{Austronesian}
\define@key{fams}{xch}{Chimakuan}
\define@key{fams}{cdm}{Sino-Tibetan}
\define@key{fams}{chr}{Iroquoian}
\define@key{fams}{chy}{Algic}
\define@key{fams}{nya}{Niger-Congo}
\define@key{fams}{pei}{Oto-Manguean}
\define@key{fams}{cic}{Muskogean}
\define@key{fams}{cob}{Mayan}
\define@key{fams}{cid}{Hokan}
\define@key{fams}{cbg}{Chibchan}
\define@key{fams}{mrh}{Sino-Tibetan}
\define@key{fams}{csy}{Sino-Tibetan}
\define@key{fams}{ctd}{Sino-Tibetan}
\define@key{fams}{cco}{Oto-Manguean}
\define@key{fams}{cle}{Oto-Manguean}
\define@key{fams}{cpa}{Oto-Manguean}
\define@key{fams}{chq}{Oto-Manguean}
\define@key{fams}{cuc}{Oto-Manguean}
\define@key{fams}{cso}{Oto-Manguean}
\define@key{fams}{cnt}{Oto-Manguean}
\define@key{fams}{csl}{other}
\define@key{fams}{chh}{Penutian}
\define@key{fams}{wac}{Penutian}
\define@key{fams}{cap}{Uru-Chipaya}
\define@key{fams}{chp}{Na-Dene}
\define@key{fams}{cax}{Isolate}
\define@key{fams}{gui}{Tupian}
\define@key{fams}{ctm}{Isolate}
\define@key{fams}{coz}{Oto-Manguean}
\define@key{fams}{cho}{Muskogean}
\define@key{fams}{ctu}{Mayan}
\define@key{fams}{cht}{Hobitu-Cholon}
\define@key{fams}{chd}{Hokan}
\define@key{fams}{clo}{Hokan}
\define@key{fams}{chf}{Mayan}
\define@key{fams}{caa}{Mayan}
\define@key{fams}{crw}{Austro-Asiatic}
\define@key{fams}{cje}{Austronesian}
\define@key{fams}{cjv}{Trans-New Guinea}
\define@key{fams}{cac}{Mayan}
\define@key{fams}{ckt}{Chukotko-Kamchatkan}
\define@key{fams}{clw}{Altaic}
\define@key{fams}{boi}{Chumash}
\define@key{fams}{inz}{Chumash}
\define@key{fams}{ncu}{Niger-Congo}
\define@key{fams}{chk}{Austronesian}
\define@key{fams}{chv}{Altaic}
\define@key{fams}{cao}{Pano-Tacanan}
\define@key{fams}{lua}{Niger-Congo}
\define@key{fams}{clm}{Salishan}
\define@key{fams}{xcw}{Coahuiltecan}
\define@key{fams}{cod}{Tupian}
\define@key{fams}{coc}{Hokan}
\define@key{fams}{crd}{Salishan}
\define@key{fams}{con}{Isolate}
\define@key{fams}{kog}{Chibchan}
\define@key{fams}{col}{Salishan}
\define@key{fams}{com}{Uto-Aztecan}
\define@key{fams}{xcm}{Hokan}
\define@key{fams}{swb}{Niger-Congo}
\define@key{fams}{coo}{Salishan}
\define@key{fams}{csz}{Oregon Coast}
\define@key{fams}{cop}{Afro-Asiatic}
\define@key{fams}{crn}{Uto-Aztecan}
\define@key{fams}{cor}{Indo-European}
\define@key{fams}{crk}{Algic}
\define@key{fams}{csw}{Algic}
\define@key{fams}{mus}{Muskogean}
\define@key{fams}{crh}{Altaic}
\define@key{fams}{cro}{Siouan}
\define@key{fams}{cua}{Austro-Asiatic}
\define@key{fams}{cub}{Tucanoan}
\define@key{fams}{cui}{Guahiban}
\define@key{fams}{cuy}{Isolate}
\define@key{fams}{cul}{Arauan}
\define@key{fams}{cup}{Uto-Aztecan}
\define@key{fams}{kpc}{Arawakan}
\define@key{fams}{ces}{Indo-European}
\define@key{fams}{cam}{Austronesian}
\define@key{fams}{kzf}{Austronesian}
\define@key{fams}{dbq}{Afro-Asiatic}
\define@key{fams}{dav}{Niger-Congo}
\define@key{fams}{mps}{Teberan-Pawaian}
\define@key{fams}{dgz}{Trans-New Guinea}
\define@key{fams}{dga}{Niger-Congo}
\define@key{fams}{dag}{Niger-Congo}
\define@key{fams}{dta}{Altaic}
\define@key{fams}{dal}{Afro-Asiatic}
\define@key{fams}{daj}{Eastern Sudanic}
\define@key{fams}{dak}{Siouan}
\define@key{fams}{mbp}{Chibchan}
\define@key{fams}{dnj}{Mande}
\define@key{fams}{daa}{Afro-Asiatic}
\define@key{fams}{dni}{Trans-New Guinea}
\define@key{fams}{dan}{Indo-European}
\define@key{fams}{dry}{Indo-European}
\define@key{fams}{dar}{Nakh-Daghestanian}
\define@key{fams}{prs}{Indo-European}
\define@key{fams}{drd}{Sino-Tibetan}
\define@key{fams}{tcc}{Eastern Sudanic}
\define@key{fams}{dai}{Niger-Congo}
\define@key{fams}{afn}{Ijoid}
\define@key{fams}{deg}{Niger-Congo}
\define@key{fams}{ing}{Na-Dene}
\define@key{fams}{dny}{Arauan}
\define@key{fams}{des}{Tucanoan}
\define@key{fams}{shg}{Khoe-Kwadi}
\define@key{fams}{der}{Sino-Tibetan}
\define@key{fams}{gsg}{other}
\define@key{fams}{dsh}{Afro-Asiatic}
\define@key{fams}{dhl}{Pama-Nyungan}
\define@key{fams}{tbh}{Pama-Nyungan}
\define@key{fams}{dhr}{Pama-Nyungan}
\define@key{fams}{xgm}{Pama-Nyungan}
\define@key{fams}{dhi}{Sino-Tibetan}
\define@key{fams}{div}{Indo-European}
\define@key{fams}{dhu}{Pama-Nyungan}
\define@key{fams}{did}{Eastern Sudanic}
\define@key{fams}{mhu}{Sino-Tibetan}
\define@key{fams}{dur}{Niger-Congo}
\define@key{fams}{dis}{Sino-Tibetan}
\define@key{fams}{dim}{Afro-Asiatic}
\define@key{fams}{diz}{Niger-Congo}
\define@key{fams}{din}{Eastern Sudanic}
\define@key{fams}{dyo}{Niger-Congo}
\define@key{fams}{csk}{Niger-Congo}
\define@key{fams}{dif}{Pama-Nyungan}
\define@key{fams}{mdx}{Afro-Asiatic}
\define@key{fams}{dyy}{Pama-Nyungan}
\define@key{fams}{djr}{Pama-Nyungan}
\define@key{fams}{duj}{Pama-Nyungan}
\define@key{fams}{ddj}{Pama-Nyungan}
\define@key{fams}{dji}{Pama-Nyungan}
\define@key{fams}{jig}{Mirndi}
\define@key{fams}{kbv}{Senagi}
\define@key{fams}{kvo}{Austronesian}
\define@key{fams}{dgo}{Indo-European}
\define@key{fams}{dlg}{Altaic}
\define@key{fams}{dmk}{Indo-European}
\define@key{fams}{rmt}{Indo-European}
\define@key{fams}{kmc}{Tai-Kadai}
\define@key{fams}{doo}{Niger-Congo}
\define@key{fams}{dds}{Dogon}
\define@key{fams}{tds}{Lakes Plain}
\define@key{fams}{dow}{Niger-Congo}
\define@key{fams}{dhv}{Austronesian}
\define@key{fams}{dua}{Niger-Congo}
\define@key{fams}{dud}{Niger-Congo}
\define@key{fams}{gwd}{Afro-Asiatic}
\define@key{fams}{duu}{Sino-Tibetan}
\define@key{fams}{dma}{Niger-Congo}
\define@key{fams}{dgc}{Austronesian}
\define@key{fams}{dus}{Sino-Tibetan}
\define@key{fams}{vam}{Skou}
\define@key{fams}{duc}{Duna-Bogaya}
\define@key{fams}{nld}{Indo-European}
\define@key{fams}{zea}{Indo-European}
\define@key{fams}{dyi}{Niger-Congo}
\define@key{fams}{dbl}{Pama-Nyungan}
\define@key{fams}{dyu}{Mande}
\define@key{fams}{kwa}{Nadahup}
\define@key{fams}{igb}{Niger-Congo}
\define@key{fams}{etr}{Trans-New Guinea}
\define@key{fams}{erk}{Austronesian}
\define@key{fams}{efi}{Niger-Congo}
\define@key{fams}{ega}{Niger-Congo}
\define@key{fams}{eip}{Trans-New Guinea}
\define@key{fams}{etu}{Niger-Congo}
\define@key{fams}{ekg}{Trans-New Guinea}
\define@key{fams}{eko}{Niger-Congo}
\define@key{fams}{mrf}{Morwap}
\define@key{fams}{ema}{Niger-Congo}
\define@key{fams}{emb}{Austronesian}
\define@key{fams}{cmi}{Choco}
\define@key{fams}{emp}{Choco}
\define@key{fams}{amy}{Western Daly}
\define@key{fams}{enq}{Trans-New Guinea}
\define@key{fams}{enn}{Niger-Congo}
\define@key{fams}{eno}{Austronesian}
\define@key{fams}{eng}{Indo-European}
\define@key{fams}{gey}{Niger-Congo}
\define@key{fams}{sja}{Choco}
\define@key{fams}{erg}{Austronesian}
\define@key{fams}{ese}{Pano-Tacanan}
\define@key{fams}{esq}{Isolate}
\define@key{fams}{ekk}{Uralic}
\define@key{fams}{ets}{Niger-Congo}
\define@key{fams}{eve}{Altaic}
\define@key{fams}{ewe}{Niger-Congo}
\define@key{fams}{ewo}{Niger-Congo}
\define@key{fams}{eya}{Na-Dene}
\define@key{fams}{fao}{Indo-European}
\define@key{fams}{faa}{Trans-New Guinea}
\define@key{fams}{fmp}{Niger-Congo}
\define@key{fams}{fij}{Austronesian}
\define@key{fams}{fin}{Uralic}
\define@key{fams}{fse}{other}
\define@key{fams}{foi}{Trans-New Guinea}
\define@key{fams}{ppo}{Teberan-Pawaian}
\define@key{fams}{fon}{Niger-Congo}
\define@key{fams}{frd}{Austronesian}
\define@key{fams}{for}{Trans-New Guinea}
\define@key{fams}{sac}{Algic}
\define@key{fams}{fra}{Indo-European}
\define@key{fams}{fry}{Indo-European}
\define@key{fams}{frs}{Indo-European}
\define@key{fams}{frr}{Indo-European}
\define@key{fams}{fuh}{Niger-Congo}
\define@key{fams}{fuf}{Niger-Congo}
\define@key{fams}{fub}{Niger-Congo}
\define@key{fams}{ffm}{Niger-Congo}
\define@key{fams}{fuv}{Niger-Congo}
\define@key{fams}{fun}{Isolate}
\define@key{fams}{fvr}{Isolate}
\define@key{fams}{fud}{Austronesian}
\define@key{fams}{fut}{Austronesian}
\define@key{fams}{cdo}{Sino-Tibetan}
\define@key{fams}{pym}{Niger-Congo}
\define@key{fams}{gqa}{Afro-Asiatic}
\define@key{fams}{gbu}{Isolate}
\define@key{fams}{dhg}{Pama-Nyungan}
\define@key{fams}{gdb}{Dravidian}
\define@key{fams}{ged}{Niger-Congo}
\define@key{fams}{gaj}{Trans-New Guinea}
\define@key{fams}{gla}{Indo-European}
\define@key{fams}{gag}{Altaic}
\define@key{fams}{gah}{Trans-New Guinea}
\define@key{fams}{gbi}{North Halmaheran}
\define@key{fams}{glg}{Indo-European}
\define@key{fams}{adl}{Sino-Tibetan}
\define@key{fams}{kld}{Pama-Nyungan}
\define@key{fams}{gmv}{Afro-Asiatic}
\define@key{fams}{pwg}{Austronesian}
\define@key{fams}{grt}{Sino-Tibetan}
\define@key{fams}{wrk}{Garrwan}
\define@key{fams}{gyb}{Trans-New Guinea}
\define@key{fams}{cab}{Arawakan}
\define@key{fams}{gvo}{Tupian}
\define@key{fams}{gay}{Austronesian}
\define@key{fams}{gya}{Niger-Congo}
\define@key{fams}{gso}{Niger-Congo}
\define@key{fams}{gbp}{Niger-Congo}
\define@key{fams}{nlg}{Austronesian}
\define@key{fams}{gqu}{Tai-Kadai}
\define@key{fams}{kat}{Kartvelian}
\define@key{fams}{deu}{Indo-European}
\define@key{fams}{bar}{Indo-European}
\define@key{fams}{ksh}{Indo-European}
\define@key{fams}{wep}{Indo-European}
\define@key{fams}{aaa}{Niger-Congo}
\define@key{fams}{ghl}{Eastern Sudanic}
\define@key{fams}{gih}{Pama-Nyungan}
\define@key{fams}{gid}{Afro-Asiatic}
\define@key{fams}{glk}{Indo-European}
\define@key{fams}{bcq}{Afro-Asiatic}
\define@key{fams}{git}{Tsimshianic}
\define@key{fams}{gis}{Afro-Asiatic}
\define@key{fams}{guc}{Arawakan}
\define@key{fams}{god}{Niger-Congo}
\define@key{fams}{gdo}{Nakh-Daghestanian}
\define@key{fams}{ank}{Afro-Asiatic}
\define@key{fams}{ggw}{Trans-New Guinea}
\define@key{fams}{gju}{Indo-European}
\define@key{fams}{gkn}{Niger-Congo}
\define@key{fams}{gol}{Niger-Congo}
\define@key{fams}{gvf}{Trans-New Guinea}
\define@key{fams}{gno}{Dravidian}
\define@key{fams}{gni}{Bunuban}
\define@key{fams}{gor}{Austronesian}
\define@key{fams}{gow}{Afro-Asiatic}
\define@key{fams}{grj}{Niger-Congo}
\define@key{fams}{ell}{Indo-European}
\define@key{fams}{gss}{other}
\define@key{fams}{kal}{Eskimo-Aleut}
\define@key{fams}{guh}{Guahiban}
\define@key{fams}{gub}{Tupian}
\define@key{fams}{gum}{Barbacoan}
\define@key{fams}{gva}{Mascoian}
\define@key{fams}{gvc}{Tucanoan}
\define@key{fams}{gug}{Tupian}
\define@key{fams}{var}{Uto-Aztecan}
\define@key{fams}{gta}{Isolate}
\define@key{fams}{guo}{Guahiban}
\define@key{fams}{gde}{Afro-Asiatic}
\define@key{fams}{gdf}{Afro-Asiatic}
\define@key{fams}{ktd}{Pama-Nyungan}
\define@key{fams}{ggd}{Pama-Nyungan}
\define@key{fams}{ghs}{Trans-New Guinea}
\define@key{fams}{gcr}{other}
\define@key{fams}{pov}{other}
\define@key{fams}{guj}{Indo-European}
\define@key{fams}{kcm}{Central Sudanic}
\define@key{fams}{glj}{Niger-Congo}
\define@key{fams}{gnn}{Pama-Nyungan}
\define@key{fams}{gvs}{Austronesian}
\define@key{fams}{kgs}{Pama-Nyungan}
\define@key{fams}{guk}{Isolate}
\define@key{fams}{wlg}{Gunwinyguan}
\define@key{fams}{guw}{Niger-Congo}
\define@key{fams}{gww}{Worrorran}
\define@key{fams}{yas}{Niger-Congo}
\define@key{fams}{gyy}{Pama-Nyungan}
\define@key{fams}{guf}{Pama-Nyungan}
\define@key{fams}{gnr}{Pama-Nyungan}
\define@key{fams}{gur}{Niger-Congo}
\define@key{fams}{gue}{Pama-Nyungan}
\define@key{fams}{gux}{Niger-Congo}
\define@key{fams}{goa}{Mande}
\define@key{fams}{gge}{Mangrida}
\define@key{fams}{guz}{Niger-Congo}
\define@key{fams}{gbj}{Austro-Asiatic}
\define@key{fams}{kky}{Pama-Nyungan}
\define@key{fams}{gbr}{Niger-Congo}
\define@key{fams}{kcg}{Niger-Congo}
\define@key{fams}{gaa}{Niger-Congo}
\define@key{fams}{pue}{Chonan}
\define@key{fams}{hts}{Isolate}
\define@key{fams}{hai}{Isolate}
\define@key{fams}{hdn}{Haida}
\define@key{fams}{has}{Wakashan}
\define@key{fams}{hat}{other}
\define@key{fams}{hak}{Sino-Tibetan}
\define@key{fams}{hal}{Austro-Asiatic}
\define@key{fams}{hlb}{Indo-European}
\define@key{fams}{hla}{Austronesian}
\define@key{fams}{amf}{Afro-Asiatic}
\define@key{fams}{hmt}{Trans-New Guinea}
\define@key{fams}{wos}{Sepik}
\define@key{fams}{hni}{Sino-Tibetan}
\define@key{fams}{hnn}{Austronesian}
\define@key{fams}{har}{Afro-Asiatic}
\define@key{fams}{hss}{Afro-Asiatic}
\define@key{fams}{tmd}{Piawi}
\define@key{fams}{had}{Hatim-Mansim}
\define@key{fams}{hau}{Afro-Asiatic}
\define@key{fams}{haw}{Austronesian}
\define@key{fams}{hwc}{other}
\define@key{fams}{hac}{Indo-European}
\define@key{fams}{hay}{Niger-Congo}
\define@key{fams}{vay}{Sino-Tibetan}
\define@key{fams}{xed}{Afro-Asiatic}
\define@key{fams}{heb}{Afro-Asiatic}
\define@key{fams}{heh}{Niger-Congo}
\define@key{fams}{hei}{Wakashan}
\define@key{fams}{hem}{Niger-Congo}
\define@key{fams}{her}{Niger-Congo}
\define@key{fams}{hid}{Siouan}
\define@key{fams}{hil}{Austronesian}
\define@key{fams}{hin}{Indo-European}
\define@key{fams}{gin}{Nakh-Daghestanian}
\define@key{fams}{hix}{Cariban}
\define@key{fams}{lic}{Tai-Kadai}
\define@key{fams}{hmr}{Sino-Tibetan}
\define@key{fams}{mww}{Hmong-Mien}
\define@key{fams}{hnj}{Hmong-Mien}
\define@key{fams}{hoc}{Austro-Asiatic}
\define@key{fams}{hoa}{Austronesian}
\define@key{fams}{hoo}{Niger-Congo}
\define@key{fams}{hks}{other}
\define@key{fams}{hop}{Uto-Aztecan}
\define@key{fams}{hre}{Austro-Asiatic}
\define@key{fams}{ygr}{Trans-New Guinea}
\define@key{fams}{hub}{Jivaroan}
\define@key{fams}{hus}{Mayan}
\define@key{fams}{huv}{Huavean}
\define@key{fams}{hch}{Uto-Aztecan}
\define@key{fams}{hto}{Witotoan}
\define@key{fams}{hux}{Witotoan}
\define@key{fams}{huu}{Witotoan}
\define@key{fams}{hke}{Niger-Congo}
\define@key{fams}{hun}{Uralic}
\define@key{fams}{huz}{Nakh-Daghestanian}
\define@key{fams}{jup}{Nadahup}
\define@key{fams}{hup}{Na-Dene}
\define@key{fams}{csh}{Sino-Tibetan}
\define@key{fams}{ksi}{Skou}
\define@key{fams}{iai}{Austronesian}
\define@key{fams}{ian}{Sepik}
\define@key{fams}{tmu}{Lakes Plain}
\define@key{fams}{iba}{Austronesian}
\define@key{fams}{ibg}{Austronesian}
\define@key{fams}{ibb}{Niger-Congo}
\define@key{fams}{isl}{Indo-European}
\define@key{fams}{icl}{other}
\define@key{fams}{idu}{Niger-Congo}
\define@key{fams}{clk}{Sino-Tibetan}
\define@key{fams}{viv}{Austronesian}
\define@key{fams}{mxe}{Austronesian}
\define@key{fams}{ifb}{Austronesian}
\define@key{fams}{ifm}{Niger-Congo}
\define@key{fams}{ibo}{Niger-Congo}
\define@key{fams}{ige}{Niger-Congo}
\define@key{fams}{ign}{Arawakan}
\define@key{fams}{ihp}{Greater West Bomberai}
\define@key{fams}{ijc}{Ijoid}
\define@key{fams}{ikx}{Eastern Sudanic}
\define@key{fams}{arh}{Chibchan}
\define@key{fams}{ilb}{Niger-Congo}
\define@key{fams}{mia}{Algic}
\define@key{fams}{ilo}{Austronesian}
\define@key{fams}{imn}{Border}
\define@key{fams}{szp}{South Bird's Head}
\define@key{fams}{ins}{other}
\define@key{fams}{pks}{other}
\define@key{fams}{ind}{Austronesian}
\define@key{fams}{pmy}{Austronesian}
\define@key{fams}{inb}{Quechuan}
\define@key{fams}{tbi}{Eastern Sudanic}
\define@key{fams}{inh}{Nakh-Daghestanian}
\define@key{fams}{ynd}{Pama-Nyungan}
\define@key{fams}{ils}{other}
\define@key{fams}{ike}{Eskimo-Aleut}
\define@key{fams}{iqu}{Zaparoan}
\define@key{fams}{irn}{Isolate}
\define@key{fams}{irk}{Afro-Asiatic}
\define@key{fams}{irh}{Austronesian}
\define@key{fams}{gle}{Indo-European}
\define@key{fams}{isg}{other}
\define@key{fams}{its}{Niger-Congo}
\define@key{fams}{isk}{Indo-European}
\define@key{fams}{srl}{Tor-Kwerba}
\define@key{fams}{isd}{Austronesian}
\define@key{fams}{iso}{Niger-Congo}
\define@key{fams}{isr}{other}
\define@key{fams}{ita}{Indo-European}
\define@key{fams}{egl}{Indo-European}
\define@key{fams}{lij}{Indo-European}
\define@key{fams}{nap}{Indo-European}
\define@key{fams}{pms}{Indo-European}
\define@key{fams}{itv}{Austronesian}
\define@key{fams}{itl}{Chukotko-Kamchatkan}
\define@key{fams}{ito}{Isolate}
\define@key{fams}{itz}{Mayan}
\define@key{fams}{ivb}{Austronesian}
\define@key{fams}{ibd}{Iwaidjan}
\define@key{fams}{iwm}{Sepik}
\define@key{fams}{yom}{Niger-Congo}
\define@key{fams}{ixc}{Oto-Manguean}
\define@key{fams}{ixl}{Mayan}
\define@key{fams}{izr}{Niger-Congo}
\define@key{fams}{izh}{Uralic}
\define@key{fams}{izz}{Niger-Congo}
\define@key{fams}{esi}{Eskimo-Aleut}
\define@key{fams}{jbt}{Macro-Ge}
\define@key{fams}{jae}{Austronesian}
\define@key{fams}{jda}{Sino-Tibetan}
\define@key{fams}{jhi}{Austro-Asiatic}
\define@key{fams}{jac}{Mayan}
\define@key{fams}{jam}{other}
\define@key{fams}{djd}{Mirndi}
\define@key{fams}{djm}{Dogon}
\define@key{fams}{jpn}{Isolate}
\define@key{fams}{jru}{Cariban}
\define@key{fams}{jqr}{Aymaran}
\define@key{fams}{anq}{South Andamanese}
\define@key{fams}{jav}{Austronesian}
\define@key{fams}{jeb}{Cahuapanan}
\define@key{fams}{jeh}{Austro-Asiatic}
\define@key{fams}{jek}{Mande}
\define@key{fams}{tow}{Kiowa-Tanoan}
\define@key{fams}{jya}{Sino-Tibetan}
\define@key{fams}{shv}{Afro-Asiatic}
\define@key{fams}{kac}{Sino-Tibetan}
\define@key{fams}{jiu}{Sino-Tibetan}
\define@key{fams}{jiv}{Jivaroan}
\define@key{fams}{rgk}{Sino-Tibetan}
\define@key{fams}{tlo}{Kordofanian}
\define@key{fams}{jun}{Austro-Asiatic}
\define@key{fams}{nst}{Sino-Tibetan}
\define@key{fams}{jbu}{Niger-Congo}
\define@key{fams}{bex}{Central Sudanic}
\define@key{fams}{juc}{Altaic}
\define@key{fams}{jur}{Tupian}
\define@key{fams}{ktz}{Kxa}
\define@key{fams}{jua}{Tupian}
\define@key{fams}{kek}{Mayan}
\define@key{fams}{kbd}{Northwest Caucasian}
\define@key{fams}{xkp}{Indo-European}
\define@key{fams}{kbp}{Niger-Congo}
\define@key{fams}{nbu}{Sino-Tibetan}
\define@key{fams}{kab}{Afro-Asiatic}
\define@key{fams}{xac}{Sino-Tibetan}
\define@key{fams}{kzj}{Austronesian}
\define@key{fams}{kbc}{Guaicuruan}
\define@key{fams}{kdm}{Niger-Congo}
\define@key{fams}{kki}{Niger-Congo}
\define@key{fams}{kct}{Lower Sepik-Ramu}
\define@key{fams}{lew}{Austronesian}
\define@key{fams}{kgp}{Macro-Ge}
\define@key{fams}{kxa}{Austronesian}
\define@key{fams}{kgk}{Tupian}
\define@key{fams}{tbd}{Tate}
\define@key{fams}{mwp}{Pama-Nyungan}
\define@key{fams}{kmh}{Trans-New Guinea}
\define@key{fams}{gwc}{Indo-European}
\define@key{fams}{kck}{Niger-Congo}
\define@key{fams}{kyl}{Kalapuyan}
\define@key{fams}{kls}{Indo-European}
\define@key{fams}{fla}{Salishan}
\define@key{fams}{ktg}{Pama-Nyungan}
\define@key{fams}{bco}{Trans-New Guinea}
\define@key{fams}{kay}{Tupian}
\define@key{fams}{kbq}{Trans-New Guinea}
\define@key{fams}{kms}{Torricelli}
\define@key{fams}{xas}{Uralic}
\define@key{fams}{kam}{Niger-Congo}
\define@key{fams}{xbr}{Austronesian}
\define@key{fams}{kbx}{Lower Sepik-Ramu}
\define@key{fams}{kcu}{Niger-Congo}
\define@key{fams}{kgq}{Asmat-Kamrau Bay}
\define@key{fams}{xmu}{Eastern Daly}
\define@key{fams}{ogo}{Niger-Congo}
\define@key{fams}{kna}{Afro-Asiatic}
\define@key{fams}{xns}{Sino-Tibetan}
\define@key{fams}{kbl}{Saharan}
\define@key{fams}{ikt}{Eskimo-Aleut}
\define@key{fams}{kjb}{Mayan}
\define@key{fams}{knj}{Mayan}
\define@key{fams}{kne}{Austronesian}
\define@key{fams}{kan}{Dravidian}
\define@key{fams}{kxo}{Kapixana}
\define@key{fams}{khd}{Yam}
\define@key{fams}{kcd}{Yam}
\define@key{fams}{knc}{Saharan}
\define@key{fams}{kny}{Niger-Congo}
\define@key{fams}{pam}{Austronesian}
\define@key{fams}{kpg}{Austronesian}
\define@key{fams}{kah}{Central Sudanic}
\define@key{fams}{leu}{Austronesian}
\define@key{fams}{krc}{Altaic}
\define@key{fams}{gbd}{Pama-Nyungan}
\define@key{fams}{kdr}{Altaic}
\define@key{fams}{kpj}{Macro-Ge}
\define@key{fams}{kaa}{Altaic}
\define@key{fams}{zkk}{Isolate}
\define@key{fams}{kyj}{Austronesian}
\define@key{fams}{kpt}{Nakh-Daghestanian}
\define@key{fams}{krl}{Uralic}
\define@key{fams}{bwe}{Sino-Tibetan}
\define@key{fams}{kjp}{Sino-Tibetan}
\define@key{fams}{ksw}{Sino-Tibetan}
\define@key{fams}{vka}{Pama-Nyungan}
\define@key{fams}{kdj}{Eastern Sudanic}
\define@key{fams}{ktn}{Tupian}
\define@key{fams}{yuj}{Pauwasi}
\define@key{fams}{kyh}{Hokan}
\define@key{fams}{arr}{Tupian}
\define@key{fams}{xsm}{Niger-Congo}
\define@key{fams}{kju}{Hokan}
\define@key{fams}{kas}{Indo-European}
\define@key{fams}{csb}{Indo-European}
\define@key{fams}{cog}{Austro-Asiatic}
\define@key{fams}{bqy}{other}
\define@key{fams}{xtc}{Kadu}
\define@key{fams}{bsh}{Indo-European}
\define@key{fams}{kts}{Trans-New Guinea}
\define@key{fams}{kcr}{Kordofanian}
\define@key{fams}{ktw}{Na-Dene}
\define@key{fams}{pss}{Austronesian}
\define@key{fams}{bpp}{Isolate}
\define@key{fams}{zku}{Pama-Nyungan}
\define@key{fams}{xaw}{Uto-Aztecan}
\define@key{fams}{kyz}{Tupian}
\define@key{fams}{eky}{Sino-Tibetan}
\define@key{fams}{kys}{Austronesian}
\define@key{fams}{txu}{Macro-Ge}
\define@key{fams}{gyd}{Tangkic}
\define@key{fams}{gbb}{Pama-Nyungan}
\define@key{fams}{kaz}{Altaic}
\define@key{fams}{ksx}{Austronesian}
\define@key{fams}{kbr}{Afro-Asiatic}
\define@key{fams}{kei}{Austronesian}
\define@key{fams}{kcl}{Austronesian}
\define@key{fams}{kzi}{Austronesian}
\define@key{fams}{sbc}{Austronesian}
\define@key{fams}{ahg}{Afro-Asiatic}
\define@key{fams}{kmt}{Nimboran}
\define@key{fams}{kyq}{Central Sudanic}
\define@key{fams}{keu}{Austronesian}
\define@key{fams}{xki}{other}
\define@key{fams}{ken}{Niger-Congo}
\define@key{fams}{xxk}{Austronesian}
\define@key{fams}{ker}{Afro-Asiatic}
\define@key{fams}{krk}{Chukotko-Kamchatkan}
\define@key{fams}{kee}{Keresan}
\define@key{fams}{ket}{Yeniseian}
\define@key{fams}{xdy}{Austronesian}
\define@key{fams}{kcv}{Niger-Congo}
\define@key{fams}{xte}{Trans-New Guinea}
\define@key{fams}{kew}{Trans-New Guinea}
\define@key{fams}{kjh}{Altaic}
\define@key{fams}{klj}{Altaic}
\define@key{fams}{klr}{Sino-Tibetan}
\define@key{fams}{khk}{Altaic}
\define@key{fams}{kjl}{Sino-Tibetan}
\define@key{fams}{khg}{Sino-Tibetan}
\define@key{fams}{kca}{Uralic}
\define@key{fams}{khr}{Austro-Asiatic}
\define@key{fams}{kha}{Austro-Asiatic}
\define@key{fams}{kjj}{Nakh-Daghestanian}
\define@key{fams}{khm}{Austro-Asiatic}
\define@key{fams}{kjg}{Austro-Asiatic}
\define@key{fams}{khw}{Indo-European}
\define@key{fams}{cnk}{Sino-Tibetan}
\define@key{fams}{khv}{Nakh-Daghestanian}
\define@key{fams}{kkh}{Tai-Kadai}
\define@key{fams}{kic}{Algic}
\define@key{fams}{kik}{Niger-Congo}
\define@key{fams}{hbb}{Afro-Asiatic}
\define@key{fams}{kij}{Austronesian}
\define@key{fams}{klb}{Hokan}
\define@key{fams}{lub}{Niger-Congo}
\define@key{fams}{kig}{Kolopom}
\define@key{fams}{zga}{Niger-Congo}
\define@key{fams}{kfk}{Sino-Tibetan}
\define@key{fams}{kin}{Niger-Congo}
\define@key{fams}{kio}{Kiowa-Tanoan}
\define@key{fams}{kzw}{Kariri}
\define@key{fams}{geb}{Lower Sepik-Ramu}
\define@key{fams}{kir}{Altaic}
\define@key{fams}{gil}{Austronesian}
\define@key{fams}{kiy}{Lakes Plain}
\define@key{fams}{cme}{Niger-Congo}
\define@key{fams}{kje}{Austronesian}
\define@key{fams}{kss}{Niger-Congo}
\define@key{fams}{gia}{Jarrakan}
\define@key{fams}{kii}{Caddoan}
\define@key{fams}{ktu}{other}
\define@key{fams}{kjd}{Trans-New Guinea}
\define@key{fams}{kla}{Penutian}
\define@key{fams}{klu}{Niger-Congo}
\define@key{fams}{yak}{Penutian}
\define@key{fams}{kst}{Niger-Congo}
\define@key{fams}{cku}{Muskogean}
\define@key{fams}{kpw}{Trans-New Guinea}
\define@key{fams}{kfa}{Dravidian}
\define@key{fams}{xwg}{Eastern Sudanic}
\define@key{fams}{xuo}{Niger-Congo}
\define@key{fams}{bcs}{Niger-Congo}
\define@key{fams}{kpx}{Trans-New Guinea}
\define@key{fams}{kbk}{Trans-New Guinea}
\define@key{fams}{kqi}{Trans-New Guinea}
\define@key{fams}{trp}{Sino-Tibetan}
\define@key{fams}{kex}{Indo-European}
\define@key{fams}{kkk}{Austronesian}
\define@key{fams}{kvv}{Austronesian}
\define@key{fams}{kfb}{Dravidian}
\define@key{fams}{kvw}{Greater West Bomberai}
\define@key{fams}{shm}{Indo-European}
\define@key{fams}{bkm}{Niger-Congo}
\define@key{fams}{xbi}{Torricelli}
\define@key{fams}{kge}{Austronesian}
\define@key{fams}{koi}{Uralic}
\define@key{fams}{xom}{Koman}
\define@key{fams}{kfc}{Dravidian}
\define@key{fams}{kng}{Niger-Congo}
\define@key{fams}{kjc}{Austronesian}
\define@key{fams}{knn}{Indo-European}
\define@key{fams}{xon}{Niger-Congo}
\define@key{fams}{mjd}{Penutian}
\define@key{fams}{kma}{Niger-Congo}
\define@key{fams}{kyx}{West Bougainville}
\define@key{fams}{cou}{Niger-Congo}
\define@key{fams}{kqy}{Afro-Asiatic}
\define@key{fams}{kpr}{Trans-New Guinea}
\define@key{fams}{kqz}{Khoe-Kwadi}
\define@key{fams}{knk}{Mande}
\define@key{fams}{kor}{Isolate}
\define@key{fams}{coe}{Tucanoan}
\define@key{fams}{kfq}{Austro-Asiatic}
\define@key{fams}{kfz}{Niger-Congo}
\define@key{fams}{khe}{Trans-New Guinea}
\define@key{fams}{kpy}{Chukotko-Kamchatkan}
\define@key{fams}{kia}{Niger-Congo}
\define@key{fams}{kos}{Austronesian}
\define@key{fams}{kfe}{Dravidian}
\define@key{fams}{aal}{Afro-Asiatic}
\define@key{fams}{kff}{Dravidian}
\define@key{fams}{khq}{Songhay}
\define@key{fams}{ses}{Songhay}
\define@key{fams}{koy}{Na-Dene}
\define@key{fams}{kpk}{Niger-Congo}
\define@key{fams}{xpe}{Mande}
\define@key{fams}{kpo}{Niger-Congo}
\define@key{fams}{xra}{Macro-Ge}
\define@key{fams}{kqq}{Macro-Ge}
\define@key{fams}{krs}{Central Sudanic}
\define@key{fams}{rop}{other}
\define@key{fams}{kgo}{Kadu}
\define@key{fams}{jct}{Altaic}
\define@key{fams}{kry}{Nakh-Daghestanian}
\define@key{fams}{puo}{Austro-Asiatic}
\define@key{fams}{sdm}{Austronesian}
\define@key{fams}{uwa}{Pama-Nyungan}
\define@key{fams}{kxu}{Dravidian}
\define@key{fams}{kvd}{Greater West Bomberai}
\define@key{fams}{kui}{Cariban}
\define@key{fams}{gvn}{Pama-Nyungan}
\define@key{fams}{mbt}{Austronesian}
\define@key{fams}{dwr}{Afro-Asiatic}
\define@key{fams}{kle}{Sino-Tibetan}
\define@key{fams}{kue}{Trans-New Guinea}
\define@key{fams}{kfy}{Indo-European}
\define@key{fams}{kum}{Altaic}
\define@key{fams}{kvn}{Chibchan}
\define@key{fams}{kun}{Isolate}
\define@key{fams}{kup}{Trans-New Guinea}
\define@key{fams}{kjn}{Pama-Nyungan}
\define@key{fams}{cmn}{Sino-Tibetan}
\define@key{fams}{kto}{Isolate}
\define@key{fams}{ckb}{Indo-European}
\define@key{fams}{kmr}{Indo-European}
\define@key{fams}{kru}{Dravidian}
\define@key{fams}{kgg}{Isolate}
\define@key{fams}{vkt}{Austronesian}
\define@key{fams}{gwi}{Na-Dene}
\define@key{fams}{kut}{Isolate}
\define@key{fams}{thd}{Pama-Nyungan}
\define@key{fams}{kuy}{Pama-Nyungan}
\define@key{fams}{kxv}{Dravidian}
\define@key{fams}{kwd}{Austronesian}
\define@key{fams}{kwk}{Wakashan}
\define@key{fams}{tnk}{Austronesian}
\define@key{fams}{ksq}{Afro-Asiatic}
\define@key{fams}{kwn}{Niger-Congo}
\define@key{fams}{xwa}{Isolate}
\define@key{fams}{kwe}{Tor-Kwerba}
\define@key{fams}{kmo}{Sepik}
\define@key{fams}{kwo}{Isolate}
\define@key{fams}{xuu}{Khoe-Kwadi}
\define@key{fams}{kyc}{Trans-New Guinea}
\define@key{fams}{kgy}{Sino-Tibetan}
\define@key{fams}{nuk}{Wakashan}
\define@key{fams}{kmg}{Trans-New Guinea}
\define@key{fams}{gdm}{Isolate}
\define@key{fams}{lbu}{Austronesian}
\define@key{fams}{lac}{Mayan}
\define@key{fams}{lbt}{Tai-Kadai}
\define@key{fams}{lbj}{Sino-Tibetan}
\define@key{fams}{lld}{Indo-European}
\define@key{fams}{lad}{Indo-European}
\define@key{fams}{laf}{Kordofanian}
\define@key{fams}{kot}{Afro-Asiatic}
\define@key{fams}{lha}{Tai-Kadai}
\define@key{fams}{lhu}{Sino-Tibetan}
\define@key{fams}{cnh}{Sino-Tibetan}
\define@key{fams}{lbe}{Nakh-Daghestanian}
\define@key{fams}{lkt}{Siouan}
\define@key{fams}{lbc}{Tai-Kadai}
\define@key{fams}{ywt}{Sino-Tibetan}
\define@key{fams}{slp}{Austronesian}
\define@key{fams}{hia}{Afro-Asiatic}
\define@key{fams}{lmn}{Indo-European}
\define@key{fams}{lam}{Niger-Congo}
\define@key{fams}{lmu}{Austronesian}
\define@key{fams}{lns}{Niger-Congo}
\define@key{fams}{ljp}{Austronesian}
\define@key{fams}{lby}{Pama-Nyungan}
\define@key{fams}{lme}{Afro-Asiatic}
\define@key{fams}{lag}{Niger-Congo}
\define@key{fams}{laj}{Eastern Sudanic}
\define@key{fams}{fsl}{other}
\define@key{fams}{fcs}{other}
\define@key{fams}{lao}{Tai-Kadai}
\define@key{fams}{lrg}{Darwin Region}
\define@key{fams}{lbz}{Tangkic}
\define@key{fams}{alo}{Austronesian}
\define@key{fams}{lav}{Indo-European}
\define@key{fams}{llu}{Austronesian}
\define@key{fams}{law}{Austronesian}
\define@key{fams}{lvk}{Solomons East Papuan}
\define@key{fams}{lzz}{Kartvelian}
\define@key{fams}{agh}{Niger-Congo}
\define@key{fams}{lea}{Niger-Congo}
\define@key{fams}{agb}{Niger-Congo}
\define@key{fams}{lec}{Isolate}
\define@key{fams}{lln}{Afro-Asiatic}
\define@key{fams}{lef}{Niger-Congo}
\define@key{fams}{tnl}{Austronesian}
\define@key{fams}{led}{Central Sudanic}
\define@key{fams}{enx}{Mascoian}
\define@key{fams}{aed}{other}
\define@key{fams}{ssp}{other}
\define@key{fams}{lep}{Sino-Tibetan}
\define@key{fams}{les}{Central Sudanic}
\define@key{fams}{lti}{Austronesian}
\define@key{fams}{lww}{Austronesian}
\define@key{fams}{lez}{Nakh-Daghestanian}
\define@key{fams}{lhm}{Sino-Tibetan}
\define@key{fams}{lil}{Salishan}
\define@key{fams}{lif}{Sino-Tibetan}
\define@key{fams}{lmc}{Darwin Region}
\define@key{fams}{liy}{Niger-Congo}
\define@key{fams}{lin}{Niger-Congo}
\define@key{fams}{ise}{other}
\define@key{fams}{lnj}{Pama-Nyungan}
\define@key{fams}{lis}{Sino-Tibetan}
\define@key{fams}{lit}{Indo-European}
\define@key{fams}{liv}{Uralic}
\define@key{fams}{lob}{Niger-Congo}
\define@key{fams}{log}{Central Sudanic}
\define@key{fams}{lok}{Mande}
\define@key{fams}{arw}{Arawakan}
\define@key{fams}{lom}{Mande}
\define@key{fams}{bdu}{Niger-Congo}
\define@key{fams}{lgu}{Austronesian}
\define@key{fams}{los}{Austronesian}
\define@key{fams}{crc}{Austronesian}
\define@key{fams}{njh}{Sino-Tibetan}
\define@key{fams}{loj}{Austronesian}
\define@key{fams}{lbo}{Austro-Asiatic}
\define@key{fams}{nds}{Indo-European}
\define@key{fams}{loz}{Niger-Congo}
\define@key{fams}{nie}{Niger-Congo}
\define@key{fams}{ojv}{Austronesian}
\define@key{fams}{lch}{Niger-Congo}
\define@key{fams}{lug}{Niger-Congo}
\define@key{fams}{lgg}{Central Sudanic}
\define@key{fams}{jos}{other}
\define@key{fams}{lui}{Uto-Aztecan}
\define@key{fams}{ule}{Isolate}
\define@key{fams}{str}{Salishan}
\define@key{fams}{lnd}{Austronesian}
\define@key{fams}{lun}{Niger-Congo}
\define@key{fams}{luo}{Eastern Sudanic}
\define@key{fams}{lrc}{Indo-European}
\define@key{fams}{lut}{Salishan}
\define@key{fams}{khl}{Austronesian}
\define@key{fams}{lue}{Niger-Congo}
\define@key{fams}{lwo}{Eastern Sudanic}
\define@key{fams}{ltz}{Indo-European}
\define@key{fams}{luy}{Niger-Congo}
\define@key{fams}{lee}{Niger-Congo}
\define@key{fams}{psr}{other}
\define@key{fams}{bzs}{other}
\define@key{fams}{khb}{Tai-Kadai}
\define@key{fams}{msj}{Niger-Congo}
\define@key{fams}{mhy}{Austronesian}
\define@key{fams}{mhi}{Central Sudanic}
\define@key{fams}{slz}{Austronesian}
\define@key{fams}{mdy}{Afro-Asiatic}
\define@key{fams}{mas}{Eastern Sudanic}
\define@key{fams}{mde}{Maban}
\define@key{fams}{mca}{Matacoan}
\define@key{fams}{mbn}{Guahiban}
\define@key{fams}{mkd}{Indo-European}
\define@key{fams}{mcb}{Arawakan}
\define@key{fams}{myy}{Tucanoan}
\define@key{fams}{mbc}{Cariban}
\define@key{fams}{mxu}{Afro-Asiatic}
\define@key{fams}{mda}{Niger-Congo}
\define@key{fams}{dmd}{Pama-Nyungan}
\define@key{fams}{mad}{Austronesian}
\define@key{fams}{mmw}{Austronesian}
\define@key{fams}{mag}{Indo-European}
\define@key{fams}{mgp}{Sino-Tibetan}
\define@key{fams}{mrd}{Sino-Tibetan}
\define@key{fams}{mgu}{Trans-New Guinea}
\define@key{fams}{mdh}{Austronesian}
\define@key{fams}{mhe}{Austro-Asiatic}
\define@key{fams}{xpq}{Algic}
\define@key{fams}{nmu}{Penutian}
\define@key{fams}{zrs}{Mairasic}
\define@key{fams}{mbq}{Austronesian}
\define@key{fams}{mai}{Indo-European}
\define@key{fams}{mpe}{Eastern Sudanic}
\define@key{fams}{mcp}{Niger-Congo}
\define@key{fams}{myh}{Wakashan}
\define@key{fams}{mkz}{Greater West Bomberai}
\define@key{fams}{mak}{Austronesian}
\define@key{fams}{mgf}{Bulaka River}
\define@key{fams}{kde}{Niger-Congo}
\define@key{fams}{mgh}{Niger-Congo}
\define@key{fams}{mcm}{other}
\define@key{fams}{plt}{Austronesian}
\define@key{fams}{mpb}{Northern Daly}
\define@key{fams}{zsm}{Austronesian}
\define@key{fams}{zlm}{Austronesian}
\define@key{fams}{zmi}{Austronesian}
\define@key{fams}{mal}{Dravidian}
\define@key{fams}{mgl}{Austronesian}
\define@key{fams}{gcc}{Baining}
\define@key{fams}{mlt}{Afro-Asiatic}
\define@key{fams}{kmj}{Dravidian}
\define@key{fams}{mam}{Mayan}
\define@key{fams}{mmn}{Austronesian}
\define@key{fams}{mqj}{Austronesian}
\define@key{fams}{mcs}{Niger-Congo}
\define@key{fams}{mgr}{Niger-Congo}
\define@key{fams}{maw}{Niger-Congo}
\define@key{fams}{mdi}{Central Sudanic}
\define@key{fams}{xmm}{Austronesian}
\define@key{fams}{mva}{Austronesian}
\define@key{fams}{mle}{Sepik}
\define@key{fams}{nmm}{Sino-Tibetan}
\define@key{fams}{mnc}{Altaic}
\define@key{fams}{mid}{Afro-Asiatic}
\define@key{fams}{mhq}{Siouan}
\define@key{fams}{mdr}{Austronesian}
\define@key{fams}{mnk}{Mande}
\define@key{fams}{jet}{Border}
\define@key{fams}{mna}{Austronesian}
\define@key{fams}{mpc}{Mangarrayi-Maran}
\define@key{fams}{mdj}{Central Sudanic}
\define@key{fams}{mqy}{Austronesian}
\define@key{fams}{mjg}{Altaic}
\define@key{fams}{mge}{Central Sudanic}
\define@key{fams}{emk}{Mande}
\define@key{fams}{mlq}{Mande}
\define@key{fams}{mfv}{Niger-Congo}
\define@key{fams}{knf}{Niger-Congo}
\define@key{fams}{nge}{Niger-Congo}
\define@key{fams}{mev}{Mande}
\define@key{fams}{mbb}{Austronesian}
\define@key{fams}{mns}{Uralic}
\define@key{fams}{glv}{Indo-European}
\define@key{fams}{mri}{Austronesian}
\define@key{fams}{mcg}{Cariban}
\define@key{fams}{arn}{Araucanian}
\define@key{fams}{mec}{Mangarrayi-Maran}
\define@key{fams}{mrw}{Austronesian}
\define@key{fams}{zmr}{Western Daly}
\define@key{fams}{mar}{Indo-European}
\define@key{fams}{rnp}{Sino-Tibetan}
\define@key{fams}{zmc}{Pama-Nyungan}
\define@key{fams}{mrt}{Afro-Asiatic}
\define@key{fams}{mrj}{Uralic}
\define@key{fams}{mhr}{Uralic}
\define@key{fams}{mrc}{Hokan}
\define@key{fams}{mrz}{Trans-New Guinea}
\define@key{fams}{mbw}{Trans-New Guinea}
\define@key{fams}{zmt}{Western Daly}
\define@key{fams}{mfr}{Western Daly}
\define@key{fams}{mah}{Austronesian}
\define@key{fams}{gcf}{other}
\define@key{fams}{vma}{Pama-Nyungan}
\define@key{fams}{mhx}{Sino-Tibetan}
\define@key{fams}{mcn}{Afro-Asiatic}
\define@key{fams}{jle}{Kordofanian}
\define@key{fams}{mls}{Maban}
\define@key{fams}{wam}{Algic}
\define@key{fams}{mpq}{Pano-Tacanan}
\define@key{fams}{zml}{Eastern Daly}
\define@key{fams}{mcf}{Pano-Tacanan}
\define@key{fams}{mvb}{Na-Dene}
\define@key{fams}{mjk}{Austronesian}
\define@key{fams}{mgw}{Niger-Congo}
\define@key{fams}{mxx}{Mande}
\define@key{fams}{mph}{Iwaidjan}
\define@key{fams}{mfe}{other}
\define@key{fams}{mke}{Indo-European}
\define@key{fams}{mbl}{Macro-Ge}
\define@key{fams}{yan}{Misumalpan}
\define@key{fams}{ayz}{Isolate}
\define@key{fams}{xyj}{Pama-Nyungan}
\define@key{fams}{mfy}{Uto-Aztecan}
\define@key{fams}{mdm}{Niger-Congo}
\define@key{fams}{maz}{Oto-Manguean}
\define@key{fams}{mzn}{Indo-European}
\define@key{fams}{maq}{Oto-Manguean}
\define@key{fams}{mau}{Oto-Manguean}
\define@key{fams}{mfc}{Niger-Congo}
\define@key{fams}{vmb}{Pama-Nyungan}
\define@key{fams}{lnb}{Niger-Congo}
\define@key{fams}{mpk}{Afro-Asiatic}
\define@key{fams}{myb}{Central Sudanic}
\define@key{fams}{mtk}{Niger-Congo}
\define@key{fams}{mdt}{Niger-Congo}
\define@key{fams}{baw}{Niger-Congo}
\define@key{fams}{gmm}{Niger-Congo}
\define@key{fams}{mdq}{Niger-Congo}
\define@key{fams}{mdw}{Niger-Congo}
\define@key{fams}{mhd}{Afro-Asiatic}
\define@key{fams}{mdd}{Niger-Congo}
\define@key{fams}{mym}{Eastern Sudanic}
\define@key{fams}{nux}{Sepik}
\define@key{fams}{gdq}{Afro-Asiatic}
\define@key{fams}{mni}{Sino-Tibetan}
\define@key{fams}{skf}{Tupian}
\define@key{fams}{mek}{Austronesian}
\define@key{fams}{mel}{Austronesian}
\define@key{fams}{bew}{other}
\define@key{fams}{men}{Mande}
\define@key{fams}{mez}{Algic}
\define@key{fams}{mwv}{Austronesian}
\define@key{fams}{sdo}{Austronesian}
\define@key{fams}{mcr}{Trans-New Guinea}
\define@key{fams}{ulk}{Eastern Trans-Fly}
\define@key{fams}{mej}{East Bird's Head}
\define@key{fams}{mpt}{Trans-New Guinea}
\define@key{fams}{crg}{Algic}
\define@key{fams}{mic}{Algic}
\define@key{fams}{mei}{Eastern Sudanic}
\define@key{fams}{ium}{Hmong-Mien}
\define@key{fams}{mmy}{Afro-Asiatic}
\define@key{fams}{mxj}{Sino-Tibetan}
\define@key{fams}{msy}{Lower Sepik-Ramu}
\define@key{fams}{mik}{Muskogean}
\define@key{fams}{mjw}{Sino-Tibetan}
\define@key{fams}{hna}{Afro-Asiatic}
\define@key{fams}{min}{Austronesian}
\define@key{fams}{mvn}{Austronesian}
\define@key{fams}{xmf}{Kartvelian}
\define@key{fams}{mep}{Jarrakan}
\define@key{fams}{nju}{Pama-Nyungan}
\define@key{fams}{mrg}{Sino-Tibetan}
\define@key{fams}{miq}{Misumalpan}
\define@key{fams}{zmq}{Niger-Congo}
\define@key{fams}{csi}{Penutian}
\define@key{fams}{csm}{Penutian}
\define@key{fams}{lmw}{Penutian}
\define@key{fams}{nsq}{Penutian}
\define@key{fams}{pmw}{Penutian}
\define@key{fams}{skd}{Penutian}
\define@key{fams}{mxp}{Mixe-Zoque}
\define@key{fams}{mco}{Mixe-Zoque}
\define@key{fams}{mto}{Mixe-Zoque}
\define@key{fams}{mim}{Oto-Manguean}
\define@key{fams}{mib}{Oto-Manguean}
\define@key{fams}{miy}{Oto-Manguean}
\define@key{fams}{mih}{Oto-Manguean}
\define@key{fams}{miz}{Oto-Manguean}
\define@key{fams}{mxt}{Oto-Manguean}
\define@key{fams}{mio}{Oto-Manguean}
\define@key{fams}{mig}{Oto-Manguean}
\define@key{fams}{mie}{Oto-Manguean}
\define@key{fams}{mil}{Oto-Manguean}
\define@key{fams}{mjc}{Oto-Manguean}
\define@key{fams}{mks}{Oto-Manguean}
\define@key{fams}{mpm}{Oto-Manguean}
\define@key{fams}{mkf}{Afro-Asiatic}
\define@key{fams}{lus}{Sino-Tibetan}
\define@key{fams}{mra}{Austro-Asiatic}
\define@key{fams}{moy}{Afro-Asiatic}
\define@key{fams}{omc}{Isolate}
\define@key{fams}{moc}{Guaicuruan}
\define@key{fams}{mif}{Afro-Asiatic}
\define@key{fams}{mhj}{Altaic}
\define@key{fams}{moh}{Iroquoian}
\define@key{fams}{mov}{Hokan}
\define@key{fams}{mkj}{Austronesian}
\define@key{fams}{moz}{Afro-Asiatic}
\define@key{fams}{mbe}{Penutian}
\define@key{fams}{mso}{Isolate}
\define@key{fams}{fqs}{Baibai-Fas}
\define@key{fams}{mqf}{Trans-New Guinea}
\define@key{fams}{mnw}{Austro-Asiatic}
\define@key{fams}{ndt}{Niger-Congo}
\define@key{fams}{lol}{Niger-Congo}
\define@key{fams}{mog}{Austronesian}
\define@key{fams}{mnz}{Trans-New Guinea}
\define@key{fams}{mnr}{Uto-Aztecan}
\define@key{fams}{mte}{Austronesian}
\define@key{fams}{moe}{Algic}
\define@key{fams}{mxk}{Bogia}
\define@key{fams}{mos}{Niger-Congo}
\define@key{fams}{mop}{Mayan}
\define@key{fams}{mhz}{Austronesian}
\define@key{fams}{mok}{Isolate}
\define@key{fams}{myv}{Uralic}
\define@key{fams}{mdf}{Uralic}
\define@key{fams}{mor}{Kordofanian}
\define@key{fams}{mgd}{Central Sudanic}
\define@key{fams}{cas}{Mosetenan}
\define@key{fams}{meu}{Austronesian}
\define@key{fams}{siw}{South Bougainville}
\define@key{fams}{mzp}{Isolate}
\define@key{fams}{mye}{Niger-Congo}
\define@key{fams}{akc}{Isolate}
\define@key{fams}{dmw}{Pama-Nyungan}
\define@key{fams}{aoj}{Torricelli}
\define@key{fams}{sgw}{Afro-Asiatic}
\define@key{fams}{bmr}{Boran}
\define@key{fams}{chb}{Chibchan}
\define@key{fams}{mlm}{Tai-Kadai}
\define@key{fams}{mzm}{Niger-Congo}
\define@key{fams}{mji}{Hmong-Mien}
\define@key{fams}{mnb}{Austronesian}
\define@key{fams}{mua}{Niger-Congo}
\define@key{fams}{mnf}{Niger-Congo}
\define@key{fams}{myu}{Tupian}
\define@key{fams}{mhk}{Niger-Congo}
\define@key{fams}{umu}{Algic}
\define@key{fams}{moj}{Niger-Congo}
\define@key{fams}{mtq}{Austro-Asiatic}
\define@key{fams}{sur}{Afro-Asiatic}
\define@key{fams}{mtf}{Lower Sepik-Ramu}
\define@key{fams}{mur}{Eastern Sudanic}
\define@key{fams}{mwf}{Southern Daly}
\define@key{fams}{muz}{Eastern Sudanic}
\define@key{fams}{zmu}{Pama-Nyungan}
\define@key{fams}{mug}{Afro-Asiatic}
\define@key{fams}{msu}{Austronesian}
\define@key{fams}{hur}{Salishan}
\define@key{fams}{emi}{Austronesian}
\define@key{fams}{css}{Penutian}
\define@key{fams}{myw}{Austronesian}
\define@key{fams}{mwe}{Niger-Congo}
\define@key{fams}{mlv}{Austronesian}
\define@key{fams}{xak}{Isolate}
\define@key{fams}{bzk}{other}
\define@key{fams}{muh}{Niger-Congo}
\define@key{fams}{naf}{Trans-New Guinea}
\define@key{fams}{wyy}{Austronesian}
\define@key{fams}{mbj}{Nadahup}
\define@key{fams}{nfr}{Niger-Congo}
\define@key{fams}{nbi}{Sino-Tibetan}
\define@key{fams}{nmf}{Sino-Tibetan}
\define@key{fams}{nzm}{Sino-Tibetan}
\define@key{fams}{nag}{other}
\define@key{fams}{nce}{Yale}
\define@key{fams}{nll}{Isolate}
\define@key{fams}{nhn}{Uto-Aztecan}
\define@key{fams}{ncj}{Uto-Aztecan}
\define@key{fams}{nhx}{Uto-Aztecan}
\define@key{fams}{ncl}{Uto-Aztecan}
\define@key{fams}{nhm}{Uto-Aztecan}
\define@key{fams}{nhp}{Uto-Aztecan}
\define@key{fams}{xpo}{Uto-Aztecan}
\define@key{fams}{azz}{Uto-Aztecan}
\define@key{fams}{nhg}{Uto-Aztecan}
\define@key{fams}{ngu}{Uto-Aztecan}
\define@key{fams}{bio}{Kwomtari}
\define@key{fams}{nak}{Austronesian}
\define@key{fams}{nck}{Mangrida}
\define@key{fams}{nal}{Austronesian}
\define@key{fams}{naq}{Khoe-Kwadi}
\define@key{fams}{nmb}{Austronesian}
\define@key{fams}{nab}{Nambikuaran}
\define@key{fams}{nnm}{Sepik}
\define@key{fams}{gld}{Altaic}
\define@key{fams}{ncb}{Austro-Asiatic}
\define@key{fams}{nnb}{Niger-Congo}
\define@key{fams}{niq}{Eastern Sudanic}
\define@key{fams}{sen}{Niger-Congo}
\define@key{fams}{nnk}{Trans-New Guinea}
\define@key{fams}{nnt}{Algic}
\define@key{fams}{tvl}{Austronesian}
\define@key{fams}{npy}{Austronesian}
\define@key{fams}{npa}{Sino-Tibetan}
\define@key{fams}{nrb}{Eastern Sudanic}
\define@key{fams}{nrm}{Austronesian}
\define@key{fams}{nas}{South Bougainville}
\define@key{fams}{nsk}{Algic}
\define@key{fams}{ncz}{Isolate}
\define@key{fams}{ntm}{Niger-Congo}
\define@key{fams}{ntu}{Austronesian}
\define@key{fams}{nau}{Austronesian}
\define@key{fams}{nav}{Na-Dene}
\define@key{fams}{nxq}{Sino-Tibetan}
\define@key{fams}{bud}{Niger-Congo}
\define@key{fams}{nde}{Niger-Congo}
\define@key{fams}{djj}{Mangrida}
\define@key{fams}{ndz}{Niger-Congo}
\define@key{fams}{ndo}{Niger-Congo}
\define@key{fams}{nmd}{Niger-Congo}
\define@key{fams}{ndv}{Niger-Congo}
\define@key{fams}{djk}{other}
\define@key{fams}{dse}{other}
\define@key{fams}{neg}{Altaic}
\define@key{fams}{nsn}{Austronesian}
\define@key{fams}{nee}{Austronesian}
\define@key{fams}{anh}{Trans-New Guinea}
\define@key{fams}{yrk}{Uralic}
\define@key{fams}{nen}{Austronesian}
\define@key{fams}{aij}{Afro-Asiatic}
\define@key{fams}{aii}{Afro-Asiatic}
\define@key{fams}{trg}{Afro-Asiatic}
\define@key{fams}{npi}{Indo-European}
\define@key{fams}{pia}{Uto-Aztecan}
\define@key{fams}{nzs}{other}
\define@key{fams}{new}{Sino-Tibetan}
\define@key{fams}{ney}{Niger-Congo}
\define@key{fams}{nez}{Penutian}
\define@key{fams}{ntj}{Pama-Nyungan}
\define@key{fams}{nxg}{Austronesian}
\define@key{fams}{nig}{Gunwinyguan}
\define@key{fams}{ngk}{Gunwinyguan}
\define@key{fams}{sba}{Central Sudanic}
\define@key{fams}{nam}{Southern Daly}
\define@key{fams}{nio}{Uralic}
\define@key{fams}{nid}{Gunwinyguan}
\define@key{fams}{nay}{Pama-Nyungan}
\define@key{fams}{nrk}{Pama-Nyungan}
\define@key{fams}{nrl}{Pama-Nyungan}
\define@key{fams}{nxn}{Pama-Nyungan}
\define@key{fams}{nbm}{Niger-Congo}
\define@key{fams}{nga}{Niger-Congo}
\define@key{fams}{ngb}{Niger-Congo}
\define@key{fams}{niy}{Central Sudanic}
\define@key{fams}{wyb}{Pama-Nyungan}
\define@key{fams}{ngi}{Afro-Asiatic}
\define@key{fams}{ngo}{Niger-Congo}
\define@key{fams}{llp}{Austronesian}
\define@key{fams}{gym}{Chibchan}
\define@key{fams}{nha}{Pama-Nyungan}
\define@key{fams}{nhr}{Khoe-Kwadi}
\define@key{fams}{nia}{Austronesian}
\define@key{fams}{caq}{Austro-Asiatic}
\define@key{fams}{pcm}{other}
\define@key{fams}{jsl}{other}
\define@key{fams}{nir}{Isolate}
\define@key{fams}{niz}{Torricelli}
\define@key{fams}{nsz}{Penutian}
\define@key{fams}{ncg}{Tsimshianic}
\define@key{fams}{dtd}{Wakashan}
\define@key{fams}{num}{Austronesian}
\define@key{fams}{niu}{Austronesian}
\define@key{fams}{cag}{Matacoan}
\define@key{fams}{niv}{Isolate}
\define@key{fams}{isi}{Niger-Congo}
\define@key{fams}{nko}{Niger-Congo}
\define@key{fams}{cgg}{Niger-Congo}
\define@key{fams}{fia}{Eastern Sudanic}
\define@key{fams}{njb}{Sino-Tibetan}
\define@key{fams}{nog}{Altaic}
\define@key{fams}{not}{Arawakan}
\define@key{fams}{nhu}{Niger-Congo}
\define@key{fams}{snf}{Niger-Congo}
\define@key{fams}{nsl}{other}
\define@key{fams}{nor}{Indo-European}
\define@key{fams}{nse}{Niger-Congo}
\define@key{fams}{nto}{Niger-Congo}
\define@key{fams}{nxl}{Austronesian}
\define@key{fams}{kcn}{other}
\define@key{fams}{dgl}{Eastern Sudanic}
\define@key{fams}{xnz}{Eastern Sudanic}
\define@key{fams}{nus}{Eastern Sudanic}
\define@key{fams}{mbr}{Cacua-Nukak}
\define@key{fams}{nkr}{Austronesian}
\define@key{fams}{nut}{Tai-Kadai}
\define@key{fams}{nuy}{Gunwinyguan}
\define@key{fams}{nuv}{Niger-Congo}
\define@key{fams}{iii}{Sino-Tibetan}
\define@key{fams}{nup}{Niger-Congo}
\define@key{fams}{nuf}{Sino-Tibetan}
\define@key{fams}{cbn}{Austro-Asiatic}
\define@key{fams}{nly}{Pama-Nyungan}
\define@key{fams}{now}{Niger-Congo}
\define@key{fams}{tpq}{Sino-Tibetan}
\define@key{fams}{nym}{Niger-Congo}
\define@key{fams}{nyj}{Niger-Congo}
\define@key{fams}{nyp}{Eastern Sudanic}
\define@key{fams}{nna}{Pama-Nyungan}
\define@key{fams}{nyt}{Pama-Nyungan}
\define@key{fams}{yly}{Austronesian}
\define@key{fams}{nyh}{Nyulnyulan}
\define@key{fams}{nih}{Niger-Congo}
\define@key{fams}{nyi}{Eastern Sudanic}
\define@key{fams}{njz}{Sino-Tibetan}
\define@key{fams}{nyv}{Nyulnyulan}
\define@key{fams}{nys}{Pama-Nyungan}
\define@key{fams}{nzk}{Niger-Congo}
\define@key{fams}{ood}{Uto-Aztecan}
\define@key{fams}{afz}{Lakes Plain}
\define@key{fams}{ann}{Niger-Congo}
\define@key{fams}{oca}{Witotoan}
\define@key{fams}{oci}{Indo-European}
\define@key{fams}{ocu}{Oto-Manguean}
\define@key{fams}{ogb}{Niger-Congo}
\define@key{fams}{ogu}{Niger-Congo}
\define@key{fams}{oyb}{Austro-Asiatic}
\define@key{fams}{xal}{Altaic}
\define@key{fams}{ojs}{Algic}
\define@key{fams}{ciw}{Algic}
\define@key{fams}{oka}{Salishan}
\define@key{fams}{opm}{Trans-New Guinea}
\define@key{fams}{oku}{Niger-Congo}
\define@key{fams}{ong}{Torricelli}
\define@key{fams}{plo}{Mixe-Zoque}
\define@key{fams}{omg}{Tupian}
\define@key{fams}{oma}{Siouan}
\define@key{fams}{aun}{Torricelli}
\define@key{fams}{one}{Iroquoian}
\define@key{fams}{oon}{South Andamanese}
\define@key{fams}{ons}{Trans-New Guinea}
\define@key{fams}{ono}{Iroquoian}
\define@key{fams}{mvf}{Altaic}
\define@key{fams}{ore}{Tucanoan}
\define@key{fams}{tag}{Kordofanian}
\define@key{fams}{ory}{Indo-European}
\define@key{fams}{ort}{Indo-European}
\define@key{fams}{oru}{Indo-European}
\define@key{fams}{oac}{Altaic}
\define@key{fams}{oaa}{Altaic}
\define@key{fams}{okv}{Trans-New Guinea}
\define@key{fams}{oro}{Eleman}
\define@key{fams}{gax}{Afro-Asiatic}
\define@key{fams}{hae}{Afro-Asiatic}
\define@key{fams}{ssn}{Afro-Asiatic}
\define@key{fams}{gaz}{Afro-Asiatic}
\define@key{fams}{ury}{Tor-Kwerba}
\define@key{fams}{osa}{Siouan}
\define@key{fams}{oss}{Indo-European}
\define@key{fams}{iow}{Siouan}
\define@key{fams}{otz}{Oto-Manguean}
\define@key{fams}{ote}{Oto-Manguean}
\define@key{fams}{otq}{Oto-Manguean}
\define@key{fams}{otm}{Oto-Manguean}
\define@key{fams}{otr}{Kordofanian}
\define@key{fams}{owi}{Left May}
\define@key{fams}{pqa}{Afro-Asiatic}
\define@key{fams}{drl}{Pama-Nyungan}
\define@key{fams}{pma}{Austronesian}
\define@key{fams}{pac}{Austro-Asiatic}
\define@key{fams}{pdo}{Austronesian}
\define@key{fams}{pgu}{North Halmaheran}
\define@key{fams}{duf}{Austronesian}
\define@key{fams}{pck}{Sino-Tibetan}
\define@key{fams}{pao}{Uto-Aztecan}
\define@key{fams}{pwn}{Austronesian}
\define@key{fams}{pkn}{Pama-Nyungan}
\define@key{fams}{pau}{Austronesian}
\define@key{fams}{pll}{Austro-Asiatic}
\define@key{fams}{plu}{Arawakan}
\define@key{fams}{fap}{Niger-Congo}
\define@key{fams}{nad}{Pama-Nyungan}
\define@key{fams}{pmz}{Oto-Manguean}
\define@key{fams}{pmf}{Austronesian}
\define@key{fams}{pbh}{Cariban}
\define@key{fams}{kre}{Macro-Ge}
\define@key{fams}{pag}{Austronesian}
\define@key{fams}{pbr}{Niger-Congo}
\define@key{fams}{pan}{Indo-European}
\define@key{fams}{pnw}{Pama-Nyungan}
\define@key{fams}{pap}{other}
\define@key{fams}{prk}{Austro-Asiatic}
\define@key{fams}{asa}{Niger-Congo}
\define@key{fams}{pab}{Arawakan}
\define@key{fams}{pci}{Dravidian}
\define@key{fams}{pst}{Indo-European}
\define@key{fams}{pqm}{Algic}
\define@key{fams}{ptp}{Austronesian}
\define@key{fams}{gfk}{Austronesian}
\define@key{fams}{lae}{Sino-Tibetan}
\define@key{fams}{pwi}{Penutian}
\define@key{fams}{plh}{Austronesian}
\define@key{fams}{pad}{Arauan}
\define@key{fams}{pwa}{Teberan-Pawaian}
\define@key{fams}{paw}{Caddoan}
\define@key{fams}{pay}{Chibchan}
\define@key{fams}{aoc}{Cariban}
\define@key{fams}{peg}{Dravidian}
\define@key{fams}{pip}{Afro-Asiatic}
\define@key{fams}{pes}{Indo-European}
\define@key{fams}{pww}{Sino-Tibetan}
\define@key{fams}{pio}{Arawakan}
\define@key{fams}{pid}{Sáliban}
\define@key{fams}{plg}{Guaicuruan}
\define@key{fams}{piv}{Austronesian}
\define@key{fams}{pif}{Austronesian}
\define@key{fams}{piu}{Pama-Nyungan}
\define@key{fams}{ppl}{Uto-Aztecan}
\define@key{fams}{myp}{Mura}
\define@key{fams}{pir}{Tucanoan}
\define@key{fams}{pib}{Arawakan}
\define@key{fams}{psa}{Trans-New Guinea}
\define@key{fams}{pjt}{Pama-Nyungan}
\define@key{fams}{pit}{Pama-Nyungan}
\define@key{fams}{psd}{other}
\define@key{fams}{gob}{Guahiban}
\define@key{fams}{fwa}{Austronesian}
\define@key{fams}{pbi}{Afro-Asiatic}
\define@key{fams}{poy}{Niger-Congo}
\define@key{fams}{pon}{Austronesian}
\define@key{fams}{rwa}{Skou}
\define@key{fams}{poh}{Mayan}
\define@key{fams}{pko}{Eastern Sudanic}
\define@key{fams}{pox}{Indo-European}
\define@key{fams}{pol}{Indo-European}
\define@key{fams}{poo}{Hokan}
\define@key{fams}{peb}{Hokan}
\define@key{fams}{pej}{Hokan}
\define@key{fams}{pom}{Hokan}
\define@key{fams}{pbe}{Oto-Manguean}
\define@key{fams}{poe}{Oto-Manguean}
\define@key{fams}{pbf}{Oto-Manguean}
\define@key{fams}{poi}{Mixe-Zoque}
\define@key{fams}{poc}{Mayan}
\define@key{fams}{psw}{Austronesian}
\define@key{fams}{por}{Indo-European}
\define@key{fams}{pot}{Algic}
\define@key{fams}{pim}{Algic}
\define@key{fams}{prn}{Indo-European}
\define@key{fams}{pre}{other}
\define@key{fams}{pui}{Isolate}
\define@key{fams}{fuc}{Niger-Congo}
\define@key{fams}{nij}{Austronesian}
\define@key{fams}{puw}{Austronesian}
\define@key{fams}{pmi}{Sino-Tibetan}
\define@key{fams}{puq}{Isolate}
\define@key{fams}{prx}{Sino-Tibetan}
\define@key{fams}{tsz}{Tarascan}
\define@key{fams}{pbb}{Páezan}
\define@key{fams}{lkr}{Eastern Sudanic}
\define@key{fams}{aar}{Afro-Asiatic}
\define@key{fams}{byx}{Baining}
\define@key{fams}{alc}{Alacalufan}
\define@key{fams}{yum}{Hokan}
\define@key{fams}{qxa}{Quechuan}
\define@key{fams}{quy}{Quechuan}
\define@key{fams}{qvc}{Quechuan}
\define@key{fams}{quh}{Quechuan}
\define@key{fams}{quz}{Quechuan}
\define@key{fams}{qug}{Quechuan}
\define@key{fams}{qub}{Quechuan}
\define@key{fams}{qvi}{Quechuan}
\define@key{fams}{qvn}{Quechuan}
\define@key{fams}{quc}{Mayan}
\define@key{fams}{qui}{Chimakuan}
\define@key{fams}{rad}{Austronesian}
\define@key{fams}{lml}{Austronesian}
\define@key{fams}{rji}{Sino-Tibetan}
\define@key{fams}{ral}{Sino-Tibetan}
\define@key{fams}{rma}{Chibchan}
\define@key{fams}{bod}{Sino-Tibetan}
\define@key{fams}{rao}{Lower Sepik-Ramu}
\define@key{fams}{rap}{Austronesian}
\define@key{fams}{ras}{Kordofanian}
\define@key{fams}{rwo}{Trans-New Guinea}
\define@key{fams}{raw}{Sino-Tibetan}
\define@key{fams}{rej}{Austronesian}
\define@key{fams}{rmb}{Gunwinyguan}
\define@key{fams}{bfw}{Austro-Asiatic}
\define@key{fams}{rel}{Afro-Asiatic}
\define@key{fams}{ren}{Austro-Asiatic}
\define@key{fams}{mnv}{Austronesian}
\define@key{fams}{rgr}{Arawakan}
\define@key{fams}{tnc}{Tucanoan}
\define@key{fams}{ran}{Kolopom}
\define@key{fams}{rkb}{Macro-Ge}
\define@key{fams}{rim}{Niger-Congo}
\define@key{fams}{rit}{Pama-Nyungan}
\define@key{fams}{rog}{Austronesian}
\define@key{fams}{rmn}{Indo-European}
\define@key{fams}{rmo}{Indo-European}
\define@key{fams}{rmy}{Indo-European}
\define@key{fams}{rml}{Indo-European}
\define@key{fams}{rmw}{Indo-European}
\define@key{fams}{ron}{Indo-European}
\define@key{fams}{roh}{Indo-European}
\define@key{fams}{cla}{Afro-Asiatic}
\define@key{fams}{rng}{Niger-Congo}
\define@key{fams}{rro}{Austronesian}
\define@key{fams}{twu}{Austronesian}
\define@key{fams}{roo}{West Bougainville}
\define@key{fams}{rtm}{Austronesian}
\define@key{fams}{rug}{Austronesian}
\define@key{fams}{dru}{Austronesian}
\define@key{fams}{klq}{Trans-New Guinea}
\define@key{fams}{run}{Niger-Congo}
\define@key{fams}{rou}{Maban}
\define@key{fams}{nyn}{Niger-Congo}
\define@key{fams}{nyo}{Niger-Congo}
\define@key{fams}{rus}{Indo-European}
\define@key{fams}{rsl}{other}
\define@key{fams}{rut}{Nakh-Daghestanian}
\define@key{fams}{apb}{Austronesian}
\define@key{fams}{snv}{Austronesian}
\define@key{fams}{sma}{Uralic}
\define@key{fams}{sjd}{Uralic}
\define@key{fams}{sme}{Uralic}
\define@key{fams}{skb}{Tai-Kadai}
\define@key{fams}{uma}{Penutian}
\define@key{fams}{ssy}{Afro-Asiatic}
\define@key{fams}{saj}{North Halmaheran}
\define@key{fams}{sku}{Austronesian}
\define@key{fams}{slr}{Altaic}
\define@key{fams}{sbe}{Austronesian}
\define@key{fams}{sln}{Hokan}
\define@key{fams}{slh}{Salishan}
\define@key{fams}{sll}{Trans-New Guinea}
\define@key{fams}{sse}{Austronesian}
\define@key{fams}{ssb}{Austronesian}
\define@key{fams}{ndi}{Niger-Congo}
\define@key{fams}{smq}{Trans-New Guinea}
\define@key{fams}{smo}{Austronesian}
\define@key{fams}{sad}{Isolate}
\define@key{fams}{sxn}{Austronesian}
\define@key{fams}{sag}{Niger-Congo}
\define@key{fams}{snq}{Niger-Congo}
\define@key{fams}{sce}{Altaic}
\define@key{fams}{sat}{Austro-Asiatic}
\define@key{fams}{xsu}{Yanomam}
\define@key{fams}{spu}{Austro-Asiatic}
\define@key{fams}{srm}{other}
\define@key{fams}{srs}{Na-Dene}
\define@key{fams}{sro}{Indo-European}
\define@key{fams}{dju}{Sepik}
\define@key{fams}{ybe}{Altaic}
\define@key{fams}{sdg}{Indo-European}
\define@key{fams}{svs}{Solomons East Papuan}
\define@key{fams}{szw}{Austronesian}
\define@key{fams}{hvn}{Austronesian}
\define@key{fams}{pos}{Mixe-Zoque}
\define@key{fams}{kpz}{Eastern Sudanic}
\define@key{fams}{sey}{Tucanoan}
\define@key{fams}{sed}{Austro-Asiatic}
\define@key{fams}{trv}{Austronesian}
\define@key{fams}{slu}{Austronesian}
\define@key{fams}{sly}{Austronesian}
\define@key{fams}{spl}{Trans-New Guinea}
\define@key{fams}{ona}{Chonan}
\define@key{fams}{sel}{Uralic}
\define@key{fams}{nsm}{Sino-Tibetan}
\define@key{fams}{sea}{Austro-Asiatic}
\define@key{fams}{sif}{Niger-Congo}
\define@key{fams}{sza}{Austro-Asiatic}
\define@key{fams}{seh}{Niger-Congo}
\define@key{fams}{sef}{Niger-Congo}
\define@key{fams}{see}{Iroquoian}
\define@key{fams}{szg}{Niger-Congo}
\define@key{fams}{set}{Isolate}
\define@key{fams}{hbs}{Indo-European}
\define@key{fams}{sei}{Hokan}
\define@key{fams}{ser}{Uto-Aztecan}
\define@key{fams}{sot}{Niger-Congo}
\define@key{fams}{crs}{other}
\define@key{fams}{sbf}{Isolate}
\define@key{fams}{ksb}{Niger-Congo}
\define@key{fams}{shn}{Tai-Kadai}
\define@key{fams}{mcd}{Pano-Tacanan}
\define@key{fams}{sht}{Hokan}
\define@key{fams}{shj}{Eastern Sudanic}
\define@key{fams}{sjw}{Algic}
\define@key{fams}{swv}{Indo-European}
\define@key{fams}{sdp}{Sino-Tibetan}
\define@key{fams}{xsr}{Sino-Tibetan}
\define@key{fams}{shk}{Eastern Sudanic}
\define@key{fams}{scl}{Indo-European}
\define@key{fams}{bwo}{Afro-Asiatic}
\define@key{fams}{shp}{Pano-Tacanan}
\define@key{fams}{yuy}{Altaic}
\define@key{fams}{shb}{Yanomam}
\define@key{fams}{sii}{Isolate}
\define@key{fams}{sna}{Niger-Congo}
\define@key{fams}{cjs}{Altaic}
\define@key{fams}{shh}{Uto-Aztecan}
\define@key{fams}{sgh}{Indo-European}
\define@key{fams}{ryu}{Japanese}
\define@key{fams}{shs}{Salishan}
\define@key{fams}{snp}{Trans-New Guinea}
\define@key{fams}{sjr}{Austronesian}
\define@key{fams}{sid}{Afro-Asiatic}
\define@key{fams}{ski}{Austronesian}
\define@key{fams}{tty}{Lakes Plain}
\define@key{fams}{sip}{Sino-Tibetan}
\define@key{fams}{skh}{Austronesian}
\define@key{fams}{dau}{Eastern Sudanic}
\define@key{fams}{smr}{Austronesian}
\define@key{fams}{snc}{Austronesian}
\define@key{fams}{snd}{Indo-European}
\define@key{fams}{sin}{Indo-European}
\define@key{fams}{xsi}{Austronesian}
\define@key{fams}{snn}{Tucanoan}
\define@key{fams}{qum}{Mayan}
\define@key{fams}{fos}{Austronesian}
\define@key{fams}{sri}{Tucanoan}
\define@key{fams}{srq}{Tupian}
\define@key{fams}{ssd}{Trans-New Guinea}
\define@key{fams}{sil}{Niger-Congo}
\define@key{fams}{baa}{Austronesian}
\define@key{fams}{sis}{Oregon Coast}
\define@key{fams}{skv}{Skou}
\define@key{fams}{den}{Na-Dene}
\define@key{fams}{xsl}{Na-Dene}
\define@key{fams}{slk}{Indo-European}
\define@key{fams}{slv}{Indo-European}
\define@key{fams}{teu}{Eastern Sudanic}
\define@key{fams}{sob}{Austronesian}
\define@key{fams}{gru}{Afro-Asiatic}
\define@key{fams}{evn}{Altaic}
\define@key{fams}{som}{Afro-Asiatic}
\define@key{fams}{sop}{Niger-Congo}
\define@key{fams}{snk}{Mande}
\define@key{fams}{sov}{Austronesian}
\define@key{fams}{sqt}{Afro-Asiatic}
\define@key{fams}{srb}{Austro-Asiatic}
\define@key{fams}{dsb}{Indo-European}
\define@key{fams}{hsb}{Indo-European}
\define@key{fams}{nso}{Niger-Congo}
\define@key{fams}{mnx}{East Bird's Head}
\define@key{fams}{kvk}{other}
\define@key{fams}{tvk}{Austronesian}
\define@key{fams}{wib}{Niger-Congo}
\define@key{fams}{spa}{Indo-European}
\define@key{fams}{spt}{Sino-Tibetan}
\define@key{fams}{spo}{Salishan}
\define@key{fams}{squ}{Salishan}
\define@key{fams}{srn}{other}
\define@key{fams}{kpm}{Austro-Asiatic}
\define@key{fams}{sto}{Siouan}
\define@key{fams}{sbs}{Niger-Congo}
\define@key{fams}{tgo}{Austronesian}
\define@key{fams}{sue}{Trans-New Guinea}
\define@key{fams}{swi}{Tai-Kadai}
\define@key{fams}{sui}{Trans-New Guinea}
\define@key{fams}{sub}{Niger-Congo}
\define@key{fams}{suk}{Niger-Congo}
\define@key{fams}{sua}{Isolate}
\define@key{fams}{suv}{Isolate}
\define@key{fams}{sun}{Austronesian}
\define@key{fams}{sjg}{Eastern Sudanic}
\define@key{fams}{spp}{Niger-Congo}
\define@key{fams}{sgz}{Austronesian}
\define@key{fams}{sus}{Mande}
\define@key{fams}{sva}{Kartvelian}
\define@key{fams}{swl}{other}
\define@key{fams}{swh}{Niger-Congo}
\define@key{fams}{ssw}{Niger-Congo}
\define@key{fams}{swe}{Indo-European}
\define@key{fams}{slc}{Sáliban}
\define@key{fams}{mky}{Austronesian}
\define@key{fams}{sst}{Trans-New Guinea}
\define@key{fams}{tby}{North Halmaheran}
\define@key{fams}{tab}{Nakh-Daghestanian}
\define@key{fams}{tnm}{Sentani}
\define@key{fams}{tap}{Niger-Congo}
\define@key{fams}{tna}{Pano-Tacanan}
\define@key{fams}{tgl}{Austronesian}
\define@key{fams}{tbw}{Austronesian}
\define@key{fams}{tah}{Austronesian}
\define@key{fams}{gpn}{Gapun}
\define@key{fams}{sps}{Austronesian}
\define@key{fams}{tbg}{Trans-New Guinea}
\define@key{fams}{tss}{other}
\define@key{fams}{tgk}{Indo-European}
\define@key{fams}{tkm}{Isolate}
\define@key{fams}{tbc}{Austronesian}
\define@key{fams}{tld}{Austronesian}
\define@key{fams}{tlj}{Niger-Congo}
\define@key{fams}{tly}{Indo-European}
\define@key{fams}{tma}{Eastern Sudanic}
\define@key{fams}{mla}{Austronesian}
\define@key{fams}{tcg}{Kayagar}
\define@key{fams}{taj}{Sino-Tibetan}
\define@key{fams}{taq}{Afro-Asiatic}
\define@key{fams}{tam}{Dravidian}
\define@key{fams}{tpm}{Niger-Congo}
\define@key{fams}{tcb}{Na-Dene}
\define@key{fams}{tfn}{Na-Dene}
\define@key{fams}{taa}{Na-Dene}
\define@key{fams}{tan}{Afro-Asiatic}
\define@key{fams}{skj}{Sino-Tibetan}
\define@key{fams}{tgg}{Austronesian}
\define@key{fams}{tpg}{Greater West Bomberai}
\define@key{fams}{nwi}{Austronesian}
\define@key{fams}{tza}{other}
\define@key{fams}{tpj}{Tupian}
\define@key{fams}{tar}{Uto-Aztecan}
\define@key{fams}{tac}{Uto-Aztecan}
\define@key{fams}{txn}{Austronesian}
\define@key{fams}{tro}{Sino-Tibetan}
\define@key{fams}{tae}{Arawakan}
\define@key{fams}{yer}{Niger-Congo}
\define@key{fams}{shi}{Afro-Asiatic}
\define@key{fams}{ttt}{Indo-European}
\define@key{fams}{txx}{Austronesian}
\define@key{fams}{tat}{Altaic}
\define@key{fams}{tks}{Indo-European}
\define@key{fams}{tav}{Tucanoan}
\define@key{fams}{tuh}{Isolate}
\define@key{fams}{trr}{Isolate}
\define@key{fams}{tsg}{Austronesian}
\define@key{fams}{tya}{Trans-New Guinea}
\define@key{fams}{tbo}{Austronesian}
\define@key{fams}{cks}{other}
\define@key{fams}{tbl}{Austronesian}
\define@key{fams}{ttc}{Mayan}
\define@key{fams}{kps}{West Bird's Head}
\define@key{fams}{teh}{Chonan}
\define@key{fams}{kkw}{Niger-Congo}
\define@key{fams}{tlf}{Trans-New Guinea}
\define@key{fams}{tel}{Dravidian}
\define@key{fams}{kdh}{Niger-Congo}
\define@key{fams}{teq}{Eastern Sudanic}
\define@key{fams}{tea}{Austro-Asiatic}
\define@key{fams}{tem}{Niger-Congo}
\define@key{fams}{tex}{Eastern Sudanic}
\define@key{fams}{kza}{Niger-Congo}
\define@key{fams}{tio}{Austronesian}
\define@key{fams}{tep}{Uto-Aztecan}
\define@key{fams}{tee}{Totonacan}
\define@key{fams}{tpt}{Totonacan}
\define@key{fams}{ntp}{Uto-Aztecan}
\define@key{fams}{stp}{Uto-Aztecan}
\define@key{fams}{ttr}{Afro-Asiatic}
\define@key{fams}{tfr}{Chibchan}
\define@key{fams}{tft}{North Halmaheran}
\define@key{fams}{ter}{Arawakan}
\define@key{fams}{teo}{Eastern Sudanic}
\define@key{fams}{tll}{Niger-Congo}
\define@key{fams}{tet}{Austronesian}
\define@key{fams}{tew}{Kiowa-Tanoan}
\define@key{fams}{tcz}{Sino-Tibetan}
\define@key{fams}{tha}{Tai-Kadai}
\define@key{fams}{tsq}{other}
\define@key{fams}{ths}{Sino-Tibetan}
\define@key{fams}{thf}{Sino-Tibetan}
\define@key{fams}{ssf}{Austronesian}
\define@key{fams}{typ}{Pama-Nyungan}
\define@key{fams}{thp}{Salishan}
\define@key{fams}{tdh}{Sino-Tibetan}
\define@key{fams}{tca}{Isolate}
\define@key{fams}{tvo}{North Halmaheran}
\define@key{fams}{tif}{Trans-New Guinea}
\define@key{fams}{tgc}{Austronesian}
\define@key{fams}{tir}{Afro-Asiatic}
\define@key{fams}{tig}{Afro-Asiatic}
\define@key{fams}{dih}{Hokan}
\define@key{fams}{tik}{Niger-Congo}
\define@key{fams}{til}{Salishan}
\define@key{fams}{tms}{Kordofanian}
\define@key{fams}{aoz}{Austronesian}
\define@key{fams}{tjm}{Isolate}
\define@key{fams}{tih}{Austronesian}
\define@key{fams}{lbf}{Sino-Tibetan}
\define@key{fams}{tin}{Nakh-Daghestanian}
\define@key{fams}{cir}{Austronesian}
\define@key{fams}{tri}{Cariban}
\define@key{fams}{tiy}{Austronesian}
\define@key{fams}{tiv}{Niger-Congo}
\define@key{fams}{twf}{Kiowa-Tanoan}
\define@key{fams}{tix}{Kiowa-Tanoan}
\define@key{fams}{tiw}{Tiwian}
\define@key{fams}{tcf}{Oto-Manguean}
\define@key{fams}{tli}{Na-Dene}
\define@key{fams}{tqo}{Eleman}
\define@key{fams}{tob}{Guaicuruan}
\define@key{fams}{tti}{Austronesian}
\define@key{fams}{tlb}{North Halmaheran}
\define@key{fams}{sbu}{Sino-Tibetan}
\define@key{fams}{tcx}{Dravidian}
\define@key{fams}{kim}{Altaic}
\define@key{fams}{toj}{Mayan}
\define@key{fams}{tpi}{other}
\define@key{fams}{tkl}{Austronesian}
\define@key{fams}{jic}{Isolate}
\define@key{fams}{ksd}{Austronesian}
\define@key{fams}{dto}{Dogon}
\define@key{fams}{tdn}{Austronesian}
\define@key{fams}{toi}{Niger-Congo}
\define@key{fams}{ton}{Austronesian}
\define@key{fams}{tqw}{Isolate}
\define@key{fams}{tnt}{Austronesian}
\define@key{fams}{mlu}{Austronesian}
\define@key{fams}{sda}{Austronesian}
\define@key{fams}{rth}{Austronesian}
\define@key{fams}{dts}{Dogon}
\define@key{fams}{trw}{Indo-European}
\define@key{fams}{tlc}{Totonacan}
\define@key{fams}{top}{Totonacan}
\define@key{fams}{tos}{Totonacan}
\define@key{fams}{too}{Totonacan}
\define@key{fams}{trs}{Oto-Manguean}
\define@key{fams}{trc}{Oto-Manguean}
\define@key{fams}{tpy}{Isolate}
\define@key{fams}{cof}{Barbacoan}
\define@key{fams}{tkr}{Nakh-Daghestanian}
\define@key{fams}{huq}{Austronesian}
\define@key{fams}{ddo}{Nakh-Daghestanian}
\define@key{fams}{tsj}{Sino-Tibetan}
\define@key{fams}{tsi}{Tsimshianic}
\define@key{fams}{tsv}{Niger-Congo}
\define@key{fams}{tso}{Niger-Congo}
\define@key{fams}{tsu}{Austronesian}
\define@key{fams}{bbl}{Nakh-Daghestanian}
\define@key{fams}{tsn}{Niger-Congo}
\define@key{fams}{pmt}{Austronesian}
\define@key{fams}{thz}{Afro-Asiatic}
\define@key{fams}{thv}{Afro-Asiatic}
\define@key{fams}{tbu}{Uto-Aztecan}
\define@key{fams}{tuo}{Tucanoan}
\define@key{fams}{tzn}{Austronesian}
\define@key{fams}{bag}{Niger-Congo}
\define@key{fams}{tcy}{Dravidian}
\define@key{fams}{tmc}{Afro-Asiatic}
\define@key{fams}{tmq}{Austronesian}
\define@key{fams}{tuf}{Chibchan}
\define@key{fams}{tvu}{Niger-Congo}
\define@key{fams}{lcm}{Austronesian}
\define@key{fams}{tun}{Isolate}
\define@key{fams}{tpn}{Tupian}
\define@key{fams}{tui}{Niger-Congo}
\define@key{fams}{tuv}{Eastern Sudanic}
\define@key{fams}{kmz}{Altaic}
\define@key{fams}{tur}{Altaic}
\define@key{fams}{tuk}{Altaic}
\define@key{fams}{tus}{Iroquoian}
\define@key{fams}{ttm}{Na-Dene}
\define@key{fams}{tta}{Siouan}
\define@key{fams}{tvt}{Sino-Tibetan}
\define@key{fams}{tyv}{Altaic}
\define@key{fams}{tue}{Tucanoan}
\define@key{fams}{twa}{Salishan}
\define@key{fams}{woa}{Northern Daly}
\define@key{fams}{tzh}{Mayan}
\define@key{fams}{tzo}{Mayan}
\define@key{fams}{tzj}{Mayan}
\define@key{fams}{tub}{Uto-Aztecan}
\define@key{fams}{par}{Uto-Aztecan}
\define@key{fams}{tsm}{other}
\define@key{fams}{umb}{Niger-Congo}
\define@key{fams}{uby}{Northwest Caucasian}
\define@key{fams}{udi}{Nakh-Daghestanian}
\define@key{fams}{ude}{Altaic}
\define@key{fams}{udm}{Uralic}
\define@key{fams}{ugn}{other}
\define@key{fams}{ukr}{Indo-European}
\define@key{fams}{ulc}{Altaic}
\define@key{fams}{udl}{Afro-Asiatic}
\define@key{fams}{uli}{Austronesian}
\define@key{fams}{ppk}{Austronesian}
\define@key{fams}{cbd}{Cariban}
\define@key{fams}{ubu}{Trans-New Guinea}
\define@key{fams}{ump}{Pama-Nyungan}
\define@key{fams}{mtg}{Trans-New Guinea}
\define@key{fams}{unm}{Algic}
\define@key{fams}{ung}{Worrorran}
\define@key{fams}{kuu}{Na-Dene}
\define@key{fams}{uur}{Austronesian}
\define@key{fams}{urf}{Pama-Nyungan}
\define@key{fams}{urk}{Austronesian}
\define@key{fams}{ura}{Isolate}
\define@key{fams}{urt}{Torricelli}
\define@key{fams}{urd}{Indo-European}
\define@key{fams}{urh}{Niger-Congo}
\define@key{fams}{uri}{Torricelli}
\define@key{fams}{ure}{Uru-Chipaya}
\define@key{fams}{uks}{other}
\define@key{fams}{urb}{Tupian}
\define@key{fams}{uum}{Altaic}
\define@key{fams}{wnu}{Trans-New Guinea}
\define@key{fams}{usa}{Trans-New Guinea}
\define@key{fams}{ute}{Uto-Aztecan}
\define@key{fams}{uig}{Altaic}
\define@key{fams}{uzn}{Altaic}
\define@key{fams}{vaf}{Indo-European}
\define@key{fams}{vag}{Niger-Congo}
\define@key{fams}{vai}{Mande}
\define@key{fams}{vas}{Indo-European}
\define@key{fams}{dic}{Niger-Congo}
\define@key{fams}{ved}{Indo-European}
\define@key{fams}{ven}{Niger-Congo}
\define@key{fams}{vep}{Uralic}
\define@key{fams}{vie}{Austro-Asiatic}
\define@key{fams}{vif}{Niger-Congo}
\define@key{fams}{vnm}{Austronesian}
\define@key{fams}{vgt}{other}
\define@key{fams}{vot}{Uralic}
\define@key{fams}{wwa}{Niger-Congo}
\define@key{fams}{wkw}{Pama-Nyungan}
\define@key{fams}{waq}{Isolate}
\define@key{fams}{waw}{Cariban}
\define@key{fams}{wbk}{Indo-European}
\define@key{fams}{bao}{Tucanoan}
\define@key{fams}{wbl}{Indo-European}
\define@key{fams}{wls}{Austronesian}
\define@key{fams}{van}{Torricelli}
\define@key{fams}{wmt}{Pama-Nyungan}
\define@key{fams}{wmb}{Mirndi}
\define@key{fams}{wms}{Trans-New Guinea}
\define@key{fams}{wme}{Sino-Tibetan}
\define@key{fams}{wan}{Mande}
\define@key{fams}{wgg}{Pama-Nyungan}
\define@key{fams}{xwk}{Pama-Nyungan}
\define@key{fams}{wbt}{Pama-Nyungan}
\define@key{fams}{wnc}{Trans-New Guinea}
\define@key{fams}{auc}{Isolate}
\define@key{fams}{wap}{Arawakan}
\define@key{fams}{wao}{Wappo-Yukian}
\define@key{fams}{wba}{Isolate}
\define@key{fams}{wrz}{Gunwinyguan}
\define@key{fams}{war}{Austronesian}
\define@key{fams}{wrr}{Yangmanic}
\define@key{fams}{gae}{Arawakan}
\define@key{fams}{wsa}{Austronesian}
\define@key{fams}{pav}{Chapacura-Wanham}
\define@key{fams}{wrs}{Border}
\define@key{fams}{wbp}{Pama-Nyungan}
\define@key{fams}{wrb}{Pama-Nyungan}
\define@key{fams}{wnd}{Mangarrayi-Maran}
\define@key{fams}{wrp}{Austronesian}
\define@key{fams}{wgy}{Pama-Nyungan}
\define@key{fams}{gjm}{Pama-Nyungan}
\define@key{fams}{wrg}{Pama-Nyungan}
\define@key{fams}{wwr}{Nyulnyulan}
\define@key{fams}{wrm}{Pama-Nyungan}
\define@key{fams}{was}{Isolate}
\define@key{fams}{wsk}{Trans-New Guinea}
\define@key{fams}{wax}{Lower Sepik-Ramu}
\define@key{fams}{wth}{Pama-Nyungan}
\define@key{fams}{wbv}{Pama-Nyungan}
\define@key{fams}{noa}{Choco}
\define@key{fams}{wau}{Arawakan}
\define@key{fams}{oym}{Tupian}
\define@key{fams}{way}{Cariban}
\define@key{fams}{wed}{Austronesian}
\define@key{fams}{cym}{Indo-European}
\define@key{fams}{xww}{Pama-Nyungan}
\define@key{fams}{wer}{Trans-New Guinea}
\define@key{fams}{mqs}{North Halmaheran}
\define@key{fams}{lex}{Austronesian}
\define@key{fams}{wic}{Caddoan}
\define@key{fams}{mzh}{Matacoan}
\define@key{fams}{wim}{Pama-Nyungan}
\define@key{fams}{wig}{Pama-Nyungan}
\define@key{fams}{yok}{Penutian}
\define@key{fams}{win}{Siouan}
\define@key{fams}{wnw}{Penutian}
\define@key{fams}{wgu}{Pama-Nyungan}
\define@key{fams}{wiy}{Algic}
\define@key{fams}{wob}{Niger-Congo}
\define@key{fams}{wog}{Sepik}
\define@key{fams}{woi}{Greater West Bomberai}
\define@key{fams}{wyu}{Pama-Nyungan}
\define@key{fams}{wal}{Afro-Asiatic}
\define@key{fams}{woe}{Austronesian}
\define@key{fams}{wlo}{Austronesian}
\define@key{fams}{wol}{Niger-Congo}
\define@key{fams}{wmx}{Skou}
\define@key{fams}{wro}{Worrorran}
\define@key{fams}{wuu}{Sino-Tibetan}
\define@key{fams}{wya}{Iroquoian}
\define@key{fams}{wem}{Niger-Congo}
\define@key{fams}{kao}{Mande}
\define@key{fams}{xav}{Macro-Ge}
\define@key{fams}{xer}{Macro-Ge}
\define@key{fams}{xho}{Niger-Congo}
\define@key{fams}{xir}{Arawakan}
\define@key{fams}{xok}{Macro-Ge}
\define@key{fams}{ane}{Austronesian}
\define@key{fams}{yai}{Indo-European}
\define@key{fams}{yad}{Peba-Yaguan}
\define@key{fams}{yag}{Yámana}
\define@key{fams}{yaf}{Niger-Congo}
\define@key{fams}{yka}{Austronesian}
\define@key{fams}{yky}{Niger-Congo}
\define@key{fams}{sah}{Altaic}
\define@key{fams}{ylr}{Pama-Nyungan}
\define@key{fams}{kkl}{Trans-New Guinea}
\define@key{fams}{yli}{Trans-New Guinea}
\define@key{fams}{yam}{Niger-Congo}
\define@key{fams}{jmd}{Austronesian}
\define@key{fams}{tao}{Austronesian}
\define@key{fams}{yaa}{Pano-Tacanan}
\define@key{fams}{ybi}{Sino-Tibetan}
\define@key{fams}{ynn}{Hokan}
\define@key{fams}{kdd}{Pama-Nyungan}
\define@key{fams}{wca}{Yanomam}
\define@key{fams}{yns}{Niger-Congo}
\define@key{fams}{jao}{Pama-Nyungan}
\define@key{fams}{yao}{Niger-Congo}
\define@key{fams}{yap}{Austronesian}
\define@key{fams}{jaq}{Trans-New Guinea}
\define@key{fams}{yaq}{Uto-Aztecan}
\define@key{fams}{yrb}{Trans-New Guinea}
\define@key{fams}{yae}{Isolate}
\define@key{fams}{yuf}{Hokan}
\define@key{fams}{yva}{Isolate}
\define@key{fams}{ywr}{Nyulnyulan}
\define@key{fams}{pcc}{Tai-Kadai}
\define@key{fams}{xya}{Pama-Nyungan}
\define@key{fams}{yah}{Indo-European}
\define@key{fams}{kpv}{Uralic}
\define@key{fams}{jei}{Yam}
\define@key{fams}{jel}{Bulaka River}
\define@key{fams}{yle}{Yele}
\define@key{fams}{ybb}{Niger-Congo}
\define@key{fams}{jnj}{Afro-Asiatic}
\define@key{fams}{yss}{Sepik}
\define@key{fams}{yey}{Niger-Congo}
\define@key{fams}{ywq}{Sino-Tibetan}
\define@key{fams}{ydd}{Indo-European}
\define@key{fams}{yii}{Pama-Nyungan}
\define@key{fams}{yll}{Torricelli}
\define@key{fams}{yee}{Lower Sepik-Ramu}
\define@key{fams}{yij}{Pama-Nyungan}
\define@key{fams}{yia}{Pama-Nyungan}
\define@key{fams}{yyr}{Pama-Nyungan}
\define@key{fams}{xyy}{Pama-Nyungan}
\define@key{fams}{yor}{Niger-Congo}
\define@key{fams}{yua}{Mayan}
\define@key{fams}{yuc}{Isolate}
\define@key{fams}{ycn}{Arawakan}
\define@key{fams}{yug}{Yeniseian}
\define@key{fams}{yux}{Yukaghir}
\define@key{fams}{ykg}{Yukaghir}
\define@key{fams}{yuk}{Wappo-Yukian}
\define@key{fams}{yup}{Cariban}
\define@key{fams}{gcd}{Tangkic}
\define@key{fams}{mpj}{Pama-Nyungan}
\define@key{fams}{yul}{Central Sudanic}
\define@key{fams}{esu}{Eskimo-Aleut}
\define@key{fams}{ynk}{Eskimo-Aleut}
\define@key{fams}{ess}{Eskimo-Aleut}
\define@key{fams}{ysr}{Eskimo-Aleut}
\define@key{fams}{yuz}{Isolate}
\define@key{fams}{yur}{Algic}
\define@key{fams}{yui}{Tucanoan}
\define@key{fams}{zne}{Niger-Congo}
\define@key{fams}{zro}{Zaparoan}
\define@key{fams}{zai}{Oto-Manguean}
\define@key{fams}{zpd}{Oto-Manguean}
\define@key{fams}{zaa}{Oto-Manguean}
\define@key{fams}{zaw}{Oto-Manguean}
\define@key{fams}{zpm}{Oto-Manguean}
\define@key{fams}{zpi}{Oto-Manguean}
\define@key{fams}{zab}{Oto-Manguean}
\define@key{fams}{zpz}{Oto-Manguean}
\define@key{fams}{zav}{Oto-Manguean}
\define@key{fams}{zpq}{Oto-Manguean}
\define@key{fams}{dje}{Songhay}
\define@key{fams}{zay}{Afro-Asiatic}
\define@key{fams}{diq}{Indo-European}
\define@key{fams}{zen}{Afro-Asiatic}
\define@key{fams}{zgb}{Tai-Kadai}
\define@key{fams}{zik}{Trans-New Guinea}
\define@key{fams}{zoh}{Mixe-Zoque}
\define@key{fams}{zos}{Mixe-Zoque}
\define@key{fams}{zoc}{Mixe-Zoque}
\define@key{fams}{zor}{Mixe-Zoque}
\define@key{fams}{zul}{Niger-Congo}
\define@key{fams}{zun}{Isolate}
\define@key{fams}{eme}{Tupian}
\define@key{fams}{aom}{Trans-New Guinea}
\define@key{fams}{aas}{Afro-Asiatic}
\define@key{fams}{kbt}{Austronesian}
\define@key{fams}{abg}{Nuclear Trans New Guinea}
\define@key{fams}{abf}{Austronesian}
\define@key{fams}{abm}{Atlantic-Congo}
\define@key{fams}{mij}{Atlantic-Congo}
\define@key{fams}{aba}{Atlantic-Congo}
\define@key{fams}{abp}{Austronesian}
\define@key{fams}{bsa}{Isolate}
\define@key{fams}{ash}{Isolate}
\define@key{fams}{aob}{Anim}
\define@key{fams}{abo}{Atlantic-Congo}
\define@key{fams}{abr}{Atlantic-Congo}
\define@key{fams}{abn}{Atlantic-Congo}
\define@key{fams}{abu}{Atlantic-Congo}
\define@key{fams}{mgj}{Atlantic-Congo}
\define@key{fams}{ado}{Lower Sepik-Ramu}
\define@key{fams}{tpx}{Otomanguean}
\define@key{fams}{yif}{Sino-Tibetan}
\define@key{fams}{acz}{Narrow Talodi}
\define@key{fams}{acs}{Nuclear-Macro-Je}
\define@key{fams}{xad}{Isolate}
\define@key{fams}{ada}{Atlantic-Congo}
\define@key{fams}{adq}{Atlantic-Congo}
\define@key{fams}{tiu}{Austronesian}
\define@key{fams}{ade}{Atlantic-Congo}
\define@key{fams}{adh}{Nilotic}
\define@key{fams}{gas}{Indo-European}
\define@key{fams}{adr}{Austronesian}
\define@key{fams}{aez}{Nuclear Trans New Guinea}
\define@key{fams}{aeq}{Indo-European}
\define@key{fams}{afg}{Sign Language}
\define@key{fams}{aft}{Nyimang}
\define@key{fams}{afh}{Artificial Language}
\define@key{fams}{afs}{Indo-European}
\define@key{fams}{agi}{Unattested}
\define@key{fams}{agc}{Atlantic-Congo}
\define@key{fams}{avo}{Unattested}
\define@key{fams}{ggr}{Pama-Nyungan}
\define@key{fams}{xag}{Nakh-Daghestanian}
\define@key{fams}{aif}{Nuclear Torricelli}
\define@key{fams}{kit}{Pahoturi}
\define@key{fams}{ibm}{Atlantic-Congo}
\define@key{fams}{apf}{Austronesian}
\define@key{fams}{aga}{Unattested}
\define@key{fams}{aug}{Atlantic-Congo}
\define@key{fams}{msm}{Austronesian}
\define@key{fams}{agn}{Austronesian}
\define@key{fams}{yay}{Atlantic-Congo}
\define@key{fams}{aha}{Atlantic-Congo}
\define@key{fams}{ahn}{Atlantic-Congo}
\define@key{fams}{esg}{Dravidian}
\define@key{fams}{thm}{Austroasiatic}
\define@key{fams}{kak}{Austronesian}
\define@key{fams}{aho}{Tai-Kadai}
\define@key{fams}{nfd}{Atlantic-Congo}
\define@key{fams}{aih}{Tai-Kadai}
\define@key{fams}{aix}{Austronesian}
\define@key{fams}{mwg}{Austronesian}
\define@key{fams}{aiq}{Indo-European}
\define@key{fams}{ail}{Bosavi}
\define@key{fams}{aim}{Sino-Tibetan}
\define@key{fams}{aic}{Border}
\define@key{fams}{aki}{Lower Sepik-Ramu}
\define@key{fams}{air}{Greater Kwerba}
\define@key{fams}{aio}{Tai-Kadai}
\define@key{fams}{ajw}{Afro-Asiatic}
\define@key{fams}{cpc}{Arawakan}
\define@key{fams}{soh}{Eastern Jebel}
\define@key{fams}{akm}{Great Andamanese}
\define@key{fams}{akj}{Great Andamanese}
\define@key{fams}{ack}{Great Andamanese}
\define@key{fams}{aky}{Great Andamanese}
\define@key{fams}{acl}{Great Andamanese}
\define@key{fams}{aks}{Atlantic-Congo}
\define@key{fams}{aik}{Atlantic-Congo}
\define@key{fams}{tsr}{Austronesian}
\define@key{fams}{aeu}{Sino-Tibetan}
\define@key{fams}{sia}{Uralic}
\define@key{fams}{akk}{Afro-Asiatic}
\define@key{fams}{akq}{Sepik}
\define@key{fams}{akt}{Austronesian}
\define@key{fams}{bss}{Atlantic-Congo}
\define@key{fams}{miw}{Angan}
\define@key{fams}{akf}{Atlantic-Congo}
\define@key{fams}{ibe}{Atlantic-Congo}
\define@key{fams}{afi}{Lower Sepik-Ramu}
\define@key{fams}{ayk}{Atlantic-Congo}
\define@key{fams}{aku}{Atlantic-Congo}
\define@key{fams}{aqz}{Tupian}
\define@key{fams}{ako}{Cariban}
\define@key{fams}{dul}{Austronesian}
\define@key{fams}{alw}{Afro-Asiatic}
\define@key{fams}{ala}{Atlantic-Congo}
\define@key{fams}{alk}{Austroasiatic}
\define@key{fams}{alj}{Austronesian}
\define@key{fams}{apv}{Unattested}
\define@key{fams}{bhk}{Austronesian}
\define@key{fams}{sqk}{Sign Language}
\define@key{fams}{lsc}{Sign Language}
\define@key{fams}{xta}{Otomanguean}
\define@key{fams}{alf}{Atlantic-Congo}
\define@key{fams}{asp}{Sign Language}
\define@key{fams}{arq}{Afro-Asiatic}
\define@key{fams}{aao}{Afro-Asiatic}
\define@key{fams}{aiy}{Atlantic-Congo}
\define@key{fams}{all}{Dravidian}
\define@key{fams}{aid}{Pama-Nyungan}
\define@key{fams}{zaq}{Otomanguean}
\define@key{fams}{ypo}{Sino-Tibetan}
\define@key{fams}{aol}{Austronesian}
\define@key{fams}{syy}{Sign Language}
\define@key{fams}{aub}{Sino-Tibetan}
\define@key{fams}{xua}{Dravidian}
\define@key{fams}{aab}{Atlantic-Congo}
\define@key{fams}{yna}{Sino-Tibetan}
\define@key{fams}{alz}{Nilotic}
\define@key{fams}{avd}{Indo-European}
\define@key{fams}{amq}{Austronesian}
\define@key{fams}{ali}{Nuclear Trans New Guinea}
\define@key{fams}{aad}{Sepik}
\define@key{fams}{jks}{Sign Language}
\define@key{fams}{ama}{Tupian}
\define@key{fams}{amg}{Iwaidjan Proper}
\define@key{fams}{aaz}{Austronesian}
\define@key{fams}{zpo}{Otomanguean}
\define@key{fams}{rwm}{Atlantic-Congo}
\define@key{fams}{utp}{Austronesian}
\define@key{fams}{abc}{Austronesian}
\define@key{fams}{aew}{Keram}
\define@key{fams}{ael}{Atlantic-Congo}
\define@key{fams}{amv}{Austronesian}
\define@key{fams}{alm}{Austronesian}
\define@key{fams}{amb}{Atlantic-Congo}
\define@key{fams}{abs}{Austronesian}
\define@key{fams}{qva}{Quechuan}
\define@key{fams}{aag}{Nuclear Torricelli}
\define@key{fams}{amj}{Furan}
\define@key{fams}{ifa}{Austronesian}
\define@key{fams}{alx}{Nuclear Torricelli}
\define@key{fams}{mbz}{Otomanguean}
\define@key{fams}{aqd}{Dogon}
\define@key{fams}{apg}{Austronesian}
\define@key{fams}{ajz}{Sino-Tibetan}
\define@key{fams}{amt}{Amto-Musan}
\define@key{fams}{adw}{Tupian}
\define@key{fams}{anw}{Atlantic-Congo}
\define@key{fams}{akg}{Austronesian}
\define@key{fams}{anm}{Sino-Tibetan}
\define@key{fams}{pda}{Nuclear Trans New Guinea}
\define@key{fams}{aan}{Tupian}
\define@key{fams}{dti}{Dogon}
\define@key{fams}{grc}{Indo-European}
\define@key{fams}{hbo}{Afro-Asiatic}
\define@key{fams}{xna}{Afro-Asiatic}
\define@key{fams}{xlg}{Unclassifiable}
\define@key{fams}{hca}{Indo-European}
\define@key{fams}{afd}{Arafundi}
\define@key{fams}{aod}{Lower Sepik-Ramu}
\define@key{fams}{ana}{Isolate}
\define@key{fams}{xaa}{Afro-Asiatic}
\define@key{fams}{adg}{Pama-Nyungan}
\define@key{fams}{bzb}{Austronesian}
\define@key{fams}{anb}{Zaparoan}
\define@key{fams}{anx}{Austronesian}
\define@key{fams}{aby}{Yareban}
\define@key{fams}{myo}{Ta-Ne-Omotic}
\define@key{fams}{akh}{Nuclear Trans New Guinea}
\define@key{fams}{age}{Nuclear Trans New Guinea}
\define@key{fams}{aoe}{Nuclear Trans New Guinea}
\define@key{fams}{aqt}{Lengua-Mascoy}
\define@key{fams}{avm}{Pama-Nyungan}
\define@key{fams}{anp}{Indo-European}
\define@key{fams}{rme}{Indo-European}
\define@key{fams}{aog}{Lower Sepik-Ramu}
\define@key{fams}{tnd}{Chibchan}
\define@key{fams}{blo}{Atlantic-Congo}
\define@key{fams}{anf}{Atlantic-Congo}
\define@key{fams}{aqk}{Atlantic-Congo}
\define@key{fams}{ypn}{Sino-Tibetan}
\define@key{fams}{boj}{Nuclear Trans New Guinea}
\define@key{fams}{aak}{Angan}
\define@key{fams}{amx}{Pama-Nyungan}
\define@key{fams}{anj}{Lower Sepik-Ramu}
\define@key{fams}{ans}{Chocoan}
\define@key{fams}{and}{Austronesian}
\define@key{fams}{ant}{Pama-Nyungan}
\define@key{fams}{xmv}{Austronesian}
\define@key{fams}{aig}{Indo-European}
\define@key{fams}{aui}{Austronesian}
\define@key{fams}{auq}{Austronesian}
\define@key{fams}{aud}{Austronesian}
\define@key{fams}{anl}{Sino-Tibetan}
\define@key{fams}{mtb}{Atlantic-Congo}
\define@key{fams}{pni}{Austronesian}
\define@key{fams}{aor}{Austronesian}
\define@key{fams}{aou}{Tai-Kadai}
\define@key{fams}{xap}{Muskogean}
\define@key{fams}{apo}{Austronesian}
\define@key{fams}{ena}{Nuclear Trans New Guinea}
\define@key{fams}{mip}{Otomanguean}
\define@key{fams}{api}{Tupian}
\define@key{fams}{app}{Austronesian}
\define@key{fams}{apx}{Austronesian}
\define@key{fams}{arg}{Indo-European}
\define@key{fams}{stk}{Yam}
\define@key{fams}{aaf}{Dravidian}
\define@key{fams}{xrt}{Unclassifiable}
\define@key{fams}{arj}{Tucanoan}
\define@key{fams}{awm}{Nuclear Trans New Guinea}
\define@key{fams}{awt}{Tupian}
\define@key{fams}{aae}{Indo-European}
\define@key{fams}{aea}{Pama-Nyungan}
\define@key{fams}{mwc}{Austronesian}
\define@key{fams}{aem}{Austroasiatic}
\define@key{fams}{qxu}{Quechuan}
\define@key{fams}{agj}{Afro-Asiatic}
\define@key{fams}{agf}{Austronesian}
\define@key{fams}{aqr}{Austronesian}
\define@key{fams}{aok}{Austronesian}
\define@key{fams}{ylu}{Austronesian}
\define@key{fams}{aai}{Austronesian}
\define@key{fams}{aqg}{Atlantic-Congo}
\define@key{fams}{aac}{Suki-Gogodala}
\define@key{fams}{ait}{Tupian}
\define@key{fams}{ark}{Nuclear-Macro-Je}
\define@key{fams}{xrn}{Yeniseian}
\define@key{fams}{luc}{Central Sudanic}
\define@key{fams}{dth}{Pama-Nyungan}
\define@key{fams}{aoh}{Unattested}
\define@key{fams}{aen}{Sign Language}
\define@key{fams}{rup}{Indo-European}
\define@key{fams}{aps}{Austronesian}
\define@key{fams}{atz}{Austronesian}
\define@key{fams}{arx}{Tupian}
\define@key{fams}{aru}{Arawan}
\define@key{fams}{aur}{Nuclear Torricelli}
\define@key{fams}{lsr}{Nuclear Torricelli}
\define@key{fams}{atx}{Isolate}
\define@key{fams}{aat}{Indo-European}
\define@key{fams}{mtv}{Nuclear Trans New Guinea}
\define@key{fams}{cni}{Arawakan}
\define@key{fams}{ahs}{Atlantic-Congo}
\define@key{fams}{prq}{Arawakan}
\define@key{fams}{ask}{Indo-European}
\define@key{fams}{atn}{Indo-European}
\define@key{fams}{asl}{Austronesian}
\define@key{fams}{eiv}{North Bougainville}
\define@key{fams}{asv}{Central Sudanic}
\define@key{fams}{asb}{Siouan}
\define@key{fams}{asz}{Austronesian}
\define@key{fams}{aua}{Austronesian}
\define@key{fams}{aum}{Atlantic-Congo}
\define@key{fams}{zoo}{Otomanguean}
\define@key{fams}{asr}{Austroasiatic}
\define@key{fams}{atm}{Austronesian}
\define@key{fams}{amz}{Pama-Nyungan}
\define@key{fams}{atd}{Austronesian}
\define@key{fams}{ate}{Nuclear Trans New Guinea}
\define@key{fams}{atk}{Austronesian}
\define@key{fams}{aqm}{Kayagaric}
\define@key{fams}{aot}{Sino-Tibetan}
\define@key{fams}{ato}{Atlantic-Congo}
\define@key{fams}{aox}{Arawakan}
\define@key{fams}{cch}{Atlantic-Congo}
\define@key{fams}{atc}{Pano-Tacanan}
\define@key{fams}{pkr}{Dravidian}
\define@key{fams}{ati}{Atlantic-Congo}
\define@key{fams}{kud}{Austronesian}
\define@key{fams}{aux}{Tupian}
\define@key{fams}{auh}{Atlantic-Congo}
\define@key{fams}{avs}{Zaparoan}
\define@key{fams}{asq}{Sign Language}
\define@key{fams}{asw}{Sign Language}
\define@key{fams}{aut}{Austronesian}
\define@key{fams}{smf}{Border}
\define@key{fams}{auu}{Nuclear Trans New Guinea}
\define@key{fams}{auo}{Afro-Asiatic}
\define@key{fams}{avv}{Tupian}
\define@key{fams}{avb}{Austronesian}
\define@key{fams}{ave}{Indo-European}
\define@key{fams}{awk}{Pama-Nyungan}
\define@key{fams}{vwa}{Austroasiatic}
\define@key{fams}{bcu}{Austronesian}
\define@key{fams}{awo}{Atlantic-Congo}
\define@key{fams}{awx}{Nuclear Trans New Guinea}
\define@key{fams}{aya}{Lower Sepik-Ramu}
\define@key{fams}{awh}{Bayono-Awbono}
\define@key{fams}{bob}{Afro-Asiatic}
\define@key{fams}{awr}{Lakes Plain}
\define@key{fams}{awe}{Tupian}
\define@key{fams}{azo}{Atlantic-Congo}
\define@key{fams}{auj}{Afro-Asiatic}
\define@key{fams}{aww}{Sepik}
\define@key{fams}{afu}{Atlantic-Congo}
\define@key{fams}{yiu}{Sino-Tibetan}
\define@key{fams}{ahb}{Austronesian}
\define@key{fams}{yix}{Sino-Tibetan}
\define@key{fams}{ayd}{Pama-Nyungan}
\define@key{fams}{vmy}{Otomanguean}
\define@key{fams}{aye}{Atlantic-Congo}
\define@key{fams}{ayq}{Sepik}
\define@key{fams}{yyz}{Sino-Tibetan}
\define@key{fams}{ayb}{Atlantic-Congo}
\define@key{fams}{zaf}{Otomanguean}
\define@key{fams}{ayu}{Atlantic-Congo}
\define@key{fams}{aza}{Sino-Tibetan}
\define@key{fams}{yiz}{Sino-Tibetan}
\define@key{fams}{tpc}{Otomanguean}
\define@key{fams}{bvj}{Atlantic-Congo}
\define@key{fams}{bqx}{Atlantic-Congo}
\define@key{fams}{bbm}{Atlantic-Congo}
\define@key{fams}{bbw}{Atlantic-Congo}
\define@key{fams}{bbk}{Atlantic-Congo}
\define@key{fams}{mbf}{Austronesian}
\define@key{fams}{bcr}{Athabaskan-Eyak-Tlingit}
\define@key{fams}{bzg}{Austronesian}
\define@key{fams}{btj}{Austronesian}
\define@key{fams}{bcy}{Afro-Asiatic}
\define@key{fams}{xbc}{Indo-European}
\define@key{fams}{bau}{Atlantic-Congo}
\define@key{fams}{bhz}{Austronesian}
\define@key{fams}{bdz}{Unattested}
\define@key{fams}{jbi}{Pama-Nyungan}
\define@key{fams}{bac}{Austronesian}
\define@key{fams}{pbp}{Atlantic-Congo}
\define@key{fams}{bvd}{Austronesian}
\define@key{fams}{bvc}{Austronesian}
\define@key{fams}{btr}{Austronesian}
\define@key{fams}{bwt}{Atlantic-Congo}
\define@key{fams}{bfj}{Atlantic-Congo}
\define@key{fams}{bmd}{Atlantic-Congo}
\define@key{fams}{bgo}{Atlantic-Congo}
\define@key{fams}{bcg}{Atlantic-Congo}
\define@key{fams}{bfy}{Indo-European}
\define@key{fams}{fui}{Atlantic-Congo}
\define@key{fams}{bqg}{Atlantic-Congo}
\define@key{fams}{bqb}{Greater Kwerba}
\define@key{fams}{bpi}{Nuclear Trans New Guinea}
\define@key{fams}{yha}{Tai-Kadai}
\define@key{fams}{bhv}{Austronesian}
\define@key{fams}{bah}{Indo-European}
\define@key{fams}{bhj}{Sino-Tibetan}
\define@key{fams}{bsu}{Austronesian}
\define@key{fams}{bbf}{Baibai-Fas}
\define@key{fams}{bdj}{Atlantic-Congo}
\define@key{fams}{bkx}{Austronesian}
\define@key{fams}{bqh}{Sino-Tibetan}
\define@key{fams}{bmx}{Nuclear Trans New Guinea}
\define@key{fams}{bab}{Atlantic-Congo}
\define@key{fams}{bcz}{Atlantic-Congo}
\define@key{fams}{fah}{Atlantic-Congo}
\define@key{fams}{bjs}{Indo-European}
\define@key{fams}{bjm}{Indo-European}
\define@key{fams}{bqz}{Atlantic-Congo}
\define@key{fams}{bqi}{Indo-European}
\define@key{fams}{bki}{Austronesian}
\define@key{fams}{bkh}{Atlantic-Congo}
\define@key{fams}{kme}{Atlantic-Congo}
\define@key{fams}{bbs}{Atlantic-Congo}
\define@key{fams}{bkr}{Austronesian}
\define@key{fams}{bjw}{Kru}
\define@key{fams}{ble}{Atlantic-Congo}
\define@key{fams}{bjt}{Atlantic-Congo}
\define@key{fams}{bls}{Austronesian}
\define@key{fams}{bdn}{Afro-Asiatic}
\define@key{fams}{bcn}{Atlantic-Congo}
\define@key{fams}{bcp}{Atlantic-Congo}
\define@key{fams}{mhp}{Austronesian}
\define@key{fams}{bgx}{Turkic}
\define@key{fams}{biz}{Atlantic-Congo}
\define@key{fams}{bqo}{Atlantic-Congo}
\define@key{fams}{blq}{Austronesian}
\define@key{fams}{bog}{Sign Language}
\define@key{fams}{bbq}{Atlantic-Congo}
\define@key{fams}{myf}{Blue Nile Mao}
\define@key{fams}{bmo}{Atlantic-Congo}
\define@key{fams}{bce}{Atlantic-Congo}
\define@key{fams}{bqt}{Atlantic-Congo}
\define@key{fams}{bvm}{Atlantic-Congo}
\define@key{fams}{bcf}{Kiwaian}
\define@key{fams}{bmg}{Atlantic-Congo}
\define@key{fams}{bjx}{Austronesian}
\define@key{fams}{byz}{Lower Sepik-Ramu}
\define@key{fams}{bqj}{Atlantic-Congo}
\define@key{fams}{bqk}{Atlantic-Congo}
\define@key{fams}{bpd}{Atlantic-Congo}
\define@key{fams}{bfl}{Atlantic-Congo}
\define@key{fams}{yaj}{Atlantic-Congo}
\define@key{fams}{bpq}{Austronesian}
\define@key{fams}{bnd}{Austronesian}
\define@key{fams}{bbe}{Atlantic-Congo}
\define@key{fams}{bgf}{Atlantic-Congo}
\define@key{fams}{bsj}{Atlantic-Congo}
\define@key{fams}{bnx}{Atlantic-Congo}
\define@key{fams}{bxg}{Atlantic-Congo}
\define@key{fams}{bgj}{Atlantic-Congo}
\define@key{fams}{mfb}{Austronesian}
\define@key{fams}{bjn}{Austronesian}
\define@key{fams}{bfk}{Sign Language}
\define@key{fams}{bxw}{Mande}
\define@key{fams}{dbw}{Dogon}
\define@key{fams}{bap}{Sino-Tibetan}
\define@key{fams}{bno}{Austronesian}
\define@key{fams}{bfx}{Austronesian}
\define@key{fams}{brd}{Sino-Tibetan}
\define@key{fams}{bbg}{Atlantic-Congo}
\define@key{fams}{baj}{Austronesian}
\define@key{fams}{bhr}{Austronesian}
\define@key{fams}{brs}{Austronesian}
\define@key{fams}{brp}{Geelvink Bay}
\define@key{fams}{bmz}{Anim}
\define@key{fams}{bpb}{Unattested}
\define@key{fams}{gry}{Kru}
\define@key{fams}{bva}{Afro-Asiatic}
\define@key{fams}{bxo}{Pidgin}
\define@key{fams}{bch}{Austronesian}
\define@key{fams}{bjc}{Yareban}
\define@key{fams}{jbk}{Turama-Kikori}
\define@key{fams}{bbi}{Atlantic-Congo}
\define@key{fams}{bjk}{Austronesian}
\define@key{fams}{bpt}{Pama-Nyungan}
\define@key{fams}{tbn}{Chibchan}
\define@key{fams}{bjz}{Nuclear Trans New Guinea}
\define@key{fams}{bwg}{Atlantic-Congo}
\define@key{fams}{bjf}{Afro-Asiatic}
\define@key{fams}{bsl}{Atlantic-Congo}
\define@key{fams}{buj}{Atlantic-Congo}
\define@key{fams}{bzw}{Atlantic-Congo}
\define@key{fams}{bdb}{Austronesian}
\define@key{fams}{byq}{Austronesian}
\define@key{fams}{bsg}{Indo-European}
\define@key{fams}{bst}{Ta-Ne-Omotic}
\define@key{fams}{bsr}{Atlantic-Congo}
\define@key{fams}{bsi}{Atlantic-Congo}
\define@key{fams}{bnm}{Atlantic-Congo}
\define@key{fams}{bts}{Austronesian}
\define@key{fams}{akb}{Austronesian}
\define@key{fams}{btm}{Austronesian}
\define@key{fams}{btd}{Austronesian}
\define@key{fams}{ayt}{Austronesian}
\define@key{fams}{bta}{Afro-Asiatic}
\define@key{fams}{btv}{Indo-European}
\define@key{fams}{btq}{Austroasiatic}
\define@key{fams}{btc}{Atlantic-Congo}
\define@key{fams}{bvt}{Austronesian}
\define@key{fams}{btu}{Atlantic-Congo}
\define@key{fams}{bay}{Austronesian}
\define@key{fams}{zbt}{Austronesian}
\define@key{fams}{sne}{Austronesian}
\define@key{fams}{bsf}{Atlantic-Congo}
\define@key{fams}{bge}{Indo-European}
\define@key{fams}{bxa}{Austronesian}
\define@key{fams}{bwk}{Mailuan}
\define@key{fams}{bjy}{Pama-Nyungan}
\define@key{fams}{bvy}{Austronesian}
\define@key{fams}{byg}{Dajuic}
\define@key{fams}{mkq}{Miwok-Costanoan}
\define@key{fams}{bda}{Atlantic-Congo}
\define@key{fams}{byl}{Bayono-Awbono}
\define@key{fams}{bfr}{Unclassifiable}
\define@key{fams}{beo}{Bosavi}
\define@key{fams}{bea}{Athabaskan-Eyak-Tlingit}
\define@key{fams}{bfp}{Atlantic-Congo}
\define@key{fams}{beb}{Atlantic-Congo}
\define@key{fams}{bzv}{Atlantic-Congo}
\define@key{fams}{bek}{Austronesian}
\define@key{fams}{bxp}{Atlantic-Congo}
\define@key{fams}{tnr}{Atlantic-Congo}
\define@key{fams}{bjv}{Central Sudanic}
\define@key{fams}{bed}{Austronesian}
\define@key{fams}{bkf}{Atlantic-Congo}
\define@key{fams}{bxq}{Afro-Asiatic}
\define@key{fams}{bnz}{Atlantic-Congo}
\define@key{fams}{bby}{Atlantic-Congo}
\define@key{fams}{bqv}{Atlantic-Congo}
\define@key{fams}{bei}{Austronesian}
\define@key{fams}{bkv}{Atlantic-Congo}
\define@key{fams}{bkw}{Atlantic-Congo}
\define@key{fams}{bvi}{Atlantic-Congo}
\define@key{fams}{bxb}{Nilotic}
\define@key{fams}{beg}{Austronesian}
\define@key{fams}{blm}{Central Sudanic}
\define@key{fams}{bey}{Nuclear Torricelli}
\define@key{fams}{bzj}{Indo-European}
\define@key{fams}{brw}{Dravidian}
\define@key{fams}{glb}{Afro-Asiatic}
\define@key{fams}{bmb}{Atlantic-Congo}
\define@key{fams}{yun}{Atlantic-Congo}
\define@key{fams}{bez}{Atlantic-Congo}
\define@key{fams}{bdp}{Atlantic-Congo}
\define@key{fams}{bct}{Central Sudanic}
\define@key{fams}{bgy}{Austronesian}
\define@key{fams}{bnu}{Austronesian}
\define@key{fams}{dbt}{Dogon}
\define@key{fams}{byd}{Austronesian}
\define@key{fams}{bie}{Nuclear Trans New Guinea}
\define@key{fams}{bxv}{Central Sudanic}
\define@key{fams}{bve}{Austronesian}
\define@key{fams}{bit}{Sepik}
\define@key{fams}{byt}{Saharan}
\define@key{fams}{bes}{Atlantic-Congo}
\define@key{fams}{bep}{Austronesian}
\define@key{fams}{bfe}{Tor-Orya}
\define@key{fams}{byf}{Atlantic-Congo}
\define@key{fams}{btt}{Atlantic-Congo}
\define@key{fams}{eot}{Atlantic-Congo}
\define@key{fams}{bhd}{Indo-European}
\define@key{fams}{bha}{Indo-European}
\define@key{fams}{bht}{Indo-European}
\define@key{fams}{bgw}{Indo-European}
\define@key{fams}{bhe}{Indo-European}
\define@key{fams}{bhy}{Atlantic-Congo}
\define@key{fams}{bhi}{Indo-European}
\define@key{fams}{nes}{Sino-Tibetan}
\define@key{fams}{bhu}{Indo-European}
\define@key{fams}{bdf}{Koiarian}
\define@key{fams}{beh}{Atlantic-Congo}
\define@key{fams}{bpv}{Anim}
\define@key{fams}{big}{Goilalan}
\define@key{fams}{byk}{Tai-Kadai}
\define@key{fams}{bje}{Hmong-Mien}
\define@key{fams}{bmt}{Hmong-Mien}
\define@key{fams}{bym}{Pama-Nyungan}
\define@key{fams}{bjg}{Atlantic-Congo}
\define@key{fams}{bmc}{Austronesian}
\define@key{fams}{bnk}{Austronesian}
\define@key{fams}{brj}{Austronesian}
\define@key{fams}{biu}{Sino-Tibetan}
\define@key{fams}{xbe}{Pama-Nyungan}
\define@key{fams}{bhc}{Austronesian}
\define@key{fams}{ibh}{Austronesian}
\define@key{fams}{jbm}{Atlantic-Congo}
\define@key{fams}{bix}{Austroasiatic}
\define@key{fams}{byb}{Atlantic-Congo}
\define@key{fams}{kfs}{Indo-European}
\define@key{fams}{bql}{Nuclear Trans New Guinea}
\define@key{fams}{brz}{Austronesian}
\define@key{fams}{bpz}{Austronesian}
\define@key{fams}{bil}{Atlantic-Congo}
\define@key{fams}{bms}{Saharan}
\define@key{fams}{bxf}{Austronesian}
\define@key{fams}{bhl}{Nuclear Trans New Guinea}
\define@key{fams}{byj}{Atlantic-Congo}
\define@key{fams}{bmn}{Austronesian}
\define@key{fams}{bxz}{Mailuan}
\define@key{fams}{bon}{Eastern Trans-Fly}
\define@key{fams}{bpj}{Atlantic-Congo}
\define@key{fams}{itb}{Austronesian}
\define@key{fams}{bne}{Austronesian}
\define@key{fams}{bny}{Austronesian}
\define@key{fams}{biq}{Austronesian}
\define@key{fams}{bxe}{Isolate}
\define@key{fams}{brr}{Austronesian}
\define@key{fams}{btf}{Afro-Asiatic}
\define@key{fams}{biy}{Austroasiatic}
\define@key{fams}{bqq}{Lakes Plain}
\define@key{fams}{brk}{Nubian}
\define@key{fams}{brl}{Atlantic-Congo}
\define@key{fams}{ije}{Ijoid}
\define@key{fams}{bpy}{Indo-European}
\define@key{fams}{bwh}{Atlantic-Congo}
\define@key{fams}{bnw}{Sepik}
\define@key{fams}{bir}{Nuclear Trans New Guinea}
\define@key{fams}{bzi}{Sino-Tibetan}
\define@key{fams}{brt}{Atlantic-Congo}
\define@key{fams}{bgk}{Austroasiatic}
\define@key{fams}{mcc}{Anim}
\define@key{fams}{bwm}{Yuat}
\define@key{fams}{byo}{Sino-Tibetan}
\define@key{fams}{bpm}{Nuclear Trans New Guinea}
\define@key{fams}{blp}{Austronesian}
\define@key{fams}{bfh}{Yam}
\define@key{fams}{beu}{Timor-Alor-Pantar}
\define@key{fams}{blr}{Austroasiatic}
\define@key{fams}{zbl}{Artificial Language}
\define@key{fams}{bzn}{Austronesian}
\define@key{fams}{bzl}{Austronesian}
\define@key{fams}{bty}{Austronesian}
\define@key{fams}{bgb}{Austronesian}
\define@key{fams}{bdv}{Indo-European}
\define@key{fams}{boy}{Atlantic-Congo}
\define@key{fams}{bff}{Atlantic-Congo}
\define@key{fams}{boq}{Isolate}
\define@key{fams}{bvw}{Afro-Asiatic}
\define@key{fams}{bux}{Afro-Asiatic}
\define@key{fams}{bqu}{Atlantic-Congo}
\define@key{fams}{bhn}{Afro-Asiatic}
\define@key{fams}{ybk}{Sino-Tibetan}
\define@key{fams}{bdt}{Atlantic-Congo}
\define@key{fams}{bkp}{Atlantic-Congo}
\define@key{fams}{bus}{Mande}
\define@key{fams}{bky}{Atlantic-Congo}
\define@key{fams}{bnp}{Austronesian}
\define@key{fams}{bld}{Austronesian}
\define@key{fams}{xbo}{Turkic}
\define@key{fams}{bvo}{Atlantic-Congo}
\define@key{fams}{bvl}{Sign Language}
\define@key{fams}{smk}{Austronesian}
\define@key{fams}{blv}{Atlantic-Congo}
\define@key{fams}{bkt}{Atlantic-Congo}
\define@key{fams}{bzm}{Atlantic-Congo}
\define@key{fams}{bof}{Mande}
\define@key{fams}{blj}{Austronesian}
\define@key{fams}{ply}{Austroasiatic}
\define@key{fams}{boh}{Atlantic-Congo}
\define@key{fams}{bml}{Atlantic-Congo}
\define@key{fams}{bws}{Atlantic-Congo}
\define@key{fams}{zmx}{Atlantic-Congo}
\define@key{fams}{bmf}{Atlantic-Congo}
\define@key{fams}{bmq}{Atlantic-Congo}
\define@key{fams}{bmw}{Atlantic-Congo}
\define@key{fams}{kzc}{Atlantic-Congo}
\define@key{fams}{bou}{Atlantic-Congo}
\define@key{fams}{dbu}{Dogon}
\define@key{fams}{bna}{Austronesian}
\define@key{fams}{bnv}{Tor-Orya}
\define@key{fams}{glc}{Atlantic-Congo}
\define@key{fams}{bui}{Atlantic-Congo}
\define@key{fams}{bpg}{Austronesian}
\define@key{fams}{bok}{Atlantic-Congo}
\define@key{fams}{bvg}{Atlantic-Congo}
\define@key{fams}{bop}{Nuclear Trans New Guinea}
\define@key{fams}{bnb}{Austronesian}
\define@key{fams}{bnl}{Afro-Asiatic}
\define@key{fams}{bvf}{Afro-Asiatic}
\define@key{fams}{bpw}{Left May}
\define@key{fams}{gai}{Lower Sepik-Ramu}
\define@key{fams}{fue}{Atlantic-Congo}
\define@key{fams}{ksr}{Nuclear Trans New Guinea}
\define@key{fams}{xxb}{Atlantic-Congo}
\define@key{fams}{mae}{Atlantic-Congo}
\define@key{fams}{bwf}{Austronesian}
\define@key{fams}{bqs}{Lower Sepik-Ramu}
\define@key{fams}{bmj}{Indo-European}
\define@key{fams}{bph}{Nakh-Daghestanian}
\define@key{fams}{sbl}{Austronesian}
\define@key{fams}{nku}{Atlantic-Congo}
\define@key{fams}{mux}{Nuclear Trans New Guinea}
\define@key{fams}{suo}{Sko}
\define@key{fams}{kxr}{Austronesian}
\define@key{fams}{aof}{Nuclear Torricelli}
\define@key{fams}{bra}{Indo-European}
\define@key{fams}{kvl}{Sino-Tibetan}
\define@key{fams}{buq}{Nuclear Trans New Guinea}
\define@key{fams}{brq}{Lower Sepik-Ramu}
\define@key{fams}{rib}{Sign Language}
\define@key{fams}{bzt}{Artificial Language}
\define@key{fams}{sgt}{Sino-Tibetan}
\define@key{fams}{bro}{Sino-Tibetan}
\define@key{fams}{bpl}{Pidgin}
\define@key{fams}{plw}{Austronesian}
\define@key{fams}{kxd}{Austronesian}
\define@key{fams}{bsb}{Austronesian}
\define@key{fams}{rnb}{Sign Language}
\define@key{fams}{bub}{Atlantic-Congo}
\define@key{fams}{cbl}{Sino-Tibetan}
\define@key{fams}{box}{Atlantic-Congo}
\define@key{fams}{buw}{Atlantic-Congo}
\define@key{fams}{stt}{Austroasiatic}
\define@key{fams}{btp}{Austronesian}
\define@key{fams}{bdx}{Austronesian}
\define@key{fams}{bja}{Atlantic-Congo}
\define@key{fams}{bbh}{Austroasiatic}
\define@key{fams}{buk}{Austronesian}
\define@key{fams}{bgt}{Austronesian}
\define@key{fams}{bku}{Austronesian}
\define@key{fams}{bxh}{Austronesian}
\define@key{fams}{byh}{Sino-Tibetan}
\define@key{fams}{bvk}{Austronesian}
\define@key{fams}{bhh}{Indo-European}
\define@key{fams}{bvu}{Austronesian}
\define@key{fams}{bkn}{Austronesian}
\define@key{fams}{tkb}{Indo-European}
\define@key{fams}{buz}{Atlantic-Congo}
\define@key{fams}{bqn}{Sign Language}
\define@key{fams}{bmp}{Nuclear Trans New Guinea}
\define@key{fams}{buy}{Atlantic-Congo}
\define@key{fams}{sti}{Austroasiatic}
\define@key{fams}{bjl}{Austronesian}
\define@key{fams}{byp}{Atlantic-Congo}
\define@key{fams}{aon}{Nuclear Torricelli}
\define@key{fams}{bmv}{Atlantic-Congo}
\define@key{fams}{kjz}{Sino-Tibetan}
\define@key{fams}{bwx}{Hmong-Mien}
\define@key{fams}{bdd}{Austronesian}
\define@key{fams}{bvn}{Nuclear Torricelli}
\define@key{fams}{bfn}{Timor-Alor-Pantar}
\define@key{fams}{bns}{Indo-European}
\define@key{fams}{bqd}{Atlantic-Congo}
\define@key{fams}{xbg}{Pama-Nyungan}
\define@key{fams}{wun}{Atlantic-Congo}
\define@key{fams}{bkz}{Austronesian}
\define@key{fams}{but}{Nuclear Torricelli}
\define@key{fams}{buv}{Yuat}
\define@key{fams}{dgb}{Dogon}
\define@key{fams}{bnn}{Austronesian}
\define@key{fams}{blf}{Austronesian}
\define@key{fams}{bys}{Atlantic-Congo}
\define@key{fams}{bti}{Geelvink Bay}
\define@key{fams}{bxn}{Pama-Nyungan}
\define@key{fams}{bvh}{Afro-Asiatic}
\define@key{fams}{pyx}{Sino-Tibetan}
\define@key{fams}{vrt}{Austronesian}
\define@key{fams}{bzu}{Isolate}
\define@key{fams}{bqw}{Atlantic-Congo}
\define@key{fams}{bdi}{Nilotic}
\define@key{fams}{bqr}{Austronesian}
\define@key{fams}{aip}{Nuclear Trans New Guinea}
\define@key{fams}{asi}{Nuclear Trans New Guinea}
\define@key{fams}{bry}{Ndu}
\define@key{fams}{bxs}{Atlantic-Congo}
\define@key{fams}{bsm}{Austronesian}
\define@key{fams}{bfg}{Austronesian}
\define@key{fams}{buc}{Austronesian}
\define@key{fams}{bup}{Austronesian}
\define@key{fams}{dox}{Afro-Asiatic}
\define@key{fams}{bju}{Atlantic-Congo}
\define@key{fams}{kyb}{Austronesian}
\define@key{fams}{bnr}{Austronesian}
\define@key{fams}{btw}{Austronesian}
\define@key{fams}{jid}{Atlantic-Congo}
\define@key{fams}{bhs}{Afro-Asiatic}
\define@key{fams}{jiy}{Sino-Tibetan}
\define@key{fams}{byi}{Atlantic-Congo}
\define@key{fams}{bww}{Atlantic-Congo}
\define@key{fams}{bwd}{Austronesian}
\define@key{fams}{tte}{Austronesian}
\define@key{fams}{bwa}{Austronesian}
\define@key{fams}{bwl}{Atlantic-Congo}
\define@key{fams}{bwc}{Atlantic-Congo}
\define@key{fams}{bwz}{Atlantic-Congo}
\define@key{fams}{mkk}{Atlantic-Congo}
\define@key{fams}{msq}{Austronesian}
\define@key{fams}{cbb}{Arawakan}
\define@key{fams}{ccr}{Misumalpan}
\define@key{fams}{miu}{Otomanguean}
\define@key{fams}{roc}{Austronesian}
\define@key{fams}{ccd}{Indo-European}
\define@key{fams}{cah}{Zaparoan}
\define@key{fams}{qvl}{Quechuan}
\define@key{fams}{zad}{Otomanguean}
\define@key{fams}{frc}{Indo-European}
\define@key{fams}{ckx}{Atlantic-Congo}
\define@key{fams}{ckz}{Mixed Language}
\define@key{fams}{cky}{Afro-Asiatic}
\define@key{fams}{tbk}{Austronesian}
\define@key{fams}{qud}{Quechuan}
\define@key{fams}{caw}{Speech Register}
\define@key{fams}{rmq}{Indo-European}
\define@key{fams}{clu}{Austronesian}
\define@key{fams}{abd}{Austronesian}
\define@key{fams}{csx}{Sign Language}
\define@key{fams}{mcu}{Atlantic-Congo}
\define@key{fams}{wes}{Indo-European}
\define@key{fams}{cml}{Austronesian}
\define@key{fams}{cmt}{Speech Register}
\define@key{fams}{xcc}{Unclassifiable}
\define@key{fams}{qxr}{Quechuan}
\define@key{fams}{caz}{Isolate}
\define@key{fams}{mlc}{Tai-Kadai}
\define@key{fams}{cov}{Tai-Kadai}
\define@key{fams}{cps}{Austronesian}
\define@key{fams}{cpg}{Indo-European}
\define@key{fams}{cot}{Arawakan}
\define@key{fams}{cby}{Unclassifiable}
\define@key{fams}{cfd}{Atlantic-Congo}
\define@key{fams}{crf}{Chocoan}
\define@key{fams}{xcr}{Indo-European}
\define@key{fams}{hns}{Indo-European}
\define@key{fams}{jvn}{Austronesian}
\define@key{fams}{crr}{Algic}
\define@key{fams}{rmc}{Indo-European}
\define@key{fams}{asc}{Nuclear Trans New Guinea}
\define@key{fams}{csc}{Sign Language}
\define@key{fams}{xcy}{Isolate}
\define@key{fams}{xce}{Indo-European}
\define@key{fams}{cen}{Atlantic-Congo}
\define@key{fams}{hmm}{Hmong-Mien}
\define@key{fams}{cmo}{Austroasiatic}
\define@key{fams}{zch}{Tai-Kadai}
\define@key{fams}{hmc}{Hmong-Mien}
\define@key{fams}{fuq}{Atlantic-Congo}
\define@key{fams}{grv}{Kru}
\define@key{fams}{cet}{Isolate}
\define@key{fams}{pse}{Austronesian}
\define@key{fams}{mwo}{Austronesian}
\define@key{fams}{mxz}{Austronesian}
\define@key{fams}{syb}{Austronesian}
\define@key{fams}{tgt}{Austronesian}
\define@key{fams}{plc}{Austronesian}
\define@key{fams}{sml}{Austronesian}
\define@key{fams}{zbc}{Austronesian}
\define@key{fams}{dtp}{Austronesian}
\define@key{fams}{awu}{Nuclear Trans New Guinea}
\define@key{fams}{ncx}{Uto-Aztecan}
\define@key{fams}{nch}{Uto-Aztecan}
\define@key{fams}{ojc}{Algic}
\define@key{fams}{pbs}{Otomanguean}
\define@key{fams}{quk}{Quechuan}
\define@key{fams}{cds}{Sign Language}
\define@key{fams}{cdy}{Tai-Kadai}
\define@key{fams}{chg}{Turkic}
\define@key{fams}{ciy}{Cariban}
\define@key{fams}{ccp}{Indo-European}
\define@key{fams}{ckh}{Sino-Tibetan}
\define@key{fams}{cli}{Atlantic-Congo}
\define@key{fams}{tgf}{Sino-Tibetan}
\define@key{fams}{cll}{Atlantic-Congo}
\define@key{fams}{cdh}{Indo-European}
\define@key{fams}{ceg}{Zamucoan}
\define@key{fams}{ccc}{Arawakan}
\define@key{fams}{cna}{Sino-Tibetan}
\define@key{fams}{cga}{Yuat}
\define@key{fams}{cra}{Ta-Ne-Omotic}
\define@key{fams}{crv}{Austroasiatic}
\define@key{fams}{xtb}{Otomanguean}
\define@key{fams}{ruk}{Atlantic-Congo}
\define@key{fams}{cde}{Dravidian}
\define@key{fams}{cjn}{Sepik}
\define@key{fams}{cnu}{Afro-Asiatic}
\define@key{fams}{ycp}{Sino-Tibetan}
\define@key{fams}{cpn}{Atlantic-Congo}
\define@key{fams}{ych}{Sino-Tibetan}
\define@key{fams}{cwg}{Austroasiatic}
\define@key{fams}{hne}{Indo-European}
\define@key{fams}{ctn}{Sino-Tibetan}
\define@key{fams}{cur}{Sino-Tibetan}
\define@key{fams}{csd}{Sign Language}
\define@key{fams}{cip}{Otomanguean}
\define@key{fams}{zpv}{Otomanguean}
\define@key{fams}{mii}{Otomanguean}
\define@key{fams}{csg}{Sign Language}
\define@key{fams}{clh}{Indo-European}
\define@key{fams}{clc}{Athabaskan-Eyak-Tlingit}
\define@key{fams}{csa}{Otomanguean}
\define@key{fams}{cpi}{Pidgin}
\define@key{fams}{chn}{Chinookan}
\define@key{fams}{cih}{Indo-European}
\define@key{fams}{bxu}{Mongolic-Khitan}
\define@key{fams}{cnb}{Sino-Tibetan}
\define@key{fams}{qxc}{Quechuan}
\define@key{fams}{cdf}{Sino-Tibetan}
\define@key{fams}{nhd}{Tupian}
\define@key{fams}{the}{Indo-European}
\define@key{fams}{cik}{Sino-Tibetan}
\define@key{fams}{zpc}{Otomanguean}
\define@key{fams}{cgk}{Sino-Tibetan}
\define@key{fams}{cdi}{Indo-European}
\define@key{fams}{nri}{Sino-Tibetan}
\define@key{fams}{cjk}{Atlantic-Congo}
\define@key{fams}{cda}{Sino-Tibetan}
\define@key{fams}{coh}{Atlantic-Congo}
\define@key{fams}{cce}{Atlantic-Congo}
\define@key{fams}{nct}{Sino-Tibetan}
\define@key{fams}{cvg}{Sino-Tibetan}
\define@key{fams}{cuw}{Sino-Tibetan}
\define@key{fams}{cuh}{Atlantic-Congo}
\define@key{fams}{chu}{Indo-European}
\define@key{fams}{cdj}{Indo-European}
\define@key{fams}{scb}{Austroasiatic}
\define@key{fams}{xcv}{Yukaghir}
\define@key{fams}{chw}{Atlantic-Congo}
\define@key{fams}{cia}{Austronesian}
\define@key{fams}{ckl}{Afro-Asiatic}
\define@key{fams}{awc}{Atlantic-Congo}
\define@key{fams}{cib}{Atlantic-Congo}
\define@key{fams}{cim}{Indo-European}
\define@key{fams}{mkx}{Austronesian}
\define@key{fams}{cdr}{Atlantic-Congo}
\define@key{fams}{cie}{Afro-Asiatic}
\define@key{fams}{cin}{Tupian}
\define@key{fams}{xcg}{Indo-European}
\define@key{fams}{asg}{Atlantic-Congo}
\define@key{fams}{txt}{Nuclear Trans New Guinea}
\define@key{fams}{tgd}{Afro-Asiatic}
\define@key{fams}{xcl}{Indo-European}
\define@key{fams}{nci}{Uto-Aztecan}
\define@key{fams}{qwc}{Quechuan}
\define@key{fams}{syc}{Afro-Asiatic}
\define@key{fams}{myz}{Afro-Asiatic}
\define@key{fams}{xct}{Sino-Tibetan}
\define@key{fams}{dri}{Atlantic-Congo}
\define@key{fams}{naz}{Uto-Aztecan}
\define@key{fams}{zps}{Otomanguean}
\define@key{fams}{zca}{Otomanguean}
\define@key{fams}{coj}{Cochimi-Yuman}
\define@key{fams}{coa}{Austronesian}
\define@key{fams}{liw}{Austronesian}
\define@key{fams}{csn}{Sign Language}
\define@key{fams}{gct}{Indo-European}
\define@key{fams}{cfg}{Atlantic-Congo}
\define@key{fams}{swc}{Atlantic-Congo}
\define@key{fams}{cnc}{Sino-Tibetan}
\define@key{fams}{coq}{Athabaskan-Eyak-Tlingit}
\define@key{fams}{cry}{Atlantic-Congo}
\define@key{fams}{qwa}{Quechuan}
\define@key{fams}{xxr}{Nuclear-Macro-Je}
\define@key{fams}{cos}{Indo-European}
\define@key{fams}{csr}{Sign Language}
\define@key{fams}{mta}{Austronesian}
\define@key{fams}{xcn}{Isolate}
\define@key{fams}{cow}{Salishan}
\define@key{fams}{toc}{Totonacan}
\define@key{fams}{gyn}{Indo-European}
\define@key{fams}{csq}{Sign Language}
\define@key{fams}{mfn}{Atlantic-Congo}
\define@key{fams}{crz}{Chumashan}
\define@key{fams}{csf}{Sign Language}
\define@key{fams}{cbq}{Atlantic-Congo}
\define@key{fams}{cuo}{Cariban}
\define@key{fams}{xlu}{Indo-European}
\define@key{fams}{cnq}{Atlantic-Congo}
\define@key{fams}{cuq}{Tai-Kadai}
\define@key{fams}{ccl}{Atlantic-Congo}
\define@key{fams}{cuv}{Afro-Asiatic}
\define@key{fams}{xtu}{Otomanguean}
\define@key{fams}{cyo}{Austronesian}
\define@key{fams}{bwy}{Atlantic-Congo}
\define@key{fams}{cse}{Sign Language}
\define@key{fams}{dao}{Sino-Tibetan}
\define@key{fams}{lni}{South Bougainville}
\define@key{fams}{dtn}{Gumuz}
\define@key{fams}{dbr}{Afro-Asiatic}
\define@key{fams}{dbe}{Tor-Orya}
\define@key{fams}{xdc}{Indo-European}
\define@key{fams}{dbd}{Atlantic-Congo}
\define@key{fams}{dgd}{Atlantic-Congo}
\define@key{fams}{dgk}{Central Sudanic}
\define@key{fams}{dec}{Narrow Talodi}
\define@key{fams}{dgn}{Yangmanic}
\define@key{fams}{dlk}{Afro-Asiatic}
\define@key{fams}{das}{Kru}
\define@key{fams}{dij}{Austronesian}
\define@key{fams}{drb}{Nubian}
\define@key{fams}{zhd}{Tai-Kadai}
\define@key{fams}{bpa}{Austronesian}
\define@key{fams}{dkk}{Austronesian}
\define@key{fams}{dka}{Sino-Tibetan}
\define@key{fams}{qer}{Indo-European}
\define@key{fams}{dlm}{Indo-European}
\define@key{fams}{dmm}{Atlantic-Congo}
\define@key{fams}{dam}{Atlantic-Congo}
\define@key{fams}{uhn}{Isolate}
\define@key{fams}{idb}{Indo-European}
\define@key{fams}{dac}{Austronesian}
\define@key{fams}{dml}{Indo-European}
\define@key{fams}{dms}{Austronesian}
\define@key{fams}{dnu}{Austroasiatic}
\define@key{fams}{dnr}{Nuclear Trans New Guinea}
\define@key{fams}{daq}{Dravidian}
\define@key{fams}{thl}{Indo-European}
\define@key{fams}{dsl}{Sign Language}
\define@key{fams}{daf}{Mande}
\define@key{fams}{aso}{Nuclear Trans New Guinea}
\define@key{fams}{gku}{Tuu}
\define@key{fams}{dnd}{Border}
\define@key{fams}{daz}{Nuclear Trans New Guinea}
\define@key{fams}{djc}{Dajuic}
\define@key{fams}{dln}{Sino-Tibetan}
\define@key{fams}{dro}{Austronesian}
\define@key{fams}{dot}{Afro-Asiatic}
\define@key{fams}{daw}{Austronesian}
\define@key{fams}{dww}{Austronesian}
\define@key{fams}{ddw}{Austronesian}
\define@key{fams}{dax}{Pama-Nyungan}
\define@key{fams}{dzg}{Saharan}
\define@key{fams}{dzd}{Unattested}
\define@key{fams}{ded}{Nuclear Trans New Guinea}
\define@key{fams}{gbh}{Atlantic-Congo}
\define@key{fams}{dge}{Nuclear Trans New Guinea}
\define@key{fams}{mzw}{Atlantic-Congo}
\define@key{fams}{deh}{Indo-European}
\define@key{fams}{dek}{Unattested}
\define@key{fams}{row}{Austronesian}
\define@key{fams}{ntr}{Atlantic-Congo}
\define@key{fams}{dmx}{Atlantic-Congo}
\define@key{fams}{dei}{Geelvink Bay}
\define@key{fams}{dem}{Isolate}
\define@key{fams}{dmy}{Sentanic}
\define@key{fams}{deq}{Atlantic-Congo}
\define@key{fams}{ddn}{Songhay}
\define@key{fams}{dez}{Atlantic-Congo}
\define@key{fams}{dnk}{Austronesian}
\define@key{fams}{dbb}{Afro-Asiatic}
\define@key{fams}{anv}{Atlantic-Congo}
\define@key{fams}{dee}{Kru}
\define@key{fams}{def}{Indo-European}
\define@key{fams}{dgh}{Afro-Asiatic}
\define@key{fams}{dhs}{Atlantic-Congo}
\define@key{fams}{dhn}{Indo-European}
\define@key{fams}{dwz}{Indo-European}
\define@key{fams}{nfa}{Austronesian}
\define@key{fams}{mki}{Indo-European}
\define@key{fams}{dho}{Indo-European}
\define@key{fams}{adf}{Afro-Asiatic}
\define@key{fams}{ddr}{Pama-Nyungan}
\define@key{fams}{dhd}{Indo-European}
\define@key{fams}{dia}{Nuclear Torricelli}
\define@key{fams}{mbd}{Austronesian}
\define@key{fams}{dby}{Isolate}
\define@key{fams}{dio}{Atlantic-Congo}
\define@key{fams}{duy}{Austronesian}
\define@key{fams}{dig}{Atlantic-Congo}
\define@key{fams}{cfa}{Atlantic-Congo}
\define@key{fams}{dil}{Nubian}
\define@key{fams}{jma}{Dagan}
\define@key{fams}{dii}{Atlantic-Congo}
\define@key{fams}{dmc}{Nuclear Trans New Guinea}
\define@key{fams}{ddi}{Austronesian}
\define@key{fams}{gdl}{Afro-Asiatic}
\define@key{fams}{diu}{Atlantic-Congo}
\define@key{fams}{dir}{Atlantic-Congo}
\define@key{fams}{dwa}{Afro-Asiatic}
\define@key{fams}{dsi}{Central Sudanic}
\define@key{fams}{tbz}{Atlantic-Congo}
\define@key{fams}{diy}{Nuclear Trans New Guinea}
\define@key{fams}{xtd}{Otomanguean}
\define@key{fams}{dix}{Austronesian}
\define@key{fams}{djf}{Pama-Nyungan}
\define@key{fams}{djn}{Gunwinyguan}
\define@key{fams}{djw}{Nyulnyulan}
\define@key{fams}{djb}{Pama-Nyungan}
\define@key{fams}{dze}{Pama-Nyungan}
\define@key{fams}{dob}{Austronesian}
\define@key{fams}{doe}{Atlantic-Congo}
\define@key{fams}{dgg}{Austronesian}
\define@key{fams}{dgx}{Nuclear Trans New Guinea}
\define@key{fams}{dgs}{Atlantic-Congo}
\define@key{fams}{dos}{Atlantic-Congo}
\define@key{fams}{dgr}{Athabaskan-Eyak-Tlingit}
\define@key{fams}{dbg}{Dogon}
\define@key{fams}{dbi}{Atlantic-Congo}
\define@key{fams}{uya}{Atlantic-Congo}
\define@key{fams}{dre}{Sino-Tibetan}
\define@key{fams}{dov}{Atlantic-Congo}
\define@key{fams}{doq}{Sign Language}
\define@key{fams}{doa}{Nuclear Trans New Guinea}
\define@key{fams}{doy}{Atlantic-Congo}
\define@key{fams}{dof}{Mailuan}
\define@key{fams}{dev}{Nuclear Trans New Guinea}
\define@key{fams}{dok}{Austronesian}
\define@key{fams}{yik}{Sino-Tibetan}
\define@key{fams}{doh}{Atlantic-Congo}
\define@key{fams}{ddd}{Nilotic}
\define@key{fams}{dde}{Atlantic-Congo}
\define@key{fams}{dor}{Austronesian}
\define@key{fams}{kqc}{Manubaran}
\define@key{fams}{doz}{Ta-Ne-Omotic}
\define@key{fams}{dol}{Doso-Turumsa}
\define@key{fams}{dty}{Indo-European}
\define@key{fams}{dup}{Austronesian}
\define@key{fams}{dva}{Austronesian}
\define@key{fams}{dub}{Indo-European}
\define@key{fams}{dmu}{Pauwasi}
\define@key{fams}{duk}{Nuclear Trans New Guinea}
\define@key{fams}{ndu}{Atlantic-Congo}
\define@key{fams}{dbm}{Atlantic-Congo}
\define@key{fams}{dme}{Afro-Asiatic}
\define@key{fams}{kbz}{Afro-Asiatic}
\define@key{fams}{nke}{Austronesian}
\define@key{fams}{dbo}{Atlantic-Congo}
\define@key{fams}{duz}{Atlantic-Congo}
\define@key{fams}{dmv}{Austronesian}
\define@key{fams}{wtf}{Nuclear Trans New Guinea}
\define@key{fams}{dui}{Nuclear Trans New Guinea}
\define@key{fams}{duh}{Indo-European}
\define@key{fams}{raa}{Sino-Tibetan}
\define@key{fams}{dng}{Sino-Tibetan}
\define@key{fams}{dbv}{Unattested}
\define@key{fams}{drq}{Sino-Tibetan}
\define@key{fams}{mvp}{Austronesian}
\define@key{fams}{dbn}{Inanwatan}
\define@key{fams}{dug}{Atlantic-Congo}
\define@key{fams}{dsn}{Austronesian}
\define@key{fams}{duw}{Austronesian}
\define@key{fams}{duq}{Austronesian}
\define@key{fams}{dun}{Austronesian}
\define@key{fams}{dws}{Artificial Language}
\define@key{fams}{dux}{Mande}
\define@key{fams}{dae}{Atlantic-Congo}
\define@key{fams}{duv}{Lakes Plain}
\define@key{fams}{dbp}{Afro-Asiatic}
\define@key{fams}{gve}{Austronesian}
\define@key{fams}{nnu}{Atlantic-Congo}
\define@key{fams}{dyb}{Nyulnyulan}
\define@key{fams}{dyn}{Pama-Nyungan}
\define@key{fams}{dya}{Atlantic-Congo}
\define@key{fams}{dyd}{Nyulnyulan}
\define@key{fams}{jen}{Atlantic-Congo}
\define@key{fams}{dzl}{Sino-Tibetan}
\define@key{fams}{dzn}{Atlantic-Congo}
\define@key{fams}{bpn}{Hmong-Mien}
\define@key{fams}{add}{Atlantic-Congo}
\define@key{fams}{dzo}{Sino-Tibetan}
\define@key{fams}{dnn}{Mande}
\define@key{fams}{ktv}{Austroasiatic}
\define@key{fams}{bgp}{Indo-European}
\define@key{fams}{lwl}{Austroasiatic}
\define@key{fams}{mng}{Austroasiatic}
\define@key{fams}{emu}{Dravidian}
\define@key{fams}{tge}{Sino-Tibetan}
\define@key{fams}{nos}{Sino-Tibetan}
\define@key{fams}{emq}{Sino-Tibetan}
\define@key{fams}{kif}{Sino-Tibetan}
\define@key{fams}{emg}{Sino-Tibetan}
\define@key{fams}{zeh}{Tai-Kadai}
\define@key{fams}{hmq}{Hmong-Mien}
\define@key{fams}{muq}{Hmong-Mien}
\define@key{fams}{hme}{Hmong-Mien}
\define@key{fams}{lma}{Atlantic-Congo}
\define@key{fams}{gbx}{Atlantic-Congo}
\define@key{fams}{xrb}{Atlantic-Congo}
\define@key{fams}{acp}{Atlantic-Congo}
\define@key{fams}{nle}{Atlantic-Congo}
\define@key{fams}{kqo}{Kru}
\define@key{fams}{vme}{Austronesian}
\define@key{fams}{tre}{Austronesian}
\define@key{fams}{dmr}{Austronesian}
\define@key{fams}{bnj}{Austronesian}
\define@key{fams}{pez}{Austronesian}
\define@key{fams}{zbe}{Austronesian}
\define@key{fams}{kjs}{Nuclear Trans New Guinea}
\define@key{fams}{nhe}{Uto-Aztecan}
\define@key{fams}{ojg}{Algic}
\define@key{fams}{aaq}{Algic}
\define@key{fams}{qve}{Quechuan}
\define@key{fams}{cly}{Otomanguean}
\define@key{fams}{avl}{Afro-Asiatic}
\define@key{fams}{sfe}{Austronesian}
\define@key{fams}{azd}{Uto-Aztecan}
\define@key{fams}{yit}{Sino-Tibetan}
\define@key{fams}{cek}{Sino-Tibetan}
\define@key{fams}{yol}{Indo-European}
\define@key{fams}{xeb}{Afro-Asiatic}
\define@key{fams}{ebr}{Atlantic-Congo}
\define@key{fams}{ebg}{Atlantic-Congo}
\define@key{fams}{ecs}{Sign Language}
\define@key{fams}{cbj}{Atlantic-Congo}
\define@key{fams}{idd}{Atlantic-Congo}
\define@key{fams}{ijj}{Atlantic-Congo}
\define@key{fams}{ica}{Atlantic-Congo}
\define@key{fams}{nqg}{Atlantic-Congo}
\define@key{fams}{awy}{Nuclear Trans New Guinea}
\define@key{fams}{dbf}{Lakes Plain}
\define@key{fams}{eee}{Tai-Kadai}
\define@key{fams}{efa}{Atlantic-Congo}
\define@key{fams}{efe}{Central Sudanic}
\define@key{fams}{ofu}{Atlantic-Congo}
\define@key{fams}{ego}{Atlantic-Congo}
\define@key{fams}{esl}{Sign Language}
\define@key{fams}{egy}{Afro-Asiatic}
\define@key{fams}{ehu}{Atlantic-Congo}
\define@key{fams}{eit}{Nuclear Torricelli}
\define@key{fams}{eja}{Atlantic-Congo}
\define@key{fams}{eka}{Atlantic-Congo}
\define@key{fams}{eki}{Atlantic-Congo}
\define@key{fams}{eke}{Atlantic-Congo}
\define@key{fams}{ekp}{Atlantic-Congo}
\define@key{fams}{zpp}{Otomanguean}
\define@key{fams}{elx}{Isolate}
\define@key{fams}{elm}{Atlantic-Congo}
\define@key{fams}{ele}{Nuclear Torricelli}
\define@key{fams}{elh}{Nubian}
\define@key{fams}{ekm}{Atlantic-Congo}
\define@key{fams}{elk}{Nuclear Torricelli}
\define@key{fams}{elo}{Afro-Asiatic}
\define@key{fams}{zte}{Otomanguean}
\define@key{fams}{afo}{Atlantic-Congo}
\define@key{fams}{elu}{Austronesian}
\define@key{fams}{xly}{Unclassifiable}
\define@key{fams}{yzg}{Tai-Kadai}
\define@key{fams}{emn}{Atlantic-Congo}
\define@key{fams}{bdc}{Chocoan}
\define@key{fams}{tdc}{Chocoan}
\define@key{fams}{ebu}{Atlantic-Congo}
\define@key{fams}{emw}{Austronesian}
\define@key{fams}{enr}{Pauwasi}
\define@key{fams}{unk}{Arawakan}
\define@key{fams}{end}{Austronesian}
\define@key{fams}{enc}{Tai-Kadai}
\define@key{fams}{ptt}{Austronesian}
\define@key{fams}{enu}{Sino-Tibetan}
\define@key{fams}{enw}{Atlantic-Congo}
\define@key{fams}{env}{Atlantic-Congo}
\define@key{fams}{epi}{Atlantic-Congo}
\define@key{fams}{emy}{Mayan}
\define@key{fams}{era}{Dravidian}
\define@key{fams}{kjy}{Nuclear Trans New Guinea}
\define@key{fams}{twp}{Austronesian}
\define@key{fams}{ert}{Lakes Plain}
\define@key{fams}{erw}{Austronesian}
\define@key{fams}{err}{Giimbiyu}
\define@key{fams}{emx}{Speech Register}
\define@key{fams}{ers}{Sino-Tibetan}
\define@key{fams}{erh}{Atlantic-Congo}
\define@key{fams}{ish}{Atlantic-Congo}
\define@key{fams}{mcq}{Koiarian}
\define@key{fams}{esh}{Indo-European}
\define@key{fams}{ags}{Atlantic-Congo}
\define@key{fams}{esy}{Artificial Language}
\define@key{fams}{epo}{Artificial Language}
\define@key{fams}{ots}{Otomanguean}
\define@key{fams}{eso}{Sign Language}
\define@key{fams}{esm}{Unattested}
\define@key{fams}{etb}{Atlantic-Congo}
\define@key{fams}{etx}{Atlantic-Congo}
\define@key{fams}{ecr}{Unclassifiable}
\define@key{fams}{ecy}{Unclassifiable}
\define@key{fams}{eth}{Sign Language}
\define@key{fams}{ich}{Atlantic-Congo}
\define@key{fams}{eto}{Atlantic-Congo}
\define@key{fams}{etn}{Austronesian}
\define@key{fams}{ett}{Isolate}
\define@key{fams}{utr}{Atlantic-Congo}
\define@key{fams}{bzz}{Atlantic-Congo}
\define@key{fams}{gev}{Atlantic-Congo}
\define@key{fams}{nou}{Nuclear Trans New Guinea}
\define@key{fams}{ext}{Indo-European}
\define@key{fams}{fab}{Indo-European}
\define@key{fams}{faf}{Austronesian}
\define@key{fams}{fif}{Afro-Asiatic}
\define@key{fams}{azt}{Austronesian}
\define@key{fams}{faj}{Nuclear Trans New Guinea}
\define@key{fams}{fai}{Nuclear Trans New Guinea}
\define@key{fams}{fax}{Indo-European}
\define@key{fams}{cfm}{Sino-Tibetan}
\define@key{fams}{fli}{Afro-Asiatic}
\define@key{fams}{xfa}{Indo-European}
\define@key{fams}{fam}{Atlantic-Congo}
\define@key{fams}{fng}{Pidgin}
\define@key{fams}{fan}{Atlantic-Congo}
\define@key{fams}{fak}{Atlantic-Congo}
\define@key{fams}{fni}{Atlantic-Congo}
\define@key{fams}{nsf}{Sino-Tibetan}
\define@key{fams}{fmu}{Dravidian}
\define@key{fams}{far}{Austronesian}
\define@key{fams}{ddg}{Timor-Alor-Pantar}
\define@key{fams}{fau}{Lakes Plain}
\define@key{fams}{agl}{East Strickland}
\define@key{fams}{fpe}{Indo-European}
\define@key{fams}{fer}{Atlantic-Congo}
\define@key{fams}{hif}{Indo-European}
\define@key{fams}{fil}{Austronesian}
\define@key{fams}{tlp}{Totonacan}
\define@key{fams}{bkb}{Austronesian}
\define@key{fams}{fss}{Sign Language}
\define@key{fams}{fag}{Nuclear Trans New Guinea}
\define@key{fams}{fip}{Atlantic-Congo}
\define@key{fams}{fir}{Atlantic-Congo}
\define@key{fams}{fiw}{East Kutubu}
\define@key{fams}{fln}{Pama-Nyungan}
\define@key{fams}{flh}{Lakes Plain}
\define@key{fams}{fod}{Atlantic-Congo}
\define@key{fams}{frq}{Nuclear Trans New Guinea}
\define@key{fams}{enf}{Uralic}
\define@key{fams}{frt}{Austronesian}
\define@key{fams}{frp}{Indo-European}
\define@key{fams}{fur}{Indo-European}
\define@key{fams}{flr}{Atlantic-Congo}
\define@key{fams}{ula}{Atlantic-Congo}
\define@key{fams}{fuy}{Goilalan}
\define@key{fams}{fwe}{Atlantic-Congo}
\define@key{fams}{fie}{Afro-Asiatic}
\define@key{fams}{ttb}{Atlantic-Congo}
\define@key{fams}{gie}{Kru}
\define@key{fams}{gab}{Afro-Asiatic}
\define@key{fams}{gdg}{Austronesian}
\define@key{fams}{gdk}{Afro-Asiatic}
\define@key{fams}{gbk}{Indo-European}
\define@key{fams}{gad}{Austronesian}
\define@key{fams}{gda}{Indo-European}
\define@key{fams}{gdh}{Jarrakan}
\define@key{fams}{gft}{Afro-Asiatic}
\define@key{fams}{btg}{Kru}
\define@key{fams}{ggu}{Mande}
\define@key{fams}{gbf}{Ndu}
\define@key{fams}{gic}{Unclassifiable}
\define@key{fams}{gcn}{Nuclear Trans New Guinea}
\define@key{fams}{xga}{Indo-European}
\define@key{fams}{glo}{Afro-Asiatic}
\define@key{fams}{gar}{Austronesian}
\define@key{fams}{gce}{Athabaskan-Eyak-Tlingit}
\define@key{fams}{sdn}{Indo-European}
\define@key{fams}{gap}{Nuclear Trans New Guinea}
\define@key{fams}{gal}{Austronesian}
\define@key{fams}{kgj}{Sino-Tibetan}
\define@key{fams}{gma}{Worrorran}
\define@key{fams}{wof}{Atlantic-Congo}
\define@key{fams}{gbl}{Indo-European}
\define@key{fams}{gak}{North Halmahera}
\define@key{fams}{bte}{Atlantic-Congo}
\define@key{fams}{ihw}{Pama-Nyungan}
\define@key{fams}{gne}{Atlantic-Congo}
\define@key{fams}{gnk}{Khoe-Kwadi}
\define@key{fams}{gnq}{Austronesian}
\define@key{fams}{unn}{Pama-Nyungan}
\define@key{fams}{gan}{Sino-Tibetan}
\define@key{fams}{pgd}{Indo-European}
\define@key{fams}{gzn}{Austronesian}
\define@key{fams}{gnb}{Sino-Tibetan}
\define@key{fams}{gnl}{Pama-Nyungan}
\define@key{fams}{ggl}{Nuclear Trans New Guinea}
\define@key{fams}{gao}{Nuclear Trans New Guinea}
\define@key{fams}{gza}{Blue Nile Mao}
\define@key{fams}{gnz}{Atlantic-Congo}
\define@key{fams}{gga}{Austronesian}
\define@key{fams}{gbm}{Indo-European}
\define@key{fams}{ilg}{Iwaidjan Proper}
\define@key{fams}{gex}{Afro-Asiatic}
\define@key{fams}{gaq}{Austroasiatic}
\define@key{fams}{gou}{Afro-Asiatic}
\define@key{fams}{gwt}{Indo-European}
\define@key{fams}{gyl}{South Omotic}
\define@key{fams}{gzi}{Indo-European}
\define@key{fams}{gbg}{Atlantic-Congo}
\define@key{fams}{gbv}{Atlantic-Congo}
\define@key{fams}{gby}{Atlantic-Congo}
\define@key{fams}{gyg}{Atlantic-Congo}
\define@key{fams}{gbq}{Atlantic-Congo}
\define@key{fams}{gbs}{Atlantic-Congo}
\define@key{fams}{ggb}{Kru}
\define@key{fams}{xgb}{Mande}
\define@key{fams}{grh}{Atlantic-Congo}
\define@key{fams}{gec}{Kru}
\define@key{fams}{kvq}{Sino-Tibetan}
\define@key{fams}{gei}{Austronesian}
\define@key{fams}{gdd}{Austronesian}
\define@key{fams}{drs}{Afro-Asiatic}
\define@key{fams}{hmj}{Hmong-Mien}
\define@key{fams}{gez}{Afro-Asiatic}
\define@key{fams}{ghk}{Sino-Tibetan}
\define@key{fams}{giu}{Tai-Kadai}
\define@key{fams}{geq}{Atlantic-Congo}
\define@key{fams}{gaf}{Nuclear Trans New Guinea}
\define@key{fams}{gej}{Atlantic-Congo}
\define@key{fams}{ygp}{Sino-Tibetan}
\define@key{fams}{gew}{Afro-Asiatic}
\define@key{fams}{gea}{Afro-Asiatic}
\define@key{fams}{ges}{Austronesian}
\define@key{fams}{gha}{Afro-Asiatic}
\define@key{fams}{gse}{Sign Language}
\define@key{fams}{ghn}{Austronesian}
\define@key{fams}{gpe}{Indo-European}
\define@key{fams}{gds}{Sign Language}
\define@key{fams}{gri}{Austronesian}
\define@key{fams}{ajs}{Sign Language}
\define@key{fams}{bmk}{Austronesian}
\define@key{fams}{aln}{Indo-European}
\define@key{fams}{ghr}{Indo-European}
\define@key{fams}{bbj}{Atlantic-Congo}
\define@key{fams}{gho}{Afro-Asiatic}
\define@key{fams}{bgi}{Austronesian}
\define@key{fams}{gib}{Pidgin}
\define@key{fams}{kks}{Afro-Asiatic}
\define@key{fams}{acd}{Atlantic-Congo}
\define@key{fams}{gix}{Atlantic-Congo}
\define@key{fams}{gip}{Austronesian}
\define@key{fams}{gim}{Nuclear Trans New Guinea}
\define@key{fams}{kmp}{Atlantic-Congo}
\define@key{fams}{gmn}{Atlantic-Congo}
\define@key{fams}{gnm}{Dagan}
\define@key{fams}{ayg}{Atlantic-Congo}
\define@key{fams}{bbr}{Nuclear Trans New Guinea}
\define@key{fams}{gii}{Afro-Asiatic}
\define@key{fams}{nyf}{Atlantic-Congo}
\define@key{fams}{toh}{Atlantic-Congo}
\define@key{fams}{ggt}{Austronesian}
\define@key{fams}{giy}{Unattested}
\define@key{fams}{tof}{Eastern Trans-Fly}
\define@key{fams}{glr}{Kru}
\define@key{fams}{glw}{Afro-Asiatic}
\define@key{fams}{oub}{Kru}
\define@key{fams}{gnu}{Nuclear Torricelli}
\define@key{fams}{gom}{Indo-European}
\define@key{fams}{gig}{Indo-European}
\define@key{fams}{goi}{East Strickland}
\define@key{fams}{gox}{Atlantic-Congo}
\define@key{fams}{gdx}{Indo-European}
\define@key{fams}{gof}{Ta-Ne-Omotic}
\define@key{fams}{gog}{Atlantic-Congo}
\define@key{fams}{goo}{Austronesian}
\define@key{fams}{goe}{Sino-Tibetan}
\define@key{fams}{gjn}{Atlantic-Congo}
\define@key{fams}{gov}{Mande}
\define@key{fams}{goq}{Austronesian}
\define@key{fams}{goc}{Austronesian}
\define@key{fams}{grq}{Lower Sepik-Ramu}
\define@key{fams}{gqr}{Central Sudanic}
\define@key{fams}{got}{Indo-European}
\define@key{fams}{goy}{Atlantic-Congo}
\define@key{fams}{gwf}{Indo-European}
\define@key{fams}{goz}{Indo-European}
\define@key{fams}{nli}{Indo-European}
\define@key{fams}{giq}{Tai-Kadai}
\define@key{fams}{gcl}{Indo-European}
\define@key{fams}{grs}{Nimboranic}
\define@key{fams}{gro}{Sino-Tibetan}
\define@key{fams}{gos}{Indo-European}
\define@key{fams}{ats}{Algic}
\define@key{fams}{gwx}{Atlantic-Congo}
\define@key{fams}{gvj}{Tupian}
\define@key{fams}{jiq}{Sino-Tibetan}
\define@key{fams}{gnc}{Afro-Asiatic}
\define@key{fams}{gyr}{Tupian}
\define@key{fams}{gsm}{Sign Language}
\define@key{fams}{xgd}{Pama-Nyungan}
\define@key{fams}{gdu}{Afro-Asiatic}
\define@key{fams}{zpg}{Otomanguean}
\define@key{fams}{gdc}{Pama-Nyungan}
\define@key{fams}{kkp}{Pama-Nyungan}
\define@key{fams}{wrw}{Pama-Nyungan}
\define@key{fams}{zgn}{Tai-Kadai}
\define@key{fams}{bet}{Kru}
\define@key{fams}{ztu}{Otomanguean}
\define@key{fams}{gus}{Sign Language}
\define@key{fams}{gkp}{Mande}
\define@key{fams}{gqi}{Sino-Tibetan}
\define@key{fams}{gvl}{Central Sudanic}
\define@key{fams}{glu}{Central Sudanic}
\define@key{fams}{gmb}{Austronesian}
\define@key{fams}{gly}{Isolate}
\define@key{fams}{gul}{Indo-European}
\define@key{fams}{gmu}{Nuclear Trans New Guinea}
\define@key{fams}{gdi}{Atlantic-Congo}
\define@key{fams}{gyf}{Pama-Nyungan}
\define@key{fams}{rub}{Atlantic-Congo}
\define@key{fams}{gnt}{Yam}
\define@key{fams}{gpa}{Atlantic-Congo}
\define@key{fams}{grz}{Austronesian}
\define@key{fams}{gdj}{Pama-Nyungan}
\define@key{fams}{ggg}{Indo-European}
\define@key{fams}{grx}{Isolate}
\define@key{fams}{gjr}{Mixed Language}
\define@key{fams}{gvm}{Atlantic-Congo}
\define@key{fams}{gvr}{Sino-Tibetan}
\define@key{fams}{grd}{Afro-Asiatic}
\define@key{fams}{gsn}{Nuclear Trans New Guinea}
\define@key{fams}{gsl}{Atlantic-Congo}
\define@key{fams}{xgw}{Pama-Nyungan}
\define@key{fams}{gwu}{Pama-Nyungan}
\define@key{fams}{gvy}{Pama-Nyungan}
\define@key{fams}{gka}{Nuclear Trans New Guinea}
\define@key{fams}{ngs}{Afro-Asiatic}
\define@key{fams}{gwb}{Atlantic-Congo}
\define@key{fams}{dah}{Nuclear Trans New Guinea}
\define@key{fams}{bga}{Atlantic-Congo}
\define@key{fams}{gwn}{Afro-Asiatic}
\define@key{fams}{grw}{Austronesian}
\define@key{fams}{gwe}{Atlantic-Congo}
\define@key{fams}{gwr}{Atlantic-Congo}
\define@key{fams}{gwj}{Khoe-Kwadi}
\define@key{fams}{gyi}{Atlantic-Congo}
\define@key{fams}{gye}{Atlantic-Congo}
\define@key{fams}{haq}{Atlantic-Congo}
\define@key{fams}{hbu}{Austronesian}
\define@key{fams}{hdy}{Afro-Asiatic}
\define@key{fams}{hoj}{Indo-European}
\define@key{fams}{xhd}{Afro-Asiatic}
\define@key{fams}{ayh}{Afro-Asiatic}
\define@key{fams}{aek}{Austronesian}
\define@key{fams}{hah}{Austronesian}
\define@key{fams}{hgw}{Austronesian}
\define@key{fams}{bzx}{Mande}
\define@key{fams}{hgm}{Khoe-Kwadi}
\define@key{fams}{haf}{Sign Language}
\define@key{fams}{hvc}{Unclassifiable}
\define@key{fams}{hji}{Austronesian}
\define@key{fams}{haj}{Indo-European}
\define@key{fams}{hao}{Austronesian}
\define@key{fams}{hld}{Austroasiatic}
\define@key{fams}{hmu}{Timor-Alor-Pantar}
\define@key{fams}{hba}{Atlantic-Congo}
\define@key{fams}{hag}{Atlantic-Congo}
\define@key{fams}{han}{Atlantic-Congo}
\define@key{fams}{haa}{Athabaskan-Eyak-Tlingit}
\define@key{fams}{hab}{Sign Language}
\define@key{fams}{xiv}{Unattested}
\define@key{fams}{kjo}{Indo-European}
\define@key{fams}{hro}{Austronesian}
\define@key{fams}{hrk}{Austronesian}
\define@key{fams}{bgc}{Indo-European}
\define@key{fams}{hrz}{Indo-European}
\define@key{fams}{ybj}{Atlantic-Congo}
\define@key{fams}{xht}{Isolate}
\define@key{fams}{hsl}{Sign Language}
\define@key{fams}{hvk}{Austronesian}
\define@key{fams}{hav}{Atlantic-Congo}
\define@key{fams}{hps}{Sign Language}
\define@key{fams}{xda}{Pama-Nyungan}
\define@key{fams}{haz}{Indo-European}
\define@key{fams}{hbn}{Heibanic}
\define@key{fams}{scp}{Sino-Tibetan}
\define@key{fams}{heg}{Austronesian}
\define@key{fams}{nix}{Atlantic-Congo}
\define@key{fams}{hed}{Afro-Asiatic}
\define@key{fams}{llf}{Austronesian}
\define@key{fams}{hrt}{Afro-Asiatic}
\define@key{fams}{ham}{Sepik}
\define@key{fams}{auk}{Nuclear Torricelli}
\define@key{fams}{hib}{Hibito-Cholon}
\define@key{fams}{hlu}{Indo-European}
\define@key{fams}{mba}{Austronesian}
\define@key{fams}{kjk}{Austronesian}
\define@key{fams}{hij}{Atlantic-Congo}
\define@key{fams}{hir}{Unattested}
\define@key{fams}{hii}{Indo-European}
\define@key{fams}{hmo}{Pidgin}
\define@key{fams}{hit}{Indo-European}
\define@key{fams}{htu}{Austronesian}
\define@key{fams}{hiw}{Austronesian}
\define@key{fams}{yhl}{Sino-Tibetan}
\define@key{fams}{hle}{Sino-Tibetan}
\define@key{fams}{hmf}{Hmong-Mien}
\define@key{fams}{hmz}{Hmong-Mien}
\define@key{fams}{hmv}{Hmong-Mien}
\define@key{fams}{mrk}{Austronesian}
\define@key{fams}{hoh}{Afro-Asiatic}
\define@key{fams}{hos}{Sign Language}
\define@key{fams}{hhi}{Anim}
\define@key{fams}{hoy}{Dravidian}
\define@key{fams}{hoi}{Athabaskan-Eyak-Tlingit}
\define@key{fams}{hod}{Afro-Asiatic}
\define@key{fams}{hol}{Atlantic-Congo}
\define@key{fams}{hom}{Atlantic-Congo}
\define@key{fams}{hds}{Sign Language}
\define@key{fams}{juh}{Atlantic-Congo}
\define@key{fams}{how}{Sino-Tibetan}
\define@key{fams}{hrm}{Hmong-Mien}
\define@key{fams}{hoe}{Atlantic-Congo}
\define@key{fams}{hor}{Central Sudanic}
\define@key{fams}{ero}{Sino-Tibetan}
\define@key{fams}{hot}{Austronesian}
\define@key{fams}{hti}{Austronesian}
\define@key{fams}{hov}{Austronesian}
\define@key{fams}{hhy}{Anim}
\define@key{fams}{hoz}{Blue Nile Mao}
\define@key{fams}{hpo}{Sino-Tibetan}
\define@key{fams}{hra}{Sino-Tibetan}
\define@key{fams}{hru}{Isolate}
\define@key{fams}{hug}{Harakmbut}
\define@key{fams}{qvh}{Quechuan}
\define@key{fams}{hud}{Austronesian}
\define@key{fams}{nhq}{Uto-Aztecan}
\define@key{fams}{qwh}{Quechuan}
\define@key{fams}{qvw}{Quechuan}
\define@key{fams}{huh}{Araucanian}
\define@key{fams}{mxs}{Otomanguean}
\define@key{fams}{czh}{Sino-Tibetan}
\define@key{fams}{huw}{Austronesian}
\define@key{fams}{hul}{Austronesian}
\define@key{fams}{huy}{Afro-Asiatic}
\define@key{fams}{hui}{Nuclear Trans New Guinea}
\define@key{fams}{huk}{Austronesian}
\define@key{fams}{hmb}{Songhay}
\define@key{fams}{huf}{Kwalean}
\define@key{fams}{hut}{Sino-Tibetan}
\define@key{fams}{hsh}{Sign Language}
\define@key{fams}{hnu}{Austroasiatic}
\define@key{fams}{nat}{Atlantic-Congo}
\define@key{fams}{hum}{Atlantic-Congo}
\define@key{fams}{hng}{Atlantic-Congo}
\define@key{fams}{hkk}{Nuclear Trans New Guinea}
\define@key{fams}{hap}{Nuclear Trans New Guinea}
\define@key{fams}{xhu}{Hurro-Urartian}
\define@key{fams}{geh}{Indo-European}
\define@key{fams}{huo}{Austroasiatic}
\define@key{fams}{hwo}{Afro-Asiatic}
\define@key{fams}{hya}{Afro-Asiatic}
\define@key{fams}{jab}{Atlantic-Congo}
\define@key{fams}{yml}{Austronesian}
\define@key{fams}{tek}{Atlantic-Congo}
\define@key{fams}{ibl}{Austronesian}
\define@key{fams}{iby}{Ijoid}
\define@key{fams}{xib}{Isolate}
\define@key{fams}{ibn}{Atlantic-Congo}
\define@key{fams}{ibr}{Atlantic-Congo}
\define@key{fams}{ibu}{North Halmahera}
\define@key{fams}{bec}{Atlantic-Congo}
\define@key{fams}{ida}{Atlantic-Congo}
\define@key{fams}{idt}{Austronesian}
\define@key{fams}{ide}{Atlantic-Congo}
\define@key{fams}{idi}{Pahoturi}
\define@key{fams}{idc}{Atlantic-Congo}
\define@key{fams}{ido}{Artificial Language}
\define@key{fams}{ldb}{Atlantic-Congo}
\define@key{fams}{ife}{Atlantic-Congo}
\define@key{fams}{iff}{Austronesian}
\define@key{fams}{igl}{Atlantic-Congo}
\define@key{fams}{igg}{Lower Sepik-Ramu}
\define@key{fams}{ahl}{Atlantic-Congo}
\define@key{fams}{nar}{Atlantic-Congo}
\define@key{fams}{igw}{Atlantic-Congo}
\define@key{fams}{ihb}{Pidgin}
\define@key{fams}{ikk}{Atlantic-Congo}
\define@key{fams}{ikr}{Pama-Nyungan}
\define@key{fams}{ikz}{Atlantic-Congo}
\define@key{fams}{meb}{Turama-Kikori}
\define@key{fams}{ntk}{Atlantic-Congo}
\define@key{fams}{iki}{Atlantic-Congo}
\define@key{fams}{ikp}{Atlantic-Congo}
\define@key{fams}{txi}{Cariban}
\define@key{fams}{ikv}{Atlantic-Congo}
\define@key{fams}{ikl}{Atlantic-Congo}
\define@key{fams}{ikw}{Atlantic-Congo}
\define@key{fams}{ila}{Austronesian}
\define@key{fams}{mbi}{Austronesian}
\define@key{fams}{ili}{Turkic}
\define@key{fams}{ilu}{Austronesian}
\define@key{fams}{xil}{Unclassifiable}
\define@key{fams}{ilk}{Austronesian}
\define@key{fams}{ilv}{Atlantic-Congo}
\define@key{fams}{mlk}{Atlantic-Congo}
\define@key{fams}{imo}{Nuclear Trans New Guinea}
\define@key{fams}{arc}{Afro-Asiatic}
\define@key{fams}{imr}{Austronesian}
\define@key{fams}{abx}{Austronesian}
\define@key{fams}{mzu}{Lower Sepik-Ramu}
\define@key{fams}{inp}{Arawakan}
\define@key{fams}{smn}{Uralic}
\define@key{fams}{inl}{Sign Language}
\define@key{fams}{idr}{Atlantic-Congo}
\define@key{fams}{mvy}{Indo-European}
\define@key{fams}{oin}{Nuclear Torricelli}
\define@key{fams}{iti}{Austronesian}
\define@key{fams}{ino}{Nuclear Trans New Guinea}
\define@key{fams}{loc}{Austronesian}
\define@key{fams}{ior}{Afro-Asiatic}
\define@key{fams}{ina}{Artificial Language}
\define@key{fams}{ile}{Artificial Language}
\define@key{fams}{igs}{Artificial Language}
\define@key{fams}{int}{Sino-Tibetan}
\define@key{fams}{iks}{Sign Language}
\define@key{fams}{azm}{Otomanguean}
\define@key{fams}{ipo}{Anim}
\define@key{fams}{ipi}{Nuclear Trans New Guinea}
\define@key{fams}{ass}{Atlantic-Congo}
\define@key{fams}{ill}{Austronesian}
\define@key{fams}{iry}{Austronesian}
\define@key{fams}{ire}{Austronesian}
\define@key{fams}{iri}{Atlantic-Congo}
\define@key{fams}{bto}{Austronesian}
\define@key{fams}{iru}{Dravidian}
\define@key{fams}{isa}{Nuclear Trans New Guinea}
\define@key{fams}{isn}{Atlantic-Congo}
\define@key{fams}{agk}{Austronesian}
\define@key{fams}{isc}{Pano-Tacanan}
\define@key{fams}{igo}{Nuclear Trans New Guinea}
\define@key{fams}{inn}{Austronesian}
\define@key{fams}{crb}{Arawakan}
\define@key{fams}{mir}{Mixe-Zoque}
\define@key{fams}{nhk}{Uto-Aztecan}
\define@key{fams}{ist}{Indo-European}
\define@key{fams}{ruo}{Indo-European}
\define@key{fams}{szv}{Atlantic-Congo}
\define@key{fams}{isu}{Atlantic-Congo}
\define@key{fams}{ite}{Chapacuran}
\define@key{fams}{itr}{Left May}
\define@key{fams}{itx}{Tor-Orya}
\define@key{fams}{itw}{Atlantic-Congo}
\define@key{fams}{itm}{Atlantic-Congo}
\define@key{fams}{mce}{Otomanguean}
\define@key{fams}{ivv}{Austronesian}
\define@key{fams}{atg}{Atlantic-Congo}
\define@key{fams}{iwk}{Austronesian}
\define@key{fams}{kbm}{Austronesian}
\define@key{fams}{iwo}{Nuclear Trans New Guinea}
\define@key{fams}{mzi}{Otomanguean}
\define@key{fams}{vmj}{Otomanguean}
\define@key{fams}{iya}{Atlantic-Congo}
\define@key{fams}{uiv}{Atlantic-Congo}
\define@key{fams}{crt}{Matacoan}
\define@key{fams}{nca}{Nuclear Trans New Guinea}
\define@key{fams}{crq}{Matacoan}
\define@key{fams}{izi}{Atlantic-Congo}
\define@key{fams}{cbo}{Atlantic-Congo}
\define@key{fams}{rzh}{Afro-Asiatic}
\define@key{fams}{jdg}{Indo-European}
\define@key{fams}{jad}{Mande}
\define@key{fams}{jah}{Austroasiatic}
\define@key{fams}{awv}{Nuclear Trans New Guinea}
\define@key{fams}{jat}{Indo-European}
\define@key{fams}{jak}{Austronesian}
\define@key{fams}{maj}{Otomanguean}
\define@key{fams}{bxl}{Mande}
\define@key{fams}{jcs}{Sign Language}
\define@key{fams}{jls}{Sign Language}
\define@key{fams}{jax}{Austronesian}
\define@key{fams}{jnd}{Indo-European}
\define@key{fams}{jna}{Sino-Tibetan}
\define@key{fams}{djo}{Austronesian}
\define@key{fams}{jni}{Atlantic-Congo}
\define@key{fams}{jar}{Atlantic-Congo}
\define@key{fams}{jra}{Austronesian}
\define@key{fams}{jaf}{Afro-Asiatic}
\define@key{fams}{qxw}{Quechuan}
\define@key{fams}{jns}{Indo-European}
\define@key{fams}{jvd}{Indo-European}
\define@key{fams}{jaz}{Austronesian}
\define@key{fams}{jyy}{Central Sudanic}
\define@key{fams}{jje}{Koreanic}
\define@key{fams}{bze}{Mande}
\define@key{fams}{xuj}{Dravidian}
\define@key{fams}{jer}{Atlantic-Congo}
\define@key{fams}{jee}{Sino-Tibetan}
\define@key{fams}{tmr}{Afro-Asiatic}
\define@key{fams}{jhs}{Sign Language}
\define@key{fams}{jio}{Tai-Kadai}
\define@key{fams}{juo}{Atlantic-Congo}
\define@key{fams}{jib}{Atlantic-Congo}
\define@key{fams}{jii}{Afro-Asiatic}
\define@key{fams}{jie}{Afro-Asiatic}
\define@key{fams}{jil}{Nuclear Trans New Guinea}
\define@key{fams}{jim}{Afro-Asiatic}
\define@key{fams}{jmi}{Afro-Asiatic}
\define@key{fams}{jia}{Afro-Asiatic}
\define@key{fams}{cjy}{Sino-Tibetan}
\define@key{fams}{pnu}{Hmong-Mien}
\define@key{fams}{jul}{Sino-Tibetan}
\define@key{fams}{jrr}{Atlantic-Congo}
\define@key{fams}{jit}{Atlantic-Congo}
\define@key{fams}{kaj}{Atlantic-Congo}
\define@key{fams}{job}{Atlantic-Congo}
\define@key{fams}{jbr}{Tor-Orya}
\define@key{fams}{jeu}{Afro-Asiatic}
\define@key{fams}{jor}{Tupian}
\define@key{fams}{jrt}{Afro-Asiatic}
\define@key{fams}{jow}{Mande}
\define@key{fams}{itk}{Indo-European}
\define@key{fams}{jdt}{Indo-European}
\define@key{fams}{jpr}{Indo-European}
\define@key{fams}{yud}{Afro-Asiatic}
\define@key{fams}{aju}{Afro-Asiatic}
\define@key{fams}{yhd}{Afro-Asiatic}
\define@key{fams}{jye}{Afro-Asiatic}
\define@key{fams}{jum}{Nilotic}
\define@key{fams}{jml}{Indo-European}
\define@key{fams}{jus}{Sign Language}
\define@key{fams}{mxq}{Mixe-Zoque}
\define@key{fams}{juy}{Austroasiatic}
\define@key{fams}{jut}{Indo-European}
\define@key{fams}{juu}{Afro-Asiatic}
\define@key{fams}{mwb}{Nuclear Torricelli}
\define@key{fams}{vmc}{Otomanguean}
\define@key{fams}{jwi}{Atlantic-Congo}
\define@key{fams}{xku}{Atlantic-Congo}
\define@key{fams}{gna}{Atlantic-Congo}
\define@key{fams}{ldl}{Atlantic-Congo}
\define@key{fams}{ckn}{Sino-Tibetan}
\define@key{fams}{ksp}{Central Sudanic}
\define@key{fams}{kvf}{Afro-Asiatic}
\define@key{fams}{gbw}{Pama-Nyungan}
\define@key{fams}{klz}{Timor-Alor-Pantar}
\define@key{fams}{onk}{Nuclear Torricelli}
\define@key{fams}{lkb}{Atlantic-Congo}
\define@key{fams}{uka}{South Bird's Head Family}
\define@key{fams}{kbu}{Indo-European}
\define@key{fams}{kea}{Indo-European}
\define@key{fams}{cwa}{Atlantic-Congo}
\define@key{fams}{kcw}{Atlantic-Congo}
\define@key{fams}{gjk}{Indo-European}
\define@key{fams}{kfr}{Indo-European}
\define@key{fams}{kcx}{Ta-Ne-Omotic}
\define@key{fams}{xkk}{Austroasiatic}
\define@key{fams}{kej}{Dravidian}
\define@key{fams}{kdu}{Nubian}
\define@key{fams}{kad}{Atlantic-Congo}
\define@key{fams}{kzd}{Austronesian}
\define@key{fams}{kdv}{Sino-Tibetan}
\define@key{fams}{ktp}{Sino-Tibetan}
\define@key{fams}{jka}{Timor-Alor-Pantar}
\define@key{fams}{kpu}{Timor-Alor-Pantar}
\define@key{fams}{sqx}{Sign Language}
\define@key{fams}{syw}{Sino-Tibetan}
\define@key{fams}{kll}{Austronesian}
\define@key{fams}{cgc}{Austronesian}
\define@key{fams}{gel}{Atlantic-Congo}
\define@key{fams}{xkg}{Mande}
\define@key{fams}{hka}{Atlantic-Congo}
\define@key{fams}{agw}{Austronesian}
\define@key{fams}{kzb}{Austronesian}
\define@key{fams}{kzp}{Austronesian}
\define@key{fams}{kbw}{Austronesian}
\define@key{fams}{kep}{Dravidian}
\define@key{fams}{kzq}{Sino-Tibetan}
\define@key{fams}{kkq}{Atlantic-Congo}
\define@key{fams}{xai}{Unclassifiable}
\define@key{fams}{zka}{Austronesian}
\define@key{fams}{krd}{Austronesian}
\define@key{fams}{ckr}{Baining}
\define@key{fams}{kzm}{South Bird's Head Family}
\define@key{fams}{kce}{Unattested}
\define@key{fams}{tcq}{Lakes Plain}
\define@key{fams}{xkj}{Indo-European}
\define@key{fams}{kag}{Austronesian}
\define@key{fams}{ckq}{Afro-Asiatic}
\define@key{fams}{kjv}{Indo-European}
\define@key{fams}{xdq}{Nakh-Daghestanian}
\define@key{fams}{kka}{Atlantic-Congo}
\define@key{fams}{kke}{Mande}
\define@key{fams}{kqf}{Austronesian}
\define@key{fams}{kkj}{Atlantic-Congo}
\define@key{fams}{keo}{Nilotic}
\define@key{fams}{wkl}{Dravidian}
\define@key{fams}{kzz}{West Bird's Head}
\define@key{fams}{kkf}{Sino-Tibetan}
\define@key{fams}{kba}{Pama-Nyungan}
\define@key{fams}{gll}{Pama-Nyungan}
\define@key{fams}{ijn}{Ijoid}
\define@key{fams}{knz}{Atlantic-Congo}
\define@key{fams}{kqe}{Austronesian}
\define@key{fams}{kve}{Austronesian}
\define@key{fams}{kly}{Austronesian}
\define@key{fams}{lkm}{Pama-Nyungan}
\define@key{fams}{xka}{Indo-European}
\define@key{fams}{rmf}{Indo-European}
\define@key{fams}{ywa}{Sepik}
\define@key{fams}{kli}{Austronesian}
\define@key{fams}{keq}{Indo-European}
\define@key{fams}{jmr}{Atlantic-Congo}
\define@key{fams}{kci}{Atlantic-Congo}
\define@key{fams}{klp}{Angan}
\define@key{fams}{kzx}{Austronesian}
\define@key{fams}{kyk}{Austronesian}
\define@key{fams}{kgx}{Austronesian}
\define@key{fams}{vkm}{Kamakanan}
\define@key{fams}{xbw}{Unclassifiable}
\define@key{fams}{irx}{Nuclear Trans New Guinea}
\define@key{fams}{kyy}{Nuclear Trans New Guinea}
\define@key{fams}{ktb}{Afro-Asiatic}
\define@key{fams}{kmi}{Atlantic-Congo}
\define@key{fams}{kdx}{Atlantic-Congo}
\define@key{fams}{kcq}{Atlantic-Congo}
\define@key{fams}{xla}{Kamula-Elevala}
\define@key{fams}{hig}{Afro-Asiatic}
\define@key{fams}{bjj}{Indo-European}
\define@key{fams}{xnb}{Austronesian}
\define@key{fams}{soq}{Dagan}
\define@key{fams}{kbs}{Atlantic-Congo}
\define@key{fams}{kqw}{Austronesian}
\define@key{fams}{gam}{Nuclear Trans New Guinea}
\define@key{fams}{xnr}{Indo-European}
\define@key{fams}{kxs}{Mongolic-Khitan}
\define@key{fams}{kzy}{Atlantic-Congo}
\define@key{fams}{kty}{Atlantic-Congo}
\define@key{fams}{kcp}{Kadugli-Krongo}
\define@key{fams}{kkv}{Austronesian}
\define@key{fams}{igm}{Lower Sepik-Ramu}
\define@key{fams}{kev}{Dravidian}
\define@key{fams}{kdp}{Atlantic-Congo}
\define@key{fams}{kzo}{Atlantic-Congo}
\define@key{fams}{wat}{Austronesian}
\define@key{fams}{ktk}{Austronesian}
\define@key{fams}{knr}{Sepik}
\define@key{fams}{kmu}{Nuclear Trans New Guinea}
\define@key{fams}{kft}{Indo-European}
\define@key{fams}{kbe}{Pama-Nyungan}
\define@key{fams}{kxn}{Austronesian}
\define@key{fams}{ksk}{Siouan}
\define@key{fams}{xkt}{Atlantic-Congo}
\define@key{fams}{kni}{Atlantic-Congo}
\define@key{fams}{khx}{Atlantic-Congo}
\define@key{fams}{kqn}{Atlantic-Congo}
\define@key{fams}{kax}{North Halmahera}
\define@key{fams}{xpn}{Unclassifiable}
\define@key{fams}{tbx}{Austronesian}
\define@key{fams}{khp}{Isolate}
\define@key{fams}{ykm}{Austronesian}
\define@key{fams}{kbi}{Austronesian}
\define@key{fams}{klo}{Atlantic-Congo}
\define@key{fams}{xkh}{Unattested}
\define@key{fams}{kzr}{Atlantic-Congo}
\define@key{fams}{reg}{Atlantic-Congo}
\define@key{fams}{kth}{Maban}
\define@key{fams}{mry}{Austronesian}
\define@key{fams}{xrw}{Sepik}
\define@key{fams}{xar}{Isolate}
\define@key{fams}{kgv}{West Bomberai}
\define@key{fams}{kbn}{Atlantic-Congo}
\define@key{fams}{kyd}{Austronesian}
\define@key{fams}{kmf}{Nuclear Trans New Guinea}
\define@key{fams}{kai}{Afro-Asiatic}
\define@key{fams}{kmv}{Indo-European}
\define@key{fams}{kgn}{Indo-European}
\define@key{fams}{kbj}{Atlantic-Congo}
\define@key{fams}{kil}{Afro-Asiatic}
\define@key{fams}{kuq}{Tupian}
\define@key{fams}{kko}{Nubian}
\define@key{fams}{krb}{Miwok-Costanoan}
\define@key{fams}{bbv}{Austronesian}
\define@key{fams}{krx}{Atlantic-Congo}
\define@key{fams}{kxh}{South Omotic}
\define@key{fams}{xkx}{Austronesian}
\define@key{fams}{kyn}{Austronesian}
\define@key{fams}{rxw}{Pama-Nyungan}
\define@key{fams}{ccj}{Atlantic-Congo}
\define@key{fams}{ksn}{Austronesian}
\define@key{fams}{kkz}{Athabaskan-Eyak-Tlingit}
\define@key{fams}{khs}{Bosavi}
\define@key{fams}{ktq}{Unclassifiable}
\define@key{fams}{xat}{Katukinan}
\define@key{fams}{tmb}{Austronesian}
\define@key{fams}{tkt}{Indo-European}
\define@key{fams}{ykt}{Sino-Tibetan}
\define@key{fams}{kfu}{Indo-European}
\define@key{fams}{kaf}{Sino-Tibetan}
\define@key{fams}{kta}{Austroasiatic}
\define@key{fams}{vkk}{Austronesian}
\define@key{fams}{xau}{Greater Kwerba}
\define@key{fams}{ckv}{Austronesian}
\define@key{fams}{kcb}{Angan}
\define@key{fams}{kgb}{Austronesian}
\define@key{fams}{kaw}{Austronesian}
\define@key{fams}{ktx}{Pano-Tacanan}
\define@key{fams}{kbb}{Cariban}
\define@key{fams}{pdu}{Sino-Tibetan}
\define@key{fams}{xay}{Austronesian}
\define@key{fams}{xkn}{Austronesian}
\define@key{fams}{kyt}{Kayagaric}
\define@key{fams}{kzl}{Austronesian}
\define@key{fams}{kxy}{Austroasiatic}
\define@key{fams}{kzu}{Austronesian}
\define@key{fams}{kzk}{Austronesian}
\define@key{fams}{keh}{Ndu}
\define@key{fams}{khz}{Austronesian}
\define@key{fams}{meo}{Austronesian}
\define@key{fams}{kdy}{Tor-Orya}
\define@key{fams}{khh}{Isolate}
\define@key{fams}{kec}{Kadugli-Krongo}
\define@key{fams}{bmh}{Nuclear Trans New Guinea}
\define@key{fams}{eyo}{Nilotic}
\define@key{fams}{khy}{Atlantic-Congo}
\define@key{fams}{keb}{Atlantic-Congo}
\define@key{fams}{ify}{Austronesian}
\define@key{fams}{kbo}{Central Sudanic}
\define@key{fams}{xel}{Eastern Jebel}
\define@key{fams}{kyo}{Timor-Alor-Pantar}
\define@key{fams}{kem}{Austronesian}
\define@key{fams}{bzp}{South Bird's Head Family}
\define@key{fams}{xem}{Austronesian}
\define@key{fams}{xkw}{Lepki-Murkim-Kembra}
\define@key{fams}{dmo}{Atlantic-Congo}
\define@key{fams}{sjk}{Uralic}
\define@key{fams}{xbn}{Isolate}
\define@key{fams}{gat}{Nuclear Trans New Guinea}
\define@key{fams}{kvm}{Atlantic-Congo}
\define@key{fams}{klf}{Maban}
\define@key{fams}{knx}{Austronesian}
\define@key{fams}{knl}{Austronesian}
\define@key{fams}{kxi}{Austronesian}
\define@key{fams}{kns}{Austroasiatic}
\define@key{fams}{ndb}{Atlantic-Congo}
\define@key{fams}{kzh}{Nubian}
\define@key{fams}{lke}{Atlantic-Congo}
\define@key{fams}{xeu}{Eleman}
\define@key{fams}{kpn}{Tupian}
\define@key{fams}{kuk}{Austronesian}
\define@key{fams}{hhr}{Atlantic-Congo}
\define@key{fams}{ked}{Atlantic-Congo}
\define@key{fams}{xke}{Austronesian}
\define@key{fams}{kxz}{Kiwaian}
\define@key{fams}{kvr}{Austronesian}
\define@key{fams}{xes}{Nuclear Trans New Guinea}
\define@key{fams}{kae}{Austronesian}
\define@key{fams}{ktt}{Nuclear Trans New Guinea}
\define@key{fams}{kyg}{Nuclear Trans New Guinea}
\define@key{fams}{xkv}{Atlantic-Congo}
\define@key{fams}{hkh}{Indo-European}
\define@key{fams}{kbg}{Sino-Tibetan}
\define@key{fams}{kht}{Tai-Kadai}
\define@key{fams}{ksu}{Tai-Kadai}
\define@key{fams}{khn}{Indo-European}
\define@key{fams}{kjm}{Austroasiatic}
\define@key{fams}{ksy}{Indo-European}
\define@key{fams}{kfw}{Sino-Tibetan}
\define@key{fams}{lko}{Atlantic-Congo}
\define@key{fams}{kqg}{Atlantic-Congo}
\define@key{fams}{tlx}{Austronesian}
\define@key{fams}{xkf}{Sino-Tibetan}
\define@key{fams}{xhe}{Indo-European}
\define@key{fams}{nkh}{Sino-Tibetan}
\define@key{fams}{kix}{Sino-Tibetan}
\define@key{fams}{kwx}{Dravidian}
\define@key{fams}{kqm}{Atlantic-Congo}
\define@key{fams}{ykl}{Sino-Tibetan}
\define@key{fams}{xkc}{Indo-European}
\define@key{fams}{nkb}{Sino-Tibetan}
\define@key{fams}{ktc}{Afro-Asiatic}
\define@key{fams}{kho}{Indo-European}
\define@key{fams}{khf}{Austroasiatic}
\define@key{fams}{kfm}{Indo-European}
\define@key{fams}{xco}{Indo-European}
\define@key{fams}{kie}{Maban}
\define@key{fams}{prm}{Isolate}
\define@key{fams}{kzg}{Japonic}
\define@key{fams}{kih}{Border}
\define@key{fams}{kqr}{Austronesian}
\define@key{fams}{kmb}{Atlantic-Congo}
\define@key{fams}{kiv}{Atlantic-Congo}
\define@key{fams}{sbt}{Isolate}
\define@key{fams}{kqp}{Afro-Asiatic}
\define@key{fams}{krj}{Austronesian}
\define@key{fams}{kco}{Nuclear Trans New Guinea}
\define@key{fams}{cbw}{Austronesian}
\define@key{fams}{knq}{Austroasiatic}
\define@key{fams}{kkd}{Atlantic-Congo}
\define@key{fams}{ues}{Austronesian}
\define@key{fams}{kkm}{Atlantic-Congo}
\define@key{fams}{apk}{Athabaskan-Eyak-Tlingit}
\define@key{fams}{sgc}{Nilotic}
\define@key{fams}{kyi}{Austronesian}
\define@key{fams}{kkr}{Afro-Asiatic}
\define@key{fams}{okr}{Ijoid}
\define@key{fams}{kiu}{Indo-European}
\define@key{fams}{fkk}{Afro-Asiatic}
\define@key{fams}{lks}{Atlantic-Congo}
\define@key{fams}{kiz}{Atlantic-Congo}
\define@key{fams}{kis}{Austronesian}
\define@key{fams}{zkt}{Mongolic-Khitan}
\define@key{fams}{mwk}{Mande}
\define@key{fams}{mkw}{Atlantic-Congo}
\define@key{fams}{kqt}{Austronesian}
\define@key{fams}{tlh}{Artificial Language}
\define@key{fams}{kib}{Heibanic}
\define@key{fams}{kpd}{Austronesian}
\define@key{fams}{kcj}{Atlantic-Congo}
\define@key{fams}{kgu}{Nuclear Trans New Guinea}
\define@key{fams}{thq}{Indo-European}
\define@key{fams}{kdq}{Sino-Tibetan}
\define@key{fams}{dhw}{Indo-European}
\define@key{fams}{cdz}{Austroasiatic}
\define@key{fams}{ksz}{Austroasiatic}
\define@key{fams}{vko}{Austronesian}
\define@key{fams}{kwp}{Kru}
\define@key{fams}{kod}{Austronesian}
\define@key{fams}{kcs}{Afro-Asiatic}
\define@key{fams}{kpi}{Geelvink Bay}
\define@key{fams}{kwl}{Afro-Asiatic}
\define@key{fams}{zkg}{Unclassifiable}
\define@key{fams}{plk}{Indo-European}
\define@key{fams}{kkx}{Austronesian}
\define@key{fams}{kkt}{Sino-Tibetan}
\define@key{fams}{nkd}{Sino-Tibetan}
\define@key{fams}{kxt}{Ndu}
\define@key{fams}{kou}{Atlantic-Congo}
\define@key{fams}{gko}{Pama-Nyungan}
\define@key{fams}{xod}{South Bird's Head Family}
\define@key{fams}{kzn}{Atlantic-Congo}
\define@key{fams}{klc}{Atlantic-Congo}
\define@key{fams}{ekl}{Austroasiatic}
\define@key{fams}{biw}{Atlantic-Congo}
\define@key{fams}{skn}{Austronesian}
\define@key{fams}{klm}{Nuclear Trans New Guinea}
\define@key{fams}{kol}{Isolate}
\define@key{fams}{klx}{Austronesian}
\define@key{fams}{kmy}{Atlantic-Congo}
\define@key{fams}{kpf}{Nuclear Trans New Guinea}
\define@key{fams}{tyn}{Nuclear Trans New Guinea}
\define@key{fams}{kmm}{Sino-Tibetan}
\define@key{fams}{xoi}{Lower Sepik-Ramu}
\define@key{fams}{kmw}{Atlantic-Congo}
\define@key{fams}{kvh}{Austronesian}
\define@key{fams}{kvp}{Austronesian}
\define@key{fams}{kzv}{Nuclear Trans New Guinea}
\define@key{fams}{kxw}{East Strickland}
\define@key{fams}{knd}{Konda-Yahadian}
\define@key{fams}{kdw}{Mombum-Koneraw}
\define@key{fams}{klk}{Atlantic-Congo}
\define@key{fams}{kcz}{Atlantic-Congo}
\define@key{fams}{knu}{Mande}
\define@key{fams}{kno}{Mande}
\define@key{fams}{koa}{Austronesian}
\define@key{fams}{kxc}{Afro-Asiatic}
\define@key{fams}{nbe}{Sino-Tibetan}
\define@key{fams}{mku}{Mande}
\define@key{fams}{koo}{Atlantic-Congo}
\define@key{fams}{ozm}{Atlantic-Congo}
\define@key{fams}{fuj}{Heibanic}
\define@key{fams}{xop}{Lower Sepik-Ramu}
\define@key{fams}{opk}{Nuclear Trans New Guinea}
\define@key{fams}{kcy}{Songhay}
\define@key{fams}{koz}{Nuclear Trans New Guinea}
\define@key{fams}{okh}{Indo-European}
\define@key{fams}{vkp}{Indo-European}
\define@key{fams}{ktl}{Indo-European}
\define@key{fams}{krp}{Atlantic-Congo}
\define@key{fams}{kfo}{Mande}
\define@key{fams}{krf}{Austronesian}
\define@key{fams}{xkq}{Austronesian}
\define@key{fams}{kqj}{South Bougainville}
\define@key{fams}{jkr}{Sino-Tibetan}
\define@key{fams}{vkn}{Atlantic-Congo}
\define@key{fams}{vkz}{Atlantic-Congo}
\define@key{fams}{kfd}{Dravidian}
\define@key{fams}{kpq}{Nuclear Trans New Guinea}
\define@key{fams}{xor}{Pano-Tacanan}
\define@key{fams}{kfp}{Austroasiatic}
\define@key{fams}{kiq}{Kaure-Kosare}
\define@key{fams}{kid}{Atlantic-Congo}
\define@key{fams}{kqk}{Atlantic-Congo}
\define@key{fams}{koq}{Atlantic-Congo}
\define@key{fams}{mqg}{Austronesian}
\define@key{fams}{grm}{Austronesian}
\define@key{fams}{avk}{Artificial Language}
\define@key{fams}{zko}{Yeniseian}
\define@key{fams}{kyf}{Kru}
\define@key{fams}{kqb}{Nuclear Trans New Guinea}
\define@key{fams}{kvc}{Austronesian}
\define@key{fams}{xow}{Nuclear Trans New Guinea}
\define@key{fams}{kwh}{Austronesian}
\define@key{fams}{kga}{Mande}
\define@key{fams}{koh}{Atlantic-Congo}
\define@key{fams}{kqd}{Afro-Asiatic}
\define@key{fams}{kuw}{Atlantic-Congo}
\define@key{fams}{kpl}{Atlantic-Congo}
\define@key{fams}{pbn}{Atlantic-Congo}
\define@key{fams}{koc}{Atlantic-Congo}
\define@key{fams}{cpo}{Mande}
\define@key{fams}{kef}{Atlantic-Congo}
\define@key{fams}{kph}{Atlantic-Congo}
\define@key{fams}{kye}{Atlantic-Congo}
\define@key{fams}{rka}{Austroasiatic}
\define@key{fams}{xre}{Nuclear-Macro-Je}
\define@key{fams}{kri}{Indo-European}
\define@key{fams}{kxb}{Atlantic-Congo}
\define@key{fams}{tyu}{Khoe-Kwadi}
\define@key{fams}{yku}{Sino-Tibetan}
\define@key{fams}{uan}{Tai-Kadai}
\define@key{fams}{kua}{Atlantic-Congo}
\define@key{fams}{ykn}{Sino-Tibetan}
\define@key{fams}{ugh}{Nakh-Daghestanian}
\define@key{fams}{kgf}{Nuclear Trans New Guinea}
\define@key{fams}{kof}{Afro-Asiatic}
\define@key{fams}{jko}{East Strickland}
\define@key{fams}{kvb}{Austronesian}
\define@key{fams}{lkc}{Sino-Tibetan}
\define@key{fams}{kfg}{Dravidian}
\define@key{fams}{kyw}{Indo-European}
\define@key{fams}{kov}{Atlantic-Congo}
\define@key{fams}{kow}{Atlantic-Congo}
\define@key{fams}{kes}{Atlantic-Congo}
\define@key{fams}{dkr}{Austronesian}
\define@key{fams}{vkj}{Isolate}
\define@key{fams}{kux}{Pama-Nyungan}
\define@key{fams}{kez}{Atlantic-Congo}
\define@key{fams}{kfn}{Atlantic-Congo}
\define@key{fams}{ugb}{Pama-Nyungan}
\define@key{fams}{xmp}{Pama-Nyungan}
\define@key{fams}{xmh}{Pama-Nyungan}
\define@key{fams}{ukv}{Nilotic}
\define@key{fams}{kul}{Afro-Asiatic}
\define@key{fams}{kxj}{Central Sudanic}
\define@key{fams}{vkl}{Austronesian}
\define@key{fams}{xpk}{Pano-Tacanan}
\define@key{fams}{kfx}{Indo-European}
\define@key{fams}{pzh}{Austronesian}
\define@key{fams}{uon}{Austronesian}
\define@key{fams}{bbu}{Atlantic-Congo}
\define@key{fams}{kdi}{Nilotic}
\define@key{fams}{ksl}{Austronesian}
\define@key{fams}{ksm}{Atlantic-Congo}
\define@key{fams}{xks}{Austronesian}
\define@key{fams}{kra}{Indo-European}
\define@key{fams}{kuo}{Nuclear Trans New Guinea}
\define@key{fams}{zum}{Indo-European}
\define@key{fams}{wku}{Dravidian}
\define@key{fams}{kdn}{Atlantic-Congo}
\define@key{fams}{shd}{Indo-European}
\define@key{fams}{kgl}{Pama-Nyungan}
\define@key{fams}{ggk}{Isolate}
\define@key{fams}{kfl}{Atlantic-Congo}
\define@key{fams}{kse}{Austronesian}
\define@key{fams}{xug}{Japonic}
\define@key{fams}{pep}{Yam}
\define@key{fams}{njx}{Atlantic-Congo}
\define@key{fams}{kug}{Atlantic-Congo}
\define@key{fams}{mkn}{Austronesian}
\define@key{fams}{key}{Indo-European}
\define@key{fams}{nqk}{Atlantic-Congo}
\define@key{fams}{krh}{Atlantic-Congo}
\define@key{fams}{kfh}{Dravidian}
\define@key{fams}{kuj}{Atlantic-Congo}
\define@key{fams}{nbn}{Austronesian}
\define@key{fams}{kfv}{Indo-European}
\define@key{fams}{vku}{Pama-Nyungan}
\define@key{fams}{kuv}{Austronesian}
\define@key{fams}{xkz}{Sino-Tibetan}
\define@key{fams}{ktm}{Austronesian}
\define@key{fams}{kjr}{Austronesian}
\define@key{fams}{kyr}{Tupian}
\define@key{fams}{kus}{Atlantic-Congo}
\define@key{fams}{ksg}{Austronesian}
\define@key{fams}{kuh}{Afro-Asiatic}
\define@key{fams}{ksv}{Atlantic-Congo}
\define@key{fams}{ght}{Sino-Tibetan}
\define@key{fams}{kub}{Atlantic-Congo}
\define@key{fams}{xut}{Pama-Nyungan}
\define@key{fams}{kpa}{Afro-Asiatic}
\define@key{fams}{khj}{Atlantic-Congo}
\define@key{fams}{kdc}{Atlantic-Congo}
\define@key{fams}{uky}{Pama-Nyungan}
\define@key{fams}{lku}{Pama-Nyungan}
\define@key{fams}{olu}{Atlantic-Congo}
\define@key{fams}{cwt}{Atlantic-Congo}
\define@key{fams}{blh}{Kru}
\define@key{fams}{kdt}{Austroasiatic}
\define@key{fams}{fkv}{Uralic}
\define@key{fams}{kwb}{Atlantic-Congo}
\define@key{fams}{bko}{Atlantic-Congo}
\define@key{fams}{kwz}{Khoe-Kwadi}
\define@key{fams}{wka}{Afro-Asiatic}
\define@key{fams}{kdz}{Atlantic-Congo}
\define@key{fams}{kwu}{Atlantic-Congo}
\define@key{fams}{qwt}{Athabaskan-Eyak-Tlingit}
\define@key{fams}{kmq}{Koman}
\define@key{fams}{ktf}{Atlantic-Congo}
\define@key{fams}{kwm}{Atlantic-Congo}
\define@key{fams}{okk}{Nuclear Torricelli}
\define@key{fams}{knp}{Atlantic-Congo}
\define@key{fams}{kwj}{Sepik}
\define@key{fams}{kvi}{Afro-Asiatic}
\define@key{fams}{xdo}{Atlantic-Congo}
\define@key{fams}{kwf}{Austronesian}
\define@key{fams}{kop}{Nuclear Trans New Guinea}
\define@key{fams}{kya}{Atlantic-Congo}
\define@key{fams}{cwe}{Atlantic-Congo}
\define@key{fams}{xwr}{Greater Kwerba}
\define@key{fams}{kkb}{Lakes Plain}
\define@key{fams}{kwr}{Nuclear Trans New Guinea}
\define@key{fams}{kws}{Atlantic-Congo}
\define@key{fams}{kwt}{Tor-Orya}
\define@key{fams}{kuc}{Tor-Orya}
\define@key{fams}{kww}{Indo-European}
\define@key{fams}{bka}{Atlantic-Congo}
\define@key{fams}{tye}{Mande}
\define@key{fams}{kql}{Yuat}
\define@key{fams}{ldn}{Artificial Language}
\define@key{fams}{bwj}{Atlantic-Congo}
\define@key{fams}{ldi}{Atlantic-Congo}
\define@key{fams}{lbb}{Austronesian}
\define@key{fams}{lbi}{Speech Register}
\define@key{fams}{jku}{Atlantic-Congo}
\define@key{fams}{ypb}{Sino-Tibetan}
\define@key{fams}{mwi}{Austronesian}
\define@key{fams}{dtb}{Austronesian}
\define@key{fams}{zpl}{Otomanguean}
\define@key{fams}{zpa}{Otomanguean}
\define@key{fams}{lkl}{Nuclear Torricelli}
\define@key{fams}{lgh}{Sino-Tibetan}
\define@key{fams}{lgb}{Austronesian}
\define@key{fams}{lhh}{Austronesian}
\define@key{fams}{lhn}{Austronesian}
\define@key{fams}{lhl}{Indo-European}
\define@key{fams}{lhi}{Sino-Tibetan}
\define@key{fams}{lmx}{Atlantic-Congo}
\define@key{fams}{lji}{Austronesian}
\define@key{fams}{lap}{Central Sudanic}
\define@key{fams}{lka}{Austronesian}
\define@key{fams}{lkh}{Sino-Tibetan}
\define@key{fams}{lki}{Indo-European}
\define@key{fams}{lkn}{Austronesian}
\define@key{fams}{lkd}{Nambiquaran}
\define@key{fams}{lxm}{Austronesian}
\define@key{fams}{lla}{Atlantic-Congo}
\define@key{fams}{leb}{Atlantic-Congo}
\define@key{fams}{cnl}{Otomanguean}
\define@key{fams}{las}{Atlantic-Congo}
\define@key{fams}{lmr}{Austronesian}
\define@key{fams}{lmq}{Austronesian}
\define@key{fams}{lai}{Atlantic-Congo}
\define@key{fams}{lmy}{Austronesian}
\define@key{fams}{quf}{Quechuan}
\define@key{fams}{lbn}{Austroasiatic}
\define@key{fams}{bma}{Atlantic-Congo}
\define@key{fams}{ldh}{Atlantic-Congo}
\define@key{fams}{lmk}{Sino-Tibetan}
\define@key{fams}{lev}{Timor-Alor-Pantar}
\define@key{fams}{lmg}{Austronesian}
\define@key{fams}{abl}{Austronesian}
\define@key{fams}{llh}{Sino-Tibetan}
\define@key{fams}{ruu}{Austronesian}
\define@key{fams}{ldm}{Atlantic-Congo}
\define@key{fams}{sfb}{Sign Language}
\define@key{fams}{yln}{Tai-Kadai}
\define@key{fams}{lna}{Atlantic-Congo}
\define@key{fams}{lno}{Nilotic}
\define@key{fams}{lnm}{Keram}
\define@key{fams}{lnh}{Austroasiatic}
\define@key{fams}{lwm}{Sino-Tibetan}
\define@key{fams}{ztl}{Otomanguean}
\define@key{fams}{laa}{Austronesian}
\define@key{fams}{lrt}{Austronesian}
\define@key{fams}{lrv}{Austronesian}
\define@key{fams}{hmd}{Hmong-Mien}
\define@key{fams}{lrl}{Indo-European}
\define@key{fams}{lro}{Heibanic}
\define@key{fams}{lar}{Atlantic-Congo}
\define@key{fams}{lan}{Atlantic-Congo}
\define@key{fams}{llm}{Austronesian}
\define@key{fams}{lsa}{Indo-European}
\define@key{fams}{lsi}{Sino-Tibetan}
\define@key{fams}{lss}{Indo-European}
\define@key{fams}{lat}{Indo-European}
\define@key{fams}{ltu}{Austronesian}
\define@key{fams}{ltn}{Nambiquaran}
\define@key{fams}{lsl}{Sign Language}
\define@key{fams}{llx}{Austronesian}
\define@key{fams}{luf}{Mailuan}
\define@key{fams}{lre}{Iroquoian}
\define@key{fams}{clt}{Sino-Tibetan}
\define@key{fams}{lbv}{Austronesian}
\define@key{fams}{lbx}{Austronesian}
\define@key{fams}{lvi}{Austroasiatic}
\define@key{fams}{tgi}{Austronesian}
\define@key{fams}{lwu}{Sino-Tibetan}
\define@key{fams}{lya}{Sino-Tibetan}
\define@key{fams}{ldk}{Atlantic-Congo}
\define@key{fams}{lfa}{Atlantic-Congo}
\define@key{fams}{lgm}{Atlantic-Congo}
\define@key{fams}{lcc}{Austronesian}
\define@key{fams}{cae}{Atlantic-Congo}
\define@key{fams}{tql}{Austronesian}
\define@key{fams}{urr}{Austronesian}
\define@key{fams}{lzn}{Sino-Tibetan}
\define@key{fams}{lek}{Austronesian}
\define@key{fams}{llk}{Austronesian}
\define@key{fams}{lel}{Atlantic-Congo}
\define@key{fams}{llc}{Mande}
\define@key{fams}{lpa}{Austronesian}
\define@key{fams}{lle}{Austronesian}
\define@key{fams}{leq}{Nuclear Trans New Guinea}
\define@key{fams}{lrz}{Austronesian}
\define@key{fams}{lei}{Nuclear Trans New Guinea}
\define@key{fams}{xle}{Unclassifiable}
\define@key{fams}{ldj}{Atlantic-Congo}
\define@key{fams}{ley}{Austronesian}
\define@key{fams}{lej}{Atlantic-Congo}
\define@key{fams}{lgr}{Austronesian}
\define@key{fams}{lgi}{Austronesian}
\define@key{fams}{leh}{Atlantic-Congo}
\define@key{fams}{ler}{Austronesian}
\define@key{fams}{ldg}{Atlantic-Congo}
\define@key{fams}{lpe}{Lepki-Murkim-Kembra}
\define@key{fams}{xlp}{Indo-European}
\define@key{fams}{gnh}{Atlantic-Congo}
\define@key{fams}{let}{Austronesian}
\define@key{fams}{nms}{Austronesian}
\define@key{fams}{leo}{Atlantic-Congo}
\define@key{fams}{lvu}{Austronesian}
\define@key{fams}{lwe}{Austronesian}
\define@key{fams}{lwt}{Austronesian}
\define@key{fams}{ayi}{Atlantic-Congo}
\define@key{fams}{lhp}{Sino-Tibetan}
\define@key{fams}{lix}{Austronesian}
\define@key{fams}{njn}{Sino-Tibetan}
\define@key{fams}{zln}{Tai-Kadai}
\define@key{fams}{ste}{Austronesian}
\define@key{fams}{lir}{Pidgin}
\define@key{fams}{liz}{Atlantic-Congo}
\define@key{fams}{liq}{Afro-Asiatic}
\define@key{fams}{lbs}{Sign Language}
\define@key{fams}{lig}{Mande}
\define@key{fams}{lgz}{Atlantic-Congo}
\define@key{fams}{lih}{Austronesian}
\define@key{fams}{mgi}{Atlantic-Congo}
\define@key{fams}{lik}{Atlantic-Congo}
\define@key{fams}{lie}{Atlantic-Congo}
\define@key{fams}{lio}{Austronesian}
\define@key{fams}{kxx}{Atlantic-Congo}
\define@key{fams}{lib}{Austronesian}
\define@key{fams}{kwc}{Atlantic-Congo}
\define@key{fams}{lll}{Bogia}
\define@key{fams}{bme}{Atlantic-Congo}
\define@key{fams}{lim}{Indo-European}
\define@key{fams}{lmp}{Atlantic-Congo}
\define@key{fams}{ylm}{Sino-Tibetan}
\define@key{fams}{kmk}{Austronesian}
\define@key{fams}{qlm}{Indo-European}
\define@key{fams}{klw}{Austronesian}
\define@key{fams}{pml}{Pidgin}
\define@key{fams}{onb}{Tai-Kadai}
\define@key{fams}{lgk}{Austronesian}
\define@key{fams}{lfn}{Artificial Language}
\define@key{fams}{ljl}{Austronesian}
\define@key{fams}{apl}{Athabaskan-Eyak-Tlingit}
\define@key{fams}{lpo}{Sino-Tibetan}
\define@key{fams}{lcs}{Austronesian}
\define@key{fams}{lcl}{Austronesian}
\define@key{fams}{lsh}{Sino-Tibetan}
\define@key{fams}{lsd}{Afro-Asiatic}
\define@key{fams}{lzh}{Sino-Tibetan}
\define@key{fams}{lls}{Sign Language}
\define@key{fams}{lzl}{Austronesian}
\define@key{fams}{zlj}{Tai-Kadai}
\define@key{fams}{zlq}{Tai-Kadai}
\define@key{fams}{olo}{Uralic}
\define@key{fams}{loq}{Atlantic-Congo}
\define@key{fams}{lbm}{Indo-European}
\define@key{fams}{lgq}{Atlantic-Congo}
\define@key{fams}{rag}{Atlantic-Congo}
\define@key{fams}{liu}{Dajuic}
\define@key{fams}{lof}{Heibanic}
\define@key{fams}{src}{Indo-European}
\define@key{fams}{qvj}{Quechuan}
\define@key{fams}{jbo}{Artificial Language}
\define@key{fams}{yaz}{Atlantic-Congo}
\define@key{fams}{lky}{Nilotic}
\define@key{fams}{lcd}{Austronesian}
\define@key{fams}{llq}{Austronesian}
\define@key{fams}{llg}{Austronesian}
\define@key{fams}{ycl}{Sino-Tibetan}
\define@key{fams}{llb}{Atlantic-Congo}
\define@key{fams}{loa}{North Halmahera}
\define@key{fams}{rmi}{Speech Register}
\define@key{fams}{loi}{Atlantic-Congo}
\define@key{fams}{lmv}{Austronesian}
\define@key{fams}{lmi}{Central Sudanic}
\define@key{fams}{lmo}{Indo-European}
\define@key{fams}{loo}{Atlantic-Congo}
\define@key{fams}{ngl}{Atlantic-Congo}
\define@key{fams}{lce}{Austronesian}
\define@key{fams}{lpn}{Sino-Tibetan}
\define@key{fams}{wok}{Atlantic-Congo}
\define@key{fams}{lnu}{Atlantic-Congo}
\define@key{fams}{ttw}{Austronesian}
\define@key{fams}{ldo}{Atlantic-Congo}
\define@key{fams}{lop}{Atlantic-Congo}
\define@key{fams}{lpx}{Nilotic}
\define@key{fams}{lrn}{Austronesian}
\define@key{fams}{spq}{Indo-European}
\define@key{fams}{lnn}{Austronesian}
\define@key{fams}{uvl}{Austronesian}
\define@key{fams}{lht}{Austronesian}
\define@key{fams}{dtr}{Austronesian}
\define@key{fams}{lou}{Indo-European}
\define@key{fams}{lox}{Austronesian}
\define@key{fams}{xlo}{Algic}
\define@key{fams}{sli}{Indo-European}
\define@key{fams}{tto}{Austroasiatic}
\define@key{fams}{nsb}{Tuu}
\define@key{fams}{kml}{Austronesian}
\define@key{fams}{cea}{Salishan}
\define@key{fams}{axl}{Pama-Nyungan}
\define@key{fams}{ztp}{Otomanguean}
\define@key{fams}{kcc}{Atlantic-Congo}
\define@key{fams}{lcf}{Austronesian}
\define@key{fams}{knb}{Austronesian}
\define@key{fams}{luq}{Atlantic-Congo}
\define@key{fams}{lud}{Uralic}
\define@key{fams}{ldq}{Atlantic-Congo}
\define@key{fams}{ruf}{Atlantic-Congo}
\define@key{fams}{lcq}{Austronesian}
\define@key{fams}{lum}{Atlantic-Congo}
\define@key{fams}{dop}{Atlantic-Congo}
\define@key{fams}{smj}{Uralic}
\define@key{fams}{lmz}{Unattested}
\define@key{fams}{lup}{Atlantic-Congo}
\define@key{fams}{lmd}{Narrow Talodi}
\define@key{fams}{luk}{Sino-Tibetan}
\define@key{fams}{luj}{Atlantic-Congo}
\define@key{fams}{lga}{Austronesian}
\define@key{fams}{luw}{Atlantic-Congo}
\define@key{fams}{hml}{Hmong-Mien}
\define@key{fams}{ldd}{Afro-Asiatic}
\define@key{fams}{lse}{Atlantic-Congo}
\define@key{fams}{xls}{Indo-European}
\define@key{fams}{ndy}{Central Sudanic}
\define@key{fams}{luv}{Indo-European}
\define@key{fams}{lyn}{Atlantic-Congo}
\define@key{fams}{lwa}{Atlantic-Congo}
\define@key{fams}{xlc}{Indo-European}
\define@key{fams}{xld}{Indo-European}
\define@key{fams}{lyg}{Austroasiatic}
\define@key{fams}{cma}{Austroasiatic}
\define@key{fams}{mew}{Afro-Asiatic}
\define@key{fams}{ymm}{Afro-Asiatic}
\define@key{fams}{mmz}{Atlantic-Congo}
\define@key{fams}{mfz}{Nilotic}
\define@key{fams}{mqa}{Austronesian}
\define@key{fams}{kkg}{Austronesian}
\define@key{fams}{muj}{Afro-Asiatic}
\define@key{fams}{mcl}{Tucanoan}
\define@key{fams}{mzs}{Indo-European}
\define@key{fams}{mvw}{Atlantic-Congo}
\define@key{fams}{jmc}{Atlantic-Congo}
\define@key{fams}{mpd}{Arawakan}
\define@key{fams}{wpc}{Saliban}
\define@key{fams}{mzc}{Sign Language}
\define@key{fams}{mmx}{Austronesian}
\define@key{fams}{xmx}{Austronesian}
\define@key{fams}{grg}{Nuclear Trans New Guinea}
\define@key{fams}{kmd}{Austronesian}
\define@key{fams}{mme}{Austronesian}
\define@key{fams}{itt}{Austronesian}
\define@key{fams}{maf}{Afro-Asiatic}
\define@key{fams}{mkv}{Austronesian}
\define@key{fams}{sgb}{Austronesian}
\define@key{fams}{mtw}{Austronesian}
\define@key{fams}{xtm}{Otomanguean}
\define@key{fams}{gmd}{Atlantic-Congo}
\define@key{fams}{blx}{Austronesian}
\define@key{fams}{gkd}{Nuclear Trans New Guinea}
\define@key{fams}{gmg}{Nuclear Trans New Guinea}
\define@key{fams}{gmx}{Atlantic-Congo}
\define@key{fams}{zgr}{Austronesian}
\define@key{fams}{bfz}{Indo-European}
\define@key{fams}{mjx}{Austroasiatic}
\define@key{fams}{pmh}{Indo-European}
\define@key{fams}{mjy}{Algic}
\define@key{fams}{mhb}{Atlantic-Congo}
\define@key{fams}{mzz}{Austronesian}
\define@key{fams}{tnh}{Nuclear Trans New Guinea}
\define@key{fams}{sks}{Nuclear Trans New Guinea}
\define@key{fams}{mmm}{Austronesian}
\define@key{fams}{vmf}{Indo-European}
\define@key{fams}{cwb}{Atlantic-Congo}
\define@key{fams}{xkl}{Austronesian}
\define@key{fams}{mum}{Austronesian}
\define@key{fams}{wmm}{Austronesian}
\define@key{fams}{mti}{Dagan}
\define@key{fams}{xmj}{Afro-Asiatic}
\define@key{fams}{mmj}{Austroasiatic}
\define@key{fams}{mjz}{Indo-European}
\define@key{fams}{mfp}{Austronesian}
\define@key{fams}{aup}{Anim}
\define@key{fams}{mkg}{Tai-Kadai}
\define@key{fams}{vmk}{Atlantic-Congo}
\define@key{fams}{xmc}{Atlantic-Congo}
\define@key{fams}{vmw}{Atlantic-Congo}
\define@key{fams}{mhm}{Atlantic-Congo}
\define@key{fams}{xsq}{Atlantic-Congo}
\define@key{fams}{pbl}{Atlantic-Congo}
\define@key{fams}{zmh}{Baining}
\define@key{fams}{jmn}{Sino-Tibetan}
\define@key{fams}{lva}{Austronesian}
\define@key{fams}{mpu}{Tupian}
\define@key{fams}{ymk}{Atlantic-Congo}
\define@key{fams}{umn}{Sino-Tibetan}
\define@key{fams}{lon}{Atlantic-Congo}
\define@key{fams}{xml}{Sign Language}
\define@key{fams}{ima}{Dravidian}
\define@key{fams}{ymr}{Dravidian}
\define@key{fams}{mjo}{Dravidian}
\define@key{fams}{mjr}{Dravidian}
\define@key{fams}{mjq}{Dravidian}
\define@key{fams}{mjp}{Dravidian}
\define@key{fams}{ruy}{Unattested}
\define@key{fams}{swk}{Atlantic-Congo}
\define@key{fams}{ccm}{Austronesian}
\define@key{fams}{mln}{Austronesian}
\define@key{fams}{mqz}{Austronesian}
\define@key{fams}{mmt}{Austronesian}
\define@key{fams}{ped}{Nuclear Trans New Guinea}
\define@key{fams}{mkr}{Nuclear Trans New Guinea}
\define@key{fams}{lws}{Artificial Language}
\define@key{fams}{bfo}{Atlantic-Congo}
\define@key{fams}{pkt}{Austroasiatic}
\define@key{fams}{mdc}{Nuclear Trans New Guinea}
\define@key{fams}{gut}{Chibchan}
\define@key{fams}{mlx}{Austronesian}
\define@key{fams}{vml}{Pama-Nyungan}
\define@key{fams}{mxf}{Afro-Asiatic}
\define@key{fams}{mgq}{Atlantic-Congo}
\define@key{fams}{mzd}{Atlantic-Congo}
\define@key{fams}{mli}{Austronesian}
\define@key{fams}{mlf}{Austroasiatic}
\define@key{fams}{mbk}{Austronesian}
\define@key{fams}{mkb}{Indo-European}
\define@key{fams}{mdl}{Sign Language}
\define@key{fams}{mll}{Austronesian}
\define@key{fams}{mup}{Indo-European}
\define@key{fams}{myk}{Atlantic-Congo}
\define@key{fams}{mma}{Atlantic-Congo}
\define@key{fams}{mhf}{Nuclear Trans New Guinea}
\define@key{fams}{wmd}{Nambiquaran}
\define@key{fams}{mvd}{Austronesian}
\define@key{fams}{mgm}{Austronesian}
\define@key{fams}{kdf}{Austronesian}
\define@key{fams}{mqx}{Austronesian}
\define@key{fams}{znk}{Unattested}
\define@key{fams}{mjl}{Indo-European}
\define@key{fams}{mha}{Dravidian}
\define@key{fams}{zma}{Western Daly}
\define@key{fams}{zmk}{Pama-Nyungan}
\define@key{fams}{mgs}{Atlantic-Congo}
\define@key{fams}{mqu}{Nilotic}
\define@key{fams}{tbf}{Austronesian}
\define@key{fams}{mqr}{Tor-Orya}
\define@key{fams}{aax}{Nuclear Trans New Guinea}
\define@key{fams}{bwp}{Nuclear Trans New Guinea}
\define@key{fams}{mht}{Arawakan}
\define@key{fams}{zng}{Austroasiatic}
\define@key{fams}{zme}{Giimbiyu}
\define@key{fams}{mem}{Pama-Nyungan}
\define@key{fams}{myj}{Atlantic-Congo}
\define@key{fams}{mdk}{Central Sudanic}
\define@key{fams}{kby}{Saharan}
\define@key{fams}{mrv}{Austronesian}
\define@key{fams}{mbh}{Austronesian}
\define@key{fams}{mmo}{Austronesian}
\define@key{fams}{zns}{Afro-Asiatic}
\define@key{fams}{xkb}{Atlantic-Congo}
\define@key{fams}{mqp}{Austronesian}
\define@key{fams}{nlm}{Indo-European}
\define@key{fams}{mml}{Austroasiatic}
\define@key{fams}{mjv}{Dravidian}
\define@key{fams}{woo}{Austronesian}
\define@key{fams}{msw}{Atlantic-Congo}
\define@key{fams}{msk}{Austronesian}
\define@key{fams}{nty}{Sino-Tibetan}
\define@key{fams}{myg}{Atlantic-Congo}
\define@key{fams}{kxf}{Sino-Tibetan}
\define@key{fams}{wha}{Austronesian}
\define@key{fams}{mxc}{Atlantic-Congo}
\define@key{fams}{mny}{Atlantic-Congo}
\define@key{fams}{mzj}{Mande}
\define@key{fams}{mzv}{Atlantic-Congo}
\define@key{fams}{mmd}{Tai-Kadai}
\define@key{fams}{mjn}{Nuclear Trans New Guinea}
\define@key{fams}{mlh}{Nuclear Trans New Guinea}
\define@key{fams}{mnm}{Dagan}
\define@key{fams}{mpy}{Austronesian}
\define@key{fams}{mpw}{Arawakan}
\define@key{fams}{bzh}{Austronesian}
\define@key{fams}{sjm}{Austronesian}
\define@key{fams}{vmh}{Indo-European}
\define@key{fams}{nma}{Sino-Tibetan}
\define@key{fams}{lrm}{Atlantic-Congo}
\define@key{fams}{lri}{Atlantic-Congo}
\define@key{fams}{mgb}{Tamaic}
\define@key{fams}{mvr}{Austronesian}
\define@key{fams}{mrs}{Austronesian}
\define@key{fams}{mpg}{Afro-Asiatic}
\define@key{fams}{dsz}{Sign Language}
\define@key{fams}{vmr}{Atlantic-Congo}
\define@key{fams}{mrx}{Tor-Orya}
\define@key{fams}{mvu}{Maban}
\define@key{fams}{mhg}{Marrku-Wurrugu}
\define@key{fams}{qvm}{Quechuan}
\define@key{fams}{mfm}{Afro-Asiatic}
\define@key{fams}{nsr}{Sign Language}
\define@key{fams}{mrr}{Dravidian}
\define@key{fams}{nng}{Sino-Tibetan}
\define@key{fams}{zmm}{Western Daly}
\define@key{fams}{zmj}{Western Daly}
\define@key{fams}{zmd}{Western Daly}
\define@key{fams}{zmy}{Western Daly}
\define@key{fams}{mrb}{Austronesian}
\define@key{fams}{dad}{Austronesian}
\define@key{fams}{hob}{Austronesian}
\define@key{fams}{mqi}{Austronesian}
\define@key{fams}{mbx}{Sepik}
\define@key{fams}{mds}{Manubaran}
\define@key{fams}{msp}{Tupian}
\define@key{fams}{enb}{Nilotic}
\define@key{fams}{rkm}{Mande}
\define@key{fams}{mvo}{Austronesian}
\define@key{fams}{xru}{Western Daly}
\define@key{fams}{mre}{Sign Language}
\define@key{fams}{zmg}{Western Daly}
\define@key{fams}{mzr}{Pano-Tacanan}
\define@key{fams}{mve}{Indo-European}
\define@key{fams}{rwr}{Indo-European}
\define@key{fams}{myx}{Atlantic-Congo}
\define@key{fams}{tis}{Austronesian}
\define@key{fams}{bks}{Austronesian}
\define@key{fams}{msb}{Austronesian}
\define@key{fams}{mho}{Atlantic-Congo}
\define@key{fams}{jms}{Atlantic-Congo}
\define@key{fams}{cuj}{Arawakan}
\define@key{fams}{ism}{Austronesian}
\define@key{fams}{bnf}{Austronesian}
\define@key{fams}{msh}{Austronesian}
\define@key{fams}{klv}{Austronesian}
\define@key{fams}{msv}{Afro-Asiatic}
\define@key{fams}{mes}{Afro-Asiatic}
\define@key{fams}{mdg}{Maban}
\define@key{fams}{mvs}{Isolate}
\define@key{fams}{mtn}{Misumalpan}
\define@key{fams}{mfh}{Afro-Asiatic}
\define@key{fams}{xmt}{Austronesian}
\define@key{fams}{mgv}{Atlantic-Congo}
\define@key{fams}{mqe}{Nuclear Trans New Guinea}
\define@key{fams}{mzo}{Cariban}
\define@key{fams}{mtm}{Uralic}
\define@key{fams}{met}{Austronesian}
\define@key{fams}{axg}{Isolate}
\define@key{fams}{stj}{Mande}
\define@key{fams}{cty}{Dravidian}
\define@key{fams}{lsy}{Sign Language}
\define@key{fams}{mhl}{Nuclear Trans New Guinea}
\define@key{fams}{wma}{Unattested}
\define@key{fams}{mjj}{Nuclear Trans New Guinea}
\define@key{fams}{mcz}{Nuclear Trans New Guinea}
\define@key{fams}{mcw}{Afro-Asiatic}
\define@key{fams}{mgk}{Isolate}
\define@key{fams}{mxl}{Atlantic-Congo}
\define@key{fams}{xmy}{Pama-Nyungan}
\define@key{fams}{sym}{Mande}
\define@key{fams}{mnt}{Pama-Nyungan}
\define@key{fams}{ifu}{Austronesian}
\define@key{fams}{mzl}{Mixe-Zoque}
\define@key{fams}{zpy}{Otomanguean}
\define@key{fams}{vmz}{Otomanguean}
\define@key{fams}{dkx}{Afro-Asiatic}
\define@key{fams}{mdp}{Atlantic-Congo}
\define@key{fams}{mgn}{Atlantic-Congo}
\define@key{fams}{zmz}{Atlantic-Congo}
\define@key{fams}{mxg}{Atlantic-Congo}
\define@key{fams}{zmn}{Atlantic-Congo}
\define@key{fams}{zmv}{Pama-Nyungan}
\define@key{fams}{mvl}{Pama-Nyungan}
\define@key{fams}{gwa}{Atlantic-Congo}
\define@key{fams}{mdn}{Atlantic-Congo}
\define@key{fams}{xmd}{Afro-Asiatic}
\define@key{fams}{mfo}{Atlantic-Congo}
\define@key{fams}{mql}{Atlantic-Congo}
\define@key{fams}{zms}{Atlantic-Congo}
\define@key{fams}{emz}{Atlantic-Congo}
\define@key{fams}{mbo}{Atlantic-Congo}
\define@key{fams}{zmw}{Atlantic-Congo}
\define@key{fams}{moi}{Atlantic-Congo}
\define@key{fams}{mdu}{Atlantic-Congo}
\define@key{fams}{xmb}{Atlantic-Congo}
\define@key{fams}{bgu}{Atlantic-Congo}
\define@key{fams}{mxo}{Atlantic-Congo}
\define@key{fams}{mka}{Atlantic-Congo}
\define@key{fams}{mgz}{Atlantic-Congo}
\define@key{fams}{mhw}{Atlantic-Congo}
\define@key{fams}{mqb}{Afro-Asiatic}
\define@key{fams}{bpc}{Atlantic-Congo}
\define@key{fams}{mbv}{Atlantic-Congo}
\define@key{fams}{mbu}{Atlantic-Congo}
\define@key{fams}{mlb}{Atlantic-Congo}
\define@key{fams}{mgy}{Atlantic-Congo}
\define@key{fams}{mck}{Atlantic-Congo}
\define@key{fams}{bbt}{Afro-Asiatic}
\define@key{fams}{muc}{Atlantic-Congo}
\define@key{fams}{mfu}{Atlantic-Congo}
\define@key{fams}{gun}{Tupian}
\define@key{fams}{mjm}{Austronesian}
\define@key{fams}{dmf}{Speech Register}
\define@key{fams}{mue}{Mixed Language}
\define@key{fams}{mud}{Eskimo-Aleut}
\define@key{fams}{byv}{Atlantic-Congo}
\define@key{fams}{mfj}{Afro-Asiatic}
\define@key{fams}{mef}{Austroasiatic}
\define@key{fams}{ruq}{Indo-European}
\define@key{fams}{mmh}{Arawakan}
\define@key{fams}{mvk}{Yuat}
\define@key{fams}{msf}{Nimboranic}
\define@key{fams}{hkn}{Austroasiatic}
\define@key{fams}{mfx}{Ta-Ne-Omotic}
\define@key{fams}{med}{Nuclear Trans New Guinea}
\define@key{fams}{mby}{Indo-European}
\define@key{fams}{mfd}{Atlantic-Congo}
\define@key{fams}{xkd}{Austronesian}
\define@key{fams}{sim}{Sepik}
\define@key{fams}{xmg}{Atlantic-Congo}
\define@key{fams}{mee}{Austronesian}
\define@key{fams}{mea}{Atlantic-Congo}
\define@key{fams}{mvx}{Austronesian}
\define@key{fams}{mxm}{Austronesian}
\define@key{fams}{lmb}{Austronesian}
\define@key{fams}{meq}{Afro-Asiatic}
\define@key{fams}{mrm}{Austronesian}
\define@key{fams}{xmr}{Isolate}
\define@key{fams}{mnu}{Mairasic}
\define@key{fams}{mer}{Atlantic-Congo}
\define@key{fams}{wry}{Indo-European}
\define@key{fams}{iyo}{Atlantic-Congo}
\define@key{fams}{mci}{Nuclear Trans New Guinea}
\define@key{fams}{zim}{Afro-Asiatic}
\define@key{fams}{mys}{Afro-Asiatic}
\define@key{fams}{mvz}{Afro-Asiatic}
\define@key{fams}{cms}{Indo-European}
\define@key{fams}{mgo}{Atlantic-Congo}
\define@key{fams}{mxv}{Otomanguean}
\define@key{fams}{mtr}{Indo-European}
\define@key{fams}{wtm}{Indo-European}
\define@key{fams}{mfs}{Sign Language}
\define@key{fams}{zmf}{Atlantic-Congo}
\define@key{fams}{nfu}{Atlantic-Congo}
\define@key{fams}{zam}{Otomanguean}
\define@key{fams}{pla}{Nuclear Trans New Guinea}
\define@key{fams}{xmi}{Unattested}
\define@key{fams}{mmc}{Otomanguean}
\define@key{fams}{enm}{Indo-European}
\define@key{fams}{gml}{Indo-European}
\define@key{fams}{dum}{Indo-European}
\define@key{fams}{mpl}{Austronesian}
\define@key{fams}{gmh}{Indo-European}
\define@key{fams}{ltc}{Sino-Tibetan}
\define@key{fams}{xng}{Mongolic-Khitan}
\define@key{fams}{dnt}{Nuclear Trans New Guinea}
\define@key{fams}{bjo}{Atlantic-Congo}
\define@key{fams}{mpp}{Nuclear Trans New Guinea}
\define@key{fams}{ymh}{Sino-Tibetan}
\define@key{fams}{mlj}{Afro-Asiatic}
\define@key{fams}{iml}{Coosan}
\define@key{fams}{imy}{Indo-European}
\define@key{fams}{mcv}{Anim}
\define@key{fams}{inm}{Afro-Asiatic}
\define@key{fams}{mnp}{Sino-Tibetan}
\define@key{fams}{mpn}{Austronesian}
\define@key{fams}{drc}{Indo-European}
\define@key{fams}{mko}{Atlantic-Congo}
\define@key{fams}{vmg}{Austronesian}
\define@key{fams}{wii}{Nuclear Torricelli}
\define@key{fams}{xxm}{Isolate}
\define@key{fams}{omn}{Unclassifiable}
\define@key{fams}{mqq}{Austronesian}
\define@key{fams}{mnq}{Austroasiatic}
\define@key{fams}{mzt}{Austroasiatic}
\define@key{fams}{czo}{Sino-Tibetan}
\define@key{fams}{zgm}{Tai-Kadai}
\define@key{fams}{yiq}{Sino-Tibetan}
\define@key{fams}{mwl}{Indo-European}
\define@key{fams}{mvh}{Afro-Asiatic}
\define@key{fams}{mmv}{Tucanoan}
\define@key{fams}{rsm}{Sign Language}
\define@key{fams}{mjs}{Afro-Asiatic}
\define@key{fams}{mpx}{Austronesian}
\define@key{fams}{vmm}{Otomanguean}
\define@key{fams}{mwu}{Central Sudanic}
\define@key{fams}{mpo}{Austronesian}
\define@key{fams}{vmi}{Worrorran}
\define@key{fams}{mfg}{Mande}
\define@key{fams}{mix}{Otomanguean}
\define@key{fams}{mvi}{Japonic}
\define@key{fams}{ehs}{Sign Language}
\define@key{fams}{soy}{Atlantic-Congo}
\define@key{fams}{lhs}{Afro-Asiatic}
\define@key{fams}{kja}{Nimboranic}
\define@key{fams}{mlo}{Atlantic-Congo}
\define@key{fams}{mmu}{Atlantic-Congo}
\define@key{fams}{bfm}{Atlantic-Congo}
\define@key{fams}{mfq}{Atlantic-Congo}
\define@key{fams}{mod}{Pidgin}
\define@key{fams}{ahm}{Kru}
\define@key{fams}{jkm}{Sino-Tibetan}
\define@key{fams}{mhn}{Indo-European}
\define@key{fams}{mhc}{Mayan}
\define@key{fams}{gbn}{Central Sudanic}
\define@key{fams}{mxd}{Austronesian}
\define@key{fams}{mqo}{North Halmahera}
\define@key{fams}{mvq}{Nuclear Trans New Guinea}
\define@key{fams}{mou}{Afro-Asiatic}
\define@key{fams}{mof}{Algic}
\define@key{fams}{mow}{Atlantic-Congo}
\define@key{fams}{mxn}{West Bird's Head}
\define@key{fams}{mkp}{Yareban}
\define@key{fams}{mwz}{Atlantic-Congo}
\define@key{fams}{ymi}{Sino-Tibetan}
\define@key{fams}{mft}{Austronesian}
\define@key{fams}{mwt}{Austronesian}
\define@key{fams}{mqt}{Austroasiatic}
\define@key{fams}{mkm}{Austronesian}
\define@key{fams}{mkl}{Atlantic-Congo}
\define@key{fams}{vms}{Unattested}
\define@key{fams}{pwm}{Austronesian}
\define@key{fams}{vsi}{Sign Language}
\define@key{fams}{bxc}{Atlantic-Congo}
\define@key{fams}{mox}{Austronesian}
\define@key{fams}{zmo}{Eastern Jebel}
\define@key{fams}{msl}{Isolate}
\define@key{fams}{mlw}{Afro-Asiatic}
\define@key{fams}{myl}{Austronesian}
\define@key{fams}{msz}{Nuclear Trans New Guinea}
\define@key{fams}{dmb}{Dogon}
\define@key{fams}{mmb}{Somahai}
\define@key{fams}{ver}{Atlantic-Congo}
\define@key{fams}{mzg}{Sign Language}
\define@key{fams}{npn}{Austronesian}
\define@key{fams}{msr}{Sign Language}
\define@key{fams}{mgt}{Keram}
\define@key{fams}{mom}{Otomanguean}
\define@key{fams}{moo}{Austroasiatic}
\define@key{fams}{mru}{Atlantic-Congo}
\define@key{fams}{mnh}{Atlantic-Congo}
\define@key{fams}{nmh}{Sino-Tibetan}
\define@key{fams}{mtl}{Afro-Asiatic}
\define@key{fams}{gwg}{Atlantic-Congo}
\define@key{fams}{crm}{Algic}
\define@key{fams}{msg}{West Bird's Head}
\define@key{fams}{mze}{Mailuan}
\define@key{fams}{moq}{Isolate}
\define@key{fams}{msx}{Nuclear Trans New Guinea}
\define@key{fams}{xmo}{Unattested}
\define@key{fams}{xmz}{Austronesian}
\define@key{fams}{mzq}{Austronesian}
\define@key{fams}{mdb}{Kiwaian}
\define@key{fams}{xms}{Sign Language}
\define@key{fams}{bdo}{Central Sudanic}
\define@key{fams}{mgc}{Central Sudanic}
\define@key{fams}{mrp}{Austronesian}
\define@key{fams}{mqn}{Austronesian}
\define@key{fams}{mrl}{Austronesian}
\define@key{fams}{mwy}{Nilotic}
\define@key{fams}{mqv}{Nuclear Trans New Guinea}
\define@key{fams}{mtj}{East Bird's Head}
\define@key{fams}{mtt}{Austronesian}
\define@key{fams}{mwh}{Austronesian}
\define@key{fams}{jmw}{Turama-Kikori}
\define@key{fams}{ity}{Austronesian}
\define@key{fams}{nmo}{Sino-Tibetan}
\define@key{fams}{mzy}{Sign Language}
\define@key{fams}{mxi}{Indo-European}
\define@key{fams}{xnq}{Atlantic-Congo}
\define@key{fams}{mpi}{Afro-Asiatic}
\define@key{fams}{mcx}{Atlantic-Congo}
\define@key{fams}{mpz}{Sino-Tibetan}
\define@key{fams}{pnd}{Atlantic-Congo}
\define@key{fams}{mgg}{Atlantic-Congo}
\define@key{fams}{mpa}{Atlantic-Congo}
\define@key{fams}{mvt}{Austronesian}
\define@key{fams}{zmp}{Atlantic-Congo}
\define@key{fams}{cmr}{Sino-Tibetan}
\define@key{fams}{mro}{Sino-Tibetan}
\define@key{fams}{kqx}{Afro-Asiatic}
\define@key{fams}{agz}{Austronesian}
\define@key{fams}{atl}{Austronesian}
\define@key{fams}{mtd}{Austronesian}
\define@key{fams}{tsx}{Anim}
\define@key{fams}{mub}{Afro-Asiatic}
\define@key{fams}{ymd}{Sino-Tibetan}
\define@key{fams}{gau}{Dravidian}
\define@key{fams}{udg}{Dravidian}
\define@key{fams}{vmd}{Dravidian}
\define@key{fams}{wiv}{Austronesian}
\define@key{fams}{muk}{Sino-Tibetan}
\define@key{fams}{mmk}{Dravidian}
\define@key{fams}{mfw}{Kwalean}
\define@key{fams}{kpb}{Dravidian}
\define@key{fams}{vmu}{Pama-Nyungan}
\define@key{fams}{kqa}{Nuclear Trans New Guinea}
\define@key{fams}{mwq}{Sino-Tibetan}
\define@key{fams}{boe}{Atlantic-Congo}
\define@key{fams}{mmf}{Afro-Asiatic}
\define@key{fams}{mth}{Austronesian}
\define@key{fams}{mpv}{Nuclear Trans New Guinea}
\define@key{fams}{mtc}{Nuclear Trans New Guinea}
\define@key{fams}{myr}{Isolate}
\define@key{fams}{mnj}{Indo-European}
\define@key{fams}{asx}{Nuclear Trans New Guinea}
\define@key{fams}{mxr}{Austronesian}
\define@key{fams}{rmh}{Lepki-Murkim-Kembra}
\define@key{fams}{tkv}{Austronesian}
\define@key{fams}{mqw}{Nuclear Trans New Guinea}
\define@key{fams}{smm}{Indo-European}
\define@key{fams}{mmi}{Nuclear Trans New Guinea}
\define@key{fams}{mmq}{Nuclear Trans New Guinea}
\define@key{fams}{mse}{Afro-Asiatic}
\define@key{fams}{mui}{Austronesian}
\define@key{fams}{mje}{Afro-Asiatic}
\define@key{fams}{muv}{Dravidian}
\define@key{fams}{tuc}{Austronesian}
\define@key{fams}{muy}{Afro-Asiatic}
\define@key{fams}{ymz}{Sino-Tibetan}
\define@key{fams}{mcj}{Atlantic-Congo}
\define@key{fams}{mxh}{Central Sudanic}
\define@key{fams}{wlc}{Atlantic-Congo}
\define@key{fams}{wmw}{Atlantic-Congo}
\define@key{fams}{moa}{Mande}
\define@key{fams}{mwa}{Austronesian}
\define@key{fams}{mjh}{Atlantic-Congo}
\define@key{fams}{mws}{Atlantic-Congo}
\define@key{fams}{gmy}{Indo-European}
\define@key{fams}{nme}{Sino-Tibetan}
\define@key{fams}{nbt}{Sino-Tibetan}
\define@key{fams}{nao}{Sino-Tibetan}
\define@key{fams}{mne}{Central Sudanic}
\define@key{fams}{mty}{Nuclear Torricelli}
\define@key{fams}{ncd}{Sino-Tibetan}
\define@key{fams}{srf}{Austronesian}
\define@key{fams}{nxx}{Sentanic}
\define@key{fams}{jbn}{Afro-Asiatic}
\define@key{fams}{nbg}{Unattested}
\define@key{fams}{nxe}{Austronesian}
\define@key{fams}{ngv}{Atlantic-Congo}
\define@key{fams}{nlx}{Indo-European}
\define@key{fams}{nhh}{Indo-European}
\define@key{fams}{ars}{Afro-Asiatic}
\define@key{fams}{nae}{Austronesian}
\define@key{fams}{nib}{Nuclear Trans New Guinea}
\define@key{fams}{nkj}{Nuclear Trans New Guinea}
\define@key{fams}{nbk}{Nuclear Trans New Guinea}
\define@key{fams}{mff}{Atlantic-Congo}
\define@key{fams}{nax}{Left May}
\define@key{fams}{nlc}{Nuclear Trans New Guinea}
\define@key{fams}{nss}{Austronesian}
\define@key{fams}{nlz}{Austronesian}
\define@key{fams}{ylo}{Sino-Tibetan}
\define@key{fams}{naj}{Atlantic-Congo}
\define@key{fams}{nmx}{Yam}
\define@key{fams}{nkm}{Yam}
\define@key{fams}{nmk}{Austronesian}
\define@key{fams}{nmq}{Atlantic-Congo}
\define@key{fams}{ncm}{Yam}
\define@key{fams}{neo}{Unclassifiable}
\define@key{fams}{nbs}{Sign Language}
\define@key{fams}{nvm}{Koiarian}
\define@key{fams}{naa}{Namla-Tofanma}
\define@key{fams}{mxw}{Yam}
\define@key{fams}{nmt}{Austronesian}
\define@key{fams}{bwb}{Austronesian}
\define@key{fams}{nmy}{Sino-Tibetan}
\define@key{fams}{nnc}{Afro-Asiatic}
\define@key{fams}{nzz}{Dogon}
\define@key{fams}{ngr}{Austronesian}
\define@key{fams}{cox}{Arawakan}
\define@key{fams}{afk}{Arafundi}
\define@key{fams}{qvo}{Quechuan}
\define@key{fams}{nrg}{Austronesian}
\define@key{fams}{nac}{Nuclear Trans New Guinea}
\define@key{fams}{loh}{Surmic}
\define@key{fams}{nnr}{Pama-Nyungan}
\define@key{fams}{nsy}{Austronesian}
\define@key{fams}{nvh}{Austronesian}
\define@key{fams}{ntz}{Indo-European}
\define@key{fams}{nte}{Atlantic-Congo}
\define@key{fams}{nti}{Atlantic-Congo}
\define@key{fams}{nxa}{Austronesian}
\define@key{fams}{ncn}{Austronesian}
\define@key{fams}{nwo}{Pama-Nyungan}
\define@key{fams}{nsw}{Austronesian}
\define@key{fams}{nwr}{Yareban}
\define@key{fams}{nwa}{Algic}
\define@key{fams}{nmz}{Atlantic-Congo}
\define@key{fams}{naw}{Atlantic-Congo}
\define@key{fams}{nyq}{Indo-European}
\define@key{fams}{noz}{Dizoid}
\define@key{fams}{ncr}{Atlantic-Congo}
\define@key{fams}{nlu}{Atlantic-Congo}
\define@key{fams}{gke}{Atlantic-Congo}
\define@key{fams}{ndk}{Atlantic-Congo}
\define@key{fams}{ndh}{Atlantic-Congo}
\define@key{fams}{ndj}{Atlantic-Congo}
\define@key{fams}{ndm}{Afro-Asiatic}
\define@key{fams}{nxo}{Atlantic-Congo}
\define@key{fams}{nnz}{Atlantic-Congo}
\define@key{fams}{nda}{Atlantic-Congo}
\define@key{fams}{ndc}{Atlantic-Congo}
\define@key{fams}{nml}{Atlantic-Congo}
\define@key{fams}{ndg}{Atlantic-Congo}
\define@key{fams}{dne}{Atlantic-Congo}
\define@key{fams}{ndd}{Atlantic-Congo}
\define@key{fams}{eli}{Narrow Talodi}
\define@key{fams}{ndw}{Atlantic-Congo}
\define@key{fams}{nbb}{Atlantic-Congo}
\define@key{fams}{ndl}{Atlantic-Congo}
\define@key{fams}{ndq}{Atlantic-Congo}
\define@key{fams}{nqm}{Kolopom}
\define@key{fams}{ndr}{Atlantic-Congo}
\define@key{fams}{ndp}{Central Sudanic}
\define@key{fams}{dno}{Central Sudanic}
\define@key{fams}{ndx}{Nuclear Trans New Guinea}
\define@key{fams}{nuh}{Atlantic-Congo}
\define@key{fams}{nww}{Atlantic-Congo}
\define@key{fams}{njt}{Pidgin}
\define@key{fams}{wni}{Atlantic-Congo}
\define@key{fams}{nec}{Timor-Alor-Pantar}
\define@key{fams}{nef}{Pidgin}
\define@key{fams}{dcr}{Indo-European}
\define@key{fams}{nkg}{Nuclear Trans New Guinea}
\define@key{fams}{nif}{Nuclear Trans New Guinea}
\define@key{fams}{nej}{Nuclear Trans New Guinea}
\define@key{fams}{nek}{Austronesian}
\define@key{fams}{nex}{Yam}
\define@key{fams}{nem}{Austronesian}
\define@key{fams}{nqn}{Yam}
\define@key{fams}{neu}{Artificial Language}
\define@key{fams}{nsp}{Sign Language}
\define@key{fams}{net}{Nuclear Trans New Guinea}
\define@key{fams}{jas}{Austronesian}
\define@key{fams}{jui}{Pama-Nyungan}
\define@key{fams}{nnf}{Nuclear Trans New Guinea}
\define@key{fams}{hlt}{Sino-Tibetan}
\define@key{fams}{szb}{Nuclear Trans New Guinea}
\define@key{fams}{nud}{Ndu}
\define@key{fams}{nmv}{Pama-Nyungan}
\define@key{fams}{nbv}{Atlantic-Congo}
\define@key{fams}{nmc}{Central Sudanic}
\define@key{fams}{nbh}{Afro-Asiatic}
\define@key{fams}{nyx}{Pama-Nyungan}
\define@key{fams}{gng}{Atlantic-Congo}
\define@key{fams}{nne}{Atlantic-Congo}
\define@key{fams}{nxd}{Atlantic-Congo}
\define@key{fams}{ngd}{Atlantic-Congo}
\define@key{fams}{nji}{Mirndi}
\define@key{fams}{rxd}{Pama-Nyungan}
\define@key{fams}{nsg}{Nilotic}
\define@key{fams}{ngm}{Speech Register}
\define@key{fams}{cnw}{Sino-Tibetan}
\define@key{fams}{zdj}{Atlantic-Congo}
\define@key{fams}{ngg}{Atlantic-Congo}
\define@key{fams}{jgb}{Atlantic-Congo}
\define@key{fams}{nbd}{Atlantic-Congo}
\define@key{fams}{nuu}{Atlantic-Congo}
\define@key{fams}{gnj}{Mande}
\define@key{fams}{nql}{Atlantic-Congo}
\define@key{fams}{ngt}{Austroasiatic}
\define@key{fams}{nnn}{Afro-Asiatic}
\define@key{fams}{nbq}{Nuclear Trans New Guinea}
\define@key{fams}{ngx}{Afro-Asiatic}
\define@key{fams}{nnh}{Atlantic-Congo}
\define@key{fams}{ngj}{Atlantic-Congo}
\define@key{fams}{nnq}{Atlantic-Congo}
\define@key{fams}{nra}{Atlantic-Congo}
\define@key{fams}{nla}{Atlantic-Congo}
\define@key{fams}{jgo}{Atlantic-Congo}
\define@key{fams}{noq}{Atlantic-Congo}
\define@key{fams}{nsh}{Atlantic-Congo}
\define@key{fams}{nuw}{Austronesian}
\define@key{fams}{ngp}{Atlantic-Congo}
\define@key{fams}{nlo}{Atlantic-Congo}
\define@key{fams}{xnm}{Nyulnyulan}
\define@key{fams}{nui}{Atlantic-Congo}
\define@key{fams}{nue}{Atlantic-Congo}
\define@key{fams}{ndn}{Atlantic-Congo}
\define@key{fams}{ngz}{Atlantic-Congo}
\define@key{fams}{nuo}{Austroasiatic}
\define@key{fams}{nrx}{Unattested}
\define@key{fams}{nbx}{Pama-Nyungan}
\define@key{fams}{ngq}{Atlantic-Congo}
\define@key{fams}{ngw}{Afro-Asiatic}
\define@key{fams}{nwe}{Atlantic-Congo}
\define@key{fams}{ngn}{Atlantic-Congo}
\define@key{fams}{yrl}{Tupian}
\define@key{fams}{nhf}{Pama-Nyungan}
\define@key{fams}{ncs}{Sign Language}
\define@key{fams}{nsi}{Sign Language}
\define@key{fams}{mzk}{Atlantic-Congo}
\define@key{fams}{nii}{Nuclear Trans New Guinea}
\define@key{fams}{xny}{Pama-Nyungan}
\define@key{fams}{gbe}{Sepik}
\define@key{fams}{nim}{Atlantic-Congo}
\define@key{fams}{nil}{Austronesian}
\define@key{fams}{noe}{Indo-European}
\define@key{fams}{nmp}{Nyulnyulan}
\define@key{fams}{nmr}{Atlantic-Congo}
\define@key{fams}{nis}{Nuclear Trans New Guinea}
\define@key{fams}{nmw}{Austronesian}
\define@key{fams}{niw}{Left May}
\define@key{fams}{nxi}{Atlantic-Congo}
\define@key{fams}{nxr}{Nuclear Trans New Guinea}
\define@key{fams}{nby}{Border}
\define@key{fams}{nlk}{Nuclear Trans New Guinea}
\define@key{fams}{nin}{Atlantic-Congo}
\define@key{fams}{nps}{Nuclear Trans New Guinea}
\define@key{fams}{njs}{Geelvink Bay}
\define@key{fams}{yso}{Sino-Tibetan}
\define@key{fams}{nkp}{Austronesian}
\define@key{fams}{njl}{Dajuic}
\define@key{fams}{nzb}{Atlantic-Congo}
\define@key{fams}{njj}{Atlantic-Congo}
\define@key{fams}{njr}{Atlantic-Congo}
\define@key{fams}{njy}{Atlantic-Congo}
\define@key{fams}{nkq}{Atlantic-Congo}
\define@key{fams}{nkn}{Atlantic-Congo}
\define@key{fams}{nkz}{Atlantic-Congo}
\define@key{fams}{khu}{Atlantic-Congo}
\define@key{fams}{nqo}{Artificial Language}
\define@key{fams}{nkc}{Atlantic-Congo}
\define@key{fams}{nkx}{Ijoid}
\define@key{fams}{nka}{Atlantic-Congo}
\define@key{fams}{nbo}{Atlantic-Congo}
\define@key{fams}{nkw}{Atlantic-Congo}
\define@key{fams}{nbp}{Atlantic-Congo}
\define@key{fams}{ngh}{Tuu}
\define@key{fams}{gaw}{Nuclear Trans New Guinea}
\define@key{fams}{noi}{Indo-European}
\define@key{fams}{nkk}{Austronesian}
\define@key{fams}{lem}{Atlantic-Congo}
\define@key{fams}{nof}{Nuclear Trans New Guinea}
\define@key{fams}{noh}{Nuclear Trans New Guinea}
\define@key{fams}{zhn}{Tai-Kadai}
\define@key{fams}{noj}{Huitotoan}
\define@key{fams}{nok}{Salishan}
\define@key{fams}{nrc}{Indo-European}
\define@key{fams}{nrp}{Unclassifiable}
\define@key{fams}{huj}{Hmong-Mien}
\define@key{fams}{hmp}{Hmong-Mien}
\define@key{fams}{crl}{Algic}
\define@key{fams}{pbu}{Indo-European}
\define@key{fams}{hno}{Indo-European}
\define@key{fams}{glh}{Indo-European}
\define@key{fams}{aee}{Indo-European}
\define@key{fams}{kxm}{Austroasiatic}
\define@key{fams}{atv}{Turkic}
\define@key{fams}{azj}{Turkic}
\define@key{fams}{ghh}{Sino-Tibetan}
\define@key{fams}{ymx}{Sino-Tibetan}
\define@key{fams}{yiv}{Sino-Tibetan}
\define@key{fams}{cng}{Sino-Tibetan}
\define@key{fams}{bfc}{Sino-Tibetan}
\define@key{fams}{nnl}{Sino-Tibetan}
\define@key{fams}{lbr}{Sino-Tibetan}
\define@key{fams}{tji}{Sino-Tibetan}
\define@key{fams}{doc}{Tai-Kadai}
\define@key{fams}{nod}{Tai-Kadai}
\define@key{fams}{tts}{Tai-Kadai}
\define@key{fams}{hea}{Hmong-Mien}
\define@key{fams}{hmi}{Hmong-Mien}
\define@key{fams}{kqs}{Atlantic-Congo}
\define@key{fams}{fll}{Atlantic-Congo}
\define@key{fams}{dgi}{Atlantic-Congo}
\define@key{fams}{tsp}{Atlantic-Congo}
\define@key{fams}{gbo}{Kru}
\define@key{fams}{dip}{Nilotic}
\define@key{fams}{diw}{Nilotic}
\define@key{fams}{max}{Austronesian}
\define@key{fams}{mmg}{Austronesian}
\define@key{fams}{mrq}{Austronesian}
\define@key{fams}{tnn}{Austronesian}
\define@key{fams}{una}{Austronesian}
\define@key{fams}{bcd}{Austronesian}
\define@key{fams}{weo}{Austronesian}
\define@key{fams}{nni}{Austronesian}
\define@key{fams}{aqn}{Austronesian}
\define@key{fams}{xnn}{Austronesian}
\define@key{fams}{cts}{Austronesian}
\define@key{fams}{stb}{Austronesian}
\define@key{fams}{bmm}{Austronesian}
\define@key{fams}{onr}{Nuclear Torricelli}
\define@key{fams}{kti}{Nuclear Trans New Guinea}
\define@key{fams}{nks}{Nuclear Trans New Guinea}
\define@key{fams}{yir}{Nuclear Trans New Guinea}
\define@key{fams}{whg}{Nuclear Trans New Guinea}
\define@key{fams}{kiw}{Kiwaian}
\define@key{fams}{ryn}{Japonic}
\define@key{fams}{neq}{Mixe-Zoque}
\define@key{fams}{scs}{Athabaskan-Eyak-Tlingit}
\define@key{fams}{esk}{Eskimo-Aleut}
\define@key{fams}{thh}{Uto-Aztecan}
\define@key{fams}{nhy}{Uto-Aztecan}
\define@key{fams}{ojb}{Algic}
\define@key{fams}{pef}{Pomoan}
\define@key{fams}{cst}{Miwok-Costanoan}
\define@key{fams}{enl}{Lengua-Mascoy}
\define@key{fams}{qvz}{Quechuan}
\define@key{fams}{qul}{Quechuan}
\define@key{fams}{qxn}{Quechuan}
\define@key{fams}{pmq}{Otomanguean}
\define@key{fams}{xtn}{Otomanguean}
\define@key{fams}{mxa}{Otomanguean}
\define@key{fams}{mfk}{Afro-Asiatic}
\define@key{fams}{ayp}{Afro-Asiatic}
\define@key{fams}{ntd}{Austronesian}
\define@key{fams}{cnp}{Sino-Tibetan}
\define@key{fams}{ncq}{Austroasiatic}
\define@key{fams}{bly}{Atlantic-Congo}
\define@key{fams}{ncf}{Austronesian}
\define@key{fams}{ntw}{Iroquoian}
\define@key{fams}{nov}{Artificial Language}
\define@key{fams}{noy}{Atlantic-Congo}
\define@key{fams}{asj}{Atlantic-Congo}
\define@key{fams}{nsc}{Unattested}
\define@key{fams}{nsx}{Atlantic-Congo}
\define@key{fams}{baf}{Atlantic-Congo}
\define@key{fams}{kte}{Sino-Tibetan}
\define@key{fams}{wbm}{Austroasiatic}
\define@key{fams}{bsq}{Kru}
\define@key{fams}{wla}{Walioic}
\define@key{fams}{wgi}{Nuclear Trans New Guinea}
\define@key{fams}{gyz}{Afro-Asiatic}
\define@key{fams}{nqt}{Afro-Asiatic}
\define@key{fams}{nnv}{Pama-Nyungan}
\define@key{fams}{noc}{Nuclear Trans New Guinea}
\define@key{fams}{klt}{Nuclear Trans New Guinea}
\define@key{fams}{nuq}{Austronesian}
\define@key{fams}{nur}{Austronesian}
\define@key{fams}{nuc}{Pano-Tacanan}
\define@key{fams}{nbr}{Atlantic-Congo}
\define@key{fams}{nop}{Nuclear Trans New Guinea}
\define@key{fams}{sij}{Austronesian}
\define@key{fams}{tgs}{Austronesian}
\define@key{fams}{kdk}{Austronesian}
\define@key{fams}{nxm}{Unclassifiable}
\define@key{fams}{nug}{Mirndi}
\define@key{fams}{rin}{Atlantic-Congo}
\define@key{fams}{nul}{Austronesian}
\define@key{fams}{nwb}{Kru}
\define@key{fams}{nev}{Austroasiatic}
\define@key{fams}{nyy}{Atlantic-Congo}
\define@key{fams}{nlj}{Atlantic-Congo}
\define@key{fams}{mwn}{Atlantic-Congo}
\define@key{fams}{nwm}{Central Sudanic}
\define@key{fams}{nmi}{Afro-Asiatic}
\define@key{fams}{nny}{Tangkic}
\define@key{fams}{nyb}{Atlantic-Congo}
\define@key{fams}{nyc}{Atlantic-Congo}
\define@key{fams}{nyk}{Atlantic-Congo}
\define@key{fams}{nnj}{Nilotic}
\define@key{fams}{sev}{Atlantic-Congo}
\define@key{fams}{nba}{Atlantic-Congo}
\define@key{fams}{neh}{Sino-Tibetan}
\define@key{fams}{nye}{Atlantic-Congo}
\define@key{fams}{nyl}{Austroasiatic}
\define@key{fams}{nyr}{Atlantic-Congo}
\define@key{fams}{nkv}{Atlantic-Congo}
\define@key{fams}{nkt}{Atlantic-Congo}
\define@key{fams}{nyg}{Atlantic-Congo}
\define@key{fams}{lid}{Austronesian}
\define@key{fams}{nvo}{Atlantic-Congo}
\define@key{fams}{nuj}{Atlantic-Congo}
\define@key{fams}{muo}{Atlantic-Congo}
\define@key{fams}{nyd}{Atlantic-Congo}
\define@key{fams}{nyu}{Atlantic-Congo}
\define@key{fams}{nzd}{Atlantic-Congo}
\define@key{fams}{nzy}{Atlantic-Congo}
\define@key{fams}{nja}{Afro-Asiatic}
\define@key{fams}{nzi}{Atlantic-Congo}
\define@key{fams}{bzy}{Atlantic-Congo}
\define@key{fams}{obi}{Chumashan}
\define@key{fams}{obl}{Atlantic-Congo}
\define@key{fams}{obo}{Austronesian}
\define@key{fams}{obu}{Atlantic-Congo}
\define@key{fams}{zac}{Otomanguean}
\define@key{fams}{odk}{Indo-European}
\define@key{fams}{bhf}{Isolate}
\define@key{fams}{kkc}{East Strickland}
\define@key{fams}{odu}{Atlantic-Congo}
\define@key{fams}{tyh}{Austroasiatic}
\define@key{fams}{opy}{Nuclear-Macro-Je}
\define@key{fams}{ofo}{Siouan}
\define@key{fams}{ogc}{Atlantic-Congo}
\define@key{fams}{ogg}{Atlantic-Congo}
\define@key{fams}{eri}{Nuclear Trans New Guinea}
\define@key{fams}{oia}{Timor-Alor-Pantar}
\define@key{fams}{chj}{Otomanguean}
\define@key{fams}{oki}{Nilotic}
\define@key{fams}{okn}{Japonic}
\define@key{fams}{okb}{Atlantic-Congo}
\define@key{fams}{okd}{Ijoid}
\define@key{fams}{oks}{Atlantic-Congo}
\define@key{fams}{okj}{Great Andamanese}
\define@key{fams}{kqv}{Austronesian}
\define@key{fams}{oie}{Nilotic}
\define@key{fams}{opa}{Atlantic-Congo}
\define@key{fams}{okx}{Atlantic-Congo}
\define@key{fams}{oke}{Atlantic-Congo}
\define@key{fams}{oar}{Afro-Asiatic}
\define@key{fams}{obr}{Sino-Tibetan}
\define@key{fams}{och}{Sino-Tibetan}
\define@key{fams}{odt}{Indo-European}
\define@key{fams}{ang}{Indo-European}
\define@key{fams}{fro}{Indo-European}
\define@key{fams}{ofs}{Indo-European}
\define@key{fams}{oge}{Kartvelian}
\define@key{fams}{goh}{Indo-European}
\define@key{fams}{sga}{Indo-European}
\define@key{fams}{ojp}{Japonic}
\define@key{fams}{okl}{Sign Language}
\define@key{fams}{qok}{Austroasiatic}
\define@key{fams}{qkn}{Dravidian}
\define@key{fams}{qbb}{Indo-European}
\define@key{fams}{omx}{Austroasiatic}
\define@key{fams}{omr}{Indo-European}
\define@key{fams}{non}{Indo-European}
\define@key{fams}{onw}{Nubian}
\define@key{fams}{oos}{Indo-European}
\define@key{fams}{pro}{Indo-European}
\define@key{fams}{peo}{Indo-European}
\define@key{fams}{orv}{Indo-European}
\define@key{fams}{osp}{Indo-European}
\define@key{fams}{osx}{Indo-European}
\define@key{fams}{oty}{Dravidian}
\define@key{fams}{oui}{Turkic}
\define@key{fams}{owl}{Indo-European}
\define@key{fams}{ole}{Sino-Tibetan}
\define@key{fams}{olm}{Atlantic-Congo}
\define@key{fams}{lul}{Central Sudanic}
\define@key{fams}{iko}{Atlantic-Congo}
\define@key{fams}{acx}{Afro-Asiatic}
\define@key{fams}{oml}{Atlantic-Congo}
\define@key{fams}{nht}{Uto-Aztecan}
\define@key{fams}{omi}{Central Sudanic}
\define@key{fams}{omt}{Nilotic}
\define@key{fams}{omu}{Isolate}
\define@key{fams}{oog}{Austroasiatic}
\define@key{fams}{onx}{Pidgin}
\define@key{fams}{oni}{Austronesian}
\define@key{fams}{onj}{Dagan}
\define@key{fams}{onn}{Bosavi}
\define@key{fams}{oor}{Indo-European}
\define@key{fams}{opo}{Eleman}
\define@key{fams}{opt}{Uto-Aztecan}
\define@key{fams}{lgn}{Koman}
\define@key{fams}{orn}{Austronesian}
\define@key{fams}{ors}{Austronesian}
\define@key{fams}{sdr}{Indo-European}
\define@key{fams}{org}{Atlantic-Congo}
\define@key{fams}{nlv}{Uto-Aztecan}
\define@key{fams}{fnb}{Austronesian}
\define@key{fams}{orc}{Afro-Asiatic}
\define@key{fams}{orz}{Austronesian}
\define@key{fams}{ora}{Austronesian}
\define@key{fams}{orx}{Atlantic-Congo}
\define@key{fams}{orh}{Tungusic}
\define@key{fams}{bpk}{Austronesian}
\define@key{fams}{orw}{Chapacuran}
\define@key{fams}{orr}{Ijoid}
\define@key{fams}{syx}{Atlantic-Congo}
\define@key{fams}{ost}{Atlantic-Congo}
\define@key{fams}{osc}{Indo-European}
\define@key{fams}{osi}{Austronesian}
\define@key{fams}{oso}{Atlantic-Congo}
\define@key{fams}{uta}{Atlantic-Congo}
\define@key{fams}{otd}{Austronesian}
\define@key{fams}{oti}{Isolate}
\define@key{fams}{otw}{Algic}
\define@key{fams}{lot}{Nilotic}
\define@key{fams}{otu}{Bororoan}
\define@key{fams}{oum}{Austronesian}
\define@key{fams}{oue}{South Bougainville}
\define@key{fams}{stn}{Austronesian}
\define@key{fams}{wsr}{Nuclear Trans New Guinea}
\define@key{fams}{oyy}{Austronesian}
\define@key{fams}{oyd}{Ta-Ne-Omotic}
\define@key{fams}{zao}{Otomanguean}
\define@key{fams}{chz}{Otomanguean}
\define@key{fams}{pfa}{Austronesian}
\define@key{fams}{sig}{Atlantic-Congo}
\define@key{fams}{qvp}{Quechuan}
\define@key{fams}{pcp}{Pano-Tacanan}
\define@key{fams}{pdi}{Tai-Kadai}
\define@key{fams}{pkc}{Unclassifiable}
\define@key{fams}{pae}{Atlantic-Congo}
\define@key{fams}{pgi}{Border}
\define@key{fams}{phr}{Indo-European}
\define@key{fams}{phj}{Sino-Tibetan}
\define@key{fams}{lgt}{Sepik}
\define@key{fams}{phv}{Indo-European}
\define@key{fams}{pal}{Indo-European}
\define@key{fams}{pha}{Hmong-Mien}
\define@key{fams}{pri}{Austronesian}
\define@key{fams}{ppi}{Cochimi-Yuman}
\define@key{fams}{qpp}{Indo-European}
\define@key{fams}{pta}{Tupian}
\define@key{fams}{pkg}{Austronesian}
\define@key{fams}{jkp}{Sino-Tibetan}
\define@key{fams}{pku}{Austronesian}
\define@key{fams}{pfl}{Indo-European}
\define@key{fams}{plq}{Indo-European}
\define@key{fams}{plr}{Atlantic-Congo}
\define@key{fams}{pln}{Indo-European}
\define@key{fams}{pnl}{Atlantic-Congo}
\define@key{fams}{pli}{Indo-European}
\define@key{fams}{pcf}{Dravidian}
\define@key{fams}{pmd}{Pama-Nyungan}
\define@key{fams}{abw}{Nuclear Trans New Guinea}
\define@key{fams}{pmc}{Unattested}
\define@key{fams}{ple}{Austronesian}
\define@key{fams}{plz}{Austronesian}
\define@key{fams}{bpx}{Indo-European}
\define@key{fams}{pmb}{Atlantic-Congo}
\define@key{fams}{pmn}{Atlantic-Congo}
\define@key{fams}{hih}{Nuclear Trans New Guinea}
\define@key{fams}{att}{Austronesian}
\define@key{fams}{pnz}{Atlantic-Congo}
\define@key{fams}{pnq}{Atlantic-Congo}
\define@key{fams}{pwb}{Atlantic-Congo}
\define@key{fams}{psn}{Austronesian}
\define@key{fams}{qxh}{Quechuan}
\define@key{fams}{lsp}{Sign Language}
\define@key{fams}{tdb}{Indo-European}
\define@key{fams}{pnp}{Austronesian}
\define@key{fams}{bkj}{Atlantic-Congo}
\define@key{fams}{pgg}{Indo-European}
\define@key{fams}{pgs}{Atlantic-Congo}
\define@key{fams}{slm}{Austronesian}
\define@key{fams}{pcg}{Dravidian}
\define@key{fams}{pnr}{Nuclear Trans New Guinea}
\define@key{fams}{pax}{Unattested}
\define@key{fams}{pkh}{Sino-Tibetan}
\define@key{fams}{paz}{Isolate}
\define@key{fams}{pnc}{Austronesian}
\define@key{fams}{knt}{Pano-Tacanan}
\define@key{fams}{pno}{Pano-Tacanan}
\define@key{fams}{blk}{Sino-Tibetan}
\define@key{fams}{ppv}{Unattested}
\define@key{fams}{ppn}{Austronesian}
\define@key{fams}{dpp}{Austronesian}
\define@key{fams}{pas}{Lakes Plain}
\define@key{fams}{pbo}{Atlantic-Congo}
\define@key{fams}{ppe}{Isolate}
\define@key{fams}{ppu}{Austronesian}
\define@key{fams}{ppm}{Austronesian}
\define@key{fams}{pgz}{Sign Language}
\define@key{fams}{prc}{Indo-European}
\define@key{fams}{pzn}{Sino-Tibetan}
\define@key{fams}{prf}{Austronesian}
\define@key{fams}{prw}{Nuclear Trans New Guinea}
\define@key{fams}{aap}{Cariban}
\define@key{fams}{pak}{Tupian}
\define@key{fams}{paf}{Tupian}
\define@key{fams}{gvp}{Nuclear-Macro-Je}
\define@key{fams}{pbg}{Arawakan}
\define@key{fams}{pys}{Sign Language}
\define@key{fams}{pcl}{Indo-European}
\define@key{fams}{pch}{Unattested}
\define@key{fams}{pcj}{Austroasiatic}
\define@key{fams}{ppt}{Kamula-Elevala}
\define@key{fams}{kvx}{Indo-European}
\define@key{fams}{xpr}{Indo-European}
\define@key{fams}{paq}{Indo-European}
\define@key{fams}{psq}{Sepik}
\define@key{fams}{yac}{Nuclear Trans New Guinea}
\define@key{fams}{ptn}{Austronesian}
\define@key{fams}{pth}{Nuclear-Macro-Je}
\define@key{fams}{pbc}{Cariban}
\define@key{fams}{pty}{Dravidian}
\define@key{fams}{ptq}{Dravidian}
\define@key{fams}{mfa}{Austronesian}
\define@key{fams}{pnk}{Arawakan}
\define@key{fams}{bfb}{Indo-European}
\define@key{fams}{psm}{Tupian}
\define@key{fams}{pmr}{Nuclear Trans New Guinea}
\define@key{fams}{pcb}{Austroasiatic}
\define@key{fams}{xpc}{Turkic}
\define@key{fams}{pai}{Atlantic-Congo}
\define@key{fams}{pfe}{Atlantic-Congo}
\define@key{fams}{ppq}{Walioic}
\define@key{fams}{pel}{Austronesian}
\define@key{fams}{bxd}{Sino-Tibetan}
\define@key{fams}{ata}{Isolate}
\define@key{fams}{pev}{Cariban}
\define@key{fams}{psg}{Sign Language}
\define@key{fams}{pek}{Austronesian}
\define@key{fams}{ums}{Austronesian}
\define@key{fams}{pdc}{Indo-European}
\define@key{fams}{pnh}{Austronesian}
\define@key{fams}{ptw}{Salishan}
\define@key{fams}{pea}{Austronesian}
\define@key{fams}{wet}{Austronesian}
\define@key{fams}{psc}{Sign Language}
\define@key{fams}{prl}{Sign Language}
\define@key{fams}{pex}{Austronesian}
\define@key{fams}{zpe}{Otomanguean}
\define@key{fams}{pey}{Indo-European}
\define@key{fams}{prt}{Austroasiatic}
\define@key{fams}{phk}{Tai-Kadai}
\define@key{fams}{phl}{Indo-European}
\define@key{fams}{ypa}{Sino-Tibetan}
\define@key{fams}{phq}{Sino-Tibetan}
\define@key{fams}{pem}{Atlantic-Congo}
\define@key{fams}{psp}{Sign Language}
\define@key{fams}{phm}{Atlantic-Congo}
\define@key{fams}{phn}{Afro-Asiatic}
\define@key{fams}{yip}{Sino-Tibetan}
\define@key{fams}{ypg}{Sino-Tibetan}
\define@key{fams}{nph}{Sino-Tibetan}
\define@key{fams}{pnx}{Austroasiatic}
\define@key{fams}{kjt}{Sino-Tibetan}
\define@key{fams}{xpg}{Indo-European}
\define@key{fams}{phu}{Tai-Kadai}
\define@key{fams}{phd}{Indo-European}
\define@key{fams}{pug}{Atlantic-Congo}
\define@key{fams}{phh}{Sino-Tibetan}
\define@key{fams}{ypm}{Sino-Tibetan}
\define@key{fams}{pho}{Sino-Tibetan}
\define@key{fams}{phg}{Austroasiatic}
\define@key{fams}{yph}{Sino-Tibetan}
\define@key{fams}{ypp}{Sino-Tibetan}
\define@key{fams}{pht}{Tai-Kadai}
\define@key{fams}{ypz}{Sino-Tibetan}
\define@key{fams}{ptr}{Austronesian}
\define@key{fams}{pin}{Sepik}
\define@key{fams}{pcd}{Indo-European}
\define@key{fams}{cpu}{Arawakan}
\define@key{fams}{xpi}{Unclassifiable}
\define@key{fams}{dep}{Pidgin}
\define@key{fams}{pij}{Unclassifiable}
\define@key{fams}{piz}{Austronesian}
\define@key{fams}{pis}{Indo-European}
\define@key{fams}{piw}{Atlantic-Congo}
\define@key{fams}{pnn}{Piawi}
\define@key{fams}{pnv}{Pama-Nyungan}
\define@key{fams}{tjp}{Pama-Nyungan}
\define@key{fams}{pic}{Atlantic-Congo}
\define@key{fams}{pti}{Pama-Nyungan}
\define@key{fams}{pny}{Atlantic-Congo}
\define@key{fams}{bxi}{Pama-Nyungan}
\define@key{fams}{pie}{Kiowa-Tanoan}
\define@key{fams}{xpa}{Pama-Nyungan}
\define@key{fams}{tpp}{Totonacan}
\define@key{fams}{pig}{Unattested}
\define@key{fams}{psy}{Algic}
\define@key{fams}{xps}{Indo-European}
\define@key{fams}{pih}{Indo-European}
\define@key{fams}{sje}{Uralic}
\define@key{fams}{pcn}{Atlantic-Congo}
\define@key{fams}{pix}{Austronesian}
\define@key{fams}{piy}{Afro-Asiatic}
\define@key{fams}{ktj}{Kru}
\define@key{fams}{pdt}{Indo-European}
\define@key{fams}{pbv}{Austroasiatic}
\define@key{fams}{npo}{Sino-Tibetan}
\define@key{fams}{pdn}{Austronesian}
\define@key{fams}{pof}{Atlantic-Congo}
\define@key{fams}{pkb}{Atlantic-Congo}
\define@key{fams}{pld}{Unclassifiable}
\define@key{fams}{plj}{Afro-Asiatic}
\define@key{fams}{pso}{Sign Language}
\define@key{fams}{plb}{Austronesian}
\define@key{fams}{pmo}{Austronesian}
\define@key{fams}{pmm}{Atlantic-Congo}
\define@key{fams}{ncc}{Austronesian}
\define@key{fams}{png}{Atlantic-Congo}
\define@key{fams}{pns}{Austronesian}
\define@key{fams}{pnt}{Indo-European}
\define@key{fams}{prh}{Austronesian}
\define@key{fams}{ptv}{Austronesian}
\define@key{fams}{pmx}{Sino-Tibetan}
\define@key{fams}{bye}{Sepik}
\define@key{fams}{pwr}{Indo-European}
\define@key{fams}{pyn}{Pano-Tacanan}
\define@key{fams}{prz}{Sign Language}
\define@key{fams}{prg}{Indo-European}
\define@key{fams}{kvj}{Afro-Asiatic}
\define@key{fams}{pux}{Sko}
\define@key{fams}{atp}{Austronesian}
\define@key{fams}{pbm}{Otomanguean}
\define@key{fams}{psl}{Sign Language}
\define@key{fams}{pkp}{Austronesian}
\define@key{fams}{pup}{Nuclear Trans New Guinea}
\define@key{fams}{pum}{Sino-Tibetan}
\define@key{fams}{xpm}{Yeniseian}
\define@key{fams}{puj}{Austronesian}
\define@key{fams}{pud}{Austronesian}
\define@key{fams}{puf}{Austronesian}
\define@key{fams}{pna}{Austronesian}
\define@key{fams}{pnm}{Austronesian}
\define@key{fams}{xpu}{Afro-Asiatic}
\define@key{fams}{qxp}{Quechuan}
\define@key{fams}{puu}{Atlantic-Congo}
\define@key{fams}{pru}{South Bird's Head Family}
\define@key{fams}{iar}{Isolate}
\define@key{fams}{puy}{Chumashan}
\define@key{fams}{prr}{Puri-Coroado}
\define@key{fams}{pur}{Tupian}
\define@key{fams}{pub}{Sino-Tibetan}
\define@key{fams}{mfl}{Afro-Asiatic}
\define@key{fams}{afe}{Atlantic-Congo}
\define@key{fams}{cpx}{Sino-Tibetan}
\define@key{fams}{pyu}{Austronesian}
\define@key{fams}{pme}{Austronesian}
\define@key{fams}{pop}{Austronesian}
\define@key{fams}{pwo}{Sino-Tibetan}
\define@key{fams}{pcw}{Afro-Asiatic}
\define@key{fams}{pye}{Kru}
\define@key{fams}{pyy}{Sino-Tibetan}
\define@key{fams}{pby}{Isolate}
\define@key{fams}{laq}{Tai-Kadai}
\define@key{fams}{qxq}{Turkic}
\define@key{fams}{xqt}{Afro-Asiatic}
\define@key{fams}{ymq}{Sino-Tibetan}
\define@key{fams}{zqe}{Tai-Kadai}
\define@key{fams}{qua}{Siouan}
\define@key{fams}{qya}{Artificial Language}
\define@key{fams}{qvy}{Sino-Tibetan}
\define@key{fams}{zpj}{Otomanguean}
\define@key{fams}{quq}{Unclassifiable}
\define@key{fams}{qun}{Salishan}
\define@key{fams}{ztq}{Otomanguean}
\define@key{fams}{rah}{Sino-Tibetan}
\define@key{fams}{xrr}{Unclassifiable}
\define@key{fams}{raz}{Austronesian}
\define@key{fams}{mqk}{Austronesian}
\define@key{fams}{rjs}{Indo-European}
\define@key{fams}{rjg}{Austronesian}
\define@key{fams}{gra}{Indo-European}
\define@key{fams}{rkh}{Austronesian}
\define@key{fams}{rki}{Sino-Tibetan}
\define@key{fams}{rai}{Austronesian}
\define@key{fams}{kjx}{North Bougainville}
\define@key{fams}{lje}{Austronesian}
\define@key{fams}{thr}{Indo-European}
\define@key{fams}{rkt}{Indo-European}
\define@key{fams}{rnl}{Sino-Tibetan}
\define@key{fams}{rax}{Atlantic-Congo}
\define@key{fams}{ray}{Austronesian}
\define@key{fams}{rpt}{Nuclear Trans New Guinea}
\define@key{fams}{lra}{Austronesian}
\define@key{fams}{rar}{Austronesian}
\define@key{fams}{rac}{Lakes Plain}
\define@key{fams}{btn}{Austronesian}
\define@key{fams}{bgd}{Indo-European}
\define@key{fams}{rtw}{Indo-European}
\define@key{fams}{rau}{Sino-Tibetan}
\define@key{fams}{yea}{Dravidian}
\define@key{fams}{jnl}{Sino-Tibetan}
\define@key{fams}{rat}{Indo-European}
\define@key{fams}{gir}{Tai-Kadai}
\define@key{fams}{atu}{Nilotic}
\define@key{fams}{ree}{Austronesian}
\define@key{fams}{rei}{Indo-European}
\define@key{fams}{bow}{Yam}
\define@key{fams}{reb}{Austronesian}
\define@key{fams}{agv}{Austronesian}
\define@key{fams}{rem}{Pano-Tacanan}
\define@key{fams}{rmp}{Nuclear Trans New Guinea}
\define@key{fams}{lkj}{Austronesian}
\define@key{fams}{rsi}{Artificial Language}
\define@key{fams}{rea}{Nuclear Trans New Guinea}
\define@key{fams}{rer}{Unattested}
\define@key{fams}{pgk}{Austronesian}
\define@key{fams}{res}{Atlantic-Congo}
\define@key{fams}{ret}{Timor-Alor-Pantar}
\define@key{fams}{rcf}{Indo-European}
\define@key{fams}{rey}{Pano-Tacanan}
\define@key{fams}{ril}{Austroasiatic}
\define@key{fams}{ria}{Sino-Tibetan}
\define@key{fams}{rir}{Austronesian}
\define@key{fams}{zar}{Otomanguean}
\define@key{fams}{rgu}{Austronesian}
\define@key{fams}{hrx}{Indo-European}
\define@key{fams}{rri}{Austronesian}
\define@key{fams}{riu}{Austronesian}
\define@key{fams}{snj}{Atlantic-Congo}
\define@key{fams}{rod}{Atlantic-Congo}
\define@key{fams}{rhg}{Indo-European}
\define@key{fams}{rge}{Indo-European}
\define@key{fams}{rms}{Sign Language}
\define@key{fams}{rgn}{Indo-European}
\define@key{fams}{rmx}{Austroasiatic}
\define@key{fams}{rmm}{Austronesian}
\define@key{fams}{rmv}{Artificial Language}
\define@key{fams}{rof}{Atlantic-Congo}
\define@key{fams}{rol}{Austronesian}
\define@key{fams}{rmk}{Lower Sepik-Ramu}
\define@key{fams}{ror}{Austronesian}
\define@key{fams}{roe}{Austronesian}
\define@key{fams}{rnn}{Austronesian}
\define@key{fams}{rga}{Austronesian}
\define@key{fams}{pce}{Austroasiatic}
\define@key{fams}{rdb}{Indo-European}
\define@key{fams}{ruh}{Sino-Tibetan}
\define@key{fams}{rbb}{Austroasiatic}
\define@key{fams}{ruz}{Unattested}
\define@key{fams}{rna}{Unattested}
\define@key{fams}{rnw}{Atlantic-Congo}
\define@key{fams}{drg}{Austronesian}
\define@key{fams}{bxr}{Mongolic-Khitan}
\define@key{fams}{rue}{Indo-European}
\define@key{fams}{ruc}{Atlantic-Congo}
\define@key{fams}{rnd}{Atlantic-Congo}
\define@key{fams}{rwk}{Atlantic-Congo}
\define@key{fams}{rsn}{Sign Language}
\define@key{fams}{sax}{Austronesian}
\define@key{fams}{sav}{Atlantic-Congo}
\define@key{fams}{raq}{Sino-Tibetan}
\define@key{fams}{lsm}{Atlantic-Congo}
\define@key{fams}{sxr}{Austronesian}
\define@key{fams}{spy}{Nilotic}
\define@key{fams}{msi}{Austronesian}
\define@key{fams}{bsy}{Austronesian}
\define@key{fams}{sae}{Nambiquaran}
\define@key{fams}{saa}{Afro-Asiatic}
\define@key{fams}{xsa}{Afro-Asiatic}
\define@key{fams}{qhr}{Indo-European}
\define@key{fams}{sbo}{Austroasiatic}
\define@key{fams}{quv}{Mayan}
\define@key{fams}{sck}{Indo-European}
\define@key{fams}{spd}{Nuclear Trans New Guinea}
\define@key{fams}{saf}{Atlantic-Congo}
\define@key{fams}{sbk}{Atlantic-Congo}
\define@key{fams}{sbm}{Atlantic-Congo}
\define@key{fams}{tga}{Atlantic-Congo}
\define@key{fams}{aec}{Afro-Asiatic}
\define@key{fams}{acf}{Indo-European}
\define@key{fams}{xsy}{Austronesian}
\define@key{fams}{sjl}{Sino-Tibetan}
\define@key{fams}{sjb}{Austronesian}
\define@key{fams}{sch}{Sino-Tibetan}
\define@key{fams}{skt}{Atlantic-Congo}
\define@key{fams}{skg}{Austronesian}
\define@key{fams}{skm}{Nuclear Trans New Guinea}
\define@key{fams}{sak}{Atlantic-Congo}
\define@key{fams}{szy}{Austronesian}
\define@key{fams}{shq}{Atlantic-Congo}
\define@key{fams}{slx}{Atlantic-Congo}
\define@key{fams}{sgu}{Austronesian}
\define@key{fams}{qxl}{Quechuan}
\define@key{fams}{mnd}{Tupian}
\define@key{fams}{slq}{Turkic}
\define@key{fams}{sau}{Austronesian}
\define@key{fams}{loe}{Austronesian}
\define@key{fams}{esn}{Sign Language}
\define@key{fams}{tmj}{Greater Kwerba}
\define@key{fams}{ysd}{Sino-Tibetan}
\define@key{fams}{smp}{Afro-Asiatic}
\define@key{fams}{xab}{Atlantic-Congo}
\define@key{fams}{smx}{Atlantic-Congo}
\define@key{fams}{ccg}{Atlantic-Congo}
\define@key{fams}{saq}{Nilotic}
\define@key{fams}{ssx}{Nuclear Trans New Guinea}
\define@key{fams}{spv}{Indo-European}
\define@key{fams}{smh}{Sino-Tibetan}
\define@key{fams}{snx}{Nuclear Trans New Guinea}
\define@key{fams}{swm}{Nuclear Trans New Guinea}
\define@key{fams}{rav}{Sino-Tibetan}
\define@key{fams}{stu}{Austroasiatic}
\define@key{fams}{smv}{Indo-European}
\define@key{fams}{ztm}{Otomanguean}
\define@key{fams}{icr}{Indo-European}
\define@key{fams}{spn}{Lengua-Mascoy}
\define@key{fams}{zpx}{Otomanguean}
\define@key{fams}{cuk}{Chibchan}
\define@key{fams}{hve}{Huavean}
\define@key{fams}{hue}{Huavean}
\define@key{fams}{mat}{Otomanguean}
\define@key{fams}{pow}{Otomanguean}
\define@key{fams}{xso}{Unclassifiable}
\define@key{fams}{sgr}{Indo-European}
\define@key{fams}{sgk}{Sino-Tibetan}
\define@key{fams}{nsa}{Sino-Tibetan}
\define@key{fams}{xsn}{Atlantic-Congo}
\define@key{fams}{sbp}{Atlantic-Congo}
\define@key{fams}{sng}{Atlantic-Congo}
\define@key{fams}{snl}{Austronesian}
\define@key{fams}{scg}{Austronesian}
\define@key{fams}{sgy}{Indo-European}
\define@key{fams}{ysy}{Sino-Tibetan}
\define@key{fams}{ysn}{Sino-Tibetan}
\define@key{fams}{sny}{Sepik}
\define@key{fams}{xtj}{Otomanguean}
\define@key{fams}{maa}{Otomanguean}
\define@key{fams}{msc}{Mande}
\define@key{fams}{pps}{Otomanguean}
\define@key{fams}{qvs}{Quechuan}
\define@key{fams}{xtp}{Otomanguean}
\define@key{fams}{trq}{Otomanguean}
\define@key{fams}{pls}{Otomanguean}
\define@key{fams}{azg}{Otomanguean}
\define@key{fams}{zpf}{Otomanguean}
\define@key{fams}{san}{Indo-European}
\define@key{fams}{ssi}{Indo-European}
\define@key{fams}{kwy}{Atlantic-Congo}
\define@key{fams}{hvv}{Huavean}
\define@key{fams}{nhz}{Uto-Aztecan}
\define@key{fams}{cok}{Uto-Aztecan}
\define@key{fams}{qus}{Quechuan}
\define@key{fams}{mza}{Otomanguean}
\define@key{fams}{mdv}{Otomanguean}
\define@key{fams}{zpn}{Otomanguean}
\define@key{fams}{ztn}{Otomanguean}
\define@key{fams}{zas}{Otomanguean}
\define@key{fams}{zpr}{Otomanguean}
\define@key{fams}{pca}{Otomanguean}
\define@key{fams}{zpt}{Otomanguean}
\define@key{fams}{scq}{Austroasiatic}
\define@key{fams}{zkp}{Nuclear-Macro-Je}
\define@key{fams}{cri}{Indo-European}
\define@key{fams}{spr}{Austronesian}
\define@key{fams}{spc}{Isolate}
\define@key{fams}{krn}{Kru}
\define@key{fams}{spi}{Lakes Plain}
\define@key{fams}{sbz}{Central Sudanic}
\define@key{fams}{kwv}{Central Sudanic}
\define@key{fams}{kwg}{Central Sudanic}
\define@key{fams}{zsa}{Austronesian}
\define@key{fams}{bps}{Austronesian}
\define@key{fams}{mbs}{Austronesian}
\define@key{fams}{sre}{Austronesian}
\define@key{fams}{sar}{Arawakan}
\define@key{fams}{srh}{Indo-European}
\define@key{fams}{mwm}{Central Sudanic}
\define@key{fams}{onp}{Sino-Tibetan}
\define@key{fams}{sdu}{Austronesian}
\define@key{fams}{sra}{Nuclear Trans New Guinea}
\define@key{fams}{swy}{Afro-Asiatic}
\define@key{fams}{sxs}{Atlantic-Congo}
\define@key{fams}{sas}{Austronesian}
\define@key{fams}{sdc}{Indo-European}
\define@key{fams}{stw}{Austronesian}
\define@key{fams}{stq}{Indo-European}
\define@key{fams}{mav}{Tupian}
\define@key{fams}{sdl}{Sign Language}
\define@key{fams}{skc}{Nuclear Trans New Guinea}
\define@key{fams}{saz}{Indo-European}
\define@key{fams}{mjt}{Dravidian}
\define@key{fams}{srt}{Geelvink Bay}
\define@key{fams}{psu}{Indo-European}
\define@key{fams}{ssj}{Nuclear Trans New Guinea}
\define@key{fams}{sao}{Isolate}
\define@key{fams}{swr}{Yawa-Saweru}
\define@key{fams}{swt}{Timor-Alor-Pantar}
\define@key{fams}{saw}{Nuclear Trans New Guinea}
\define@key{fams}{swn}{Afro-Asiatic}
\define@key{fams}{sxw}{Atlantic-Congo}
\define@key{fams}{say}{Afro-Asiatic}
\define@key{fams}{sco}{Indo-European}
\define@key{fams}{kdg}{Atlantic-Congo}
\define@key{fams}{sbx}{Austronesian}
\define@key{fams}{sib}{Austronesian}
\define@key{fams}{sec}{Salishan}
\define@key{fams}{tvw}{Austronesian}
\define@key{fams}{sos}{Mande}
\define@key{fams}{sge}{Austronesian}
\define@key{fams}{sbg}{West Bird's Head}
\define@key{fams}{seg}{Atlantic-Congo}
\define@key{fams}{sfw}{Atlantic-Congo}
\define@key{fams}{ssg}{Austronesian}
\define@key{fams}{hik}{Austronesian}
\define@key{fams}{skz}{Austronesian}
\define@key{fams}{skp}{Austronesian}
\define@key{fams}{sek}{Athabaskan-Eyak-Tlingit}
\define@key{fams}{ske}{Austronesian}
\define@key{fams}{syi}{Atlantic-Congo}
\define@key{fams}{sko}{Austronesian}
\define@key{fams}{skx}{Austronesian}
\define@key{fams}{lip}{Atlantic-Congo}
\define@key{fams}{kgi}{Sign Language}
\define@key{fams}{snw}{Atlantic-Congo}
\define@key{fams}{sws}{Austronesian}
\define@key{fams}{slg}{Austronesian}
\define@key{fams}{szc}{Austroasiatic}
\define@key{fams}{sbr}{Austronesian}
\define@key{fams}{etz}{Mairasic}
\define@key{fams}{smy}{Indo-European}
\define@key{fams}{ssm}{Austroasiatic}
\define@key{fams}{xse}{Nuclear Trans New Guinea}
\define@key{fams}{seq}{Atlantic-Congo}
\define@key{fams}{sej}{Nuclear Trans New Guinea}
\define@key{fams}{sds}{Afro-Asiatic}
\define@key{fams}{ssz}{Austronesian}
\define@key{fams}{spk}{Ndu}
\define@key{fams}{snu}{Border}
\define@key{fams}{sjs}{Afro-Asiatic}
\define@key{fams}{sni}{Pano-Tacanan}
\define@key{fams}{std}{Unattested}
\define@key{fams}{sez}{Sino-Tibetan}
\define@key{fams}{spe}{Austronesian}
\define@key{fams}{spb}{Austronesian}
\define@key{fams}{spm}{Lower Sepik-Ramu}
\define@key{fams}{iws}{Sepik}
\define@key{fams}{skr}{Indo-European}
\define@key{fams}{sry}{Austronesian}
\define@key{fams}{srr}{Atlantic-Congo}
\define@key{fams}{swf}{Atlantic-Congo}
\define@key{fams}{sve}{Austronesian}
\define@key{fams}{seu}{Austronesian}
\define@key{fams}{srw}{Austronesian}
\define@key{fams}{srk}{Austronesian}
\define@key{fams}{stf}{Nuclear Torricelli}
\define@key{fams}{stm}{Nuclear Trans New Guinea}
\define@key{fams}{sbi}{Nuclear Torricelli}
\define@key{fams}{sta}{Pidgin}
\define@key{fams}{sew}{Austronesian}
\define@key{fams}{lsw}{Sign Language}
\define@key{fams}{sze}{Blue Nile Mao}
\define@key{fams}{scw}{Afro-Asiatic}
\define@key{fams}{sdb}{Indo-European}
\define@key{fams}{srz}{Indo-European}
\define@key{fams}{sha}{Atlantic-Congo}
\define@key{fams}{xsh}{Atlantic-Congo}
\define@key{fams}{sqa}{Atlantic-Congo}
\define@key{fams}{jih}{Sino-Tibetan}
\define@key{fams}{sho}{Mande}
\define@key{fams}{swo}{Pano-Tacanan}
\define@key{fams}{ssv}{Austronesian}
\define@key{fams}{swq}{Afro-Asiatic}
\define@key{fams}{sqh}{Atlantic-Congo}
\define@key{fams}{shx}{Hmong-Mien}
\define@key{fams}{she}{Dizoid}
\define@key{fams}{sth}{Speech Register}
\define@key{fams}{shl}{Sino-Tibetan}
\define@key{fams}{scv}{Atlantic-Congo}
\define@key{fams}{bun}{Atlantic-Congo}
\define@key{fams}{kip}{Sino-Tibetan}
\define@key{fams}{ssh}{Afro-Asiatic}
\define@key{fams}{shr}{Atlantic-Congo}
\define@key{fams}{gua}{Atlantic-Congo}
\define@key{fams}{snh}{Unattested}
\define@key{fams}{sxg}{Sino-Tibetan}
\define@key{fams}{sle}{Dravidian}
\define@key{fams}{bcv}{Atlantic-Congo}
\define@key{fams}{suj}{Atlantic-Congo}
\define@key{fams}{sts}{Indo-European}
\define@key{fams}{scu}{Sino-Tibetan}
\define@key{fams}{ksa}{Unattested}
\define@key{fams}{shw}{Heibanic}
\define@key{fams}{slw}{Nuclear Trans New Guinea}
\define@key{fams}{sya}{Austronesian}
\define@key{fams}{spg}{Austronesian}
\define@key{fams}{mmp}{Amto-Musan}
\define@key{fams}{nco}{South Bougainville}
\define@key{fams}{sty}{Turkic}
\define@key{fams}{sdx}{Austronesian}
\define@key{fams}{sxc}{Unclassifiable}
\define@key{fams}{scn}{Indo-European}
\define@key{fams}{sep}{Atlantic-Congo}
\define@key{fams}{scx}{Unclassifiable}
\define@key{fams}{xsd}{Indo-European}
\define@key{fams}{sgx}{Sign Language}
\define@key{fams}{nsu}{Uto-Aztecan}
\define@key{fams}{sxe}{Atlantic-Congo}
\define@key{fams}{snr}{Nuclear Trans New Guinea}
\define@key{fams}{qws}{Quechuan}
\define@key{fams}{sky}{Austronesian}
\define@key{fams}{slt}{Sino-Tibetan}
\define@key{fams}{szl}{Indo-European}
\define@key{fams}{sbq}{Nuclear Trans New Guinea}
\define@key{fams}{mkc}{Nuclear Torricelli}
\define@key{fams}{wul}{Nuclear Trans New Guinea}
\define@key{fams}{xsp}{Nuclear Trans New Guinea}
\define@key{fams}{stv}{Afro-Asiatic}
\define@key{fams}{sie}{Atlantic-Congo}
\define@key{fams}{sbw}{Atlantic-Congo}
\define@key{fams}{smb}{Angan}
\define@key{fams}{sbb}{Austronesian}
\define@key{fams}{smg}{Baining}
\define@key{fams}{smz}{South Bougainville}
\define@key{fams}{smt}{Sino-Tibetan}
\define@key{fams}{siu}{Nuclear Torricelli}
\define@key{fams}{sbn}{Indo-European}
\define@key{fams}{xts}{Otomanguean}
\define@key{fams}{sjn}{Artificial Language}
\define@key{fams}{sgp}{Sino-Tibetan}
\define@key{fams}{sgm}{Atlantic-Congo}
\define@key{fams}{skq}{Mande}
\define@key{fams}{xti}{Otomanguean}
\define@key{fams}{snz}{Nuclear Trans New Guinea}
\define@key{fams}{sys}{Central Sudanic}
\define@key{fams}{swj}{Atlantic-Congo}
\define@key{fams}{sir}{Afro-Asiatic}
\define@key{fams}{srx}{Indo-European}
\define@key{fams}{sld}{Atlantic-Congo}
\define@key{fams}{sso}{Austronesian}
\define@key{fams}{siy}{Indo-European}
\define@key{fams}{lsv}{Sign Language}
\define@key{fams}{akp}{Atlantic-Congo}
\define@key{fams}{skw}{Indo-European}
\define@key{fams}{sms}{Uralic}
\define@key{fams}{svm}{Indo-European}
\define@key{fams}{svk}{Sign Language}
\define@key{fams}{sfm}{Hmong-Mien}
\define@key{fams}{kxq}{Yam}
\define@key{fams}{sox}{Atlantic-Congo}
\define@key{fams}{soc}{Atlantic-Congo}
\define@key{fams}{xog}{Atlantic-Congo}
\define@key{fams}{sog}{Indo-European}
\define@key{fams}{soj}{Indo-European}
\define@key{fams}{sok}{Afro-Asiatic}
\define@key{fams}{sby}{Atlantic-Congo}
\define@key{fams}{sol}{Austronesian}
\define@key{fams}{aaw}{Austronesian}
\define@key{fams}{szs}{Sign Language}
\define@key{fams}{smc}{Nuclear Trans New Guinea}
\define@key{fams}{smu}{Austroasiatic}
\define@key{fams}{sor}{Afro-Asiatic}
\define@key{fams}{kgt}{Atlantic-Congo}
\define@key{fams}{ysg}{Sino-Tibetan}
\define@key{fams}{shc}{Atlantic-Congo}
\define@key{fams}{soo}{Atlantic-Congo}
\define@key{fams}{sod}{Atlantic-Congo}
\define@key{fams}{soe}{Atlantic-Congo}
\define@key{fams}{soi}{Indo-European}
\define@key{fams}{siq}{Bosavi}
\define@key{fams}{sss}{Austroasiatic}
\define@key{fams}{urw}{Nuclear Trans New Guinea}
\define@key{fams}{sbh}{Austronesian}
\define@key{fams}{sqo}{Indo-European}
\define@key{fams}{ays}{Unattested}
\define@key{fams}{sdk}{Ndu}
\define@key{fams}{krz}{Yam}
\define@key{fams}{sfs}{Sign Language}
\define@key{fams}{nit}{Dravidian}
\define@key{fams}{hmy}{Hmong-Mien}
\define@key{fams}{hma}{Hmong-Mien}
\define@key{fams}{sdh}{Indo-European}
\define@key{fams}{bcc}{Indo-European}
\define@key{fams}{fay}{Indo-European}
\define@key{fams}{luz}{Indo-European}
\define@key{fams}{pbt}{Indo-European}
\define@key{fams}{hnd}{Indo-European}
\define@key{fams}{psh}{Indo-European}
\define@key{fams}{psi}{Indo-European}
\define@key{fams}{vro}{Uralic}
\define@key{fams}{nik}{Austroasiatic}
\define@key{fams}{mnn}{Austroasiatic}
\define@key{fams}{uzs}{Turkic}
\define@key{fams}{ghe}{Sino-Tibetan}
\define@key{fams}{ymc}{Sino-Tibetan}
\define@key{fams}{nsd}{Sino-Tibetan}
\define@key{fams}{qxs}{Sino-Tibetan}
\define@key{fams}{pmj}{Sino-Tibetan}
\define@key{fams}{bfs}{Sino-Tibetan}
\define@key{fams}{nre}{Sino-Tibetan}
\define@key{fams}{lrr}{Sino-Tibetan}
\define@key{fams}{tjs}{Sino-Tibetan}
\define@key{fams}{sou}{Tai-Kadai}
\define@key{fams}{hms}{Hmong-Mien}
\define@key{fams}{hmh}{Hmong-Mien}
\define@key{fams}{hmg}{Hmong-Mien}
\define@key{fams}{xtv}{Pama-Nyungan}
\define@key{fams}{ijs}{Ijoid}
\define@key{fams}{fal}{Atlantic-Congo}
\define@key{fams}{nbw}{Atlantic-Congo}
\define@key{fams}{lnl}{Atlantic-Congo}
\define@key{fams}{biv}{Atlantic-Congo}
\define@key{fams}{nnw}{Atlantic-Congo}
\define@key{fams}{snm}{Central Sudanic}
\define@key{fams}{dik}{Nilotic}
\define@key{fams}{dib}{Nilotic}
\define@key{fams}{dks}{Nilotic}
\define@key{fams}{bwq}{Mande}
\define@key{fams}{sbd}{Mande}
\define@key{fams}{sns}{Austronesian}
\define@key{fams}{mqm}{Austronesian}
\define@key{fams}{mcy}{Austronesian}
\define@key{fams}{vbb}{Austronesian}
\define@key{fams}{lmf}{Austronesian}
\define@key{fams}{agy}{Austronesian}
\define@key{fams}{ksc}{Austronesian}
\define@key{fams}{bln}{Austronesian}
\define@key{fams}{plv}{Austronesian}
\define@key{fams}{bzc}{Austronesian}
\define@key{fams}{osu}{Nuclear Torricelli}
\define@key{fams}{aws}{Nuclear Trans New Guinea}
\define@key{fams}{omw}{Nuclear Trans New Guinea}
\define@key{fams}{ams}{Japonic}
\define@key{fams}{hax}{Haida}
\define@key{fams}{tce}{Athabaskan-Eyak-Tlingit}
\define@key{fams}{caf}{Athabaskan-Eyak-Tlingit}
\define@key{fams}{twr}{Uto-Aztecan}
\define@key{fams}{tcu}{Uto-Aztecan}
\define@key{fams}{npl}{Uto-Aztecan}
\define@key{fams}{tla}{Uto-Aztecan}
\define@key{fams}{crj}{Algic}
\define@key{fams}{peq}{Pomoan}
\define@key{fams}{qup}{Quechuan}
\define@key{fams}{qxo}{Quechuan}
\define@key{fams}{ayc}{Aymaran}
\define@key{fams}{meh}{Otomanguean}
\define@key{fams}{mit}{Otomanguean}
\define@key{fams}{mxy}{Otomanguean}
\define@key{fams}{rgs}{Austronesian}
\define@key{fams}{giz}{Afro-Asiatic}
\define@key{fams}{cpy}{Arawakan}
\define@key{fams}{itd}{Austronesian}
\define@key{fams}{csp}{Sino-Tibetan}
\define@key{fams}{sct}{Austroasiatic}
\define@key{fams}{sqq}{Austroasiatic}
\define@key{fams}{sww}{Austronesian}
\define@key{fams}{sow}{Border}
\define@key{fams}{vmq}{Otomanguean}
\define@key{fams}{vmp}{Otomanguean}
\define@key{fams}{sqs}{Sign Language}
\define@key{fams}{sci}{Austronesian}
\define@key{fams}{seo}{Isolate}
\define@key{fams}{swp}{Austronesian}
\define@key{fams}{sxb}{Atlantic-Congo}
\define@key{fams}{ssc}{Atlantic-Congo}
\define@key{fams}{sut}{Otomanguean}
\define@key{fams}{apd}{Afro-Asiatic}
\define@key{fams}{pga}{Afro-Asiatic}
\define@key{fams}{sgi}{Atlantic-Congo}
\define@key{fams}{sug}{Nuclear Trans New Guinea}
\define@key{fams}{kzs}{Austronesian}
\define@key{fams}{zsu}{Austronesian}
\define@key{fams}{syk}{Afro-Asiatic}
\define@key{fams}{szn}{Austronesian}
\define@key{fams}{srg}{Austronesian}
\define@key{fams}{sqm}{Atlantic-Congo}
\define@key{fams}{siv}{Sepik}
\define@key{fams}{six}{Nuclear Trans New Guinea}
\define@key{fams}{suw}{Atlantic-Congo}
\define@key{fams}{smw}{Austronesian}
\define@key{fams}{sux}{Isolate}
\define@key{fams}{csv}{Sino-Tibetan}
\define@key{fams}{ssk}{Sino-Tibetan}
\define@key{fams}{suz}{Sino-Tibetan}
\define@key{fams}{syo}{Austroasiatic}
\define@key{fams}{sbj}{Maban}
\define@key{fams}{sgd}{Austronesian}
\define@key{fams}{sjp}{Indo-European}
\define@key{fams}{tdl}{Atlantic-Congo}
\define@key{fams}{sde}{Atlantic-Congo}
\define@key{fams}{mdz}{Tupian}
\define@key{fams}{sru}{Tupian}
\define@key{fams}{swx}{Arawan}
\define@key{fams}{sqn}{Iroquoian}
\define@key{fams}{ssu}{Angan}
\define@key{fams}{sdj}{Atlantic-Congo}
\define@key{fams}{swu}{Austronesian}
\define@key{fams}{suy}{Nuclear-Macro-Je}
\define@key{fams}{swg}{Indo-European}
\define@key{fams}{slf}{Sign Language}
\define@key{fams}{sgg}{Sign Language}
\define@key{fams}{ssr}{Sign Language}
\define@key{fams}{xdk}{Pama-Nyungan}
\define@key{fams}{syl}{Indo-European}
\define@key{fams}{zoq}{Mixe-Zoque}
\define@key{fams}{nhc}{Uto-Aztecan}
\define@key{fams}{zat}{Otomanguean}
\define@key{fams}{knv}{Isolate}
\define@key{fams}{tzx}{Lower Sepik-Ramu}
\define@key{fams}{xtt}{Otomanguean}
\define@key{fams}{lts}{Atlantic-Congo}
\define@key{fams}{dsq}{Songhay}
\define@key{fams}{tdy}{Austronesian}
\define@key{fams}{rob}{Austronesian}
\define@key{fams}{tcd}{Atlantic-Congo}
\define@key{fams}{klg}{Austronesian}
\define@key{fams}{bgs}{Austronesian}
\define@key{fams}{mvv}{Austronesian}
\define@key{fams}{tgz}{Pama-Nyungan}
\define@key{fams}{tbm}{Atlantic-Congo}
\define@key{fams}{tda}{Songhay}
\define@key{fams}{tgx}{Athabaskan-Eyak-Tlingit}
\define@key{fams}{tgj}{Sino-Tibetan}
\define@key{fams}{tgw}{Atlantic-Congo}
\define@key{fams}{tht}{Athabaskan-Eyak-Tlingit}
\define@key{fams}{blt}{Tai-Kadai}
\define@key{fams}{tyj}{Tai-Kadai}
\define@key{fams}{tyr}{Tai-Kadai}
\define@key{fams}{twh}{Tai-Kadai}
\define@key{fams}{tiz}{Tai-Kadai}
\define@key{fams}{taw}{Nuclear Trans New Guinea}
\define@key{fams}{aos}{Border}
\define@key{fams}{tlq}{Austroasiatic}
\define@key{fams}{thi}{Tai-Kadai}
\define@key{fams}{tjl}{Tai-Kadai}
\define@key{fams}{tdd}{Tai-Kadai}
\define@key{fams}{ago}{Angan}
\define@key{fams}{tnq}{Arawakan}
\define@key{fams}{tpo}{Tai-Kadai}
\define@key{fams}{uar}{Eleman}
\define@key{fams}{tmm}{Tai-Kadai}
\define@key{fams}{cuu}{Tai-Kadai}
\define@key{fams}{acq}{Afro-Asiatic}
\define@key{fams}{pee}{Austronesian}
\define@key{fams}{tdj}{Austronesian}
\define@key{fams}{abh}{Afro-Asiatic}
\define@key{fams}{tja}{Kru}
\define@key{fams}{tkz}{Austroasiatic}
\define@key{fams}{nho}{Austronesian}
\define@key{fams}{tke}{Atlantic-Congo}
\define@key{fams}{tak}{Afro-Asiatic}
\define@key{fams}{tdf}{Austroasiatic}
\define@key{fams}{tlr}{Austronesian}
\define@key{fams}{tlv}{Austronesian}
\define@key{fams}{tal}{Afro-Asiatic}
\define@key{fams}{tln}{Austronesian}
\define@key{fams}{tlk}{Austronesian}
\define@key{fams}{tzl}{Artificial Language}
\define@key{fams}{yta}{Sino-Tibetan}
\define@key{fams}{tcl}{Sino-Tibetan}
\define@key{fams}{tmn}{Austronesian}
\define@key{fams}{tmz}{Cariban}
\define@key{fams}{vmx}{Otomanguean}
\define@key{fams}{ten}{Tucanoan}
\define@key{fams}{tls}{Austronesian}
\define@key{fams}{xxt}{Isolate}
\define@key{fams}{tdk}{Afro-Asiatic}
\define@key{fams}{tmy}{Austronesian}
\define@key{fams}{tax}{Afro-Asiatic}
\define@key{fams}{tml}{Nuclear Trans New Guinea}
\define@key{fams}{tpu}{Austroasiatic}
\define@key{fams}{low}{Austronesian}
\define@key{fams}{tpv}{Austronesian}
\define@key{fams}{tcm}{Isolate}
\define@key{fams}{tni}{Austronesian}
\define@key{fams}{tdx}{Austronesian}
\define@key{fams}{tgn}{Austronesian}
\define@key{fams}{tnx}{Austronesian}
\define@key{fams}{tnv}{Indo-European}
\define@key{fams}{txg}{Sino-Tibetan}
\define@key{fams}{tgp}{Austronesian}
\define@key{fams}{tkx}{Nuclear Trans New Guinea}
\define@key{fams}{tgu}{Lower Sepik-Ramu}
\define@key{fams}{tbs}{Lower Sepik-Ramu}
\define@key{fams}{ytl}{Sino-Tibetan}
\define@key{fams}{tbe}{Austronesian}
\define@key{fams}{uji}{Atlantic-Congo}
\define@key{fams}{txy}{Austronesian}
\define@key{fams}{xnj}{Atlantic-Congo}
\define@key{fams}{qcs}{Mixe-Zoque}
\define@key{fams}{afp}{Arafundi}
\define@key{fams}{taf}{Tupian}
\define@key{fams}{txj}{Saharan}
\define@key{fams}{tpf}{Austronesian}
\define@key{fams}{txr}{Unclassifiable}
\define@key{fams}{tdm}{Isolate}
\define@key{fams}{twq}{Songhay}
\define@key{fams}{tmt}{Austronesian}
\define@key{fams}{ttd}{Goilalan}
\define@key{fams}{tco}{Sino-Tibetan}
\define@key{fams}{tpa}{Austronesian}
\define@key{fams}{tad}{Lakes Plain}
\define@key{fams}{tvs}{Atlantic-Congo}
\define@key{fams}{tvn}{Sino-Tibetan}
\define@key{fams}{rmu}{Speech Register}
\define@key{fams}{twl}{Atlantic-Congo}
\define@key{fams}{xtw}{Nambiquaran}
\define@key{fams}{ttq}{Afro-Asiatic}
\define@key{fams}{twy}{Austronesian}
\define@key{fams}{tbp}{Lakes Plain}
\define@key{fams}{tcp}{Sino-Tibetan}
\define@key{fams}{ayy}{Unattested}
\define@key{fams}{tas}{Pidgin}
\define@key{fams}{tnu}{Tai-Kadai}
\define@key{fams}{tys}{Tai-Kadai}
\define@key{fams}{tyt}{Tai-Kadai}
\define@key{fams}{tyz}{Tai-Kadai}
\define@key{fams}{tck}{Atlantic-Congo}
\define@key{fams}{bqa}{Atlantic-Congo}
\define@key{fams}{dtu}{Dogon}
\define@key{fams}{tsy}{Sign Language}
\define@key{fams}{tcw}{Totonacan}
\define@key{fams}{tuq}{Saharan}
\define@key{fams}{tkq}{Atlantic-Congo}
\define@key{fams}{lor}{Atlantic-Congo}
\define@key{fams}{tfo}{Geelvink Bay}
\define@key{fams}{twe}{Timor-Alor-Pantar}
\define@key{fams}{ztt}{Otomanguean}
\define@key{fams}{teg}{Atlantic-Congo}
\define@key{fams}{tyx}{Atlantic-Congo}
\define@key{fams}{lli}{Atlantic-Congo}
\define@key{fams}{ebo}{Atlantic-Congo}
\define@key{fams}{tyi}{Atlantic-Congo}
\define@key{fams}{tvm}{Austronesian}
\define@key{fams}{tlt}{Austronesian}
\define@key{fams}{nhv}{Uto-Aztecan}
\define@key{fams}{tjo}{Afro-Asiatic}
\define@key{fams}{tbt}{Atlantic-Congo}
\define@key{fams}{tmv}{Atlantic-Congo}
\define@key{fams}{tqb}{Tupian}
\define@key{fams}{tdo}{Atlantic-Congo}
\define@key{fams}{soz}{Atlantic-Congo}
\define@key{fams}{tmo}{Austroasiatic}
\define@key{fams}{ott}{Otomanguean}
\define@key{fams}{tmw}{Austronesian}
\define@key{fams}{quw}{Quechuan}
\define@key{fams}{otn}{Otomanguean}
\define@key{fams}{dtk}{Dogon}
\define@key{fams}{tes}{Austronesian}
\define@key{fams}{pah}{Tupian}
\define@key{fams}{tqn}{Sahaptian}
\define@key{fams}{tns}{Austronesian}
\define@key{fams}{tct}{Tai-Kadai}
\define@key{fams}{tev}{Austronesian}
\define@key{fams}{cux}{Otomanguean}
\define@key{fams}{cte}{Otomanguean}
\define@key{fams}{ted}{Kru}
\define@key{fams}{tef}{Austroasiatic}
\define@key{fams}{trb}{Austronesian}
\define@key{fams}{twg}{Timor-Alor-Pantar}
\define@key{fams}{tec}{Nilotic}
\define@key{fams}{tmg}{Indo-European}
\define@key{fams}{sjt}{Uralic}
\define@key{fams}{tkg}{Austronesian}
\define@key{fams}{keg}{Temeinic}
\define@key{fams}{twc}{Afro-Asiatic}
\define@key{fams}{tez}{Afro-Asiatic}
\define@key{fams}{tdt}{Austronesian}
\define@key{fams}{tve}{Austronesian}
\define@key{fams}{cut}{Otomanguean}
\define@key{fams}{twx}{Atlantic-Congo}
\define@key{fams}{otx}{Otomanguean}
\define@key{fams}{poq}{Mixe-Zoque}
\define@key{fams}{mxb}{Otomanguean}
\define@key{fams}{thy}{Atlantic-Congo}
\define@key{fams}{thn}{Dravidian}
\define@key{fams}{soa}{Tai-Kadai}
\define@key{fams}{nki}{Sino-Tibetan}
\define@key{fams}{thk}{Atlantic-Congo}
\define@key{fams}{iin}{Pama-Nyungan}
\define@key{fams}{tou}{Austroasiatic}
\define@key{fams}{ytp}{Sino-Tibetan}
\define@key{fams}{txh}{Indo-European}
\define@key{fams}{thu}{Nilotic}
\define@key{fams}{ahi}{Kru}
\define@key{fams}{mnl}{Austronesian}
\define@key{fams}{tbj}{Austronesian}
\define@key{fams}{ngy}{Atlantic-Congo}
\define@key{fams}{lsn}{Sign Language}
\define@key{fams}{tcn}{Sino-Tibetan}
\define@key{fams}{mtx}{Otomanguean}
\define@key{fams}{tia}{Afro-Asiatic}
\define@key{fams}{tiq}{Atlantic-Congo}
\define@key{fams}{boo}{Mande}
\define@key{fams}{tii}{Atlantic-Congo}
\define@key{fams}{nza}{Atlantic-Congo}
\define@key{fams}{txq}{Austronesian}
\define@key{fams}{xtl}{Otomanguean}
\define@key{fams}{tkp}{Austronesian}
\define@key{fams}{otl}{Otomanguean}
\define@key{fams}{zts}{Otomanguean}
\define@key{fams}{tij}{Sino-Tibetan}
\define@key{fams}{tim}{Nuclear Trans New Guinea}
\define@key{fams}{tvy}{Indo-European}
\define@key{fams}{xsb}{Austronesian}
\define@key{fams}{tit}{Isolate}
\define@key{fams}{tpz}{Austronesian}
\define@key{fams}{tpe}{Sino-Tibetan}
\define@key{fams}{tra}{Indo-European}
\define@key{fams}{tic}{Heibanic}
\define@key{fams}{tde}{Dogon}
\define@key{fams}{tdq}{Atlantic-Congo}
\define@key{fams}{ttv}{Austronesian}
\define@key{fams}{lax}{Sino-Tibetan}
\define@key{fams}{tju}{Pama-Nyungan}
\define@key{fams}{tpl}{Otomanguean}
\define@key{fams}{ctl}{Otomanguean}
\define@key{fams}{zpk}{Otomanguean}
\define@key{fams}{nuz}{Uto-Aztecan}
\define@key{fams}{mqh}{Otomanguean}
\define@key{fams}{tmf}{Lengua-Mascoy}
\define@key{fams}{tng}{Afro-Asiatic}
\define@key{fams}{tgh}{Indo-European}
\define@key{fams}{tox}{Austronesian}
\define@key{fams}{tgb}{Austronesian}
\define@key{fams}{taz}{Narrow Talodi}
\define@key{fams}{tdr}{Austroasiatic}
\define@key{fams}{tlg}{Namla-Tofanma}
\define@key{fams}{tfi}{Atlantic-Congo}
\define@key{fams}{tor}{Atlantic-Congo}
\define@key{fams}{tgy}{Atlantic-Congo}
\define@key{fams}{zuh}{Nuclear Trans New Guinea}
\define@key{fams}{xto}{Indo-European}
\define@key{fams}{txb}{Indo-European}
\define@key{fams}{tok}{Artificial Language}
\define@key{fams}{tkn}{Japonic}
\define@key{fams}{lbw}{Austronesian}
\define@key{fams}{tlm}{Austronesian}
\define@key{fams}{tol}{Athabaskan-Eyak-Tlingit}
\define@key{fams}{tod}{Mande}
\define@key{fams}{tdi}{Austronesian}
\define@key{fams}{tom}{Austronesian}
\define@key{fams}{txa}{Austronesian}
\define@key{fams}{ttp}{Austronesian}
\define@key{fams}{txm}{Austronesian}
\define@key{fams}{dtm}{Dogon}
\define@key{fams}{tqp}{Austronesian}
\define@key{fams}{tst}{Songhay}
\define@key{fams}{tnz}{Austroasiatic}
\define@key{fams}{tny}{Atlantic-Congo}
\define@key{fams}{tog}{Atlantic-Congo}
\define@key{fams}{xgf}{Uto-Aztecan}
\define@key{fams}{tjn}{Mande}
\define@key{fams}{tnw}{Austronesian}
\define@key{fams}{txs}{Austronesian}
\define@key{fams}{toz}{Atlantic-Congo}
\define@key{fams}{ttj}{Atlantic-Congo}
\define@key{fams}{toq}{Nilotic}
\define@key{fams}{toy}{Austronesian}
\define@key{fams}{ttu}{Austronesian}
\define@key{fams}{trz}{Chapacuran}
\define@key{fams}{trj}{Afro-Asiatic}
\define@key{fams}{fit}{Uralic}
\define@key{fams}{tdv}{Atlantic-Congo}
\define@key{fams}{tqr}{Narrow Talodi}
\define@key{fams}{dtt}{Dogon}
\define@key{fams}{tno}{Pano-Tacanan}
\define@key{fams}{tei}{Nuclear Torricelli}
\define@key{fams}{als}{Indo-European}
\define@key{fams}{ttl}{Atlantic-Congo}
\define@key{fams}{txo}{Sino-Tibetan}
\define@key{fams}{txe}{Austronesian}
\define@key{fams}{ttk}{Barbacoan}
\define@key{fams}{zph}{Otomanguean}
\define@key{fams}{tqu}{Isolate}
\define@key{fams}{neb}{Mande}
\define@key{fams}{don}{Austronesian}
\define@key{fams}{ttn}{Pauwasi}
\define@key{fams}{xtg}{Indo-European}
\define@key{fams}{trl}{Unclassifiable}
\define@key{fams}{rmg}{Speech Register}
\define@key{fams}{rmd}{Speech Register}
\define@key{fams}{trm}{Indo-European}
\define@key{fams}{tme}{Unattested}
\define@key{fams}{stg}{Austroasiatic}
\define@key{fams}{tip}{Greater Kwerba}
\define@key{fams}{trx}{Austronesian}
\define@key{fams}{tgq}{Austronesian}
\define@key{fams}{trn}{Arawakan}
\define@key{fams}{trf}{Indo-European}
\define@key{fams}{lst}{Sign Language}
\define@key{fams}{tka}{Unattested}
\define@key{fams}{tsa}{Atlantic-Congo}
\define@key{fams}{tsd}{Indo-European}
\define@key{fams}{kvz}{Nuclear Trans New Guinea}
\define@key{fams}{tsb}{Afro-Asiatic}
\define@key{fams}{tsk}{Sino-Tibetan}
\define@key{fams}{txc}{Athabaskan-Eyak-Tlingit}
\define@key{fams}{kdl}{Atlantic-Congo}
\define@key{fams}{xmw}{Austronesian}
\define@key{fams}{tsw}{Atlantic-Congo}
\define@key{fams}{hio}{Khoe-Kwadi}
\define@key{fams}{ldp}{Atlantic-Congo}
\define@key{fams}{lto}{Atlantic-Congo}
\define@key{fams}{fly}{Speech Register}
\define@key{fams}{ttz}{Sino-Tibetan}
\define@key{fams}{tsl}{Tai-Kadai}
\define@key{fams}{tvd}{Atlantic-Congo}
\define@key{fams}{tsh}{Afro-Asiatic}
\define@key{fams}{two}{Atlantic-Congo}
\define@key{fams}{tsc}{Atlantic-Congo}
\define@key{fams}{nrt}{Kalapuyan}
\define@key{fams}{tuy}{Nilotic}
\define@key{fams}{tuj}{North Halmahera}
\define@key{fams}{khc}{Austronesian}
\define@key{fams}{bhq}{Austronesian}
\define@key{fams}{tkf}{Unattested}
\define@key{fams}{tkd}{Austronesian}
\define@key{fams}{tul}{Atlantic-Congo}
\define@key{fams}{tlu}{Austronesian}
\define@key{fams}{tey}{Kadugli-Krongo}
\define@key{fams}{rak}{Austronesian}
\define@key{fams}{krt}{Saharan}
\define@key{fams}{iou}{Nuclear Trans New Guinea}
\define@key{fams}{tum}{Atlantic-Congo}
\define@key{fams}{kku}{Unattested}
\define@key{fams}{xtq}{Indo-European}
\define@key{fams}{tbr}{Kadugli-Krongo}
\define@key{fams}{enh}{Uralic}
\define@key{fams}{trt}{Geelvink Bay}
\define@key{fams}{tse}{Sign Language}
\define@key{fams}{tug}{Atlantic-Congo}
\define@key{fams}{tjg}{Austronesian}
\define@key{fams}{tqq}{Afro-Asiatic}
\define@key{fams}{dza}{Atlantic-Congo}
\define@key{fams}{ttf}{Atlantic-Congo}
\define@key{fams}{tpr}{Tupian}
\define@key{fams}{tpw}{Tupian}
\define@key{fams}{trh}{Dagan}
\define@key{fams}{trd}{Austroasiatic}
\define@key{fams}{twt}{Tupian}
\define@key{fams}{tuz}{Atlantic-Congo}
\define@key{fams}{tch}{Indo-European}
\define@key{fams}{tru}{Afro-Asiatic}
\define@key{fams}{try}{Tai-Kadai}
\define@key{fams}{tqm}{Doso-Turumsa}
\define@key{fams}{ttg}{Austronesian}
\define@key{fams}{tmi}{Austronesian}
\define@key{fams}{mtu}{Otomanguean}
\define@key{fams}{tww}{Walioic}
\define@key{fams}{ifk}{Austronesian}
\define@key{fams}{bov}{Atlantic-Congo}
\define@key{fams}{tud}{Isolate}
\define@key{fams}{tux}{Pano-Tacanan}
\define@key{fams}{xjb}{Pama-Nyungan}
\define@key{fams}{twn}{Atlantic-Congo}
\define@key{fams}{uam}{Unclassifiable}
\define@key{fams}{ksj}{Kwalean}
\define@key{fams}{byc}{Atlantic-Congo}
\define@key{fams}{uba}{Atlantic-Congo}
\define@key{fams}{ubi}{Afro-Asiatic}
\define@key{fams}{ubr}{Austronesian}
\define@key{fams}{cpb}{Arawakan}
\define@key{fams}{uda}{Atlantic-Congo}
\define@key{fams}{udu}{Koman}
\define@key{fams}{ufi}{Nuclear Trans New Guinea}
\define@key{fams}{uga}{Afro-Asiatic}
\define@key{fams}{uge}{Austronesian}
\define@key{fams}{ugo}{Sino-Tibetan}
\define@key{fams}{uha}{Atlantic-Congo}
\define@key{fams}{uis}{South Bougainville}
\define@key{fams}{udj}{Austronesian}
\define@key{fams}{kcf}{Atlantic-Congo}
\define@key{fams}{ukh}{Atlantic-Congo}
\define@key{fams}{umi}{Austronesian}
\define@key{fams}{ukp}{Atlantic-Congo}
\define@key{fams}{akd}{Atlantic-Congo}
\define@key{fams}{ukl}{Sign Language}
\define@key{fams}{uku}{Atlantic-Congo}
\define@key{fams}{ukg}{Nuclear Trans New Guinea}
\define@key{fams}{ukq}{Atlantic-Congo}
\define@key{fams}{ukw}{Atlantic-Congo}
\define@key{fams}{svb}{Austronesian}
\define@key{fams}{ull}{Dravidian}
\define@key{fams}{ulb}{Atlantic-Congo}
\define@key{fams}{ulm}{Austronesian}
\define@key{fams}{ulw}{Misumalpan}
\define@key{fams}{ulu}{Austronesian}
\define@key{fams}{xky}{Austronesian}
\define@key{fams}{gdn}{Dagan}
\define@key{fams}{umd}{Pama-Nyungan}
\define@key{fams}{xum}{Indo-European}
\define@key{fams}{umr}{Isolate}
\define@key{fams}{umg}{Pama-Nyungan}
\define@key{fams}{upi}{Border}
\define@key{fams}{sju}{Uralic}
\define@key{fams}{due}{Austronesian}
\define@key{fams}{umm}{Atlantic-Congo}
\define@key{fams}{umo}{Bororoan}
\define@key{fams}{unz}{Austronesian}
\define@key{fams}{bbn}{Austronesian}
\define@key{fams}{une}{Atlantic-Congo}
\define@key{fams}{xgu}{Worrorran}
\define@key{fams}{uni}{Sko}
\define@key{fams}{uln}{Indo-European}
\define@key{fams}{onu}{Austronesian}
\define@key{fams}{unu}{Austronesian}
\define@key{fams}{tov}{Indo-European}
\define@key{fams}{tku}{Totonacan}
\define@key{fams}{sxu}{Indo-European}
\define@key{fams}{tth}{Austroasiatic}
\define@key{fams}{dmg}{Austronesian}
\define@key{fams}{dna}{Nuclear Trans New Guinea}
\define@key{fams}{xup}{Athabaskan-Eyak-Tlingit}
\define@key{fams}{tau}{Athabaskan-Eyak-Tlingit}
\define@key{fams}{url}{Dravidian}
\define@key{fams}{urm}{Nuclear Trans New Guinea}
\define@key{fams}{uro}{Baining}
\define@key{fams}{xur}{Hurro-Urartian}
\define@key{fams}{urg}{Nuclear Trans New Guinea}
\define@key{fams}{uvh}{Nuclear Trans New Guinea}
\define@key{fams}{urx}{Nuclear Torricelli}
\define@key{fams}{urc}{Giimbiyu}
\define@key{fams}{urv}{Austronesian}
\define@key{fams}{urn}{Austronesian}
\define@key{fams}{urz}{Tupian}
\define@key{fams}{ugy}{Sign Language}
\define@key{fams}{uru}{Tupian}
\define@key{fams}{urp}{Unclassifiable}
\define@key{fams}{usk}{Atlantic-Congo}
\define@key{fams}{ush}{Indo-European}
\define@key{fams}{ulf}{Isolate}
\define@key{fams}{usp}{Mayan}
\define@key{fams}{usi}{Sino-Tibetan}
\define@key{fams}{omo}{Nuclear Trans New Guinea}
\define@key{fams}{wsg}{Dravidian}
\define@key{fams}{utu}{Nuclear Trans New Guinea}
\define@key{fams}{uuu}{Austroasiatic}
\define@key{fams}{evh}{Atlantic-Congo}
\define@key{fams}{usu}{Nuclear Trans New Guinea}
\define@key{fams}{auz}{Afro-Asiatic}
\define@key{fams}{eze}{Atlantic-Congo}
\define@key{fams}{vaa}{Indo-European}
\define@key{fams}{kqu}{Tuu}
\define@key{fams}{vgr}{Indo-European}
\define@key{fams}{dkg}{Atlantic-Congo}
\define@key{fams}{tva}{Austronesian}
\define@key{fams}{vap}{Sino-Tibetan}
\define@key{fams}{vae}{Central Sudanic}
\define@key{fams}{vsv}{Sign Language}
\define@key{fams}{vmv}{Maiduan}
\define@key{fams}{cvn}{Otomanguean}
\define@key{fams}{vlp}{Austronesian}
\define@key{fams}{mkt}{Austronesian}
\define@key{fams}{mlr}{Afro-Asiatic}
\define@key{fams}{mpr}{Austronesian}
\define@key{fams}{vnk}{Austronesian}
\define@key{fams}{vau}{Atlantic-Congo}
\define@key{fams}{vao}{Austronesian}
\define@key{fams}{vah}{Indo-European}
\define@key{fams}{vrs}{Austronesian}
\define@key{fams}{vav}{Indo-European}
\define@key{fams}{vaj}{Kxa}
\define@key{fams}{val}{Austronesian}
\define@key{fams}{vem}{Afro-Asiatic}
\define@key{fams}{vsl}{Sign Language}
\define@key{fams}{xve}{Indo-European}
\define@key{fams}{vec}{Indo-European}
\define@key{fams}{veo}{Chumashan}
\define@key{fams}{vra}{Austronesian}
\define@key{fams}{vid}{Atlantic-Congo}
\define@key{fams}{vig}{Atlantic-Congo}
\define@key{fams}{vil}{Isolate}
\define@key{fams}{dyg}{Unattested}
\define@key{fams}{svc}{Indo-European}
\define@key{fams}{vin}{Atlantic-Congo}
\define@key{fams}{vic}{Indo-European}
\define@key{fams}{vis}{Dravidian}
\define@key{fams}{vit}{Atlantic-Congo}
\define@key{fams}{vto}{Tor-Orya}
\define@key{fams}{vls}{Indo-European}
\define@key{fams}{vol}{Artificial Language}
\define@key{fams}{kch}{Unattested}
\define@key{fams}{vor}{Atlantic-Congo}
\define@key{fams}{vum}{Atlantic-Congo}
\define@key{fams}{vnp}{Austronesian}
\define@key{fams}{vun}{Atlantic-Congo}
\define@key{fams}{msn}{Austronesian}
\define@key{fams}{vut}{Atlantic-Congo}
\define@key{fams}{wbi}{Atlantic-Congo}
\define@key{fams}{wmn}{Austronesian}
\define@key{fams}{wab}{Austronesian}
\define@key{fams}{wbb}{Austronesian}
\define@key{fams}{kmx}{Kiwaian}
\define@key{fams}{wci}{Atlantic-Congo}
\define@key{fams}{wdg}{Nuclear Trans New Guinea}
\define@key{fams}{wbq}{Dravidian}
\define@key{fams}{kxp}{Indo-European}
\define@key{fams}{wdu}{Pama-Nyungan}
\define@key{fams}{wag}{Austronesian}
\define@key{fams}{wrx}{Austronesian}
\define@key{fams}{waj}{Nuclear Trans New Guinea}
\define@key{fams}{wga}{Pama-Nyungan}
\define@key{fams}{wgb}{Austronesian}
\define@key{fams}{wbr}{Indo-European}
\define@key{fams}{fad}{Nuclear Trans New Guinea}
\define@key{fams}{whk}{Austronesian}
\define@key{fams}{wgo}{Austronesian}
\define@key{fams}{wlr}{Austronesian}
\define@key{fams}{wlk}{Athabaskan-Eyak-Tlingit}
\define@key{fams}{wmh}{Austronesian}
\define@key{fams}{atr}{Cariban}
\define@key{fams}{wli}{North Halmahera}
\define@key{fams}{wja}{Atlantic-Congo}
\define@key{fams}{wav}{Atlantic-Congo}
\define@key{fams}{wwb}{Unclassifiable}
\define@key{fams}{wkd}{Austronesian}
\define@key{fams}{waf}{Unattested}
\define@key{fams}{lgl}{Austronesian}
\define@key{fams}{wlw}{Nuclear Trans New Guinea}
\define@key{fams}{wly}{Sino-Tibetan}
\define@key{fams}{wll}{Nubian}
\define@key{fams}{wlx}{Atlantic-Congo}
\define@key{fams}{waa}{Sahaptian}
\define@key{fams}{wln}{Indo-European}
\define@key{fams}{wae}{Indo-European}
\define@key{fams}{ola}{Sino-Tibetan}
\define@key{fams}{wmc}{Nuclear Trans New Guinea}
\define@key{fams}{wmi}{Pama-Nyungan}
\define@key{fams}{lbq}{Austronesian}
\define@key{fams}{waz}{Austronesian}
\define@key{fams}{qyp}{Algic}
\define@key{fams}{wnp}{Nuclear Torricelli}
\define@key{fams}{wnb}{Nuclear Trans New Guinea}
\define@key{fams}{nnp}{Sino-Tibetan}
\define@key{fams}{wbh}{Atlantic-Congo}
\define@key{fams}{wdd}{Atlantic-Congo}
\define@key{fams}{wad}{Austronesian}
\define@key{fams}{mfi}{Afro-Asiatic}
\define@key{fams}{wne}{Indo-European}
\define@key{fams}{hwa}{Kru}
\define@key{fams}{wnm}{Pama-Nyungan}
\define@key{fams}{lwg}{Atlantic-Congo}
\define@key{fams}{wng}{Nuclear Trans New Guinea}
\define@key{fams}{jub}{Atlantic-Congo}
\define@key{fams}{wno}{Nuclear Trans New Guinea}
\define@key{fams}{wnk}{Austronesian}
\define@key{fams}{wny}{Garrwan}
\define@key{fams}{juk}{Atlantic-Congo}
\define@key{fams}{juw}{Atlantic-Congo}
\define@key{fams}{wbf}{Atlantic-Congo}
\define@key{fams}{tci}{Yam}
\define@key{fams}{srv}{Austronesian}
\define@key{fams}{bpe}{Sko}
\define@key{fams}{wre}{Unattested}
\define@key{fams}{wai}{Unattested}
\define@key{fams}{wri}{Pama-Nyungan}
\define@key{fams}{wbe}{Lakes Plain}
\define@key{fams}{aml}{Austroasiatic}
\define@key{fams}{wji}{Afro-Asiatic}
\define@key{fams}{bgv}{Anim}
\define@key{fams}{wrl}{Pama-Nyungan}
\define@key{fams}{wrn}{Heibanic}
\define@key{fams}{wru}{Austronesian}
\define@key{fams}{wrv}{Suki-Gogodala}
\define@key{fams}{wss}{Atlantic-Congo}
\define@key{fams}{gsp}{Nuclear Trans New Guinea}
\define@key{fams}{wsu}{Unattested}
\define@key{fams}{wtk}{Sepik}
\define@key{fams}{wah}{Austronesian}
\define@key{fams}{wuy}{Austronesian}
\define@key{fams}{www}{Atlantic-Congo}
\define@key{fams}{wow}{Austronesian}
\define@key{fams}{wxa}{Sino-Tibetan}
\define@key{fams}{ctt}{Dravidian}
\define@key{fams}{wyr}{Tupian}
\define@key{fams}{weh}{Atlantic-Congo}
\define@key{fams}{wew}{Austronesian}
\define@key{fams}{wlh}{Austronesian}
\define@key{fams}{klh}{Nuclear Trans New Guinea}
\define@key{fams}{wei}{Anim}
\define@key{fams}{gxx}{Kru}
\define@key{fams}{ywl}{Sino-Tibetan}
\define@key{fams}{hmw}{Hmong-Mien}
\define@key{fams}{ojw}{Algic}
\define@key{fams}{tqt}{Totonacan}
\define@key{fams}{yih}{Indo-European}
\define@key{fams}{pnb}{Indo-European}
\define@key{fams}{lcp}{Austroasiatic}
\define@key{fams}{kuf}{Austroasiatic}
\define@key{fams}{mut}{Dravidian}
\define@key{fams}{kyu}{Sino-Tibetan}
\define@key{fams}{tdg}{Sino-Tibetan}
\define@key{fams}{wmg}{Sino-Tibetan}
\define@key{fams}{raf}{Sino-Tibetan}
\define@key{fams}{mmr}{Hmong-Mien}
\define@key{fams}{lia}{Atlantic-Congo}
\define@key{fams}{xwl}{Atlantic-Congo}
\define@key{fams}{bbp}{Atlantic-Congo}
\define@key{fams}{ssl}{Atlantic-Congo}
\define@key{fams}{krw}{Kru}
\define@key{fams}{nnd}{Austronesian}
\define@key{fams}{uve}{Austronesian}
\define@key{fams}{mss}{Austronesian}
\define@key{fams}{lmj}{Austronesian}
\define@key{fams}{drn}{Austronesian}
\define@key{fams}{suc}{Austronesian}
\define@key{fams}{twb}{Austronesian}
\define@key{fams}{pne}{Austronesian}
\define@key{fams}{zbw}{Austronesian}
\define@key{fams}{dnw}{Nuclear Trans New Guinea}
\define@key{fams}{nhw}{Uto-Aztecan}
\define@key{fams}{pua}{Tarascan}
\define@key{fams}{gnw}{Tupian}
\define@key{fams}{jmx}{Otomanguean}
\define@key{fams}{tnb}{Chibchan}
\define@key{fams}{amw}{Afro-Asiatic}
\define@key{fams}{azn}{Uto-Aztecan}
\define@key{fams}{wwo}{Austronesian}
\define@key{fams}{wea}{Sino-Tibetan}
\define@key{fams}{wec}{Kru}
\define@key{fams}{woy}{Unattested}
\define@key{fams}{lwh}{Tai-Kadai}
\define@key{fams}{giw}{Tai-Kadai}
\define@key{fams}{tnp}{Austronesian}
\define@key{fams}{tua}{Nuclear Torricelli}
\define@key{fams}{mtp}{Matacoan}
\define@key{fams}{wlv}{Matacoan}
\define@key{fams}{wik}{Pama-Nyungan}
\define@key{fams}{wie}{Pama-Nyungan}
\define@key{fams}{wij}{Pama-Nyungan}
\define@key{fams}{wif}{Unattested}
\define@key{fams}{wih}{Pama-Nyungan}
\define@key{fams}{wua}{Pama-Nyungan}
\define@key{fams}{wil}{Worrorran}
\define@key{fams}{wit}{Wintuan}
\define@key{fams}{gdr}{Eastern Trans-Fly}
\define@key{fams}{wrh}{Pama-Nyungan}
\define@key{fams}{wir}{Tupian}
\define@key{fams}{wiu}{Isolate}
\define@key{fams}{xwc}{Siouan}
\define@key{fams}{woc}{Austronesian}
\define@key{fams}{wbw}{Austronesian}
\define@key{fams}{wyi}{Pama-Nyungan}
\define@key{fams}{jod}{Mande}
\define@key{fams}{wod}{Nuclear Trans New Guinea}
\define@key{fams}{wle}{Afro-Asiatic}
\define@key{fams}{wom}{Atlantic-Congo}
\define@key{fams}{wmo}{Nuclear Torricelli}
\define@key{fams}{won}{Atlantic-Congo}
\define@key{fams}{cwd}{Algic}
\define@key{fams}{kda}{Pama-Nyungan}
\define@key{fams}{wor}{Geelvink Bay}
\define@key{fams}{jud}{Mande}
\define@key{fams}{wsv}{Indo-European}
\define@key{fams}{wtw}{Austronesian}
\define@key{fams}{wud}{Atlantic-Congo}
\define@key{fams}{qgu}{Pama-Nyungan}
\define@key{fams}{wlu}{Pama-Nyungan}
\define@key{fams}{wux}{Limilngan-Wulna}
\define@key{fams}{bqm}{Atlantic-Congo}
\define@key{fams}{wum}{Atlantic-Congo}
\define@key{fams}{ywu}{Sino-Tibetan}
\define@key{fams}{bwn}{Hmong-Mien}
\define@key{fams}{wub}{Worrorran}
\define@key{fams}{wur}{Marrku-Wurrugu}
\define@key{fams}{yig}{Sino-Tibetan}
\define@key{fams}{bse}{Atlantic-Congo}
\define@key{fams}{wsi}{Austronesian}
\define@key{fams}{wuh}{Sino-Tibetan}
\define@key{fams}{wut}{Sko}
\define@key{fams}{wuv}{Austronesian}
\define@key{fams}{wym}{Indo-European}
\define@key{fams}{zax}{Otomanguean}
\define@key{fams}{xkr}{Nuclear-Macro-Je}
\define@key{fams}{xan}{Afro-Asiatic}
\define@key{fams}{ztg}{Otomanguean}
\define@key{fams}{axx}{Austronesian}
\define@key{fams}{xeg}{Tuu}
\define@key{fams}{xet}{Tupian}
\define@key{fams}{hsn}{Sino-Tibetan}
\define@key{fams}{sjo}{Tungusic}
\define@key{fams}{asn}{Tupian}
\define@key{fams}{xiy}{Tupian}
\define@key{fams}{xip}{Unattested}
\define@key{fams}{xii}{Khoe-Kwadi}
\define@key{fams}{xoo}{Isolate}
\define@key{fams}{xwe}{Atlantic-Congo}
\define@key{fams}{tyy}{Atlantic-Congo}
\define@key{fams}{muu}{Afro-Asiatic}
\define@key{fams}{yar}{Cariban}
\define@key{fams}{ybn}{Arawakan}
\define@key{fams}{ybm}{Nuclear Trans New Guinea}
\define@key{fams}{ybo}{Nuclear Trans New Guinea}
\define@key{fams}{ekr}{Atlantic-Congo}
\define@key{fams}{rys}{Japonic}
\define@key{fams}{wfg}{Pauwasi}
\define@key{fams}{ygm}{Nuclear Trans New Guinea}
\define@key{fams}{ygw}{Angan}
\define@key{fams}{rhp}{Nuclear Torricelli}
\define@key{fams}{ner}{Konda-Yahadian}
\define@key{fams}{ynu}{Tucanoan}
\define@key{fams}{iyx}{Atlantic-Congo}
\define@key{fams}{ykk}{Austronesian}
\define@key{fams}{ybh}{Sino-Tibetan}
\define@key{fams}{xyl}{Unattested}
\define@key{fams}{yba}{Atlantic-Congo}
\define@key{fams}{jal}{Austronesian}
\define@key{fams}{zpu}{Otomanguean}
\define@key{fams}{yal}{Mande}
\define@key{fams}{ymp}{Austronesian}
\define@key{fams}{yat}{Atlantic-Congo}
\define@key{fams}{ymb}{Nuclear Torricelli}
\define@key{fams}{yme}{Peba-Yagua}
\define@key{fams}{ymn}{Austronesian}
\define@key{fams}{qur}{Quechuan}
\define@key{fams}{yda}{Pama-Nyungan}
\define@key{fams}{dym}{Dogon}
\define@key{fams}{xyb}{Pama-Nyungan}
\define@key{fams}{zyg}{Tai-Kadai}
\define@key{fams}{jng}{Yangmanic}
\define@key{fams}{yng}{Atlantic-Congo}
\define@key{fams}{bsx}{Atlantic-Congo}
\define@key{fams}{yav}{Atlantic-Congo}
\define@key{fams}{ygl}{Nuclear Torricelli}
\define@key{fams}{ymo}{Nuclear Torricelli}
\define@key{fams}{yde}{Nuclear Torricelli}
\define@key{fams}{ynl}{Nuclear Trans New Guinea}
\define@key{fams}{tjj}{Pama-Nyungan}
\define@key{fams}{ysm}{Sign Language}
\define@key{fams}{jay}{Pama-Nyungan}
\define@key{fams}{guu}{Yanomamic}
\define@key{fams}{asy}{Nuclear Trans New Guinea}
\define@key{fams}{yre}{Mande}
\define@key{fams}{yev}{Nuclear Torricelli}
\define@key{fams}{yrw}{Nuclear Trans New Guinea}
\define@key{fams}{zae}{Otomanguean}
\define@key{fams}{yro}{Yanomamic}
\define@key{fams}{yko}{Atlantic-Congo}
\define@key{fams}{zty}{Otomanguean}
\define@key{fams}{yla}{Keram}
\define@key{fams}{yuw}{Nuclear Trans New Guinea}
\define@key{fams}{jau}{Austronesian}
\define@key{fams}{yyu}{Nuclear Torricelli}
\define@key{fams}{zpb}{Otomanguean}
\define@key{fams}{qux}{Quechuan}
\define@key{fams}{yvt}{Arawakan}
\define@key{fams}{yww}{Pama-Nyungan}
\define@key{fams}{ywn}{Pano-Tacanan}
\define@key{fams}{yaw}{Arawakan}
\define@key{fams}{yby}{Nuclear Trans New Guinea}
\define@key{fams}{ybx}{Walioic}
\define@key{fams}{ykr}{Nuclear Trans New Guinea}
\define@key{fams}{yel}{Atlantic-Congo}
\define@key{fams}{ylg}{Ndu}
\define@key{fams}{ynq}{Atlantic-Congo}
\define@key{fams}{yec}{Mixed Language}
\define@key{fams}{yei}{Atlantic-Congo}
\define@key{fams}{yra}{Isolate}
\define@key{fams}{gop}{Austronesian}
\define@key{fams}{yrn}{Tai-Kadai}
\define@key{fams}{yeu}{Dravidian}
\define@key{fams}{yes}{Atlantic-Congo}
\define@key{fams}{yet}{Isolate}
\define@key{fams}{yej}{Indo-European}
\define@key{fams}{ydg}{Indo-European}
\define@key{fams}{yim}{Sino-Tibetan}
\define@key{fams}{kvu}{Sino-Tibetan}
\define@key{fams}{yin}{Austroasiatic}
\define@key{fams}{yil}{Pama-Nyungan}
\define@key{fams}{ywg}{Pama-Nyungan}
\define@key{fams}{kvy}{Sino-Tibetan}
\define@key{fams}{yxm}{Pama-Nyungan}
\define@key{fams}{ljw}{Pama-Nyungan}
\define@key{fams}{yiy}{Pama-Nyungan}
\define@key{fams}{yis}{Nuclear Torricelli}
\define@key{fams}{gek}{Afro-Asiatic}
\define@key{fams}{yob}{Austronesian}
\define@key{fams}{gud}{Kru}
\define@key{fams}{yog}{Austronesian}
\define@key{fams}{ydk}{Nuclear Trans New Guinea}
\define@key{fams}{yki}{Austronesian}
\define@key{fams}{ygs}{Sign Language}
\define@key{fams}{xty}{Otomanguean}
\define@key{fams}{pil}{Atlantic-Congo}
\define@key{fams}{yoi}{Japonic}
\define@key{fams}{sxk}{Kalapuyan}
\define@key{fams}{nru}{Sino-Tibetan}
\define@key{fams}{zyn}{Tai-Kadai}
\define@key{fams}{zyb}{Tai-Kadai}
\define@key{fams}{yno}{Tai-Kadai}
\define@key{fams}{yon}{Nuclear Trans New Guinea}
\define@key{fams}{yut}{Nuclear Trans New Guinea}
\define@key{fams}{mts}{Pano-Tacanan}
\define@key{fams}{yox}{Japonic}
\define@key{fams}{yot}{Atlantic-Congo}
\define@key{fams}{zyj}{Tai-Kadai}
\define@key{fams}{ytw}{Nuclear Trans New Guinea}
\define@key{fams}{yoy}{Tai-Kadai}
\define@key{fams}{nua}{Austronesian}
\define@key{fams}{msd}{Sign Language}
\define@key{fams}{mvg}{Otomanguean}
\define@key{fams}{yub}{Pama-Nyungan}
\define@key{fams}{ysl}{Sign Language}
\define@key{fams}{ygu}{Unattested}
\define@key{fams}{yab}{Naduhup}
\define@key{fams}{omk}{Yukaghir}
\define@key{fams}{ybl}{Atlantic-Congo}
\define@key{fams}{yuq}{Tupian}
\define@key{fams}{ljx}{Pama-Nyungan}
\define@key{fams}{mab}{Otomanguean}
\define@key{fams}{yau}{Isolate}
\define@key{fams}{ztx}{Otomanguean}
\define@key{fams}{kji}{Austronesian}
\define@key{fams}{nhi}{Uto-Aztecan}
\define@key{fams}{ctz}{Otomanguean}
\define@key{fams}{atb}{Sino-Tibetan}
\define@key{fams}{zkr}{Sino-Tibetan}
\define@key{fams}{zsl}{Sign Language}
\define@key{fams}{zak}{Atlantic-Congo}
\define@key{fams}{zau}{Sino-Tibetan}
\define@key{fams}{zna}{Atlantic-Congo}
\define@key{fams}{zah}{Afro-Asiatic}
\define@key{fams}{zpw}{Otomanguean}
\define@key{fams}{zaj}{Atlantic-Congo}
\define@key{fams}{zbu}{Afro-Asiatic}
\define@key{fams}{zaz}{Afro-Asiatic}
\define@key{fams}{zal}{Sino-Tibetan}
\define@key{fams}{kxk}{Sino-Tibetan}
\define@key{fams}{zwa}{Afro-Asiatic}
\define@key{fams}{jaj}{Austronesian}
\define@key{fams}{zua}{Afro-Asiatic}
\define@key{fams}{dhm}{Atlantic-Congo}
\define@key{fams}{zeg}{Austronesian}
\define@key{fams}{czn}{Otomanguean}
\define@key{fams}{zhb}{Sino-Tibetan}
\define@key{fams}{xzh}{Sino-Tibetan}
\define@key{fams}{zhi}{Atlantic-Congo}
\define@key{fams}{zhw}{Atlantic-Congo}
\define@key{fams}{zia}{Nuclear Trans New Guinea}
\define@key{fams}{zil}{Mande}
\define@key{fams}{ziw}{Atlantic-Congo}
\define@key{fams}{zib}{Sign Language}
\define@key{fams}{zmb}{Atlantic-Congo}
\define@key{fams}{zin}{Atlantic-Congo}
\define@key{fams}{sih}{Austronesian}
\define@key{fams}{zrn}{Afro-Asiatic}
\define@key{fams}{ziz}{Afro-Asiatic}
\define@key{fams}{pto}{Tupian}
\define@key{fams}{yzk}{Sino-Tibetan}
\define@key{fams}{gbz}{Indo-European}
\define@key{fams}{czt}{Sino-Tibetan}
\define@key{fams}{zom}{Sino-Tibetan}
\define@key{fams}{zla}{Atlantic-Congo}
\define@key{fams}{gnd}{Afro-Asiatic}
\define@key{fams}{zuy}{Afro-Asiatic}
\define@key{fams}{jmb}{Afro-Asiatic}
\define@key{fams}{zzj}{Tai-Kadai}
\define@key{fams}{zyp}{Sino-Tibetan}
}
\DeclareOption{wals}{\define@key{names}{knw}{North-Central Ju}
\define@key{names}{nmn}{East Taa}
\define@key{names}{alu}{'Are'are}
\define@key{names}{hnh}{//Ani}
\define@key{names}{xam}{/Xam}
\define@key{names}{huc}{Amkoe}
\define@key{names}{apq}{Apucikwar}
\define@key{names}{aiw}{Aari}
\define@key{names}{aau}{Abau}
\define@key{names}{abq}{Abaza}
\define@key{names}{abe}{Western Abenaki}
\define@key{names}{abi}{Abidji}
\define@key{names}{axb}{Abipon}
\define@key{names}{abk}{Abkhaz}
\define@key{names}{abz}{Abui}
\define@key{names}{kgr}{Abun}
\define@key{names}{ace}{Acehnese}
\define@key{names}{aca}{Achagua}
\define@key{names}{acn}{Longchuan Achang}
\define@key{names}{ach}{Acoli}
\define@key{names}{acu}{Achuar-Shiwiar}
\define@key{names}{acv}{Achumawi}
\define@key{names}{guq}{Aché}
\define@key{names}{acr}{Achi}
\define@key{names}{kjq}{Western Keres}
\define@key{names}{ads}{Adamorobe Sign Language}
\define@key{names}{adn}{Adang}
\define@key{names}{adj}{Adioukrou}
\define@key{names}{ady}{Adyghe}
\define@key{names}{adt}{Adnyamathanha}
\define@key{names}{adz}{Adzera}
\define@key{names}{awi}{Aekyom}
\define@key{names}{afr}{Afrikaans}
\define@key{names}{agd}{Agarabi}
\define@key{names}{agq}{Aghem}
\define@key{names}{ahh}{Aghu}
\define@key{names}{agx}{Aghul}
\define@key{names}{agt}{Central Cagayan Agta}
\define@key{names}{duo}{Dupaninan Agta}
\define@key{names}{agu}{Aguacateco}
\define@key{names}{agr}{Aguaruna}
\define@key{names}{aht}{Ahtena}
\define@key{names}{tba}{Aikanã}
\define@key{names}{ain}{Hokkaido Ainu}
\define@key{names}{ahp}{Aproumu Aizi}
\define@key{names}{aja}{Aja (South Sudan)}
\define@key{names}{ajg}{Aja (Benin)}
\define@key{names}{aji}{Ajië}
\define@key{names}{axk}{Yaka (Central African Republic)}
\define@key{names}{abj}{Akabea}
\define@key{names}{aci}{Akacari}
\define@key{names}{akx}{Akakede}
\define@key{names}{aka}{Akan}
\define@key{names}{ake}{Akawaio-Ingariko}
\define@key{names}{ahk}{Akha}
\define@key{names}{akv}{Akhvakh}
\define@key{names}{akl}{Aklanon}
\define@key{names}{akw}{Akwa}
\define@key{names}{nrz}{Lala}
\define@key{names}{akz}{Alabama}
\define@key{names}{wbj}{Alagwa}
\define@key{names}{amp}{Alamblak}
\define@key{names}{btz}{Batak Alas-Kluet}
\define@key{names}{alh}{Alawa}
\define@key{names}{sqi}{Albanian}
\define@key{names}{ale}{Aleut}
\define@key{names}{alq}{Algonquin}
\define@key{names}{ald}{Alladian}
\define@key{names}{gsw}{Central Alemannic}
\define@key{names}{aes}{Alsea-Yaquina}
\define@key{names}{alt}{Southern Altai}
\define@key{names}{alp}{Alune}
\define@key{names}{ems}{Pacific Gulf Yupik}
\define@key{names}{alr}{Alutor}
\define@key{names}{aly}{Alyawarr}
\define@key{names}{amm}{Ama (Papua New Guinea)}
\define@key{names}{amc}{Amahuaca}
\define@key{names}{amn}{Amanab}
\define@key{names}{aie}{Amara}
\define@key{names}{amr}{Amarakaeri}
\define@key{names}{omb}{East Ambae}
\define@key{names}{amk}{Ambai}
\define@key{names}{abt}{Ambulas}
\define@key{names}{adx}{Amdo Tibetan}
\define@key{names}{aey}{Amele}
\define@key{names}{ase}{American Sign Language}
\define@key{names}{amh}{Amharic}
\define@key{names}{ami}{Amis}
\define@key{names}{amo}{Amo}
\define@key{names}{apz}{Safeyoka}
\define@key{names}{ame}{Yanesha'}
\define@key{names}{amu}{Guerrero Amuzgo}
\define@key{names}{imi}{Anamuxra}
\define@key{names}{ani}{Andi}
\define@key{names}{ano}{Andoque}
\define@key{names}{aty}{Aneityum}
\define@key{names}{agm}{Angaataha}
\define@key{names}{njm}{Angami Naga}
\define@key{names}{anc}{Ngas}
\define@key{names}{agg}{Angor}
\define@key{names}{aoa}{Angolar}
\define@key{names}{awg}{Anguthimri}
\define@key{names}{aoi}{Anindilyakwa}
\define@key{names}{nun}{Nung (Myanmar)}
\define@key{names}{cko}{Anufo}
\define@key{names}{any}{Anyin}
\define@key{names}{anu}{Anuak}
\define@key{names}{anz}{Anem}
\define@key{names}{njo}{Ao Naga}
\define@key{names}{apm}{Mescalero-Chiricahua Apache}
\define@key{names}{apj}{Jicarilla Apache}
\define@key{names}{apw}{Western Apache}
\define@key{names}{apy}{Apalaí}
\define@key{names}{apt}{Apatani}
\define@key{names}{apn}{Apinayé}
\define@key{names}{apu}{Apurinã}
\define@key{names}{ard}{Arabana}
\define@key{names}{arl}{Arabela}
\define@key{names}{abv}{Baharna Arabic}
\define@key{names}{mey}{Hassaniyya}
\define@key{names}{shu}{Chadian Arabic}
\define@key{names}{ayl}{Libyan Arabic}
\define@key{names}{arz}{Egyptian Arabic}
\define@key{names}{afb}{Gulf Arabic}
\define@key{names}{acw}{Hijazi Arabic}
\define@key{names}{acm}{Gilit Mesopotamian Arabic}
\define@key{names}{acy}{Cypriot Arabic}
\define@key{names}{arb}{Standard Arabic}
\define@key{names}{ary}{Moroccan Arabic}
\define@key{names}{ajp}{South Levantine Arabic}
\define@key{names}{ayn}{Sanaani Arabic}
\define@key{names}{apc}{North Levantine Arabic}
\define@key{names}{aeb}{Tunisian Arabic}
\define@key{names}{rmz}{Marma}
\define@key{names}{akr}{Araki}
\define@key{names}{atq}{Aralle-Tabulahan}
\define@key{names}{jbj}{Dombano}
\define@key{names}{aro}{Araona}
\define@key{names}{arp}{Arapaho}
\define@key{names}{aah}{Abu' Arapesh}
\define@key{names}{ape}{Bukiyip}
\define@key{names}{arv}{Arbore}
\define@key{names}{aqc}{Archi}
\define@key{names}{laz}{Aribwatsa}
\define@key{names}{ari}{Arikara}
\define@key{names}{hye}{Eastern Armenian}
\define@key{names}{hyw}{Western Armenian}
\define@key{names}{apr}{Arop-Lokep}
\define@key{names}{aia}{Arosi}
\define@key{names}{aer}{Eastern Arrernte}
\define@key{names}{are}{Western Arrarnta}
\define@key{names}{cns}{Central Asmat}
\define@key{names}{asm}{Assamese}
\define@key{names}{ast}{Asturian-Leonese-Cantabrian}
\define@key{names}{asu}{Tocantins Asurini}
\define@key{names}{kuz}{Kunza}
\define@key{names}{aqp}{Atakapa}
\define@key{names}{tay}{Atayal}
\define@key{names}{upv}{Uripiv-Wala-Rano-Atchin}
\define@key{names}{aph}{Athpariya}
\define@key{names}{atj}{Atikamekw}
\define@key{names}{atw}{Atsugewi}
\define@key{names}{avt}{Au}
\define@key{names}{aul}{Aulua}
\define@key{names}{asf}{Auslan}
\define@key{names}{auy}{Awiyaana}
\define@key{names}{ava}{Avar}
\define@key{names}{avn}{Avatime}
\define@key{names}{avi}{Avikam}
\define@key{names}{avu}{Avokaya}
\define@key{names}{awb}{Awa (Papua New Guinea)}
\define@key{names}{kwi}{Awa-Cuaiquer}
\define@key{names}{awa}{Awadhi}
\define@key{names}{awn}{Awngi}
\define@key{names}{kmn}{Awtuw}
\define@key{names}{auw}{Awyi}
\define@key{names}{nfl}{Äiwoo}
\define@key{names}{ayr}{Central Aymara}
\define@key{names}{aib}{Ainu (China)}
\define@key{names}{ayo}{Ayoreo}
\define@key{names}{azb}{South Azerbaijani}
\define@key{names}{koe}{Kacipo-Balesi}
\define@key{names}{bvx}{Dibole}
\define@key{names}{bav}{Vengo}
\define@key{names}{wdj}{Wadjiginy}
\define@key{names}{bfq}{Badaga}
\define@key{names}{bde}{Bade}
\define@key{names}{bia}{Badimaya}
\define@key{names}{ksf}{Bafia}
\define@key{names}{bfd}{Bafut}
\define@key{names}{bsp}{Baga Sitemu}
\define@key{names}{bmi}{Bagirmi}
\define@key{names}{fuu}{Furu}
\define@key{names}{bgq}{Bagri}
\define@key{names}{kva}{Bagvalal}
\define@key{names}{bdw}{Baham}
\define@key{names}{bjh}{Bahinemo}
\define@key{names}{bdq}{Bahnar}
\define@key{names}{bca}{Central Bai}
\define@key{names}{bdl}{Indonesian Bajau}
\define@key{names}{bdr}{West Coast Bajau}
\define@key{names}{bkc}{Baka (Cameroon)}
\define@key{names}{bdh}{Baka (South Sudan)}
\define@key{names}{bkq}{Bakairí}
\define@key{names}{bri}{Mokpwe}
\define@key{names}{blw}{Balangao}
\define@key{names}{blz}{Balantak}
\define@key{names}{ban}{Balinese}
\define@key{names}{bft}{Balti}
\define@key{names}{bgn}{Western Balochi}
\define@key{names}{ptu}{Bambam}
\define@key{names}{bam}{Bambara}
\define@key{names}{bax}{Bamun}
\define@key{names}{bcw}{Bana}
\define@key{names}{jaa}{Madi}
\define@key{names}{bza}{Bandi}
\define@key{names}{bdy}{Middle Clarence Bandjalang}
\define@key{names}{bgz}{Banggai}
\define@key{names}{bjb}{Banggarla}
\define@key{names}{bdg}{Bonggi}
\define@key{names}{dba}{Bangime}
\define@key{names}{bvv}{Baniva}
\define@key{names}{bwi}{Baniwa do Icana}
\define@key{names}{abb}{Bankon}
\define@key{names}{bcm}{Bannoni}
\define@key{names}{bnq}{Bantik}
\define@key{names}{peh}{Bonan}
\define@key{names}{bci}{Baoulé}
\define@key{names}{loy}{Lowa}
\define@key{names}{bbb}{Barai}
\define@key{names}{brm}{Barambu}
\define@key{names}{bsn}{Barasana-Eduria}
\define@key{names}{bcj}{Bardi}
\define@key{names}{mlp}{Bargam}
\define@key{names}{bfa}{Bari}
\define@key{names}{bba}{Baatonum}
\define@key{names}{wra}{Barupu}
\define@key{names}{byr}{Baruya}
\define@key{names}{bae}{Baré}
\define@key{names}{mot}{Barí}
\define@key{names}{bsc}{Bassari-Tanda}
\define@key{names}{bas}{Basa (Cameroon)}
\define@key{names}{bak}{Bashkir}
\define@key{names}{eus}{Basque}
\define@key{names}{bya}{Batak}
\define@key{names}{btx}{Batak Karo}
\define@key{names}{bbc}{Batak Toba}
\define@key{names}{bhm}{Bathari}
\define@key{names}{bbd}{Bau}
\define@key{names}{brg}{Baure}
\define@key{names}{bvz}{Bauzi}
\define@key{names}{bgr}{Bawm Chin}
\define@key{names}{bsw}{Baiso}
\define@key{names}{bxj}{Bayungu}
\define@key{names}{beq}{Beembe}
\define@key{names}{dbj}{Ida'an}
\define@key{names}{bej}{Beja}
\define@key{names}{byw}{Belhariya}
\define@key{names}{blc}{Bella Coola}
\define@key{names}{bel}{Belarusian}
\define@key{names}{bem}{Bemba (Zambia)}
\define@key{names}{bef}{Benabena}
\define@key{names}{nhb}{Beng}
\define@key{names}{bng}{Benga}
\define@key{names}{ben}{Bengali}
\define@key{names}{ctg}{Chittagonian}
\define@key{names}{bue}{Beothuk}
\define@key{names}{brf}{Bera}
\define@key{names}{shy}{Chaouia of the Aures}
\define@key{names}{grr}{Sud Oranais-Gourara}
\define@key{names}{tzm}{Central Moroccan Berber}
\define@key{names}{mzb}{Tumzabt}
\define@key{names}{rif}{Tarifiyt-Beni-Iznasen-Eastern Middle Atlas Berber}
\define@key{names}{siz}{Siwi}
\define@key{names}{oua}{Ouargli}
\define@key{names}{brc}{Berbice Creole Dutch}
\define@key{names}{zag}{Beria}
\define@key{names}{bkl}{Berik}
\define@key{names}{wti}{Berta}
\define@key{names}{xub}{Betta Kurumba}
\define@key{names}{kap}{Bezhta}
\define@key{names}{bhb}{Bhili}
\define@key{names}{bho}{Bhojpuri}
\define@key{names}{unr}{Mundari}
\define@key{names}{bif}{Biafada}
\define@key{names}{bhw}{Biak}
\define@key{names}{bth}{Biatah Bidayuh}
\define@key{names}{bid}{Bidiyo}
\define@key{names}{bcl}{Coastal-Naga Bikol}
\define@key{names}{bip}{Bila}
\define@key{names}{bpr}{Koronadal Blaan}
\define@key{names}{byn}{Bilin}
\define@key{names}{nbj}{Ngarinman}
\define@key{names}{bll}{Biloxi}
\define@key{names}{blb}{Bilua}
\define@key{names}{bhp}{Bima}
\define@key{names}{bim}{Bimoba}
\define@key{names}{bhg}{Binandere}
\define@key{names}{bin}{Bini}
\define@key{names}{gup}{Bininj Kun-Wok}
\define@key{names}{bkd}{Talaandig-Binukid}
\define@key{names}{bjr}{Binumarien}
\define@key{names}{bzr}{Biri}
\define@key{names}{bom}{Berom}
\define@key{names}{bvq}{Birri}
\define@key{names}{bib}{Bissa}
\define@key{names}{bis}{Bislama}
\define@key{names}{bla}{Siksika}
\define@key{names}{kvg}{Kuni-Boazi}
\define@key{names}{bni}{Bobangi}
\define@key{names}{bbo}{Northern Bobo Madaré}
\define@key{names}{brx}{Bodo-Mech}
\define@key{names}{bzf}{Boikin}
\define@key{names}{bqc}{Boko (Benin)}
\define@key{names}{bol}{Bole}
\define@key{names}{bli}{Bolia}
\define@key{names}{bot}{Bongo}
\define@key{names}{bpu}{Bongu}
\define@key{names}{lbk}{Central Bontoc}
\define@key{names}{boa}{Bora}
\define@key{names}{adi}{Bori-Karko}
\define@key{names}{bor}{Bororo}
\define@key{names}{brn}{Boruca}
\define@key{names}{bos}{Bosnian}
\define@key{names}{boz}{Tiéyaxo Bozo}
\define@key{names}{brh}{Brahui}
\define@key{names}{brb}{Brao}
\define@key{names}{bre}{Breton}
\define@key{names}{bzd}{Bribri}
\define@key{names}{bfi}{British Sign Language}
\define@key{names}{tcs}{Torres Strait-Lockhart River Creole}
\define@key{names}{bkk}{Brokskat}
\define@key{names}{bru}{Eastern Bru}
\define@key{names}{brv}{Western Bru}
\define@key{names}{bvb}{Bube}
\define@key{names}{buu}{Budu}
\define@key{names}{bdk}{Budukh}
\define@key{names}{bdm}{Buduma}
\define@key{names}{bug}{Buginese}
\define@key{names}{sab}{Buglere}
\define@key{names}{bgg}{Bugun}
\define@key{names}{buo}{Terei}
\define@key{names}{nmg}{Kwasio}
\define@key{names}{bxk}{Bukusu}
\define@key{names}{bul}{Bulgarian}
\define@key{names}{bwu}{Buli (Ghana)}
\define@key{names}{bzq}{Buli (Indonesia)}
\define@key{names}{bum}{Bulu (Cameroon)}
\define@key{names}{tkw}{Teanu}
\define@key{names}{bfu}{Bunan}
\define@key{names}{buh}{Younuo Bunu}
\define@key{names}{bck}{Bunaba}
\define@key{names}{bwr}{Bura-Pabir}
\define@key{names}{bvr}{Burarra}
\define@key{names}{bxm}{Mongolia Buriat}
\define@key{names}{bji}{Burji}
\define@key{names}{mya}{Burmese}
\define@key{names}{mhs}{Buru (Indonesia)}
\define@key{names}{bmu}{Burum-Mindik}
\define@key{names}{bds}{Burunge}
\define@key{names}{bsk}{Burushaski}
\define@key{names}{bqp}{Busa}
\define@key{names}{buf}{Bushoong}
\define@key{names}{ngc}{Ngombe (Democratic Republic of Congo)}
\define@key{names}{bee}{Byangsi}
\define@key{names}{bev}{Daloa Bété}
\define@key{names}{cjp}{Cabécar}
\define@key{names}{cbv}{Kakua}
\define@key{names}{cad}{Caddo}
\define@key{names}{chl}{Cahuilla}
\define@key{names}{cak}{Kaqchikel}
\define@key{names}{rab}{Camling}
\define@key{names}{cjo}{Ashéninka Pajonal}
\define@key{names}{kbh}{Camsá}
\define@key{names}{knm}{Katukína-Kanamarí}
\define@key{names}{cbu}{Candoshi-Shapra}
\define@key{names}{ram}{Canela-Krahô}
\define@key{names}{yue}{Yue Chinese}
\define@key{names}{kaq}{Capanahua}
\define@key{names}{cbc}{Carapana}
\define@key{names}{car}{Galibi Carib}
\define@key{names}{mch}{Ye'kwana}
\define@key{names}{cal}{Carolinian}
\define@key{names}{crx}{Central Carrier}
\define@key{names}{cbr}{Cashibo-Cacataibo}
\define@key{names}{cbs}{Cashinahua}
\define@key{names}{cat}{Catalan}
\define@key{names}{chc}{Catawba}
\define@key{names}{cto}{Emberá-Catío}
\define@key{names}{cav}{Cavineña}
\define@key{names}{cbi}{Cha'palaa}
\define@key{names}{cay}{Cayuga}
\define@key{names}{cyb}{Cayubaba}
\define@key{names}{ceb}{Cebuano}
\define@key{names}{old}{Mochi}
\define@key{names}{suq}{Tirma-Chai}
\define@key{names}{cld}{Chaldean Neo-Aramaic}
\define@key{names}{cjm}{Eastern Cham}
\define@key{names}{cja}{Western Cham}
\define@key{names}{cji}{Chamalal}
\define@key{names}{can}{Chambri}
\define@key{names}{cha}{Chamorro}
\define@key{names}{nbc}{Chang Naga}
\define@key{names}{chx}{Chantyal}
\define@key{names}{tuu}{Tututni}
\define@key{names}{cya}{Nopala Chatino}
\define@key{names}{cta}{Tataltepec Chatino}
\define@key{names}{ctp}{Western Highland Chatino}
\define@key{names}{cdn}{Chaudangsi}
\define@key{names}{cbk}{Chavacano}
\define@key{names}{cbt}{Shawi}
\define@key{names}{che}{Chechen}
\define@key{names}{cjh}{Upper Chehalis}
\define@key{names}{mrn}{Cheke Holo}
\define@key{names}{xch}{Chimakum}
\define@key{names}{cdm}{Chepang}
\define@key{names}{chr}{Cherokee}
\define@key{names}{chy}{Cheyenne}
\define@key{names}{nya}{Nyanja}
\define@key{names}{pei}{Chichimeca-Jonaz}
\define@key{names}{cic}{Chickasaw}
\define@key{names}{cob}{Chicomuceltec}
\define@key{names}{cid}{Chimariko}
\define@key{names}{cbg}{Chimila}
\define@key{names}{mrh}{Mara Chin}
\define@key{names}{csy}{Sizang Chin}
\define@key{names}{ctd}{Tedim Chin}
\define@key{names}{cco}{Comaltepec Chinantec}
\define@key{names}{cle}{Lealao Chinantec}
\define@key{names}{cpa}{Palantla Chinantec}
\define@key{names}{chq}{Quiotepec Chinantec}
\define@key{names}{cuc}{Usila Chinantec}
\define@key{names}{cso}{Sochiapam Chinantec}
\define@key{names}{cnt}{Tepetotutla Chinantec}
\define@key{names}{csl}{Chinese Sign Language}
\define@key{names}{chh}{Clatsop-Shoalwater Chinook}
\define@key{names}{wac}{Upper Chinook}
\define@key{names}{cap}{Chipaya}
\define@key{names}{chp}{Chipewyan}
\define@key{names}{cax}{Lomeriano-Ignaciano Chiquitano}
\define@key{names}{gui}{Eastern Bolivian Guaraní}
\define@key{names}{ctm}{Chitimacha}
\define@key{names}{coz}{Chochotec}
\define@key{names}{cho}{Choctaw}
\define@key{names}{ctu}{Chol}
\define@key{names}{cht}{Cholón}
\define@key{names}{chd}{Highland Oaxaca Chontal}
\define@key{names}{clo}{Lowland Oaxaca Chontal}
\define@key{names}{chf}{Tabasco Chontal}
\define@key{names}{caa}{Chortí}
\define@key{names}{crw}{Chrau}
\define@key{names}{cje}{Chru}
\define@key{names}{cjv}{Chuave}
\define@key{names}{cac}{Chuj}
\define@key{names}{ckt}{Chukchi}
\define@key{names}{clw}{Chulym Turkic}
\define@key{names}{boi}{Barbareño}
\define@key{names}{inz}{Ineseño}
\define@key{names}{ncu}{Chumburung}
\define@key{names}{chk}{Chuukese}
\define@key{names}{chv}{Chuvash}
\define@key{names}{cao}{Chácobo}
\define@key{names}{lua}{Luba-Lulua}
\define@key{names}{clm}{Clallam}
\define@key{names}{xcw}{Coahuilteco}
\define@key{names}{cod}{Cocama-Cocamilla}
\define@key{names}{coc}{Cocopa}
\define@key{names}{crd}{Coeur d'Alene}
\define@key{names}{con}{Cofán}
\define@key{names}{kog}{Cogui}
\define@key{names}{col}{Columbia-Wenatchi}
\define@key{names}{com}{Comanche}
\define@key{names}{xcm}{Comecrudan}
\define@key{names}{swb}{Maore Comorian}
\define@key{names}{coo}{Comox}
\define@key{names}{csz}{Hanis}
\define@key{names}{cop}{Coptic}
\define@key{names}{crn}{El Nayar Cora}
\define@key{names}{cor}{Cornish}
\define@key{names}{crk}{Plains Cree}
\define@key{names}{csw}{Swampy Cree}
\define@key{names}{mus}{Creek}
\define@key{names}{crh}{Crimean Tatar}
\define@key{names}{cro}{Crow}
\define@key{names}{cua}{Cua}
\define@key{names}{cub}{Cubeo}
\define@key{names}{cui}{Cuiba}
\define@key{names}{cuy}{Cuitlatec}
\define@key{names}{cul}{Culina}
\define@key{names}{cup}{Cupeño}
\define@key{names}{kpc}{Curripaco}
\define@key{names}{ces}{Czech}
\define@key{names}{cam}{Cemuhî}
\define@key{names}{kzf}{Da'a Kaili}
\define@key{names}{dbq}{Daba}
\define@key{names}{dav}{Taita}
\define@key{names}{mps}{Dadibi}
\define@key{names}{dgz}{Daga}
\define@key{names}{dga}{Central Dagaare}
\define@key{names}{dag}{Dagbani}
\define@key{names}{dta}{Dagur}
\define@key{names}{dal}{Dahalo}
\define@key{names}{daj}{Dar Fur Daju}
\define@key{names}{dak}{Dakota}
\define@key{names}{mbp}{Malayo}
\define@key{names}{dnj}{Dan}
\define@key{names}{daa}{Dangaleat}
\define@key{names}{dni}{Lower Grand Valley Dani}
\define@key{names}{dan}{Danish}
\define@key{names}{dry}{Darai}
\define@key{names}{dar}{North-Central Dargwa}
\define@key{names}{prs}{Dari}
\define@key{names}{drd}{Darma}
\define@key{names}{tcc}{Barabayiiga-Gisamjanga}
\define@key{names}{dai}{Day}
\define@key{names}{afn}{Defaka}
\define@key{names}{deg}{Degema}
\define@key{names}{ing}{Degexit'an}
\define@key{names}{dny}{Deni}
\define@key{names}{des}{Desano}
\define@key{names}{shg}{Shua}
\define@key{names}{der}{Deori}
\define@key{names}{gsg}{German Sign Language}
\define@key{names}{dsh}{Daasanach}
\define@key{names}{dhl}{Dhalandji}
\define@key{names}{tbh}{Thurawal}
\define@key{names}{dhr}{Dhargari}
\define@key{names}{xgm}{Dharumbal}
\define@key{names}{dhi}{Dhimal}
\define@key{names}{div}{Dhivehi}
\define@key{names}{dhu}{Dhurga}
\define@key{names}{did}{Didinga}
\define@key{names}{mhu}{Tawra}
\define@key{names}{dur}{Dii}
\define@key{names}{dis}{Dimasa}
\define@key{names}{dim}{Dime}
\define@key{names}{diz}{Ding}
\define@key{names}{din}{Dinka}
\define@key{names}{dyo}{Jola-Fonyi}
\define@key{names}{csk}{Jola-Esulalu}
\define@key{names}{dif}{Dieri}
\define@key{names}{mdx}{Dizin}
\define@key{names}{dyy}{Dyaabugay}
\define@key{names}{djr}{Djambarrpuyngu}
\define@key{names}{duj}{Dhuwal}
\define@key{names}{ddj}{Jaru}
\define@key{names}{dji}{Djinang}
\define@key{names}{jig}{Jingulu}
\define@key{names}{kbv}{Dera (Indonesia)}
\define@key{names}{kvo}{Dobel}
\define@key{names}{dgo}{Dogri}
\define@key{names}{dlg}{Dolgan}
\define@key{names}{dmk}{Domaaki}
\define@key{names}{rmt}{Domari}
\define@key{names}{kmc}{Southern Dong}
\define@key{names}{doo}{Dongo}
\define@key{names}{dds}{Donno So Dogon}
\define@key{names}{tds}{Doutai}
\define@key{names}{dow}{Doyayo}
\define@key{names}{dhv}{Dehu}
\define@key{names}{dua}{Duala}
\define@key{names}{dud}{Hun-Saare}
\define@key{names}{gwd}{Ale-Gawwada}
\define@key{names}{duu}{Drung}
\define@key{names}{dma}{Duma}
\define@key{names}{dgc}{Casiguran-Nagtipunan Agta}
\define@key{names}{dus}{Dumi}
\define@key{names}{vam}{Vanimo}
\define@key{names}{duc}{Duna}
\define@key{names}{nld}{Dutch}
\define@key{names}{zea}{Zeeuws}
\define@key{names}{dyi}{Djimini Senoufo}
\define@key{names}{dbl}{Dyirbal}
\define@key{names}{dyu}{Dyula}
\define@key{names}{kwa}{Dâw}
\define@key{names}{igb}{Ebira}
\define@key{names}{etr}{Edolo}
\define@key{names}{erk}{South Efate}
\define@key{names}{efi}{Efik}
\define@key{names}{ega}{Ega}
\define@key{names}{eip}{Eipomek}
\define@key{names}{etu}{Ejagham}
\define@key{names}{ekg}{Ekari}
\define@key{names}{eko}{Koti}
\define@key{names}{mrf}{Elseng}
\define@key{names}{ema}{Emai-Iuleha-Ora}
\define@key{names}{emb}{Embaloh}
\define@key{names}{cmi}{Emberá-Chamí}
\define@key{names}{emp}{Northern Emberá}
\define@key{names}{amy}{Ami}
\define@key{names}{enq}{Enga}
\define@key{names}{enn}{Egene}
\define@key{names}{eno}{Enggano}
\define@key{names}{eng}{English}
\define@key{names}{gey}{Enya}
\define@key{names}{sja}{Epena}
\define@key{names}{erg}{Sie}
\define@key{names}{ese}{Ese Ejja}
\define@key{names}{esq}{Esselen}
\define@key{names}{ekk}{Estonian}
\define@key{names}{ets}{Yekhee}
\define@key{names}{eve}{Even}
\define@key{names}{ewe}{Ewe}
\define@key{names}{ewo}{Ewondo}
\define@key{names}{eya}{Eyak}
\define@key{names}{fao}{Faroese}
\define@key{names}{faa}{Fasu}
\define@key{names}{fmp}{Fe'fe'}
\define@key{names}{fij}{Fijian}
\define@key{names}{fin}{Finnish}
\define@key{names}{fse}{Finnish Sign Language}
\define@key{names}{foi}{Foi}
\define@key{names}{ppo}{Folopa}
\define@key{names}{fon}{Fon}
\define@key{names}{frd}{Fordata}
\define@key{names}{for}{Fore}
\define@key{names}{sac}{Meskwaki}
\define@key{names}{fra}{French}
\define@key{names}{fry}{Western Frisian}
\define@key{names}{frs}{German Northern Low Saxon}
\define@key{names}{frr}{Northern Frisian}
\define@key{names}{fuh}{Western Niger Fulfulde}
\define@key{names}{fuf}{Pular}
\define@key{names}{fub}{Adamawa Fulfulde}
\define@key{names}{ffm}{Maasina Fulfulde}
\define@key{names}{fuv}{Hausa States Fulfulde}
\define@key{names}{fun}{Fulniô}
\define@key{names}{fvr}{Fur}
\define@key{names}{fud}{East Futuna}
\define@key{names}{fut}{Futuna-Aniwa}
\define@key{names}{cdo}{Min Dong Chinese}
\define@key{names}{pym}{Fyam}
\define@key{names}{gqa}{Ga'anda}
\define@key{names}{gbu}{Gaagudju}
\define@key{names}{dhg}{Dhangu}
\define@key{names}{gdb}{Pottangi Ollar Gadaba}
\define@key{names}{ged}{Gade}
\define@key{names}{gaj}{Gadsup}
\define@key{names}{gla}{Scottish Gaelic}
\define@key{names}{gag}{Gagauz}
\define@key{names}{gah}{Alekano}
\define@key{names}{gbi}{Galela}
\define@key{names}{glg}{Galician}
\define@key{names}{adl}{Galo}
\define@key{names}{kld}{Yuwaalaraay-Gamilaraay}
\define@key{names}{gmv}{Gamo}
\define@key{names}{pwg}{Gapapaiwa}
\define@key{names}{grt}{Garo}
\define@key{names}{wrk}{Garrwa}
\define@key{names}{gyb}{Garus}
\define@key{names}{cab}{Garifuna}
\define@key{names}{gvo}{Gavião Do Jiparaná}
\define@key{names}{gay}{Gayo}
\define@key{names}{gya}{Northwest Gbaya}
\define@key{names}{gso}{Southwest Gbaya}
\define@key{names}{gbp}{Gbaya-Bossangoa}
\define@key{names}{nlg}{Gela}
\define@key{names}{gqu}{Central Gelao-Qau}
\define@key{names}{kat}{Georgian}
\define@key{names}{deu}{German}
\define@key{names}{bar}{Bavarian}
\define@key{names}{ksh}{Kölsch}
\define@key{names}{wep}{Westphalic}
\define@key{names}{aaa}{Ghotuo}
\define@key{names}{ghl}{Uncunwee}
\define@key{names}{gih}{Condamine-Upper Clarence Bandjalang}
\define@key{names}{gid}{Gidar}
\define@key{names}{glk}{Gilaki}
\define@key{names}{bcq}{Bench}
\define@key{names}{git}{Gitxsan}
\define@key{names}{gis}{North Giziga}
\define@key{names}{guc}{Wayuu}
\define@key{names}{god}{Godié}
\define@key{names}{gdo}{Godoberi}
\define@key{names}{ank}{Goemai}
\define@key{names}{ggw}{Gogodala}
\define@key{names}{gju}{Gujari}
\define@key{names}{gkn}{Gokana}
\define@key{names}{gol}{Gola}
\define@key{names}{gvf}{Golin}
\define@key{names}{gno}{Northern Gondi}
\define@key{names}{gni}{Gooniyandi}
\define@key{names}{gor}{Gorontalo}
\define@key{names}{gow}{Gorowa}
\define@key{names}{grj}{Southern Grebo}
\define@key{names}{ell}{Modern Greek}
\define@key{names}{gss}{Greek Sign Language}
\define@key{names}{kal}{Kalaallisut}
\define@key{names}{guh}{Guahibo}
\define@key{names}{gub}{Guajajara}
\define@key{names}{gum}{Guambiano}
\define@key{names}{gva}{Guaná (Paraguay)}
\define@key{names}{gvc}{Kotiria}
\define@key{names}{gug}{Paraguayan Guaraní}
\define@key{names}{var}{Huarijio}
\define@key{names}{gta}{Guató}
\define@key{names}{guo}{Guayabero}
\define@key{names}{gde}{Gude}
\define@key{names}{gdf}{Guduf-Gava}
\define@key{names}{ktd}{Kokata}
\define@key{names}{ggd}{Gugadj}
\define@key{names}{ghs}{Guhu-Samane}
\define@key{names}{gcr}{Guianese Creole French}
\define@key{names}{pov}{Upper Guinea Crioulo}
\define@key{names}{guj}{Gujarati}
\define@key{names}{kcm}{Gula (Central African Republic)}
\define@key{names}{glj}{Gula Iro}
\define@key{names}{gnn}{Gumatj}
\define@key{names}{gvs}{Gumawana}
\define@key{names}{kgs}{Kumbainggar}
\define@key{names}{guk}{Northern Gumuz}
\define@key{names}{wlg}{Kunbarlang}
\define@key{names}{guw}{Gun}
\define@key{names}{gww}{Kwini}
\define@key{names}{yas}{Nugunu (Cameroon)}
\define@key{names}{gyy}{Gunya}
\define@key{names}{guf}{Gupapuyngu}
\define@key{names}{gnr}{Gureng Gureng}
\define@key{names}{gur}{Farefare}
\define@key{names}{gue}{Gurindji}
\define@key{names}{gux}{Gourmanchéma}
\define@key{names}{goa}{Guro}
\define@key{names}{gge}{Guragone}
\define@key{names}{guz}{Gusii}
\define@key{names}{gbj}{Bodo Gadaba}
\define@key{names}{kky}{Guugu Yimidhirr}
\define@key{names}{gbr}{Gbagyi}
\define@key{names}{kcg}{Tyap}
\define@key{names}{gaa}{Ga}
\define@key{names}{pue}{Puelche}
\define@key{names}{hts}{Hadza}
\define@key{names}{hai}{Haida}
\define@key{names}{hdn}{Northern Haida}
\define@key{names}{has}{Haisla}
\define@key{names}{hat}{Haitian}
\define@key{names}{hak}{Hakka Chinese}
\define@key{names}{hal}{Halang}
\define@key{names}{hlb}{Halbi}
\define@key{names}{hla}{Halia}
\define@key{names}{amf}{Hamer-Banna}
\define@key{names}{hmt}{Hamtai}
\define@key{names}{wos}{Hanga Hundi}
\define@key{names}{hni}{Hani}
\define@key{names}{hnn}{Hanunoo}
\define@key{names}{har}{Harari}
\define@key{names}{hss}{Harsusi}
\define@key{names}{tmd}{Haruai}
\define@key{names}{had}{Hatam}
\define@key{names}{hau}{Hausa}
\define@key{names}{haw}{Hawaiian}
\define@key{names}{hwc}{Hawai'i Creole English}
\define@key{names}{hac}{Gurani}
\define@key{names}{hay}{Haya}
\define@key{names}{vay}{Wayu}
\define@key{names}{xed}{Hdi}
\define@key{names}{heb}{Modern Hebrew}
\define@key{names}{heh}{Hehe}
\define@key{names}{hei}{Heiltsuk-Oowekyala}
\define@key{names}{hem}{Hemba-Yazi}
\define@key{names}{her}{Herero}
\define@key{names}{hid}{Hidatsa}
\define@key{names}{hil}{Hiligaynon}
\define@key{names}{hin}{Hindi}
\define@key{names}{gin}{Hinuq}
\define@key{names}{hix}{Hixkaryána}
\define@key{names}{lic}{Hlai}
\define@key{names}{hmr}{Hmar}
\define@key{names}{mww}{Hmong Daw}
\define@key{names}{hnj}{Hmong Njua}
\define@key{names}{hoc}{Ho}
\define@key{names}{hoa}{Hoava}
\define@key{names}{hoo}{Holoholo}
\define@key{names}{hks}{Hong Kong-Macau Sign Language}
\define@key{names}{hop}{Hopi}
\define@key{names}{hre}{Hre}
\define@key{names}{ygr}{Yagaria}
\define@key{names}{hub}{Huambisa}
\define@key{names}{hus}{Huastec}
\define@key{names}{huv}{San Mateo del Mar Huave}
\define@key{names}{hch}{Huichol}
\define@key{names}{hto}{Minica Huitoto}
\define@key{names}{hux}{Nüpode Huitoto}
\define@key{names}{huu}{Murui Huitoto}
\define@key{names}{hke}{Hunde}
\define@key{names}{hun}{Hungarian}
\define@key{names}{huz}{Hunzib}
\define@key{names}{jup}{Hup}
\define@key{names}{hup}{Hupa-Chilula}
\define@key{names}{csh}{Asho Chin}
\define@key{names}{ksi}{I'saka}
\define@key{names}{iai}{Iaai}
\define@key{names}{ian}{Iatmul}
\define@key{names}{tmu}{Iau}
\define@key{names}{iba}{Iban}
\define@key{names}{ibg}{Ibanag}
\define@key{names}{ibb}{Ibibio}
\define@key{names}{isl}{Icelandic}
\define@key{names}{icl}{Icelandic Sign Language}
\define@key{names}{idu}{Idoma}
\define@key{names}{clk}{Idu}
\define@key{names}{viv}{Iduna}
\define@key{names}{mxe}{Mele-Fila}
\define@key{names}{ifb}{Batad Ifugao}
\define@key{names}{ifm}{Teke-Fuumu}
\define@key{names}{ibo}{Igbo}
\define@key{names}{ige}{Igede}
\define@key{names}{ign}{Ignaciano}
\define@key{names}{ihp}{Iha}
\define@key{names}{ijc}{Izon}
\define@key{names}{ikx}{Ik}
\define@key{names}{arh}{Arhuaco}
\define@key{names}{ilb}{Ila}
\define@key{names}{mia}{Miami}
\define@key{names}{ilo}{Iloko}
\define@key{names}{imn}{Imonda}
\define@key{names}{szp}{Suabo}
\define@key{names}{ins}{Indian Sign Language}
\define@key{names}{pks}{Pakistan Sign Language}
\define@key{names}{ind}{Standard Indonesian}
\define@key{names}{pmy}{Papuan Malay}
\define@key{names}{inb}{Colombian Inga}
\define@key{names}{tbi}{Gaam}
\define@key{names}{inh}{Ingush}
\define@key{names}{ynd}{Yandruwandha}
\define@key{names}{ils}{International Sign}
\define@key{names}{ike}{Eastern Canadian Inuktitut}
\define@key{names}{iqu}{Iquito}
\define@key{names}{irn}{Irántxe-Münkü}
\define@key{names}{irk}{Iraqw}
\define@key{names}{irh}{Irarutu}
\define@key{names}{gle}{Irish}
\define@key{names}{isg}{Irish Sign Language}
\define@key{names}{its}{Isekiri}
\define@key{names}{isk}{Ishkashimi}
\define@key{names}{srl}{Isirawa}
\define@key{names}{isd}{Isnag}
\define@key{names}{iso}{Isoko}
\define@key{names}{isr}{Israeli Sign Language}
\define@key{names}{ita}{Italian}
\define@key{names}{egl}{Emiliano}
\define@key{names}{lij}{Ligurian}
\define@key{names}{nap}{Continental Southern Italian}
\define@key{names}{pms}{Piemontese}
\define@key{names}{itv}{Itawit}
\define@key{names}{itl}{West Itelmen}
\define@key{names}{ito}{Itonama}
\define@key{names}{itz}{Itzá}
\define@key{names}{ivb}{Ibatan}
\define@key{names}{ibd}{Iwaidja}
\define@key{names}{iwm}{Iwam}
\define@key{names}{yom}{Yombe}
\define@key{names}{ixc}{Ixcatec}
\define@key{names}{ixl}{Ixil}
\define@key{names}{izr}{Izere}
\define@key{names}{izh}{Ingrian}
\define@key{names}{izz}{Izi}
\define@key{names}{esi}{North Alaskan Inupiatun}
\define@key{names}{jbt}{Djeoromitxí}
\define@key{names}{jae}{Yabem}
\define@key{names}{jda}{Jad}
\define@key{names}{jhi}{Jehai}
\define@key{names}{jac}{Popti'}
\define@key{names}{jam}{Jamaican Creole English}
\define@key{names}{djd}{Jaminjung-Ngaliwurru}
\define@key{names}{djm}{Jamsay Dogon}
\define@key{names}{jpn}{Japanese}
\define@key{names}{jru}{Japrería}
\define@key{names}{jqr}{Jaqaru}
\define@key{names}{anq}{Jarawa (India)}
\define@key{names}{jav}{Javanese}
\define@key{names}{jeb}{Jebero}
\define@key{names}{jeh}{Jeh}
\define@key{names}{jek}{Jeli}
\define@key{names}{tow}{Towa}
\define@key{names}{jya}{Jiarong}
\define@key{names}{shv}{Jibbali}
\define@key{names}{kac}{Southern Jinghpaw}
\define@key{names}{jiu}{Youle Jinuo}
\define@key{names}{jiv}{Shuar}
\define@key{names}{rgk}{Rangkas}
\define@key{names}{tlo}{Tasomi-Tata}
\define@key{names}{jun}{Juang}
\define@key{names}{nst}{Pangwa Naga}
\define@key{names}{jbu}{Jukun Takum}
\define@key{names}{bex}{Jur Modo}
\define@key{names}{juc}{Jurchen}
\define@key{names}{jur}{Jurúna}
\define@key{names}{ktz}{South-Eastern Ju}
\define@key{names}{jua}{Júma}
\define@key{names}{kek}{Kekchí}
\define@key{names}{kbd}{Kabardian}
\define@key{names}{xkp}{Kabatei}
\define@key{names}{kbp}{Kabiyé}
\define@key{names}{nbu}{Rongmei Naga}
\define@key{names}{kab}{Kabyle}
\define@key{names}{xac}{Kachari}
\define@key{names}{kzj}{Kadazan}
\define@key{names}{kbc}{Kadiwéu}
\define@key{names}{kdm}{Kagoma}
\define@key{names}{kki}{Kagulu}
\define@key{names}{kct}{Kaian}
\define@key{names}{lew}{Ledo Kaili}
\define@key{names}{kgp}{Kaingang}
\define@key{names}{kxa}{Kairiru}
\define@key{names}{kgk}{Kaiwá}
\define@key{names}{tbd}{Kaki Ae}
\define@key{names}{mwp}{Kala Lagaw Ya}
\define@key{names}{kmh}{Kalam}
\define@key{names}{gwc}{Gawri}
\define@key{names}{kck}{Kalanga}
\define@key{names}{kyl}{Central Kalapuya}
\define@key{names}{kls}{Chitral Kalasha}
\define@key{names}{fla}{Kalispel-Pend d'Oreille}
\define@key{names}{ktg}{Kalkutung}
\define@key{names}{bco}{Kaluli}
\define@key{names}{kay}{Kamayurá}
\define@key{names}{kbq}{Kamano}
\define@key{names}{kms}{Kamasau}
\define@key{names}{xas}{Kamas-Koibal}
\define@key{names}{kam}{Kamba (Kenya)}
\define@key{names}{xbr}{Kambera}
\define@key{names}{kbx}{Ap Ma}
\define@key{names}{kcu}{Kami (Tanzania)}
\define@key{names}{kgq}{Kamoro}
\define@key{names}{xmu}{Kamu}
\define@key{names}{ogo}{Khana}
\define@key{names}{kna}{Dera (Nigeria)}
\define@key{names}{xns}{Kanashi}
\define@key{names}{kbl}{Kanembu}
\define@key{names}{ikt}{Western Canadian Inuktitut}
\define@key{names}{kjb}{Q'anjob'al}
\define@key{names}{knj}{Akateko}
\define@key{names}{kne}{Kankanaey}
\define@key{names}{kan}{Kannada}
\define@key{names}{kxo}{Kanoê}
\define@key{names}{khd}{Ngkontar Baedi}
\define@key{names}{kcd}{Ngkontar Ngkolmpu}
\define@key{names}{knc}{Central Kanuri}
\define@key{names}{kny}{Kanyok}
\define@key{names}{pam}{Pampanga}
\define@key{names}{kpg}{Kapingamarangi}
\define@key{names}{kah}{Kara (Central African Republic)}
\define@key{names}{leu}{Kara (Papua New Guinea)}
\define@key{names}{krc}{Karachay-Balkar}
\define@key{names}{gbd}{Karadjeri}
\define@key{names}{kdr}{Karaim}
\define@key{names}{kpj}{Karajá}
\define@key{names}{kaa}{Kara-Kalpak}
\define@key{names}{zkk}{Karankawa}
\define@key{names}{kyj}{Karao}
\define@key{names}{kpt}{Karata-Tukita}
\define@key{names}{krl}{Karelian}
\define@key{names}{bwe}{Bwe Karen}
\define@key{names}{kjp}{Pwo Eastern Karen}
\define@key{names}{ksw}{S'gaw Karen}
\define@key{names}{vka}{Kariyarra}
\define@key{names}{kdj}{Karamojong}
\define@key{names}{ktn}{Karitiâna}
\define@key{names}{yuj}{Karkar-Yuri}
\define@key{names}{kyh}{Karok}
\define@key{names}{arr}{Karo (Brazil)}
\define@key{names}{xsm}{Kasem}
\define@key{names}{kju}{Kashaya}
\define@key{names}{kas}{Kashmiri}
\define@key{names}{csb}{Kashubian}
\define@key{names}{cog}{Chong of Chanthaburi}
\define@key{names}{bqy}{Kata Kolok}
\define@key{names}{xtc}{Katcha-Kadugli-Miri}
\define@key{names}{bsh}{Katë}
\define@key{names}{kts}{South Muyu}
\define@key{names}{kcr}{Katla}
\define@key{names}{ktw}{Kato}
\define@key{names}{pss}{Kaulong}
\define@key{names}{bpp}{Kaure-Narau}
\define@key{names}{zku}{Kaurna}
\define@key{names}{xaw}{Kawaiisu}
\define@key{names}{kyz}{Kayabí}
\define@key{names}{eky}{Eastern Kayah}
\define@key{names}{kys}{Baram Kayan}
\define@key{names}{txu}{Kayapó}
\define@key{names}{gyd}{Kayardild}
\define@key{names}{gbb}{Kaytetye}
\define@key{names}{kaz}{Kazakh}
\define@key{names}{ksx}{Kedang}
\define@key{names}{kbr}{Kafa}
\define@key{names}{kei}{Kei}
\define@key{names}{kcl}{Kela (Papua New Guinea)}
\define@key{names}{kzi}{Kelabit}
\define@key{names}{sbc}{Kele (Papua New Guinea)}
\define@key{names}{ahg}{Qimant}
\define@key{names}{kmt}{Kemtuik}
\define@key{names}{kyq}{Kenga}
\define@key{names}{keu}{Akebu}
\define@key{names}{xki}{Kenya-Somali Sign Language}
\define@key{names}{ken}{Kenyang}
\define@key{names}{xxk}{Kéo}
\define@key{names}{ker}{Kera}
\define@key{names}{krk}{Kerek}
\define@key{names}{kee}{Eastern Keres}
\define@key{names}{ket}{Ket}
\define@key{names}{xdy}{Malayic Dayak}
\define@key{names}{kcv}{Kete}
\define@key{names}{xte}{Ketengban}
\define@key{names}{kew}{West Kewa}
\define@key{names}{kjh}{Khakas}
\define@key{names}{klj}{Turkic Khalaj}
\define@key{names}{klr}{Khaling}
\define@key{names}{khk}{Halh Mongolian}
\define@key{names}{kjl}{Western Parbate Kham}
\define@key{names}{khg}{Khams Tibetan}
\define@key{names}{kca}{Kazym-Berezover-Suryskarer Khanty}
\define@key{names}{khr}{Kharia}
\define@key{names}{kha}{Khasi}
\define@key{names}{kjj}{Khinalug}
\define@key{names}{khm}{Central Khmer}
\define@key{names}{kjg}{Khmu}
\define@key{names}{khw}{Khowar}
\define@key{names}{cnk}{Khumi Chin}
\define@key{names}{khv}{Khwarshi-Inkhoqwari}
\define@key{names}{kkh}{Khün}
\define@key{names}{kic}{Kickapoo}
\define@key{names}{kik}{Kikuyu}
\define@key{names}{hbb}{Huba}
\define@key{names}{kij}{Kilivila}
\define@key{names}{klb}{Kiliwa}
\define@key{names}{lub}{Luba-Katanga}
\define@key{names}{kig}{Kimaama}
\define@key{names}{zga}{Kinga}
\define@key{names}{kfk}{Kinnauri}
\define@key{names}{kin}{Kinyarwanda}
\define@key{names}{kio}{Kiowa}
\define@key{names}{kzw}{Karirí-Xocó}
\define@key{names}{geb}{Kire}
\define@key{names}{kir}{Kirghiz}
\define@key{names}{gil}{Gilbertese}
\define@key{names}{kiy}{Kirikiri}
\define@key{names}{cme}{Cerma}
\define@key{names}{kje}{Kisar}
\define@key{names}{kss}{Southern Kisi}
\define@key{names}{gia}{Kitja}
\define@key{names}{kii}{Kitsai}
\define@key{names}{ktu}{Kituba (Democratic Republic of Congo)}
\define@key{names}{kjd}{Southern Kiwai}
\define@key{names}{kla}{Klamath-Modoc}
\define@key{names}{klu}{Klao}
\define@key{names}{yak}{Northwest Sahaptin}
\define@key{names}{kst}{Winyé}
\define@key{names}{cku}{Koasati}
\define@key{names}{kpw}{Kobon}
\define@key{names}{kfa}{Kodava}
\define@key{names}{xwg}{Kwegu}
\define@key{names}{xuo}{Kuo}
\define@key{names}{bcs}{Kohumono}
\define@key{names}{kpx}{Mountain Koiali}
\define@key{names}{kbk}{Grass Koiari}
\define@key{names}{kqi}{Koitabu}
\define@key{names}{trp}{Kok Borok}
\define@key{names}{kex}{Kokni}
\define@key{names}{kkk}{Kokota}
\define@key{names}{kvv}{Kola}
\define@key{names}{kfb}{Northwestern Kolami}
\define@key{names}{kvw}{Wersing}
\define@key{names}{shm}{Shahrudi-Southern Talysh}
\define@key{names}{bkm}{Kom (Cameroon)}
\define@key{names}{xbi}{Kombio}
\define@key{names}{kge}{Komering}
\define@key{names}{koi}{Komi-Permyak}
\define@key{names}{xom}{Komo (Sudan-Ethiopia)}
\define@key{names}{kfc}{Konda-Dora}
\define@key{names}{kng}{South-Central Koongo}
\define@key{names}{kjc}{Coastal Konjo}
\define@key{names}{knn}{Konkan Marathi}
\define@key{names}{xon}{Konkomba}
\define@key{names}{mjd}{Northwest Maidu}
\define@key{names}{kma}{Konni}
\define@key{names}{kyx}{Rapoisi}
\define@key{names}{cou}{Wamey}
\define@key{names}{kqy}{Koorete}
\define@key{names}{kpr}{Korafe-Yegha}
\define@key{names}{kqz}{Korana}
\define@key{names}{knk}{Kuranko}
\define@key{names}{kor}{Korean}
\define@key{names}{coe}{Koreguaje}
\define@key{names}{kfq}{Korku}
\define@key{names}{kfz}{Koromfé}
\define@key{names}{khe}{Korowai}
\define@key{names}{kpy}{Koryak}
\define@key{names}{kia}{Kim}
\define@key{names}{kos}{Kosraean}
\define@key{names}{kfe}{Kota (India)}
\define@key{names}{aal}{Afade}
\define@key{names}{kff}{Koya}
\define@key{names}{khq}{Koyra Chiini Songhay}
\define@key{names}{ses}{Koyraboro Senni Songhai}
\define@key{names}{koy}{Koyukon}
\define@key{names}{kpk}{Kpan}
\define@key{names}{xpe}{Liberia Kpelle}
\define@key{names}{kpo}{Ikposo}
\define@key{names}{xra}{Krahô}
\define@key{names}{kqq}{Krenak}
\define@key{names}{krs}{Kresh-Woro}
\define@key{names}{rop}{Kriol}
\define@key{names}{kgo}{Krongo}
\define@key{names}{jct}{Krymchak}
\define@key{names}{kry}{Kryz}
\define@key{names}{puo}{Ksingmul}
\define@key{names}{sdm}{Onya Darat}
\define@key{names}{uwa}{Kuku-Uwanh}
\define@key{names}{kxu}{Kui (India)}
\define@key{names}{kvd}{Kui (Indonesia)}
\define@key{names}{kui}{Kuikúro-Kalapálo}
\define@key{names}{gvn}{Kuku-Yalanji}
\define@key{names}{mbt}{Matigsalug Manobo}
\define@key{names}{dwr}{Dawro}
\define@key{names}{kle}{Kulung (Nepal)}
\define@key{names}{kue}{Kuman}
\define@key{names}{kfy}{Kumaoni}
\define@key{names}{kum}{Kumyk}
\define@key{names}{kvn}{Border Kuna}
\define@key{names}{kun}{Kunama}
\define@key{names}{kup}{Kunimaipa}
\define@key{names}{kjn}{Kunjen}
\define@key{names}{cmn}{Mandarin Chinese}
\define@key{names}{kto}{Kuot}
\define@key{names}{ckb}{Central Kurdish}
\define@key{names}{kmr}{Northern Kurdish}
\define@key{names}{kru}{Kurukh}
\define@key{names}{kgg}{Kusunda}
\define@key{names}{vkt}{Tenggarong Kutai Malay}
\define@key{names}{gwi}{Gwich'in}
\define@key{names}{kut}{Kutenai}
\define@key{names}{thd}{Thayore}
\define@key{names}{kuy}{Kuuku-Ya'u}
\define@key{names}{kxv}{Kuvi}
\define@key{names}{kwd}{Kwaio}
\define@key{names}{kwk}{Kwak'wala}
\define@key{names}{tnk}{Kwamera}
\define@key{names}{ksq}{Kwaami}
\define@key{names}{kwn}{Kwangali}
\define@key{names}{xwa}{Kwaza}
\define@key{names}{kwe}{Kwerba}
\define@key{names}{kmo}{Kwoma}
\define@key{names}{kwo}{Kwomtari}
\define@key{names}{xuu}{Kxoe}
\define@key{names}{kyc}{Kyaka}
\define@key{names}{kgy}{Kyerung}
\define@key{names}{nuk}{Nuu-chah-nulth}
\define@key{names}{kmg}{Kâte}
\define@key{names}{gdm}{Laal}
\define@key{names}{lbu}{Labu}
\define@key{names}{lac}{Lacandon}
\define@key{names}{lbt}{Lachi}
\define@key{names}{lbj}{Leh Ladakhi}
\define@key{names}{lld}{Ladin}
\define@key{names}{lad}{Ladino}
\define@key{names}{laf}{Lafofa}
\define@key{names}{kot}{Lagwan}
\define@key{names}{lha}{Laha (Viet Nam)}
\define@key{names}{lhu}{Lahu}
\define@key{names}{cnh}{Haka Chin}
\define@key{names}{lbe}{Lak}
\define@key{names}{lkt}{Lakota}
\define@key{names}{lbc}{Lakkia}
\define@key{names}{ywt}{Xishanba Lalo}
\define@key{names}{slp}{Lamaholot}
\define@key{names}{hia}{Lamang}
\define@key{names}{lmn}{Lambadi}
\define@key{names}{lam}{Lamba}
\define@key{names}{lmu}{Lamenu}
\define@key{names}{lns}{Lamnso'}
\define@key{names}{ljp}{Lampung Api}
\define@key{names}{lby}{Lamalama}
\define@key{names}{lme}{Peve}
\define@key{names}{lag}{Langi}
\define@key{names}{laj}{Lango (Uganda)}
\define@key{names}{fsl}{French Sign Language}
\define@key{names}{fcs}{Quebec Sign Language}
\define@key{names}{lao}{Lao}
\define@key{names}{lrg}{Laragia}
\define@key{names}{lbz}{Lardil}
\define@key{names}{alo}{Larike-Wakasihu}
\define@key{names}{lav}{Latvian}
\define@key{names}{llu}{Lau}
\define@key{names}{law}{Lauje}
\define@key{names}{lvk}{Lavukaleve}
\define@key{names}{lzz}{Laz}
\define@key{names}{agh}{Ngelima}
\define@key{names}{lea}{Lega-Shabunda}
\define@key{names}{agb}{Legbo}
\define@key{names}{lec}{Leco}
\define@key{names}{lln}{Lele (Chad)}
\define@key{names}{lef}{Lelemi}
\define@key{names}{tnl}{Lenakel}
\define@key{names}{led}{Lendu}
\define@key{names}{enx}{Enxet Sur}
\define@key{names}{aed}{Argentine Sign Language}
\define@key{names}{ssp}{Spanish Sign Language}
\define@key{names}{lep}{Lepcha}
\define@key{names}{les}{Lese}
\define@key{names}{lti}{Leti (Indonesia)}
\define@key{names}{lww}{Lewo}
\define@key{names}{lez}{Lezgian}
\define@key{names}{lhm}{Lhomi}
\define@key{names}{lil}{Lillooet}
\define@key{names}{lif}{Limbu}
\define@key{names}{lmc}{Limilngan}
\define@key{names}{liy}{Banda-Bambari}
\define@key{names}{lin}{Kinshasa Lingala}
\define@key{names}{ise}{Italian Sign Language}
\define@key{names}{lnj}{Linngithigh}
\define@key{names}{lis}{Lisu}
\define@key{names}{lit}{Lithuanian}
\define@key{names}{liv}{Liv}
\define@key{names}{lob}{Lobi}
\define@key{names}{log}{Logo}
\define@key{names}{lok}{Loko}
\define@key{names}{arw}{Lokono}
\define@key{names}{lom}{Loma (Liberia)}
\define@key{names}{bdu}{Oroko}
\define@key{names}{lgu}{Longgu}
\define@key{names}{los}{Loniu}
\define@key{names}{crc}{Lonwolwol}
\define@key{names}{njh}{Lotha Naga}
\define@key{names}{loj}{Lou}
\define@key{names}{lbo}{Laven}
\define@key{names}{nds}{Eastern Low German}
\define@key{names}{loz}{Lozi}
\define@key{names}{nie}{Niellim}
\define@key{names}{ojv}{Luangiua}
\define@key{names}{lch}{Luchazi}
\define@key{names}{lug}{Ganda}
\define@key{names}{lgg}{Lugbara}
\define@key{names}{jos}{Levantine Arabic Sign Language}
\define@key{names}{lui}{Luiseno-Juaneño}
\define@key{names}{ule}{Lule}
\define@key{names}{str}{Northern Straits Salish}
\define@key{names}{lnd}{Lundayeh}
\define@key{names}{lun}{Lunda}
\define@key{names}{luo}{Luo (Kenya and Tanzania)}
\define@key{names}{lrc}{Northern Luri}
\define@key{names}{lut}{Northern Lushootseed}
\define@key{names}{khl}{Lusi}
\define@key{names}{lue}{Luvale}
\define@key{names}{lwo}{Luwo}
\define@key{names}{ltz}{Moselle Franconian}
\define@key{names}{luy}{Luyia}
\define@key{names}{lee}{Lyélé}
\define@key{names}{psr}{Portuguese Sign Language}
\define@key{names}{bzs}{Brazilian Sign Language}
\define@key{names}{khb}{Lü}
\define@key{names}{msj}{Ma (Democratic Republic of Congo)}
\define@key{names}{mhy}{Ma'anyan}
\define@key{names}{mhi}{Ma'di}
\define@key{names}{slz}{Misool-Salawati Ma'ya}
\define@key{names}{mdy}{Male (Ethiopia)}
\define@key{names}{mas}{Masai}
\define@key{names}{mde}{Maba (Chad)}
\define@key{names}{mca}{Maca}
\define@key{names}{mbn}{Macaguán}
\define@key{names}{mkd}{Macedonian}
\define@key{names}{mcb}{Machiguenga}
\define@key{names}{myy}{Macuna}
\define@key{names}{mbc}{Macushi}
\define@key{names}{mxu}{Mada (Cameroon)}
\define@key{names}{mda}{Mada (Nigeria)}
\define@key{names}{dmd}{Madimadi}
\define@key{names}{mad}{Madurese}
\define@key{names}{mmw}{Emae}
\define@key{names}{mag}{Magahi}
\define@key{names}{mgp}{Eastern Magar}
\define@key{names}{mrd}{Western Magar}
\define@key{names}{mgu}{Mailu}
\define@key{names}{mdh}{Maguindanao}
\define@key{names}{mhe}{Besisi}
\define@key{names}{xpq}{Mahican}
\define@key{names}{nmu}{Northeast Maidu}
\define@key{names}{zrs}{Mairasi}
\define@key{names}{mbq}{Maisin}
\define@key{names}{mai}{Maithili}
\define@key{names}{mpe}{Majang}
\define@key{names}{mcp}{Makaa}
\define@key{names}{myh}{Makah}
\define@key{names}{mkz}{Makasae-Makalero}
\define@key{names}{mak}{Makasar}
\define@key{names}{mgf}{Maklew}
\define@key{names}{kde}{Makonde}
\define@key{names}{mgh}{Makhuwa-Meetto}
\define@key{names}{mcm}{Malacca-Batavia Portuguese Creole}
\define@key{names}{plt}{Plateau Malagasy}
\define@key{names}{mpb}{Mullukmulluk}
\define@key{names}{zsm}{Standard Malay}
\define@key{names}{zlm}{Central Malay}
\define@key{names}{zmi}{Negeri Sembilan Malay}
\define@key{names}{mal}{Malayalam}
\define@key{names}{mgl}{Maleu-Kilenge}
\define@key{names}{gcc}{Mali}
\define@key{names}{mlt}{Maltese}
\define@key{names}{kmj}{Kumarbhag Paharia}
\define@key{names}{mam}{Mam}
\define@key{names}{mmn}{Mamanwa}
\define@key{names}{mqj}{Mamasa}
\define@key{names}{mcs}{Mambai}
\define@key{names}{mgr}{Mambwe-Lungu}
\define@key{names}{maw}{Mampruli}
\define@key{names}{mdi}{Mamvu}
\define@key{names}{xmm}{Manado Malay}
\define@key{names}{mva}{Manam}
\define@key{names}{mle}{Manambu}
\define@key{names}{nmm}{Manange}
\define@key{names}{mnc}{Manchu}
\define@key{names}{mid}{Neo-Mandaic}
\define@key{names}{mhq}{Mandan}
\define@key{names}{mdr}{Mandar}
\define@key{names}{mnk}{Mandinka}
\define@key{names}{jet}{Manem}
\define@key{names}{mna}{Mbula}
\define@key{names}{mpc}{Mangarrayi}
\define@key{names}{mdj}{Mangbetu}
\define@key{names}{mqy}{Manggarai}
\define@key{names}{mjg}{Mongghul}
\define@key{names}{mge}{Mango}
\define@key{names}{emk}{Eastern Maninkakan}
\define@key{names}{mlq}{Western Maninkakan}
\define@key{names}{mfv}{Mandjak}
\define@key{names}{knf}{Mankanya}
\define@key{names}{nge}{Ngemba}
\define@key{names}{mev}{Mann}
\define@key{names}{mbb}{Western Bukidnon Manobo}
\define@key{names}{mns}{Northern Mansi}
\define@key{names}{glv}{Manx}
\define@key{names}{mri}{Maori}
\define@key{names}{mcg}{Mapoyo}
\define@key{names}{arn}{Mapudungun}
\define@key{names}{mec}{Marra}
\define@key{names}{mrw}{Maranao}
\define@key{names}{zmr}{Maranunggu}
\define@key{names}{mar}{Marathi}
\define@key{names}{rnp}{Rongpo}
\define@key{names}{zmc}{Margany}
\define@key{names}{mrt}{Marghi Central}
\define@key{names}{mrj}{Western Mari}
\define@key{names}{mhr}{Eastern Mari}
\define@key{names}{mrc}{Maricopa}
\define@key{names}{mrz}{Marind}
\define@key{names}{mbw}{Maring}
\define@key{names}{zmt}{Maringarr}
\define@key{names}{mfr}{Marithiel}
\define@key{names}{mah}{Marshallese}
\define@key{names}{gcf}{Guadeloupe-Martinique Creole French}
\define@key{names}{vma}{Martuthunira}
\define@key{names}{mhx}{Maru}
\define@key{names}{mcn}{Masana}
\define@key{names}{jle}{Ngile}
\define@key{names}{mls}{Masalit}
\define@key{names}{wam}{Wampanoag}
\define@key{names}{mpq}{Matís}
\define@key{names}{zml}{Madngele}
\define@key{names}{mcf}{Matsés}
\define@key{names}{mvb}{Mattole-Bear River}
\define@key{names}{mjk}{Matukar}
\define@key{names}{mgw}{Matumbi}
\define@key{names}{mxx}{Mahou}
\define@key{names}{mph}{Mawng}
\define@key{names}{mfe}{Morisyen}
\define@key{names}{mke}{Mawchi}
\define@key{names}{mbl}{Maxakalí}
\define@key{names}{yan}{Mayangna}
\define@key{names}{ayz}{Maybrat-Karon}
\define@key{names}{xyj}{Mayi-Yapi}
\define@key{names}{mfy}{Mayo}
\define@key{names}{mdm}{Mayogo}
\define@key{names}{maz}{Central Mazahua}
\define@key{names}{mzn}{Mazanderani}
\define@key{names}{maq}{Chiquihuitlán Mazatec}
\define@key{names}{mau}{Huautla Mazatec}
\define@key{names}{mfc}{Mba}
\define@key{names}{vmb}{Mbabaram}
\define@key{names}{lnb}{Central Wambo}
\define@key{names}{mpk}{Mbara (Chad)}
\define@key{names}{myb}{Mbay}
\define@key{names}{mtk}{Mbe'}
\define@key{names}{mdt}{Mbere-Mbamba}
\define@key{names}{baw}{Bambili-Bambui}
\define@key{names}{gmm}{Gbaya-Mbodomo}
\define@key{names}{mdq}{Mbole}
\define@key{names}{mdw}{Mbosi}
\define@key{names}{mhd}{Mbugu}
\define@key{names}{mdd}{Mbum}
\define@key{names}{mym}{Me'en}
\define@key{names}{nux}{Mehek}
\define@key{names}{gdq}{Mehri}
\define@key{names}{mni}{Manipuri}
\define@key{names}{skf}{Mekens}
\define@key{names}{mek}{Mekeo}
\define@key{names}{mel}{Central Melanau}
\define@key{names}{bew}{Betawi}
\define@key{names}{men}{Mende (Sierra Leone)}
\define@key{names}{mez}{Menominee}
\define@key{names}{mwv}{Mentawai}
\define@key{names}{sdo}{Bukar-Sadung Bidayuh}
\define@key{names}{mcr}{Menya}
\define@key{names}{ulk}{Meriam}
\define@key{names}{mej}{Meyah}
\define@key{names}{mpt}{Mian}
\define@key{names}{crg}{Michif}
\define@key{names}{mic}{Mi'kmaq}
\define@key{names}{mei}{Midob}
\define@key{names}{ium}{Iu Mien}
\define@key{names}{mmy}{Migaama}
\define@key{names}{mxj}{Kman}
\define@key{names}{msy}{Aruamu}
\define@key{names}{mik}{Mikasuki}
\define@key{names}{mjw}{Hills Karbi}
\define@key{names}{hna}{Mina (Cameroon)}
\define@key{names}{min}{Minangkabau}
\define@key{names}{mvn}{Minaveha}
\define@key{names}{xmf}{Mingrelian}
\define@key{names}{mep}{Miriwung}
\define@key{names}{nju}{Ngadjunmaya}
\define@key{names}{mrg}{Mising-Padam-Miri-Minyong}
\define@key{names}{miq}{Mískito}
\define@key{names}{zmq}{Mituku}
\define@key{names}{csi}{Coast Miwok}
\define@key{names}{csm}{Central Sierra Miwok}
\define@key{names}{lmw}{Lake Miwok}
\define@key{names}{nsq}{Northern Sierra Miwok}
\define@key{names}{pmw}{Plains Miwok}
\define@key{names}{skd}{Southern Sierra Miwok}
\define@key{names}{mxp}{Tlahuitoltepec Mixe}
\define@key{names}{mco}{Coatlán Mixe}
\define@key{names}{mto}{Totontepec Mixe}
\define@key{names}{mim}{Alacatlatzala Mixtec}
\define@key{names}{mib}{Atatláhuca Mixtec}
\define@key{names}{miy}{Ayutla Mixtec}
\define@key{names}{mih}{Chayuco Mixtec}
\define@key{names}{miz}{Coatzospan Mixtec}
\define@key{names}{mxt}{Jamiltepec Mixtec}
\define@key{names}{mio}{Pinotepa Nacional Mixtec}
\define@key{names}{mig}{San Miguel El Grande Mixtec}
\define@key{names}{mie}{Ocotepec Mixtec}
\define@key{names}{mil}{Peñoles Mixtec}
\define@key{names}{mjc}{San Juan Colorado Mixtec}
\define@key{names}{mks}{Silacayoapan Mixtec}
\define@key{names}{mpm}{Yosondúa Mixtec}
\define@key{names}{mkf}{Miya}
\define@key{names}{lus}{Mizo}
\define@key{names}{mra}{Mlabri}
\define@key{names}{moy}{Shekkacho}
\define@key{names}{omc}{Mochica}
\define@key{names}{moc}{Mocoví}
\define@key{names}{mif}{Mofu-Gudur}
\define@key{names}{mhj}{Mogholi}
\define@key{names}{moh}{Mohawk}
\define@key{names}{mov}{Mohave}
\define@key{names}{mkj}{Mokilese}
\define@key{names}{moz}{Mukulu}
\define@key{names}{mbe}{Molale}
\define@key{names}{mso}{Mombum}
\define@key{names}{fqs}{Momu-Fas}
\define@key{names}{mqf}{Momuna}
\define@key{names}{mnw}{Mon}
\define@key{names}{ndt}{Ndunga}
\define@key{names}{lol}{Mongo}
\define@key{names}{mog}{Mongondow}
\define@key{names}{mnz}{Moni}
\define@key{names}{mnr}{Mono (USA)}
\define@key{names}{mte}{Mono-Alu}
\define@key{names}{moe}{Montagnais}
\define@key{names}{mxk}{Monumbo}
\define@key{names}{mos}{Mossi}
\define@key{names}{mop}{Mopán Maya}
\define@key{names}{mhz}{Mor (Mor Islands)}
\define@key{names}{mok}{Marori}
\define@key{names}{myv}{Erzya}
\define@key{names}{mdf}{Moksha}
\define@key{names}{mor}{Moro}
\define@key{names}{mgd}{Moru}
\define@key{names}{cas}{Mosetén-Chimané}
\define@key{names}{meu}{Motu}
\define@key{names}{siw}{Siwai}
\define@key{names}{mzp}{Movima}
\define@key{names}{mye}{Myene}
\define@key{names}{akc}{Mpur}
\define@key{names}{dmw}{Mudburra}
\define@key{names}{aoj}{Mufian}
\define@key{names}{sgw}{Sebat Bet Gurage}
\define@key{names}{bmr}{Muinane}
\define@key{names}{chb}{Chibcha}
\define@key{names}{mlm}{Mulam}
\define@key{names}{mzm}{Mumuye}
\define@key{names}{mji}{Kim Mun}
\define@key{names}{mnb}{Muna}
\define@key{names}{mua}{Mundang}
\define@key{names}{mnf}{Mundani}
\define@key{names}{myu}{Mundurukú}
\define@key{names}{mhk}{Mungaka}
\define@key{names}{umu}{Munsee}
\define@key{names}{moj}{Monzombo}
\define@key{names}{mtq}{Muong}
\define@key{names}{sur}{Mwaghavul}
\define@key{names}{mtf}{Murik (Papua New Guinea)}
\define@key{names}{mur}{Murle}
\define@key{names}{mwf}{Murriny Patha}
\define@key{names}{muz}{Mursi}
\define@key{names}{zmu}{Muruwari}
\define@key{names}{mug}{Musgu}
\define@key{names}{msu}{Musom}
\define@key{names}{hur}{Halkomelem}
\define@key{names}{emi}{Mussau-Emira}
\define@key{names}{css}{Mutsun}
\define@key{names}{myw}{Muyuw}
\define@key{names}{mwe}{Mwera (Chimwera)}
\define@key{names}{mlv}{Mwotlap}
\define@key{names}{xak}{Máku}
\define@key{names}{bzk}{Nicaragua Creole English}
\define@key{names}{muh}{Mündü}
\define@key{names}{naf}{Nabak}
\define@key{names}{wyy}{Western Fijian}
\define@key{names}{mbj}{Nadëb}
\define@key{names}{nfr}{Nafanan}
\define@key{names}{nbi}{Mao Naga}
\define@key{names}{nmf}{East-Central Tangkhul Naga}
\define@key{names}{nzm}{Zeme Naga}
\define@key{names}{nag}{Naga Pidgin}
\define@key{names}{nce}{Yale}
\define@key{names}{nll}{Nihali}
\define@key{names}{nhn}{Tlaxcala-Puebla-Central Nahuatl}
\define@key{names}{ncj}{Northern Puebla Nahuatl}
\define@key{names}{nhx}{Isthmus-Mecayapan Nahuatl}
\define@key{names}{ncl}{Michoacán Nahuatl}
\define@key{names}{nhm}{Morelos Nahuatl}
\define@key{names}{nhp}{Isthmus-Pajapan Nahuatl}
\define@key{names}{xpo}{Pochutec}
\define@key{names}{azz}{Highland Puebla Nahuatl}
\define@key{names}{nhg}{Tetelcingo Nahuatl}
\define@key{names}{ngu}{Central Guerrero Nahuatl}
\define@key{names}{bio}{Nai}
\define@key{names}{nak}{Nakanai}
\define@key{names}{nck}{Nakara}
\define@key{names}{nal}{Nalik}
\define@key{names}{naq}{Nama (Namibia)}
\define@key{names}{nmb}{Big Nambas}
\define@key{names}{nab}{Southern Nambikuára}
\define@key{names}{nnm}{Namia}
\define@key{names}{gld}{Nanai}
\define@key{names}{ncb}{Central Nicobarese}
\define@key{names}{nnb}{Nande}
\define@key{names}{niq}{Nandi}
\define@key{names}{sen}{Nanerigé Sénoufo}
\define@key{names}{nnk}{Nankina}
\define@key{names}{nnt}{Nanticoke}
\define@key{names}{tvl}{Tuvalu}
\define@key{names}{npy}{Napu}
\define@key{names}{npa}{Nar Phu}
\define@key{names}{nrb}{Nara}
\define@key{names}{nrm}{Narom}
\define@key{names}{nas}{Naasioi}
\define@key{names}{nsk}{Naskapi}
\define@key{names}{ncz}{Natchez}
\define@key{names}{ntm}{Nateni}
\define@key{names}{ntu}{Natügu}
\define@key{names}{nau}{Nauru}
\define@key{names}{nav}{Navajo}
\define@key{names}{nxq}{Naxi}
\define@key{names}{bud}{Ntcham}
\define@key{names}{nde}{Zimbabwean Ndebele}
\define@key{names}{djj}{Djeebbana}
\define@key{names}{ndz}{Ndogo}
\define@key{names}{ndo}{Ndonga}
\define@key{names}{nmd}{Ndumu}
\define@key{names}{ndv}{Ndut}
\define@key{names}{djk}{Aukan}
\define@key{names}{dse}{Dutch Sign Language}
\define@key{names}{neg}{Negidal}
\define@key{names}{nsn}{Nehan}
\define@key{names}{nee}{Nêlêmwa-Nixumwak}
\define@key{names}{anh}{Nend}
\define@key{names}{yrk}{Tundra Nenets}
\define@key{names}{nen}{Nengone}
\define@key{names}{aij}{Lishanid Noshan}
\define@key{names}{aii}{Assyrian Neo-Aramaic}
\define@key{names}{trg}{Lishán Didán}
\define@key{names}{npi}{Nepali}
\define@key{names}{pia}{Pima Bajo}
\define@key{names}{nzs}{New Zealand Sign Language}
\define@key{names}{new}{Kathmandu Valley Newari}
\define@key{names}{ney}{Neyo}
\define@key{names}{nez}{Nez Perce}
\define@key{names}{ntj}{Ngaanyatjarra}
\define@key{names}{nxg}{Ngad'a}
\define@key{names}{nig}{Ngalakgan}
\define@key{names}{ngk}{Ngalkbun}
\define@key{names}{sba}{Ngambay}
\define@key{names}{nam}{Nangikurrunggurr}
\define@key{names}{nio}{Nganasan}
\define@key{names}{nid}{Ngandi}
\define@key{names}{nay}{Narrinyeri}
\define@key{names}{nrk}{Ngarla}
\define@key{names}{nrl}{Ngarluma}
\define@key{names}{nxn}{Ngawun}
\define@key{names}{nbm}{Ngbaka Ma'bo}
\define@key{names}{nga}{Ngbaka Minagende}
\define@key{names}{ngb}{Northern Ngbandi}
\define@key{names}{niy}{Ngiti}
\define@key{names}{wyb}{Ngiyambaa}
\define@key{names}{ngi}{Ngizim}
\define@key{names}{ngo}{Ngoni}
\define@key{names}{llp}{North Efate}
\define@key{names}{gym}{Ngäbere}
\define@key{names}{nha}{Nhanda}
\define@key{names}{nhr}{Naro}
\define@key{names}{nia}{Nias}
\define@key{names}{caq}{Car Nicobarese}
\define@key{names}{pcm}{Nigerian Pidgin}
\define@key{names}{jsl}{Japanese Sign Language}
\define@key{names}{nir}{Nimboran}
\define@key{names}{niz}{Ningil}
\define@key{names}{nsz}{Nisenan}
\define@key{names}{ncg}{Nisga'a}
\define@key{names}{dtd}{Ditidaht}
\define@key{names}{num}{Niuafo'ou}
\define@key{names}{niu}{Niuean}
\define@key{names}{cag}{Nivaclé}
\define@key{names}{niv}{Amur Nivkh}
\define@key{names}{isi}{Nkem-Nkum}
\define@key{names}{nko}{Nkonya}
\define@key{names}{cgg}{Chiga}
\define@key{names}{fia}{Nobiin}
\define@key{names}{njb}{Nocte Naga}
\define@key{names}{nog}{Nogai}
\define@key{names}{not}{Nomatsiguenga}
\define@key{names}{nhu}{Noone}
\define@key{names}{snf}{Noon}
\define@key{names}{nsl}{Norwegian Sign Language}
\define@key{names}{nor}{Norwegian}
\define@key{names}{nse}{Nsenga}
\define@key{names}{nto}{Ntomba}
\define@key{names}{nxl}{South Nuaulu}
\define@key{names}{kcn}{Nubi}
\define@key{names}{dgl}{Nubian (Dongolese)}
\define@key{names}{xnz}{Nubian (Kunuz)}
\define@key{names}{nus}{Nuer}
\define@key{names}{mbr}{Nukak Makú}
\define@key{names}{nkr}{Nukuoro}
\define@key{names}{nut}{Nung (Viet Nam)}
\define@key{names}{nuy}{Wubuy}
\define@key{names}{nuv}{Northern Nuni}
\define@key{names}{iii}{Sichuan Yi}
\define@key{names}{nup}{Nupe-Nupe-Tako}
\define@key{names}{nuf}{Nusu}
\define@key{names}{cbn}{Nyahkur}
\define@key{names}{nly}{Nyamal}
\define@key{names}{now}{Nyambo}
\define@key{names}{tpq}{Nyamkad}
\define@key{names}{nym}{Nyamwezi}
\define@key{names}{nyj}{Nyanga}
\define@key{names}{nyp}{Nyang'i}
\define@key{names}{nna}{Nyangumarta}
\define@key{names}{nyt}{Nyawaygi}
\define@key{names}{yly}{Belep}
\define@key{names}{nyh}{Nyigina}
\define@key{names}{nih}{Nyiha (Tanzania)}
\define@key{names}{nyi}{Ama (Sudan)}
\define@key{names}{njz}{Nyishi-Hill Miri}
\define@key{names}{nyv}{Nyulnyul}
\define@key{names}{nys}{Nyunga}
\define@key{names}{nzk}{Nzakara}
\define@key{names}{ood}{Tohono O'odham}
\define@key{names}{afz}{Obokuitai}
\define@key{names}{ann}{Obolo}
\define@key{names}{oca}{Ocaina}
\define@key{names}{oci}{Occitan}
\define@key{names}{ocu}{Atzingo Matlatzinca}
\define@key{names}{ogb}{Ogbia}
\define@key{names}{ogu}{Ogbronuagum}
\define@key{names}{oyb}{Oy}
\define@key{names}{xal}{Oirad-Kalmyk-Darkhat}
\define@key{names}{ojs}{Severn Ojibwa}
\define@key{names}{ciw}{Chippewa}
\define@key{names}{oka}{Okanagan}
\define@key{names}{opm}{Oksapmin}
\define@key{names}{oku}{Oku}
\define@key{names}{ong}{Olo}
\define@key{names}{plo}{Oluta Popoluca}
\define@key{names}{omg}{Omagua}
\define@key{names}{oma}{Omaha-Ponca}
\define@key{names}{aun}{Molmo One}
\define@key{names}{one}{Oneida}
\define@key{names}{oon}{Önge}
\define@key{names}{ons}{Ono}
\define@key{names}{ono}{Onondaga}
\define@key{names}{mvf}{Peripheral Mongolian}
\define@key{names}{ore}{Maijiki}
\define@key{names}{tag}{Tagoi}
\define@key{names}{ory}{Odia}
\define@key{names}{ort}{Kotia-Adivasi Oriya-Desiya}
\define@key{names}{oru}{Ormuri}
\define@key{names}{oac}{Oroch}
\define@key{names}{oaa}{Orok}
\define@key{names}{okv}{Orokaiva}
\define@key{names}{oro}{Orokolo}
\define@key{names}{gax}{Borana-Arsi-Guji Oromo}
\define@key{names}{hae}{Eastern Oromo}
\define@key{names}{ssn}{Waata}
\define@key{names}{gaz}{West Central Oromo}
\define@key{names}{ury}{Orya}
\define@key{names}{osa}{Osage}
\define@key{names}{oss}{Iron Ossetian}
\define@key{names}{iow}{Iowa-Oto}
\define@key{names}{otz}{Ixtenco Otomi}
\define@key{names}{ote}{Mezquital Otomi}
\define@key{names}{otq}{Querétaro Otomi}
\define@key{names}{otm}{Eastern Highland Otomi}
\define@key{names}{otr}{Otoro}
\define@key{names}{owi}{Owiniga}
\define@key{names}{pqa}{Pa'a}
\define@key{names}{drl}{Paakantyi}
\define@key{names}{pma}{Paama}
\define@key{names}{pac}{Pacoh}
\define@key{names}{pdo}{Padoe}
\define@key{names}{pgu}{Pagu}
\define@key{names}{duf}{Dumbea}
\define@key{names}{pck}{Paite Chin}
\define@key{names}{pao}{Northern Paiute}
\define@key{names}{pwn}{Paiwan}
\define@key{names}{pkn}{Pakanha}
\define@key{names}{pau}{Palauan}
\define@key{names}{pll}{Shwe Palaung}
\define@key{names}{plu}{Palikúr}
\define@key{names}{fap}{Palor}
\define@key{names}{nad}{Palyku}
\define@key{names}{pmz}{Southern Pame}
\define@key{names}{pmf}{Pamona}
\define@key{names}{pbh}{Panare}
\define@key{names}{kre}{Panará}
\define@key{names}{pag}{Pangasinan}
\define@key{names}{pbr}{Pangwa}
\define@key{names}{pan}{Eastern Panjabi}
\define@key{names}{pnw}{Panytyima}
\define@key{names}{pap}{Papiamento}
\define@key{names}{prk}{South Wa}
\define@key{names}{asa}{Asu (Tanzania)}
\define@key{names}{pab}{Parecís}
\define@key{names}{pci}{Duruwa}
\define@key{names}{pst}{Central Pashto}
\define@key{names}{pqm}{Malecite-Passamaquoddy}
\define@key{names}{ptp}{Patep}
\define@key{names}{gfk}{Patpatar}
\define@key{names}{lae}{Pattani}
\define@key{names}{pwi}{Patwin}
\define@key{names}{plh}{Paulohi}
\define@key{names}{pad}{Paumari}
\define@key{names}{pwa}{Pawaia}
\define@key{names}{paw}{Pawnee}
\define@key{names}{pay}{Pech}
\define@key{names}{aoc}{Pemon}
\define@key{names}{peg}{Pengo}
\define@key{names}{pip}{Pero}
\define@key{names}{pes}{Western Farsi}
\define@key{names}{pww}{Pwo Northern Karen}
\define@key{names}{pio}{Piapoco}
\define@key{names}{pid}{Piaroa}
\define@key{names}{plg}{Pilagá}
\define@key{names}{piv}{Vaeakau-Taumako}
\define@key{names}{pif}{Pingelapese}
\define@key{names}{piu}{Pintupi-Luritja}
\define@key{names}{ppl}{Pipil}
\define@key{names}{myp}{Pirahã}
\define@key{names}{pir}{Wa'ikhana}
\define@key{names}{pib}{Yine}
\define@key{names}{psa}{Asue Awyu}
\define@key{names}{pjt}{Pitjantjatjara}
\define@key{names}{pit}{Pitta Pitta}
\define@key{names}{psd}{Plains Indian Sign Language}
\define@key{names}{gob}{Playero}
\define@key{names}{fwa}{Fwâi}
\define@key{names}{pbi}{Parkwa}
\define@key{names}{poy}{Pogolo}
\define@key{names}{pon}{Pohnpeian}
\define@key{names}{rwa}{Rawo}
\define@key{names}{poh}{Poqomchi'}
\define@key{names}{pko}{Pökoot}
\define@key{names}{pox}{Polabian}
\define@key{names}{pol}{Polish}
\define@key{names}{poo}{Central Pomo}
\define@key{names}{peb}{Eastern Pomo}
\define@key{names}{pej}{Northern Pomo}
\define@key{names}{pom}{Southeastern Pomo}
\define@key{names}{pbe}{Mezontla Popoloca}
\define@key{names}{poe}{San Juan Atzingo Popoloca}
\define@key{names}{pbf}{Coyotepec Popoloca}
\define@key{names}{poi}{Highland Popoluca}
\define@key{names}{poc}{Poqomam}
\define@key{names}{psw}{Port Sandwich}
\define@key{names}{por}{Portuguese}
\define@key{names}{pot}{Potawatomi}
\define@key{names}{pim}{Powhatan}
\define@key{names}{prn}{Prasun}
\define@key{names}{pre}{Principense}
\define@key{names}{pui}{Puinave}
\define@key{names}{fuc}{Pulaar}
\define@key{names}{nij}{Ngaju}
\define@key{names}{puw}{Puluwatese}
\define@key{names}{pmi}{Northern Pumi}
\define@key{names}{puq}{Puquina}
\define@key{names}{prx}{Purik-Sham-Nubra}
\define@key{names}{tsz}{Purepecha}
\define@key{names}{pbb}{Páez}
\define@key{names}{lkr}{Päri}
\define@key{names}{aar}{Afar}
\define@key{names}{byx}{Qaqet}
\define@key{names}{alc}{Qawasqar}
\define@key{names}{yum}{Quechan}
\define@key{names}{qxa}{Chiquián Ancash Quechua}
\define@key{names}{quy}{Ayacucho Quechua}
\define@key{names}{qvc}{Cajamarca Quechua}
\define@key{names}{quh}{South Bolivian Quechua}
\define@key{names}{quz}{Cusco Quechua}
\define@key{names}{qug}{Bolivar-North Chimborazo Highland Quichua}
\define@key{names}{qub}{Huallaga Huánuco Quechua}
\define@key{names}{qvi}{Imbabura Highland Quichua}
\define@key{names}{qvn}{North Junín Quechua}
\define@key{names}{quc}{K'iche'}
\define@key{names}{qui}{Quileute}
\define@key{names}{rad}{Rade}
\define@key{names}{lml}{Hano}
\define@key{names}{rji}{Raji}
\define@key{names}{ral}{Ralte}
\define@key{names}{rma}{Rama}
\define@key{names}{bod}{Tibetan}
\define@key{names}{rao}{Rao}
\define@key{names}{rap}{Rapanui}
\define@key{names}{ras}{Tegali}
\define@key{names}{rwo}{Rawa}
\define@key{names}{raw}{Rawang}
\define@key{names}{rej}{Rejang}
\define@key{names}{rmb}{Rembarrnga}
\define@key{names}{bfw}{Bondo}
\define@key{names}{rel}{Rendille}
\define@key{names}{ren}{Rengao}
\define@key{names}{mnv}{Rennell-Bellona}
\define@key{names}{rgr}{Resígaro}
\define@key{names}{tnc}{Tanimuca-Retuarã}
\define@key{names}{ran}{Riantana}
\define@key{names}{rkb}{Rikbaktsa}
\define@key{names}{rim}{Nyaturu}
\define@key{names}{rit}{Ritarungo}
\define@key{names}{rog}{Northern Roglai}
\define@key{names}{rmn}{Balkan Romani}
\define@key{names}{rmo}{Sinte-Manus Romani}
\define@key{names}{rmy}{Vlax Romani}
\define@key{names}{rml}{Baltic Romani}
\define@key{names}{rmw}{Welsh Romani}
\define@key{names}{ron}{Romanian}
\define@key{names}{roh}{Romansh}
\define@key{names}{cla}{Ron}
\define@key{names}{rng}{Ronga}
\define@key{names}{rro}{Waima}
\define@key{names}{twu}{Termanu}
\define@key{names}{roo}{Rotokas}
\define@key{names}{rtm}{Rotuman}
\define@key{names}{rug}{Roviana}
\define@key{names}{dru}{Rukai}
\define@key{names}{klq}{Rumu}
\define@key{names}{run}{Rundi}
\define@key{names}{rou}{Runga}
\define@key{names}{nyn}{Nyankole}
\define@key{names}{nyo}{Nyoro}
\define@key{names}{rus}{Russian}
\define@key{names}{rsl}{Russian-Tajik Sign Language}
\define@key{names}{rut}{Rutul}
\define@key{names}{apb}{Sa'a}
\define@key{names}{snv}{Sa'ban}
\define@key{names}{sma}{South Saami}
\define@key{names}{sjd}{Kildin Saami}
\define@key{names}{sme}{North Saami}
\define@key{names}{skb}{Saek}
\define@key{names}{uma}{Umatilla}
\define@key{names}{ssy}{Saho}
\define@key{names}{saj}{Sahu}
\define@key{names}{sku}{Wanohe}
\define@key{names}{slr}{Salar}
\define@key{names}{sbe}{Saliba}
\define@key{names}{sln}{Salinan}
\define@key{names}{slh}{Southern Puget Sound Salish}
\define@key{names}{sll}{Salt-Yui}
\define@key{names}{sse}{Balangingi}
\define@key{names}{ssb}{Southern Sama}
\define@key{names}{ndi}{Samba Leko}
\define@key{names}{smq}{Samo}
\define@key{names}{smo}{Samoan}
\define@key{names}{sad}{Sandawe}
\define@key{names}{sxn}{Sangir}
\define@key{names}{sag}{Sango}
\define@key{names}{snq}{Sangu (Gabon)}
\define@key{names}{sce}{Dongxiang}
\define@key{names}{sat}{Santali}
\define@key{names}{xsu}{Sanumá}
\define@key{names}{spu}{Sapuan}
\define@key{names}{srm}{Saramaccan}
\define@key{names}{srs}{Sarsi}
\define@key{names}{sro}{Campidanese Sardinian}
\define@key{names}{dju}{Kapriman}
\define@key{names}{ybe}{West Yugur}
\define@key{names}{sdg}{Savi}
\define@key{names}{svs}{Savosavo}
\define@key{names}{szw}{Sawai}
\define@key{names}{hvn}{Hawu}
\define@key{names}{pos}{Sayula Popoluca}
\define@key{names}{kpz}{Kupsabiny}
\define@key{names}{sey}{Secoya}
\define@key{names}{sed}{Sedang}
\define@key{names}{trv}{Seediq}
\define@key{names}{slu}{Selaru}
\define@key{names}{sly}{Selayar}
\define@key{names}{spl}{Selepet}
\define@key{names}{ona}{Selk'nam}
\define@key{names}{sel}{Selkup}
\define@key{names}{nsm}{Sumi Naga}
\define@key{names}{sea}{Semai}
\define@key{names}{sif}{Siamou}
\define@key{names}{sza}{Semelai}
\define@key{names}{seh}{Sena}
\define@key{names}{sef}{Senari}
\define@key{names}{see}{Seneca}
\define@key{names}{szg}{Sengele}
\define@key{names}{set}{Sentani}
\define@key{names}{hbs}{Serbian-Croatian-Bosnian}
\define@key{names}{sei}{Seri}
\define@key{names}{ser}{Serrano}
\define@key{names}{sot}{Southern Sotho}
\define@key{names}{crs}{Seselwa Creole French}
\define@key{names}{sbf}{Shabo}
\define@key{names}{ksb}{Shambala}
\define@key{names}{shn}{Shan}
\define@key{names}{mcd}{Sharanahua}
\define@key{names}{sht}{Shasta}
\define@key{names}{shj}{Shatt}
\define@key{names}{sjw}{Shawnee}
\define@key{names}{swv}{Shekhawati}
\define@key{names}{sdp}{Sherdukpen}
\define@key{names}{xsr}{Solu-Khumbu Sherpa}
\define@key{names}{shk}{Shilluk}
\define@key{names}{scl}{Shina}
\define@key{names}{bwo}{Boro (Ethiopia)}
\define@key{names}{shp}{Shipibo-Conibo}
\define@key{names}{yuy}{East Yugur}
\define@key{names}{shb}{Ninam}
\define@key{names}{sii}{Shom Peng}
\define@key{names}{sna}{Shona}
\define@key{names}{cjs}{Shor}
\define@key{names}{shh}{Shoshoni}
\define@key{names}{sgh}{Shughni}
\define@key{names}{ryu}{Central Okinawan}
\define@key{names}{shs}{Shuswap}
\define@key{names}{snp}{Siane}
\define@key{names}{sjr}{Siar-Lak}
\define@key{names}{sid}{Sidamo}
\define@key{names}{ski}{Sika}
\define@key{names}{tty}{Sikaritai}
\define@key{names}{sip}{Sikkimese}
\define@key{names}{skh}{Sikule}
\define@key{names}{dau}{Dar Sila Daju}
\define@key{names}{smr}{Simeulue}
\define@key{names}{snc}{Sinaugoro}
\define@key{names}{snd}{Sindhi}
\define@key{names}{sin}{Sinhala}
\define@key{names}{xsi}{Sio}
\define@key{names}{snn}{Siona-Tetete}
\define@key{names}{qum}{Sipacapense}
\define@key{names}{fos}{Sirayaic}
\define@key{names}{sri}{Siriano}
\define@key{names}{srq}{Sirionó}
\define@key{names}{ssd}{Siroi}
\define@key{names}{sil}{Tumulung Sisaala}
\define@key{names}{baa}{Babatana}
\define@key{names}{sis}{Siuslaw}
\define@key{names}{skv}{Skou}
\define@key{names}{den}{Slave}
\define@key{names}{xsl}{South Slavey}
\define@key{names}{slk}{Slovak}
\define@key{names}{slv}{Slovenian}
\define@key{names}{teu}{Soo}
\define@key{names}{sob}{Sobei}
\define@key{names}{gru}{Kistane}
\define@key{names}{evn}{Evenki}
\define@key{names}{som}{Somali}
\define@key{names}{sop}{Songe}
\define@key{names}{snk}{Soninke}
\define@key{names}{sov}{Sonsorol}
\define@key{names}{sqt}{Soqotri}
\define@key{names}{srb}{Sora}
\define@key{names}{dsb}{Lower Sorbian}
\define@key{names}{hsb}{Upper Sorbian}
\define@key{names}{nso}{Pedi}
\define@key{names}{mnx}{Sougb}
\define@key{names}{kvk}{Korean Sign Language}
\define@key{names}{tvk}{Southeast Ambrym}
\define@key{names}{wib}{Southern Toussian}
\define@key{names}{spa}{Spanish}
\define@key{names}{spt}{Spiti Bhoti}
\define@key{names}{spo}{Spokane}
\define@key{names}{squ}{Squamish}
\define@key{names}{srn}{Sranan Tongo}
\define@key{names}{kpm}{Koho}
\define@key{names}{sto}{Stoney}
\define@key{names}{sbs}{Subiya}
\define@key{names}{tgo}{Sudest}
\define@key{names}{sue}{Suena}
\define@key{names}{swi}{Sui}
\define@key{names}{sui}{Suki}
\define@key{names}{sub}{Suku}
\define@key{names}{suk}{Sukuma}
\define@key{names}{sua}{Sulka}
\define@key{names}{suv}{Eastern Puroik}
\define@key{names}{sun}{Sundanese}
\define@key{names}{sjg}{Assangori}
\define@key{names}{spp}{Supyire Senoufo}
\define@key{names}{sgz}{Sursurunga}
\define@key{names}{sus}{Susu}
\define@key{names}{sva}{Svan}
\define@key{names}{swl}{Swedish Sign Language}
\define@key{names}{swh}{Swahili}
\define@key{names}{ssw}{Swati}
\define@key{names}{swe}{Swedish}
\define@key{names}{slc}{Sáliba}
\define@key{names}{mky}{East Makian}
\define@key{names}{sst}{Sinasina}
\define@key{names}{tby}{Tabaru}
\define@key{names}{tab}{Tabasaran}
\define@key{names}{tnm}{Tabla}
\define@key{names}{tap}{Taabwa}
\define@key{names}{tna}{Tacana}
\define@key{names}{tgl}{Tagalog}
\define@key{names}{tbw}{Tagbanwa}
\define@key{names}{tah}{Tahitian}
\define@key{names}{gpn}{Taiap}
\define@key{names}{sps}{Saposa}
\define@key{names}{tbg}{North Tairora}
\define@key{names}{tss}{Taiwan Sign Language}
\define@key{names}{tgk}{Tajik}
\define@key{names}{tkm}{Takelma}
\define@key{names}{tbc}{Takia}
\define@key{names}{tld}{Talaud}
\define@key{names}{tlj}{Talinga-Bwisi}
\define@key{names}{tly}{North-Central Talysh}
\define@key{names}{tma}{Tama (Chad)}
\define@key{names}{mla}{Tamambo}
\define@key{names}{tcg}{Tamagario}
\define@key{names}{taj}{Eastern Tamang}
\define@key{names}{taq}{Tamasheq}
\define@key{names}{tam}{Tamil}
\define@key{names}{tpm}{Tampulma}
\define@key{names}{tcb}{Tanacross}
\define@key{names}{tfn}{Dena'ina}
\define@key{names}{taa}{Lower Tanana}
\define@key{names}{tan}{Tangale}
\define@key{names}{skj}{Seke (Nepal)}
\define@key{names}{tgg}{Tangga}
\define@key{names}{tpg}{Kula (Indonesia)}
\define@key{names}{nwi}{Southwest Tanna}
\define@key{names}{tza}{Tanzanian Sign Language}
\define@key{names}{tpj}{Tapieté}
\define@key{names}{tar}{Central Tarahumara}
\define@key{names}{tac}{Lowland Tarahumara}
\define@key{names}{txn}{West Tarangan}
\define@key{names}{tro}{Tarao}
\define@key{names}{tae}{Tariana}
\define@key{names}{yer}{Tarok}
\define@key{names}{shi}{Tachelhit}
\define@key{names}{ttt}{Muslim Tat}
\define@key{names}{txx}{Tatana}
\define@key{names}{tat}{Tatar}
\define@key{names}{tks}{Takestani}
\define@key{names}{tav}{Tatuyo}
\define@key{names}{tuh}{Taulil}
\define@key{names}{trr}{Taushiro}
\define@key{names}{tsg}{Tausug}
\define@key{names}{tya}{Tauya}
\define@key{names}{tbo}{Tawala}
\define@key{names}{cks}{Tayo}
\define@key{names}{tbl}{Tboli}
\define@key{names}{ttc}{Tektiteko}
\define@key{names}{kps}{Tehit}
\define@key{names}{teh}{Tehuelche}
\define@key{names}{kkw}{Teke-Kukuya}
\define@key{names}{tlf}{Telefol}
\define@key{names}{tel}{Telugu}
\define@key{names}{kdh}{Tem}
\define@key{names}{teq}{Temein}
\define@key{names}{tea}{Temiar}
\define@key{names}{tem}{Timne}
\define@key{names}{tex}{Tennet}
\define@key{names}{kza}{Western Karaboro}
\define@key{names}{tio}{Teop}
\define@key{names}{tep}{Tepecano}
\define@key{names}{tee}{Huehuetla Tepehua}
\define@key{names}{tpt}{Tlachichilco Tepehua}
\define@key{names}{ntp}{Northern Tepehuan}
\define@key{names}{stp}{Southeastern Tepehuan}
\define@key{names}{ttr}{Tera}
\define@key{names}{tfr}{Teribe}
\define@key{names}{tft}{Ternate}
\define@key{names}{ter}{Terena-Kinikinao-Chane}
\define@key{names}{teo}{Teso}
\define@key{names}{tll}{Tetela}
\define@key{names}{tet}{Tetum}
\define@key{names}{tew}{Rio Grande Tewa}
\define@key{names}{tcz}{Thado Chin}
\define@key{names}{tha}{Thai}
\define@key{names}{tsq}{Thai Sign Language}
\define@key{names}{ths}{Thakali}
\define@key{names}{thf}{Thangmi}
\define@key{names}{ssf}{Thao}
\define@key{names}{typ}{Thaypan}
\define@key{names}{thp}{Thompson}
\define@key{names}{tdh}{Thulung}
\define@key{names}{tca}{Ticuna}
\define@key{names}{tvo}{Tidore}
\define@key{names}{tif}{Tifal}
\define@key{names}{tgc}{Tigak}
\define@key{names}{tir}{Tigrinya}
\define@key{names}{tig}{Tigre}
\define@key{names}{dih}{Tipai}
\define@key{names}{tik}{Tikar}
\define@key{names}{til}{Tillamook}
\define@key{names}{tms}{Tima}
\define@key{names}{aoz}{Uab Meto}
\define@key{names}{tjm}{Timucua}
\define@key{names}{tih}{Timugon Murut}
\define@key{names}{lbf}{Tinani}
\define@key{names}{tin}{Tindi}
\define@key{names}{cir}{Tiri-Mea}
\define@key{names}{tri}{Trió}
\define@key{names}{tiy}{Tiruray}
\define@key{names}{tiv}{Tiv}
\define@key{names}{twf}{Taos Northern Tiwa}
\define@key{names}{tix}{Southern Tiwa}
\define@key{names}{tiw}{Tiwi}
\define@key{names}{tcf}{Malinaltepec Me'phaa}
\define@key{names}{tli}{Tlingit}
\define@key{names}{tqo}{Toaripi}
\define@key{names}{tob}{Toba}
\define@key{names}{tti}{Tobati}
\define@key{names}{tlb}{Tobelo}
\define@key{names}{sbu}{Stod Bhoti}
\define@key{names}{tcx}{Toda}
\define@key{names}{kim}{Taiga Sayan Turkic}
\define@key{names}{toj}{Tojolabal}
\define@key{names}{tpi}{Tok Pisin}
\define@key{names}{tkl}{Tokelau}
\define@key{names}{jic}{Tol}
\define@key{names}{ksd}{Kuanua}
\define@key{names}{dto}{Tommo So Dogon}
\define@key{names}{tdn}{Tondano}
\define@key{names}{toi}{Tonga (Zambia)}
\define@key{names}{ton}{Tonga (Tonga Islands)}
\define@key{names}{tqw}{Tonkawa}
\define@key{names}{tnt}{Tontemboan}
\define@key{names}{mlu}{To'abaita}
\define@key{names}{sda}{Toraja-Sa'dan}
\define@key{names}{rth}{Ratahan}
\define@key{names}{dts}{Toro So Dogon}
\define@key{names}{trw}{Torwali}
\define@key{names}{tlc}{Yecuatla Totonac}
\define@key{names}{top}{Papantla Totonac}
\define@key{names}{tos}{Highland Totonac}
\define@key{names}{too}{Xicotepec De Juárez Totonac}
\define@key{names}{trs}{Chicahuaxtla Triqui}
\define@key{names}{trc}{Copala Triqui}
\define@key{names}{tpy}{Trumai}
\define@key{names}{cof}{Tsafiki}
\define@key{names}{tkr}{Tsakhur}
\define@key{names}{huq}{Tsat}
\define@key{names}{ddo}{Tsez}
\define@key{names}{tsj}{Tshangla}
\define@key{names}{tsi}{Southern-Coastal Tsimshian}
\define@key{names}{tsv}{Tsogo}
\define@key{names}{tso}{Tsonga}
\define@key{names}{tsu}{Tsou}
\define@key{names}{bbl}{Bats}
\define@key{names}{tsn}{Tswana}
\define@key{names}{pmt}{Tuamotuan}
\define@key{names}{thz}{Tayart Tamajeq}
\define@key{names}{thv}{Tahaggart Tamahaq}
\define@key{names}{tbu}{Tubar}
\define@key{names}{tuo}{Tucano}
\define@key{names}{tzn}{Tugun}
\define@key{names}{bag}{Tuki}
\define@key{names}{tcy}{Tulu}
\define@key{names}{tmc}{Tumak}
\define@key{names}{tmq}{Tumleo}
\define@key{names}{tuf}{Central Tunebo}
\define@key{names}{tvu}{Tunen}
\define@key{names}{lcm}{Tungag}
\define@key{names}{tun}{Tunica}
\define@key{names}{tpn}{Tupinambá}
\define@key{names}{tui}{Tupuri}
\define@key{names}{tuv}{Turkana}
\define@key{names}{kmz}{Khorasan Turkic}
\define@key{names}{tur}{Turkish}
\define@key{names}{tuk}{Turkmen}
\define@key{names}{tus}{Tuscarora}
\define@key{names}{ttm}{Northern Tutchone}
\define@key{names}{tta}{Tutelo}
\define@key{names}{tvt}{Tutsa Naga}
\define@key{names}{tyv}{Tuvinian}
\define@key{names}{tue}{Tuyuca}
\define@key{names}{twa}{Twana}
\define@key{names}{woa}{Tyaraity}
\define@key{names}{tzh}{Tzeltal}
\define@key{names}{tzo}{Tzotzil}
\define@key{names}{tzj}{Tz'utujil}
\define@key{names}{tub}{Tübatulabal}
\define@key{names}{par}{Panamint}
\define@key{names}{tsm}{Turkish Sign Language}
\define@key{names}{umb}{Umbundu}
\define@key{names}{uby}{Ubykh}
\define@key{names}{udi}{Udi}
\define@key{names}{ude}{Udihe}
\define@key{names}{udm}{Udmurt}
\define@key{names}{ugn}{Ugandan Sign Language}
\define@key{names}{ukr}{Ukrainian}
\define@key{names}{ulc}{Ulch}
\define@key{names}{udl}{Wuzlam}
\define@key{names}{uli}{Ulithian}
\define@key{names}{ppk}{Uma}
\define@key{names}{cbd}{Carijona}
\define@key{names}{ubu}{Umbu-Ungu}
\define@key{names}{ump}{Umpila}
\define@key{names}{mtg}{Una}
\define@key{names}{unm}{Unami}
\define@key{names}{ung}{Ngarinyin}
\define@key{names}{kuu}{Upper Kuskokwim}
\define@key{names}{uur}{Ura (Vanuatu)}
\define@key{names}{urf}{Uradhi}
\define@key{names}{urk}{Urak Lawoi'}
\define@key{names}{ura}{Urarina}
\define@key{names}{urt}{Urat}
\define@key{names}{urd}{Urdu}
\define@key{names}{urh}{Urhobo}
\define@key{names}{uri}{Urim}
\define@key{names}{ure}{Uru}
\define@key{names}{uks}{Urubú-Kaapor Sign Language}
\define@key{names}{urb}{Urubú-Kaapor}
\define@key{names}{uum}{Urum}
\define@key{names}{wnu}{Usan}
\define@key{names}{usa}{Usarufa}
\define@key{names}{ute}{Ute-Southern Paiute}
\define@key{names}{uig}{Uighur}
\define@key{names}{uzn}{Northern Uzbek}
\define@key{names}{vaf}{Vafsi}
\define@key{names}{vag}{Vagla}
\define@key{names}{vai}{Vai}
\define@key{names}{vas}{Vasavi}
\define@key{names}{dic}{Lakota Dida}
\define@key{names}{ved}{Veddah}
\define@key{names}{ven}{Venda}
\define@key{names}{vep}{Veps}
\define@key{names}{vie}{Vietnamese}
\define@key{names}{vif}{Vili}
\define@key{names}{vnm}{Neve'ei}
\define@key{names}{vgt}{Vlaamse Gebarentaal}
\define@key{names}{vot}{Votic}
\define@key{names}{wwa}{Waama}
\define@key{names}{wkw}{Wakawaka}
\define@key{names}{waq}{Wageman}
\define@key{names}{waw}{Waiwai}
\define@key{names}{wbk}{Nuristani Kalasha}
\define@key{names}{bao}{Waimaha}
\define@key{names}{wbl}{Wakhi}
\define@key{names}{wls}{East Uvean}
\define@key{names}{van}{Walman}
\define@key{names}{wmt}{Walmajarri}
\define@key{names}{wmb}{Wambayan}
\define@key{names}{wms}{Wambon}
\define@key{names}{wme}{Wambule}
\define@key{names}{wan}{Wan}
\define@key{names}{wgg}{Wangganguru}
\define@key{names}{xwk}{Wangkumara}
\define@key{names}{wbt}{Wanman}
\define@key{names}{wnc}{Wantoat}
\define@key{names}{auc}{Waorani}
\define@key{names}{wap}{Wapishana}
\define@key{names}{wao}{Wappo}
\define@key{names}{wba}{Warao}
\define@key{names}{wrz}{Warray}
\define@key{names}{war}{Waray (Philippines)}
\define@key{names}{wrr}{Wardaman}
\define@key{names}{gae}{Baniva de Maroa}
\define@key{names}{wsa}{Warembori}
\define@key{names}{pav}{Wari'}
\define@key{names}{wrs}{Waris}
\define@key{names}{wbp}{Warlpiri}
\define@key{names}{wrb}{Warluwara}
\define@key{names}{wnd}{Wandarang}
\define@key{names}{wrp}{Waropen}
\define@key{names}{wgy}{Warrgamay}
\define@key{names}{gjm}{Warrnambool}
\define@key{names}{wrg}{Warrongo}
\define@key{names}{wwr}{Warrwa}
\define@key{names}{wrm}{Warumungu}
\define@key{names}{was}{Washo}
\define@key{names}{wsk}{Waskia}
\define@key{names}{wax}{Watam}
\define@key{names}{wth}{Wathawurrung}
\define@key{names}{wbv}{Wajarri}
\define@key{names}{noa}{Woun Meu}
\define@key{names}{wau}{Waurá}
\define@key{names}{oym}{Wayampi}
\define@key{names}{way}{Wayana}
\define@key{names}{wed}{Wedau}
\define@key{names}{cym}{Welsh}
\define@key{names}{xww}{Wembawemba}
\define@key{names}{wer}{Weri}
\define@key{names}{mqs}{West Makian}
\define@key{names}{lex}{Luang}
\define@key{names}{wic}{Wichita}
\define@key{names}{mzh}{Wichí Lhamtés Güisnay}
\define@key{names}{wim}{Wik-Mungkan}
\define@key{names}{wig}{Wik-Ngathana}
\define@key{names}{yok}{Northern Yokuts}
\define@key{names}{win}{Ho-Chunk}
\define@key{names}{wnw}{Wintu}
\define@key{names}{wgu}{Wirangu}
\define@key{names}{wiy}{Wiyot}
\define@key{names}{wob}{Wobe-Wè Northern}
\define@key{names}{wog}{Wogamusin}
\define@key{names}{woi}{Kamang}
\define@key{names}{wyu}{Woiwurrung}
\define@key{names}{wal}{Wolaytta}
\define@key{names}{woe}{Woleaian}
\define@key{names}{wlo}{Wolio}
\define@key{names}{wol}{Wolof}
\define@key{names}{wmx}{Womo-Sumararu}
\define@key{names}{wro}{Worrorra}
\define@key{names}{wuu}{Wu Chinese}
\define@key{names}{wya}{Huron-Wyandot}
\define@key{names}{wem}{Weme Gbe}
\define@key{names}{kao}{Xaasongaxango}
\define@key{names}{xav}{Xavánte}
\define@key{names}{xer}{Xerénte}
\define@key{names}{xho}{Xhosa}
\define@key{names}{xir}{Xiriâna}
\define@key{names}{xok}{Xokleng}
\define@key{names}{ane}{Xârâcùù}
\define@key{names}{yai}{Yagnobi}
\define@key{names}{yad}{Yagua}
\define@key{names}{yag}{Yámana}
\define@key{names}{yaf}{Yaka-Pelende-Lonzo}
\define@key{names}{yka}{Yakan}
\define@key{names}{yky}{Yakoma}
\define@key{names}{sah}{Sakha}
\define@key{names}{ylr}{Yalarnnga}
\define@key{names}{kkl}{Kosarek Yale}
\define@key{names}{yli}{Angguruk Yali}
\define@key{names}{yam}{Yamba}
\define@key{names}{jmd}{Yamdena}
\define@key{names}{tao}{Yami}
\define@key{names}{yaa}{Yaminahua}
\define@key{names}{ybi}{Yamphu}
\define@key{names}{ynn}{Yana}
\define@key{names}{kdd}{Yankunytjatjara}
\define@key{names}{wca}{Yanomám}
\define@key{names}{yns}{Yansi}
\define@key{names}{jao}{Yanyuwa}
\define@key{names}{yao}{Yao}
\define@key{names}{yap}{Yapese}
\define@key{names}{jaq}{Yaqay}
\define@key{names}{yaq}{Yaqui}
\define@key{names}{yrb}{Yareba}
\define@key{names}{yae}{Pumé}
\define@key{names}{yuf}{Havasupai-Walapai-Yavapai}
\define@key{names}{yva}{Yawa}
\define@key{names}{ywr}{Yawuru}
\define@key{names}{pcc}{Bouyei}
\define@key{names}{xya}{Yaygir}
\define@key{names}{yah}{Yazgulyam}
\define@key{names}{kpv}{Komi-Zyrian}
\define@key{names}{jei}{Yei}
\define@key{names}{jel}{Southern Yelmek}
\define@key{names}{yle}{Yele}
\define@key{names}{ybb}{Yemba}
\define@key{names}{jnj}{Yemsa}
\define@key{names}{yss}{Yessan-Mayo}
\define@key{names}{yey}{Yeyi}
\define@key{names}{ywq}{Wuding-Luquan Yi}
\define@key{names}{ydd}{Eastern Yiddish}
\define@key{names}{yii}{Yidiñ}
\define@key{names}{yll}{Yil}
\define@key{names}{yee}{Yimas}
\define@key{names}{yij}{Yindjibarndi}
\define@key{names}{yia}{Yinggarda}
\define@key{names}{yyr}{Yir Yoront}
\define@key{names}{xyy}{Yorta Yorta}
\define@key{names}{yor}{Yoruba}
\define@key{names}{yua}{Yucatec Maya}
\define@key{names}{yuc}{Yuchi}
\define@key{names}{ycn}{Yucuna}
\define@key{names}{yug}{Yugh}
\define@key{names}{yux}{Southern Yukaghir}
\define@key{names}{ykg}{Northern Yukaghir}
\define@key{names}{yuk}{Northern Yukian}
\define@key{names}{yup}{Yukpa}
\define@key{names}{gcd}{Ganggalida}
\define@key{names}{mpj}{Martu Wangka}
\define@key{names}{yul}{Yulu-Binga}
\define@key{names}{esu}{Central Alaskan Yupik}
\define@key{names}{ynk}{Naukan Yupik}
\define@key{names}{ess}{Central Siberian Yupik}
\define@key{names}{ysr}{Old Sirenik}
\define@key{names}{yuz}{Yuracaré}
\define@key{names}{yur}{Yurok}
\define@key{names}{yui}{Yurutí}
\define@key{names}{zne}{Zande}
\define@key{names}{zro}{Záparo}
\define@key{names}{zai}{Isthmus Zapotec}
\define@key{names}{zpd}{Southeastern Ixtlán Zapotec}
\define@key{names}{zaa}{Sierra de Juárez Zapotec}
\define@key{names}{zaw}{Mitla Zapotec}
\define@key{names}{zpm}{Mixtepec Zapotec}
\define@key{names}{zpi}{Santa María Quiegolani Zapotec}
\define@key{names}{zab}{Western Tlacolula Valley Zapotec}
\define@key{names}{zpz}{Texmelucan Zapotec}
\define@key{names}{zav}{Yatzachi Zapotec}
\define@key{names}{zpq}{Zoogocho Zapotec}
\define@key{names}{dje}{Zarma}
\define@key{names}{zay}{Zayse-Zergulla}
\define@key{names}{diq}{Dimli}
\define@key{names}{zen}{Zenaga}
\define@key{names}{zgb}{Guibei Zhuang}
\define@key{names}{zik}{Zimakani}
\define@key{names}{zoh}{Chimalapa Zoque}
\define@key{names}{zos}{Francisco León Zoque}
\define@key{names}{zoc}{Copainalá Zoque}
\define@key{names}{zor}{Rayón Zoque}
\define@key{names}{zul}{Zulu}
\define@key{names}{zun}{Zuni}
\define@key{names}{eme}{Teko}
\define@key{names}{aom}{Ömie}
\define@key{names}{aas}{Aasax}
\define@key{names}{kbt}{Abadi}
\define@key{names}{abg}{Abaga}
\define@key{names}{abf}{Abai Sungai}
\define@key{names}{abm}{Abanyom}
\define@key{names}{mij}{Mungbam}
\define@key{names}{aba}{Abé}
\define@key{names}{abp}{Abenlen Ayta}
\define@key{names}{bsa}{Abinomn}
\define@key{names}{ash}{Aewa}
\define@key{names}{aob}{Abom}
\define@key{names}{abo}{Abon}
\define@key{names}{abr}{Abron}
\define@key{names}{abn}{Abua}
\define@key{names}{abu}{Abure}
\define@key{names}{mgj}{Abureni}
\define@key{names}{ado}{Abu}
\define@key{names}{tpx}{Acatepec Me'phaa}
\define@key{names}{yif}{Ache}
\define@key{names}{acz}{Acheron}
\define@key{names}{acs}{Acroá}
\define@key{names}{xad}{Adai}
\define@key{names}{ada}{Adangme}
\define@key{names}{adq}{Adangbe}
\define@key{names}{tiu}{Adasen}
\define@key{names}{ade}{Adele}
\define@key{names}{adh}{Adhola}
\define@key{names}{gas}{Adiwasi Garasia}
\define@key{names}{adr}{Adonara}
\define@key{names}{aez}{Aeka}
\define@key{names}{aeq}{Aer}
\define@key{names}{afg}{Afghan Sign Language}
\define@key{names}{aft}{Afitti}
\define@key{names}{afh}{Afrihili}
\define@key{names}{afs}{Afro-Seminole Creole}
\define@key{names}{agi}{Agariya}
\define@key{names}{agc}{Agatu}
\define@key{names}{avo}{Agavotaguerra}
\define@key{names}{ggr}{Aghu Tharnggalu}
\define@key{names}{xag}{Aghwan}
\define@key{names}{aif}{Agi}
\define@key{names}{kit}{Agob-Ende-Kawam}
\define@key{names}{ibm}{Agoi}
\define@key{names}{apf}{Agta-Pahanan}
\define@key{names}{aga}{Aguano}
\define@key{names}{aug}{Aguna}
\define@key{names}{msm}{Agusan Manobo}
\define@key{names}{agn}{Agutaynen}
\define@key{names}{yay}{Agwagwune}
\define@key{names}{aha}{Ahanta}
\define@key{names}{ahn}{Àhàn}
\define@key{names}{esg}{Aheri Gondi}
\define@key{names}{thm}{Thavung}
\define@key{names}{kak}{Ahin-Kayapa Kalanguya}
\define@key{names}{aho}{Ahom}
\define@key{names}{nfd}{Ndunic}
\define@key{names}{aih}{Ai-Cham}
\define@key{names}{aix}{Aighon}
\define@key{names}{mwg}{Aiklep}
\define@key{names}{aiq}{Aimaq}
\define@key{names}{ail}{Aimele}
\define@key{names}{aim}{Aimol}
\define@key{names}{aic}{Ainbai}
\define@key{names}{aki}{Aiome}
\define@key{names}{air}{Airoran}
\define@key{names}{aio}{Aiton}
\define@key{names}{ajw}{Ajawa}
\define@key{names}{cpc}{Ajyíninka Apurucayali}
\define@key{names}{soh}{Aka}
\define@key{names}{akm}{Akabo}
\define@key{names}{akj}{Akajeru}
\define@key{names}{ack}{Akakora}
\define@key{names}{aky}{Akakol}
\define@key{names}{acl}{Akarbale}
\define@key{names}{aks}{Akaselem}
\define@key{names}{aik}{Akye}
\define@key{names}{tsr}{Akei}
\define@key{names}{aeu}{Akeu}
\define@key{names}{sia}{Akkala Saami}
\define@key{names}{akk}{Akkadian}
\define@key{names}{akq}{Ak}
\define@key{names}{akt}{Akolet}
\define@key{names}{bss}{Akoose}
\define@key{names}{miw}{Akoye}
\define@key{names}{akf}{Akpa}
\define@key{names}{ibe}{Akpes}
\define@key{names}{afi}{Chini}
\define@key{names}{ayk}{Akuku}
\define@key{names}{aku}{Akum}
\define@key{names}{aqz}{Akuntsu}
\define@key{names}{ako}{Akurio}
\define@key{names}{dul}{Alabat Island Agta}
\define@key{names}{alw}{Alaba-K'abeena}
\define@key{names}{ala}{Alago}
\define@key{names}{alk}{Alak}
\define@key{names}{alj}{Alangan}
\define@key{names}{apv}{Alapmunte}
\define@key{names}{bhk}{Inland-Buhi-Daraga Bikol}
\define@key{names}{sqk}{Albanian Sign Language}
\define@key{names}{lsc}{Albarradas Sign Language}
\define@key{names}{xta}{Alcozauca Mixtec}
\define@key{names}{alf}{Alege}
\define@key{names}{asp}{Algerian Sign Language}
\define@key{names}{arq}{Algerian Arabic}
\define@key{names}{aao}{Algerian Saharan Arabic}
\define@key{names}{aiy}{Ali}
\define@key{names}{all}{Allar}
\define@key{names}{aid}{Alngith}
\define@key{names}{zaq}{Aloápam Zapotec}
\define@key{names}{ypo}{Alo Phola}
\define@key{names}{aol}{Alorese}
\define@key{names}{syy}{Al-Sayyid Bedouin Sign Language}
\define@key{names}{aub}{Alugu}
\define@key{names}{xua}{Alu Kurumba}
\define@key{names}{aab}{Arum}
\define@key{names}{yna}{Aluo}
\define@key{names}{alz}{Alur}
\define@key{names}{avd}{Alviri-Vidari}
\define@key{names}{amq}{Amahai}
\define@key{names}{ali}{Amaimon}
\define@key{names}{aad}{Amal}
\define@key{names}{jks}{Amami O Shima Sign Language}
\define@key{names}{ama}{Amanayé}
\define@key{names}{amg}{Amurdak}
\define@key{names}{aaz}{Amarasi}
\define@key{names}{zpo}{Amatlán Zapotec}
\define@key{names}{rwm}{Amba (Uganda)}
\define@key{names}{utp}{Amba (Solomon Islands)}
\define@key{names}{abc}{Ambala Ayta}
\define@key{names}{aew}{Ambakich}
\define@key{names}{ael}{Ambele}
\define@key{names}{amv}{Ambelau}
\define@key{names}{alm}{Amblong}
\define@key{names}{amb}{Ambo}
\define@key{names}{abs}{Ambonese Malay}
\define@key{names}{qva}{Ambo-Pasco Quechua}
\define@key{names}{aag}{Ambrak}
\define@key{names}{amj}{Amdang}
\define@key{names}{ifa}{Amganad Ifugao}
\define@key{names}{alx}{Mol}
\define@key{names}{mbz}{Amoltepec Mixtec}
\define@key{names}{aqd}{Ampari Dogon}
\define@key{names}{apg}{Ampanang}
\define@key{names}{ajz}{Amri Karbi}
\define@key{names}{amt}{Amto}
\define@key{names}{adw}{Amundava}
\define@key{names}{anw}{Anaang}
\define@key{names}{akg}{Anakalangu}
\define@key{names}{anm}{Anal}
\define@key{names}{pda}{Anam}
\define@key{names}{aan}{Anambé}
\define@key{names}{dti}{Ana Tinga Dogon}
\define@key{names}{grc}{Ancient Greek}
\define@key{names}{hbo}{Ancient Hebrew}
\define@key{names}{xna}{Ancient North Arabian}
\define@key{names}{xlg}{Ancient Ligurian}
\define@key{names}{hca}{Andaman Creole Hindi}
\define@key{names}{afd}{Andai}
\define@key{names}{aod}{Andarum}
\define@key{names}{ana}{Andaqui}
\define@key{names}{xaa}{Andalusian Arabic}
\define@key{names}{adg}{Andegerebinha}
\define@key{names}{bzb}{Andio}
\define@key{names}{anb}{Andoa}
\define@key{names}{anx}{Andra-Hus}
\define@key{names}{aby}{Aneme Wake}
\define@key{names}{myo}{Anfillo}
\define@key{names}{akh}{Angal Heneng}
\define@key{names}{age}{Angal}
\define@key{names}{aoe}{Angal Enen}
\define@key{names}{aqt}{Angaité}
\define@key{names}{avm}{Angkamuthi}
\define@key{names}{anp}{Angika}
\define@key{names}{rme}{Archaic Angloromani}
\define@key{names}{aog}{Angoram}
\define@key{names}{tnd}{Angosturas Tunebo}
\define@key{names}{blo}{Anii}
\define@key{names}{anf}{Animere}
\define@key{names}{aqk}{Aninka}
\define@key{names}{ypn}{Ani Phowa}
\define@key{names}{boj}{Anjam}
\define@key{names}{aak}{Ankave}
\define@key{names}{amx}{Anmatyerre}
\define@key{names}{anj}{Anor}
\define@key{names}{ans}{Anserma}
\define@key{names}{and}{Ansus}
\define@key{names}{ant}{Antakarinya}
\define@key{names}{xmv}{Antankarana Malagasy}
\define@key{names}{aig}{Antigua and Barbuda Creole English}
\define@key{names}{aui}{Anuki}
\define@key{names}{auq}{Anus}
\define@key{names}{aud}{Anuta}
\define@key{names}{anl}{Anu-Hkongso}
\define@key{names}{mtb}{Anyin Morofo}
\define@key{names}{pni}{Aoheng-Seputan}
\define@key{names}{aor}{Aore}
\define@key{names}{aou}{A'ou}
\define@key{names}{xap}{Apalachee}
\define@key{names}{apo}{Apalik}
\define@key{names}{ena}{Apali}
\define@key{names}{mip}{Apasco-Apoala Mixtec}
\define@key{names}{api}{Apiaká}
\define@key{names}{app}{Apma}
\define@key{names}{apx}{Aputai}
\define@key{names}{arg}{Aragonese}
\define@key{names}{stk}{Arammba}
\define@key{names}{aaf}{Aranadan}
\define@key{names}{xrt}{Aranama}
\define@key{names}{arj}{Arapaso}
\define@key{names}{awm}{Arawum}
\define@key{names}{awt}{Araweté}
\define@key{names}{aae}{Arbëreshë Albanian}
\define@key{names}{aea}{Areba}
\define@key{names}{mwc}{Are}
\define@key{names}{aem}{Arem}
\define@key{names}{qxu}{Arequipa-La Unión Quechua}
\define@key{names}{agj}{Argobba}
\define@key{names}{agf}{Arguni}
\define@key{names}{aqr}{Arhâ}
\define@key{names}{aok}{Arhö}
\define@key{names}{ylu}{Aribwaung}
\define@key{names}{aai}{Arifama-Miniafia}
\define@key{names}{aqg}{Arigidi}
\define@key{names}{aac}{Ari}
\define@key{names}{ait}{Arikem}
\define@key{names}{ark}{Arikapú}
\define@key{names}{xrn}{Arin}
\define@key{names}{luc}{Aringa}
\define@key{names}{dth}{Aritinngitigh}
\define@key{names}{aoh}{Arma}
\define@key{names}{aen}{Armenian Sign Language}
\define@key{names}{rup}{Aromanian}
\define@key{names}{aps}{Arop-Sissano}
\define@key{names}{atz}{Arta}
\define@key{names}{arx}{Aruá (Rondonia State)}
\define@key{names}{aru}{Aruá (Amazonas State)}
\define@key{names}{aur}{Aruek}
\define@key{names}{lsr}{Srenge}
\define@key{names}{atx}{Arutani}
\define@key{names}{aat}{Arvanitika Albanian}
\define@key{names}{mtv}{Asaro'o}
\define@key{names}{cni}{Asháninka}
\define@key{names}{ahs}{Ashe}
\define@key{names}{prq}{Ashéninka Perené}
\define@key{names}{ask}{Ashkun}
\define@key{names}{atn}{Ashtiani}
\define@key{names}{asl}{Asilulu}
\define@key{names}{eiv}{Askopan}
\define@key{names}{asv}{Asoa}
\define@key{names}{asb}{Assiniboine}
\define@key{names}{asz}{As}
\define@key{names}{aua}{Asumboa}
\define@key{names}{aum}{Asu (Nigeria)}
\define@key{names}{zoo}{Asunción Mixtepec Zapotec}
\define@key{names}{asr}{Asuri}
\define@key{names}{atm}{Ata}
\define@key{names}{amz}{Atampaya}
\define@key{names}{atd}{Ata Manobo}
\define@key{names}{ate}{Mand}
\define@key{names}{atk}{Ati}
\define@key{names}{aqm}{Atohwaim}
\define@key{names}{aot}{Atong (India)}
\define@key{names}{ato}{Atong}
\define@key{names}{aox}{Atorada}
\define@key{names}{cch}{Atsam}
\define@key{names}{atc}{Atsahuaca}
\define@key{names}{pkr}{Attapady Kurumba}
\define@key{names}{ati}{Attié}
\define@key{names}{kud}{'Auhelawa}
\define@key{names}{aux}{Aurê y Aurá}
\define@key{names}{auh}{Aushi}
\define@key{names}{avs}{Aushiri}
\define@key{names}{asq}{Austrian Sign Language}
\define@key{names}{asw}{Australian Aborigines Sign Language}
\define@key{names}{aut}{Austral}
\define@key{names}{smf}{Auwe}
\define@key{names}{auu}{Auye}
\define@key{names}{auo}{Auyokawa}
\define@key{names}{avv}{Avá-Canoeiro}
\define@key{names}{avb}{Avau}
\define@key{names}{ave}{Avestan}
\define@key{names}{awk}{Awabakal}
\define@key{names}{vwa}{Lavia-Awalai-Damangnuo Awa}
\define@key{names}{bcu}{Awad Bing}
\define@key{names}{awo}{Awak}
\define@key{names}{awx}{Awara}
\define@key{names}{aya}{Awar}
\define@key{names}{awh}{Awbono}
\define@key{names}{bob}{Aweer}
\define@key{names}{awr}{Awera}
\define@key{names}{awe}{Awetí}
\define@key{names}{azo}{Awing}
\define@key{names}{auj}{Awjilah}
\define@key{names}{aww}{Auwon}
\define@key{names}{afu}{Awutu}
\define@key{names}{yiu}{Southern Awu (Lope)}
\define@key{names}{ahb}{Axamb}
\define@key{names}{yix}{Axi Yi}
\define@key{names}{ayd}{Yintyinka-Ayabadhu}
\define@key{names}{vmy}{Ayautla Mazatec}
\define@key{names}{aye}{Ayere}
\define@key{names}{ayq}{Ayi (Papua New Guinea)}
\define@key{names}{yyz}{Ayizi}
\define@key{names}{ayb}{Ayizo Gbe}
\define@key{names}{zaf}{Ayoquesco Zapotec}
\define@key{names}{ayu}{Ayu}
\define@key{names}{aza}{Azha}
\define@key{names}{yiz}{Azhe}
\define@key{names}{tpc}{Azoyú Me'phaa}
\define@key{names}{bvj}{Baan}
\define@key{names}{bqx}{Baangi}
\define@key{names}{bbm}{Babango}
\define@key{names}{bbw}{Baba}
\define@key{names}{bbk}{Babanki}
\define@key{names}{mbf}{Baba Malay}
\define@key{names}{bcr}{Witsuwit'en-Babine}
\define@key{names}{bzg}{Babuza}
\define@key{names}{btj}{Bacanese Malay}
\define@key{names}{bcy}{Bacama}
\define@key{names}{xbc}{Bactrian}
\define@key{names}{bau}{Bada (Nigeria)}
\define@key{names}{bhz}{Bada (Indonesia)}
\define@key{names}{bdz}{Badeshi}
\define@key{names}{jbi}{Badjirri}
\define@key{names}{bac}{Badui}
\define@key{names}{pbp}{Jaad-Badyara}
\define@key{names}{bvd}{Baeggu}
\define@key{names}{bvc}{Baelelea}
\define@key{names}{btr}{Baetora}
\define@key{names}{bwt}{Bafaw-Balong}
\define@key{names}{bfj}{Bafanji}
\define@key{names}{bmd}{Baga Manduri}
\define@key{names}{bgo}{Baga Koga}
\define@key{names}{bcg}{Pukur}
\define@key{names}{bfy}{Bagheli}
\define@key{names}{fui}{Bagirmi Fulfulde}
\define@key{names}{bqg}{Bago-Kusuntu}
\define@key{names}{bqb}{Bagusa}
\define@key{names}{bpi}{Bagupi}
\define@key{names}{yha}{Baha Buyang}
\define@key{names}{bhv}{Bahau}
\define@key{names}{bah}{Bahamas Creole English}
\define@key{names}{bhj}{Bahing}
\define@key{names}{bsu}{Bahonsuai}
\define@key{names}{bbf}{Baibai}
\define@key{names}{bdj}{Bai}
\define@key{names}{bkx}{Baikeno}
\define@key{names}{bqh}{Baima}
\define@key{names}{bmx}{Baimak}
\define@key{names}{bab}{Bainounk-Gujaher}
\define@key{names}{bcz}{Bainouk-Gunyaamolo-Gutobor}
\define@key{names}{fah}{Baissa Fali}
\define@key{names}{bjs}{Bajan}
\define@key{names}{bjm}{Bajelani}
\define@key{names}{bqz}{Bakaka}
\define@key{names}{bqi}{Bakhtiari}
\define@key{names}{bki}{Baki}
\define@key{names}{bkh}{Bakoko}
\define@key{names}{kme}{Bakole}
\define@key{names}{bbs}{Bakpinka}
\define@key{names}{bkr}{Bakumpai}
\define@key{names}{bjw}{Bakwé}
\define@key{names}{ble}{Balanta-Kentohe}
\define@key{names}{bjt}{Balanta-Ganja}
\define@key{names}{bls}{Balaesang}
\define@key{names}{bdn}{Baldemu}
\define@key{names}{bcn}{Bali (Nigeria)}
\define@key{names}{bcp}{Bali (Democratic Republic of Congo)}
\define@key{names}{mhp}{Balinese Malay}
\define@key{names}{bgx}{Rumelian Turkish}
\define@key{names}{biz}{Loi-Likila}
\define@key{names}{bqo}{Balo}
\define@key{names}{blq}{Paluai}
\define@key{names}{bog}{Langue de Signes Malienne}
\define@key{names}{bbq}{Bamali}
\define@key{names}{myf}{Bambassi}
\define@key{names}{bmo}{Bambalang}
\define@key{names}{bce}{Bamenyam}
\define@key{names}{bqt}{Bamukumbit}
\define@key{names}{bvm}{Bamunka}
\define@key{names}{bcf}{Bamu}
\define@key{names}{bmg}{Bamwe}
\define@key{names}{bjx}{Banao Itneg}
\define@key{names}{byz}{Banaro}
\define@key{names}{bqj}{Bandial}
\define@key{names}{bqk}{Banda-Mbrès}
\define@key{names}{bpd}{Banda-Banda}
\define@key{names}{bfl}{Banda-Ndélé}
\define@key{names}{yaj}{Banda-Yangere}
\define@key{names}{bpq}{Banda Malay}
\define@key{names}{bnd}{Banda (Indonesia)}
\define@key{names}{bbe}{Bangba}
\define@key{names}{bgf}{Ngombe-Bangandu}
\define@key{names}{bsj}{Bangwinji}
\define@key{names}{bnx}{Bangubangu}
\define@key{names}{bxg}{Bangala}
\define@key{names}{bgj}{Bangolan}
\define@key{names}{mfb}{Bangka}
\define@key{names}{bjn}{Banjar}
\define@key{names}{bfk}{Ban Khor Sign Language}
\define@key{names}{bxw}{Bankagooma}
\define@key{names}{dbw}{Bankan Tey Dogon}
\define@key{names}{bap}{Bantawa}
\define@key{names}{bno}{Bantoanon}
\define@key{names}{bfx}{Bantayanon}
\define@key{names}{brd}{Baraamu}
\define@key{names}{bbg}{Barama}
\define@key{names}{baj}{Barakai}
\define@key{names}{bhr}{Bara Malagasy}
\define@key{names}{brs}{Baras}
\define@key{names}{brp}{Barapasi}
\define@key{names}{bmz}{Baramu}
\define@key{names}{bpb}{Barbacoas}
\define@key{names}{gry}{Barclayville Grebo}
\define@key{names}{bva}{Barain}
\define@key{names}{bxo}{Barikanchi}
\define@key{names}{bch}{Bariai}
\define@key{names}{bjc}{Bariji}
\define@key{names}{jbk}{Barikewa}
\define@key{names}{bbi}{Barombi}
\define@key{names}{bjk}{Barok}
\define@key{names}{bpt}{Barrow Point}
\define@key{names}{tbn}{Barro Negro Tunebo}
\define@key{names}{bjz}{Baruga}
\define@key{names}{bwg}{Barwe}
\define@key{names}{bjf}{Barzani Jewish Neo-Aramaic}
\define@key{names}{bsl}{Basa-Gumna}
\define@key{names}{buj}{Basa-Gurmana}
\define@key{names}{bzw}{Basa (Nigeria)}
\define@key{names}{bdb}{Basap}
\define@key{names}{byq}{Basay}
\define@key{names}{bsg}{Bashkardi}
\define@key{names}{bst}{Basketo}
\define@key{names}{bsr}{Bassa-Kontagora}
\define@key{names}{bsi}{Bassossi}
\define@key{names}{bnm}{Batanga}
\define@key{names}{bts}{Batak Simalungun}
\define@key{names}{akb}{Batak Angkola}
\define@key{names}{btm}{Batak Mandailing}
\define@key{names}{btd}{Batak Dairi}
\define@key{names}{ayt}{Bataan Ayta}
\define@key{names}{bta}{Bata}
\define@key{names}{btv}{Bateri}
\define@key{names}{btq}{Batek}
\define@key{names}{btc}{Bati (Cameroon)}
\define@key{names}{bvt}{Bati (Indonesia)}
\define@key{names}{btu}{Batu}
\define@key{names}{bay}{Batuley}
\define@key{names}{zbt}{Batui}
\define@key{names}{sne}{Bau-Jagoi Bidayuh}
\define@key{names}{bsf}{Bauchi}
\define@key{names}{bge}{Bauria}
\define@key{names}{bxa}{Bauro}
\define@key{names}{bwk}{Bauwaki}
\define@key{names}{bjy}{Bayali}
\define@key{names}{bvy}{Baybayanon}
\define@key{names}{byg}{Baygo}
\define@key{names}{mkq}{Bay Miwok}
\define@key{names}{bda}{Kugere-Kuxinge}
\define@key{names}{byl}{Bayono}
\define@key{names}{bfr}{Bazigar}
\define@key{names}{beo}{Beami}
\define@key{names}{bea}{Beaver}
\define@key{names}{bfp}{Beba}
\define@key{names}{beb}{Bebele}
\define@key{names}{bzv}{Bebe}
\define@key{names}{bek}{Bebeli}
\define@key{names}{bxp}{Bebil}
\define@key{names}{tnr}{Bedik}
\define@key{names}{bjv}{Nangnda}
\define@key{names}{bed}{Bedoanas}
\define@key{names}{bkf}{Beeke}
\define@key{names}{bxq}{Beele}
\define@key{names}{bnz}{Beezen}
\define@key{names}{bby}{Menchum}
\define@key{names}{bqv}{Begbere-Ejar}
\define@key{names}{bei}{Riuk Bekati'}
\define@key{names}{bkv}{Bekwarra}
\define@key{names}{bkw}{Bekwil}
\define@key{names}{bvi}{Belanda Viri}
\define@key{names}{bxb}{Belanda Bor}
\define@key{names}{beg}{Lemeting}
\define@key{names}{blm}{Beli (South Sudan)}
\define@key{names}{bey}{Beli (Papua New Guinea)}
\define@key{names}{bzj}{Belize Kriol English}
\define@key{names}{brw}{Bellari}
\define@key{names}{glb}{Belneng}
\define@key{names}{bmb}{Bembe}
\define@key{names}{yun}{Bena (Nigeria)}
\define@key{names}{bez}{Bena (Tanzania)}
\define@key{names}{bdp}{Bende}
\define@key{names}{bct}{Bendi}
\define@key{names}{bgy}{Benggoi}
\define@key{names}{bnu}{Bentong}
\define@key{names}{dbt}{Ben Tey Dogon}
\define@key{names}{byd}{Benyadu'}
\define@key{names}{bie}{Bepour}
\define@key{names}{bxv}{Berakou}
\define@key{names}{bve}{Berau Malay}
\define@key{names}{bit}{Berinomo}
\define@key{names}{byt}{Berti}
\define@key{names}{bes}{Besme}
\define@key{names}{bep}{Besoa}
\define@key{names}{bfe}{Betaf}
\define@key{names}{byf}{Bete (Yukubenic)}
\define@key{names}{btt}{Bete-Bendi}
\define@key{names}{eot}{Beti (Côte d'Ivoire)}
\define@key{names}{bhd}{Bhadrawahi}
\define@key{names}{bha}{Bharia}
\define@key{names}{bht}{Bhattiyali}
\define@key{names}{bgw}{Bhatri}
\define@key{names}{bhe}{Bhaya}
\define@key{names}{bhy}{Bhele}
\define@key{names}{bhi}{Bhilali}
\define@key{names}{nes}{Bhoti Kinnauri}
\define@key{names}{bhu}{Bhunjia}
\define@key{names}{bdf}{Biage}
\define@key{names}{beh}{Biali}
\define@key{names}{bpv}{Bian Marind}
\define@key{names}{big}{Biangai}
\define@key{names}{byk}{Shidong Biao}
\define@key{names}{bje}{Biao-Jiao Mien}
\define@key{names}{bmt}{Biao Mon}
\define@key{names}{bym}{Bidyara}
\define@key{names}{bjg}{Kanyaki-Kagbaaga-Kajoko Bidyogo}
\define@key{names}{bmc}{Biem}
\define@key{names}{bnk}{Bierebo}
\define@key{names}{brj}{Bieria}
\define@key{names}{biu}{Biete}
\define@key{names}{xbe}{Bigambal}
\define@key{names}{bhc}{Biga}
\define@key{names}{ibh}{Bih}
\define@key{names}{jbm}{Bijim}
\define@key{names}{bix}{Bijori}
\define@key{names}{byb}{Bikya}
\define@key{names}{kfs}{Bilaspuri}
\define@key{names}{bql}{Karen}
\define@key{names}{brz}{Bilibil}
\define@key{names}{bpz}{Bilba}
\define@key{names}{bil}{Bile}
\define@key{names}{bms}{Bilma Kanuri}
\define@key{names}{bxf}{Bilur}
\define@key{names}{bhl}{Bimin}
\define@key{names}{byj}{Bina (Nigeria)}
\define@key{names}{bmn}{Bina (Papua New Guinea)}
\define@key{names}{bxz}{Binahari-Neme}
\define@key{names}{bon}{Bine}
\define@key{names}{bpj}{Binji}
\define@key{names}{itb}{Binongan Itneg}
\define@key{names}{bne}{Bintauna}
\define@key{names}{bny}{Bintulu}
\define@key{names}{biq}{Bipi}
\define@key{names}{bxe}{Ongota}
\define@key{names}{brr}{Birao}
\define@key{names}{btf}{Birgit}
\define@key{names}{biy}{Birhor}
\define@key{names}{bqq}{Biritai}
\define@key{names}{brk}{Birked}
\define@key{names}{brl}{Birwa}
\define@key{names}{ije}{Biseni}
\define@key{names}{bpy}{Bishnupriya Manipuri}
\define@key{names}{bwh}{Bishuo}
\define@key{names}{bnw}{Bisis}
\define@key{names}{bir}{Bisorio}
\define@key{names}{bzi}{Bisu}
\define@key{names}{brt}{Bitare}
\define@key{names}{bgk}{Bit}
\define@key{names}{mcc}{Bitur}
\define@key{names}{bwm}{Biwat}
\define@key{names}{byo}{Biyo}
\define@key{names}{bpm}{Biyom}
\define@key{names}{blp}{Blablanga}
\define@key{names}{bfh}{Mblafe-Ránmo}
\define@key{names}{beu}{Blagar}
\define@key{names}{blr}{Blang}
\define@key{names}{zbl}{Blissymbols}
\define@key{names}{bzn}{Boano (Maluku)}
\define@key{names}{bzl}{Boano (Sulawesi)}
\define@key{names}{bty}{Bobot}
\define@key{names}{bgb}{Bobongko}
\define@key{names}{bdv}{Bodo Parja}
\define@key{names}{boy}{Bodo (Central African Republic)}
\define@key{names}{bff}{Bofi}
\define@key{names}{boq}{Bogaya}
\define@key{names}{bvw}{Boga}
\define@key{names}{bux}{Boghom}
\define@key{names}{bqu}{Boguru}
\define@key{names}{bhn}{Gardabani Bohtan Neo-Aramaic}
\define@key{names}{ybk}{Bokha}
\define@key{names}{bdt}{Bokoto}
\define@key{names}{bkp}{Boko (Democratic Republic of Congo)}
\define@key{names}{bus}{Bokobaru}
\define@key{names}{bky}{Bokyi}
\define@key{names}{bnp}{Bola}
\define@key{names}{bld}{Bolango}
\define@key{names}{xbo}{Bolgarian}
\define@key{names}{bvo}{Bolgo}
\define@key{names}{bvl}{Bolivian Sign Language}
\define@key{names}{smk}{Bolinao}
\define@key{names}{blv}{Kibala}
\define@key{names}{bkt}{Boloki}
\define@key{names}{bzm}{Bolondo}
\define@key{names}{bof}{Bolon}
\define@key{names}{blj}{Bolongan}
\define@key{names}{ply}{Bolyu}
\define@key{names}{boh}{Boma Yumu}
\define@key{names}{bml}{Bomboli-Bozaba}
\define@key{names}{bws}{Bomboma}
\define@key{names}{zmx}{Bomitaba}
\define@key{names}{bmf}{Bom-Kim}
\define@key{names}{bmq}{Bomu}
\define@key{names}{bmw}{Bomwali}
\define@key{names}{kzc}{Bondoukou Kulango}
\define@key{names}{bou}{Bondei}
\define@key{names}{dbu}{Najamba-Kindige}
\define@key{names}{bna}{Bonerate}
\define@key{names}{bnv}{Bonerif}
\define@key{names}{glc}{Bon Gula}
\define@key{names}{bui}{Bongili}
\define@key{names}{bpg}{Bonggo}
\define@key{names}{bok}{Impfondo}
\define@key{names}{bvg}{Bonkeng}
\define@key{names}{bop}{Bonkiman}
\define@key{names}{bnb}{Bookan}
\define@key{names}{bnl}{Boon}
\define@key{names}{bvf}{Boor}
\define@key{names}{bpw}{Bo (Papua New Guinea)}
\define@key{names}{gai}{Borei}
\define@key{names}{fue}{Borgu Fulfulde}
\define@key{names}{ksr}{Borong}
\define@key{names}{xxb}{Boro}
\define@key{names}{mae}{Bo-Rukul}
\define@key{names}{bwf}{Boselewa}
\define@key{names}{bqs}{Bosngun}
\define@key{names}{bmj}{Bote}
\define@key{names}{bph}{Botlikh}
\define@key{names}{sbl}{Botolan Sambal}
\define@key{names}{nku}{Bouna Kulango}
\define@key{names}{mux}{Bo-Ung}
\define@key{names}{suo}{Bouni-Bobe}
\define@key{names}{kxr}{Manus Koro}
\define@key{names}{aof}{Bragat}
\define@key{names}{bra}{Braj}
\define@key{names}{kvl}{Brek Karen}
\define@key{names}{buq}{Barem}
\define@key{names}{brq}{Breri}
\define@key{names}{rib}{Bribri Sign Language}
\define@key{names}{bzt}{Brithenig}
\define@key{names}{sgt}{Brokpake}
\define@key{names}{bro}{Dur Brokkat}
\define@key{names}{bpl}{Broome Pearling Lugger Pidgin}
\define@key{names}{plw}{Brooke's Point Palawano}
\define@key{names}{kxd}{Brunei}
\define@key{names}{bsb}{Brunei Bisaya-Dusun}
\define@key{names}{rnb}{Brunca Sign Language}
\define@key{names}{bub}{Bua}
\define@key{names}{cbl}{Bualkhaw Chin}
\define@key{names}{box}{Buamu}
\define@key{names}{buw}{Bubi}
\define@key{names}{stt}{Budeh Stieng}
\define@key{names}{btp}{Budibud}
\define@key{names}{bdx}{Budong-Budong}
\define@key{names}{bja}{Budza}
\define@key{names}{bbh}{Bugan}
\define@key{names}{buk}{Bugawac}
\define@key{names}{bgt}{Bughotu}
\define@key{names}{bku}{Buhid}
\define@key{names}{bxh}{Buhutu}
\define@key{names}{byh}{Bujhyal}
\define@key{names}{bvk}{Bukat}
\define@key{names}{bhh}{Bukharic}
\define@key{names}{bvu}{Bukit Malay}
\define@key{names}{bkn}{Bukitan}
\define@key{names}{tkb}{Buksa}
\define@key{names}{buz}{Bukwen}
\define@key{names}{bqn}{Bulgarian Sign Language}
\define@key{names}{bmp}{Bulgebi}
\define@key{names}{buy}{Bullom So}
\define@key{names}{sti}{Bulo Stieng}
\define@key{names}{bjl}{Bulu (Papua New Guinea)}
\define@key{names}{byp}{Bumaji}
\define@key{names}{aon}{Bumbita Arapesh}
\define@key{names}{bmv}{Bum}
\define@key{names}{kjz}{Bumthangkha}
\define@key{names}{bwx}{Bu-Nao Bunu}
\define@key{names}{bdd}{Bunama}
\define@key{names}{bvn}{Buna}
\define@key{names}{bfn}{Bunak}
\define@key{names}{bns}{Bundeli}
\define@key{names}{bqd}{Bung}
\define@key{names}{xbg}{Bunganditj}
\define@key{names}{wun}{Bungu}
\define@key{names}{bkz}{Bungku}
\define@key{names}{but}{Bungain}
\define@key{names}{buv}{Bun}
\define@key{names}{dgb}{Bunoge Dogon}
\define@key{names}{bnn}{Bunun}
\define@key{names}{blf}{Buol}
\define@key{names}{bys}{Burak}
\define@key{names}{bti}{Burate}
\define@key{names}{bxn}{Burduna}
\define@key{names}{bvh}{Bure}
\define@key{names}{pyx}{Burma Pyu}
\define@key{names}{vrt}{Burmbar}
\define@key{names}{bzu}{Burmeso}
\define@key{names}{bqw}{Buru-Angwe}
\define@key{names}{bdi}{Northern Burun}
\define@key{names}{bqr}{Burusu}
\define@key{names}{aip}{Burumakok}
\define@key{names}{asi}{Buruwai}
\define@key{names}{bry}{Burui}
\define@key{names}{bxs}{Busam}
\define@key{names}{bsm}{Busami}
\define@key{names}{bfg}{Busang Kayan}
\define@key{names}{buc}{Kibosy Kiantalaotsy-Majunga}
\define@key{names}{bup}{Busoa}
\define@key{names}{dox}{Bussa}
\define@key{names}{bju}{Busuu}
\define@key{names}{kyb}{Butbut Kalinga}
\define@key{names}{bnr}{Farafi}
\define@key{names}{btw}{Butuanon}
\define@key{names}{jid}{Bu}
\define@key{names}{bhs}{Buwal}
\define@key{names}{jiy}{Buyuan Jinuo}
\define@key{names}{byi}{Buyu}
\define@key{names}{bww}{Bwa}
\define@key{names}{bwd}{Bwaidoka}
\define@key{names}{tte}{Bwanabwana}
\define@key{names}{bwa}{Bwatoo}
\define@key{names}{bwl}{Bwela}
\define@key{names}{bwc}{Bwile}
\define@key{names}{bwz}{Bwisi}
\define@key{names}{mkk}{Byep-Besep}
\define@key{names}{msq}{Caac}
\define@key{names}{cbb}{Cabiyarí}
\define@key{names}{ccr}{Cacaopera}
\define@key{names}{miu}{Cacaloxtepec Mixtec}
\define@key{names}{roc}{Cacgia Roglai}
\define@key{names}{ccd}{Cafundo}
\define@key{names}{cah}{Cahuarano}
\define@key{names}{qvl}{Cajatambo North Lima Quechua}
\define@key{names}{zad}{Cajonos Zapotec}
\define@key{names}{frc}{Cajun French}
\define@key{names}{ckx}{Caka}
\define@key{names}{ckz}{Cakchiquel-Quiché Mixed Language}
\define@key{names}{cky}{Cakfem-Mushere-Jibyal}
\define@key{names}{tbk}{Calamian Tagbanwa}
\define@key{names}{qud}{Calderón Highland Quichua}
\define@key{names}{caw}{Callawalla}
\define@key{names}{rmq}{Caló}
\define@key{names}{clu}{Caluyanun}
\define@key{names}{abd}{Camarines Norte Agta}
\define@key{names}{csx}{Cambodian Sign Language}
\define@key{names}{mcu}{Donga Mambila}
\define@key{names}{wes}{Cameroon Pidgin}
\define@key{names}{cml}{Campalagian}
\define@key{names}{cmt}{Camtho}
\define@key{names}{xcc}{Camunic}
\define@key{names}{qxr}{Cañar-Azuay-South Chimborazo Highland Quichua}
\define@key{names}{caz}{Canichana}
\define@key{names}{mlc}{Cao Lan}
\define@key{names}{cov}{Cao Miao}
\define@key{names}{cps}{Capiznon}
\define@key{names}{cpg}{Cappadocian Greek}
\define@key{names}{cot}{Caquinte}
\define@key{names}{cby}{Carabayo}
\define@key{names}{cfd}{Cara}
\define@key{names}{crf}{Caramanta}
\define@key{names}{xcr}{Carian}
\define@key{names}{hns}{Caribbean Hindustani}
\define@key{names}{jvn}{Caribbean Javanese}
\define@key{names}{crr}{Carolina Algonquian}
\define@key{names}{rmc}{Central Romani}
\define@key{names}{asc}{Casuarina Coast Asmat}
\define@key{names}{csc}{Catalan Sign Language}
\define@key{names}{xcy}{Cayuse}
\define@key{names}{xce}{Celtiberian}
\define@key{names}{cen}{Cen}
\define@key{names}{hmm}{Central Mashan Hmong}
\define@key{names}{cmo}{Central Mnong}
\define@key{names}{zch}{Central Hongshuihe Zhuang}
\define@key{names}{hmc}{Central Huishui Hmong}
\define@key{names}{fuq}{Central-Eastern Niger Fulfulde}
\define@key{names}{grv}{Central Grebo}
\define@key{names}{cet}{Jalaa}
\define@key{names}{pse}{South Barisan Malay}
\define@key{names}{mwo}{Central Maewo}
\define@key{names}{mxz}{Central Masela}
\define@key{names}{syb}{Central Subanen}
\define@key{names}{tgt}{Central Tagbanwa}
\define@key{names}{plc}{Central Palawano}
\define@key{names}{sml}{Central Sama}
\define@key{names}{zbc}{Central Berawan}
\define@key{names}{dtp}{Kadazan Dusun}
\define@key{names}{awu}{Central Awyu}
\define@key{names}{ncx}{Central Puebla Nahuatl}
\define@key{names}{nch}{Central Huasteca Nahuatl}
\define@key{names}{ojc}{Central Ojibwa}
\define@key{names}{pbs}{Central Pame}
\define@key{names}{quk}{Chachapoyas Quechua}
\define@key{names}{cds}{Chadian Sign Language}
\define@key{names}{cdy}{Chadong}
\define@key{names}{chg}{Chagatai}
\define@key{names}{ciy}{Chaima}
\define@key{names}{ccp}{Chakma}
\define@key{names}{ckh}{Chak}
\define@key{names}{cli}{Chakali}
\define@key{names}{tgf}{Chalikha}
\define@key{names}{cll}{Chala}
\define@key{names}{cdh}{Chambeali}
\define@key{names}{ceg}{Chamacoco}
\define@key{names}{ccc}{Chamicuro}
\define@key{names}{cna}{Changthang}
\define@key{names}{cga}{Changriwa}
\define@key{names}{cra}{Chara}
\define@key{names}{crv}{Chaura}
\define@key{names}{xtb}{Chazumba Mixtec}
\define@key{names}{ruk}{Che}
\define@key{names}{cde}{Chenchu}
\define@key{names}{cjn}{Chenapian}
\define@key{names}{cnu}{Western Algerian Berber}
\define@key{names}{ycp}{Chepya}
\define@key{names}{cpn}{Cherepon}
\define@key{names}{ych}{Chesu}
\define@key{names}{cwg}{Chewong}
\define@key{names}{hne}{Chhattisgarhi}
\define@key{names}{ctn}{Chintang}
\define@key{names}{cur}{Chhulung}
\define@key{names}{csd}{Chiangmai Sign Language}
\define@key{names}{cip}{Chiapanec}
\define@key{names}{zpv}{Chichicapan Zapotec}
\define@key{names}{mii}{Chigmecatitlán Mixtec}
\define@key{names}{csg}{Chilean Sign Language}
\define@key{names}{clh}{Chilisso}
\define@key{names}{clc}{Chilcotin-Nicola}
\define@key{names}{csa}{Chiltepec Chinantec}
\define@key{names}{cpi}{Chinese Pidgin English}
\define@key{names}{chn}{Creolized Grand Ronde Chinook Jargon}
\define@key{names}{cih}{Chinali}
\define@key{names}{bxu}{China Buriat}
\define@key{names}{cnb}{Chinbon Chin}
\define@key{names}{qxc}{Chincha Quechua}
\define@key{names}{cdf}{Chiru}
\define@key{names}{nhd}{Chiripá}
\define@key{names}{the}{Chitwania Tharu}
\define@key{names}{cik}{Chhitkul-Rakchham}
\define@key{names}{zpc}{Choapan Zapotec}
\define@key{names}{cgk}{Chocangacakha}
\define@key{names}{cdi}{Chodri}
\define@key{names}{nri}{Chokri Naga}
\define@key{names}{cjk}{Chokwe}
\define@key{names}{cda}{Choni}
\define@key{names}{coh}{Chonyi-Dzihana-Kauma}
\define@key{names}{cce}{Chopi}
\define@key{names}{nct}{Chothe}
\define@key{names}{cvg}{Duhumbi}
\define@key{names}{cuw}{Chukwa}
\define@key{names}{cuh}{Chuka}
\define@key{names}{chu}{Church Slavic}
\define@key{names}{cdj}{Churahi}
\define@key{names}{scb}{Chut}
\define@key{names}{xcv}{Chuvantsy}
\define@key{names}{chw}{Chuwabu}
\define@key{names}{cia}{Cia-Cia}
\define@key{names}{ckl}{Cibak}
\define@key{names}{awc}{Cicipu}
\define@key{names}{cib}{Ci Gbe}
\define@key{names}{cim}{Cimbrian}
\define@key{names}{mkx}{Cinamiguin Manobo}
\define@key{names}{cdr}{Yara}
\define@key{names}{cie}{Cineni}
\define@key{names}{cin}{Cinta Larga}
\define@key{names}{xcg}{Cisalpine Gaulish}
\define@key{names}{asg}{Western-Kambari-Cishingini}
\define@key{names}{txt}{Citak}
\define@key{names}{tgd}{Ciwogai}
\define@key{names}{xcl}{Classical-Middle Armenian}
\define@key{names}{nci}{Classical Nahuatl}
\define@key{names}{qwc}{Classical Quechua}
\define@key{names}{syc}{Classical Syriac}
\define@key{names}{myz}{Classical Mandaic}
\define@key{names}{xct}{Classical Tibetan}
\define@key{names}{dri}{C'lela}
\define@key{names}{naz}{Coatepec Nahuatl}
\define@key{names}{zps}{Coatlán Zapotec}
\define@key{names}{zca}{Coatecas Altas Zapotec}
\define@key{names}{coj}{Cochimi}
\define@key{names}{coa}{Cocos Islands Malay}
\define@key{names}{liw}{Col}
\define@key{names}{csn}{Colombian Sign Language}
\define@key{names}{gct}{Colonia Tovar German}
\define@key{names}{cfg}{Como Karim}
\define@key{names}{swc}{Congo Swahili}
\define@key{names}{cnc}{Côông}
\define@key{names}{coq}{Coquille}
\define@key{names}{cry}{Kyoli}
\define@key{names}{qwa}{Corongo Ancash Quechua}
\define@key{names}{xxr}{Koropó}
\define@key{names}{cos}{Corsican}
\define@key{names}{csr}{Costa Rican Sign Language}
\define@key{names}{mta}{Cotabato Manobo}
\define@key{names}{xcn}{Cotoname}
\define@key{names}{cow}{Cowlitz}
\define@key{names}{toc}{Coyutla Totonac}
\define@key{names}{gyn}{Guyanese Creole English}
\define@key{names}{csq}{Croatian Sign Language}
\define@key{names}{mfn}{Cross River Mbembe}
\define@key{names}{crz}{Cruzeño}
\define@key{names}{csf}{Cuba Sign Language}
\define@key{names}{cbq}{Cuba}
\define@key{names}{cuo}{Cumanagoto}
\define@key{names}{xlu}{Cuneiform Luwian}
\define@key{names}{cnq}{Chung}
\define@key{names}{cuq}{Cun}
\define@key{names}{ccl}{Cutchi-Swahili}
\define@key{names}{cuv}{Cuvok}
\define@key{names}{xtu}{Cuyamecalco Mixtec}
\define@key{names}{cyo}{Cuyonon}
\define@key{names}{bwy}{Cwi Bwamu}
\define@key{names}{cse}{Czech Sign Language}
\define@key{names}{dao}{Daai Chin}
\define@key{names}{lni}{Daantanai'}
\define@key{names}{dtn}{Daats'iin}
\define@key{names}{dbr}{Dabarre}
\define@key{names}{dbe}{Dabe}
\define@key{names}{xdc}{Dacian}
\define@key{names}{dbd}{Dadiya}
\define@key{names}{dgd}{Dagaari Dioula}
\define@key{names}{dgk}{Dagba}
\define@key{names}{dec}{Dagik}
\define@key{names}{dgn}{Dagoman}
\define@key{names}{dlk}{Dahalik}
\define@key{names}{das}{Daho-Doo}
\define@key{names}{dij}{Dai}
\define@key{names}{drb}{Dair}
\define@key{names}{zhd}{Dai Zhuang}
\define@key{names}{bpa}{Dakaka}
\define@key{names}{dkk}{Dakka}
\define@key{names}{dka}{Dakpakha}
\define@key{names}{qer}{Dalecarlian}
\define@key{names}{dlm}{Dalmatian}
\define@key{names}{dmm}{Dama (Cameroon)}
\define@key{names}{dam}{Damakawa}
\define@key{names}{uhn}{Damal}
\define@key{names}{idb}{Daman-Diu Portuguese}
\define@key{names}{dac}{Dambi}
\define@key{names}{dml}{Dameli}
\define@key{names}{dms}{Dampelas}
\define@key{names}{dnu}{Danau}
\define@key{names}{dnr}{Danaru}
\define@key{names}{daq}{Dandami Maria}
\define@key{names}{thl}{Dangaura Tharu}
\define@key{names}{dsl}{Danish Sign Language}
\define@key{names}{daf}{Dan}
\define@key{names}{aso}{Dano}
\define@key{names}{gku}{Danster !Ui}
\define@key{names}{dnd}{Daonda}
\define@key{names}{daz}{Dao}
\define@key{names}{djc}{Dar Daju Daju}
\define@key{names}{dln}{Darlong}
\define@key{names}{dro}{Daro-Matu Melanau}
\define@key{names}{dot}{Dass}
\define@key{names}{daw}{Davawenyo}
\define@key{names}{dww}{Dawawa}
\define@key{names}{ddw}{Dawera-Daweloor}
\define@key{names}{dax}{Dayi}
\define@key{names}{dzg}{Dazaga}
\define@key{names}{dzd}{Daza}
\define@key{names}{ded}{Dedua}
\define@key{names}{gbh}{Defi Gbe}
\define@key{names}{dge}{Degenan}
\define@key{names}{mzw}{Deg}
\define@key{names}{deh}{Dehwari}
\define@key{names}{dek}{Dek}
\define@key{names}{row}{Dela-Oenale}
\define@key{names}{ntr}{Delo}
\define@key{names}{dmx}{Dema}
\define@key{names}{dei}{Demisa}
\define@key{names}{dem}{Dem}
\define@key{names}{dmy}{Demta}
\define@key{names}{deq}{Dendi (Central African Republic)}
\define@key{names}{ddn}{Dendi (Benin)}
\define@key{names}{dez}{Dengese}
\define@key{names}{dnk}{Dengka}
\define@key{names}{dbb}{Deno}
\define@key{names}{anv}{Denya}
\define@key{names}{dee}{Dewoin}
\define@key{names}{def}{Dezfuli-Shushtari}
\define@key{names}{dgh}{Dghwede}
\define@key{names}{dhs}{Dhaiso}
\define@key{names}{dhn}{Dhanki}
\define@key{names}{dwz}{Dewas-Done Danuwar}
\define@key{names}{nfa}{Dhao}
\define@key{names}{mki}{Dhatki}
\define@key{names}{dho}{Dhodia-Kukna}
\define@key{names}{adf}{Dhofari Arabic}
\define@key{names}{ddr}{Dhudhuroa}
\define@key{names}{dhd}{Dhundari}
\define@key{names}{dia}{Alu-Sinagen}
\define@key{names}{mbd}{Dibabawon Manobo}
\define@key{names}{dby}{Dibiyaso}
\define@key{names}{dio}{Dibo}
\define@key{names}{duy}{Dicamay Agta}
\define@key{names}{dig}{Digo}
\define@key{names}{cfa}{Dijim-Bwilim}
\define@key{names}{dil}{Dilling}
\define@key{names}{jma}{Dima}
\define@key{names}{dii}{Dimbong}
\define@key{names}{dmc}{Gavak}
\define@key{names}{ddi}{Diodio}
\define@key{names}{gdl}{Dirasha}
\define@key{names}{diu}{Diriku-Shambyu}
\define@key{names}{dir}{Dirim}
\define@key{names}{dwa}{Diri}
\define@key{names}{dsi}{Dissa-Canton Mufa}
\define@key{names}{tbz}{Ditammari}
\define@key{names}{diy}{Diuwe}
\define@key{names}{xtd}{Diuxi-Tilantongo Mixtec}
\define@key{names}{dix}{Dixon Reef}
\define@key{names}{djf}{Djangun}
\define@key{names}{djn}{Jawoyn}
\define@key{names}{djw}{Djawi}
\define@key{names}{djb}{Djinba}
\define@key{names}{dze}{Djiwarli}
\define@key{names}{dob}{Dobu}
\define@key{names}{doe}{Doe}
\define@key{names}{dgg}{Doga}
\define@key{names}{dgx}{Doghoro}
\define@key{names}{dgs}{Dogoso}
\define@key{names}{dos}{Dogosé}
\define@key{names}{dgr}{Dogrib}
\define@key{names}{dbg}{Dogul Dom Dogon}
\define@key{names}{dbi}{Doka}
\define@key{names}{uya}{Doko-Uyanga}
\define@key{names}{dre}{Dolpo}
\define@key{names}{dov}{Toka-Leya-Dombe}
\define@key{names}{doq}{Dominican Sign Language}
\define@key{names}{doa}{Dom}
\define@key{names}{doy}{Dompo}
\define@key{names}{dof}{Domu}
\define@key{names}{dev}{Domung}
\define@key{names}{dok}{Dondo}
\define@key{names}{yik}{Dongshanba Lalo}
\define@key{names}{doh}{Dong}
\define@key{names}{ddd}{Dongotono}
\define@key{names}{dde}{Doondo}
\define@key{names}{dor}{Dori'o}
\define@key{names}{kqc}{Doromu-Koki}
\define@key{names}{doz}{Dorze}
\define@key{names}{dol}{Doso}
\define@key{names}{dty}{Dotyali}
\define@key{names}{dup}{Duano}
\define@key{names}{dva}{Duau}
\define@key{names}{dub}{Dubli}
\define@key{names}{dmu}{Dubu}
\define@key{names}{duk}{Duduela}
\define@key{names}{ndu}{Dugun}
\define@key{names}{dbm}{Duguri}
\define@key{names}{dme}{Dugwor}
\define@key{names}{kbz}{Duhwa}
\define@key{names}{nke}{Duke}
\define@key{names}{dbo}{Dulbu}
\define@key{names}{duz}{Duli-Gewe}
\define@key{names}{dmv}{Dumpas}
\define@key{names}{wtf}{Dumpu}
\define@key{names}{dui}{Dumun}
\define@key{names}{duh}{Dungra Bhil}
\define@key{names}{raa}{Dungmali}
\define@key{names}{dng}{Dungan}
\define@key{names}{dbv}{Dungu}
\define@key{names}{drq}{Dura}
\define@key{names}{mvp}{Duri}
\define@key{names}{dbn}{Duriankere}
\define@key{names}{dug}{Duruma}
\define@key{names}{dsn}{Dusner}
\define@key{names}{duw}{Dusun Witu}
\define@key{names}{duq}{Dusun Malang}
\define@key{names}{dun}{Dusun Deyah}
\define@key{names}{dws}{Dutton World Speedwords}
\define@key{names}{dux}{Duungooma}
\define@key{names}{dae}{Duupa}
\define@key{names}{duv}{Duvle}
\define@key{names}{dbp}{Duwai}
\define@key{names}{gve}{Duwet}
\define@key{names}{nnu}{Dwang}
\define@key{names}{dyb}{Dyaberdyaber}
\define@key{names}{dyn}{Dyangadi}
\define@key{names}{dya}{Dyan}
\define@key{names}{dyd}{Dyugun}
\define@key{names}{jen}{Dza}
\define@key{names}{dzl}{Dzalakha}
\define@key{names}{dzn}{Dzando}
\define@key{names}{bpn}{Dzao Min}
\define@key{names}{add}{Dzodinka}
\define@key{names}{dzo}{Dzongkha}
\define@key{names}{dnn}{Dzùùngoo}
\define@key{names}{ktv}{Eastern Katu}
\define@key{names}{bgp}{Eastern Balochi}
\define@key{names}{lwl}{Eastern Lawa}
\define@key{names}{mng}{Eastern Mnong}
\define@key{names}{emu}{Eastern Muria}
\define@key{names}{tge}{Eastern Gorkha Tamang}
\define@key{names}{nos}{Eastern Nisu}
\define@key{names}{emq}{Eastern Muya}
\define@key{names}{kif}{Eastern Parbate Kham}
\define@key{names}{emg}{Eastern Meohang}
\define@key{names}{zeh}{Eastern Hongshuihe Zhuang}
\define@key{names}{hmq}{Eastern Qiandong Miao}
\define@key{names}{muq}{Eastern Xiangxi Miao}
\define@key{names}{hme}{Eastern Huishui Hmong}
\define@key{names}{lma}{East Limba}
\define@key{names}{gbx}{Eastern Xwla Gbe}
\define@key{names}{xrb}{Eastern Karaboro}
\define@key{names}{acp}{Eastern Acipa}
\define@key{names}{nle}{East Nyala}
\define@key{names}{kqo}{Konobo-Eastern Krahn}
\define@key{names}{vme}{East Masela}
\define@key{names}{tre}{East Tarangan}
\define@key{names}{dmr}{East Damar}
\define@key{names}{bnj}{Eastern Tawbuid}
\define@key{names}{pez}{Eastern Penan}
\define@key{names}{zbe}{East Berawan}
\define@key{names}{kjs}{East Kewa}
\define@key{names}{nhe}{Eastern Huasteca Nahuatl}
\define@key{names}{ojg}{Eastern Ojibwa}
\define@key{names}{aaq}{Eastern Abenaki}
\define@key{names}{qve}{Eastern Apurímac Quechua}
\define@key{names}{cly}{Eastern Highland Chatino}
\define@key{names}{avl}{Eastern Egyptian Bedawi Arabic}
\define@key{names}{sfe}{Eastern Subanen}
\define@key{names}{azd}{Eastern Durango Nahuatl}
\define@key{names}{yit}{Eastern Lalu}
\define@key{names}{cek}{Eastern Khumi Chin}
\define@key{names}{yol}{Irish Anglo-Norman}
\define@key{names}{xeb}{Eblaite}
\define@key{names}{ebr}{Ebrié}
\define@key{names}{ebg}{Ebughu}
\define@key{names}{ecs}{Ecuadorian Sign Language}
\define@key{names}{cbj}{Ede Cabe}
\define@key{names}{idd}{Ede Idaca}
\define@key{names}{ijj}{Ede Ije}
\define@key{names}{ica}{Ede Ica}
\define@key{names}{nqg}{Ede Nago}
\define@key{names}{awy}{Edera Awyu}
\define@key{names}{dbf}{Edopi}
\define@key{names}{eee}{E}
\define@key{names}{efa}{Efai}
\define@key{names}{efe}{Efe}
\define@key{names}{ofu}{Efutop}
\define@key{names}{ego}{Eggon}
\define@key{names}{esl}{Egypt Sign Language}
\define@key{names}{egy}{Egyptian (Ancient)}
\define@key{names}{ehu}{Ehueun}
\define@key{names}{eit}{Eitiep}
\define@key{names}{eja}{Ejamat}
\define@key{names}{eka}{Ekajuk}
\define@key{names}{eki}{Eki}
\define@key{names}{eke}{Ekit}
\define@key{names}{ekp}{Ekpeye}
\define@key{names}{zpp}{El Alto Zapotec}
\define@key{names}{elx}{Elamite}
\define@key{names}{elm}{Eleme}
\define@key{names}{ele}{Elepi}
\define@key{names}{elh}{El Hugeirat}
\define@key{names}{ekm}{Elip}
\define@key{names}{elk}{Elkei}
\define@key{names}{elo}{El Molo}
\define@key{names}{zte}{Elotepec Zapotec}
\define@key{names}{afo}{Ajiri}
\define@key{names}{elu}{Elu}
\define@key{names}{xly}{Elymian}
\define@key{names}{yzg}{E'ma Buyang}
\define@key{names}{emn}{Eman}
\define@key{names}{bdc}{Emberá-Baudó}
\define@key{names}{tdc}{Emberá-Tadó}
\define@key{names}{ebu}{Embu}
\define@key{names}{emw}{Emplawas}
\define@key{names}{enr}{Emumu}
\define@key{names}{unk}{Enawené-Nawé}
\define@key{names}{end}{Ende}
\define@key{names}{enc}{En}
\define@key{names}{ptt}{Enrekang}
\define@key{names}{enu}{Enu}
\define@key{names}{enw}{Enwan (Akwa Ibom State)}
\define@key{names}{env}{Enwan (Edo State)}
\define@key{names}{epi}{Epie}
\define@key{names}{emy}{Epigraphic Mayan}
\define@key{names}{era}{Eravallan}
\define@key{names}{kjy}{Erave}
\define@key{names}{twp}{Ere}
\define@key{names}{ert}{Eritai}
\define@key{names}{erw}{Erokwanas}
\define@key{names}{err}{Erre}
\define@key{names}{emx}{Erromintxela}
\define@key{names}{ers}{Ersu}
\define@key{names}{erh}{Eruwa}
\define@key{names}{ish}{Esan}
\define@key{names}{mcq}{Ese}
\define@key{names}{esh}{Eshtehardi}
\define@key{names}{ags}{Esimbi}
\define@key{names}{esy}{Eskayan}
\define@key{names}{epo}{Esperanto}
\define@key{names}{ots}{Estado de México Otomi}
\define@key{names}{eso}{Estonian Sign Language}
\define@key{names}{esm}{Esuma}
\define@key{names}{etb}{Etebi}
\define@key{names}{etx}{Eten}
\define@key{names}{ecr}{Eteocretan}
\define@key{names}{ecy}{Eteocypriot}
\define@key{names}{eth}{Ethiopian Sign Language}
\define@key{names}{ich}{Etkywan}
\define@key{names}{eto}{Eton-Mengisa}
\define@key{names}{etn}{Eton (Vanuatu)}
\define@key{names}{ett}{Etruscan}
\define@key{names}{utr}{Etulo}
\define@key{names}{bzz}{Evant}
\define@key{names}{gev}{Viya}
\define@key{names}{nou}{Ewage-Notu}
\define@key{names}{ext}{Extremaduran}
\define@key{names}{fab}{Annobonese}
\define@key{names}{faf}{Fagani}
\define@key{names}{fif}{Faifi}
\define@key{names}{azt}{Faire Atta}
\define@key{names}{faj}{Kulsab}
\define@key{names}{fai}{Faiwol}
\define@key{names}{fax}{Fala}
\define@key{names}{cfm}{Falam Chin}
\define@key{names}{fli}{Fali}
\define@key{names}{xfa}{Faliscan}
\define@key{names}{fam}{Fam}
\define@key{names}{fng}{Fanagalo}
\define@key{names}{fan}{Fang (Equatorial Guinea)}
\define@key{names}{fak}{Fang (Cameroon)}
\define@key{names}{fni}{Fania}
\define@key{names}{nsf}{Far Northwestern Nisu}
\define@key{names}{fmu}{Far Western Muria}
\define@key{names}{far}{Fataleka}
\define@key{names}{ddg}{Fataluku}
\define@key{names}{fau}{Fayu}
\define@key{names}{agl}{Fembe}
\define@key{names}{fpe}{Pichi}
\define@key{names}{fer}{Feroge}
\define@key{names}{hif}{Fiji Hindi}
\define@key{names}{fil}{Filipino}
\define@key{names}{tlp}{Filomeno Mata Totonac}
\define@key{names}{bkb}{Eastern-Southern Bontok}
\define@key{names}{fss}{Finland-Swedish Sign Language}
\define@key{names}{fag}{Finongan}
\define@key{names}{fip}{Fipa}
\define@key{names}{fir}{Firan}
\define@key{names}{fiw}{Fiwaga}
\define@key{names}{fln}{Flinders Island}
\define@key{names}{flh}{Abawiri}
\define@key{names}{fod}{Foodo}
\define@key{names}{frq}{Forak}
\define@key{names}{enf}{Forest Enets}
\define@key{names}{frt}{Kiai}
\define@key{names}{frp}{Arpitan}
\define@key{names}{fur}{Friulian}
\define@key{names}{flr}{Fuliiru}
\define@key{names}{ula}{Fungwa}
\define@key{names}{fuy}{Fuyug}
\define@key{names}{fwe}{Fwe}
\define@key{names}{fie}{Fyer}
\define@key{names}{ttb}{Gaa}
\define@key{names}{gie}{Gabogbo}
\define@key{names}{gab}{Gabri}
\define@key{names}{gdg}{Ga'dang}
\define@key{names}{gdk}{Gadang}
\define@key{names}{gbk}{Gaddi}
\define@key{names}{gad}{Gaddang}
\define@key{names}{gda}{Gade Lohar}
\define@key{names}{gdh}{Gajirrabeng}
\define@key{names}{gft}{Gafat}
\define@key{names}{btg}{Gagnoa Bété}
\define@key{names}{ggu}{Gban}
\define@key{names}{gbf}{Gaikundi}
\define@key{names}{gic}{Gail}
\define@key{names}{gcn}{Gaina}
\define@key{names}{xga}{Galatian}
\define@key{names}{glo}{Galambu}
\define@key{names}{gar}{Galeya}
\define@key{names}{gce}{Galice}
\define@key{names}{sdn}{Gallurese Sardinian}
\define@key{names}{gap}{Gal}
\define@key{names}{gal}{Galoli-Talur}
\define@key{names}{kgj}{Gamale Kham}
\define@key{names}{gma}{Gambera}
\define@key{names}{wof}{Gambian Wolof}
\define@key{names}{gbl}{Gamit}
\define@key{names}{gak}{Gamkonora}
\define@key{names}{bte}{Gamo-Ningi}
\define@key{names}{ihw}{Birrdhawal}
\define@key{names}{gne}{Ganang}
\define@key{names}{gnk}{//Gana}
\define@key{names}{gnq}{Gana}
\define@key{names}{unn}{Ganai}
\define@key{names}{gan}{Gan Chinese}
\define@key{names}{pgd}{Gandhari}
\define@key{names}{gzn}{Gane}
\define@key{names}{gnb}{Gangte}
\define@key{names}{gnl}{Gangulu}
\define@key{names}{ggl}{Ganglau}
\define@key{names}{gao}{Gants}
\define@key{names}{gza}{Ganza}
\define@key{names}{gnz}{Ganzi}
\define@key{names}{gga}{Gao}
\define@key{names}{gbm}{Garhwali}
\define@key{names}{ilg}{Garig-Ilgar}
\define@key{names}{gex}{Garre}
\define@key{names}{gaq}{Gata'}
\define@key{names}{gou}{Gavar}
\define@key{names}{gwt}{Gawar-Bati}
\define@key{names}{gyl}{Gayil}
\define@key{names}{gzi}{Gazic}
\define@key{names}{gbg}{Gbanziri-Boraka}
\define@key{names}{gbv}{Gbanu}
\define@key{names}{gby}{Gbari}
\define@key{names}{gyg}{Gbayi}
\define@key{names}{gbq}{Gbaya-Bozoum}
\define@key{names}{gbs}{Gbesi Gbe}
\define@key{names}{ggb}{Gbii}
\define@key{names}{xgb}{Gbin}
\define@key{names}{grh}{Gbiri-Niragu}
\define@key{names}{gec}{Gboloo Grebo}
\define@key{names}{kvq}{Geba Karen}
\define@key{names}{gei}{Gebe}
\define@key{names}{gdd}{Gedaged}
\define@key{names}{drs}{Gedeo}
\define@key{names}{hmj}{Ge}
\define@key{names}{gez}{Geez}
\define@key{names}{ghk}{Geko Karen}
\define@key{names}{giu}{Gelao Mulao}
\define@key{names}{geq}{Geme}
\define@key{names}{gaf}{Gende}
\define@key{names}{gej}{Gen}
\define@key{names}{ygp}{Gepo}
\define@key{names}{gew}{Gera}
\define@key{names}{gea}{Geruma}
\define@key{names}{ges}{Geser-Gorom}
\define@key{names}{gha}{Ghadames}
\define@key{names}{gse}{Ghanaian Sign Language}
\define@key{names}{ghn}{Ghanongga}
\define@key{names}{gpe}{Ghanaian Pidgin English}
\define@key{names}{gds}{Ghandruk Sign Language}
\define@key{names}{gri}{Ghari}
\define@key{names}{ajs}{Ghardaia Sign Language}
\define@key{names}{bmk}{Ghayavi}
\define@key{names}{aln}{Gheg Albanian}
\define@key{names}{ghr}{Ghera}
\define@key{names}{bbj}{Ghomálá'}
\define@key{names}{gho}{Ghomara}
\define@key{names}{bgi}{Giangan}
\define@key{names}{gib}{Gibanawa}
\define@key{names}{kks}{Giiwo}
\define@key{names}{acd}{Gikyode}
\define@key{names}{gix}{Gilima}
\define@key{names}{gip}{Gimi (West New Britain)}
\define@key{names}{gim}{Gimi (Eastern Highlands)}
\define@key{names}{kmp}{Gimme}
\define@key{names}{gmn}{Gimnime}
\define@key{names}{gnm}{Ginuman}
\define@key{names}{ayg}{Ginyanga}
\define@key{names}{bbr}{Girawa}
\define@key{names}{gii}{Girirra}
\define@key{names}{nyf}{Giryama}
\define@key{names}{toh}{Gitonga}
\define@key{names}{ggt}{Gitua}
\define@key{names}{giy}{Giyug}
\define@key{names}{tof}{Gizrra}
\define@key{names}{glr}{Glaro-Twabo}
\define@key{names}{glw}{Glavda}
\define@key{names}{oub}{Glio-Oubi}
\define@key{names}{gnu}{Gnau}
\define@key{names}{gom}{Goan Konkani}
\define@key{names}{gig}{Goaria}
\define@key{names}{goi}{Gobasi}
\define@key{names}{gox}{Gobu}
\define@key{names}{gdx}{Godwari}
\define@key{names}{gof}{Gofa}
\define@key{names}{gog}{Gogo}
\define@key{names}{goo}{Gone Dau}
\define@key{names}{goe}{Gongduk}
\define@key{names}{gjn}{Gonja}
\define@key{names}{gov}{Goo}
\define@key{names}{goq}{Gorap}
\define@key{names}{goc}{Gorakor}
\define@key{names}{grq}{Gorovu}
\define@key{names}{gqr}{Gor}
\define@key{names}{got}{Gothic}
\define@key{names}{goy}{Goundo}
\define@key{names}{gwf}{Gowro}
\define@key{names}{goz}{Alamuti}
\define@key{names}{nli}{Grangali-Ningalami}
\define@key{names}{giq}{Hagei Gelao}
\define@key{names}{gcl}{Grenadian Creole English}
\define@key{names}{grs}{Gresi}
\define@key{names}{gro}{Groma}
\define@key{names}{gos}{Gronings}
\define@key{names}{ats}{Gros Ventre}
\define@key{names}{gwx}{Gua}
\define@key{names}{gvj}{Guajá}
\define@key{names}{jiq}{Khroskyabs}
\define@key{names}{gnc}{Guanche}
\define@key{names}{gyr}{Guarayu}
\define@key{names}{gsm}{Guatemalan Sign Language}
\define@key{names}{xgd}{Gudang}
\define@key{names}{gdu}{Gudu}
\define@key{names}{zpg}{Guevea De Humboldt Zapotec}
\define@key{names}{gdc}{Gugu Badhun}
\define@key{names}{kkp}{Gugubera}
\define@key{names}{wrw}{Roth's Gugu Warra}
\define@key{names}{zgn}{Guibian Zhuang}
\define@key{names}{bet}{Guiberoua Béte}
\define@key{names}{ztu}{Güilá Zapotec}
\define@key{names}{gus}{Guinean Sign Language}
\define@key{names}{gkp}{Guinea Kpelle}
\define@key{names}{gqi}{Guiqiong}
\define@key{names}{gvl}{Gulay}
\define@key{names}{glu}{Gula (Chad)}
\define@key{names}{gmb}{Gula'alaa}
\define@key{names}{gly}{Gule}
\define@key{names}{gul}{Sea Island Creole English}
\define@key{names}{gmu}{Gumalu}
\define@key{names}{gdi}{Gundi}
\define@key{names}{gyf}{Gungabula}
\define@key{names}{rub}{Gungu}
\define@key{names}{gnt}{Warta Thuntai}
\define@key{names}{gpa}{Gupa-Abawa}
\define@key{names}{grz}{Guramalum}
\define@key{names}{gdj}{Gurdjar}
\define@key{names}{ggg}{Gurgula}
\define@key{names}{grx}{Guriaso}
\define@key{names}{gjr}{Gurindji Kriol}
\define@key{names}{gvm}{Gurmana}
\define@key{names}{gvr}{Gurung}
\define@key{names}{grd}{Guruntum-Mbaaru}
\define@key{names}{gsn}{Gusan}
\define@key{names}{gsl}{Gusilay}
\define@key{names}{xgw}{Guwa}
\define@key{names}{gwu}{Guwamu}
\define@key{names}{gvy}{Guyani}
\define@key{names}{gka}{Guya}
\define@key{names}{ngs}{Gvoko}
\define@key{names}{gwb}{Gwa}
\define@key{names}{dah}{Gwahatike}
\define@key{names}{bga}{Gwamhi-Wuri}
\define@key{names}{gwn}{Gwandara}
\define@key{names}{grw}{Gweda}
\define@key{names}{gwe}{Gweno}
\define@key{names}{gwr}{Gwere}
\define@key{names}{gwj}{/Gwi}
\define@key{names}{gyi}{Gyele}
\define@key{names}{gye}{Gyem}
\define@key{names}{haq}{Ha}
\define@key{names}{hbu}{Habu}
\define@key{names}{hdy}{Hadiyya}
\define@key{names}{hoj}{Hadothi}
\define@key{names}{xhd}{Hadrami}
\define@key{names}{ayh}{Hadrami Arabic}
\define@key{names}{aek}{Haeke}
\define@key{names}{hah}{Hahon}
\define@key{names}{hgw}{Haigwai}
\define@key{names}{bzx}{Hainyaxo Bozo}
\define@key{names}{hgm}{Hai//om-Akhoe}
\define@key{names}{haf}{Haiphong Sign Language}
\define@key{names}{hvc}{Haitian Vodoun Culture Language}
\define@key{names}{hji}{Haji}
\define@key{names}{haj}{Hajong}
\define@key{names}{hao}{Hakö}
\define@key{names}{hld}{Halang Doan}
\define@key{names}{hmu}{Hamap}
\define@key{names}{hba}{Hamba de Lomela}
\define@key{names}{hag}{Hanga}
\define@key{names}{han}{Hangaza}
\define@key{names}{haa}{Han}
\define@key{names}{hab}{Hanoi Sign Language}
\define@key{names}{xiv}{Harappan}
\define@key{names}{kjo}{Indo-Aryan Kinnauri}
\define@key{names}{hro}{Haroi}
\define@key{names}{hrk}{Haruku}
\define@key{names}{bgc}{Haryanvi}
\define@key{names}{hrz}{Harzani-Kilit}
\define@key{names}{ybj}{Hasha}
\define@key{names}{xht}{Hattic}
\define@key{names}{hsl}{Hausa Sign Language}
\define@key{names}{hvk}{Haveke}
\define@key{names}{hav}{Havu}
\define@key{names}{hps}{Hawai'i Pidgin Sign Language}
\define@key{names}{xda}{Hawkesbury}
\define@key{names}{haz}{Hazaragi}
\define@key{names}{hbn}{Ebang}
\define@key{names}{scp}{Lamjung-Melamchi Yolmo}
\define@key{names}{heg}{Helong}
\define@key{names}{nix}{Hema}
\define@key{names}{hed}{Herde}
\define@key{names}{llf}{Hermit}
\define@key{names}{hrt}{Hertevin}
\define@key{names}{ham}{Hewa}
\define@key{names}{auk}{Heyo}
\define@key{names}{hib}{Hibito}
\define@key{names}{hlu}{Hieroglyphic Luwian}
\define@key{names}{mba}{Higaonon}
\define@key{names}{kjk}{Highland Konjo}
\define@key{names}{hij}{Hijuk}
\define@key{names}{hir}{Himarimã}
\define@key{names}{hii}{Hinduri}
\define@key{names}{hmo}{Hiri Motu}
\define@key{names}{hit}{Hittite}
\define@key{names}{htu}{Hitu}
\define@key{names}{hiw}{Hiw}
\define@key{names}{yhl}{Hlepho Phowa}
\define@key{names}{hle}{Hlersu}
\define@key{names}{hmf}{Hmong Don}
\define@key{names}{hmz}{Sinicized Miao}
\define@key{names}{hmv}{Hmong Dô}
\define@key{names}{mrk}{Hmwaveke}
\define@key{names}{hoh}{Hobyót}
\define@key{names}{hos}{Ho Chi Minh City Sign Language}
\define@key{names}{hhi}{Hoia Hoia-Ukusi-Koperami}
\define@key{names}{hoy}{Holiya}
\define@key{names}{hoi}{Holikachuk}
\define@key{names}{hod}{Holma}
\define@key{names}{hol}{Holu}
\define@key{names}{hom}{Homa}
\define@key{names}{hds}{Honduras Sign Language}
\define@key{names}{juh}{Hõne}
\define@key{names}{how}{Honi}
\define@key{names}{hrm}{Horned Miao}
\define@key{names}{hoe}{Horom}
\define@key{names}{hor}{Horo}
\define@key{names}{ero}{Stau-Dgebshes}
\define@key{names}{hot}{Hote}
\define@key{names}{hti}{Hoti of East Seram}
\define@key{names}{hov}{Hobongan}
\define@key{names}{hhy}{Hoyahoya-Matakaia}
\define@key{names}{hoz}{Hozo}
\define@key{names}{hpo}{Hpon}
\define@key{names}{hra}{Hrangkhol}
\define@key{names}{hru}{Hruso}
\define@key{names}{hug}{Huachipaeri}
\define@key{names}{qvh}{Huamalíes-Dos de Mayo Huánuco Quechua}
\define@key{names}{hud}{Huaulu}
\define@key{names}{nhq}{Huaxcaleca Nahuatl}
\define@key{names}{qwh}{Huaylas Ancash Quechua}
\define@key{names}{qvw}{Huaylla Wanca Quechua}
\define@key{names}{huh}{Huilliche}
\define@key{names}{mxs}{Huitepec Mixtec}
\define@key{names}{czh}{Hui Chinese}
\define@key{names}{huw}{Hukumina}
\define@key{names}{hul}{Hula}
\define@key{names}{huy}{Hulaulá}
\define@key{names}{hui}{Huli}
\define@key{names}{huk}{Hulung}
\define@key{names}{hmb}{Humburi Senni Songhay}
\define@key{names}{huf}{Humene}
\define@key{names}{hut}{Humla}
\define@key{names}{hsh}{Hungarian Sign Language}
\define@key{names}{hnu}{Hung}
\define@key{names}{nat}{Hungworo}
\define@key{names}{hum}{Hungan}
\define@key{names}{hng}{Hungu-Pombo}
\define@key{names}{hkk}{Hunjara-Kaina Ke}
\define@key{names}{hap}{Hupla}
\define@key{names}{xhu}{Hurrian}
\define@key{names}{geh}{Hutterite German}
\define@key{names}{huo}{Hu}
\define@key{names}{hwo}{Hwana}
\define@key{names}{hya}{Hya}
\define@key{names}{jab}{Hyam}
\define@key{names}{yml}{Iamalele}
\define@key{names}{tek}{Kwa South}
\define@key{names}{ibl}{Ibaloi}
\define@key{names}{iby}{Ibani}
\define@key{names}{xib}{Iberian}
\define@key{names}{ibn}{Ibino}
\define@key{names}{ibr}{Ibuoro}
\define@key{names}{ibu}{Ibu}
\define@key{names}{bec}{Iceve-Maci}
\define@key{names}{ida}{Idakho-Isukha-Tiriki}
\define@key{names}{idt}{Idaté}
\define@key{names}{ide}{Idere}
\define@key{names}{idi}{Idi-Taeme}
\define@key{names}{idc}{Idon}
\define@key{names}{ido}{Ido}
\define@key{names}{ldb}{Dũya}
\define@key{names}{ife}{Ifè}
\define@key{names}{iff}{Ifo}
\define@key{names}{igl}{Igala}
\define@key{names}{igg}{Igana}
\define@key{names}{ahl}{Igo}
\define@key{names}{nar}{Iguta}
\define@key{names}{igw}{Igwe}
\define@key{names}{ihb}{Iha-based Pidgin}
\define@key{names}{ikk}{Ika}
\define@key{names}{ikr}{Ikaranggal}
\define@key{names}{ikz}{Ikizu}
\define@key{names}{meb}{Ikobi}
\define@key{names}{ntk}{Ikoma-Nata}
\define@key{names}{iki}{Iko}
\define@key{names}{ikp}{Ikpeshi}
\define@key{names}{txi}{Ikpeng}
\define@key{names}{ikv}{Iku-Gora-Ankwa}
\define@key{names}{ikl}{Ikulu}
\define@key{names}{ikw}{Ikwere}
\define@key{names}{ila}{Ile Ape}
\define@key{names}{mbi}{Ilianen Manobo}
\define@key{names}{ili}{Ili Turki}
\define@key{names}{ilu}{Ili'uun}
\define@key{names}{xil}{Illyrian}
\define@key{names}{ilk}{Ilongot}
\define@key{names}{ilv}{Ilue}
\define@key{names}{mlk}{Ilwana}
\define@key{names}{imo}{Imbongu}
\define@key{names}{arc}{Imperial Aramaic (700-300 BCE)}
\define@key{names}{imr}{Imroing}
\define@key{names}{abx}{Inabaknon}
\define@key{names}{mzu}{Itutang-Inapang}
\define@key{names}{inp}{Iñapari}
\define@key{names}{smn}{Inari Saami}
\define@key{names}{inl}{Jakartan Sign Language}
\define@key{names}{idr}{Indri}
\define@key{names}{mvy}{Indus Kohistani}
\define@key{names}{oin}{Inebu One}
\define@key{names}{iti}{Inlaod Itneg}
\define@key{names}{ino}{Inoke-Yate}
\define@key{names}{loc}{Inonhan}
\define@key{names}{ior}{Inoric}
\define@key{names}{ina}{Interlingua (International Auxiliary Language Association)}
\define@key{names}{ile}{Interlingue (Occidental)}
\define@key{names}{igs}{Interglossa}
\define@key{names}{int}{Intha-Danu}
\define@key{names}{iks}{Inuit Sign Language}
\define@key{names}{azm}{Ipalapa Amuzgo}
\define@key{names}{ipo}{Ipiko}
\define@key{names}{ipi}{Ipili}
\define@key{names}{ass}{Ipulo-Olulu}
\define@key{names}{ill}{Iranun}
\define@key{names}{iry}{Iraya}
\define@key{names}{ire}{Yerisiam}
\define@key{names}{iri}{Irigwe}
\define@key{names}{bto}{Iriga Bicolano}
\define@key{names}{iru}{Irula of the Nilgiri}
\define@key{names}{isa}{Isabi}
\define@key{names}{isn}{Isanzu}
\define@key{names}{agk}{Isarog Agta}
\define@key{names}{isc}{Isconahua}
\define@key{names}{igo}{Isebe}
\define@key{names}{inn}{Isinai}
\define@key{names}{crb}{Island Carib}
\define@key{names}{mir}{Isthmus Mixe}
\define@key{names}{nhk}{Isthmus-Cosoleacaque Nahuatl}
\define@key{names}{ist}{Istriot}
\define@key{names}{ruo}{Istro Romanian}
\define@key{names}{szv}{Isu (Fako Division)}
\define@key{names}{isu}{Isu (Menchum Division)}
\define@key{names}{ite}{Itene}
\define@key{names}{itr}{Iteri}
\define@key{names}{itx}{Itik}
\define@key{names}{itw}{Ito}
\define@key{names}{itm}{Itu Mbon Uzo}
\define@key{names}{mce}{Itundujia Mixtec}
\define@key{names}{ivv}{Itbayat}
\define@key{names}{atg}{Ivbie North-Okpela-Arhe}
\define@key{names}{iwk}{I-Wak}
\define@key{names}{kbm}{Iwal}
\define@key{names}{iwo}{Morop-Dintere}
\define@key{names}{mzi}{Ixcatlán Mazatec}
\define@key{names}{vmj}{Ixtayutla Mixtec}
\define@key{names}{iya}{Iyayu}
\define@key{names}{uiv}{Iyive}
\define@key{names}{crt}{Iyojwa'ja Chorote}
\define@key{names}{nca}{Iyo}
\define@key{names}{crq}{Iyo'wujwa Chorote}
\define@key{names}{izi}{Izi-Ezaa-Ikwo-Mgbo}
\define@key{names}{cbo}{Izora}
\define@key{names}{rzh}{Jabal Razih}
\define@key{names}{jdg}{Jadgali}
\define@key{names}{jad}{Jahanka}
\define@key{names}{jah}{Jah Hut}
\define@key{names}{awv}{Kia River Awyu}
\define@key{names}{jat}{Inku}
\define@key{names}{jak}{Jakun}
\define@key{names}{maj}{Jalapa De Díaz Mazatec}
\define@key{names}{bxl}{Jalkunan}
\define@key{names}{jcs}{Jamaican Country Sign Language}
\define@key{names}{jls}{Jamaican Sign Language}
\define@key{names}{jax}{Jambi Malay}
\define@key{names}{jnd}{Jandavra}
\define@key{names}{jna}{Jangshung}
\define@key{names}{djo}{Jangkang}
\define@key{names}{jni}{Janji}
\define@key{names}{jar}{Jarawa (Nigeria)}
\define@key{names}{jra}{Jarai}
\define@key{names}{jaf}{Jara}
\define@key{names}{qxw}{Jauja Wanca Quechua}
\define@key{names}{jns}{Jaunsari}
\define@key{names}{jvd}{Javindo}
\define@key{names}{jaz}{Jawe}
\define@key{names}{jyy}{Jaya}
\define@key{names}{jje}{Jejueo}
\define@key{names}{bze}{Jenaama Bozo}
\define@key{names}{xuj}{Jennu Kurumba}
\define@key{names}{jer}{Jere}
\define@key{names}{jee}{Jerung}
\define@key{names}{tmr}{Jewish Babylonian Aramaic (ca. 200-1200 CE)}
\define@key{names}{jhs}{Jhankot Sign Language}
\define@key{names}{jio}{Jiamao}
\define@key{names}{juo}{Jiba}
\define@key{names}{jib}{Jibu}
\define@key{names}{jii}{Jiiddu}
\define@key{names}{jie}{Jilbe}
\define@key{names}{jil}{Jilim}
\define@key{names}{jim}{Jimi (Cameroon)}
\define@key{names}{jmi}{Jimi (Nigeria)}
\define@key{names}{jia}{Jina}
\define@key{names}{cjy}{Jinyu Chinese}
\define@key{names}{pnu}{Jiongnai Bunu}
\define@key{names}{jul}{Jirel}
\define@key{names}{jrr}{Jiru}
\define@key{names}{jit}{Jita}
\define@key{names}{kaj}{Jju}
\define@key{names}{job}{Joba}
\define@key{names}{jbr}{Jofotek-Bromnya}
\define@key{names}{jeu}{Jonkor Bourmataguil}
\define@key{names}{jor}{Jorá}
\define@key{names}{jrt}{Jakattoe}
\define@key{names}{jow}{Jowulu}
\define@key{names}{itk}{Judeo-Italian}
\define@key{names}{jdt}{Judeo-Tat}
\define@key{names}{jpr}{Judeo-Persian}
\define@key{names}{yud}{Judeo-Tripolitanian Arabic}
\define@key{names}{aju}{Judeo-Moroccan Arabic}
\define@key{names}{yhd}{Judeo-Iraqi Arabic}
\define@key{names}{jye}{Judeo-Yemeni Arabic}
\define@key{names}{jum}{Jumjum}
\define@key{names}{jml}{Jumli}
\define@key{names}{jus}{Jumla Sign Language}
\define@key{names}{mxq}{Juquila Mixe}
\define@key{names}{juy}{Juray}
\define@key{names}{jut}{Jutish}
\define@key{names}{juu}{Ju}
\define@key{names}{mwb}{Juwal}
\define@key{names}{vmc}{Juxtlahuaca Mixtec}
\define@key{names}{jwi}{Jwira-Pepesa}
\define@key{names}{xku}{Kaamba}
\define@key{names}{gna}{Kaansa}
\define@key{names}{ldl}{Kaan}
\define@key{names}{ckn}{Kaang Chin}
\define@key{names}{ksp}{Kaba}
\define@key{names}{kvf}{Kabalai}
\define@key{names}{gbw}{Kabikabi}
\define@key{names}{klz}{Kabola}
\define@key{names}{onk}{Kabore One}
\define@key{names}{lkb}{Kabras}
\define@key{names}{uka}{Kaburi}
\define@key{names}{kbu}{Kabutra}
\define@key{names}{kea}{Kabuverdianu}
\define@key{names}{cwa}{Kabwa}
\define@key{names}{kcw}{Kabwari}
\define@key{names}{gjk}{Kachi Koli}
\define@key{names}{kfr}{Kachchi}
\define@key{names}{kcx}{Kachama-Ganjule-Haro}
\define@key{names}{xkk}{Kaco'}
\define@key{names}{kej}{Kadar}
\define@key{names}{kdu}{Kadaru}
\define@key{names}{kad}{Kadara}
\define@key{names}{kzd}{Kadai}
\define@key{names}{kdv}{Kado}
\define@key{names}{ktp}{Kaduo}
\define@key{names}{jka}{Kaera}
\define@key{names}{kpu}{Kafoa}
\define@key{names}{sqx}{Kafr Qasem Sign Language}
\define@key{names}{syw}{Kagate}
\define@key{names}{kll}{Kagan Kalagan}
\define@key{names}{cgc}{Kagayanen}
\define@key{names}{gel}{Ut-Main}
\define@key{names}{xkg}{Kagoro}
\define@key{names}{hka}{Kahe}
\define@key{names}{agw}{Kahua}
\define@key{names}{kzb}{Kaibobo}
\define@key{names}{kzp}{Kaidipang}
\define@key{names}{kbw}{Kaiep}
\define@key{names}{kep}{Kaikadi}
\define@key{names}{kzq}{Kaike}
\define@key{names}{kkq}{Kaiku}
\define@key{names}{xai}{Kaimbé}
\define@key{names}{zka}{Kaimbulawa}
\define@key{names}{krd}{Kairui-Midiki}
\define@key{names}{ckr}{Kairak}
\define@key{names}{kzm}{Kais}
\define@key{names}{kce}{Kaivi}
\define@key{names}{tcq}{Kaiy}
\define@key{names}{xkj}{Kajali}
\define@key{names}{kag}{Kajaman}
\define@key{names}{ckq}{Kajakse}
\define@key{names}{kjv}{Kajkavian}
\define@key{names}{xdq}{Kajtak}
\define@key{names}{kka}{Kakanda}
\define@key{names}{kke}{Kakabe}
\define@key{names}{kqf}{Kakabai}
\define@key{names}{kkj}{Kako}
\define@key{names}{keo}{Kakwa}
\define@key{names}{wkl}{Kalanadi}
\define@key{names}{kzz}{Kalabra}
\define@key{names}{kkf}{Kalaktang Monpa}
\define@key{names}{kba}{Kalarko-Mirniny}
\define@key{names}{gll}{Bulloo River}
\define@key{names}{ijn}{Kalabari}
\define@key{names}{knz}{Kalamsé}
\define@key{names}{kqe}{Kalagan}
\define@key{names}{kve}{Kalabakan}
\define@key{names}{kly}{Kalao}
\define@key{names}{lkm}{Kalaamaya}
\define@key{names}{xka}{Kalkoti}
\define@key{names}{rmf}{Kalo Finnish Romani}
\define@key{names}{ywa}{Kalou}
\define@key{names}{kli}{Kalumpang}
\define@key{names}{keq}{Kamar}
\define@key{names}{jmr}{Kamara}
\define@key{names}{kci}{Kamantan}
\define@key{names}{klp}{Kamasa}
\define@key{names}{kzx}{Kamarian}
\define@key{names}{kyk}{Kamayo}
\define@key{names}{kgx}{Kamaru}
\define@key{names}{vkm}{Kamakan}
\define@key{names}{xbw}{Kambiwá}
\define@key{names}{irx}{Kamberau}
\define@key{names}{kyy}{Kambaira}
\define@key{names}{ktb}{Kambaata}
\define@key{names}{kmi}{Kami (Nigeria)}
\define@key{names}{kdx}{Kam}
\define@key{names}{kcq}{Kamo}
\define@key{names}{xla}{Kamula}
\define@key{names}{hig}{Kamwe}
\define@key{names}{bjj}{Kanauji}
\define@key{names}{xnb}{Kanakanavu}
\define@key{names}{soq}{Kanasi}
\define@key{names}{kbs}{Kande}
\define@key{names}{kqw}{Kandas}
\define@key{names}{gam}{Kandawo}
\define@key{names}{xnr}{Kangri}
\define@key{names}{kxs}{Kangjia}
\define@key{names}{kzy}{Kango (Tshopo District)}
\define@key{names}{kty}{Kango (Bas-Uélé District)}
\define@key{names}{kcp}{Kanga}
\define@key{names}{kkv}{Kangean}
\define@key{names}{igm}{Kanggape}
\define@key{names}{kev}{Kanikkaran}
\define@key{names}{kdp}{Kaningdon-Nindem}
\define@key{names}{kzo}{Kaningi}
\define@key{names}{wat}{Kaninuwa}
\define@key{names}{ktk}{Kaniet}
\define@key{names}{knr}{Kaningra}
\define@key{names}{kmu}{Kanite}
\define@key{names}{kft}{Kanjari}
\define@key{names}{kbe}{Kanju}
\define@key{names}{kxn}{Kanowit-Tanjong Melanau}
\define@key{names}{ksk}{Kansa}
\define@key{names}{xkt}{Kantosi}
\define@key{names}{kni}{Kanufi}
\define@key{names}{khx}{Kanu}
\define@key{names}{kqn}{Kaonde}
\define@key{names}{kax}{Kao}
\define@key{names}{xpn}{Kapinawá}
\define@key{names}{tbx}{Kapin}
\define@key{names}{khp}{Kapori}
\define@key{names}{ykm}{Kap}
\define@key{names}{kbi}{Kaptiau}
\define@key{names}{klo}{Kapya}
\define@key{names}{xkh}{Karahawyana}
\define@key{names}{kzr}{Karang}
\define@key{names}{reg}{Kara (Tanzania)}
\define@key{names}{kth}{Karanga}
\define@key{names}{mry}{Mandaya}
\define@key{names}{xrw}{Karawa}
\define@key{names}{xar}{Karami}
\define@key{names}{kgv}{Kalamang}
\define@key{names}{kbn}{Kare (Central African Republic)}
\define@key{names}{kyd}{Karey}
\define@key{names}{kmf}{Kare (Papua New Guinea)}
\define@key{names}{kai}{Karekare}
\define@key{names}{kmv}{Uaçá Creole French}
\define@key{names}{kgn}{Karingani-Kalasuri-Khoynarudi}
\define@key{names}{kbj}{Kari}
\define@key{names}{kil}{Kariya}
\define@key{names}{kuq}{Karipúna}
\define@key{names}{kko}{Karko}
\define@key{names}{krb}{Karkin}
\define@key{names}{bbv}{Karnai}
\define@key{names}{krx}{Karon}
\define@key{names}{kxh}{Karo (Ethiopia)}
\define@key{names}{xkx}{Karore}
\define@key{names}{kyn}{Northern Binukidnon}
\define@key{names}{rxw}{Karruwali}
\define@key{names}{ccj}{Kasanga}
\define@key{names}{ksn}{Kasiguranin}
\define@key{names}{kkz}{Kaska}
\define@key{names}{khs}{Kasua}
\define@key{names}{ktq}{Katabaga}
\define@key{names}{xat}{Katawixi}
\define@key{names}{tmb}{Avava}
\define@key{names}{tkt}{Kathoriya Tharu}
\define@key{names}{ykt}{Thou-Kathu}
\define@key{names}{kfu}{Katkari}
\define@key{names}{kaf}{Katso}
\define@key{names}{kta}{Katua}
\define@key{names}{vkk}{Kaur}
\define@key{names}{xau}{Kauwera}
\define@key{names}{ckv}{Kavalan}
\define@key{names}{kcb}{Kawacha}
\define@key{names}{kgb}{Kawe}
\define@key{names}{kaw}{Kawi}
\define@key{names}{ktx}{Kaxararí}
\define@key{names}{kbb}{Kaxuiâna}
\define@key{names}{pdu}{Kayan Lahwi}
\define@key{names}{xay}{Kayan Mahakam}
\define@key{names}{xkn}{Kayan River Kayan}
\define@key{names}{kyt}{Kayagar}
\define@key{names}{kzl}{Kayeli}
\define@key{names}{kxy}{Kayong}
\define@key{names}{kzu}{Kayupulau}
\define@key{names}{kzk}{Kazukuru}
\define@key{names}{keh}{Keak}
\define@key{names}{khz}{Keapara}
\define@key{names}{meo}{Kedah-Perak Malay}
\define@key{names}{kdy}{Keder}
\define@key{names}{khh}{Kehu}
\define@key{names}{kec}{Keiga}
\define@key{names}{bmh}{Kein}
\define@key{names}{eyo}{Keiyo}
\define@key{names}{khy}{Kele-Foma}
\define@key{names}{keb}{Kélé}
\define@key{names}{ify}{Keley-i Kallahan}
\define@key{names}{kbo}{Keliko}
\define@key{names}{xel}{Kelo}
\define@key{names}{kyo}{Klon}
\define@key{names}{kem}{Kemak}
\define@key{names}{bzp}{Kemberano}
\define@key{names}{xem}{Mateq}
\define@key{names}{xkw}{Kembra}
\define@key{names}{dmo}{Kemezung}
\define@key{names}{sjk}{Kemi Saami}
\define@key{names}{xbn}{Kenaboi}
\define@key{names}{gat}{Kenati}
\define@key{names}{kvm}{Kendem}
\define@key{names}{klf}{Kendeje}
\define@key{names}{knx}{Kendayan-Belangin}
\define@key{names}{knl}{Keninjal}
\define@key{names}{kxi}{Keningau Murut}
\define@key{names}{kns}{Kensiu}
\define@key{names}{ndb}{Kenswei Nsei}
\define@key{names}{kzh}{Kenuzi-Dongola}
\define@key{names}{lke}{Kenyi}
\define@key{names}{xeu}{Keoru-Ahia}
\define@key{names}{kpn}{Kepkiriwát}
\define@key{names}{kuk}{Kepo'}
\define@key{names}{hhr}{Keerak}
\define@key{names}{ked}{Kerewe}
\define@key{names}{xke}{Kereho}
\define@key{names}{kxz}{Kerewo}
\define@key{names}{kvr}{Kerinci}
\define@key{names}{xes}{Kesawai}
\define@key{names}{kae}{Ketangalan}
\define@key{names}{ktt}{Ketum}
\define@key{names}{kyg}{Keyagana}
\define@key{names}{xkv}{Kgalagadi}
\define@key{names}{hkh}{Khah}
\define@key{names}{kbg}{Khamba}
\define@key{names}{kht}{Khamti}
\define@key{names}{ksu}{Khamyang}
\define@key{names}{khn}{Khandesi}
\define@key{names}{kjm}{Kháng}
\define@key{names}{ksy}{Kharia Thar}
\define@key{names}{kfw}{Kharam Naga}
\define@key{names}{lko}{Khayo}
\define@key{names}{kqg}{Khe}
\define@key{names}{tlx}{Khehek}
\define@key{names}{xkf}{Khengkha}
\define@key{names}{xhe}{Khetrani}
\define@key{names}{nkh}{Khezha Naga}
\define@key{names}{kix}{Khiamniungan Naga}
\define@key{names}{kwx}{Khirwar}
\define@key{names}{kqm}{Khisa}
\define@key{names}{ykl}{Khlula}
\define@key{names}{xkc}{Kho'ini}
\define@key{names}{nkb}{Khoibu}
\define@key{names}{ktc}{Kholok}
\define@key{names}{kho}{Khotanese}
\define@key{names}{khf}{Khuen}
\define@key{names}{kfm}{Khunsaric}
\define@key{names}{xco}{Khwarezmian}
\define@key{names}{kie}{Kibet}
\define@key{names}{prm}{Kibiri}
\define@key{names}{kzg}{Kikai}
\define@key{names}{kih}{Kilmeri}
\define@key{names}{kqr}{Kimaragang}
\define@key{names}{kmb}{Kimbundu}
\define@key{names}{kiv}{Kimbu}
\define@key{names}{sbt}{Kimki}
\define@key{names}{kqp}{Kimre}
\define@key{names}{krj}{Kinaray-a}
\define@key{names}{kco}{Kinalakna}
\define@key{names}{cbw}{Kinabalian}
\define@key{names}{knq}{Kintaq}
\define@key{names}{kkd}{Kinuku}
\define@key{names}{ues}{Kioko}
\define@key{names}{kkm}{Kiong}
\define@key{names}{apk}{Kiowa Apache}
\define@key{names}{sgc}{Kipsigis}
\define@key{names}{kyi}{Kiput}
\define@key{names}{kkr}{Kir-Balar}
\define@key{names}{okr}{Kirike}
\define@key{names}{kiu}{Kirmanjki}
\define@key{names}{fkk}{Kirya-Konzel}
\define@key{names}{lks}{Kisa}
\define@key{names}{kiz}{Kisi}
\define@key{names}{kis}{Kis}
\define@key{names}{zkt}{Kitan}
\define@key{names}{mwk}{Kita Maninkakan}
\define@key{names}{mkw}{Kituba (Congo)}
\define@key{names}{kqt}{Klias River Kadazan}
\define@key{names}{tlh}{Klingon}
\define@key{names}{kib}{Koalib-Rere}
\define@key{names}{kpd}{Koba}
\define@key{names}{kcj}{Kobiana}
\define@key{names}{kgu}{Kobol}
\define@key{names}{thq}{Kochila Tharu}
\define@key{names}{kdq}{Koch}
\define@key{names}{dhw}{Kochariya-East Danuwar}
\define@key{names}{cdz}{Koda}
\define@key{names}{ksz}{Kodaku}
\define@key{names}{vko}{Kodeoha}
\define@key{names}{kwp}{Kodia}
\define@key{names}{kod}{Kodi-Gaura}
\define@key{names}{kcs}{Koenoem}
\define@key{names}{kpi}{Kofei}
\define@key{names}{kwl}{Pan}
\define@key{names}{zkg}{Koguryo}
\define@key{names}{plk}{Kohistani Shina}
\define@key{names}{kkx}{Kohin}
\define@key{names}{kkt}{Koi}
\define@key{names}{nkd}{Koireng}
\define@key{names}{kxt}{Koiwat}
\define@key{names}{kou}{Koke}
\define@key{names}{gko}{Kok-Nar}
\define@key{names}{xod}{Kokoda}
\define@key{names}{kzn}{Kokola}
\define@key{names}{klc}{Kolbila}
\define@key{names}{ekl}{Kol (Bangladesh)}
\define@key{names}{biw}{Kol (Cameroon)}
\define@key{names}{skn}{Kolibugan Subanon}
\define@key{names}{klm}{Kolom}
\define@key{names}{kol}{Kol (Papua New Guinea)}
\define@key{names}{klx}{Koluwawa}
\define@key{names}{kmy}{Koma Ndera}
\define@key{names}{kpf}{Komba}
\define@key{names}{tyn}{Kombai}
\define@key{names}{kmm}{Kom (India)}
\define@key{names}{xoi}{Kominimung}
\define@key{names}{kmw}{Komo (Democratic Republic of Congo)}
\define@key{names}{kvh}{Komodo}
\define@key{names}{kvp}{Kompane}
\define@key{names}{kzv}{Komyandaret}
\define@key{names}{kxw}{Konai}
\define@key{names}{knd}{Yaben (Konda)}
\define@key{names}{kdw}{Koneraw}
\define@key{names}{klk}{Kono (Nigeria)}
\define@key{names}{kcz}{Konongo-Ruwila}
\define@key{names}{knu}{Kono (Guinea)}
\define@key{names}{kno}{Kono (Sierra Leone)}
\define@key{names}{koa}{Konomala}
\define@key{names}{kxc}{Konso}
\define@key{names}{nbe}{Konyak Naga}
\define@key{names}{mku}{Konyanka Maninka}
\define@key{names}{koo}{Konzo}
\define@key{names}{ozm}{Koonzime}
\define@key{names}{fuj}{Ko}
\define@key{names}{xop}{Kopar}
\define@key{names}{opk}{Kopkaka}
\define@key{names}{kcy}{Korandje}
\define@key{names}{koz}{Korak}
\define@key{names}{okh}{Karanic}
\define@key{names}{vkp}{Korlai Portuguese}
\define@key{names}{ktl}{Koroshi}
\define@key{names}{krp}{Korop}
\define@key{names}{kfo}{Koro (Côte d'Ivoire)}
\define@key{names}{krf}{Koro-Olrat}
\define@key{names}{xkq}{Koroni}
\define@key{names}{kqj}{Koromira}
\define@key{names}{jkr}{Koro}
\define@key{names}{vkn}{Koro Nulu}
\define@key{names}{vkz}{Koro Zuba}
\define@key{names}{kfd}{Korra Koraga}
\define@key{names}{kpq}{Korupun-Sela}
\define@key{names}{xor}{Korubo}
\define@key{names}{kfp}{Korwa}
\define@key{names}{kiq}{Kosadle}
\define@key{names}{kid}{Koshin}
\define@key{names}{kqk}{Kotafon Gbe}
\define@key{names}{koq}{Kota (Gabon)}
\define@key{names}{mqg}{Kota Bangun Kutai Malay}
\define@key{names}{grm}{Kota Marudu Talantang}
\define@key{names}{avk}{Kotava}
\define@key{names}{zko}{Kott-Assan}
\define@key{names}{kyf}{Kouya}
\define@key{names}{kqb}{Kovai}
\define@key{names}{kvc}{Kove}
\define@key{names}{xow}{Kowaki}
\define@key{names}{kwh}{Kowiai}
\define@key{names}{kga}{Koyaga}
\define@key{names}{koh}{Koyo}
\define@key{names}{kqd}{Koy Sanjaq Jewish Neo-Aramaic}
\define@key{names}{kuw}{Kpagua}
\define@key{names}{kpl}{Kpala}
\define@key{names}{pbn}{Kpasam}
\define@key{names}{koc}{Kpati}
\define@key{names}{cpo}{Kpeego}
\define@key{names}{kef}{Kpessi}
\define@key{names}{kph}{Kplang}
\define@key{names}{kye}{Krache}
\define@key{names}{rka}{Kraol}
\define@key{names}{xre}{Northeastern Timbira}
\define@key{names}{kri}{Krio}
\define@key{names}{kxb}{Krobu}
\define@key{names}{tyu}{Southern Tshwa}
\define@key{names}{yku}{Kuamasi}
\define@key{names}{uan}{Kuan}
\define@key{names}{kua}{Kuanyama}
\define@key{names}{ykn}{Kua-nsi}
\define@key{names}{ugh}{Kubachi}
\define@key{names}{kgf}{Kulungtfu-Yuanggeng-Tobo}
\define@key{names}{kof}{Kubi}
\define@key{names}{jko}{Kubo}
\define@key{names}{kvb}{Kubu}
\define@key{names}{lkc}{Kucong}
\define@key{names}{kfg}{Kudiya}
\define@key{names}{kyw}{Kudmali}
\define@key{names}{kov}{Kudu-Camo}
\define@key{names}{kow}{Gengle-Kugama}
\define@key{names}{kes}{Kugbo}
\define@key{names}{dkr}{Kuijau}
\define@key{names}{vkj}{Kujarge}
\define@key{names}{kux}{Kukatja}
\define@key{names}{kez}{Kukele}
\define@key{names}{kfn}{Kuk}
\define@key{names}{ugb}{Kuku-Ugbanh}
\define@key{names}{xmp}{Kuku-Mu'inh}
\define@key{names}{xmh}{Kuku-Muminh}
\define@key{names}{ukv}{Kuku}
\define@key{names}{kul}{Kulere}
\define@key{names}{kxj}{Kulfa}
\define@key{names}{vkl}{Kulisusu}
\define@key{names}{xpk}{Kulina Pano}
\define@key{names}{kfx}{Kullu Pahari}
\define@key{names}{pzh}{Pazeh-Kahabu}
\define@key{names}{uon}{Kulon}
\define@key{names}{bbu}{Kulung (Nigeria)}
\define@key{names}{kdi}{Kumam}
\define@key{names}{ksl}{Kumalu}
\define@key{names}{ksm}{Kumba}
\define@key{names}{xks}{Kumbewaha}
\define@key{names}{kra}{Kumhali}
\define@key{names}{kuo}{Kumukio}
\define@key{names}{zum}{Kumzari}
\define@key{names}{wku}{Kunduvadi}
\define@key{names}{kdn}{Chikunda}
\define@key{names}{shd}{Kundal Shahi}
\define@key{names}{kgl}{Kunggari}
\define@key{names}{ggk}{Kungarakany}
\define@key{names}{kfl}{Kung}
\define@key{names}{kse}{Kuni}
\define@key{names}{xug}{Kunigami}
\define@key{names}{pep}{Kánchá}
\define@key{names}{njx}{Kunyi}
\define@key{names}{kug}{Kupa}
\define@key{names}{mkn}{Kupang Malay}
\define@key{names}{key}{Kupia}
\define@key{names}{nqk}{Kura Ede Nago}
\define@key{names}{krh}{Kurama}
\define@key{names}{kfh}{Kurichiya}
\define@key{names}{kuj}{Kuria}
\define@key{names}{nbn}{Nabi}
\define@key{names}{kfv}{Kurmukar}
\define@key{names}{vku}{Kurrama}
\define@key{names}{kuv}{Kur}
\define@key{names}{xkz}{Kurtokha}
\define@key{names}{ktm}{Kurti}
\define@key{names}{kjr}{Kurudu}
\define@key{names}{kyr}{Kuruáya}
\define@key{names}{kus}{Kusaal}
\define@key{names}{ksg}{Kusaghe-Njela}
\define@key{names}{kuh}{Kushi}
\define@key{names}{ksv}{Kusu}
\define@key{names}{ght}{Kutang Ghale}
\define@key{names}{kub}{Kutep}
\define@key{names}{xut}{Kuthant}
\define@key{names}{kpa}{Kutto}
\define@key{names}{khj}{Kuturmi}
\define@key{names}{kdc}{Kutu}
\define@key{names}{uky}{Kuuk-Yak}
\define@key{names}{lku}{Kuungkari of Barcoo River}
\define@key{names}{olu}{Kuvale}
\define@key{names}{cwt}{Kuwaataay}
\define@key{names}{blh}{Kuwaa}
\define@key{names}{kdt}{Kuy}
\define@key{names}{fkv}{Kven Finnish}
\define@key{names}{kwb}{Baa}
\define@key{names}{bko}{Kwa'}
\define@key{names}{kwz}{Kwadi}
\define@key{names}{wka}{Kw'adza}
\define@key{names}{kdz}{Kwaja-Ndaktup}
\define@key{names}{kwu}{Kwakum}
\define@key{names}{qwt}{Kwalhioqua-Clatskanie}
\define@key{names}{kmq}{Gwama}
\define@key{names}{ktf}{Kwami}
\define@key{names}{kwm}{Kwambi}
\define@key{names}{okk}{Kwamtim One}
\define@key{names}{knp}{Kwanja}
\define@key{names}{kwj}{Kwanga}
\define@key{names}{kvi}{Kwang}
\define@key{names}{xdo}{Kwandu}
\define@key{names}{kwf}{Kwara'ae}
\define@key{names}{kop}{Kwato}
\define@key{names}{kya}{Kwaya}
\define@key{names}{cwe}{Kwere}
\define@key{names}{xwr}{Kwerba Mamberamo}
\define@key{names}{kkb}{Kwerisa}
\define@key{names}{kwr}{Kwer}
\define@key{names}{kws}{Kwese}
\define@key{names}{kwt}{Kwesten}
\define@key{names}{kuc}{Kwinsu}
\define@key{names}{kww}{Kwinti}
\define@key{names}{bka}{Kyak}
\define@key{names}{tye}{Kyenga}
\define@key{names}{kql}{Kyenele}
\define@key{names}{ldn}{Láadan}
\define@key{names}{bwj}{Láá Láá Bwamu}
\define@key{names}{ldi}{Laari}
\define@key{names}{lbb}{Label}
\define@key{names}{lbi}{La'bi}
\define@key{names}{jku}{Labir}
\define@key{names}{ypb}{Labo Phowa}
\define@key{names}{mwi}{Ninde}
\define@key{names}{dtb}{Labuk-Kinabatangan Kadazan}
\define@key{names}{zpl}{Lachixío Zapotec}
\define@key{names}{zpa}{Lachiguiri Zapotec}
\define@key{names}{lkl}{Laeko-Libuat}
\define@key{names}{lgh}{Laghuu}
\define@key{names}{lgb}{Laghu}
\define@key{names}{lhh}{Laha (Indonesia)}
\define@key{names}{lhn}{Lahanan}
\define@key{names}{lhl}{Lahul Lohar}
\define@key{names}{lhi}{Lahu Shi}
\define@key{names}{lmx}{Laimbue}
\define@key{names}{lji}{Laiyolo}
\define@key{names}{lap}{Laka (Chad)}
\define@key{names}{lka}{Lakalei}
\define@key{names}{lkh}{Lakha}
\define@key{names}{lki}{Laki}
\define@key{names}{lkn}{Lakon}
\define@key{names}{lkd}{Lakondê}
\define@key{names}{lxm}{Lakuramau}
\define@key{names}{lla}{Lala-Roba}
\define@key{names}{leb}{Lala-Bisa}
\define@key{names}{cnl}{Lalana Chinantec}
\define@key{names}{las}{Lama (Togo)}
\define@key{names}{lmr}{Peripheral Lembata}
\define@key{names}{lmq}{Lamatuka}
\define@key{names}{lai}{Lambya}
\define@key{names}{lmy}{Lamboya}
\define@key{names}{quf}{Lambayeque Quechua}
\define@key{names}{lbn}{Lamet}
\define@key{names}{bma}{Lame}
\define@key{names}{ldh}{Lamja-Dengsa-Tola}
\define@key{names}{lmk}{Lamkang}
\define@key{names}{lev}{Western Pantar}
\define@key{names}{lmg}{Lamogai}
\define@key{names}{abl}{Lampung Nyo}
\define@key{names}{llh}{Lamu}
\define@key{names}{ruu}{Lanas Lobu}
\define@key{names}{ldm}{Landoma}
\define@key{names}{sfb}{Langue des signes de Belgique Francophone}
\define@key{names}{yln}{Langnian Buyang}
\define@key{names}{lna}{Langbashe}
\define@key{names}{lno}{Lango-Logire-Logir}
\define@key{names}{lnm}{Pondi}
\define@key{names}{lnh}{Lanoh}
\define@key{names}{lwm}{Laomian}
\define@key{names}{ztl}{Lapaguía-Guivini Zapotec}
\define@key{names}{laa}{Lapuyan Subanun}
\define@key{names}{lrt}{Larantuka Malay}
\define@key{names}{lrv}{Larevat}
\define@key{names}{hmd}{Diandongbei-Large Flowery Miao}
\define@key{names}{lrl}{Larestani}
\define@key{names}{lro}{Laru (North Sudan)}
\define@key{names}{lar}{Larteh}
\define@key{names}{lan}{Laru (Nigeria)}
\define@key{names}{llm}{Lasalimu}
\define@key{names}{lsa}{Lasgerdi}
\define@key{names}{lsi}{Lashi}
\define@key{names}{lss}{Lasi}
\define@key{names}{lat}{Latin}
\define@key{names}{ltu}{Latu}
\define@key{names}{ltn}{Latundê}
\define@key{names}{lsl}{Latvian Sign Language}
\define@key{names}{llx}{Lauan}
\define@key{names}{luf}{Laua}
\define@key{names}{lre}{Laurentian}
\define@key{names}{clt}{Lautu}
\define@key{names}{lbv}{Lavatbura-Lamusong}
\define@key{names}{lbx}{Lawangan}
\define@key{names}{lvi}{Lawi}
\define@key{names}{tgi}{Lawunuia}
\define@key{names}{lwu}{Lawu}
\define@key{names}{lya}{Layakha}
\define@key{names}{ldk}{Leelau}
\define@key{names}{lfa}{Lefa}
\define@key{names}{lgm}{Lega-Mwenga}
\define@key{names}{lcc}{Legenyem}
\define@key{names}{cae}{Lehar}
\define@key{names}{tql}{Lehali}
\define@key{names}{urr}{Lehalurup}
\define@key{names}{lzn}{Leinong Naga}
\define@key{names}{lek}{Leipon}
\define@key{names}{llk}{Lelak}
\define@key{names}{lel}{Lele (Democratic Republic of Congo)}
\define@key{names}{llc}{Lele (Guinea)}
\define@key{names}{lpa}{Lelepa}
\define@key{names}{lle}{Lele (Papua New Guinea)}
\define@key{names}{leq}{Lembena}
\define@key{names}{lrz}{Lemerig}
\define@key{names}{lei}{Lemio}
\define@key{names}{xle}{Lemnian}
\define@key{names}{ldj}{Lemoro}
\define@key{names}{ley}{Lemolang}
\define@key{names}{lej}{Lengola}
\define@key{names}{lgr}{Lengo}
\define@key{names}{lgi}{Lengilu}
\define@key{names}{leh}{Lenje}
\define@key{names}{ler}{Lenkau}
\define@key{names}{ldg}{Lenyima}
\define@key{names}{lpe}{Lepki}
\define@key{names}{xlp}{Lepontic}
\define@key{names}{gnh}{Lere}
\define@key{names}{let}{Lesing-Gelimi}
\define@key{names}{nms}{Letemboi-Repanbitip}
\define@key{names}{leo}{Leti (Cameroon)}
\define@key{names}{lvu}{Central Lembata-Lewokukun}
\define@key{names}{lwe}{Lewo Eleng}
\define@key{names}{lwt}{Lewotobi}
\define@key{names}{ayi}{Leyigha}
\define@key{names}{lhp}{Lhokpu}
\define@key{names}{lix}{Liabuku}
\define@key{names}{njn}{Liangmai Naga}
\define@key{names}{zln}{Lianshan Zhuang}
\define@key{names}{ste}{Liana-Seti}
\define@key{names}{lir}{Kru Pidgin English}
\define@key{names}{liz}{Libinza}
\define@key{names}{liq}{Libido}
\define@key{names}{lbs}{Libyan Sign Language}
\define@key{names}{lig}{Ligbi}
\define@key{names}{lgz}{Ligenza}
\define@key{names}{lih}{Lihir}
\define@key{names}{mgi}{Lijili}
\define@key{names}{lik}{Liko}
\define@key{names}{lie}{Balobo}
\define@key{names}{lio}{Liki}
\define@key{names}{kxx}{Likuba}
\define@key{names}{lib}{Likum}
\define@key{names}{kwc}{Likwala}
\define@key{names}{lll}{Lilau}
\define@key{names}{bme}{Limassa}
\define@key{names}{lim}{Limburgan}
\define@key{names}{lmp}{Limbum}
\define@key{names}{ylm}{Limi}
\define@key{names}{kmk}{Limos Kalinga}
\define@key{names}{qlm}{Limonese Creole}
\define@key{names}{klw}{Tado-Lindu}
\define@key{names}{pml}{Mediterranean Lingua Franca}
\define@key{names}{onb}{Western Ong-Be}
\define@key{names}{lgk}{Neverver}
\define@key{names}{lfn}{Lingua Franca Nova}
\define@key{names}{ljl}{Li'o}
\define@key{names}{apl}{Lipan Apache}
\define@key{names}{lpo}{Lipo}
\define@key{names}{lcs}{Lisabata-Nuniali}
\define@key{names}{lcl}{Lisela}
\define@key{names}{lsh}{Khispi}
\define@key{names}{lsd}{Lishana Deni}
\define@key{names}{lzh}{Literary Chinese}
\define@key{names}{lls}{Lithuanian Sign Language}
\define@key{names}{lzl}{Naman}
\define@key{names}{zlj}{Liujiang Zhuang}
\define@key{names}{zlq}{Liuqian Zhuang}
\define@key{names}{olo}{Livvi}
\define@key{names}{loq}{Lobala}
\define@key{names}{lbm}{Lodhi}
\define@key{names}{lgq}{Ikpana}
\define@key{names}{rag}{Logooli}
\define@key{names}{liu}{Logorik}
\define@key{names}{lof}{Logol}
\define@key{names}{src}{Logudorese Sardinian}
\define@key{names}{qvj}{Loja Highland Quichua}
\define@key{names}{jbo}{Lojban}
\define@key{names}{yaz}{Lokaa}
\define@key{names}{lky}{Lokoya}
\define@key{names}{lcd}{Lola}
\define@key{names}{llq}{Lolak}
\define@key{names}{llg}{Lole}
\define@key{names}{ycl}{Lolopo}
\define@key{names}{llb}{Lolo}
\define@key{names}{loa}{Loloda-Laba}
\define@key{names}{rmi}{Lomavren}
\define@key{names}{loi}{Loma (Côte d'Ivoire)}
\define@key{names}{lmv}{Lomaiviti}
\define@key{names}{lmi}{Lombi}
\define@key{names}{lmo}{Lombard}
\define@key{names}{loo}{Lombo}
\define@key{names}{ngl}{Mozambique Lomwe}
\define@key{names}{lce}{Loncong}
\define@key{names}{lpn}{Long Phuri Naga}
\define@key{names}{wok}{Longto}
\define@key{names}{lnu}{Longuda}
\define@key{names}{ttw}{Western Lowland Kenyah}
\define@key{names}{ldo}{Loo}
\define@key{names}{lop}{Lopa}
\define@key{names}{lpx}{Lopit}
\define@key{names}{lrn}{Lorang}
\define@key{names}{spq}{Peruvian Amazonian Spanish}
\define@key{names}{lnn}{Nethalp}
\define@key{names}{uvl}{Lote}
\define@key{names}{lht}{Lo-Toga}
\define@key{names}{dtr}{Lotud}
\define@key{names}{lou}{Louisiana Creole French}
\define@key{names}{lox}{Loun}
\define@key{names}{xlo}{Loup A}
\define@key{names}{sli}{Lower Silesian}
\define@key{names}{tto}{Lower Ta'oih}
\define@key{names}{nsb}{Lower-Nosop}
\define@key{names}{kml}{Tanudan Kalinga}
\define@key{names}{cea}{Lower Chehalis}
\define@key{names}{axl}{Lower Southern Aranda}
\define@key{names}{ztp}{Loxicha Zapotec}
\define@key{names}{kcc}{Lubila}
\define@key{names}{lcf}{Lubu}
\define@key{names}{knb}{Lubuagan Kalinga}
\define@key{names}{luq}{Lucumi}
\define@key{names}{lud}{Ludian}
\define@key{names}{ldq}{Lufu}
\define@key{names}{ruf}{Luguru}
\define@key{names}{lcq}{Luhu-Piru}
\define@key{names}{lum}{Luimbi}
\define@key{names}{dop}{Lukpa}
\define@key{names}{smj}{Lule Saami}
\define@key{names}{lmz}{Lumbee}
\define@key{names}{lup}{Lumbu}
\define@key{names}{lmd}{Lumun}
\define@key{names}{luk}{Lunanakha}
\define@key{names}{luj}{Luna}
\define@key{names}{lga}{Lungga}
\define@key{names}{luw}{Luo (Cameroon)}
\define@key{names}{hml}{Luopohe Hmong}
\define@key{names}{ldd}{Luri}
\define@key{names}{lse}{Lusengo}
\define@key{names}{xls}{Lusitanian}
\define@key{names}{ndy}{Lutos}
\define@key{names}{luv}{Luwati}
\define@key{names}{lyn}{Luyi}
\define@key{names}{lwa}{Lwalu}
\define@key{names}{xlc}{Lycian A}
\define@key{names}{xld}{Lydian}
\define@key{names}{lyg}{India Lyngam}
\define@key{names}{cma}{Maa}
\define@key{names}{mew}{Maaka}
\define@key{names}{ymm}{Maay}
\define@key{names}{mmz}{Mabaale}
\define@key{names}{mfz}{Mabaan}
\define@key{names}{mqa}{Maba (Indonesia)}
\define@key{names}{kkg}{Mabaka Valley Kalinga}
\define@key{names}{muj}{Mabire}
\define@key{names}{mcl}{Macaguaje}
\define@key{names}{mzs}{Macanese}
\define@key{names}{mvw}{Machinga}
\define@key{names}{jmc}{Machame}
\define@key{names}{mpd}{Machinere}
\define@key{names}{wpc}{Maco}
\define@key{names}{mzc}{Madagascar Sign Language}
\define@key{names}{mmx}{Madak}
\define@key{names}{xmx}{Salawati}
\define@key{names}{grg}{Madi (Papua New Guinea)}
\define@key{names}{kmd}{Madukayang Kalinga}
\define@key{names}{mme}{Tirax}
\define@key{names}{itt}{Maeng Itneg}
\define@key{names}{maf}{Mafa}
\define@key{names}{mkv}{Mafea}
\define@key{names}{sgb}{Mag-Anchi Ayta}
\define@key{names}{mtw}{Southern Binukidnon}
\define@key{names}{xtm}{Magdalena Peñasco Mixtec}
\define@key{names}{gmd}{Mághdì}
\define@key{names}{blx}{Mag-Indi Ayta}
\define@key{names}{gkd}{Magi}
\define@key{names}{gmg}{Magiyi}
\define@key{names}{gmx}{Magoma}
\define@key{names}{zgr}{Magori}
\define@key{names}{bfz}{Mahasu Pahari}
\define@key{names}{mjx}{Mahali}
\define@key{names}{pmh}{Maharastri Prakrit}
\define@key{names}{mjy}{Mohican}
\define@key{names}{mhb}{Mahongwe}
\define@key{names}{mzz}{Maiadomu}
\define@key{names}{tnh}{Maiani}
\define@key{names}{sks}{Maia}
\define@key{names}{mmm}{Maii}
\define@key{names}{vmf}{Ostfränkisch}
\define@key{names}{cwb}{Maindo}
\define@key{names}{xkl}{Usun Apau Kenyah}
\define@key{names}{mum}{Maiwala}
\define@key{names}{wmm}{Maiwa (Indonesia)}
\define@key{names}{mti}{Maiwa (Papua New Guinea)}
\define@key{names}{xmj}{Majera}
\define@key{names}{mmj}{Majhwar}
\define@key{names}{mjz}{Majhi}
\define@key{names}{mfp}{Makassar Malay}
\define@key{names}{aup}{Makayam}
\define@key{names}{mkg}{Mak (China)}
\define@key{names}{vmk}{Makhuwa-Shirima}
\define@key{names}{xmc}{Makhuwa-Marrevone}
\define@key{names}{vmw}{Makhuwa}
\define@key{names}{mhm}{Makhuwa-Moniga}
\define@key{names}{xsq}{Makhuwa-Saka}
\define@key{names}{pbl}{Mak (Nigeria)}
\define@key{names}{zmh}{Makolkol}
\define@key{names}{jmn}{Makuri Naga}
\define@key{names}{lva}{Maku'a}
\define@key{names}{mpu}{Makuráp}
\define@key{names}{ymk}{Makwe}
\define@key{names}{umn}{Makyan Naga}
\define@key{names}{lon}{Malawi Lomwe}
\define@key{names}{xml}{Malaysian Sign Language}
\define@key{names}{ima}{Mala Malasar}
\define@key{names}{ymr}{Malasar}
\define@key{names}{mjo}{Malankuravan}
\define@key{names}{mjr}{Malavedan}
\define@key{names}{mjq}{Malaryan}
\define@key{names}{mjp}{Malapandaram}
\define@key{names}{ruy}{Mala (Nigeria)}
\define@key{names}{swk}{Malawi Sena}
\define@key{names}{ccm}{Malaccan Creole Malay}
\define@key{names}{mln}{Malango}
\define@key{names}{mqz}{Malasanga}
\define@key{names}{mmt}{Malalamai}
\define@key{names}{ped}{Mala (Papua New Guinea)}
\define@key{names}{mkr}{Manep}
\define@key{names}{lws}{Malawian Sign Language}
\define@key{names}{bfo}{Malba Birifor}
\define@key{names}{pkt}{Maleng}
\define@key{names}{mdc}{Male (Papua New Guinea)}
\define@key{names}{gut}{Maléku Jaíka}
\define@key{names}{mlx}{Na'ahai}
\define@key{names}{vml}{Malgana}
\define@key{names}{mxf}{Malgbe}
\define@key{names}{mgq}{Malila}
\define@key{names}{mzd}{Malimba}
\define@key{names}{mli}{Malimpung}
\define@key{names}{mlf}{Mal}
\define@key{names}{mbk}{Malol}
\define@key{names}{mkb}{Mar Paharia of Dumka}
\define@key{names}{mdl}{Maltese Sign Language}
\define@key{names}{mll}{Malua Bay}
\define@key{names}{mup}{Malvi}
\define@key{names}{myk}{Mamara Senoufo}
\define@key{names}{mma}{Mama}
\define@key{names}{mhf}{Mamaa}
\define@key{names}{wmd}{Mamaindé}
\define@key{names}{mvd}{Mamboru}
\define@key{names}{mgm}{Mambae}
\define@key{names}{kdf}{Mamusi}
\define@key{names}{mqx}{Mamuju}
\define@key{names}{znk}{Manangkari}
\define@key{names}{mjl}{Mandeali}
\define@key{names}{mha}{Manda (India)}
\define@key{names}{zma}{Manda (Australia)}
\define@key{names}{zmk}{Mandandanyi}
\define@key{names}{mgs}{Manda-Matumba}
\define@key{names}{mqu}{Mandari}
\define@key{names}{tbf}{Mandara}
\define@key{names}{mqr}{Mander}
\define@key{names}{aax}{Mandobo Atas}
\define@key{names}{bwp}{Mandobo Bawah}
\define@key{names}{mht}{Mandahuaca}
\define@key{names}{zng}{Mang}
\define@key{names}{zme}{Mangerr}
\define@key{names}{mem}{Mangala}
\define@key{names}{myj}{Mangayat}
\define@key{names}{mdk}{Mangbutu}
\define@key{names}{kby}{Manga Kanuri}
\define@key{names}{mrv}{Mangareva}
\define@key{names}{mbh}{Mangseng}
\define@key{names}{mmo}{Mangga Buang}
\define@key{names}{zns}{Mangas}
\define@key{names}{xkb}{Manigri-Kambolé Ede Nago}
\define@key{names}{mqp}{Manipa}
\define@key{names}{nlm}{Mankiyali}
\define@key{names}{mml}{Man Met}
\define@key{names}{mjv}{Mannan}
\define@key{names}{woo}{Manombai}
\define@key{names}{msw}{Mansoanka}
\define@key{names}{msk}{Mansaka}
\define@key{names}{nty}{Mantsi}
\define@key{names}{myg}{Manta}
\define@key{names}{kxf}{Manumanaw Karen}
\define@key{names}{wha}{Manusela}
\define@key{names}{mxc}{Manyika}
\define@key{names}{mny}{Manyawa}
\define@key{names}{mzj}{Manya}
\define@key{names}{mzv}{Manza}
\define@key{names}{mmd}{Maonan}
\define@key{names}{mjn}{Ma (Papua New Guinea)}
\define@key{names}{mlh}{Mape}
\define@key{names}{mnm}{Mapena}
\define@key{names}{mpy}{Mapia}
\define@key{names}{mpw}{Mapidian-Mawayana}
\define@key{names}{bzh}{Mapos Buang}
\define@key{names}{sjm}{Mapun}
\define@key{names}{vmh}{Maraghei}
\define@key{names}{nma}{Maram Naga}
\define@key{names}{lrm}{Marama}
\define@key{names}{lri}{Marachi}
\define@key{names}{mgb}{Mararit}
\define@key{names}{mvr}{Marau}
\define@key{names}{mrs}{Maragus}
\define@key{names}{mpg}{Marba}
\define@key{names}{dsz}{Mardin Sign Language}
\define@key{names}{vmr}{Marenje}
\define@key{names}{mrx}{Maremgi}
\define@key{names}{mvu}{Marfa}
\define@key{names}{mhg}{Margu}
\define@key{names}{qvm}{Margos-Yarowilca-Lauricocha Quechua}
\define@key{names}{mfm}{Marghi South}
\define@key{names}{nsr}{Maritime Sign Language}
\define@key{names}{mrr}{Maria (India)}
\define@key{names}{nng}{Maring Naga}
\define@key{names}{zmm}{Marimanindji}
\define@key{names}{zmj}{Maridjabin}
\define@key{names}{zmd}{Maridan}
\define@key{names}{zmy}{Mariyedi}
\define@key{names}{mrb}{Sunwadia}
\define@key{names}{dad}{Marik}
\define@key{names}{hob}{Mari (Madang Province)}
\define@key{names}{mqi}{Mariri}
\define@key{names}{mbx}{Mari (East Sepik Province)}
\define@key{names}{mds}{Maria (Papua New Guinea)}
\define@key{names}{msp}{Maritsauá}
\define@key{names}{enb}{Markweeta}
\define@key{names}{rkm}{Marka}
\define@key{names}{mvo}{Marovo}
\define@key{names}{xru}{Marriammu}
\define@key{names}{mre}{Martha's Vineyard Sign Language}
\define@key{names}{zmg}{Marti Ke}
\define@key{names}{mzr}{Marúbo}
\define@key{names}{mve}{Marwari (Pakistan)}
\define@key{names}{rwr}{Marwari (India)}
\define@key{names}{myx}{Masaaba}
\define@key{names}{tis}{Masadiit Itneg}
\define@key{names}{bks}{Masbate Sorsogon}
\define@key{names}{msb}{Masbatenyo}
\define@key{names}{mho}{Mashi (Zambia)}
\define@key{names}{jms}{Mashi (Nigeria)}
\define@key{names}{cuj}{Mashco Piro}
\define@key{names}{ism}{Masimasi}
\define@key{names}{bnf}{Masiwang}
\define@key{names}{msh}{Masikoro Malagasy}
\define@key{names}{klv}{Maskelynes}
\define@key{names}{msv}{Maslam}
\define@key{names}{mes}{Masmaje}
\define@key{names}{mdg}{Massalat}
\define@key{names}{mvs}{Massep}
\define@key{names}{mtn}{Matagalpa}
\define@key{names}{mfh}{Matal}
\define@key{names}{xmt}{Matbat}
\define@key{names}{mgv}{Matengo}
\define@key{names}{mqe}{Matepi}
\define@key{names}{mzo}{Matipuhy}
\define@key{names}{mtm}{Mator-Taigi-Karagas}
\define@key{names}{met}{Mato}
\define@key{names}{axg}{Mato Grosso Arára}
\define@key{names}{stj}{Matya Samo}
\define@key{names}{cty}{Maundadan Chetti}
\define@key{names}{lsy}{Mauritian Sign Language}
\define@key{names}{mhl}{Mauwake}
\define@key{names}{wma}{Mawa (Nigeria)}
\define@key{names}{mjj}{Mawak}
\define@key{names}{mcz}{Mawan}
\define@key{names}{mcw}{Mawa (Chad)}
\define@key{names}{mgk}{Mawes}
\define@key{names}{mxl}{Maxi Gbe}
\define@key{names}{xmy}{Mayaguduna}
\define@key{names}{sym}{Maya Samo}
\define@key{names}{mnt}{Maykulan}
\define@key{names}{ifu}{Mayoyao Ifugao}
\define@key{names}{mzl}{Mazatlán Mixe}
\define@key{names}{zpy}{Mazaltepec Zapotec}
\define@key{names}{vmz}{Mazatlán Mazatec}
\define@key{names}{dkx}{Mazagway}
\define@key{names}{mdp}{Mbala}
\define@key{names}{mgn}{Mbangi}
\define@key{names}{zmz}{Mbandja}
\define@key{names}{mxg}{Mbangala}
\define@key{names}{zmn}{Mbangwe}
\define@key{names}{zmv}{Rimanggudhinma}
\define@key{names}{mvl}{Mbara-Yanga}
\define@key{names}{gwa}{Mbato}
\define@key{names}{mdn}{Mbati}
\define@key{names}{xmd}{Mbedam}
\define@key{names}{mfo}{Mbe}
\define@key{names}{mql}{Mbelime}
\define@key{names}{zms}{Mbesa}
\define@key{names}{emz}{Mbessa}
\define@key{names}{mbo}{Mbo (Cameroon)}
\define@key{names}{zmw}{Mbo (Democratic Republic of Congo)}
\define@key{names}{moi}{Mboi}
\define@key{names}{mdu}{Mboko}
\define@key{names}{xmb}{Mbonga}
\define@key{names}{bgu}{Mbongno}
\define@key{names}{mxo}{Mbowe}
\define@key{names}{mka}{Mbre}
\define@key{names}{mgz}{Mbugwe}
\define@key{names}{mhw}{Mbukushu}
\define@key{names}{mqb}{Mbuko}
\define@key{names}{bpc}{Mbuk}
\define@key{names}{mbv}{Mbulungish}
\define@key{names}{mbu}{Mbula-Bwazza}
\define@key{names}{mlb}{Mbule}
\define@key{names}{mgy}{Mbunga}
\define@key{names}{mck}{Mbunda}
\define@key{names}{bbt}{Mburku}
\define@key{names}{muc}{Ajumbu}
\define@key{names}{mfu}{Mbwela}
\define@key{names}{gun}{Mbyá Guaraní}
\define@key{names}{mjm}{Medebur}
\define@key{names}{dmf}{Medefidrin}
\define@key{names}{mue}{Media Lengua}
\define@key{names}{mud}{Mednyj Aleut}
\define@key{names}{byv}{Medumba}
\define@key{names}{mfj}{Mefele}
\define@key{names}{mef}{Bangladesh Lyngam}
\define@key{names}{ruq}{Megleno Romanian}
\define@key{names}{mmh}{Mehináku}
\define@key{names}{mvk}{Mekmek}
\define@key{names}{msf}{Mekwei}
\define@key{names}{hkn}{Mel-Khaonh}
\define@key{names}{mfx}{Melo}
\define@key{names}{med}{Melpa}
\define@key{names}{mby}{Memoni}
\define@key{names}{mfd}{Mendankwe-Nkwen}
\define@key{names}{xkd}{Mendalam Kayan}
\define@key{names}{sim}{Mende (Papua New Guinea)}
\define@key{names}{xmg}{Mengaka}
\define@key{names}{mee}{Mengen}
\define@key{names}{mea}{Menka}
\define@key{names}{mvx}{Meoswar}
\define@key{names}{mxm}{Meramera}
\define@key{names}{lmb}{Merei}
\define@key{names}{meq}{Merey}
\define@key{names}{mrm}{Merlav}
\define@key{names}{xmr}{Meroitic}
\define@key{names}{mnu}{Mer}
\define@key{names}{mer}{Meru}
\define@key{names}{wry}{Merwari}
\define@key{names}{iyo}{Mesaka}
\define@key{names}{mci}{Mese}
\define@key{names}{zim}{Mesme}
\define@key{names}{mys}{Mesmes}
\define@key{names}{mvz}{Mesqan}
\define@key{names}{cms}{Messapic}
\define@key{names}{mgo}{Meta'}
\define@key{names}{mxv}{Metlatónoc Mixtec}
\define@key{names}{mtr}{Mewari}
\define@key{names}{wtm}{Mewati}
\define@key{names}{mfs}{Mexican Sign Language}
\define@key{names}{zmf}{Mfinu}
\define@key{names}{nfu}{Southern Mfumte}
\define@key{names}{zam}{Cuixtla-Xitla Zapotec}
\define@key{names}{pla}{Miani}
\define@key{names}{xmi}{Miarrã}
\define@key{names}{mmc}{Michoacán Mazahua}
\define@key{names}{enm}{Middle English}
\define@key{names}{gml}{Middle Low German}
\define@key{names}{dum}{Middle Dutch}
\define@key{names}{mpl}{Middle Watut}
\define@key{names}{gmh}{Middle High German}
\define@key{names}{ltc}{Middle Chinese}
\define@key{names}{xng}{Middle Mongol}
\define@key{names}{dnt}{Mid Grand Valley Dani}
\define@key{names}{bjo}{Mid-Southern Banda}
\define@key{names}{mpp}{Migabac}
\define@key{names}{ymh}{Mili}
\define@key{names}{mlj}{Miltu}
\define@key{names}{iml}{Miluk}
\define@key{names}{imy}{Milyan}
\define@key{names}{mcv}{Minanibai-Foia Foia}
\define@key{names}{inm}{Minaean}
\define@key{names}{mnp}{Min Bei Chinese}
\define@key{names}{mpn}{Mindiri}
\define@key{names}{drc}{Minderico}
\define@key{names}{mko}{Mingang Doso}
\define@key{names}{vmg}{Minigir}
\define@key{names}{wii}{Minidien}
\define@key{names}{xxm}{Minkin}
\define@key{names}{omn}{Minoan}
\define@key{names}{mqq}{Minokok}
\define@key{names}{mnq}{Minriq}
\define@key{names}{mzt}{Mintil}
\define@key{names}{czo}{Min Zhong Chinese}
\define@key{names}{zgm}{Minz Zhuang}
\define@key{names}{yiq}{Miqie}
\define@key{names}{mwl}{Mirandese}
\define@key{names}{mvh}{Mire}
\define@key{names}{mmv}{Miriti}
\define@key{names}{rsm}{Miriwoong Sign Language}
\define@key{names}{mjs}{Miship}
\define@key{names}{mpx}{Misima-Paneati}
\define@key{names}{vmm}{Mitlatongo Mixtec}
\define@key{names}{mwu}{Mittu}
\define@key{names}{mpo}{Miu}
\define@key{names}{vmi}{Miwa}
\define@key{names}{mfg}{Mixifore}
\define@key{names}{mix}{Mixtepec Mixtec}
\define@key{names}{mvi}{Miyako}
\define@key{names}{ehs}{Miyakubo Sign Language}
\define@key{names}{soy}{Miyobe}
\define@key{names}{lhs}{Mlahsô}
\define@key{names}{kja}{Mlap}
\define@key{names}{mlo}{Mlomp}
\define@key{names}{mmu}{Mmaala}
\define@key{names}{bfm}{Mmen}
\define@key{names}{mfq}{Moba}
\define@key{names}{mod}{Mobilian}
\define@key{names}{ahm}{Mobumrin Aizi}
\define@key{names}{jkm}{Mobwa Karen}
\define@key{names}{mhn}{Mòcheno}
\define@key{names}{mhc}{Mocho}
\define@key{names}{gbn}{Mo'da}
\define@key{names}{mxd}{Modang}
\define@key{names}{mqo}{Modole}
\define@key{names}{mvq}{Moere}
\define@key{names}{mou}{Mogum}
\define@key{names}{mof}{Mohegan-Montauk-Narragansett}
\define@key{names}{mow}{Moi (Congo)}
\define@key{names}{mxn}{Moi (Indonesia)}
\define@key{names}{mkp}{Moikodi}
\define@key{names}{mwz}{Moingi}
\define@key{names}{ymi}{Moji}
\define@key{names}{mft}{Mokerang}
\define@key{names}{mwt}{Moken}
\define@key{names}{mqt}{Mok}
\define@key{names}{mkm}{Moklen}
\define@key{names}{mkl}{Mokole}
\define@key{names}{vms}{Moksela}
\define@key{names}{pwm}{Molbog}
\define@key{names}{vsi}{Moldova Sign Language}
\define@key{names}{bxc}{Molengue}
\define@key{names}{mox}{Molima}
\define@key{names}{zmo}{Molo}
\define@key{names}{msl}{Molof}
\define@key{names}{mlw}{Moloko}
\define@key{names}{myl}{Moma}
\define@key{names}{msz}{Momare}
\define@key{names}{dmb}{Mombo Dogon}
\define@key{names}{mmb}{Momina}
\define@key{names}{ver}{Mom Jango}
\define@key{names}{mzg}{Monastic Sign Language}
\define@key{names}{npn}{Mondropolon}
\define@key{names}{msr}{Mongolian Sign Language}
\define@key{names}{mgt}{Mwakai}
\define@key{names}{mom}{Mangue}
\define@key{names}{moo}{Monom}
\define@key{names}{mru}{Mono (Cameroon)}
\define@key{names}{mnh}{Mono (Democratic Republic of Congo)}
\define@key{names}{nmh}{Monsang Naga}
\define@key{names}{mtl}{Montol}
\define@key{names}{gwg}{Moo}
\define@key{names}{crm}{Moose Cree}
\define@key{names}{msg}{Moraid}
\define@key{names}{mze}{Morawa}
\define@key{names}{moq}{Mor (Bomberai Peninsula)}
\define@key{names}{msx}{Moresada}
\define@key{names}{xmo}{Morerebi}
\define@key{names}{xmz}{Mori Bawah}
\define@key{names}{mzq}{Mori Atas}
\define@key{names}{mdb}{Morigi}
\define@key{names}{xms}{Moroccan Sign Language}
\define@key{names}{bdo}{Morom}
\define@key{names}{mgc}{Morokodo}
\define@key{names}{mrp}{Morouas}
\define@key{names}{mqn}{Moronene}
\define@key{names}{mrl}{Mortlockese}
\define@key{names}{mwy}{Akie}
\define@key{names}{mqv}{Mosimo}
\define@key{names}{mtj}{Moskona}
\define@key{names}{mtt}{Mota}
\define@key{names}{mwh}{Mouk-Aria}
\define@key{names}{jmw}{Mouwase}
\define@key{names}{ity}{Moyadan Itneg}
\define@key{names}{nmo}{Moyon}
\define@key{names}{mzy}{Mozambican Sign Language}
\define@key{names}{mxi}{Mozarabic}
\define@key{names}{xnq}{Mozambican Ngoni}
\define@key{names}{mpi}{Mpade}
\define@key{names}{mcx}{Mpiemo}
\define@key{names}{mpz}{Mpi}
\define@key{names}{pnd}{Mpinda}
\define@key{names}{mgg}{Mpongmpong}
\define@key{names}{mpa}{Mpoto}
\define@key{names}{mvt}{Mpotovoro}
\define@key{names}{zmp}{Mbuun}
\define@key{names}{cmr}{Mro Chin}
\define@key{names}{mro}{Mru}
\define@key{names}{kqx}{Mser}
\define@key{names}{agz}{Mt. Iriga Agta}
\define@key{names}{atl}{Mt. Iraya Agta}
\define@key{names}{mtd}{Mualang}
\define@key{names}{tsx}{Mubami}
\define@key{names}{mub}{Mubi}
\define@key{names}{ymd}{Muda}
\define@key{names}{gau}{Mudhili Gadaba}
\define@key{names}{udg}{Muduga}
\define@key{names}{vmd}{Mudu Koraga}
\define@key{names}{wiv}{Muduapa}
\define@key{names}{muk}{Mugom}
\define@key{names}{mmk}{Mukha-Dora}
\define@key{names}{mfw}{Mulaha}
\define@key{names}{kpb}{Mullu Kurumba}
\define@key{names}{vmu}{Muluridyi}
\define@key{names}{kqa}{Mum}
\define@key{names}{mwq}{Mün Chin}
\define@key{names}{boe}{Mundabli-Mufu}
\define@key{names}{mmf}{Mindat}
\define@key{names}{mth}{Munggui}
\define@key{names}{mpv}{Mungkip}
\define@key{names}{mtc}{Munit}
\define@key{names}{myr}{Muniche}
\define@key{names}{mnj}{Munji}
\define@key{names}{asx}{Muratayak}
\define@key{names}{mxr}{Murik (Malaysia)}
\define@key{names}{rmh}{Murkim}
\define@key{names}{tkv}{Mur Pano}
\define@key{names}{mqw}{Murupi}
\define@key{names}{smm}{Musasa}
\define@key{names}{mmi}{Hember Avu}
\define@key{names}{mmq}{Aisi}
\define@key{names}{mse}{Musey}
\define@key{names}{mui}{Musi}
\define@key{names}{mje}{Muskum}
\define@key{names}{muv}{Muthuvan}
\define@key{names}{tuc}{Mutu}
\define@key{names}{muy}{Muyang}
\define@key{names}{ymz}{Muzi}
\define@key{names}{mcj}{Mvano}
\define@key{names}{mxh}{Mvuba}
\define@key{names}{wlc}{Mwali Comorian}
\define@key{names}{wmw}{Mwani}
\define@key{names}{moa}{Mwan}
\define@key{names}{mwa}{Mwatebu}
\define@key{names}{mjh}{Mwera (Nyasa)}
\define@key{names}{mws}{Mwimbi-Muthambi}
\define@key{names}{gmy}{Mycenaean Greek}
\define@key{names}{nme}{Mzieme Naga}
\define@key{names}{nbt}{Na}
\define@key{names}{nao}{Naaba}
\define@key{names}{mne}{Naba}
\define@key{names}{mty}{Nabi-Metan}
\define@key{names}{ncd}{Nachering}
\define@key{names}{srf}{Nafi}
\define@key{names}{nxx}{Nafri}
\define@key{names}{jbn}{Nafusi}
\define@key{names}{nbg}{Nagarchal}
\define@key{names}{nxe}{Nage}
\define@key{names}{ngv}{Nagumi}
\define@key{names}{nlx}{Nahali-Baglani}
\define@key{names}{nhh}{Nahari}
\define@key{names}{ars}{Najdi Arabic}
\define@key{names}{nae}{Naka'ela}
\define@key{names}{nib}{Nakama}
\define@key{names}{nkj}{Nakai}
\define@key{names}{nbk}{Nake}
\define@key{names}{mff}{Naki}
\define@key{names}{nax}{Nakwi}
\define@key{names}{nlc}{Nalca}
\define@key{names}{nss}{Nali}
\define@key{names}{nlz}{Nalögo}
\define@key{names}{ylo}{Naluo Yi}
\define@key{names}{naj}{Nalu}
\define@key{names}{nmx}{Nama (Papua New Guinea)}
\define@key{names}{nkm}{Namat}
\define@key{names}{nmk}{Namakura}
\define@key{names}{nmq}{Nambya}
\define@key{names}{ncm}{Nambo}
\define@key{names}{neo}{Ná-Meo}
\define@key{names}{nbs}{Namibian Sign Language}
\define@key{names}{nvm}{Namiae}
\define@key{names}{naa}{Namla}
\define@key{names}{mxw}{Namo}
\define@key{names}{nmt}{Namonuito}
\define@key{names}{bwb}{Namosi-Naitasiri-Serua}
\define@key{names}{nmy}{Namuyi}
\define@key{names}{nnc}{Nancere}
\define@key{names}{nzz}{Nanga}
\define@key{names}{ngr}{Nanggu}
\define@key{names}{cox}{Nanti}
\define@key{names}{afk}{Nanubae-Imangae}
\define@key{names}{qvo}{Napo Lowland Quechua}
\define@key{names}{nrg}{Narango}
\define@key{names}{nac}{Narak}
\define@key{names}{loh}{Narim}
\define@key{names}{nnr}{Narungga}
\define@key{names}{nsy}{Nasal}
\define@key{names}{nvh}{Nasarian}
\define@key{names}{ntz}{Natanzic}
\define@key{names}{nte}{Nathembo}
\define@key{names}{nti}{Natioro}
\define@key{names}{nxa}{Nauete}
\define@key{names}{ncn}{Nauna}
\define@key{names}{nwo}{Nauo}
\define@key{names}{nsw}{Navut}
\define@key{names}{nwr}{Nawaru}
\define@key{names}{nwa}{Nawathinehena}
\define@key{names}{nmz}{Nawdm}
\define@key{names}{naw}{Nawuri}
\define@key{names}{nyq}{Nayinic}
\define@key{names}{noz}{Nayi}
\define@key{names}{ncr}{Ncane-Mungong}
\define@key{names}{nlu}{Nchumbulu}
\define@key{names}{gke}{Ndai}
\define@key{names}{ndk}{Ndaka}
\define@key{names}{ndh}{Ndali}
\define@key{names}{ndj}{Ndamba}
\define@key{names}{ndm}{Ndam}
\define@key{names}{nxo}{Ndambomo}
\define@key{names}{nnz}{Nda'nda'}
\define@key{names}{nda}{Ndasa}
\define@key{names}{ndc}{Ndau}
\define@key{names}{nml}{Ndemli}
\define@key{names}{ndg}{Ndengereko}
\define@key{names}{dne}{Ndendeule}
\define@key{names}{ndd}{Nde-Nsele-Nta}
\define@key{names}{eli}{Nding}
\define@key{names}{ndw}{Ndobo}
\define@key{names}{nbb}{Ndoe}
\define@key{names}{ndl}{Ndolo}
\define@key{names}{ndq}{Ndombe}
\define@key{names}{nqm}{Ndom}
\define@key{names}{ndr}{Ndoola}
\define@key{names}{ndp}{Ndo}
\define@key{names}{dno}{Ndrulo}
\define@key{names}{ndx}{Nduga}
\define@key{names}{nuh}{Ndunda}
\define@key{names}{nww}{Ndwewe}
\define@key{names}{njt}{Ndyuka-Trio Pidgin}
\define@key{names}{wni}{Ndzwani Comorian}
\define@key{names}{nec}{Klamu}
\define@key{names}{nef}{Nefamese}
\define@key{names}{dcr}{Negerhollands}
\define@key{names}{nkg}{Nekgini}
\define@key{names}{nif}{Nek}
\define@key{names}{nej}{Neko}
\define@key{names}{nek}{Neku}
\define@key{names}{nex}{Neme}
\define@key{names}{nem}{Nemi}
\define@key{names}{nqn}{Nen}
\define@key{names}{neu}{Neo (Artificial Language)}
\define@key{names}{nsp}{Nepalese Sign Language}
\define@key{names}{net}{Nete}
\define@key{names}{jas}{New Caledonian Javanese}
\define@key{names}{jui}{Ngadjuri}
\define@key{names}{nnf}{Ngaing}
\define@key{names}{hlt}{Nga La}
\define@key{names}{szb}{Ngalum}
\define@key{names}{nud}{Ngala}
\define@key{names}{nmv}{Ngamini-Yarluyandi-Karangura}
\define@key{names}{nbv}{Ngamambo}
\define@key{names}{nmc}{Ngam}
\define@key{names}{nbh}{Ngamo}
\define@key{names}{nyx}{Nganyaywana}
\define@key{names}{gng}{Ngangam}
\define@key{names}{nne}{Ngandyera}
\define@key{names}{nxd}{Ngando-Lalia}
\define@key{names}{ngd}{Ngando (Central African Republic)}
\define@key{names}{nji}{Ngarnka}
\define@key{names}{rxd}{Ngardi}
\define@key{names}{nsg}{Ngasa}
\define@key{names}{ngm}{Ngatik Men's Creole}
\define@key{names}{cnw}{Ngawn Chin}
\define@key{names}{zdj}{Ngazidja Comorian}
\define@key{names}{ngg}{Ngbaka Manza}
\define@key{names}{jgb}{Ngbee}
\define@key{names}{nbd}{Ngbinda-Mayeka}
\define@key{names}{nuu}{Ngbundu}
\define@key{names}{gnj}{Ngen of Djonkro}
\define@key{names}{nql}{Ngendelengo}
\define@key{names}{ngt}{Kriang-Khlor}
\define@key{names}{nnn}{Ngete}
\define@key{names}{nbq}{Nggem}
\define@key{names}{ngx}{Nggwahyi}
\define@key{names}{nnh}{Ngiemboon}
\define@key{names}{ngj}{Ngie}
\define@key{names}{nnq}{Ngindo}
\define@key{names}{nra}{Ngom}
\define@key{names}{nla}{Ngombale}
\define@key{names}{jgo}{Ngomba}
\define@key{names}{noq}{Ngongo}
\define@key{names}{nsh}{Ngoshie}
\define@key{names}{nuw}{Nguluwan}
\define@key{names}{ngp}{Ngulu}
\define@key{names}{nlo}{Ngwii}
\define@key{names}{xnm}{Ngumbarl}
\define@key{names}{nui}{Ngumbi}
\define@key{names}{nue}{Ngundu}
\define@key{names}{ndn}{Ngundi}
\define@key{names}{ngz}{Ngungwel}
\define@key{names}{nuo}{Nguôn}
\define@key{names}{nrx}{Ngurmbur}
\define@key{names}{nbx}{Wilson River (Grey Range)}
\define@key{names}{ngq}{Ngoreme}
\define@key{names}{ngw}{Ngwaba}
\define@key{names}{nwe}{Ngwe}
\define@key{names}{ngn}{Ngwo}
\define@key{names}{yrl}{Nhengatu}
\define@key{names}{nhf}{Nhuwala}
\define@key{names}{ncs}{Nicaraguan Sign Language}
\define@key{names}{nsi}{Nigerian Sign Language}
\define@key{names}{mzk}{Western Mambila}
\define@key{names}{nii}{Nii}
\define@key{names}{xny}{Nyiyaparli-Palyku}
\define@key{names}{gbe}{Niksek}
\define@key{names}{nim}{Nilamba}
\define@key{names}{nil}{Nila}
\define@key{names}{noe}{Nimadi}
\define@key{names}{nmp}{Nimanbur}
\define@key{names}{nmr}{Nimbari}
\define@key{names}{nis}{Nimi}
\define@key{names}{nmw}{Nimoa}
\define@key{names}{niw}{Nimo}
\define@key{names}{nxi}{Nindi}
\define@key{names}{nxr}{Ninggerum}
\define@key{names}{nby}{Ningera}
\define@key{names}{nlk}{Ninia Yali}
\define@key{names}{nin}{Ninzo}
\define@key{names}{nps}{Nipsan}
\define@key{names}{njs}{Nisa-Anasi}
\define@key{names}{yso}{Nisi (China)}
\define@key{names}{nkp}{Niuatoputapu}
\define@key{names}{njl}{Njalgulgule}
\define@key{names}{nzb}{Njebi}
\define@key{names}{njj}{Njen}
\define@key{names}{njr}{Njerep}
\define@key{names}{njy}{Njyem}
\define@key{names}{nkq}{Nkami}
\define@key{names}{nkn}{Nkangala}
\define@key{names}{nkz}{Nkari}
\define@key{names}{khu}{Nkhumbi}
\define@key{names}{nqo}{N'Ko}
\define@key{names}{nkc}{Nkongho}
\define@key{names}{nkx}{Nkoroo}
\define@key{names}{nka}{Nkoya}
\define@key{names}{nbo}{Nkukoli}
\define@key{names}{nkw}{Nkutu}
\define@key{names}{nbp}{Nnam}
\define@key{names}{ngh}{N||ng}
\define@key{names}{gaw}{Nobonob}
\define@key{names}{noi}{Noiri}
\define@key{names}{nkk}{Nokuku}
\define@key{names}{lem}{Nomaande}
\define@key{names}{nof}{Nomane}
\define@key{names}{noh}{Nomu}
\define@key{names}{zhn}{Nong Zhuang}
\define@key{names}{noj}{Nonuya}
\define@key{names}{nok}{Nooksack}
\define@key{names}{nrc}{Noric}
\define@key{names}{nrp}{North Picene}
\define@key{names}{huj}{Northern Guiyang Hmong}
\define@key{names}{hmp}{Northern Mashan Hmong}
\define@key{names}{crl}{Northern East Cree}
\define@key{names}{pbu}{Northern Pashto}
\define@key{names}{hno}{Northern Hindko}
\define@key{names}{glh}{Northwest Pashayi}
\define@key{names}{aee}{Northeast Pashayi}
\define@key{names}{kxm}{Northern Khmer}
\define@key{names}{atv}{Northern Altai}
\define@key{names}{azj}{North Azerbaijani}
\define@key{names}{ghh}{Northern Ghale}
\define@key{names}{ymx}{Northern Muji}
\define@key{names}{yiv}{Northern Nisu}
\define@key{names}{cng}{Northern Qiang}
\define@key{names}{bfc}{Northern Bai}
\define@key{names}{nnl}{Northern Rengma Naga}
\define@key{names}{lbr}{Lohorung}
\define@key{names}{tji}{Northern Tujia}
\define@key{names}{doc}{Northern Dong}
\define@key{names}{nod}{Northern Thai}
\define@key{names}{tts}{Northeastern Thai}
\define@key{names}{hea}{Northern Qiandong Miao}
\define@key{names}{hmi}{Northern Huishui Hmong}
\define@key{names}{kqs}{Northern Kissi}
\define@key{names}{fll}{North Fali}
\define@key{names}{dgi}{Northern Dagara}
\define@key{names}{tsp}{Northern Toussian}
\define@key{names}{gbo}{Northern Grebo}
\define@key{names}{dip}{Northeastern Dinka}
\define@key{names}{diw}{Northwestern Dinka}
\define@key{names}{max}{North Moluccan Malay}
\define@key{names}{mmg}{North Ambrym}
\define@key{names}{mrq}{North Marquesan}
\define@key{names}{tnn}{North Tanna}
\define@key{names}{una}{North Watut}
\define@key{names}{bcd}{North Babar}
\define@key{names}{weo}{Wemale}
\define@key{names}{nni}{North Nuaulu}
\define@key{names}{aqn}{Northern Alta}
\define@key{names}{xnn}{Northern Kankanay}
\define@key{names}{cts}{Northern Catanduanes Bicolano}
\define@key{names}{stb}{Northern Subanen}
\define@key{names}{bmm}{Northern Betsimisaraka Malagasy}
\define@key{names}{onr}{Northern One}
\define@key{names}{kti}{North Muyu}
\define@key{names}{nks}{Momogo-Pupis-Irogo}
\define@key{names}{yir}{North Awyu}
\define@key{names}{whg}{North Wahgi}
\define@key{names}{kiw}{Northeast Kiwai}
\define@key{names}{ryn}{Northern Amami-Oshima}
\define@key{names}{neq}{North Central Mixe}
\define@key{names}{scs}{North Slavey}
\define@key{names}{esk}{Seward Alaska Inupiatun}
\define@key{names}{thh}{Northern Tarahumara}
\define@key{names}{nhy}{Northern Oaxaca Nahuatl}
\define@key{names}{ojb}{Northwestern Ojibwa}
\define@key{names}{pef}{Northeastern Russian River Pomo}
\define@key{names}{cst}{San Francisco Bay Ohlone}
\define@key{names}{enl}{Enlhet Norte}
\define@key{names}{qvz}{Northern Pastaza Quichua}
\define@key{names}{qul}{North Bolivian Quechua}
\define@key{names}{qxn}{Northern Conchucos Ancash Quechua}
\define@key{names}{pmq}{Northern Pame}
\define@key{names}{xtn}{Northern Tlaxiaco Mixtec}
\define@key{names}{mxa}{Northwest Oaxaca Mixtec}
\define@key{names}{mfk}{North Mofu}
\define@key{names}{ayp}{North Mesopotamian Arabic}
\define@key{names}{ntd}{Northern Tidung}
\define@key{names}{cnp}{Northern Pinghua}
\define@key{names}{ncq}{Northern Katang}
\define@key{names}{bly}{Notre}
\define@key{names}{ncf}{Notsi}
\define@key{names}{ntw}{Nottoway}
\define@key{names}{nov}{Novial}
\define@key{names}{noy}{Noy}
\define@key{names}{asj}{Nsari}
\define@key{names}{nsc}{Nshi}
\define@key{names}{nsx}{Nsongo}
\define@key{names}{baf}{Nubaca}
\define@key{names}{kte}{Gyalsumdo-Nubri}
\define@key{names}{wbm}{Zhenkang Wa}
\define@key{names}{bsq}{Bassa}
\define@key{names}{wla}{Walio}
\define@key{names}{wgi}{Wahgi}
\define@key{names}{gyz}{Gyaazi}
\define@key{names}{nqt}{Nteng}
\define@key{names}{nnv}{Nugunu (Australia)}
\define@key{names}{noc}{Nuk}
\define@key{names}{klt}{Nukna}
\define@key{names}{nuq}{Nukumanu}
\define@key{names}{nur}{Nukuria}
\define@key{names}{nuc}{Nukuini}
\define@key{names}{nbr}{Numana}
\define@key{names}{nop}{Numanggang}
\define@key{names}{sij}{Numbami}
\define@key{names}{tgs}{Nume}
\define@key{names}{kdk}{Numee}
\define@key{names}{nxm}{Numidian}
\define@key{names}{nug}{Nungali}
\define@key{names}{rin}{Nungu}
\define@key{names}{nul}{Nusa Laut}
\define@key{names}{nwb}{Nyabwa}
\define@key{names}{nev}{Nyaheun}
\define@key{names}{nyy}{Nyakyusa-Ngonde}
\define@key{names}{nlj}{Nyali}
\define@key{names}{mwn}{Nyamwanga}
\define@key{names}{nwm}{Nyamusa-Molo}
\define@key{names}{nmi}{Nyam}
\define@key{names}{nny}{Yangkaal}
\define@key{names}{nyb}{Nyangbo}
\define@key{names}{nyc}{Nyanga-li}
\define@key{names}{nyk}{Nyaneka}
\define@key{names}{nnj}{Nyangatom}
\define@key{names}{sev}{Nyarafolo Senoufo}
\define@key{names}{nba}{Nyemba}
\define@key{names}{neh}{Upper Mangdep}
\define@key{names}{nye}{Nyengo}
\define@key{names}{nyl}{Nyeu}
\define@key{names}{nyr}{Nyiha (Malawi)}
\define@key{names}{nkv}{Nyika (Malawi and Zambia)}
\define@key{names}{nkt}{Nyika (Tanzania)}
\define@key{names}{nyg}{Nyindu}
\define@key{names}{lid}{Nyindrou}
\define@key{names}{nvo}{Nyokon}
\define@key{names}{nuj}{Nyole}
\define@key{names}{muo}{Nyong}
\define@key{names}{nyd}{Nyore}
\define@key{names}{nyu}{Nyungwe}
\define@key{names}{nzd}{Nzadi}
\define@key{names}{nzy}{Nzakambay}
\define@key{names}{nja}{Nzanyi}
\define@key{names}{nzi}{Nzima}
\define@key{names}{bzy}{Obanliku}
\define@key{names}{obi}{Obispeño}
\define@key{names}{obl}{Oblo}
\define@key{names}{obo}{Obo Manobo}
\define@key{names}{obu}{Obulom-Ochichi}
\define@key{names}{zac}{Ocotlán Zapotec}
\define@key{names}{odk}{Od}
\define@key{names}{bhf}{Odiai}
\define@key{names}{kkc}{Odoodee}
\define@key{names}{odu}{Odual}
\define@key{names}{tyh}{O'du}
\define@key{names}{opy}{Ofayé}
\define@key{names}{ofo}{Ofo}
\define@key{names}{ogc}{Ogbah}
\define@key{names}{ogg}{Ogbogolo}
\define@key{names}{eri}{Ogea}
\define@key{names}{oia}{Oirata}
\define@key{names}{chj}{Ojitlán Chinantec}
\define@key{names}{oki}{Okiek}
\define@key{names}{okn}{Oki-No-Erabu}
\define@key{names}{okb}{Okobo}
\define@key{names}{okd}{Okodia}
\define@key{names}{oks}{Oko-Eni-Osayen}
\define@key{names}{okj}{Okojuwoi}
\define@key{names}{kqv}{Okolod}
\define@key{names}{oie}{Okolie}
\define@key{names}{opa}{Okpamheri}
\define@key{names}{okx}{Okpe (Northwestern Edo)}
\define@key{names}{oke}{Okpe (Southwestern Edo)}
\define@key{names}{oar}{Old Aramaic-Sam'alian}
\define@key{names}{obr}{Old Burmese}
\define@key{names}{och}{Old Chinese}
\define@key{names}{odt}{Old Dutch-Old Frankish}
\define@key{names}{ang}{Old English (ca. 450-1100)}
\define@key{names}{fro}{Old French (842-ca. 1400)}
\define@key{names}{ofs}{Old Frisian}
\define@key{names}{oge}{Old Georgian}
\define@key{names}{goh}{Old High German (ca. 750-1050)}
\define@key{names}{sga}{Early Irish}
\define@key{names}{ojp}{Old Japanese}
\define@key{names}{okl}{Old Kentish Sign Language}
\define@key{names}{qok}{Old Khmer}
\define@key{names}{qkn}{Old Kannada}
\define@key{names}{qbb}{Old Latin}
\define@key{names}{omx}{Old Mon}
\define@key{names}{omr}{Old Marathi}
\define@key{names}{non}{Old Norse}
\define@key{names}{onw}{Old Nubian}
\define@key{names}{oos}{Old Ossetic}
\define@key{names}{pro}{Old Provençal}
\define@key{names}{peo}{Old Persian (ca. 600-400 B.C.)}
\define@key{names}{orv}{Old Russian}
\define@key{names}{osp}{Old Spanish}
\define@key{names}{osx}{Old Saxon}
\define@key{names}{oty}{Old Tamil}
\define@key{names}{oui}{Old Turkic}
\define@key{names}{owl}{Old-Middle Welsh}
\define@key{names}{ole}{Olekha}
\define@key{names}{olm}{Oloma}
\define@key{names}{lul}{Olu'bo}
\define@key{names}{iko}{Olulumo-Ikom}
\define@key{names}{acx}{Omani Arabic}
\define@key{names}{oml}{Ombo}
\define@key{names}{nht}{Ometepec Nahuatl}
\define@key{names}{omi}{Omi}
\define@key{names}{omt}{Omotik}
\define@key{names}{omu}{Omurano}
\define@key{names}{oog}{Ong-Ir}
\define@key{names}{onx}{Onin Pidgin}
\define@key{names}{oni}{Onin}
\define@key{names}{onj}{Onjob}
\define@key{names}{onn}{Onobasulu}
\define@key{names}{oor}{Oorlams}
\define@key{names}{opo}{Opao}
\define@key{names}{opt}{Teguima}
\define@key{names}{lgn}{Opo}
\define@key{names}{orn}{Orang Kanaq}
\define@key{names}{ors}{Orang Seletar}
\define@key{names}{sdr}{Oraon Sadri}
\define@key{names}{org}{Oring}
\define@key{names}{nlv}{Orizaba Nahuatl}
\define@key{names}{fnb}{Orkon-Fanbak}
\define@key{names}{orc}{Orma}
\define@key{names}{orz}{Ormu}
\define@key{names}{ora}{Oroha}
\define@key{names}{orx}{Oro}
\define@key{names}{orh}{Oroqen}
\define@key{names}{bpk}{Orowe}
\define@key{names}{orw}{Oro Win}
\define@key{names}{orr}{Oruma}
\define@key{names}{syx}{Osamayi}
\define@key{names}{ost}{Osatu}
\define@key{names}{osc}{Oscan}
\define@key{names}{osi}{Osing}
\define@key{names}{oso}{Ososo}
\define@key{names}{uta}{Otank}
\define@key{names}{otd}{Ot Danum}
\define@key{names}{oti}{Oti}
\define@key{names}{otw}{Ottawa}
\define@key{names}{lot}{Otuho}
\define@key{names}{otu}{Otuke}
\define@key{names}{oum}{Ouma}
\define@key{names}{oue}{Ounge}
\define@key{names}{stn}{Owa}
\define@key{names}{wsr}{Oweina}
\define@key{names}{oyy}{Oya'oya}
\define@key{names}{oyd}{Oyda}
\define@key{names}{zao}{Ozolotepec Zapotec}
\define@key{names}{chz}{Ozumacín Chinantec}
\define@key{names}{pfa}{Pááfang}
\define@key{names}{sig}{Paasaal}
\define@key{names}{qvp}{Pacaraos Quechua}
\define@key{names}{pcp}{Pacahuara}
\define@key{names}{pdi}{Pa Di}
\define@key{names}{pkc}{Paekche}
\define@key{names}{pae}{Pagibete}
\define@key{names}{pgi}{Pagi}
\define@key{names}{phr}{Pahari Potwari}
\define@key{names}{phj}{Pahari Newari}
\define@key{names}{lgt}{Pahi}
\define@key{names}{phv}{Pahlavani}
\define@key{names}{pal}{Pahlavi}
\define@key{names}{pha}{Pa-Hng}
\define@key{names}{pri}{Paicî}
\define@key{names}{ppi}{Paipai}
\define@key{names}{qpp}{Paisaci Prakrit}
\define@key{names}{pta}{Pai Tavytera}
\define@key{names}{pkg}{Pak-Tong}
\define@key{names}{jkp}{Paku Karen}
\define@key{names}{pku}{Paku}
\define@key{names}{pfl}{Pfaelzisch-Lothringisch}
\define@key{names}{plq}{Palaic}
\define@key{names}{plr}{Palaka Senoufo}
\define@key{names}{pln}{Palenquero}
\define@key{names}{pnl}{Palen}
\define@key{names}{pli}{Pali}
\define@key{names}{pcf}{Paliyan}
\define@key{names}{pmd}{Pallanganmiddang}
\define@key{names}{abw}{Pal}
\define@key{names}{pmc}{Palumata}
\define@key{names}{ple}{Palu'e}
\define@key{names}{plz}{Paluan}
\define@key{names}{bpx}{Palya Bareli}
\define@key{names}{pmb}{Pambia}
\define@key{names}{pmn}{Pam (Cameroon)}
\define@key{names}{hih}{Pamosu}
\define@key{names}{att}{Pamplona Atta}
\define@key{names}{pnz}{Pana (Central African Republic)}
\define@key{names}{pnq}{Pana (Burkina Faso)}
\define@key{names}{pwb}{Panawa}
\define@key{names}{psn}{Panasuan}
\define@key{names}{qxh}{Panao Huánuco Quechua}
\define@key{names}{lsp}{Panamanian Sign Language}
\define@key{names}{tdb}{Panchpargania}
\define@key{names}{pnp}{Pancana}
\define@key{names}{bkj}{Pande}
\define@key{names}{pgg}{Pangwali}
\define@key{names}{pgs}{Pangseng}
\define@key{names}{slm}{Pangutaran Sama}
\define@key{names}{pcg}{Paniya}
\define@key{names}{pnr}{Panim}
\define@key{names}{pax}{Pankararé}
\define@key{names}{pkh}{Pangkhua}
\define@key{names}{paz}{Pankararú}
\define@key{names}{pnc}{Pannei}
\define@key{names}{knt}{Panoan Katukína}
\define@key{names}{pno}{Panobo}
\define@key{names}{blk}{Pa'o Karen}
\define@key{names}{ppv}{Papavô}
\define@key{names}{ppn}{Papapana}
\define@key{names}{dpp}{Papar}
\define@key{names}{pas}{Papasena}
\define@key{names}{pbo}{Papel}
\define@key{names}{ppe}{Papi}
\define@key{names}{ppu}{Papora-Hoanya}
\define@key{names}{ppm}{Papuma}
\define@key{names}{pgz}{Papua New Guinean Sign Language}
\define@key{names}{prc}{Parachi}
\define@key{names}{pzn}{Jejara Naga}
\define@key{names}{prf}{Paranan}
\define@key{names}{prw}{Parawen}
\define@key{names}{aap}{Pará Arára}
\define@key{names}{pak}{Parakanã}
\define@key{names}{paf}{Paranawát}
\define@key{names}{gvp}{Pará-Maranhão Gavião}
\define@key{names}{pbg}{Paraujano}
\define@key{names}{pys}{Paraguayan Sign Language}
\define@key{names}{pcl}{Pardhi}
\define@key{names}{pch}{Pardhan}
\define@key{names}{pcj}{Gorum-Parenga}
\define@key{names}{ppt}{Pare}
\define@key{names}{kvx}{Parkari Koli}
\define@key{names}{xpr}{Parthian}
\define@key{names}{paq}{Parya}
\define@key{names}{psq}{Pasi}
\define@key{names}{yac}{Pass Valley Yali}
\define@key{names}{ptn}{Patani}
\define@key{names}{pth}{Pataxó Hã-Ha-Hãe}
\define@key{names}{pbc}{Patamona}
\define@key{names}{pty}{Pathiya}
\define@key{names}{ptq}{Pattapu}
\define@key{names}{mfa}{Kelantan-Pattani Malay}
\define@key{names}{pnk}{Paunaka}
\define@key{names}{bfb}{Pauri Bareli}
\define@key{names}{psm}{Warázu}
\define@key{names}{pmr}{Manat}
\define@key{names}{pcb}{Pear}
\define@key{names}{xpc}{Pecheneg}
\define@key{names}{pai}{Pye}
\define@key{names}{pfe}{Peere}
\define@key{names}{ppq}{Pei}
\define@key{names}{pel}{Pekal}
\define@key{names}{bxd}{Pela}
\define@key{names}{ata}{Pele-Ata}
\define@key{names}{pev}{Pémono}
\define@key{names}{psg}{Penang Sign Language}
\define@key{names}{pek}{Penchal}
\define@key{names}{ums}{Pendau}
\define@key{names}{pdc}{Pennsylvania German}
\define@key{names}{pnh}{Māngarongaro}
\define@key{names}{ptw}{Pentlatch}
\define@key{names}{pea}{Peranakan Indonesian}
\define@key{names}{wet}{Perai}
\define@key{names}{psc}{Zaban Eshareh Irani}
\define@key{names}{prl}{Peruvian Sign Language}
\define@key{names}{pex}{Petats}
\define@key{names}{zpe}{Petapa Zapotec}
\define@key{names}{pey}{Petjo}
\define@key{names}{prt}{Prai}
\define@key{names}{phk}{Phake}
\define@key{names}{phl}{Palula}
\define@key{names}{ypa}{Phala}
\define@key{names}{phq}{Phana'}
\define@key{names}{pem}{Phende}
\define@key{names}{psp}{Philippine Sign Language}
\define@key{names}{phm}{Phimbi}
\define@key{names}{phn}{Phoenician}
\define@key{names}{yip}{Pholo}
\define@key{names}{ypg}{Phola}
\define@key{names}{nph}{Phom Naga}
\define@key{names}{pnx}{Phong-Kniang}
\define@key{names}{kjt}{Phrae Pwo Karen}
\define@key{names}{xpg}{Phrygian}
\define@key{names}{phu}{Phuan}
\define@key{names}{phd}{Phudagi}
\define@key{names}{pug}{Phuie}
\define@key{names}{phh}{Phukha}
\define@key{names}{ypm}{Phuma}
\define@key{names}{pho}{Phunoi}
\define@key{names}{phg}{Phuong}
\define@key{names}{yph}{Phupha}
\define@key{names}{ypp}{Phupa}
\define@key{names}{pht}{Phu Thai}
\define@key{names}{ypz}{Phuza}
\define@key{names}{ptr}{Piamatsina}
\define@key{names}{pin}{Piame}
\define@key{names}{pcd}{Picard}
\define@key{names}{cpu}{Pichis Ashéninka}
\define@key{names}{xpi}{Pictish}
\define@key{names}{dep}{Pidgin Delaware}
\define@key{names}{pij}{Pijao}
\define@key{names}{piz}{Pije}
\define@key{names}{pis}{Pijin}
\define@key{names}{piw}{Pimbwe}
\define@key{names}{pnn}{Pinai-Hagahai}
\define@key{names}{pnv}{Pinigura}
\define@key{names}{tjp}{Lake Carnegie Western Desert}
\define@key{names}{pic}{Pinji}
\define@key{names}{pti}{Pintiini}
\define@key{names}{pny}{Pinyin}
\define@key{names}{bxi}{Pirlatapa}
\define@key{names}{pie}{Piro}
\define@key{names}{xpa}{Pirriya}
\define@key{names}{tpp}{Pisaflores Tepehua}
\define@key{names}{pig}{Pisabo}
\define@key{names}{psy}{Piscataway}
\define@key{names}{xps}{Pisidian}
\define@key{names}{pih}{Pitcairn-Norfolk}
\define@key{names}{sje}{Pite Saami}
\define@key{names}{pcn}{Piti}
\define@key{names}{pix}{Piu}
\define@key{names}{piy}{Piya-Kwonci}
\define@key{names}{ktj}{Plapo Krumen}
\define@key{names}{pdt}{Plautdietsch}
\define@key{names}{pbv}{Pnar}
\define@key{names}{npo}{Pochuri Naga}
\define@key{names}{pdn}{Podena}
\define@key{names}{pof}{Poke}
\define@key{names}{pkb}{Pokomo}
\define@key{names}{pld}{Polari}
\define@key{names}{plj}{Pesse}
\define@key{names}{pso}{Polish Sign Language}
\define@key{names}{plb}{Polonombauk}
\define@key{names}{pmo}{Pom}
\define@key{names}{pmm}{Pol}
\define@key{names}{ncc}{Ponam}
\define@key{names}{png}{Pongu}
\define@key{names}{pns}{Ponosakan}
\define@key{names}{pnt}{Pontic}
\define@key{names}{prh}{Porohanon}
\define@key{names}{ptv}{Daakie}
\define@key{names}{pmx}{Poumei Naga}
\define@key{names}{bye}{Pouye}
\define@key{names}{pwr}{Powari}
\define@key{names}{pyn}{Poyanáwa}
\define@key{names}{prz}{Providencia Sign Language}
\define@key{names}{prg}{Old Prussian}
\define@key{names}{kvj}{Psikye}
\define@key{names}{pux}{Puare}
\define@key{names}{atp}{Pudtol Atta}
\define@key{names}{pbm}{Puebla and Northeastern Mazatec}
\define@key{names}{psl}{Puerto Rican Sign Language}
\define@key{names}{pkp}{Pukapuka}
\define@key{names}{pup}{Pulabu}
\define@key{names}{pum}{Puma}
\define@key{names}{xpm}{Pumpokol}
\define@key{names}{puj}{Punan Tubu}
\define@key{names}{pud}{Punan Aput}
\define@key{names}{puf}{Punan Merah}
\define@key{names}{pna}{Punan Bah-Biau}
\define@key{names}{pnm}{Punan Batu 1}
\define@key{names}{xpu}{Punic}
\define@key{names}{qxp}{Puno Quechua}
\define@key{names}{puu}{Punu}
\define@key{names}{pru}{Puragi}
\define@key{names}{iar}{Purari}
\define@key{names}{puy}{Purisimeño}
\define@key{names}{prr}{Puri}
\define@key{names}{pur}{Puruborá}
\define@key{names}{pub}{Purum}
\define@key{names}{mfl}{Putai}
\define@key{names}{afe}{Utugwang-Irungene-Afrike}
\define@key{names}{cpx}{Pu-Xian Chinese}
\define@key{names}{pyu}{Puyuma}
\define@key{names}{pme}{Pwaamei}
\define@key{names}{pop}{Pwapwa}
\define@key{names}{pwo}{Pwo Western Karen}
\define@key{names}{pcw}{Pyapun}
\define@key{names}{pye}{Pye Krumen}
\define@key{names}{pyy}{Pyen}
\define@key{names}{pby}{Pyu}
\define@key{names}{laq}{Pubiao-Qabiao}
\define@key{names}{qxq}{Qashqa'i}
\define@key{names}{xqt}{Qatabanian}
\define@key{names}{ymq}{Qila Muji}
\define@key{names}{zqe}{Qiubei Zhuang}
\define@key{names}{qua}{Quapaw}
\define@key{names}{qya}{Quenya}
\define@key{names}{qvy}{Queyu}
\define@key{names}{zpj}{Quiavicuzas Zapotec}
\define@key{names}{quq}{Quinqui}
\define@key{names}{qun}{Quinault}
\define@key{names}{ztq}{Quioquitani-Quieri Zapotec}
\define@key{names}{rah}{Rabha}
\define@key{names}{xrr}{Raetic}
\define@key{names}{raz}{Rahambuu}
\define@key{names}{mqk}{Rajah Kabunsuwan Manobo}
\define@key{names}{rjs}{Rajbanshi}
\define@key{names}{rjg}{Rajong}
\define@key{names}{gra}{Rajput Garasia}
\define@key{names}{rkh}{Rakahanga-Manihiki}
\define@key{names}{rki}{Rakhine}
\define@key{names}{rai}{Ramoaaina}
\define@key{names}{kjx}{Ramopa}
\define@key{names}{lje}{Rampi}
\define@key{names}{thr}{Rana Tharu}
\define@key{names}{rkt}{Central-Eastern Kamta}
\define@key{names}{rnl}{Halam}
\define@key{names}{rax}{Rang}
\define@key{names}{ray}{Mangaia-Old Rapa}
\define@key{names}{rpt}{Rapting}
\define@key{names}{lra}{Rara Bakati'}
\define@key{names}{rar}{Southern Cook Island Maori}
\define@key{names}{rac}{Rasawa}
\define@key{names}{btn}{Ratagnon}
\define@key{names}{bgd}{Rathwi Bareli}
\define@key{names}{rtw}{Rathawi}
\define@key{names}{rau}{Raute}
\define@key{names}{yea}{Ravula}
\define@key{names}{jnl}{Rawat}
\define@key{names}{rat}{Razajerdi}
\define@key{names}{gir}{Red Gelao}
\define@key{names}{atu}{Reel}
\define@key{names}{ree}{Rejang Kayan}
\define@key{names}{rei}{Reli}
\define@key{names}{bow}{Rema}
\define@key{names}{reb}{Rembong-Wangka}
\define@key{names}{agv}{Hatang Kayi}
\define@key{names}{rem}{Remo of the Moa river}
\define@key{names}{rmp}{Rempi}
\define@key{names}{lkj}{Remun}
\define@key{names}{rsi}{Rennellese Sign Language}
\define@key{names}{rea}{Rerau}
\define@key{names}{rer}{Rer Bare}
\define@key{names}{pgk}{Rerep}
\define@key{names}{res}{Reshe}
\define@key{names}{ret}{Reta}
\define@key{names}{rcf}{Réunion Creole French}
\define@key{names}{rey}{Reyesano}
\define@key{names}{ril}{Riang (Myanmar)}
\define@key{names}{ria}{Riang (India)}
\define@key{names}{rir}{Ribun}
\define@key{names}{zar}{Rincón Zapotec}
\define@key{names}{rgu}{Ringgou}
\define@key{names}{hrx}{Hunsrik}
\define@key{names}{rri}{Ririo}
\define@key{names}{riu}{Riung}
\define@key{names}{snj}{Riverain Sango}
\define@key{names}{rod}{Rogo}
\define@key{names}{rhg}{Rohingya}
\define@key{names}{rge}{Romano-Greek}
\define@key{names}{rms}{Romanian Sign Language}
\define@key{names}{rgn}{Romagnol}
\define@key{names}{rmx}{Romam}
\define@key{names}{rmm}{Roma}
\define@key{names}{rmv}{Romanova}
\define@key{names}{rof}{Rombo}
\define@key{names}{rol}{Romblomanon}
\define@key{names}{rmk}{Romkun}
\define@key{names}{ror}{Rongga}
\define@key{names}{roe}{Ronji}
\define@key{names}{rnn}{Roon}
\define@key{names}{rga}{Mores}
\define@key{names}{pce}{Ruching Palaung}
\define@key{names}{rdb}{Rudbari}
\define@key{names}{ruh}{Ruga}
\define@key{names}{rbb}{Rumai Palaung}
\define@key{names}{ruz}{Ruma}
\define@key{names}{rna}{Runa}
\define@key{names}{rnw}{Rungwa}
\define@key{names}{drg}{Rungus}
\define@key{names}{bxr}{Russia Buriat}
\define@key{names}{rue}{Rusyn}
\define@key{names}{ruc}{Ruuli}
\define@key{names}{rnd}{Ruund}
\define@key{names}{rwk}{Rwa}
\define@key{names}{rsn}{Rwandan Sign Language}
\define@key{names}{sax}{Sa}
\define@key{names}{sav}{Saafi-Saafi}
\define@key{names}{raq}{Saam}
\define@key{names}{lsm}{Saamia}
\define@key{names}{sxr}{Saaroa}
\define@key{names}{spy}{Sabaot}
\define@key{names}{msi}{Sabah Malay}
\define@key{names}{bsy}{Sabah Bisaya}
\define@key{names}{sae}{Sabanê}
\define@key{names}{saa}{Saba}
\define@key{names}{xsa}{Sabaic}
\define@key{names}{qhr}{Old Sabellic}
\define@key{names}{sbo}{Sabüm}
\define@key{names}{quv}{Sacapulteco}
\define@key{names}{sck}{Sadri}
\define@key{names}{spd}{Saep}
\define@key{names}{saf}{Safaliba}
\define@key{names}{sbk}{Safwa}
\define@key{names}{sbm}{Sagala}
\define@key{names}{tga}{Sagalla}
\define@key{names}{aec}{Saidi Arabic}
\define@key{names}{acf}{Saint Lucian Creole French}
\define@key{names}{xsy}{Saisiyat}
\define@key{names}{sjl}{Sajolang}
\define@key{names}{sjb}{Sajau-Latti}
\define@key{names}{sch}{Sakachep-Chorei}
\define@key{names}{skt}{Sakata}
\define@key{names}{skg}{West Malagasy Sakalava}
\define@key{names}{skm}{Sakam}
\define@key{names}{sak}{Sake}
\define@key{names}{szy}{Sakizaya}
\define@key{names}{shq}{Sala}
\define@key{names}{slx}{Salampasu}
\define@key{names}{sgu}{Salas}
\define@key{names}{qxl}{Tungurahua Highland Quichua}
\define@key{names}{mnd}{Salamãi}
\define@key{names}{slq}{Salchuq}
\define@key{names}{sau}{Saleman}
\define@key{names}{loe}{Saluan}
\define@key{names}{esn}{Salvadoran Sign Language}
\define@key{names}{tmj}{Samarokena}
\define@key{names}{ysd}{Samatao}
\define@key{names}{smp}{Samaritan}
\define@key{names}{xab}{Sambe}
\define@key{names}{smx}{Samba}
\define@key{names}{ccg}{Samba Daka}
\define@key{names}{saq}{Samburu}
\define@key{names}{ssx}{Samberigi}
\define@key{names}{spv}{Sambalpuri}
\define@key{names}{smh}{Samei}
\define@key{names}{snx}{Sam}
\define@key{names}{swm}{Samosa}
\define@key{names}{rav}{Sampang}
\define@key{names}{stu}{Samtao}
\define@key{names}{smv}{Samvedi}
\define@key{names}{ztm}{San Agustín Mixtepec Zapotec}
\define@key{names}{icr}{San Andres Creole English}
\define@key{names}{spn}{Sanapaná}
\define@key{names}{zpx}{San Baltazar Loxicha Zapotec}
\define@key{names}{cuk}{San Blas Kuna}
\define@key{names}{hve}{San Dionisio del Mar Huave}
\define@key{names}{hue}{San Francisco del Mar Huave}
\define@key{names}{mat}{San Francisco Matlatzinca}
\define@key{names}{pow}{San Felipe Otlaltepec Popoloca}
\define@key{names}{xso}{San Francisco Solano}
\define@key{names}{sgr}{Sangisari}
\define@key{names}{sgk}{Sangkong}
\define@key{names}{nsa}{Sangtam Naga}
\define@key{names}{xsn}{Sanga (Nigeria)}
\define@key{names}{sbp}{Sangu (Tanzania)}
\define@key{names}{sng}{Sanga (Democratic Republic of Congo)}
\define@key{names}{snl}{Sangil}
\define@key{names}{scg}{Sanggau}
\define@key{names}{sgy}{Sanglechi}
\define@key{names}{ysy}{Sanie}
\define@key{names}{ysn}{Sani}
\define@key{names}{sny}{Saniyo-Hiyewe}
\define@key{names}{xtj}{San Juan Teita Mixtec}
\define@key{names}{maa}{San Jerónimo Tecóatl Mazatec}
\define@key{names}{msc}{Sankaran Maninka}
\define@key{names}{pps}{San Luís Temalacayuca Popoloca}
\define@key{names}{qvs}{San Martín Quechua}
\define@key{names}{xtp}{San Miguel Piedras Mixtec}
\define@key{names}{trq}{San Martín Itunyoso Triqui}
\define@key{names}{pls}{San Marcos Tlalcoyalco Popoloca}
\define@key{names}{azg}{San Pedro Amuzgos Amuzgo}
\define@key{names}{zpf}{San Pedro Quiatoni Zapotec}
\define@key{names}{san}{Sanskrit}
\define@key{names}{ssi}{Sansi}
\define@key{names}{kwy}{San Salvador Kongo}
\define@key{names}{hvv}{Santa María del Mar Huave}
\define@key{names}{nhz}{Santa María La Alta Nahuatl}
\define@key{names}{cok}{Santa Teresa Cora}
\define@key{names}{qus}{Santiago del Estero Quichua}
\define@key{names}{mza}{Santa María Zacatepec Mixtec}
\define@key{names}{mdv}{Santa Lucía Monteverde Mixtec}
\define@key{names}{zpn}{Santa Inés Yatzechi Zapotec}
\define@key{names}{ztn}{Santa Catarina Albarradas Zapotec}
\define@key{names}{zas}{Santo Domingo Albarradas Zapotec}
\define@key{names}{zpr}{Santiago Xanica Zapotec}
\define@key{names}{pca}{Santa Inés Ahuatempan Popoloca}
\define@key{names}{zpt}{San Vicente Coatlán Zapotec}
\define@key{names}{scq}{Sa'och}
\define@key{names}{zkp}{São Paulo Kaingáng}
\define@key{names}{cri}{Sãotomense}
\define@key{names}{spr}{Saparua}
\define@key{names}{spc}{Sapé}
\define@key{names}{krn}{Sapo}
\define@key{names}{spi}{Saponi}
\define@key{names}{sbz}{Sara Kaba}
\define@key{names}{kwv}{Sara Kaba Náà}
\define@key{names}{kwg}{Sara Kaba Deme}
\define@key{names}{zsa}{Sarasira}
\define@key{names}{bps}{Sarangani Blaan}
\define@key{names}{mbs}{Sarangani Manobo}
\define@key{names}{sre}{Sara Bakati'}
\define@key{names}{sar}{Saraveca}
\define@key{names}{srh}{Sarikoli}
\define@key{names}{mwm}{Sar}
\define@key{names}{onp}{Sartang}
\define@key{names}{sdu}{Sarudu}
\define@key{names}{sra}{Saruga}
\define@key{names}{swy}{Sarua}
\define@key{names}{sxs}{Sasaru}
\define@key{names}{sas}{Sasak}
\define@key{names}{sdc}{Sassarese Sardinian}
\define@key{names}{stw}{Satawalese}
\define@key{names}{stq}{Ems-Weser Frisian}
\define@key{names}{mav}{Sateré-Mawé}
\define@key{names}{sdl}{Saudi Arabian Sign Language}
\define@key{names}{skc}{Ma Manda}
\define@key{names}{saz}{Saurashtra}
\define@key{names}{mjt}{Sauria Paharia}
\define@key{names}{srt}{Sauri}
\define@key{names}{psu}{Sauraseni Prakrit}
\define@key{names}{ssj}{Sausi}
\define@key{names}{sao}{Sause}
\define@key{names}{swr}{Saweru}
\define@key{names}{swt}{Sawila}
\define@key{names}{saw}{Sawi}
\define@key{names}{swn}{Sawknah-Fogaha}
\define@key{names}{sxw}{Saxwe Gbe}
\define@key{names}{say}{Saya}
\define@key{names}{sco}{Scots}
\define@key{names}{kdg}{Seba}
\define@key{names}{sbx}{Seberuang}
\define@key{names}{sib}{Sebop}
\define@key{names}{sec}{Sechelt}
\define@key{names}{tvw}{Sedoa}
\define@key{names}{sos}{Seeku}
\define@key{names}{sge}{Segai}
\define@key{names}{sbg}{Seget}
\define@key{names}{seg}{Segeju}
\define@key{names}{sfw}{Sehwi}
\define@key{names}{ssg}{Seimat}
\define@key{names}{hik}{Seit-Kaitetu}
\define@key{names}{skz}{Sekar}
\define@key{names}{skp}{Sekapan}
\define@key{names}{sek}{Sekani}
\define@key{names}{ske}{Seke (Vanuatu)}
\define@key{names}{syi}{Seki}
\define@key{names}{sko}{Seko Tengah}
\define@key{names}{skx}{Seko Padang}
\define@key{names}{lip}{Sekpele}
\define@key{names}{kgi}{Selangor Sign Language}
\define@key{names}{snw}{Selee}
\define@key{names}{sws}{Seluwasan}
\define@key{names}{slg}{Selungai Murut}
\define@key{names}{szc}{Semaq Beri}
\define@key{names}{sbr}{Sembakung Murut}
\define@key{names}{etz}{Semimi}
\define@key{names}{smy}{Semnani-Biyabuneki}
\define@key{names}{ssm}{Semnam}
\define@key{names}{xse}{Sempan}
\define@key{names}{seq}{Senar de Kankalaba}
\define@key{names}{sej}{Sene}
\define@key{names}{sds}{Sened}
\define@key{names}{ssz}{Sengseng}
\define@key{names}{spk}{Sengo}
\define@key{names}{snu}{Senggi}
\define@key{names}{sjs}{Senhaja De Srair}
\define@key{names}{sni}{Sensi}
\define@key{names}{std}{Sentinel}
\define@key{names}{sez}{Senthang Chin}
\define@key{names}{spe}{Sepa (Papua New Guinea)}
\define@key{names}{spb}{Sepa (Indonesia)}
\define@key{names}{spm}{Sepen}
\define@key{names}{iws}{Sepik Iwam}
\define@key{names}{skr}{Saraiki}
\define@key{names}{sry}{Sera}
\define@key{names}{srr}{Sereer}
\define@key{names}{swf}{Sere}
\define@key{names}{sve}{Serili}
\define@key{names}{seu}{Serui-Laut}
\define@key{names}{srw}{Serua}
\define@key{names}{srk}{Serudung Murut}
\define@key{names}{stf}{Seta}
\define@key{names}{stm}{Setaman}
\define@key{names}{sbi}{Seti}
\define@key{names}{sta}{KiSetla}
\define@key{names}{sew}{Sewa Bay}
\define@key{names}{lsw}{Seychelles Sign Language}
\define@key{names}{sze}{Seze}
\define@key{names}{scw}{Sya}
\define@key{names}{sdb}{Shabaki}
\define@key{names}{srz}{Shahmirzadi}
\define@key{names}{sha}{Shall-Zwall}
\define@key{names}{xsh}{Shamang}
\define@key{names}{sqa}{Shama-Sambuga}
\define@key{names}{jih}{Stodsde}
\define@key{names}{sho}{Shanga}
\define@key{names}{swo}{Shanenawa}
\define@key{names}{ssv}{Ngen}
\define@key{names}{swq}{Sharwa}
\define@key{names}{sqh}{Shau}
\define@key{names}{shx}{She}
\define@key{names}{she}{Sheko}
\define@key{names}{sth}{Shelta}
\define@key{names}{shl}{Shendu}
\define@key{names}{scv}{Sheni-Ziriya}
\define@key{names}{bun}{Sherbro}
\define@key{names}{kip}{Sheshi Kham}
\define@key{names}{ssh}{Shihhi Arabic}
\define@key{names}{shr}{Shi}
\define@key{names}{gua}{Shiki}
\define@key{names}{snh}{Shinabo}
\define@key{names}{sxg}{Shixing}
\define@key{names}{sle}{Sholaga}
\define@key{names}{bcv}{Shoo-Minda-Nye}
\define@key{names}{suj}{Shubi}
\define@key{names}{sts}{Shumashti}
\define@key{names}{scu}{Shumcho}
\define@key{names}{ksa}{Shuwa-Zamani}
\define@key{names}{shw}{Shwai}
\define@key{names}{slw}{Sialum}
\define@key{names}{sya}{Siang}
\define@key{names}{spg}{Sihan}
\define@key{names}{mmp}{Siawi}
\define@key{names}{nco}{Sibe (Nasioi)}
\define@key{names}{sty}{Siberian Tatar}
\define@key{names}{sdx}{Sibu Melanau}
\define@key{names}{sxc}{Sicana}
\define@key{names}{scn}{Sicilian}
\define@key{names}{sep}{Sìcìté Sénoufo}
\define@key{names}{scx}{Sicula}
\define@key{names}{xsd}{Sidetic}
\define@key{names}{sgx}{Sierra Leone Sign Language}
\define@key{names}{nsu}{Sierra Negra Nahuatl}
\define@key{names}{sxe}{Sighu}
\define@key{names}{snr}{Sihan (Gum)}
\define@key{names}{qws}{Sihuas Ancash Quechua}
\define@key{names}{sky}{Sikaiana}
\define@key{names}{slt}{Sila}
\define@key{names}{szl}{Silesian}
\define@key{names}{sbq}{Sirva}
\define@key{names}{mkc}{Siliput}
\define@key{names}{wul}{Silimo}
\define@key{names}{xsp}{Silopi}
\define@key{names}{stv}{Silt'e}
\define@key{names}{sie}{Simaa}
\define@key{names}{sbw}{Simba}
\define@key{names}{smb}{Simbari}
\define@key{names}{sbb}{Simbo}
\define@key{names}{smg}{Simbali}
\define@key{names}{smz}{Simeku}
\define@key{names}{smt}{Simte}
\define@key{names}{siu}{Galu}
\define@key{names}{sbn}{Sindhi Bhil}
\define@key{names}{xts}{Sindihui Mixtec}
\define@key{names}{sjn}{Sindarin}
\define@key{names}{sgp}{Northern Jinghpaw}
\define@key{names}{sgm}{Singa}
\define@key{names}{skq}{Sininkere}
\define@key{names}{xti}{Sinicahua Mixtec}
\define@key{names}{snz}{Kou}
\define@key{names}{sys}{Sinyar}
\define@key{names}{swj}{Sira}
\define@key{names}{sir}{Siri}
\define@key{names}{srx}{Sirmauri}
\define@key{names}{sld}{Sissala of Burkina Faso}
\define@key{names}{sso}{Sissano}
\define@key{names}{siy}{Sivandi}
\define@key{names}{lsv}{Sivia Sign Language}
\define@key{names}{akp}{Siwu}
\define@key{names}{skw}{Skepi Creole Dutch}
\define@key{names}{sms}{Skolt Saami}
\define@key{names}{svm}{Slavomolisano}
\define@key{names}{svk}{Slovakian Sign Language}
\define@key{names}{sfm}{Gha-mu}
\define@key{names}{kxq}{Smärky Kanum}
\define@key{names}{sox}{So (Cameroon)}
\define@key{names}{soc}{So (Democratic Republic of Congo)}
\define@key{names}{xog}{Soga}
\define@key{names}{sog}{Sogdian}
\define@key{names}{soj}{Soic}
\define@key{names}{sok}{Sokoro}
\define@key{names}{sby}{Soli}
\define@key{names}{sol}{Solos}
\define@key{names}{aaw}{Solong}
\define@key{names}{szs}{Solomon Islands Sign Language}
\define@key{names}{smc}{Som}
\define@key{names}{smu}{Somray of Battambang-Somre of Siem Reap}
\define@key{names}{sor}{Somrai}
\define@key{names}{kgt}{Somyev}
\define@key{names}{ysg}{Sonaga}
\define@key{names}{shc}{Sonde}
\define@key{names}{soo}{Nsong-Mpiin}
\define@key{names}{sod}{Songoora}
\define@key{names}{soe}{Ohendo}
\define@key{names}{soi}{Sonha}
\define@key{names}{siq}{Sonia}
\define@key{names}{sss}{Sô}
\define@key{names}{urw}{Sop}
\define@key{names}{sbh}{Sori-Harengan}
\define@key{names}{sqo}{Sorkhei-Aftari}
\define@key{names}{ays}{Sorsogon Ayta}
\define@key{names}{sdk}{Sos Kundi}
\define@key{names}{krz}{Sota Kanum}
\define@key{names}{sfs}{South African Sign Language}
\define@key{names}{nit}{Southeastern Kolami}
\define@key{names}{hmy}{Southern Guiyang Hmong}
\define@key{names}{hma}{Southern Mashan Hmong}
\define@key{names}{sdh}{Southern Kurdish}
\define@key{names}{bcc}{Southern Balochi}
\define@key{names}{fay}{Fars Dialects}
\define@key{names}{luz}{Southern Luri}
\define@key{names}{pbt}{Southern Pashto}
\define@key{names}{hnd}{Southern Hindko}
\define@key{names}{psh}{Southwest Pashayi}
\define@key{names}{psi}{Southeast Pashayi}
\define@key{names}{vro}{South Estonian}
\define@key{names}{nik}{Southern Nicobarese}
\define@key{names}{mnn}{Southern Mnong}
\define@key{names}{uzs}{Southern Uzbek}
\define@key{names}{ghe}{Southern Ghale}
\define@key{names}{ymc}{Southern Muji}
\define@key{names}{nsd}{Southern Nisu}
\define@key{names}{qxs}{Southern Qiang}
\define@key{names}{pmj}{Southern Pumi}
\define@key{names}{bfs}{Southern Bai}
\define@key{names}{nre}{Southern Rengma Naga}
\define@key{names}{lrr}{Southern Yamphu}
\define@key{names}{tjs}{Southern Tujia}
\define@key{names}{sou}{Southern Thai}
\define@key{names}{hms}{Southern Qiandong Miao}
\define@key{names}{hmh}{Southwestern Huishui Hmong}
\define@key{names}{hmg}{Southwestern Guiyang Hmong}
\define@key{names}{xtv}{Southern Coastal Yuin}
\define@key{names}{ijs}{Southeast Ijo}
\define@key{names}{fal}{South Fali}
\define@key{names}{nbw}{Southern Ngbandi}
\define@key{names}{lnl}{South Central Banda}
\define@key{names}{biv}{Southern Birifor}
\define@key{names}{nnw}{Southern Nuni}
\define@key{names}{snm}{Southern Ma'di}
\define@key{names}{dik}{Southwestern Dinka}
\define@key{names}{dib}{South Central Dinka}
\define@key{names}{dks}{Southeastern Dinka}
\define@key{names}{bwq}{Southern Bobo Madaré}
\define@key{names}{sbd}{Southern Samo}
\define@key{names}{sns}{Nahavaq}
\define@key{names}{mqm}{South Marquesan}
\define@key{names}{mcy}{South Watut}
\define@key{names}{vbb}{Southeast Babar}
\define@key{names}{lmf}{Eastern Atadei}
\define@key{names}{agy}{Southern Alta}
\define@key{names}{ksc}{Bangad}
\define@key{names}{bln}{Coastal-Virac Bikol}
\define@key{names}{plv}{Southwest Palawano}
\define@key{names}{bzc}{Southern Betsimisaraka Malagasy}
\define@key{names}{osu}{Southern One}
\define@key{names}{aws}{South Awyu}
\define@key{names}{omw}{South Tairora}
\define@key{names}{ams}{Southern Amami-Oshima}
\define@key{names}{hax}{Southern Haida}
\define@key{names}{tce}{Southern Tutchone}
\define@key{names}{caf}{Southern Carrier}
\define@key{names}{twr}{Southwestern Tarahumara}
\define@key{names}{tcu}{Southeastern Tarahumara}
\define@key{names}{npl}{Nahuatl, Southeastern Puebla}
\define@key{names}{tla}{Southwestern Tepehuan}
\define@key{names}{crj}{Southern East Cree}
\define@key{names}{peq}{Southern Pomo}
\define@key{names}{qup}{Southern Pastaza Quechua}
\define@key{names}{qxo}{Southern Conchucos Ancash Quechua}
\define@key{names}{ayc}{Southern Aymara}
\define@key{names}{meh}{Southwestern Tlaxiaco Mixtec}
\define@key{names}{mit}{Southern Puebla Mixtec}
\define@key{names}{mxy}{Southeastern Nochixtlán Mixtec}
\define@key{names}{rgs}{Southern Roglai}
\define@key{names}{giz}{South Giziga}
\define@key{names}{cpy}{South Ucayali Ashéninka}
\define@key{names}{itd}{Southern Tidung}
\define@key{names}{csp}{Southern Pinghua}
\define@key{names}{sct}{Southern Katang}
\define@key{names}{sqq}{Sou}
\define@key{names}{sww}{Sowa}
\define@key{names}{sow}{Sowanda}
\define@key{names}{vmq}{Soyaltepec Mixtec}
\define@key{names}{vmp}{Soyaltepec Mazatec}
\define@key{names}{sqs}{Sri Lankan Sign Language}
\define@key{names}{sci}{Sri Lanka Malay}
\define@key{names}{seo}{Asabano}
\define@key{names}{swp}{Suau}
\define@key{names}{sxb}{Suba}
\define@key{names}{ssc}{Suba-Simbiti}
\define@key{names}{sut}{Subtiaba}
\define@key{names}{apd}{Sudanese Arabic}
\define@key{names}{pga}{South Sudanese Creole Arabic}
\define@key{names}{sgi}{Nizaa}
\define@key{names}{sug}{Suganga}
\define@key{names}{kzs}{Sugut Dusun}
\define@key{names}{zsu}{Sukurum}
\define@key{names}{syk}{Sukur}
\define@key{names}{szn}{Sula}
\define@key{names}{srg}{Sulod}
\define@key{names}{sqm}{Suma}
\define@key{names}{siv}{Sumariup}
\define@key{names}{six}{Sumau}
\define@key{names}{suw}{Sumbwa}
\define@key{names}{smw}{Sumbawa}
\define@key{names}{sux}{Sumerian}
\define@key{names}{csv}{Sumtu Chin}
\define@key{names}{ssk}{Sunam}
\define@key{names}{suz}{Sunwar}
\define@key{names}{syo}{Suoy}
\define@key{names}{sbj}{Surbakhal}
\define@key{names}{sgd}{Surigaonon}
\define@key{names}{sjp}{Surjapuri}
\define@key{names}{tdl}{Sur}
\define@key{names}{sde}{Vori}
\define@key{names}{mdz}{Suruí Do Pará}
\define@key{names}{sru}{Suruí}
\define@key{names}{swx}{Suruahá}
\define@key{names}{sqn}{Susquehannock}
\define@key{names}{ssu}{Susuami}
\define@key{names}{sdj}{Suundi}
\define@key{names}{swu}{Suwawa}
\define@key{names}{suy}{Suyá}
\define@key{names}{swg}{Swabian}
\define@key{names}{slf}{Swiss-Italian Sign Language}
\define@key{names}{sgg}{Swiss-German Sign Language}
\define@key{names}{ssr}{Swiss-French Sign Language}
\define@key{names}{xdk}{Sydney}
\define@key{names}{syl}{Sylheti}
\define@key{names}{zoq}{Tabasco Zoque}
\define@key{names}{nhc}{Tabasco Nahuatl}
\define@key{names}{zat}{Tabaa Zapotec}
\define@key{names}{knv}{Tabo}
\define@key{names}{tzx}{Tabriak}
\define@key{names}{xtt}{Tacahua-Yolotepec Mixtec}
\define@key{names}{lts}{Tachoni}
\define@key{names}{dsq}{Tadaksahak}
\define@key{names}{tdy}{Tadyawan}
\define@key{names}{rob}{Tae'}
\define@key{names}{tcd}{Tafi}
\define@key{names}{klg}{Tagakaulu Kalagan}
\define@key{names}{bgs}{Tagabawa}
\define@key{names}{mvv}{Tagal Murut}
\define@key{names}{tgz}{Tagalaka}
\define@key{names}{tbm}{Tagbu}
\define@key{names}{tda}{Tagdal}
\define@key{names}{tgx}{Tagish}
\define@key{names}{tgj}{Tagin}
\define@key{names}{tgw}{Tagwana Senoufo}
\define@key{names}{tht}{Tahltan}
\define@key{names}{blt}{Tai Dam}
\define@key{names}{tyj}{Tai Do-Mene-Yo}
\define@key{names}{tyr}{Tai Daeng-Meuay}
\define@key{names}{twh}{Tai Dón}
\define@key{names}{tiz}{Tai Hongjin}
\define@key{names}{taw}{Tai}
\define@key{names}{aos}{Taikat}
\define@key{names}{tlq}{Muak}
\define@key{names}{thi}{Tai Long}
\define@key{names}{tjl}{Tai Laing}
\define@key{names}{tdd}{Tai Nüa}
\define@key{names}{ago}{Tainae}
\define@key{names}{tnq}{Taino}
\define@key{names}{tpo}{Tai Pao}
\define@key{names}{uar}{Tairuma}
\define@key{names}{tmm}{Tai Thanh}
\define@key{names}{cuu}{Tai Ya}
\define@key{names}{acq}{Ta'izzi-Adeni Arabic}
\define@key{names}{pee}{Taje}
\define@key{names}{tdj}{Tajio}
\define@key{names}{abh}{Tajiki Arabic}
\define@key{names}{tja}{Tajuasohn}
\define@key{names}{tkz}{Takua}
\define@key{names}{nho}{Takuu}
\define@key{names}{tke}{Takwane}
\define@key{names}{tak}{Tala}
\define@key{names}{tdf}{Talieng}
\define@key{names}{tlr}{Talise}
\define@key{names}{tlv}{Taliabu}
\define@key{names}{tal}{Tal}
\define@key{names}{tln}{Talondo'}
\define@key{names}{tlk}{Taloki}
\define@key{names}{tzl}{Talossan}
\define@key{names}{yta}{Lavu-Yongsheng-Talu}
\define@key{names}{tcl}{Taman (Myanmar)}
\define@key{names}{tmn}{Taman (Indonesia)}
\define@key{names}{tmz}{Tamanaku}
\define@key{names}{vmx}{Tamazola Mixtec}
\define@key{names}{ten}{Tama (Colombia)}
\define@key{names}{tls}{Tambotalo}
\define@key{names}{xxt}{Tambora}
\define@key{names}{tdk}{Tambas}
\define@key{names}{tmy}{Tami}
\define@key{names}{tax}{Tamki}
\define@key{names}{tml}{Tamnim Citak}
\define@key{names}{tpu}{Tampuan}
\define@key{names}{low}{Tampias Lobu}
\define@key{names}{tpv}{Tanapag}
\define@key{names}{tcm}{Tanahmerah}
\define@key{names}{tni}{Tandia}
\define@key{names}{tdx}{Tandroy Malagasy}
\define@key{names}{tgn}{Tandaganon}
\define@key{names}{tnx}{Tanema}
\define@key{names}{tnv}{Tangchangya}
\define@key{names}{txg}{Tangut}
\define@key{names}{tgp}{Movono}
\define@key{names}{tkx}{Tangko}
\define@key{names}{tgu}{Tanggu}
\define@key{names}{tbs}{Tanguat}
\define@key{names}{ytl}{Tanglang-Toloza}
\define@key{names}{tbe}{Tanimbili}
\define@key{names}{uji}{Rjili}
\define@key{names}{txy}{Tanosy Malagasy}
\define@key{names}{xnj}{Tanzanian Ngoni}
\define@key{names}{qcs}{Tapachultec}
\define@key{names}{afp}{Tapei}
\define@key{names}{taf}{Tapirapé}
\define@key{names}{txj}{Tarjumo}
\define@key{names}{tpf}{Tarpia}
\define@key{names}{txr}{Tartessian}
\define@key{names}{tdm}{Taruma}
\define@key{names}{twq}{Tasawaq}
\define@key{names}{tmt}{Tasmate}
\define@key{names}{ttd}{Tauade}
\define@key{names}{tco}{Taungyo}
\define@key{names}{tpa}{Taupota}
\define@key{names}{tad}{Tause}
\define@key{names}{tvs}{Taveta}
\define@key{names}{tvn}{Tavoyan}
\define@key{names}{rmu}{Tavringer Romani}
\define@key{names}{twl}{Tawara}
\define@key{names}{xtw}{Tawandê}
\define@key{names}{ttq}{Tawallammat Tamajaq}
\define@key{names}{twy}{Tawoyan}
\define@key{names}{tbp}{Taworta}
\define@key{names}{tcp}{Laamtuk Thet}
\define@key{names}{ayy}{Tayabas Ayta near Lucena City in Western Quezon}
\define@key{names}{tas}{Tay Boi}
\define@key{names}{tnu}{Tay Khang}
\define@key{names}{tys}{Tày Sa Pa}
\define@key{names}{tyt}{Tày Tac}
\define@key{names}{tyz}{Tày}
\define@key{names}{tck}{Tchitchege}
\define@key{names}{bqa}{Tchumbuli}
\define@key{names}{dtu}{Tebul Ure Dogon}
\define@key{names}{tsy}{Tebul Sign Language}
\define@key{names}{tcw}{Tecpatlán Totonac}
\define@key{names}{tuq}{Tedaga}
\define@key{names}{tkq}{Tee}
\define@key{names}{lor}{Téén}
\define@key{names}{tfo}{Tefaro}
\define@key{names}{twe}{Teiwa}
\define@key{names}{ztt}{Tejalapan Zapotec}
\define@key{names}{teg}{Latege}
\define@key{names}{tyx}{Teke-Tyee}
\define@key{names}{lli}{Teke-Laali}
\define@key{names}{ebo}{Teke-Eboo-Nzikou}
\define@key{names}{tyi}{Teke-Tsaayi}
\define@key{names}{tvm}{Tela-Masbuar}
\define@key{names}{tlt}{Teluti}
\define@key{names}{nhv}{Temascaltepec Nahuatl}
\define@key{names}{tjo}{Oued Righ}
\define@key{names}{tbt}{Tembo (Kitembo)}
\define@key{names}{tmv}{Motembo-Kunda}
\define@key{names}{tqb}{Tenetehara}
\define@key{names}{tdo}{Teme}
\define@key{names}{soz}{Temi}
\define@key{names}{tmo}{Temoq}
\define@key{names}{ott}{Temoaya Otomi}
\define@key{names}{tmw}{Temuan}
\define@key{names}{quw}{Tena Lowland Quichua}
\define@key{names}{otn}{Tenango Otomi}
\define@key{names}{dtk}{Tengou-Togo Dogon}
\define@key{names}{tes}{Tengger}
\define@key{names}{pah}{Tenharim-Parintintin-Diahoi}
\define@key{names}{tqn}{Tenino}
\define@key{names}{tns}{Tenis}
\define@key{names}{tct}{T'en}
\define@key{names}{tev}{Teor}
\define@key{names}{cux}{Tepeuxila Cuicatec}
\define@key{names}{cte}{Tepinapa Chinantec}
\define@key{names}{ted}{Tepo Krumen}
\define@key{names}{tef}{Teressa}
\define@key{names}{trb}{Terebu}
\define@key{names}{twg}{Tereweng}
\define@key{names}{tec}{Terik}
\define@key{names}{tmg}{Ternateño}
\define@key{names}{sjt}{Ter Saami}
\define@key{names}{tkg}{Tesaka Malagasy}
\define@key{names}{keg}{Tese}
\define@key{names}{twc}{Teshenawa}
\define@key{names}{tez}{Tetserret}
\define@key{names}{tdt}{Tetun Dili}
\define@key{names}{tve}{Te'un}
\define@key{names}{cut}{Teutila Cuicatec}
\define@key{names}{twx}{Tewe}
\define@key{names}{otx}{Texcatepec Otomi}
\define@key{names}{poq}{Texistepec Popoluca}
\define@key{names}{mxb}{Tezoatlán Mixtec}
\define@key{names}{thy}{Tha}
\define@key{names}{thn}{Thachanadan}
\define@key{names}{soa}{Thai Song}
\define@key{names}{nki}{Thangal Naga}
\define@key{names}{thk}{Tharaka}
\define@key{names}{iin}{Thiin}
\define@key{names}{tou}{Tho}
\define@key{names}{ytp}{Thopho}
\define@key{names}{txh}{Thracian}
\define@key{names}{thu}{Thuri}
\define@key{names}{ahi}{Tiagbamrin Aizi}
\define@key{names}{mnl}{Tiale}
\define@key{names}{tbj}{Tiang}
\define@key{names}{ngy}{Tibea}
\define@key{names}{lsn}{Tibetan Sign Language}
\define@key{names}{tcn}{Tichurong}
\define@key{names}{mtx}{Tidaá Mixtec}
\define@key{names}{tia}{Tidikelt-Tuat Tamazight}
\define@key{names}{tiq}{Tiefo-Daramandugu}
\define@key{names}{boo}{Tiemacèwè Bozo}
\define@key{names}{tii}{Tiene}
\define@key{names}{nza}{Tigon Mbembe}
\define@key{names}{txq}{Tii}
\define@key{names}{xtl}{Tijaltepec Mixtec}
\define@key{names}{tkp}{Tikopia}
\define@key{names}{otl}{Tilapa Otomi}
\define@key{names}{zts}{Tilquiapan Zapotec}
\define@key{names}{tij}{Tilung}
\define@key{names}{tim}{Timbe}
\define@key{names}{tvy}{Timor Pidgin}
\define@key{names}{xsb}{Tinà Sambal}
\define@key{names}{tit}{Tinigua}
\define@key{names}{tpz}{Tinputz}
\define@key{names}{tpe}{Tippera}
\define@key{names}{tra}{Tirahi}
\define@key{names}{tic}{Tira}
\define@key{names}{tde}{Tiranige Diga Dogon}
\define@key{names}{tdq}{Tita}
\define@key{names}{ttv}{Titan}
\define@key{names}{lax}{Tiwa (India)}
\define@key{names}{tju}{Tjurruru}
\define@key{names}{tpl}{Tlacoapa Me'phaa}
\define@key{names}{ctl}{Tlacoatzintepec Chinantec}
\define@key{names}{zpk}{Tlacolulita Zapotec}
\define@key{names}{nuz}{Tlamacazapa Nahuatl}
\define@key{names}{mqh}{Tlazoyaltepec Mixtec}
\define@key{names}{tmf}{Toba-Enenlhet}
\define@key{names}{tng}{Tobanga}
\define@key{names}{tgh}{Tobagonian Creole English}
\define@key{names}{tox}{Tobian}
\define@key{names}{tgb}{Tobilung}
\define@key{names}{taz}{Tocho}
\define@key{names}{tdr}{Todrah}
\define@key{names}{tlg}{Tofanma}
\define@key{names}{tfi}{Tofin Gbe}
\define@key{names}{tor}{Togbo-Vara Banda}
\define@key{names}{tgy}{Togoyo}
\define@key{names}{zuh}{Tokano}
\define@key{names}{xto}{Tokharian A}
\define@key{names}{txb}{Tokharian B}
\define@key{names}{tok}{Toki Pona}
\define@key{names}{tkn}{Toku-No-Shima}
\define@key{names}{lbw}{Tolaki}
\define@key{names}{tlm}{Tolomako}
\define@key{names}{tol}{Tolowa-Chetco}
\define@key{names}{tod}{Toma}
\define@key{names}{tdi}{Tomadino}
\define@key{names}{tom}{Tombulu}
\define@key{names}{txa}{Tombonuo}
\define@key{names}{ttp}{Tombelala}
\define@key{names}{txm}{Tomini}
\define@key{names}{dtm}{Tomo Kan Dogon}
\define@key{names}{tqp}{Tomoip}
\define@key{names}{tst}{Tondi Songway Kiini}
\define@key{names}{tnz}{Maniq}
\define@key{names}{tny}{Tongwe}
\define@key{names}{tog}{Tonga (Nyasa)}
\define@key{names}{xgf}{Tongva}
\define@key{names}{tjn}{Tonjon}
\define@key{names}{tnw}{Tonsawang}
\define@key{names}{txs}{Tonsea}
\define@key{names}{toz}{To}
\define@key{names}{ttj}{Tooro}
\define@key{names}{toq}{Toposa}
\define@key{names}{toy}{Topoiyo}
\define@key{names}{ttu}{Torau}
\define@key{names}{trz}{Torá}
\define@key{names}{trj}{Toram}
\define@key{names}{fit}{Meänkieli}
\define@key{names}{tdv}{Toro}
\define@key{names}{tqr}{Torona}
\define@key{names}{dtt}{Toro Tegu Dogon}
\define@key{names}{tno}{Toromono}
\define@key{names}{tei}{Aro}
\define@key{names}{als}{Northern Tosk Albanian}
\define@key{names}{ttl}{Totela}
\define@key{names}{txo}{Toto}
\define@key{names}{txe}{Totoli}
\define@key{names}{ttk}{Totoro}
\define@key{names}{zph}{Totomachapan Zapotec}
\define@key{names}{tqu}{Touo}
\define@key{names}{neb}{Toura (Côte d'Ivoire)}
\define@key{names}{don}{Toura (Papua New Guinea)}
\define@key{names}{ttn}{Towei}
\define@key{names}{xtg}{Transalpine Gaulish}
\define@key{names}{trl}{Traveller Scottish}
\define@key{names}{rmg}{Traveller Norwegian}
\define@key{names}{rmd}{Traveller Danish}
\define@key{names}{trm}{Tregami}
\define@key{names}{tme}{Tremembé}
\define@key{names}{stg}{Trieng}
\define@key{names}{tip}{Trimuris}
\define@key{names}{trx}{Tringgus-Sembaan Bidayuh}
\define@key{names}{tgq}{Tring}
\define@key{names}{trn}{Trinitario-Javeriano-Loretano}
\define@key{names}{trf}{Trinidadian Creole English}
\define@key{names}{lst}{Trinidad and Tobago Sign Language}
\define@key{names}{tka}{Truká}
\define@key{names}{tsa}{Tsaangi}
\define@key{names}{tsd}{Tsakonian}
\define@key{names}{kvz}{Tsaukambo}
\define@key{names}{tsb}{Tsamai}
\define@key{names}{tsk}{Tseku}
\define@key{names}{txc}{Tsetsaut}
\define@key{names}{kdl}{Tsikimba}
\define@key{names}{xmw}{Tsimihety Malagasy}
\define@key{names}{tsw}{Salka-Tsishingini}
\define@key{names}{hio}{Northern Tshwa}
\define@key{names}{ldp}{Tso}
\define@key{names}{lto}{Tsotso}
\define@key{names}{fly}{Tsotsitaal}
\define@key{names}{ttz}{Tsum}
\define@key{names}{tsl}{Ts'ün-Lao}
\define@key{names}{tvd}{Tsuvadi}
\define@key{names}{tsh}{Tsuvan}
\define@key{names}{two}{Tswapong}
\define@key{names}{tsc}{Tswa}
\define@key{names}{nrt}{Tualatin-Yamhill}
\define@key{names}{tuy}{Tugen}
\define@key{names}{tuj}{Tugutil}
\define@key{names}{khc}{Tukang Besi North}
\define@key{names}{bhq}{Tukang Besi South}
\define@key{names}{tkf}{Tukumanféd}
\define@key{names}{tkd}{Tukudede}
\define@key{names}{tul}{Tula}
\define@key{names}{tlu}{Tulehu}
\define@key{names}{tey}{Tulishi}
\define@key{names}{rak}{Tulu-Bohuai}
\define@key{names}{krt}{Tumari Kanuri}
\define@key{names}{iou}{Tuma-Irumu}
\define@key{names}{tum}{Tumbuka}
\define@key{names}{kku}{Tumi}
\define@key{names}{xtq}{Tumshuqese}
\define@key{names}{tbr}{Tumtum}
\define@key{names}{enh}{Tundra Enets}
\define@key{names}{trt}{Tunggare}
\define@key{names}{tse}{Tunisian Sign Language}
\define@key{names}{tug}{Tunia}
\define@key{names}{tjg}{Tunjung}
\define@key{names}{tqq}{Tunni}
\define@key{names}{dza}{Tunzu}
\define@key{names}{ttf}{Tuotomb}
\define@key{names}{tpr}{Tuparí}
\define@key{names}{tpw}{Lingua Geral Paulista}
\define@key{names}{trh}{Turaka}
\define@key{names}{trd}{Turi}
\define@key{names}{twt}{Turiwára}
\define@key{names}{tuz}{Turka}
\define@key{names}{tch}{Turks And Caicos Creole English}
\define@key{names}{tru}{Turoyo}
\define@key{names}{try}{Turung}
\define@key{names}{tqm}{Turumsa}
\define@key{names}{ttg}{Tutong}
\define@key{names}{tmi}{Tutuba}
\define@key{names}{mtu}{Tututepec Mixtec}
\define@key{names}{tww}{Tuwari}
\define@key{names}{ifk}{Tuwali Ifugao}
\define@key{names}{bov}{Tuwuli}
\define@key{names}{tud}{Tuxá}
\define@key{names}{tux}{Tuxináwa}
\define@key{names}{xjb}{Tweed-Albert}
\define@key{names}{twn}{Cambap-Langa}
\define@key{names}{uam}{Uamué}
\define@key{names}{ksj}{Uare}
\define@key{names}{byc}{Ubaghara}
\define@key{names}{uba}{Ubang}
\define@key{names}{ubi}{Ubi}
\define@key{names}{ubr}{Ubir}
\define@key{names}{cpb}{Ucayali-Yurúa Ashéninka}
\define@key{names}{uda}{Uda}
\define@key{names}{udu}{Uduk}
\define@key{names}{ufi}{Ufim}
\define@key{names}{uga}{Ugaritic}
\define@key{names}{uge}{Ughele}
\define@key{names}{ugo}{Ugong}
\define@key{names}{uha}{Uhami}
\define@key{names}{uis}{Uisai}
\define@key{names}{udj}{Ujir}
\define@key{names}{kcf}{Ukaan}
\define@key{names}{ukh}{Ukhwejo}
\define@key{names}{umi}{Ukit}
\define@key{names}{ukp}{Ukpe-Bayobiri}
\define@key{names}{akd}{Ukpet-Ehom}
\define@key{names}{ukl}{Ukrainian Sign Language}
\define@key{names}{uku}{Ukue}
\define@key{names}{ukg}{Ukuriguma}
\define@key{names}{ukq}{Ukwa}
\define@key{names}{ukw}{Ukwuani-Aboh-Ndoni}
\define@key{names}{svb}{Ulau-Suain}
\define@key{names}{ull}{Ullatan}
\define@key{names}{ulb}{Ulukwumi}
\define@key{names}{ulm}{Ulumanda'}
\define@key{names}{ulw}{Ulwa}
\define@key{names}{ulu}{Uma' Lung}
\define@key{names}{xky}{Uma' Lasan}
\define@key{names}{gdn}{Umanakaina}
\define@key{names}{umd}{Umbindhamu}
\define@key{names}{xum}{Umbrian}
\define@key{names}{umr}{Umbugarla}
\define@key{names}{umg}{Umbuygamu}
\define@key{names}{upi}{Umeda-Punda}
\define@key{names}{sju}{Ume Saami}
\define@key{names}{due}{Umiray Dumaget Agta}
\define@key{names}{umm}{Umon}
\define@key{names}{umo}{Umotína}
\define@key{names}{unz}{Unde Kaili}
\define@key{names}{bbn}{Uneapa}
\define@key{names}{une}{Uneme}
\define@key{names}{xgu}{Unggumi}
\define@key{names}{uni}{Uni}
\define@key{names}{uln}{Unserdeutsch}
\define@key{names}{onu}{Unua}
\define@key{names}{unu}{Unubahe}
\define@key{names}{tov}{Upper Taromi}
\define@key{names}{tku}{Upper Necaxa Totonac}
\define@key{names}{sxu}{Central East Middle German}
\define@key{names}{tth}{Upper Ta'oih}
\define@key{names}{dmg}{Upper Kinabatangan}
\define@key{names}{dna}{Upper Grand Valley Dani}
\define@key{names}{xup}{Upper Umpqua}
\define@key{names}{tau}{Upper Tanana}
\define@key{names}{url}{Urali of Idukki}
\define@key{names}{urm}{Urapmin}
\define@key{names}{uro}{Ura (Papua New Guinea)}
\define@key{names}{xur}{Urartian}
\define@key{names}{urg}{Urigina}
\define@key{names}{uvh}{Uri}
\define@key{names}{urx}{Urimo}
\define@key{names}{urc}{Urningangg}
\define@key{names}{urv}{Uruava}
\define@key{names}{urn}{Uruangnirin}
\define@key{names}{urz}{Uru-Eu-Wau-Wau}
\define@key{names}{ugy}{Uruguayan Sign Language}
\define@key{names}{uru}{Urumi}
\define@key{names}{urp}{Uru-Pa-In}
\define@key{names}{usk}{Usaghade}
\define@key{names}{ush}{Ushojo}
\define@key{names}{ulf}{Usku}
\define@key{names}{usp}{Uspanteco}
\define@key{names}{usi}{Usui}
\define@key{names}{omo}{Utarmbung}
\define@key{names}{wsg}{Adilabad Gondi}
\define@key{names}{utu}{Utu}
\define@key{names}{uuu}{U}
\define@key{names}{evh}{Uvbie}
\define@key{names}{usu}{Uya}
\define@key{names}{auz}{Uzbeki Arabic}
\define@key{names}{eze}{Uzekwe}
\define@key{names}{vaa}{Vaagri Booli}
\define@key{names}{kqu}{Vaal-Orange}
\define@key{names}{vgr}{Vaghri}
\define@key{names}{dkg}{Kadung}
\define@key{names}{tva}{Vaghua}
\define@key{names}{vap}{Vaiphei}
\define@key{names}{vae}{Vale}
\define@key{names}{vsv}{Valencian Sign Language}
\define@key{names}{vmv}{Valley Maidu}
\define@key{names}{cvn}{Valle Nacional Chinantec}
\define@key{names}{vlp}{Valpei}
\define@key{names}{mkt}{Vamale}
\define@key{names}{mlr}{Vame}
\define@key{names}{mpr}{Vangunu}
\define@key{names}{vnk}{Lovono}
\define@key{names}{vau}{Vanuma}
\define@key{names}{vao}{Vao}
\define@key{names}{vah}{Varhadi-Nagpuri}
\define@key{names}{vrs}{Varisi}
\define@key{names}{vav}{Dungar Varli}
\define@key{names}{vaj}{Northern Ju}
\define@key{names}{val}{Vehes}
\define@key{names}{vem}{Vemgo-Mabas}
\define@key{names}{vsl}{Venezuelan Sign Language}
\define@key{names}{xve}{Venetic}
\define@key{names}{vec}{Venetian}
\define@key{names}{veo}{Ventureño}
\define@key{names}{vra}{Vera'a}
\define@key{names}{vid}{Vidunda}
\define@key{names}{vig}{Viemo}
\define@key{names}{vil}{Vilela}
\define@key{names}{dyg}{Villa Viciosa Agta}
\define@key{names}{svc}{Vincentian Creole English}
\define@key{names}{vin}{Vinza}
\define@key{names}{vic}{Virgin Islands Creole English}
\define@key{names}{vis}{Vishavan}
\define@key{names}{vit}{Viti}
\define@key{names}{vto}{Vitou}
\define@key{names}{vls}{Western Flemish}
\define@key{names}{vol}{Volapük}
\define@key{names}{kch}{Vono}
\define@key{names}{vor}{Voro}
\define@key{names}{vum}{Vumbu}
\define@key{names}{vnp}{Vunapu}
\define@key{names}{vun}{Vunjo}
\define@key{names}{msn}{Vurës}
\define@key{names}{vut}{Vute}
\define@key{names}{wbi}{Vwanji}
\define@key{names}{wmn}{Waamwang}
\define@key{names}{wab}{Wab}
\define@key{names}{wbb}{Wabo}
\define@key{names}{kmx}{Waboda}
\define@key{names}{wci}{Waci Gbe}
\define@key{names}{wdg}{Wadaginam}
\define@key{names}{wbq}{Waddar}
\define@key{names}{kxp}{Wadiyara Koli}
\define@key{names}{wdu}{Wadjigu}
\define@key{names}{wag}{Wa'ema}
\define@key{names}{wrx}{Kolor}
\define@key{names}{waj}{Waffa}
\define@key{names}{wga}{Wagaya}
\define@key{names}{wgb}{Wagawaga}
\define@key{names}{wbr}{Wagdi}
\define@key{names}{fad}{Wagi (Papua New Guinea)}
\define@key{names}{whk}{Eastern Lowland Kenyah}
\define@key{names}{wgo}{Waigeo}
\define@key{names}{wlr}{Ale}
\define@key{names}{wlk}{Eel River Athabaskan}
\define@key{names}{wmh}{Waima'a}
\define@key{names}{atr}{Waimiri-Atroari}
\define@key{names}{wli}{Waioli}
\define@key{names}{wja}{Waja}
\define@key{names}{wav}{Waka}
\define@key{names}{wwb}{Wakabunga}
\define@key{names}{wkd}{Wakde}
\define@key{names}{waf}{Wakoná}
\define@key{names}{lgl}{Wala}
\define@key{names}{wlw}{Walak}
\define@key{names}{wly}{Waling}
\define@key{names}{wll}{Wali (Sudan)}
\define@key{names}{wlx}{Wali (Ghana)}
\define@key{names}{waa}{Northeast Sahaptin}
\define@key{names}{wln}{Walloon}
\define@key{names}{wae}{Walser}
\define@key{names}{ola}{Walungge}
\define@key{names}{wmc}{Wamas}
\define@key{names}{wmi}{Wamin}
\define@key{names}{lbq}{Wampar}
\define@key{names}{waz}{Wampur}
\define@key{names}{qyp}{Wampano}
\define@key{names}{wnp}{Wanap}
\define@key{names}{wnb}{Mokati}
\define@key{names}{nnp}{Wancho Naga}
\define@key{names}{wbh}{Wanda}
\define@key{names}{wdd}{Wandji}
\define@key{names}{wad}{Wandamen}
\define@key{names}{mfi}{Wandala}
\define@key{names}{wne}{Waneci}
\define@key{names}{hwa}{Wané}
\define@key{names}{wnm}{Wanggamala}
\define@key{names}{lwg}{Wanga}
\define@key{names}{wng}{Wanggom}
\define@key{names}{jub}{Wannu}
\define@key{names}{wno}{Wano}
\define@key{names}{wnk}{Wanukaka}
\define@key{names}{wny}{Wanyi}
\define@key{names}{juk}{Wapan}
\define@key{names}{juw}{Wãpha}
\define@key{names}{wbf}{Samue}
\define@key{names}{tci}{Anta-Komnzo-Wára-Wérè-Kémä}
\define@key{names}{srv}{Waray Sorsogon}
\define@key{names}{bpe}{Bauni}
\define@key{names}{wre}{Ware}
\define@key{names}{wai}{Wares}
\define@key{names}{wri}{Wariyangga}
\define@key{names}{wbe}{Waritai}
\define@key{names}{aml}{War-Jaintia}
\define@key{names}{wji}{Warji}
\define@key{names}{bgv}{Warkay-Bipim}
\define@key{names}{wrl}{Warlmanpa}
\define@key{names}{wrn}{Warnang}
\define@key{names}{wru}{Waru}
\define@key{names}{wrv}{Waruna}
\define@key{names}{wss}{Wasa}
\define@key{names}{gsp}{Wasembo}
\define@key{names}{wsu}{Wasu}
\define@key{names}{wtk}{Watakataui}
\define@key{names}{wah}{Watubela}
\define@key{names}{wuy}{Wauyai}
\define@key{names}{www}{Wawa}
\define@key{names}{wow}{Wawonii}
\define@key{names}{wxa}{Waxianghua}
\define@key{names}{ctt}{Wayanad Chetti}
\define@key{names}{wyr}{Wayoró}
\define@key{names}{weh}{Weh}
\define@key{names}{wew}{Wewewa}
\define@key{names}{wlh}{Welaun}
\define@key{names}{klh}{Weliki}
\define@key{names}{wei}{Were}
\define@key{names}{gxx}{Wè Southern}
\define@key{names}{ywl}{Western Lalu}
\define@key{names}{hmw}{Western Mashan Hmong}
\define@key{names}{ojw}{Western Ojibwa}
\define@key{names}{tqt}{Ozumatlán Totonac}
\define@key{names}{yih}{Western Yiddish}
\define@key{names}{pnb}{Western Panjabi}
\define@key{names}{lcp}{Western Lawa}
\define@key{names}{kuf}{Western Katu}
\define@key{names}{mut}{Western Muria}
\define@key{names}{kyu}{Western Kayah}
\define@key{names}{tdg}{Western Tamang}
\define@key{names}{wmg}{Western Muya}
\define@key{names}{raf}{Western Meohang}
\define@key{names}{mmr}{Western Xiangxi Miao}
\define@key{names}{lia}{West-Central Limba}
\define@key{names}{xwl}{Western Xwla Gbe}
\define@key{names}{bbp}{West Central Banda}
\define@key{names}{ssl}{Western Sisaala}
\define@key{names}{krw}{Western Krahn}
\define@key{names}{nnd}{West Ambae}
\define@key{names}{uve}{West Uvean}
\define@key{names}{mss}{West Masela}
\define@key{names}{lmj}{West Lembata}
\define@key{names}{drn}{West Damar}
\define@key{names}{suc}{Western Subanon}
\define@key{names}{twb}{Western Tawbuid}
\define@key{names}{pne}{Western Penan}
\define@key{names}{zbw}{West Berawan}
\define@key{names}{dnw}{Western Dani}
\define@key{names}{nhw}{Western Huasteca Nahuatl}
\define@key{names}{pua}{Western Highland Purepecha}
\define@key{names}{gnw}{Western Bolivian Guaraní}
\define@key{names}{jmx}{Western Juxtlahuaca Mixtec}
\define@key{names}{tnb}{Western Tunebo}
\define@key{names}{amw}{Western Neo-Aramaic}
\define@key{names}{azn}{Western Durango Nahuatl}
\define@key{names}{wwo}{Dorig}
\define@key{names}{wea}{Wewaw}
\define@key{names}{wec}{Wè Western}
\define@key{names}{woy}{Weyto}
\define@key{names}{lwh}{White Lachi}
\define@key{names}{giw}{Duoluo Gelao}
\define@key{names}{tnp}{Whitesands}
\define@key{names}{tua}{Wiarumus}
\define@key{names}{mtp}{Wichí Lhamtés Nocten}
\define@key{names}{wlv}{Wichí Lhamtés Vejoz}
\define@key{names}{wik}{Wikalkan}
\define@key{names}{wie}{Wik-Epa}
\define@key{names}{wij}{Wik-Iiyanh}
\define@key{names}{wif}{Wik-Keyangan}
\define@key{names}{wih}{Wik-Me'anha}
\define@key{names}{wua}{Wikngenchera}
\define@key{names}{wil}{Wilawila}
\define@key{names}{wit}{Wintu}
\define@key{names}{gdr}{Wipi}
\define@key{names}{wrh}{Wiradhuri}
\define@key{names}{wir}{Wiraféd}
\define@key{names}{wiu}{Wiru}
\define@key{names}{xwc}{Woccon}
\define@key{names}{woc}{Wogeo}
\define@key{names}{wbw}{Woi}
\define@key{names}{wyi}{Woiwurrung-Thagungwurrung}
\define@key{names}{jod}{Wojenaka}
\define@key{names}{wod}{Wolani}
\define@key{names}{wle}{Wolane}
\define@key{names}{wom}{Wom (Nigeria)}
\define@key{names}{wmo}{Wom (Papua New Guinea)}
\define@key{names}{won}{Wongo}
\define@key{names}{cwd}{Woods Cree}
\define@key{names}{kda}{Worimi}
\define@key{names}{wor}{Woria}
\define@key{names}{jud}{Worodougou}
\define@key{names}{wsv}{Wotapuri-Katarqalai}
\define@key{names}{wtw}{Wotu}
\define@key{names}{wud}{Wudu}
\define@key{names}{qgu}{Wulguru}
\define@key{names}{wlu}{Wuliwuli}
\define@key{names}{wux}{Wulna}
\define@key{names}{bqm}{Wumboko-Bubia}
\define@key{names}{wum}{Wumbvu}
\define@key{names}{ywu}{Wumeng Nasu}
\define@key{names}{bwn}{Wunai Bunu}
\define@key{names}{wub}{Wunambal}
\define@key{names}{wur}{Wurrugu}
\define@key{names}{yig}{Wusa Nasu}
\define@key{names}{bse}{Wushi}
\define@key{names}{wsi}{Kula (Vanuatu)}
\define@key{names}{wuh}{Wutunhua}
\define@key{names}{wut}{Wutung}
\define@key{names}{wuv}{Wuvulu-Aua}
\define@key{names}{wym}{Wymysorys}
\define@key{names}{zax}{Xadani Zapotec}
\define@key{names}{xkr}{Xakriabá}
\define@key{names}{xan}{Xamtanga}
\define@key{names}{ztg}{Xanaguía Zapotec}
\define@key{names}{axx}{Xaragure}
\define@key{names}{xeg}{//Xegwi}
\define@key{names}{xet}{Xetá}
\define@key{names}{hsn}{Xiang Chinese}
\define@key{names}{sjo}{Xibe}
\define@key{names}{asn}{Xingú Asuriní}
\define@key{names}{xiy}{Xipaya}
\define@key{names}{xip}{Xipináwa}
\define@key{names}{xii}{Xiri}
\define@key{names}{xoo}{Xukurú}
\define@key{names}{xwe}{Xwela Gbe}
\define@key{names}{tyy}{Tiyaa}
\define@key{names}{muu}{Yaaku}
\define@key{names}{yar}{Yabarana}
\define@key{names}{ybn}{Yabaâna-Mainatari}
\define@key{names}{ybm}{Yaben}
\define@key{names}{ybo}{Yabong}
\define@key{names}{ekr}{Yace}
\define@key{names}{rys}{Yaeyama}
\define@key{names}{wfg}{Yafi}
\define@key{names}{ygm}{Yagomi}
\define@key{names}{ygw}{Yagwoia}
\define@key{names}{rhp}{Yahang}
\define@key{names}{ner}{Yahadian}
\define@key{names}{ynu}{Yahuna}
\define@key{names}{iyx}{Yaka (Congo)}
\define@key{names}{ykk}{Yakaikeke}
\define@key{names}{ybh}{Yakkha}
\define@key{names}{xyl}{Yalakalore}
\define@key{names}{yba}{Yala}
\define@key{names}{jal}{Yalahatan-Haruru-Awaiya}
\define@key{names}{zpu}{Yalálag Zapotec}
\define@key{names}{yal}{Yalunka}
\define@key{names}{ymp}{Yamap}
\define@key{names}{yat}{Yambeta}
\define@key{names}{ymb}{Yambes}
\define@key{names}{yme}{Yameo}
\define@key{names}{ymn}{Yamna}
\define@key{names}{qur}{Chaupihuaranga Quechua}
\define@key{names}{yda}{Yanda}
\define@key{names}{dym}{Yanda Dom Dogon}
\define@key{names}{xyb}{Yandjibara}
\define@key{names}{zyg}{Yang Zhuang}
\define@key{names}{jng}{Yangman}
\define@key{names}{yng}{Yango}
\define@key{names}{bsx}{Yangkam}
\define@key{names}{yav}{Yangben}
\define@key{names}{ygl}{Yangum Gel}
\define@key{names}{ymo}{Yangum Mon}
\define@key{names}{yde}{Yangum Dey}
\define@key{names}{ynl}{Yangulam}
\define@key{names}{tjj}{Yangathimri}
\define@key{names}{ysm}{Yangon Myanmar Sign Language}
\define@key{names}{jay}{Nhangu}
\define@key{names}{guu}{Yanomamö}
\define@key{names}{asy}{Yaosakor Asmat}
\define@key{names}{yre}{Yaouré}
\define@key{names}{yev}{Yeri}
\define@key{names}{yrw}{Yarawata}
\define@key{names}{zae}{Yareni Zapotec}
\define@key{names}{yro}{Yaroame}
\define@key{names}{yko}{Yasa}
\define@key{names}{zty}{Yatee Zapotec}
\define@key{names}{yla}{Ulwa (Papua New Guinea)}
\define@key{names}{yuw}{Yau-Nungon}
\define@key{names}{jau}{Yaur}
\define@key{names}{yyu}{Yau (Sandaun Province)}
\define@key{names}{zpb}{Yautepec Zapotec}
\define@key{names}{qux}{Yauyos Quechua}
\define@key{names}{yvt}{Yavitero-Pareni}
\define@key{names}{yww}{Yawarawarga}
\define@key{names}{ywn}{Yawanawa}
\define@key{names}{yaw}{Yawalapití}
\define@key{names}{yby}{Yaweyuha}
\define@key{names}{ybx}{Yawiyo}
\define@key{names}{ykr}{Yekora}
\define@key{names}{yel}{Yela-Kela}
\define@key{names}{ylg}{Yalaku}
\define@key{names}{ynq}{Yendang}
\define@key{names}{yec}{Yeniche}
\define@key{names}{yei}{Yeni}
\define@key{names}{yra}{Yerakai}
\define@key{names}{gop}{Yeretuar}
\define@key{names}{yrn}{Yerong-Southern Buyang}
\define@key{names}{yeu}{Yerukula}
\define@key{names}{yes}{Yeskwa}
\define@key{names}{yet}{Yetfa}
\define@key{names}{yej}{Yevanic}
\define@key{names}{ydg}{Yidgha}
\define@key{names}{yim}{Yimchungru Naga}
\define@key{names}{kvu}{Yinbaw Karen}
\define@key{names}{yin}{Yinchia}
\define@key{names}{yil}{Yindjilandji}
\define@key{names}{ywg}{Yinhawangka}
\define@key{names}{kvy}{Yintale Karen}
\define@key{names}{yxm}{Yinwum}
\define@key{names}{ljw}{Yirandhali}
\define@key{names}{yiy}{Yir-Yoront}
\define@key{names}{yis}{Yis}
\define@key{names}{gek}{Yiwom}
\define@key{names}{yob}{Yoba}
\define@key{names}{gud}{Yocoboué Dida}
\define@key{names}{yog}{Yogad}
\define@key{names}{ydk}{Yoidik}
\define@key{names}{yki}{Yoke}
\define@key{names}{ygs}{Yolngu Sign Language}
\define@key{names}{xty}{Yoloxochitl Mixtec}
\define@key{names}{pil}{Yom}
\define@key{names}{yoi}{Yonaguni}
\define@key{names}{sxk}{Yoncalla}
\define@key{names}{nru}{Narua}
\define@key{names}{zyn}{Yongnan Zhuang}
\define@key{names}{zyb}{Yongbei Zhuang}
\define@key{names}{yno}{Yong}
\define@key{names}{yon}{Yonggom}
\define@key{names}{yut}{Yopno}
\define@key{names}{mts}{Yora}
\define@key{names}{yox}{Yoron}
\define@key{names}{yot}{Yotti}
\define@key{names}{zyj}{Youjiang Zhuang}
\define@key{names}{ytw}{Yout Wam}
\define@key{names}{yoy}{Yoy}
\define@key{names}{nua}{Yuaga}
\define@key{names}{msd}{Yucatec Maya Sign Language}
\define@key{names}{mvg}{Yucuañe Mixtec}
\define@key{names}{yub}{Yugambal}
\define@key{names}{ysl}{Yugoslavian Sign Language}
\define@key{names}{ygu}{Yugul}
\define@key{names}{yab}{Yuhup}
\define@key{names}{omk}{Malyj Anjuj Omok}
\define@key{names}{ybl}{Yukuben}
\define@key{names}{yuq}{Yuqui}
\define@key{names}{ljx}{Yuru}
\define@key{names}{mab}{Yutanduchi Mixtec}
\define@key{names}{yau}{Hoti}
\define@key{names}{ztx}{Zaachila Zapotec}
\define@key{names}{kji}{Zabana}
\define@key{names}{nhi}{Zacatlán-Ahuacatlán-Tepetzintla Nahuatl}
\define@key{names}{ctz}{Zacatepec Chatino}
\define@key{names}{atb}{Zaiwa}
\define@key{names}{zkr}{Zakhring}
\define@key{names}{zsl}{Zambian Sign Language}
\define@key{names}{zak}{Zanaki}
\define@key{names}{zau}{Zangskari}
\define@key{names}{zna}{Zan Gula}
\define@key{names}{zah}{Zangwal}
\define@key{names}{zpw}{Zaniza Zapotec}
\define@key{names}{zaj}{Zaramo}
\define@key{names}{zbu}{Bu (Zaranda)}
\define@key{names}{zaz}{Zari}
\define@key{names}{zal}{Zauzou}
\define@key{names}{kxk}{Lahta-Zayein Karen}
\define@key{names}{zwa}{Zay}
\define@key{names}{jaj}{Zazao}
\define@key{names}{zua}{Zeem}
\define@key{names}{dhm}{Zemba}
\define@key{names}{zeg}{Zenag}
\define@key{names}{czn}{Zenzontepec Chatino}
\define@key{names}{zhb}{Zhaba}
\define@key{names}{xzh}{Zhangzhung}
\define@key{names}{zhi}{Zhire}
\define@key{names}{zhw}{Zhoa}
\define@key{names}{zia}{Zia}
\define@key{names}{zil}{Zialo}
\define@key{names}{ziw}{Zigula-Mushungulu}
\define@key{names}{zib}{Zimbabwe Sign Language}
\define@key{names}{zmb}{Zimba}
\define@key{names}{zin}{Zinza}
\define@key{names}{sih}{Zire}
\define@key{names}{zrn}{Zirenkel}
\define@key{names}{ziz}{Zizilivakan}
\define@key{names}{pto}{Zo'é}
\define@key{names}{yzk}{Zokhuo}
\define@key{names}{gbz}{Zoroastrian Yazdi}
\define@key{names}{czt}{Zotung Chin}
\define@key{names}{zom}{Zou}
\define@key{names}{zla}{Zula}
\define@key{names}{gnd}{Zulgo-Gemzek}
\define@key{names}{zuy}{Zumaya}
\define@key{names}{jmb}{Zumbun}
\define@key{names}{zzj}{Zuojiang Zhuang}
\define@key{names}{zyp}{Zyphe}

                     \define@key{fams}{knw}{Kxa}
\define@key{fams}{nmn}{Tuu}
\define@key{fams}{alu}{Austronesian}
\define@key{fams}{hnh}{Khoe-Kwadi}
\define@key{fams}{xam}{Tuu}
\define@key{fams}{huc}{Kxa}
\define@key{fams}{apq}{Great Andamanese}
\define@key{fams}{aiw}{South Omotic}
\define@key{fams}{aau}{Sepik}
\define@key{fams}{abq}{Abkhaz-Adyge}
\define@key{fams}{abe}{Algic}
\define@key{fams}{abi}{Atlantic-Congo}
\define@key{fams}{axb}{Guaicuruan}
\define@key{fams}{abk}{Abkhaz-Adyge}
\define@key{fams}{abz}{Timor-Alor-Pantar}
\define@key{fams}{kgr}{Isolate}
\define@key{fams}{ace}{Austronesian}
\define@key{fams}{aca}{Arawakan}
\define@key{fams}{acn}{Sino-Tibetan}
\define@key{fams}{ach}{Nilotic}
\define@key{fams}{acu}{Chicham}
\define@key{fams}{acv}{Palaihnihan}
\define@key{fams}{guq}{Tupian}
\define@key{fams}{acr}{Mayan}
\define@key{fams}{kjq}{Keresan}
\define@key{fams}{ads}{Sign Language}
\define@key{fams}{adn}{Timor-Alor-Pantar}
\define@key{fams}{adj}{Atlantic-Congo}
\define@key{fams}{ady}{Abkhaz-Adyge}
\define@key{fams}{adt}{Pama-Nyungan}
\define@key{fams}{adz}{Austronesian}
\define@key{fams}{awi}{Kamula-Elevala}
\define@key{fams}{afr}{Indo-European}
\define@key{fams}{agd}{Nuclear Trans New Guinea}
\define@key{fams}{agq}{Atlantic-Congo}
\define@key{fams}{ahh}{Nuclear Trans New Guinea}
\define@key{fams}{agx}{Nakh-Daghestanian}
\define@key{fams}{agt}{Austronesian}
\define@key{fams}{duo}{Austronesian}
\define@key{fams}{agu}{Mayan}
\define@key{fams}{agr}{Chicham}
\define@key{fams}{aht}{Athabaskan-Eyak-Tlingit}
\define@key{fams}{tba}{Isolate}
\define@key{fams}{ain}{Ainu}
\define@key{fams}{ahp}{Atlantic-Congo}
\define@key{fams}{aja}{Kresh-Aja}
\define@key{fams}{ajg}{Atlantic-Congo}
\define@key{fams}{aji}{Austronesian}
\define@key{fams}{axk}{Atlantic-Congo}
\define@key{fams}{abj}{Great Andamanese}
\define@key{fams}{aci}{Great Andamanese}
\define@key{fams}{akx}{Great Andamanese}
\define@key{fams}{aka}{Atlantic-Congo}
\define@key{fams}{ake}{Cariban}
\define@key{fams}{ahk}{Sino-Tibetan}
\define@key{fams}{akv}{Nakh-Daghestanian}
\define@key{fams}{akl}{Austronesian}
\define@key{fams}{akw}{Atlantic-Congo}
\define@key{fams}{nrz}{Austronesian}
\define@key{fams}{akz}{Muskogean}
\define@key{fams}{wbj}{Afro-Asiatic}
\define@key{fams}{amp}{Sepik}
\define@key{fams}{btz}{Austronesian}
\define@key{fams}{alh}{Mangarrayi-Maran}
\define@key{fams}{sqi}{nan}
\define@key{fams}{ale}{Eskimo-Aleut}
\define@key{fams}{alq}{Algic}
\define@key{fams}{ald}{Atlantic-Congo}
\define@key{fams}{gsw}{Indo-European}
\define@key{fams}{aes}{Isolate}
\define@key{fams}{alt}{Turkic}
\define@key{fams}{alp}{Austronesian}
\define@key{fams}{ems}{Eskimo-Aleut}
\define@key{fams}{alr}{Chukotko-Kamchatkan}
\define@key{fams}{aly}{Pama-Nyungan}
\define@key{fams}{amm}{Left May}
\define@key{fams}{amc}{Pano-Tacanan}
\define@key{fams}{amn}{Border}
\define@key{fams}{aie}{Austronesian}
\define@key{fams}{amr}{Harakmbut}
\define@key{fams}{omb}{Austronesian}
\define@key{fams}{amk}{Austronesian}
\define@key{fams}{abt}{Ndu}
\define@key{fams}{adx}{Sino-Tibetan}
\define@key{fams}{aey}{Nuclear Trans New Guinea}
\define@key{fams}{ase}{Sign Language}
\define@key{fams}{amh}{Afro-Asiatic}
\define@key{fams}{ami}{Austronesian}
\define@key{fams}{amo}{Atlantic-Congo}
\define@key{fams}{apz}{Angan}
\define@key{fams}{ame}{Arawakan}
\define@key{fams}{amu}{Otomanguean}
\define@key{fams}{imi}{Nuclear Trans New Guinea}
\define@key{fams}{ani}{Nakh-Daghestanian}
\define@key{fams}{ano}{Isolate}
\define@key{fams}{aty}{Austronesian}
\define@key{fams}{agm}{Angan}
\define@key{fams}{njm}{Sino-Tibetan}
\define@key{fams}{anc}{Afro-Asiatic}
\define@key{fams}{agg}{Senagi}
\define@key{fams}{aoa}{Indo-European}
\define@key{fams}{awg}{Pama-Nyungan}
\define@key{fams}{aoi}{Gunwinyguan}
\define@key{fams}{nun}{Sino-Tibetan}
\define@key{fams}{cko}{Atlantic-Congo}
\define@key{fams}{any}{Atlantic-Congo}
\define@key{fams}{anu}{Nilotic}
\define@key{fams}{anz}{Isolate}
\define@key{fams}{njo}{Sino-Tibetan}
\define@key{fams}{apm}{Athabaskan-Eyak-Tlingit}
\define@key{fams}{apj}{Athabaskan-Eyak-Tlingit}
\define@key{fams}{apw}{Athabaskan-Eyak-Tlingit}
\define@key{fams}{apy}{Cariban}
\define@key{fams}{apt}{Sino-Tibetan}
\define@key{fams}{apn}{Nuclear-Macro-Je}
\define@key{fams}{apu}{Arawakan}
\define@key{fams}{ard}{Pama-Nyungan}
\define@key{fams}{arl}{Zaparoan}
\define@key{fams}{abv}{Afro-Asiatic}
\define@key{fams}{mey}{Afro-Asiatic}
\define@key{fams}{shu}{Afro-Asiatic}
\define@key{fams}{ayl}{Afro-Asiatic}
\define@key{fams}{arz}{Afro-Asiatic}
\define@key{fams}{afb}{Afro-Asiatic}
\define@key{fams}{acw}{Afro-Asiatic}
\define@key{fams}{acm}{Afro-Asiatic}
\define@key{fams}{acy}{Afro-Asiatic}
\define@key{fams}{arb}{Afro-Asiatic}
\define@key{fams}{ary}{Afro-Asiatic}
\define@key{fams}{ajp}{Afro-Asiatic}
\define@key{fams}{ayn}{Afro-Asiatic}
\define@key{fams}{apc}{Afro-Asiatic}
\define@key{fams}{aeb}{Afro-Asiatic}
\define@key{fams}{rmz}{Sino-Tibetan}
\define@key{fams}{akr}{Austronesian}
\define@key{fams}{atq}{Austronesian}
\define@key{fams}{jbj}{South Bird's Head Family}
\define@key{fams}{aro}{Pano-Tacanan}
\define@key{fams}{arp}{Algic}
\define@key{fams}{aah}{Nuclear Torricelli}
\define@key{fams}{ape}{Nuclear Torricelli}
\define@key{fams}{arv}{Afro-Asiatic}
\define@key{fams}{aqc}{Nakh-Daghestanian}
\define@key{fams}{laz}{Austronesian}
\define@key{fams}{ari}{Caddoan}
\define@key{fams}{hye}{Indo-European}
\define@key{fams}{hyw}{Indo-European}
\define@key{fams}{apr}{Austronesian}
\define@key{fams}{aia}{Austronesian}
\define@key{fams}{aer}{Pama-Nyungan}
\define@key{fams}{are}{Pama-Nyungan}
\define@key{fams}{cns}{Nuclear Trans New Guinea}
\define@key{fams}{asm}{Indo-European}
\define@key{fams}{ast}{Indo-European}
\define@key{fams}{asu}{Tupian}
\define@key{fams}{kuz}{Isolate}
\define@key{fams}{aqp}{Isolate}
\define@key{fams}{tay}{Austronesian}
\define@key{fams}{upv}{Austronesian}
\define@key{fams}{aph}{Sino-Tibetan}
\define@key{fams}{atj}{Algic}
\define@key{fams}{atw}{Palaihnihan}
\define@key{fams}{avt}{Nuclear Torricelli}
\define@key{fams}{aul}{Austronesian}
\define@key{fams}{asf}{Sign Language}
\define@key{fams}{auy}{Nuclear Trans New Guinea}
\define@key{fams}{ava}{Nakh-Daghestanian}
\define@key{fams}{avn}{Atlantic-Congo}
\define@key{fams}{avi}{Atlantic-Congo}
\define@key{fams}{avu}{Central Sudanic}
\define@key{fams}{awb}{Nuclear Trans New Guinea}
\define@key{fams}{kwi}{Barbacoan}
\define@key{fams}{awa}{Indo-European}
\define@key{fams}{awn}{Afro-Asiatic}
\define@key{fams}{kmn}{Sepik}
\define@key{fams}{auw}{Border}
\define@key{fams}{nfl}{Austronesian}
\define@key{fams}{ayr}{Aymaran}
\define@key{fams}{aib}{Turkic}
\define@key{fams}{ayo}{Zamucoan}
\define@key{fams}{azb}{Turkic}
\define@key{fams}{koe}{Surmic}
\define@key{fams}{bvx}{Atlantic-Congo}
\define@key{fams}{bav}{Atlantic-Congo}
\define@key{fams}{wdj}{Isolate}
\define@key{fams}{bfq}{Dravidian}
\define@key{fams}{bde}{Afro-Asiatic}
\define@key{fams}{bia}{Pama-Nyungan}
\define@key{fams}{ksf}{Atlantic-Congo}
\define@key{fams}{bfd}{Atlantic-Congo}
\define@key{fams}{bsp}{Atlantic-Congo}
\define@key{fams}{bmi}{Central Sudanic}
\define@key{fams}{fuu}{Central Sudanic}
\define@key{fams}{bgq}{Indo-European}
\define@key{fams}{kva}{Nakh-Daghestanian}
\define@key{fams}{bdw}{West Bomberai}
\define@key{fams}{bjh}{Sepik}
\define@key{fams}{bdq}{Austroasiatic}
\define@key{fams}{bca}{Sino-Tibetan}
\define@key{fams}{bdl}{Austronesian}
\define@key{fams}{bdr}{Austronesian}
\define@key{fams}{bkc}{Atlantic-Congo}
\define@key{fams}{bdh}{Central Sudanic}
\define@key{fams}{bkq}{Cariban}
\define@key{fams}{bri}{Atlantic-Congo}
\define@key{fams}{blw}{Austronesian}
\define@key{fams}{blz}{Austronesian}
\define@key{fams}{ban}{Austronesian}
\define@key{fams}{bft}{Sino-Tibetan}
\define@key{fams}{bgn}{Indo-European}
\define@key{fams}{ptu}{Austronesian}
\define@key{fams}{bam}{Mande}
\define@key{fams}{bax}{Atlantic-Congo}
\define@key{fams}{bcw}{Afro-Asiatic}
\define@key{fams}{jaa}{Arawan}
\define@key{fams}{bza}{Mande}
\define@key{fams}{bdy}{Pama-Nyungan}
\define@key{fams}{bgz}{Austronesian}
\define@key{fams}{bjb}{Pama-Nyungan}
\define@key{fams}{bdg}{Austronesian}
\define@key{fams}{dba}{Isolate}
\define@key{fams}{bvv}{nan}
\define@key{fams}{bwi}{Arawakan}
\define@key{fams}{abb}{Atlantic-Congo}
\define@key{fams}{bcm}{Austronesian}
\define@key{fams}{bnq}{Austronesian}
\define@key{fams}{peh}{Mongolic-Khitan}
\define@key{fams}{bci}{Atlantic-Congo}
\define@key{fams}{loy}{Sino-Tibetan}
\define@key{fams}{bbb}{Koiarian}
\define@key{fams}{brm}{Atlantic-Congo}
\define@key{fams}{bsn}{Tucanoan}
\define@key{fams}{bcj}{Nyulnyulan}
\define@key{fams}{mlp}{Nuclear Trans New Guinea}
\define@key{fams}{bfa}{Nilotic}
\define@key{fams}{bba}{Atlantic-Congo}
\define@key{fams}{wra}{nan}
\define@key{fams}{byr}{Angan}
\define@key{fams}{bae}{Arawakan}
\define@key{fams}{mot}{Chibchan}
\define@key{fams}{bsc}{Atlantic-Congo}
\define@key{fams}{bas}{Atlantic-Congo}
\define@key{fams}{bak}{Turkic}
\define@key{fams}{eus}{Isolate}
\define@key{fams}{bya}{Austronesian}
\define@key{fams}{btx}{Austronesian}
\define@key{fams}{bbc}{Austronesian}
\define@key{fams}{bhm}{Afro-Asiatic}
\define@key{fams}{bbd}{Nuclear Trans New Guinea}
\define@key{fams}{brg}{Arawakan}
\define@key{fams}{bvz}{Geelvink Bay}
\define@key{fams}{bgr}{Sino-Tibetan}
\define@key{fams}{bsw}{Afro-Asiatic}
\define@key{fams}{bxj}{Pama-Nyungan}
\define@key{fams}{beq}{Atlantic-Congo}
\define@key{fams}{dbj}{Austronesian}
\define@key{fams}{bej}{Afro-Asiatic}
\define@key{fams}{byw}{Sino-Tibetan}
\define@key{fams}{blc}{Salishan}
\define@key{fams}{bel}{Indo-European}
\define@key{fams}{bem}{Atlantic-Congo}
\define@key{fams}{bef}{Nuclear Trans New Guinea}
\define@key{fams}{nhb}{Mande}
\define@key{fams}{bng}{Atlantic-Congo}
\define@key{fams}{ben}{Indo-European}
\define@key{fams}{ctg}{Indo-European}
\define@key{fams}{bue}{Isolate}
\define@key{fams}{brf}{Atlantic-Congo}
\define@key{fams}{shy}{Afro-Asiatic}
\define@key{fams}{grr}{Afro-Asiatic}
\define@key{fams}{tzm}{Afro-Asiatic}
\define@key{fams}{mzb}{Afro-Asiatic}
\define@key{fams}{rif}{Afro-Asiatic}
\define@key{fams}{siz}{Afro-Asiatic}
\define@key{fams}{oua}{Afro-Asiatic}
\define@key{fams}{brc}{Indo-European}
\define@key{fams}{zag}{Saharan}
\define@key{fams}{bkl}{Tor-Orya}
\define@key{fams}{wti}{Isolate}
\define@key{fams}{xub}{Dravidian}
\define@key{fams}{kap}{Nakh-Daghestanian}
\define@key{fams}{bhb}{Indo-European}
\define@key{fams}{bho}{Indo-European}
\define@key{fams}{unr}{Austroasiatic}
\define@key{fams}{bif}{Atlantic-Congo}
\define@key{fams}{bhw}{Austronesian}
\define@key{fams}{bth}{Austronesian}
\define@key{fams}{bid}{Afro-Asiatic}
\define@key{fams}{bcl}{Austronesian}
\define@key{fams}{bip}{Atlantic-Congo}
\define@key{fams}{bpr}{Austronesian}
\define@key{fams}{byn}{Afro-Asiatic}
\define@key{fams}{nbj}{Pama-Nyungan}
\define@key{fams}{bll}{Siouan}
\define@key{fams}{blb}{Isolate}
\define@key{fams}{bhp}{Austronesian}
\define@key{fams}{bim}{Atlantic-Congo}
\define@key{fams}{bhg}{Nuclear Trans New Guinea}
\define@key{fams}{bin}{Atlantic-Congo}
\define@key{fams}{gup}{Gunwinyguan}
\define@key{fams}{bkd}{Austronesian}
\define@key{fams}{bjr}{Nuclear Trans New Guinea}
\define@key{fams}{bzr}{Pama-Nyungan}
\define@key{fams}{bom}{Atlantic-Congo}
\define@key{fams}{bvq}{Central Sudanic}
\define@key{fams}{bib}{Mande}
\define@key{fams}{bis}{Indo-European}
\define@key{fams}{bla}{Algic}
\define@key{fams}{kvg}{Anim}
\define@key{fams}{bni}{Atlantic-Congo}
\define@key{fams}{bbo}{Mande}
\define@key{fams}{brx}{Sino-Tibetan}
\define@key{fams}{bzf}{Ndu}
\define@key{fams}{bqc}{Mande}
\define@key{fams}{bol}{Afro-Asiatic}
\define@key{fams}{bli}{Atlantic-Congo}
\define@key{fams}{bot}{Central Sudanic}
\define@key{fams}{bpu}{Nuclear Trans New Guinea}
\define@key{fams}{lbk}{Austronesian}
\define@key{fams}{boa}{Boran}
\define@key{fams}{adi}{Sino-Tibetan}
\define@key{fams}{bor}{Bororoan}
\define@key{fams}{brn}{Chibchan}
\define@key{fams}{bos}{nan}
\define@key{fams}{boz}{Mande}
\define@key{fams}{brh}{Dravidian}
\define@key{fams}{brb}{Austroasiatic}
\define@key{fams}{bre}{Indo-European}
\define@key{fams}{bzd}{Chibchan}
\define@key{fams}{bfi}{Sign Language}
\define@key{fams}{tcs}{Indo-European}
\define@key{fams}{bkk}{Indo-European}
\define@key{fams}{bru}{Austroasiatic}
\define@key{fams}{brv}{Austroasiatic}
\define@key{fams}{bvb}{Atlantic-Congo}
\define@key{fams}{buu}{Atlantic-Congo}
\define@key{fams}{bdk}{Nakh-Daghestanian}
\define@key{fams}{bdm}{Afro-Asiatic}
\define@key{fams}{bug}{Austronesian}
\define@key{fams}{sab}{Chibchan}
\define@key{fams}{bgg}{Sino-Tibetan}
\define@key{fams}{buo}{South Bougainville}
\define@key{fams}{nmg}{Atlantic-Congo}
\define@key{fams}{bxk}{Atlantic-Congo}
\define@key{fams}{bul}{Indo-European}
\define@key{fams}{bwu}{Atlantic-Congo}
\define@key{fams}{bzq}{Austronesian}
\define@key{fams}{bum}{Atlantic-Congo}
\define@key{fams}{tkw}{Austronesian}
\define@key{fams}{bfu}{Sino-Tibetan}
\define@key{fams}{buh}{Hmong-Mien}
\define@key{fams}{bck}{Bunaban}
\define@key{fams}{bwr}{Afro-Asiatic}
\define@key{fams}{bvr}{Maningrida}
\define@key{fams}{bxm}{Mongolic-Khitan}
\define@key{fams}{bji}{Afro-Asiatic}
\define@key{fams}{mya}{Sino-Tibetan}
\define@key{fams}{mhs}{Austronesian}
\define@key{fams}{bmu}{Nuclear Trans New Guinea}
\define@key{fams}{bds}{Afro-Asiatic}
\define@key{fams}{bsk}{Isolate}
\define@key{fams}{bqp}{Mande}
\define@key{fams}{buf}{Atlantic-Congo}
\define@key{fams}{ngc}{Atlantic-Congo}
\define@key{fams}{bee}{Sino-Tibetan}
\define@key{fams}{bev}{Kru}
\define@key{fams}{cjp}{Chibchan}
\define@key{fams}{cbv}{Kakua-Nukak}
\define@key{fams}{cad}{Caddoan}
\define@key{fams}{chl}{Uto-Aztecan}
\define@key{fams}{cak}{Mayan}
\define@key{fams}{rab}{Sino-Tibetan}
\define@key{fams}{cjo}{Arawakan}
\define@key{fams}{kbh}{Isolate}
\define@key{fams}{knm}{Katukinan}
\define@key{fams}{cbu}{Isolate}
\define@key{fams}{ram}{Nuclear-Macro-Je}
\define@key{fams}{yue}{Sino-Tibetan}
\define@key{fams}{kaq}{Pano-Tacanan}
\define@key{fams}{cbc}{Tucanoan}
\define@key{fams}{car}{Cariban}
\define@key{fams}{mch}{Cariban}
\define@key{fams}{cal}{Austronesian}
\define@key{fams}{crx}{Athabaskan-Eyak-Tlingit}
\define@key{fams}{cbr}{Pano-Tacanan}
\define@key{fams}{cbs}{Pano-Tacanan}
\define@key{fams}{cat}{Indo-European}
\define@key{fams}{chc}{Siouan}
\define@key{fams}{cto}{Chocoan}
\define@key{fams}{cav}{Pano-Tacanan}
\define@key{fams}{cbi}{Barbacoan}
\define@key{fams}{cay}{Iroquoian}
\define@key{fams}{cyb}{Isolate}
\define@key{fams}{ceb}{Austronesian}
\define@key{fams}{old}{Atlantic-Congo}
\define@key{fams}{suq}{Surmic}
\define@key{fams}{cld}{Afro-Asiatic}
\define@key{fams}{cjm}{Austronesian}
\define@key{fams}{cja}{Austronesian}
\define@key{fams}{cji}{Nakh-Daghestanian}
\define@key{fams}{can}{Lower Sepik-Ramu}
\define@key{fams}{cha}{Austronesian}
\define@key{fams}{nbc}{Sino-Tibetan}
\define@key{fams}{chx}{Sino-Tibetan}
\define@key{fams}{tuu}{Athabaskan-Eyak-Tlingit}
\define@key{fams}{cya}{Otomanguean}
\define@key{fams}{cta}{Otomanguean}
\define@key{fams}{ctp}{Otomanguean}
\define@key{fams}{cdn}{Sino-Tibetan}
\define@key{fams}{cbk}{Indo-European}
\define@key{fams}{cbt}{Cahuapanan}
\define@key{fams}{che}{Nakh-Daghestanian}
\define@key{fams}{cjh}{Salishan}
\define@key{fams}{mrn}{Austronesian}
\define@key{fams}{xch}{Chimakuan}
\define@key{fams}{cdm}{Sino-Tibetan}
\define@key{fams}{chr}{Iroquoian}
\define@key{fams}{chy}{Algic}
\define@key{fams}{nya}{Atlantic-Congo}
\define@key{fams}{pei}{Otomanguean}
\define@key{fams}{cic}{Muskogean}
\define@key{fams}{cob}{Mayan}
\define@key{fams}{cid}{Isolate}
\define@key{fams}{cbg}{Chibchan}
\define@key{fams}{mrh}{Sino-Tibetan}
\define@key{fams}{csy}{Sino-Tibetan}
\define@key{fams}{ctd}{Sino-Tibetan}
\define@key{fams}{cco}{Otomanguean}
\define@key{fams}{cle}{Otomanguean}
\define@key{fams}{cpa}{Otomanguean}
\define@key{fams}{chq}{Otomanguean}
\define@key{fams}{cuc}{Otomanguean}
\define@key{fams}{cso}{Otomanguean}
\define@key{fams}{cnt}{Otomanguean}
\define@key{fams}{csl}{Sign Language}
\define@key{fams}{chh}{Chinookan}
\define@key{fams}{wac}{Chinookan}
\define@key{fams}{cap}{Uru-Chipaya}
\define@key{fams}{chp}{Athabaskan-Eyak-Tlingit}
\define@key{fams}{cax}{Chiquitano}
\define@key{fams}{gui}{Tupian}
\define@key{fams}{ctm}{Isolate}
\define@key{fams}{coz}{Otomanguean}
\define@key{fams}{cho}{Muskogean}
\define@key{fams}{ctu}{Mayan}
\define@key{fams}{cht}{Hibito-Cholon}
\define@key{fams}{chd}{Tequistlatecan}
\define@key{fams}{clo}{Tequistlatecan}
\define@key{fams}{chf}{Mayan}
\define@key{fams}{caa}{Mayan}
\define@key{fams}{crw}{Austroasiatic}
\define@key{fams}{cje}{Austronesian}
\define@key{fams}{cjv}{Nuclear Trans New Guinea}
\define@key{fams}{cac}{Mayan}
\define@key{fams}{ckt}{Chukotko-Kamchatkan}
\define@key{fams}{clw}{Turkic}
\define@key{fams}{boi}{Chumashan}
\define@key{fams}{inz}{Chumashan}
\define@key{fams}{ncu}{Atlantic-Congo}
\define@key{fams}{chk}{Austronesian}
\define@key{fams}{chv}{Turkic}
\define@key{fams}{cao}{Pano-Tacanan}
\define@key{fams}{lua}{Atlantic-Congo}
\define@key{fams}{clm}{Salishan}
\define@key{fams}{xcw}{Isolate}
\define@key{fams}{cod}{Tupian}
\define@key{fams}{coc}{Cochimi-Yuman}
\define@key{fams}{crd}{Salishan}
\define@key{fams}{con}{Isolate}
\define@key{fams}{kog}{Chibchan}
\define@key{fams}{col}{Salishan}
\define@key{fams}{com}{Uto-Aztecan}
\define@key{fams}{xcm}{Isolate}
\define@key{fams}{swb}{Atlantic-Congo}
\define@key{fams}{coo}{Salishan}
\define@key{fams}{csz}{Coosan}
\define@key{fams}{cop}{Afro-Asiatic}
\define@key{fams}{crn}{Uto-Aztecan}
\define@key{fams}{cor}{Indo-European}
\define@key{fams}{crk}{Algic}
\define@key{fams}{csw}{Algic}
\define@key{fams}{mus}{Muskogean}
\define@key{fams}{crh}{Turkic}
\define@key{fams}{cro}{Siouan}
\define@key{fams}{cua}{Austroasiatic}
\define@key{fams}{cub}{Tucanoan}
\define@key{fams}{cui}{Guahiboan}
\define@key{fams}{cuy}{Isolate}
\define@key{fams}{cul}{Arawan}
\define@key{fams}{cup}{Uto-Aztecan}
\define@key{fams}{kpc}{Arawakan}
\define@key{fams}{ces}{Indo-European}
\define@key{fams}{cam}{Austronesian}
\define@key{fams}{kzf}{Austronesian}
\define@key{fams}{dbq}{Afro-Asiatic}
\define@key{fams}{dav}{Atlantic-Congo}
\define@key{fams}{mps}{Teberan}
\define@key{fams}{dgz}{Dagan}
\define@key{fams}{dga}{Atlantic-Congo}
\define@key{fams}{dag}{Atlantic-Congo}
\define@key{fams}{dta}{Mongolic-Khitan}
\define@key{fams}{dal}{Afro-Asiatic}
\define@key{fams}{daj}{Dajuic}
\define@key{fams}{dak}{Siouan}
\define@key{fams}{mbp}{Chibchan}
\define@key{fams}{dnj}{nan}
\define@key{fams}{daa}{Afro-Asiatic}
\define@key{fams}{dni}{Nuclear Trans New Guinea}
\define@key{fams}{dan}{Indo-European}
\define@key{fams}{dry}{Indo-European}
\define@key{fams}{dar}{Nakh-Daghestanian}
\define@key{fams}{prs}{Indo-European}
\define@key{fams}{drd}{Sino-Tibetan}
\define@key{fams}{tcc}{Nilotic}
\define@key{fams}{dai}{Atlantic-Congo}
\define@key{fams}{afn}{Ijoid}
\define@key{fams}{deg}{Atlantic-Congo}
\define@key{fams}{ing}{Athabaskan-Eyak-Tlingit}
\define@key{fams}{dny}{Arawan}
\define@key{fams}{des}{Tucanoan}
\define@key{fams}{shg}{Khoe-Kwadi}
\define@key{fams}{der}{Sino-Tibetan}
\define@key{fams}{gsg}{Sign Language}
\define@key{fams}{dsh}{Afro-Asiatic}
\define@key{fams}{dhl}{Pama-Nyungan}
\define@key{fams}{tbh}{Pama-Nyungan}
\define@key{fams}{dhr}{Pama-Nyungan}
\define@key{fams}{xgm}{Pama-Nyungan}
\define@key{fams}{dhi}{Sino-Tibetan}
\define@key{fams}{div}{Indo-European}
\define@key{fams}{dhu}{Pama-Nyungan}
\define@key{fams}{did}{Surmic}
\define@key{fams}{mhu}{Sino-Tibetan}
\define@key{fams}{dur}{Atlantic-Congo}
\define@key{fams}{dis}{Sino-Tibetan}
\define@key{fams}{dim}{South Omotic}
\define@key{fams}{diz}{Atlantic-Congo}
\define@key{fams}{din}{nan}
\define@key{fams}{dyo}{Atlantic-Congo}
\define@key{fams}{csk}{Atlantic-Congo}
\define@key{fams}{dif}{Pama-Nyungan}
\define@key{fams}{mdx}{Dizoid}
\define@key{fams}{dyy}{Pama-Nyungan}
\define@key{fams}{djr}{Pama-Nyungan}
\define@key{fams}{duj}{Pama-Nyungan}
\define@key{fams}{ddj}{Pama-Nyungan}
\define@key{fams}{dji}{Pama-Nyungan}
\define@key{fams}{jig}{Mirndi}
\define@key{fams}{kbv}{Senagi}
\define@key{fams}{kvo}{Austronesian}
\define@key{fams}{dgo}{Indo-European}
\define@key{fams}{dlg}{Turkic}
\define@key{fams}{dmk}{Indo-European}
\define@key{fams}{rmt}{Indo-European}
\define@key{fams}{kmc}{Tai-Kadai}
\define@key{fams}{doo}{Atlantic-Congo}
\define@key{fams}{dds}{Dogon}
\define@key{fams}{tds}{Lakes Plain}
\define@key{fams}{dow}{Atlantic-Congo}
\define@key{fams}{dhv}{Austronesian}
\define@key{fams}{dua}{Atlantic-Congo}
\define@key{fams}{dud}{Atlantic-Congo}
\define@key{fams}{gwd}{Afro-Asiatic}
\define@key{fams}{duu}{Sino-Tibetan}
\define@key{fams}{dma}{Atlantic-Congo}
\define@key{fams}{dgc}{Austronesian}
\define@key{fams}{dus}{Sino-Tibetan}
\define@key{fams}{vam}{Sko}
\define@key{fams}{duc}{Isolate}
\define@key{fams}{nld}{Indo-European}
\define@key{fams}{zea}{Indo-European}
\define@key{fams}{dyi}{Atlantic-Congo}
\define@key{fams}{dbl}{Pama-Nyungan}
\define@key{fams}{dyu}{Mande}
\define@key{fams}{kwa}{Naduhup}
\define@key{fams}{igb}{Atlantic-Congo}
\define@key{fams}{etr}{Bosavi}
\define@key{fams}{erk}{Austronesian}
\define@key{fams}{efi}{Atlantic-Congo}
\define@key{fams}{ega}{Atlantic-Congo}
\define@key{fams}{eip}{Nuclear Trans New Guinea}
\define@key{fams}{etu}{Atlantic-Congo}
\define@key{fams}{ekg}{Nuclear Trans New Guinea}
\define@key{fams}{eko}{Atlantic-Congo}
\define@key{fams}{mrf}{Isolate}
\define@key{fams}{ema}{Atlantic-Congo}
\define@key{fams}{emb}{Austronesian}
\define@key{fams}{cmi}{Chocoan}
\define@key{fams}{emp}{Chocoan}
\define@key{fams}{amy}{Western Daly}
\define@key{fams}{enq}{Nuclear Trans New Guinea}
\define@key{fams}{enn}{Atlantic-Congo}
\define@key{fams}{eno}{Austronesian}
\define@key{fams}{eng}{Indo-European}
\define@key{fams}{gey}{Atlantic-Congo}
\define@key{fams}{sja}{Chocoan}
\define@key{fams}{erg}{Austronesian}
\define@key{fams}{ese}{Pano-Tacanan}
\define@key{fams}{esq}{Isolate}
\define@key{fams}{ekk}{Uralic}
\define@key{fams}{ets}{Atlantic-Congo}
\define@key{fams}{eve}{Tungusic}
\define@key{fams}{ewe}{Atlantic-Congo}
\define@key{fams}{ewo}{Atlantic-Congo}
\define@key{fams}{eya}{Athabaskan-Eyak-Tlingit}
\define@key{fams}{fao}{Indo-European}
\define@key{fams}{faa}{Isolate}
\define@key{fams}{fmp}{Atlantic-Congo}
\define@key{fams}{fij}{Austronesian}
\define@key{fams}{fin}{Uralic}
\define@key{fams}{fse}{Sign Language}
\define@key{fams}{foi}{East Kutubu}
\define@key{fams}{ppo}{Teberan}
\define@key{fams}{fon}{Atlantic-Congo}
\define@key{fams}{frd}{Austronesian}
\define@key{fams}{for}{Nuclear Trans New Guinea}
\define@key{fams}{sac}{Algic}
\define@key{fams}{fra}{Indo-European}
\define@key{fams}{fry}{Indo-European}
\define@key{fams}{frs}{Indo-European}
\define@key{fams}{frr}{Indo-European}
\define@key{fams}{fuh}{Atlantic-Congo}
\define@key{fams}{fuf}{Atlantic-Congo}
\define@key{fams}{fub}{Atlantic-Congo}
\define@key{fams}{ffm}{Atlantic-Congo}
\define@key{fams}{fuv}{Atlantic-Congo}
\define@key{fams}{fun}{Isolate}
\define@key{fams}{fvr}{Furan}
\define@key{fams}{fud}{Austronesian}
\define@key{fams}{fut}{Austronesian}
\define@key{fams}{cdo}{Sino-Tibetan}
\define@key{fams}{pym}{Atlantic-Congo}
\define@key{fams}{gqa}{Afro-Asiatic}
\define@key{fams}{gbu}{Isolate}
\define@key{fams}{dhg}{Pama-Nyungan}
\define@key{fams}{gdb}{Dravidian}
\define@key{fams}{ged}{Atlantic-Congo}
\define@key{fams}{gaj}{Nuclear Trans New Guinea}
\define@key{fams}{gla}{Indo-European}
\define@key{fams}{gag}{Turkic}
\define@key{fams}{gah}{Nuclear Trans New Guinea}
\define@key{fams}{gbi}{North Halmahera}
\define@key{fams}{glg}{Indo-European}
\define@key{fams}{adl}{Sino-Tibetan}
\define@key{fams}{kld}{Pama-Nyungan}
\define@key{fams}{gmv}{Ta-Ne-Omotic}
\define@key{fams}{pwg}{Austronesian}
\define@key{fams}{grt}{Sino-Tibetan}
\define@key{fams}{wrk}{Garrwan}
\define@key{fams}{gyb}{Nuclear Trans New Guinea}
\define@key{fams}{cab}{Arawakan}
\define@key{fams}{gvo}{Tupian}
\define@key{fams}{gay}{Austronesian}
\define@key{fams}{gya}{Atlantic-Congo}
\define@key{fams}{gso}{Atlantic-Congo}
\define@key{fams}{gbp}{Atlantic-Congo}
\define@key{fams}{nlg}{Austronesian}
\define@key{fams}{gqu}{Tai-Kadai}
\define@key{fams}{kat}{Kartvelian}
\define@key{fams}{deu}{Indo-European}
\define@key{fams}{bar}{Indo-European}
\define@key{fams}{ksh}{Indo-European}
\define@key{fams}{wep}{Indo-European}
\define@key{fams}{aaa}{Atlantic-Congo}
\define@key{fams}{ghl}{Nubian}
\define@key{fams}{gih}{Pama-Nyungan}
\define@key{fams}{gid}{Afro-Asiatic}
\define@key{fams}{glk}{Indo-European}
\define@key{fams}{bcq}{Ta-Ne-Omotic}
\define@key{fams}{git}{Tsimshian}
\define@key{fams}{gis}{Afro-Asiatic}
\define@key{fams}{guc}{Arawakan}
\define@key{fams}{god}{Kru}
\define@key{fams}{gdo}{Nakh-Daghestanian}
\define@key{fams}{ank}{Afro-Asiatic}
\define@key{fams}{ggw}{Suki-Gogodala}
\define@key{fams}{gju}{Indo-European}
\define@key{fams}{gkn}{Atlantic-Congo}
\define@key{fams}{gol}{Atlantic-Congo}
\define@key{fams}{gvf}{Nuclear Trans New Guinea}
\define@key{fams}{gno}{Dravidian}
\define@key{fams}{gni}{Bunaban}
\define@key{fams}{gor}{Austronesian}
\define@key{fams}{gow}{Afro-Asiatic}
\define@key{fams}{grj}{Kru}
\define@key{fams}{ell}{Indo-European}
\define@key{fams}{gss}{Sign Language}
\define@key{fams}{kal}{Eskimo-Aleut}
\define@key{fams}{guh}{Guahiboan}
\define@key{fams}{gub}{nan}
\define@key{fams}{gum}{Barbacoan}
\define@key{fams}{gva}{Lengua-Mascoy}
\define@key{fams}{gvc}{Tucanoan}
\define@key{fams}{gug}{Tupian}
\define@key{fams}{var}{Uto-Aztecan}
\define@key{fams}{gta}{Isolate}
\define@key{fams}{guo}{Guahiboan}
\define@key{fams}{gde}{Afro-Asiatic}
\define@key{fams}{gdf}{Afro-Asiatic}
\define@key{fams}{ktd}{Pama-Nyungan}
\define@key{fams}{ggd}{Pama-Nyungan}
\define@key{fams}{ghs}{Nuclear Trans New Guinea}
\define@key{fams}{gcr}{Indo-European}
\define@key{fams}{pov}{Indo-European}
\define@key{fams}{guj}{Indo-European}
\define@key{fams}{kcm}{Central Sudanic}
\define@key{fams}{glj}{Atlantic-Congo}
\define@key{fams}{gnn}{Pama-Nyungan}
\define@key{fams}{gvs}{Austronesian}
\define@key{fams}{kgs}{Pama-Nyungan}
\define@key{fams}{guk}{Gumuz}
\define@key{fams}{wlg}{Gunwinyguan}
\define@key{fams}{guw}{Atlantic-Congo}
\define@key{fams}{gww}{Worrorran}
\define@key{fams}{yas}{Atlantic-Congo}
\define@key{fams}{gyy}{Pama-Nyungan}
\define@key{fams}{guf}{Pama-Nyungan}
\define@key{fams}{gnr}{Pama-Nyungan}
\define@key{fams}{gur}{Atlantic-Congo}
\define@key{fams}{gue}{Pama-Nyungan}
\define@key{fams}{gux}{Atlantic-Congo}
\define@key{fams}{goa}{Mande}
\define@key{fams}{gge}{Maningrida}
\define@key{fams}{guz}{Atlantic-Congo}
\define@key{fams}{gbj}{Austroasiatic}
\define@key{fams}{kky}{Pama-Nyungan}
\define@key{fams}{gbr}{Atlantic-Congo}
\define@key{fams}{kcg}{Atlantic-Congo}
\define@key{fams}{gaa}{Atlantic-Congo}
\define@key{fams}{pue}{Isolate}
\define@key{fams}{hts}{Isolate}
\define@key{fams}{hai}{nan}
\define@key{fams}{hdn}{Haida}
\define@key{fams}{has}{Wakashan}
\define@key{fams}{hat}{Indo-European}
\define@key{fams}{hak}{Sino-Tibetan}
\define@key{fams}{hal}{Austroasiatic}
\define@key{fams}{hlb}{Indo-European}
\define@key{fams}{hla}{Austronesian}
\define@key{fams}{amf}{South Omotic}
\define@key{fams}{hmt}{Angan}
\define@key{fams}{wos}{Ndu}
\define@key{fams}{hni}{Sino-Tibetan}
\define@key{fams}{hnn}{Austronesian}
\define@key{fams}{har}{Afro-Asiatic}
\define@key{fams}{hss}{Afro-Asiatic}
\define@key{fams}{tmd}{Piawi}
\define@key{fams}{had}{Hatam-Mansim}
\define@key{fams}{hau}{Afro-Asiatic}
\define@key{fams}{haw}{Austronesian}
\define@key{fams}{hwc}{Indo-European}
\define@key{fams}{hac}{Indo-European}
\define@key{fams}{hay}{Atlantic-Congo}
\define@key{fams}{vay}{Sino-Tibetan}
\define@key{fams}{xed}{Afro-Asiatic}
\define@key{fams}{heb}{Afro-Asiatic}
\define@key{fams}{heh}{Atlantic-Congo}
\define@key{fams}{hei}{Wakashan}
\define@key{fams}{hem}{Atlantic-Congo}
\define@key{fams}{her}{Atlantic-Congo}
\define@key{fams}{hid}{Siouan}
\define@key{fams}{hil}{Austronesian}
\define@key{fams}{hin}{Indo-European}
\define@key{fams}{gin}{Nakh-Daghestanian}
\define@key{fams}{hix}{Cariban}
\define@key{fams}{lic}{Tai-Kadai}
\define@key{fams}{hmr}{Sino-Tibetan}
\define@key{fams}{mww}{Hmong-Mien}
\define@key{fams}{hnj}{Hmong-Mien}
\define@key{fams}{hoc}{Austroasiatic}
\define@key{fams}{hoa}{Austronesian}
\define@key{fams}{hoo}{Atlantic-Congo}
\define@key{fams}{hks}{Sign Language}
\define@key{fams}{hop}{Uto-Aztecan}
\define@key{fams}{hre}{Austroasiatic}
\define@key{fams}{ygr}{Nuclear Trans New Guinea}
\define@key{fams}{hub}{Chicham}
\define@key{fams}{hus}{Mayan}
\define@key{fams}{huv}{Huavean}
\define@key{fams}{hch}{Uto-Aztecan}
\define@key{fams}{hto}{Huitotoan}
\define@key{fams}{hux}{Huitotoan}
\define@key{fams}{huu}{Huitotoan}
\define@key{fams}{hke}{Atlantic-Congo}
\define@key{fams}{hun}{Uralic}
\define@key{fams}{huz}{Nakh-Daghestanian}
\define@key{fams}{jup}{Naduhup}
\define@key{fams}{hup}{Athabaskan-Eyak-Tlingit}
\define@key{fams}{csh}{Sino-Tibetan}
\define@key{fams}{ksi}{Sko}
\define@key{fams}{iai}{Austronesian}
\define@key{fams}{ian}{Ndu}
\define@key{fams}{tmu}{Lakes Plain}
\define@key{fams}{iba}{Austronesian}
\define@key{fams}{ibg}{Austronesian}
\define@key{fams}{ibb}{Atlantic-Congo}
\define@key{fams}{isl}{Indo-European}
\define@key{fams}{icl}{Sign Language}
\define@key{fams}{idu}{Atlantic-Congo}
\define@key{fams}{clk}{Sino-Tibetan}
\define@key{fams}{viv}{Austronesian}
\define@key{fams}{mxe}{Austronesian}
\define@key{fams}{ifb}{Austronesian}
\define@key{fams}{ifm}{Atlantic-Congo}
\define@key{fams}{ibo}{Atlantic-Congo}
\define@key{fams}{ige}{Atlantic-Congo}
\define@key{fams}{ign}{Arawakan}
\define@key{fams}{ihp}{West Bomberai}
\define@key{fams}{ijc}{Ijoid}
\define@key{fams}{ikx}{Kuliak}
\define@key{fams}{arh}{Chibchan}
\define@key{fams}{ilb}{Atlantic-Congo}
\define@key{fams}{mia}{Algic}
\define@key{fams}{ilo}{Austronesian}
\define@key{fams}{imn}{Border}
\define@key{fams}{szp}{Inanwatan}
\define@key{fams}{ins}{Sign Language}
\define@key{fams}{pks}{Sign Language}
\define@key{fams}{ind}{Austronesian}
\define@key{fams}{pmy}{Austronesian}
\define@key{fams}{inb}{Quechuan}
\define@key{fams}{tbi}{Eastern Jebel}
\define@key{fams}{inh}{Nakh-Daghestanian}
\define@key{fams}{ynd}{Pama-Nyungan}
\define@key{fams}{ils}{Sign Language}
\define@key{fams}{ike}{Eskimo-Aleut}
\define@key{fams}{iqu}{Zaparoan}
\define@key{fams}{irn}{Isolate}
\define@key{fams}{irk}{Afro-Asiatic}
\define@key{fams}{irh}{Austronesian}
\define@key{fams}{gle}{Indo-European}
\define@key{fams}{isg}{Sign Language}
\define@key{fams}{its}{Atlantic-Congo}
\define@key{fams}{isk}{Indo-European}
\define@key{fams}{srl}{Greater Kwerba}
\define@key{fams}{isd}{Austronesian}
\define@key{fams}{iso}{Atlantic-Congo}
\define@key{fams}{isr}{Sign Language}
\define@key{fams}{ita}{Indo-European}
\define@key{fams}{egl}{Indo-European}
\define@key{fams}{lij}{Indo-European}
\define@key{fams}{nap}{Indo-European}
\define@key{fams}{pms}{Indo-European}
\define@key{fams}{itv}{Austronesian}
\define@key{fams}{itl}{Chukotko-Kamchatkan}
\define@key{fams}{ito}{Isolate}
\define@key{fams}{itz}{Mayan}
\define@key{fams}{ivb}{Austronesian}
\define@key{fams}{ibd}{Iwaidjan Proper}
\define@key{fams}{iwm}{Sepik}
\define@key{fams}{yom}{Atlantic-Congo}
\define@key{fams}{ixc}{Otomanguean}
\define@key{fams}{ixl}{Mayan}
\define@key{fams}{izr}{Atlantic-Congo}
\define@key{fams}{izh}{Uralic}
\define@key{fams}{izz}{nan}
\define@key{fams}{esi}{Eskimo-Aleut}
\define@key{fams}{jbt}{Nuclear-Macro-Je}
\define@key{fams}{jae}{Austronesian}
\define@key{fams}{jda}{Sino-Tibetan}
\define@key{fams}{jhi}{Austroasiatic}
\define@key{fams}{jac}{Mayan}
\define@key{fams}{jam}{Indo-European}
\define@key{fams}{djd}{Mirndi}
\define@key{fams}{djm}{Dogon}
\define@key{fams}{jpn}{Japonic}
\define@key{fams}{jru}{Cariban}
\define@key{fams}{jqr}{Aymaran}
\define@key{fams}{anq}{Jarawa-Onge}
\define@key{fams}{jav}{Austronesian}
\define@key{fams}{jeb}{Cahuapanan}
\define@key{fams}{jeh}{Austroasiatic}
\define@key{fams}{jek}{Mande}
\define@key{fams}{tow}{Kiowa-Tanoan}
\define@key{fams}{jya}{nan}
\define@key{fams}{shv}{Afro-Asiatic}
\define@key{fams}{kac}{Sino-Tibetan}
\define@key{fams}{jiu}{Sino-Tibetan}
\define@key{fams}{jiv}{Chicham}
\define@key{fams}{rgk}{Sino-Tibetan}
\define@key{fams}{tlo}{Narrow Talodi}
\define@key{fams}{jun}{Austroasiatic}
\define@key{fams}{nst}{Sino-Tibetan}
\define@key{fams}{jbu}{Atlantic-Congo}
\define@key{fams}{bex}{Central Sudanic}
\define@key{fams}{juc}{Tungusic}
\define@key{fams}{jur}{Tupian}
\define@key{fams}{ktz}{Kxa}
\define@key{fams}{jua}{Tupian}
\define@key{fams}{kek}{Mayan}
\define@key{fams}{kbd}{Abkhaz-Adyge}
\define@key{fams}{xkp}{Indo-European}
\define@key{fams}{kbp}{Atlantic-Congo}
\define@key{fams}{nbu}{Sino-Tibetan}
\define@key{fams}{kab}{Afro-Asiatic}
\define@key{fams}{xac}{Sino-Tibetan}
\define@key{fams}{kzj}{nan}
\define@key{fams}{kbc}{Guaicuruan}
\define@key{fams}{kdm}{Atlantic-Congo}
\define@key{fams}{kki}{Atlantic-Congo}
\define@key{fams}{kct}{Lower Sepik-Ramu}
\define@key{fams}{lew}{Austronesian}
\define@key{fams}{kgp}{Nuclear-Macro-Je}
\define@key{fams}{kxa}{Austronesian}
\define@key{fams}{kgk}{Tupian}
\define@key{fams}{tbd}{Isolate}
\define@key{fams}{mwp}{Pama-Nyungan}
\define@key{fams}{kmh}{Nuclear Trans New Guinea}
\define@key{fams}{gwc}{Indo-European}
\define@key{fams}{kck}{Atlantic-Congo}
\define@key{fams}{kyl}{Kalapuyan}
\define@key{fams}{kls}{Indo-European}
\define@key{fams}{fla}{Salishan}
\define@key{fams}{ktg}{Pama-Nyungan}
\define@key{fams}{bco}{Bosavi}
\define@key{fams}{kay}{Tupian}
\define@key{fams}{kbq}{Nuclear Trans New Guinea}
\define@key{fams}{kms}{Nuclear Torricelli}
\define@key{fams}{xas}{Uralic}
\define@key{fams}{kam}{Atlantic-Congo}
\define@key{fams}{xbr}{Austronesian}
\define@key{fams}{kbx}{Keram}
\define@key{fams}{kcu}{Atlantic-Congo}
\define@key{fams}{kgq}{Nuclear Trans New Guinea}
\define@key{fams}{xmu}{Eastern Daly}
\define@key{fams}{ogo}{Atlantic-Congo}
\define@key{fams}{kna}{Afro-Asiatic}
\define@key{fams}{xns}{Sino-Tibetan}
\define@key{fams}{kbl}{Saharan}
\define@key{fams}{ikt}{Eskimo-Aleut}
\define@key{fams}{kjb}{Mayan}
\define@key{fams}{knj}{Mayan}
\define@key{fams}{kne}{Austronesian}
\define@key{fams}{kan}{Dravidian}
\define@key{fams}{kxo}{Isolate}
\define@key{fams}{khd}{Yam}
\define@key{fams}{kcd}{Yam}
\define@key{fams}{knc}{Saharan}
\define@key{fams}{kny}{Atlantic-Congo}
\define@key{fams}{pam}{Austronesian}
\define@key{fams}{kpg}{Austronesian}
\define@key{fams}{kah}{Central Sudanic}
\define@key{fams}{leu}{Austronesian}
\define@key{fams}{krc}{Turkic}
\define@key{fams}{gbd}{Pama-Nyungan}
\define@key{fams}{kdr}{Turkic}
\define@key{fams}{kpj}{Nuclear-Macro-Je}
\define@key{fams}{kaa}{Turkic}
\define@key{fams}{zkk}{Isolate}
\define@key{fams}{kyj}{Austronesian}
\define@key{fams}{kpt}{Nakh-Daghestanian}
\define@key{fams}{krl}{Uralic}
\define@key{fams}{bwe}{Sino-Tibetan}
\define@key{fams}{kjp}{Sino-Tibetan}
\define@key{fams}{ksw}{Sino-Tibetan}
\define@key{fams}{vka}{Pama-Nyungan}
\define@key{fams}{kdj}{Nilotic}
\define@key{fams}{ktn}{Tupian}
\define@key{fams}{yuj}{Pauwasi}
\define@key{fams}{kyh}{Isolate}
\define@key{fams}{arr}{Tupian}
\define@key{fams}{xsm}{Atlantic-Congo}
\define@key{fams}{kju}{Pomoan}
\define@key{fams}{kas}{Indo-European}
\define@key{fams}{csb}{Indo-European}
\define@key{fams}{cog}{Austroasiatic}
\define@key{fams}{bqy}{Sign Language}
\define@key{fams}{xtc}{Kadugli-Krongo}
\define@key{fams}{bsh}{Indo-European}
\define@key{fams}{kts}{Nuclear Trans New Guinea}
\define@key{fams}{kcr}{Katla-Tima}
\define@key{fams}{ktw}{Athabaskan-Eyak-Tlingit}
\define@key{fams}{pss}{Austronesian}
\define@key{fams}{bpp}{Kaure-Kosare}
\define@key{fams}{zku}{Pama-Nyungan}
\define@key{fams}{xaw}{Uto-Aztecan}
\define@key{fams}{kyz}{Tupian}
\define@key{fams}{eky}{Sino-Tibetan}
\define@key{fams}{kys}{Austronesian}
\define@key{fams}{txu}{Nuclear-Macro-Je}
\define@key{fams}{gyd}{Tangkic}
\define@key{fams}{gbb}{Pama-Nyungan}
\define@key{fams}{kaz}{Turkic}
\define@key{fams}{ksx}{Austronesian}
\define@key{fams}{kbr}{Ta-Ne-Omotic}
\define@key{fams}{kei}{Austronesian}
\define@key{fams}{kcl}{Austronesian}
\define@key{fams}{kzi}{Austronesian}
\define@key{fams}{sbc}{Austronesian}
\define@key{fams}{ahg}{Afro-Asiatic}
\define@key{fams}{kmt}{Nimboranic}
\define@key{fams}{kyq}{Central Sudanic}
\define@key{fams}{keu}{Atlantic-Congo}
\define@key{fams}{xki}{Sign Language}
\define@key{fams}{ken}{Atlantic-Congo}
\define@key{fams}{xxk}{Austronesian}
\define@key{fams}{ker}{Afro-Asiatic}
\define@key{fams}{krk}{Chukotko-Kamchatkan}
\define@key{fams}{kee}{Keresan}
\define@key{fams}{ket}{Yeniseian}
\define@key{fams}{xdy}{Austronesian}
\define@key{fams}{kcv}{Atlantic-Congo}
\define@key{fams}{xte}{Nuclear Trans New Guinea}
\define@key{fams}{kew}{Nuclear Trans New Guinea}
\define@key{fams}{kjh}{Turkic}
\define@key{fams}{klj}{Turkic}
\define@key{fams}{klr}{Sino-Tibetan}
\define@key{fams}{khk}{Mongolic-Khitan}
\define@key{fams}{kjl}{Sino-Tibetan}
\define@key{fams}{khg}{Sino-Tibetan}
\define@key{fams}{kca}{Uralic}
\define@key{fams}{khr}{Austroasiatic}
\define@key{fams}{kha}{Austroasiatic}
\define@key{fams}{kjj}{Nakh-Daghestanian}
\define@key{fams}{khm}{Austroasiatic}
\define@key{fams}{kjg}{Austroasiatic}
\define@key{fams}{khw}{Indo-European}
\define@key{fams}{cnk}{Sino-Tibetan}
\define@key{fams}{khv}{Nakh-Daghestanian}
\define@key{fams}{kkh}{Tai-Kadai}
\define@key{fams}{kic}{Algic}
\define@key{fams}{kik}{Atlantic-Congo}
\define@key{fams}{hbb}{Afro-Asiatic}
\define@key{fams}{kij}{Austronesian}
\define@key{fams}{klb}{Cochimi-Yuman}
\define@key{fams}{lub}{Atlantic-Congo}
\define@key{fams}{kig}{Kolopom}
\define@key{fams}{zga}{Atlantic-Congo}
\define@key{fams}{kfk}{Sino-Tibetan}
\define@key{fams}{kin}{Atlantic-Congo}
\define@key{fams}{kio}{Kiowa-Tanoan}
\define@key{fams}{kzw}{Unclassifiable}
\define@key{fams}{geb}{Lower Sepik-Ramu}
\define@key{fams}{kir}{Turkic}
\define@key{fams}{gil}{Austronesian}
\define@key{fams}{kiy}{Lakes Plain}
\define@key{fams}{cme}{Atlantic-Congo}
\define@key{fams}{kje}{Austronesian}
\define@key{fams}{kss}{Atlantic-Congo}
\define@key{fams}{gia}{Jarrakan}
\define@key{fams}{kii}{Caddoan}
\define@key{fams}{ktu}{Atlantic-Congo}
\define@key{fams}{kjd}{Kiwaian}
\define@key{fams}{kla}{Isolate}
\define@key{fams}{klu}{Kru}
\define@key{fams}{yak}{Sahaptian}
\define@key{fams}{kst}{Atlantic-Congo}
\define@key{fams}{cku}{Muskogean}
\define@key{fams}{kpw}{Nuclear Trans New Guinea}
\define@key{fams}{kfa}{Dravidian}
\define@key{fams}{xwg}{Surmic}
\define@key{fams}{xuo}{Atlantic-Congo}
\define@key{fams}{bcs}{Atlantic-Congo}
\define@key{fams}{kpx}{Koiarian}
\define@key{fams}{kbk}{Koiarian}
\define@key{fams}{kqi}{Koiarian}
\define@key{fams}{trp}{Sino-Tibetan}
\define@key{fams}{kex}{nan}
\define@key{fams}{kkk}{Austronesian}
\define@key{fams}{kvv}{Austronesian}
\define@key{fams}{kfb}{Dravidian}
\define@key{fams}{kvw}{Timor-Alor-Pantar}
\define@key{fams}{shm}{Indo-European}
\define@key{fams}{bkm}{Atlantic-Congo}
\define@key{fams}{xbi}{Nuclear Torricelli}
\define@key{fams}{kge}{Austronesian}
\define@key{fams}{koi}{Uralic}
\define@key{fams}{xom}{Koman}
\define@key{fams}{kfc}{Dravidian}
\define@key{fams}{kng}{Atlantic-Congo}
\define@key{fams}{kjc}{Austronesian}
\define@key{fams}{knn}{Indo-European}
\define@key{fams}{xon}{Atlantic-Congo}
\define@key{fams}{mjd}{Maiduan}
\define@key{fams}{kma}{Atlantic-Congo}
\define@key{fams}{kyx}{North Bougainville}
\define@key{fams}{cou}{Atlantic-Congo}
\define@key{fams}{kqy}{Ta-Ne-Omotic}
\define@key{fams}{kpr}{Nuclear Trans New Guinea}
\define@key{fams}{kqz}{Khoe-Kwadi}
\define@key{fams}{knk}{Mande}
\define@key{fams}{kor}{Koreanic}
\define@key{fams}{coe}{Tucanoan}
\define@key{fams}{kfq}{Austroasiatic}
\define@key{fams}{kfz}{Atlantic-Congo}
\define@key{fams}{khe}{Nuclear Trans New Guinea}
\define@key{fams}{kpy}{Chukotko-Kamchatkan}
\define@key{fams}{kia}{Atlantic-Congo}
\define@key{fams}{kos}{Austronesian}
\define@key{fams}{kfe}{Dravidian}
\define@key{fams}{aal}{Afro-Asiatic}
\define@key{fams}{kff}{Dravidian}
\define@key{fams}{khq}{Songhay}
\define@key{fams}{ses}{Songhay}
\define@key{fams}{koy}{Athabaskan-Eyak-Tlingit}
\define@key{fams}{kpk}{Atlantic-Congo}
\define@key{fams}{xpe}{Mande}
\define@key{fams}{kpo}{Atlantic-Congo}
\define@key{fams}{xra}{nan}
\define@key{fams}{kqq}{Nuclear-Macro-Je}
\define@key{fams}{krs}{Kresh-Aja}
\define@key{fams}{rop}{Indo-European}
\define@key{fams}{kgo}{Kadugli-Krongo}
\define@key{fams}{jct}{Turkic}
\define@key{fams}{kry}{Nakh-Daghestanian}
\define@key{fams}{puo}{Austroasiatic}
\define@key{fams}{sdm}{Austronesian}
\define@key{fams}{uwa}{Pama-Nyungan}
\define@key{fams}{kxu}{Dravidian}
\define@key{fams}{kvd}{Timor-Alor-Pantar}
\define@key{fams}{kui}{Cariban}
\define@key{fams}{gvn}{Pama-Nyungan}
\define@key{fams}{mbt}{Austronesian}
\define@key{fams}{dwr}{Ta-Ne-Omotic}
\define@key{fams}{kle}{Sino-Tibetan}
\define@key{fams}{kue}{Nuclear Trans New Guinea}
\define@key{fams}{kfy}{Indo-European}
\define@key{fams}{kum}{Turkic}
\define@key{fams}{kvn}{Chibchan}
\define@key{fams}{kun}{Isolate}
\define@key{fams}{kup}{Goilalan}
\define@key{fams}{kjn}{Pama-Nyungan}
\define@key{fams}{cmn}{Sino-Tibetan}
\define@key{fams}{kto}{Isolate}
\define@key{fams}{ckb}{Indo-European}
\define@key{fams}{kmr}{Indo-European}
\define@key{fams}{kru}{Dravidian}
\define@key{fams}{kgg}{Isolate}
\define@key{fams}{vkt}{Austronesian}
\define@key{fams}{gwi}{Athabaskan-Eyak-Tlingit}
\define@key{fams}{kut}{Isolate}
\define@key{fams}{thd}{Pama-Nyungan}
\define@key{fams}{kuy}{Pama-Nyungan}
\define@key{fams}{kxv}{Dravidian}
\define@key{fams}{kwd}{Austronesian}
\define@key{fams}{kwk}{Wakashan}
\define@key{fams}{tnk}{Austronesian}
\define@key{fams}{ksq}{Afro-Asiatic}
\define@key{fams}{kwn}{Atlantic-Congo}
\define@key{fams}{xwa}{Isolate}
\define@key{fams}{kwe}{Greater Kwerba}
\define@key{fams}{kmo}{Sepik}
\define@key{fams}{kwo}{Kwomtari-Nai}
\define@key{fams}{xuu}{Khoe-Kwadi}
\define@key{fams}{kyc}{Nuclear Trans New Guinea}
\define@key{fams}{kgy}{Sino-Tibetan}
\define@key{fams}{nuk}{Wakashan}
\define@key{fams}{kmg}{Nuclear Trans New Guinea}
\define@key{fams}{gdm}{Isolate}
\define@key{fams}{lbu}{Austronesian}
\define@key{fams}{lac}{Mayan}
\define@key{fams}{lbt}{Tai-Kadai}
\define@key{fams}{lbj}{Sino-Tibetan}
\define@key{fams}{lld}{Indo-European}
\define@key{fams}{lad}{Indo-European}
\define@key{fams}{laf}{Isolate}
\define@key{fams}{kot}{Afro-Asiatic}
\define@key{fams}{lha}{Tai-Kadai}
\define@key{fams}{lhu}{Sino-Tibetan}
\define@key{fams}{cnh}{Sino-Tibetan}
\define@key{fams}{lbe}{Nakh-Daghestanian}
\define@key{fams}{lkt}{Siouan}
\define@key{fams}{lbc}{Tai-Kadai}
\define@key{fams}{ywt}{Sino-Tibetan}
\define@key{fams}{slp}{Austronesian}
\define@key{fams}{hia}{Afro-Asiatic}
\define@key{fams}{lmn}{Indo-European}
\define@key{fams}{lam}{Atlantic-Congo}
\define@key{fams}{lmu}{Austronesian}
\define@key{fams}{lns}{Atlantic-Congo}
\define@key{fams}{ljp}{Austronesian}
\define@key{fams}{lby}{Pama-Nyungan}
\define@key{fams}{lme}{Afro-Asiatic}
\define@key{fams}{lag}{Atlantic-Congo}
\define@key{fams}{laj}{Nilotic}
\define@key{fams}{fsl}{Sign Language}
\define@key{fams}{fcs}{Sign Language}
\define@key{fams}{lao}{Tai-Kadai}
\define@key{fams}{lrg}{Isolate}
\define@key{fams}{lbz}{Tangkic}
\define@key{fams}{alo}{Austronesian}
\define@key{fams}{lav}{Indo-European}
\define@key{fams}{llu}{Austronesian}
\define@key{fams}{law}{Austronesian}
\define@key{fams}{lvk}{Isolate}
\define@key{fams}{lzz}{Kartvelian}
\define@key{fams}{agh}{Atlantic-Congo}
\define@key{fams}{lea}{Atlantic-Congo}
\define@key{fams}{agb}{Atlantic-Congo}
\define@key{fams}{lec}{Isolate}
\define@key{fams}{lln}{Afro-Asiatic}
\define@key{fams}{lef}{Atlantic-Congo}
\define@key{fams}{tnl}{Austronesian}
\define@key{fams}{led}{Central Sudanic}
\define@key{fams}{enx}{Lengua-Mascoy}
\define@key{fams}{aed}{Sign Language}
\define@key{fams}{ssp}{Sign Language}
\define@key{fams}{lep}{Sino-Tibetan}
\define@key{fams}{les}{Central Sudanic}
\define@key{fams}{lti}{Austronesian}
\define@key{fams}{lww}{Austronesian}
\define@key{fams}{lez}{Nakh-Daghestanian}
\define@key{fams}{lhm}{Sino-Tibetan}
\define@key{fams}{lil}{Salishan}
\define@key{fams}{lif}{Sino-Tibetan}
\define@key{fams}{lmc}{Limilngan-Wulna}
\define@key{fams}{liy}{Atlantic-Congo}
\define@key{fams}{lin}{Atlantic-Congo}
\define@key{fams}{ise}{Sign Language}
\define@key{fams}{lnj}{Pama-Nyungan}
\define@key{fams}{lis}{Sino-Tibetan}
\define@key{fams}{lit}{Indo-European}
\define@key{fams}{liv}{Uralic}
\define@key{fams}{lob}{Atlantic-Congo}
\define@key{fams}{log}{Central Sudanic}
\define@key{fams}{lok}{Mande}
\define@key{fams}{arw}{Arawakan}
\define@key{fams}{lom}{Mande}
\define@key{fams}{bdu}{Atlantic-Congo}
\define@key{fams}{lgu}{Austronesian}
\define@key{fams}{los}{Austronesian}
\define@key{fams}{crc}{Austronesian}
\define@key{fams}{njh}{Sino-Tibetan}
\define@key{fams}{loj}{Austronesian}
\define@key{fams}{lbo}{Austroasiatic}
\define@key{fams}{nds}{Indo-European}
\define@key{fams}{loz}{Atlantic-Congo}
\define@key{fams}{nie}{Atlantic-Congo}
\define@key{fams}{ojv}{Austronesian}
\define@key{fams}{lch}{Atlantic-Congo}
\define@key{fams}{lug}{Atlantic-Congo}
\define@key{fams}{lgg}{Central Sudanic}
\define@key{fams}{jos}{Sign Language}
\define@key{fams}{lui}{Uto-Aztecan}
\define@key{fams}{ule}{Isolate}
\define@key{fams}{str}{Salishan}
\define@key{fams}{lnd}{Austronesian}
\define@key{fams}{lun}{Atlantic-Congo}
\define@key{fams}{luo}{Nilotic}
\define@key{fams}{lrc}{Indo-European}
\define@key{fams}{lut}{Salishan}
\define@key{fams}{khl}{Austronesian}
\define@key{fams}{lue}{Atlantic-Congo}
\define@key{fams}{lwo}{Nilotic}
\define@key{fams}{ltz}{Indo-European}
\define@key{fams}{luy}{nan}
\define@key{fams}{lee}{Atlantic-Congo}
\define@key{fams}{psr}{Sign Language}
\define@key{fams}{bzs}{Sign Language}
\define@key{fams}{khb}{Tai-Kadai}
\define@key{fams}{msj}{Atlantic-Congo}
\define@key{fams}{mhy}{Austronesian}
\define@key{fams}{mhi}{Central Sudanic}
\define@key{fams}{slz}{Austronesian}
\define@key{fams}{mdy}{Ta-Ne-Omotic}
\define@key{fams}{mas}{Nilotic}
\define@key{fams}{mde}{Maban}
\define@key{fams}{mca}{Matacoan}
\define@key{fams}{mbn}{Guahiboan}
\define@key{fams}{mkd}{Indo-European}
\define@key{fams}{mcb}{Arawakan}
\define@key{fams}{myy}{Tucanoan}
\define@key{fams}{mbc}{Cariban}
\define@key{fams}{mxu}{Afro-Asiatic}
\define@key{fams}{mda}{Atlantic-Congo}
\define@key{fams}{dmd}{nan}
\define@key{fams}{mad}{Austronesian}
\define@key{fams}{mmw}{Austronesian}
\define@key{fams}{mag}{Indo-European}
\define@key{fams}{mgp}{Sino-Tibetan}
\define@key{fams}{mrd}{Sino-Tibetan}
\define@key{fams}{mgu}{Mailuan}
\define@key{fams}{mdh}{Austronesian}
\define@key{fams}{mhe}{Austroasiatic}
\define@key{fams}{xpq}{nan}
\define@key{fams}{nmu}{Maiduan}
\define@key{fams}{zrs}{Mairasic}
\define@key{fams}{mbq}{Austronesian}
\define@key{fams}{mai}{Indo-European}
\define@key{fams}{mpe}{Surmic}
\define@key{fams}{mcp}{Atlantic-Congo}
\define@key{fams}{myh}{Wakashan}
\define@key{fams}{mkz}{Timor-Alor-Pantar}
\define@key{fams}{mak}{Austronesian}
\define@key{fams}{mgf}{Bulaka River}
\define@key{fams}{kde}{Atlantic-Congo}
\define@key{fams}{mgh}{Atlantic-Congo}
\define@key{fams}{mcm}{Indo-European}
\define@key{fams}{plt}{Austronesian}
\define@key{fams}{mpb}{Northern Daly}
\define@key{fams}{zsm}{Austronesian}
\define@key{fams}{zlm}{Austronesian}
\define@key{fams}{zmi}{Austronesian}
\define@key{fams}{mal}{Dravidian}
\define@key{fams}{mgl}{Austronesian}
\define@key{fams}{gcc}{Baining}
\define@key{fams}{mlt}{Afro-Asiatic}
\define@key{fams}{kmj}{Dravidian}
\define@key{fams}{mam}{Mayan}
\define@key{fams}{mmn}{Austronesian}
\define@key{fams}{mqj}{Austronesian}
\define@key{fams}{mcs}{Atlantic-Congo}
\define@key{fams}{mgr}{Atlantic-Congo}
\define@key{fams}{maw}{Atlantic-Congo}
\define@key{fams}{mdi}{Central Sudanic}
\define@key{fams}{xmm}{Austronesian}
\define@key{fams}{mva}{Austronesian}
\define@key{fams}{mle}{Ndu}
\define@key{fams}{nmm}{Sino-Tibetan}
\define@key{fams}{mnc}{Tungusic}
\define@key{fams}{mid}{Afro-Asiatic}
\define@key{fams}{mhq}{Siouan}
\define@key{fams}{mdr}{Austronesian}
\define@key{fams}{mnk}{Mande}
\define@key{fams}{jet}{Border}
\define@key{fams}{mna}{Austronesian}
\define@key{fams}{mpc}{Mangarrayi-Maran}
\define@key{fams}{mdj}{Central Sudanic}
\define@key{fams}{mqy}{Austronesian}
\define@key{fams}{mjg}{Mongolic-Khitan}
\define@key{fams}{mge}{Central Sudanic}
\define@key{fams}{emk}{Mande}
\define@key{fams}{mlq}{Mande}
\define@key{fams}{mfv}{Atlantic-Congo}
\define@key{fams}{knf}{Atlantic-Congo}
\define@key{fams}{nge}{Atlantic-Congo}
\define@key{fams}{mev}{Mande}
\define@key{fams}{mbb}{Austronesian}
\define@key{fams}{mns}{Uralic}
\define@key{fams}{glv}{Indo-European}
\define@key{fams}{mri}{Austronesian}
\define@key{fams}{mcg}{Cariban}
\define@key{fams}{arn}{Araucanian}
\define@key{fams}{mec}{Mangarrayi-Maran}
\define@key{fams}{mrw}{Austronesian}
\define@key{fams}{zmr}{Western Daly}
\define@key{fams}{mar}{Indo-European}
\define@key{fams}{rnp}{Sino-Tibetan}
\define@key{fams}{zmc}{Pama-Nyungan}
\define@key{fams}{mrt}{Afro-Asiatic}
\define@key{fams}{mrj}{Uralic}
\define@key{fams}{mhr}{Uralic}
\define@key{fams}{mrc}{Cochimi-Yuman}
\define@key{fams}{mrz}{Anim}
\define@key{fams}{mbw}{Nuclear Trans New Guinea}
\define@key{fams}{zmt}{Western Daly}
\define@key{fams}{mfr}{Western Daly}
\define@key{fams}{mah}{Austronesian}
\define@key{fams}{gcf}{Indo-European}
\define@key{fams}{vma}{Pama-Nyungan}
\define@key{fams}{mhx}{Sino-Tibetan}
\define@key{fams}{mcn}{Afro-Asiatic}
\define@key{fams}{jle}{Narrow Talodi}
\define@key{fams}{mls}{Maban}
\define@key{fams}{wam}{Algic}
\define@key{fams}{mpq}{Pano-Tacanan}
\define@key{fams}{zml}{Eastern Daly}
\define@key{fams}{mcf}{Pano-Tacanan}
\define@key{fams}{mvb}{Athabaskan-Eyak-Tlingit}
\define@key{fams}{mjk}{Austronesian}
\define@key{fams}{mgw}{Atlantic-Congo}
\define@key{fams}{mxx}{Mande}
\define@key{fams}{mph}{Iwaidjan Proper}
\define@key{fams}{mfe}{Indo-European}
\define@key{fams}{mke}{Indo-European}
\define@key{fams}{mbl}{Nuclear-Macro-Je}
\define@key{fams}{yan}{Misumalpan}
\define@key{fams}{ayz}{Isolate}
\define@key{fams}{xyj}{nan}
\define@key{fams}{mfy}{Uto-Aztecan}
\define@key{fams}{mdm}{Atlantic-Congo}
\define@key{fams}{maz}{Otomanguean}
\define@key{fams}{mzn}{Indo-European}
\define@key{fams}{maq}{Otomanguean}
\define@key{fams}{mau}{Otomanguean}
\define@key{fams}{mfc}{Atlantic-Congo}
\define@key{fams}{vmb}{Pama-Nyungan}
\define@key{fams}{lnb}{Atlantic-Congo}
\define@key{fams}{mpk}{Afro-Asiatic}
\define@key{fams}{myb}{Central Sudanic}
\define@key{fams}{mtk}{Atlantic-Congo}
\define@key{fams}{mdt}{Atlantic-Congo}
\define@key{fams}{baw}{Atlantic-Congo}
\define@key{fams}{gmm}{Atlantic-Congo}
\define@key{fams}{mdq}{Atlantic-Congo}
\define@key{fams}{mdw}{Atlantic-Congo}
\define@key{fams}{mhd}{Atlantic-Congo}
\define@key{fams}{mdd}{Atlantic-Congo}
\define@key{fams}{mym}{Surmic}
\define@key{fams}{nux}{Sepik}
\define@key{fams}{gdq}{Afro-Asiatic}
\define@key{fams}{mni}{Sino-Tibetan}
\define@key{fams}{skf}{Tupian}
\define@key{fams}{mek}{Austronesian}
\define@key{fams}{mel}{Austronesian}
\define@key{fams}{bew}{Austronesian}
\define@key{fams}{men}{Mande}
\define@key{fams}{mez}{Algic}
\define@key{fams}{mwv}{Austronesian}
\define@key{fams}{sdo}{Austronesian}
\define@key{fams}{mcr}{Angan}
\define@key{fams}{ulk}{Eastern Trans-Fly}
\define@key{fams}{mej}{East Bird's Head}
\define@key{fams}{mpt}{Nuclear Trans New Guinea}
\define@key{fams}{crg}{Algic}
\define@key{fams}{mic}{Algic}
\define@key{fams}{mei}{Nubian}
\define@key{fams}{ium}{Hmong-Mien}
\define@key{fams}{mmy}{Afro-Asiatic}
\define@key{fams}{mxj}{Sino-Tibetan}
\define@key{fams}{msy}{Lower Sepik-Ramu}
\define@key{fams}{mik}{Muskogean}
\define@key{fams}{mjw}{Sino-Tibetan}
\define@key{fams}{hna}{Afro-Asiatic}
\define@key{fams}{min}{Austronesian}
\define@key{fams}{mvn}{Austronesian}
\define@key{fams}{xmf}{Kartvelian}
\define@key{fams}{mep}{Jarrakan}
\define@key{fams}{nju}{Pama-Nyungan}
\define@key{fams}{mrg}{Sino-Tibetan}
\define@key{fams}{miq}{Misumalpan}
\define@key{fams}{zmq}{Atlantic-Congo}
\define@key{fams}{csi}{Miwok-Costanoan}
\define@key{fams}{csm}{Miwok-Costanoan}
\define@key{fams}{lmw}{Miwok-Costanoan}
\define@key{fams}{nsq}{Miwok-Costanoan}
\define@key{fams}{pmw}{Miwok-Costanoan}
\define@key{fams}{skd}{Miwok-Costanoan}
\define@key{fams}{mxp}{Mixe-Zoque}
\define@key{fams}{mco}{Mixe-Zoque}
\define@key{fams}{mto}{Mixe-Zoque}
\define@key{fams}{mim}{Otomanguean}
\define@key{fams}{mib}{Otomanguean}
\define@key{fams}{miy}{Otomanguean}
\define@key{fams}{mih}{Otomanguean}
\define@key{fams}{miz}{Otomanguean}
\define@key{fams}{mxt}{Otomanguean}
\define@key{fams}{mio}{Otomanguean}
\define@key{fams}{mig}{Otomanguean}
\define@key{fams}{mie}{Otomanguean}
\define@key{fams}{mil}{Otomanguean}
\define@key{fams}{mjc}{Otomanguean}
\define@key{fams}{mks}{Otomanguean}
\define@key{fams}{mpm}{Otomanguean}
\define@key{fams}{mkf}{Afro-Asiatic}
\define@key{fams}{lus}{Sino-Tibetan}
\define@key{fams}{mra}{Austroasiatic}
\define@key{fams}{moy}{Ta-Ne-Omotic}
\define@key{fams}{omc}{Isolate}
\define@key{fams}{moc}{Guaicuruan}
\define@key{fams}{mif}{Afro-Asiatic}
\define@key{fams}{mhj}{Mongolic-Khitan}
\define@key{fams}{moh}{Iroquoian}
\define@key{fams}{mov}{Cochimi-Yuman}
\define@key{fams}{mkj}{Austronesian}
\define@key{fams}{moz}{Afro-Asiatic}
\define@key{fams}{mbe}{Isolate}
\define@key{fams}{mso}{Mombum-Koneraw}
\define@key{fams}{fqs}{Baibai-Fas}
\define@key{fams}{mqf}{Somahai}
\define@key{fams}{mnw}{Austroasiatic}
\define@key{fams}{ndt}{Atlantic-Congo}
\define@key{fams}{lol}{Atlantic-Congo}
\define@key{fams}{mog}{Austronesian}
\define@key{fams}{mnz}{Nuclear Trans New Guinea}
\define@key{fams}{mnr}{Uto-Aztecan}
\define@key{fams}{mte}{Austronesian}
\define@key{fams}{moe}{Algic}
\define@key{fams}{mxk}{Bogia}
\define@key{fams}{mos}{Atlantic-Congo}
\define@key{fams}{mop}{Mayan}
\define@key{fams}{mhz}{Austronesian}
\define@key{fams}{mok}{Isolate}
\define@key{fams}{myv}{Uralic}
\define@key{fams}{mdf}{Uralic}
\define@key{fams}{mor}{Heibanic}
\define@key{fams}{mgd}{Central Sudanic}
\define@key{fams}{cas}{Isolate}
\define@key{fams}{meu}{Austronesian}
\define@key{fams}{siw}{South Bougainville}
\define@key{fams}{mzp}{Isolate}
\define@key{fams}{mye}{Atlantic-Congo}
\define@key{fams}{akc}{Isolate}
\define@key{fams}{dmw}{Pama-Nyungan}
\define@key{fams}{aoj}{Nuclear Torricelli}
\define@key{fams}{sgw}{Afro-Asiatic}
\define@key{fams}{bmr}{Boran}
\define@key{fams}{chb}{Chibchan}
\define@key{fams}{mlm}{Tai-Kadai}
\define@key{fams}{mzm}{Atlantic-Congo}
\define@key{fams}{mji}{Hmong-Mien}
\define@key{fams}{mnb}{Austronesian}
\define@key{fams}{mua}{Atlantic-Congo}
\define@key{fams}{mnf}{Atlantic-Congo}
\define@key{fams}{myu}{Tupian}
\define@key{fams}{mhk}{Atlantic-Congo}
\define@key{fams}{umu}{Algic}
\define@key{fams}{moj}{Atlantic-Congo}
\define@key{fams}{mtq}{Austroasiatic}
\define@key{fams}{sur}{Afro-Asiatic}
\define@key{fams}{mtf}{Lower Sepik-Ramu}
\define@key{fams}{mur}{Surmic}
\define@key{fams}{mwf}{Southern Daly}
\define@key{fams}{muz}{Surmic}
\define@key{fams}{zmu}{Pama-Nyungan}
\define@key{fams}{mug}{Afro-Asiatic}
\define@key{fams}{msu}{Austronesian}
\define@key{fams}{hur}{Salishan}
\define@key{fams}{emi}{Austronesian}
\define@key{fams}{css}{Miwok-Costanoan}
\define@key{fams}{myw}{Austronesian}
\define@key{fams}{mwe}{Atlantic-Congo}
\define@key{fams}{mlv}{Austronesian}
\define@key{fams}{xak}{Isolate}
\define@key{fams}{bzk}{Indo-European}
\define@key{fams}{muh}{Atlantic-Congo}
\define@key{fams}{naf}{Nuclear Trans New Guinea}
\define@key{fams}{wyy}{Austronesian}
\define@key{fams}{mbj}{Naduhup}
\define@key{fams}{nfr}{Atlantic-Congo}
\define@key{fams}{nbi}{Sino-Tibetan}
\define@key{fams}{nmf}{Sino-Tibetan}
\define@key{fams}{nzm}{Sino-Tibetan}
\define@key{fams}{nag}{Indo-European}
\define@key{fams}{nce}{Isolate}
\define@key{fams}{nll}{Isolate}
\define@key{fams}{nhn}{Uto-Aztecan}
\define@key{fams}{ncj}{Uto-Aztecan}
\define@key{fams}{nhx}{Uto-Aztecan}
\define@key{fams}{ncl}{Uto-Aztecan}
\define@key{fams}{nhm}{Uto-Aztecan}
\define@key{fams}{nhp}{Uto-Aztecan}
\define@key{fams}{xpo}{Uto-Aztecan}
\define@key{fams}{azz}{Uto-Aztecan}
\define@key{fams}{nhg}{Uto-Aztecan}
\define@key{fams}{ngu}{Uto-Aztecan}
\define@key{fams}{bio}{Kwomtari-Nai}
\define@key{fams}{nak}{Austronesian}
\define@key{fams}{nck}{Maningrida}
\define@key{fams}{nal}{Austronesian}
\define@key{fams}{naq}{Khoe-Kwadi}
\define@key{fams}{nmb}{Austronesian}
\define@key{fams}{nab}{Nambiquaran}
\define@key{fams}{nnm}{Sepik}
\define@key{fams}{gld}{Tungusic}
\define@key{fams}{ncb}{Austroasiatic}
\define@key{fams}{nnb}{Atlantic-Congo}
\define@key{fams}{niq}{Nilotic}
\define@key{fams}{sen}{Atlantic-Congo}
\define@key{fams}{nnk}{Nuclear Trans New Guinea}
\define@key{fams}{nnt}{Algic}
\define@key{fams}{tvl}{Austronesian}
\define@key{fams}{npy}{Austronesian}
\define@key{fams}{npa}{Sino-Tibetan}
\define@key{fams}{nrb}{Isolate}
\define@key{fams}{nrm}{Austronesian}
\define@key{fams}{nas}{South Bougainville}
\define@key{fams}{nsk}{Algic}
\define@key{fams}{ncz}{Isolate}
\define@key{fams}{ntm}{Atlantic-Congo}
\define@key{fams}{ntu}{Austronesian}
\define@key{fams}{nau}{Austronesian}
\define@key{fams}{nav}{Athabaskan-Eyak-Tlingit}
\define@key{fams}{nxq}{Sino-Tibetan}
\define@key{fams}{bud}{Atlantic-Congo}
\define@key{fams}{nde}{Atlantic-Congo}
\define@key{fams}{djj}{Maningrida}
\define@key{fams}{ndz}{Atlantic-Congo}
\define@key{fams}{ndo}{Atlantic-Congo}
\define@key{fams}{nmd}{Atlantic-Congo}
\define@key{fams}{ndv}{Atlantic-Congo}
\define@key{fams}{djk}{Indo-European}
\define@key{fams}{dse}{Sign Language}
\define@key{fams}{neg}{Tungusic}
\define@key{fams}{nsn}{Austronesian}
\define@key{fams}{nee}{Austronesian}
\define@key{fams}{anh}{Nuclear Trans New Guinea}
\define@key{fams}{yrk}{Uralic}
\define@key{fams}{nen}{Austronesian}
\define@key{fams}{aij}{Afro-Asiatic}
\define@key{fams}{aii}{Afro-Asiatic}
\define@key{fams}{trg}{Afro-Asiatic}
\define@key{fams}{npi}{Indo-European}
\define@key{fams}{pia}{Uto-Aztecan}
\define@key{fams}{nzs}{Sign Language}
\define@key{fams}{new}{Sino-Tibetan}
\define@key{fams}{ney}{Kru}
\define@key{fams}{nez}{Sahaptian}
\define@key{fams}{ntj}{Pama-Nyungan}
\define@key{fams}{nxg}{Austronesian}
\define@key{fams}{nig}{Gunwinyguan}
\define@key{fams}{ngk}{Gunwinyguan}
\define@key{fams}{sba}{Central Sudanic}
\define@key{fams}{nam}{Southern Daly}
\define@key{fams}{nio}{Uralic}
\define@key{fams}{nid}{Gunwinyguan}
\define@key{fams}{nay}{Pama-Nyungan}
\define@key{fams}{nrk}{Pama-Nyungan}
\define@key{fams}{nrl}{Pama-Nyungan}
\define@key{fams}{nxn}{Pama-Nyungan}
\define@key{fams}{nbm}{Atlantic-Congo}
\define@key{fams}{nga}{Atlantic-Congo}
\define@key{fams}{ngb}{Atlantic-Congo}
\define@key{fams}{niy}{Central Sudanic}
\define@key{fams}{wyb}{Pama-Nyungan}
\define@key{fams}{ngi}{Afro-Asiatic}
\define@key{fams}{ngo}{nan}
\define@key{fams}{llp}{Austronesian}
\define@key{fams}{gym}{Chibchan}
\define@key{fams}{nha}{Pama-Nyungan}
\define@key{fams}{nhr}{Khoe-Kwadi}
\define@key{fams}{nia}{Austronesian}
\define@key{fams}{caq}{Austroasiatic}
\define@key{fams}{pcm}{Indo-European}
\define@key{fams}{jsl}{Sign Language}
\define@key{fams}{nir}{Nimboranic}
\define@key{fams}{niz}{Nuclear Torricelli}
\define@key{fams}{nsz}{Maiduan}
\define@key{fams}{ncg}{Tsimshian}
\define@key{fams}{dtd}{Wakashan}
\define@key{fams}{num}{Austronesian}
\define@key{fams}{niu}{Austronesian}
\define@key{fams}{cag}{Matacoan}
\define@key{fams}{niv}{Nivkh}
\define@key{fams}{isi}{Atlantic-Congo}
\define@key{fams}{nko}{Atlantic-Congo}
\define@key{fams}{cgg}{Atlantic-Congo}
\define@key{fams}{fia}{Nubian}
\define@key{fams}{njb}{Sino-Tibetan}
\define@key{fams}{nog}{Turkic}
\define@key{fams}{not}{Arawakan}
\define@key{fams}{nhu}{Atlantic-Congo}
\define@key{fams}{snf}{Atlantic-Congo}
\define@key{fams}{nsl}{Sign Language}
\define@key{fams}{nor}{Indo-European}
\define@key{fams}{nse}{Atlantic-Congo}
\define@key{fams}{nto}{Atlantic-Congo}
\define@key{fams}{nxl}{Austronesian}
\define@key{fams}{kcn}{Afro-Asiatic}
\define@key{fams}{dgl}{nan}
\define@key{fams}{xnz}{nan}
\define@key{fams}{nus}{Nilotic}
\define@key{fams}{mbr}{Kakua-Nukak}
\define@key{fams}{nkr}{Austronesian}
\define@key{fams}{nut}{Tai-Kadai}
\define@key{fams}{nuy}{Gunwinyguan}
\define@key{fams}{nuv}{Atlantic-Congo}
\define@key{fams}{iii}{Sino-Tibetan}
\define@key{fams}{nup}{Atlantic-Congo}
\define@key{fams}{nuf}{Sino-Tibetan}
\define@key{fams}{cbn}{Austroasiatic}
\define@key{fams}{nly}{Pama-Nyungan}
\define@key{fams}{now}{Atlantic-Congo}
\define@key{fams}{tpq}{nan}
\define@key{fams}{nym}{Atlantic-Congo}
\define@key{fams}{nyj}{Atlantic-Congo}
\define@key{fams}{nyp}{Kuliak}
\define@key{fams}{nna}{Pama-Nyungan}
\define@key{fams}{nyt}{Pama-Nyungan}
\define@key{fams}{yly}{Austronesian}
\define@key{fams}{nyh}{Nyulnyulan}
\define@key{fams}{nih}{Atlantic-Congo}
\define@key{fams}{nyi}{Nyimang}
\define@key{fams}{njz}{Sino-Tibetan}
\define@key{fams}{nyv}{Nyulnyulan}
\define@key{fams}{nys}{Pama-Nyungan}
\define@key{fams}{nzk}{Atlantic-Congo}
\define@key{fams}{ood}{Uto-Aztecan}
\define@key{fams}{afz}{Lakes Plain}
\define@key{fams}{ann}{Atlantic-Congo}
\define@key{fams}{oca}{Huitotoan}
\define@key{fams}{oci}{Indo-European}
\define@key{fams}{ocu}{Otomanguean}
\define@key{fams}{ogb}{Atlantic-Congo}
\define@key{fams}{ogu}{Atlantic-Congo}
\define@key{fams}{oyb}{Austroasiatic}
\define@key{fams}{xal}{Mongolic-Khitan}
\define@key{fams}{ojs}{Algic}
\define@key{fams}{ciw}{Algic}
\define@key{fams}{oka}{Salishan}
\define@key{fams}{opm}{Nuclear Trans New Guinea}
\define@key{fams}{oku}{Atlantic-Congo}
\define@key{fams}{ong}{Nuclear Torricelli}
\define@key{fams}{plo}{Mixe-Zoque}
\define@key{fams}{omg}{Tupian}
\define@key{fams}{oma}{Siouan}
\define@key{fams}{aun}{Nuclear Torricelli}
\define@key{fams}{one}{Iroquoian}
\define@key{fams}{oon}{Jarawa-Onge}
\define@key{fams}{ons}{Nuclear Trans New Guinea}
\define@key{fams}{ono}{Iroquoian}
\define@key{fams}{mvf}{Mongolic-Khitan}
\define@key{fams}{ore}{Tucanoan}
\define@key{fams}{tag}{Rashad}
\define@key{fams}{ory}{Indo-European}
\define@key{fams}{ort}{Indo-European}
\define@key{fams}{oru}{Indo-European}
\define@key{fams}{oac}{Tungusic}
\define@key{fams}{oaa}{Tungusic}
\define@key{fams}{okv}{Nuclear Trans New Guinea}
\define@key{fams}{oro}{Eleman}
\define@key{fams}{gax}{Afro-Asiatic}
\define@key{fams}{hae}{Afro-Asiatic}
\define@key{fams}{ssn}{Afro-Asiatic}
\define@key{fams}{gaz}{Afro-Asiatic}
\define@key{fams}{ury}{Tor-Orya}
\define@key{fams}{osa}{Siouan}
\define@key{fams}{oss}{Indo-European}
\define@key{fams}{iow}{Siouan}
\define@key{fams}{otz}{Otomanguean}
\define@key{fams}{ote}{Otomanguean}
\define@key{fams}{otq}{Otomanguean}
\define@key{fams}{otm}{Otomanguean}
\define@key{fams}{otr}{Heibanic}
\define@key{fams}{owi}{Left May}
\define@key{fams}{pqa}{Afro-Asiatic}
\define@key{fams}{drl}{Pama-Nyungan}
\define@key{fams}{pma}{Austronesian}
\define@key{fams}{pac}{Austroasiatic}
\define@key{fams}{pdo}{Austronesian}
\define@key{fams}{pgu}{North Halmahera}
\define@key{fams}{duf}{Austronesian}
\define@key{fams}{pck}{Sino-Tibetan}
\define@key{fams}{pao}{Uto-Aztecan}
\define@key{fams}{pwn}{Austronesian}
\define@key{fams}{pkn}{Pama-Nyungan}
\define@key{fams}{pau}{Austronesian}
\define@key{fams}{pll}{Austroasiatic}
\define@key{fams}{plu}{Arawakan}
\define@key{fams}{fap}{Atlantic-Congo}
\define@key{fams}{nad}{nan}
\define@key{fams}{pmz}{Otomanguean}
\define@key{fams}{pmf}{Austronesian}
\define@key{fams}{pbh}{Cariban}
\define@key{fams}{kre}{Nuclear-Macro-Je}
\define@key{fams}{pag}{Austronesian}
\define@key{fams}{pbr}{Atlantic-Congo}
\define@key{fams}{pan}{Indo-European}
\define@key{fams}{pnw}{Pama-Nyungan}
\define@key{fams}{pap}{Indo-European}
\define@key{fams}{prk}{Austroasiatic}
\define@key{fams}{asa}{Atlantic-Congo}
\define@key{fams}{pab}{Arawakan}
\define@key{fams}{pci}{Dravidian}
\define@key{fams}{pst}{Indo-European}
\define@key{fams}{pqm}{Algic}
\define@key{fams}{ptp}{Austronesian}
\define@key{fams}{gfk}{Austronesian}
\define@key{fams}{lae}{Sino-Tibetan}
\define@key{fams}{pwi}{Wintuan}
\define@key{fams}{plh}{Austronesian}
\define@key{fams}{pad}{Arawan}
\define@key{fams}{pwa}{Isolate}
\define@key{fams}{paw}{Caddoan}
\define@key{fams}{pay}{Chibchan}
\define@key{fams}{aoc}{Cariban}
\define@key{fams}{peg}{Dravidian}
\define@key{fams}{pip}{Afro-Asiatic}
\define@key{fams}{pes}{Indo-European}
\define@key{fams}{pww}{Sino-Tibetan}
\define@key{fams}{pio}{Arawakan}
\define@key{fams}{pid}{Saliban}
\define@key{fams}{plg}{Guaicuruan}
\define@key{fams}{piv}{Austronesian}
\define@key{fams}{pif}{Austronesian}
\define@key{fams}{piu}{Pama-Nyungan}
\define@key{fams}{ppl}{Uto-Aztecan}
\define@key{fams}{myp}{Isolate}
\define@key{fams}{pir}{Tucanoan}
\define@key{fams}{pib}{Arawakan}
\define@key{fams}{psa}{Nuclear Trans New Guinea}
\define@key{fams}{pjt}{Pama-Nyungan}
\define@key{fams}{pit}{Pama-Nyungan}
\define@key{fams}{psd}{Sign Language}
\define@key{fams}{gob}{Guahiboan}
\define@key{fams}{fwa}{Austronesian}
\define@key{fams}{pbi}{Afro-Asiatic}
\define@key{fams}{poy}{Atlantic-Congo}
\define@key{fams}{pon}{Austronesian}
\define@key{fams}{rwa}{Sko}
\define@key{fams}{poh}{Mayan}
\define@key{fams}{pko}{Nilotic}
\define@key{fams}{pox}{Indo-European}
\define@key{fams}{pol}{Indo-European}
\define@key{fams}{poo}{Pomoan}
\define@key{fams}{peb}{Pomoan}
\define@key{fams}{pej}{Pomoan}
\define@key{fams}{pom}{Pomoan}
\define@key{fams}{pbe}{Otomanguean}
\define@key{fams}{poe}{Otomanguean}
\define@key{fams}{pbf}{Otomanguean}
\define@key{fams}{poi}{Mixe-Zoque}
\define@key{fams}{poc}{Mayan}
\define@key{fams}{psw}{Austronesian}
\define@key{fams}{por}{Indo-European}
\define@key{fams}{pot}{Algic}
\define@key{fams}{pim}{Algic}
\define@key{fams}{prn}{Indo-European}
\define@key{fams}{pre}{Indo-European}
\define@key{fams}{pui}{Isolate}
\define@key{fams}{fuc}{Atlantic-Congo}
\define@key{fams}{nij}{Austronesian}
\define@key{fams}{puw}{Austronesian}
\define@key{fams}{pmi}{Sino-Tibetan}
\define@key{fams}{puq}{Isolate}
\define@key{fams}{prx}{Sino-Tibetan}
\define@key{fams}{tsz}{Tarascan}
\define@key{fams}{pbb}{Isolate}
\define@key{fams}{lkr}{Nilotic}
\define@key{fams}{aar}{Afro-Asiatic}
\define@key{fams}{byx}{Baining}
\define@key{fams}{alc}{Kawesqar}
\define@key{fams}{yum}{Cochimi-Yuman}
\define@key{fams}{qxa}{Quechuan}
\define@key{fams}{quy}{Quechuan}
\define@key{fams}{qvc}{Quechuan}
\define@key{fams}{quh}{Quechuan}
\define@key{fams}{quz}{Quechuan}
\define@key{fams}{qug}{Quechuan}
\define@key{fams}{qub}{Quechuan}
\define@key{fams}{qvi}{Quechuan}
\define@key{fams}{qvn}{Quechuan}
\define@key{fams}{quc}{Mayan}
\define@key{fams}{qui}{Chimakuan}
\define@key{fams}{rad}{Austronesian}
\define@key{fams}{lml}{Austronesian}
\define@key{fams}{rji}{Sino-Tibetan}
\define@key{fams}{ral}{Sino-Tibetan}
\define@key{fams}{rma}{Chibchan}
\define@key{fams}{bod}{Sino-Tibetan}
\define@key{fams}{rao}{Lower Sepik-Ramu}
\define@key{fams}{rap}{Austronesian}
\define@key{fams}{ras}{Rashad}
\define@key{fams}{rwo}{Nuclear Trans New Guinea}
\define@key{fams}{raw}{Sino-Tibetan}
\define@key{fams}{rej}{Austronesian}
\define@key{fams}{rmb}{Gunwinyguan}
\define@key{fams}{bfw}{Austroasiatic}
\define@key{fams}{rel}{Afro-Asiatic}
\define@key{fams}{ren}{Austroasiatic}
\define@key{fams}{mnv}{Austronesian}
\define@key{fams}{rgr}{Arawakan}
\define@key{fams}{tnc}{Tucanoan}
\define@key{fams}{ran}{Kolopom}
\define@key{fams}{rkb}{Nuclear-Macro-Je}
\define@key{fams}{rim}{Atlantic-Congo}
\define@key{fams}{rit}{Pama-Nyungan}
\define@key{fams}{rog}{Austronesian}
\define@key{fams}{rmn}{Indo-European}
\define@key{fams}{rmo}{Indo-European}
\define@key{fams}{rmy}{Indo-European}
\define@key{fams}{rml}{Indo-European}
\define@key{fams}{rmw}{Indo-European}
\define@key{fams}{ron}{Indo-European}
\define@key{fams}{roh}{Indo-European}
\define@key{fams}{cla}{Afro-Asiatic}
\define@key{fams}{rng}{Atlantic-Congo}
\define@key{fams}{rro}{Austronesian}
\define@key{fams}{twu}{Austronesian}
\define@key{fams}{roo}{North Bougainville}
\define@key{fams}{rtm}{Austronesian}
\define@key{fams}{rug}{Austronesian}
\define@key{fams}{dru}{Austronesian}
\define@key{fams}{klq}{Turama-Kikori}
\define@key{fams}{run}{Atlantic-Congo}
\define@key{fams}{rou}{Maban}
\define@key{fams}{nyn}{Atlantic-Congo}
\define@key{fams}{nyo}{Atlantic-Congo}
\define@key{fams}{rus}{Indo-European}
\define@key{fams}{rsl}{Sign Language}
\define@key{fams}{rut}{Nakh-Daghestanian}
\define@key{fams}{apb}{Austronesian}
\define@key{fams}{snv}{Austronesian}
\define@key{fams}{sma}{Uralic}
\define@key{fams}{sjd}{Uralic}
\define@key{fams}{sme}{Uralic}
\define@key{fams}{skb}{Tai-Kadai}
\define@key{fams}{uma}{Sahaptian}
\define@key{fams}{ssy}{Afro-Asiatic}
\define@key{fams}{saj}{North Halmahera}
\define@key{fams}{sku}{Austronesian}
\define@key{fams}{slr}{Turkic}
\define@key{fams}{sbe}{Austronesian}
\define@key{fams}{sln}{Isolate}
\define@key{fams}{slh}{Salishan}
\define@key{fams}{sll}{Nuclear Trans New Guinea}
\define@key{fams}{sse}{Austronesian}
\define@key{fams}{ssb}{Austronesian}
\define@key{fams}{ndi}{Atlantic-Congo}
\define@key{fams}{smq}{East Strickland}
\define@key{fams}{smo}{Austronesian}
\define@key{fams}{sad}{Isolate}
\define@key{fams}{sxn}{Austronesian}
\define@key{fams}{sag}{Atlantic-Congo}
\define@key{fams}{snq}{Atlantic-Congo}
\define@key{fams}{sce}{Mongolic-Khitan}
\define@key{fams}{sat}{Austroasiatic}
\define@key{fams}{xsu}{Yanomamic}
\define@key{fams}{spu}{Austroasiatic}
\define@key{fams}{srm}{Indo-European}
\define@key{fams}{srs}{Athabaskan-Eyak-Tlingit}
\define@key{fams}{sro}{Indo-European}
\define@key{fams}{dju}{Sepik}
\define@key{fams}{ybe}{Turkic}
\define@key{fams}{sdg}{Indo-European}
\define@key{fams}{svs}{Isolate}
\define@key{fams}{szw}{Austronesian}
\define@key{fams}{hvn}{Austronesian}
\define@key{fams}{pos}{Mixe-Zoque}
\define@key{fams}{kpz}{Nilotic}
\define@key{fams}{sey}{Tucanoan}
\define@key{fams}{sed}{Austroasiatic}
\define@key{fams}{trv}{Austronesian}
\define@key{fams}{slu}{Austronesian}
\define@key{fams}{sly}{Austronesian}
\define@key{fams}{spl}{Nuclear Trans New Guinea}
\define@key{fams}{ona}{Chonan}
\define@key{fams}{sel}{Uralic}
\define@key{fams}{nsm}{Sino-Tibetan}
\define@key{fams}{sea}{Austroasiatic}
\define@key{fams}{sif}{Isolate}
\define@key{fams}{sza}{Austroasiatic}
\define@key{fams}{seh}{Atlantic-Congo}
\define@key{fams}{sef}{Atlantic-Congo}
\define@key{fams}{see}{Iroquoian}
\define@key{fams}{szg}{Atlantic-Congo}
\define@key{fams}{set}{Sentanic}
\define@key{fams}{hbs}{Indo-European}
\define@key{fams}{sei}{Isolate}
\define@key{fams}{ser}{Uto-Aztecan}
\define@key{fams}{sot}{Atlantic-Congo}
\define@key{fams}{crs}{Indo-European}
\define@key{fams}{sbf}{Isolate}
\define@key{fams}{ksb}{Atlantic-Congo}
\define@key{fams}{shn}{Tai-Kadai}
\define@key{fams}{mcd}{Pano-Tacanan}
\define@key{fams}{sht}{Shastan}
\define@key{fams}{shj}{Dajuic}
\define@key{fams}{sjw}{Algic}
\define@key{fams}{swv}{Indo-European}
\define@key{fams}{sdp}{Sino-Tibetan}
\define@key{fams}{xsr}{Sino-Tibetan}
\define@key{fams}{shk}{Nilotic}
\define@key{fams}{scl}{Indo-European}
\define@key{fams}{bwo}{Ta-Ne-Omotic}
\define@key{fams}{shp}{Pano-Tacanan}
\define@key{fams}{yuy}{Mongolic-Khitan}
\define@key{fams}{shb}{Yanomamic}
\define@key{fams}{sii}{Isolate}
\define@key{fams}{sna}{Atlantic-Congo}
\define@key{fams}{cjs}{Turkic}
\define@key{fams}{shh}{Uto-Aztecan}
\define@key{fams}{sgh}{Indo-European}
\define@key{fams}{ryu}{Japonic}
\define@key{fams}{shs}{Salishan}
\define@key{fams}{snp}{Nuclear Trans New Guinea}
\define@key{fams}{sjr}{Austronesian}
\define@key{fams}{sid}{Afro-Asiatic}
\define@key{fams}{ski}{Austronesian}
\define@key{fams}{tty}{Lakes Plain}
\define@key{fams}{sip}{Sino-Tibetan}
\define@key{fams}{skh}{Austronesian}
\define@key{fams}{dau}{Dajuic}
\define@key{fams}{smr}{Austronesian}
\define@key{fams}{snc}{Austronesian}
\define@key{fams}{snd}{Indo-European}
\define@key{fams}{sin}{Indo-European}
\define@key{fams}{xsi}{Austronesian}
\define@key{fams}{snn}{Tucanoan}
\define@key{fams}{qum}{Mayan}
\define@key{fams}{fos}{Austronesian}
\define@key{fams}{sri}{Tucanoan}
\define@key{fams}{srq}{Tupian}
\define@key{fams}{ssd}{Nuclear Trans New Guinea}
\define@key{fams}{sil}{Atlantic-Congo}
\define@key{fams}{baa}{Austronesian}
\define@key{fams}{sis}{Isolate}
\define@key{fams}{skv}{Sko}
\define@key{fams}{den}{nan}
\define@key{fams}{xsl}{Athabaskan-Eyak-Tlingit}
\define@key{fams}{slk}{Indo-European}
\define@key{fams}{slv}{Indo-European}
\define@key{fams}{teu}{Kuliak}
\define@key{fams}{sob}{Austronesian}
\define@key{fams}{gru}{Afro-Asiatic}
\define@key{fams}{evn}{Tungusic}
\define@key{fams}{som}{Afro-Asiatic}
\define@key{fams}{sop}{Atlantic-Congo}
\define@key{fams}{snk}{Mande}
\define@key{fams}{sov}{Austronesian}
\define@key{fams}{sqt}{Afro-Asiatic}
\define@key{fams}{srb}{Austroasiatic}
\define@key{fams}{dsb}{Indo-European}
\define@key{fams}{hsb}{Indo-European}
\define@key{fams}{nso}{Atlantic-Congo}
\define@key{fams}{mnx}{East Bird's Head}
\define@key{fams}{kvk}{Sign Language}
\define@key{fams}{tvk}{Austronesian}
\define@key{fams}{wib}{Atlantic-Congo}
\define@key{fams}{spa}{Indo-European}
\define@key{fams}{spt}{Sino-Tibetan}
\define@key{fams}{spo}{Salishan}
\define@key{fams}{squ}{Salishan}
\define@key{fams}{srn}{Indo-European}
\define@key{fams}{kpm}{Austroasiatic}
\define@key{fams}{sto}{Siouan}
\define@key{fams}{sbs}{Atlantic-Congo}
\define@key{fams}{tgo}{Austronesian}
\define@key{fams}{sue}{Nuclear Trans New Guinea}
\define@key{fams}{swi}{Tai-Kadai}
\define@key{fams}{sui}{Suki-Gogodala}
\define@key{fams}{sub}{Atlantic-Congo}
\define@key{fams}{suk}{Atlantic-Congo}
\define@key{fams}{sua}{Isolate}
\define@key{fams}{suv}{Sino-Tibetan}
\define@key{fams}{sun}{Austronesian}
\define@key{fams}{sjg}{Tamaic}
\define@key{fams}{spp}{Atlantic-Congo}
\define@key{fams}{sgz}{Austronesian}
\define@key{fams}{sus}{Mande}
\define@key{fams}{sva}{Kartvelian}
\define@key{fams}{swl}{Sign Language}
\define@key{fams}{swh}{Atlantic-Congo}
\define@key{fams}{ssw}{Atlantic-Congo}
\define@key{fams}{swe}{Indo-European}
\define@key{fams}{slc}{Saliban}
\define@key{fams}{mky}{Austronesian}
\define@key{fams}{sst}{Nuclear Trans New Guinea}
\define@key{fams}{tby}{North Halmahera}
\define@key{fams}{tab}{Nakh-Daghestanian}
\define@key{fams}{tnm}{Sentanic}
\define@key{fams}{tap}{Atlantic-Congo}
\define@key{fams}{tna}{Pano-Tacanan}
\define@key{fams}{tgl}{Austronesian}
\define@key{fams}{tbw}{Austronesian}
\define@key{fams}{tah}{Austronesian}
\define@key{fams}{gpn}{Isolate}
\define@key{fams}{sps}{Austronesian}
\define@key{fams}{tbg}{Nuclear Trans New Guinea}
\define@key{fams}{tss}{Sign Language}
\define@key{fams}{tgk}{Indo-European}
\define@key{fams}{tkm}{Isolate}
\define@key{fams}{tbc}{Austronesian}
\define@key{fams}{tld}{Austronesian}
\define@key{fams}{tlj}{Atlantic-Congo}
\define@key{fams}{tly}{Indo-European}
\define@key{fams}{tma}{Tamaic}
\define@key{fams}{mla}{Austronesian}
\define@key{fams}{tcg}{Kayagaric}
\define@key{fams}{taj}{Sino-Tibetan}
\define@key{fams}{taq}{Afro-Asiatic}
\define@key{fams}{tam}{Dravidian}
\define@key{fams}{tpm}{Atlantic-Congo}
\define@key{fams}{tcb}{Athabaskan-Eyak-Tlingit}
\define@key{fams}{tfn}{Athabaskan-Eyak-Tlingit}
\define@key{fams}{taa}{Athabaskan-Eyak-Tlingit}
\define@key{fams}{tan}{Afro-Asiatic}
\define@key{fams}{skj}{Sino-Tibetan}
\define@key{fams}{tgg}{Austronesian}
\define@key{fams}{tpg}{Timor-Alor-Pantar}
\define@key{fams}{nwi}{Austronesian}
\define@key{fams}{tza}{Sign Language}
\define@key{fams}{tpj}{Tupian}
\define@key{fams}{tar}{Uto-Aztecan}
\define@key{fams}{tac}{Uto-Aztecan}
\define@key{fams}{txn}{Austronesian}
\define@key{fams}{tro}{Sino-Tibetan}
\define@key{fams}{tae}{Arawakan}
\define@key{fams}{yer}{Atlantic-Congo}
\define@key{fams}{shi}{Afro-Asiatic}
\define@key{fams}{ttt}{Indo-European}
\define@key{fams}{txx}{Austronesian}
\define@key{fams}{tat}{Turkic}
\define@key{fams}{tks}{Indo-European}
\define@key{fams}{tav}{Tucanoan}
\define@key{fams}{tuh}{Taulil-Butam}
\define@key{fams}{trr}{Isolate}
\define@key{fams}{tsg}{Austronesian}
\define@key{fams}{tya}{Nuclear Trans New Guinea}
\define@key{fams}{tbo}{Austronesian}
\define@key{fams}{cks}{Indo-European}
\define@key{fams}{tbl}{Austronesian}
\define@key{fams}{ttc}{Mayan}
\define@key{fams}{kps}{West Bird's Head}
\define@key{fams}{teh}{Chonan}
\define@key{fams}{kkw}{Atlantic-Congo}
\define@key{fams}{tlf}{Nuclear Trans New Guinea}
\define@key{fams}{tel}{Dravidian}
\define@key{fams}{kdh}{Atlantic-Congo}
\define@key{fams}{teq}{Temeinic}
\define@key{fams}{tea}{Austroasiatic}
\define@key{fams}{tem}{Atlantic-Congo}
\define@key{fams}{tex}{Surmic}
\define@key{fams}{kza}{Atlantic-Congo}
\define@key{fams}{tio}{Austronesian}
\define@key{fams}{tep}{Uto-Aztecan}
\define@key{fams}{tee}{Totonacan}
\define@key{fams}{tpt}{Totonacan}
\define@key{fams}{ntp}{Uto-Aztecan}
\define@key{fams}{stp}{Uto-Aztecan}
\define@key{fams}{ttr}{Afro-Asiatic}
\define@key{fams}{tfr}{Chibchan}
\define@key{fams}{tft}{North Halmahera}
\define@key{fams}{ter}{Arawakan}
\define@key{fams}{teo}{Nilotic}
\define@key{fams}{tll}{Atlantic-Congo}
\define@key{fams}{tet}{Austronesian}
\define@key{fams}{tew}{Kiowa-Tanoan}
\define@key{fams}{tcz}{Sino-Tibetan}
\define@key{fams}{tha}{Tai-Kadai}
\define@key{fams}{tsq}{Sign Language}
\define@key{fams}{ths}{Sino-Tibetan}
\define@key{fams}{thf}{Sino-Tibetan}
\define@key{fams}{ssf}{Austronesian}
\define@key{fams}{typ}{Pama-Nyungan}
\define@key{fams}{thp}{Salishan}
\define@key{fams}{tdh}{Sino-Tibetan}
\define@key{fams}{tca}{Ticuna-Yuri}
\define@key{fams}{tvo}{North Halmahera}
\define@key{fams}{tif}{Nuclear Trans New Guinea}
\define@key{fams}{tgc}{Austronesian}
\define@key{fams}{tir}{Afro-Asiatic}
\define@key{fams}{tig}{Afro-Asiatic}
\define@key{fams}{dih}{Cochimi-Yuman}
\define@key{fams}{tik}{Atlantic-Congo}
\define@key{fams}{til}{Salishan}
\define@key{fams}{tms}{Katla-Tima}
\define@key{fams}{aoz}{Austronesian}
\define@key{fams}{tjm}{Isolate}
\define@key{fams}{tih}{Austronesian}
\define@key{fams}{lbf}{Sino-Tibetan}
\define@key{fams}{tin}{Nakh-Daghestanian}
\define@key{fams}{cir}{Austronesian}
\define@key{fams}{tri}{Cariban}
\define@key{fams}{tiy}{Austronesian}
\define@key{fams}{tiv}{Atlantic-Congo}
\define@key{fams}{twf}{Kiowa-Tanoan}
\define@key{fams}{tix}{Kiowa-Tanoan}
\define@key{fams}{tiw}{Isolate}
\define@key{fams}{tcf}{Otomanguean}
\define@key{fams}{tli}{Athabaskan-Eyak-Tlingit}
\define@key{fams}{tqo}{Eleman}
\define@key{fams}{tob}{Guaicuruan}
\define@key{fams}{tti}{Austronesian}
\define@key{fams}{tlb}{North Halmahera}
\define@key{fams}{sbu}{Sino-Tibetan}
\define@key{fams}{tcx}{Dravidian}
\define@key{fams}{kim}{Turkic}
\define@key{fams}{toj}{Mayan}
\define@key{fams}{tpi}{Indo-European}
\define@key{fams}{tkl}{Austronesian}
\define@key{fams}{jic}{Jicaquean}
\define@key{fams}{ksd}{Austronesian}
\define@key{fams}{dto}{Dogon}
\define@key{fams}{tdn}{Austronesian}
\define@key{fams}{toi}{Atlantic-Congo}
\define@key{fams}{ton}{Austronesian}
\define@key{fams}{tqw}{Isolate}
\define@key{fams}{tnt}{Austronesian}
\define@key{fams}{mlu}{Austronesian}
\define@key{fams}{sda}{Austronesian}
\define@key{fams}{rth}{Austronesian}
\define@key{fams}{dts}{Dogon}
\define@key{fams}{trw}{Indo-European}
\define@key{fams}{tlc}{Totonacan}
\define@key{fams}{top}{Totonacan}
\define@key{fams}{tos}{Totonacan}
\define@key{fams}{too}{Totonacan}
\define@key{fams}{trs}{Otomanguean}
\define@key{fams}{trc}{Otomanguean}
\define@key{fams}{tpy}{Isolate}
\define@key{fams}{cof}{Barbacoan}
\define@key{fams}{tkr}{Nakh-Daghestanian}
\define@key{fams}{huq}{Austronesian}
\define@key{fams}{ddo}{Nakh-Daghestanian}
\define@key{fams}{tsj}{Sino-Tibetan}
\define@key{fams}{tsi}{Tsimshian}
\define@key{fams}{tsv}{Atlantic-Congo}
\define@key{fams}{tso}{Atlantic-Congo}
\define@key{fams}{tsu}{Austronesian}
\define@key{fams}{bbl}{Nakh-Daghestanian}
\define@key{fams}{tsn}{Atlantic-Congo}
\define@key{fams}{pmt}{Austronesian}
\define@key{fams}{thz}{Afro-Asiatic}
\define@key{fams}{thv}{Afro-Asiatic}
\define@key{fams}{tbu}{Uto-Aztecan}
\define@key{fams}{tuo}{Tucanoan}
\define@key{fams}{tzn}{Austronesian}
\define@key{fams}{bag}{Atlantic-Congo}
\define@key{fams}{tcy}{Dravidian}
\define@key{fams}{tmc}{Afro-Asiatic}
\define@key{fams}{tmq}{Austronesian}
\define@key{fams}{tuf}{Chibchan}
\define@key{fams}{tvu}{Atlantic-Congo}
\define@key{fams}{lcm}{Austronesian}
\define@key{fams}{tun}{Isolate}
\define@key{fams}{tpn}{Tupian}
\define@key{fams}{tui}{Atlantic-Congo}
\define@key{fams}{tuv}{Nilotic}
\define@key{fams}{kmz}{Turkic}
\define@key{fams}{tur}{Turkic}
\define@key{fams}{tuk}{Turkic}
\define@key{fams}{tus}{Iroquoian}
\define@key{fams}{ttm}{Athabaskan-Eyak-Tlingit}
\define@key{fams}{tta}{Siouan}
\define@key{fams}{tvt}{Sino-Tibetan}
\define@key{fams}{tyv}{Turkic}
\define@key{fams}{tue}{Tucanoan}
\define@key{fams}{twa}{Salishan}
\define@key{fams}{woa}{Northern Daly}
\define@key{fams}{tzh}{Mayan}
\define@key{fams}{tzo}{Mayan}
\define@key{fams}{tzj}{Mayan}
\define@key{fams}{tub}{Uto-Aztecan}
\define@key{fams}{par}{Uto-Aztecan}
\define@key{fams}{tsm}{Sign Language}
\define@key{fams}{umb}{Atlantic-Congo}
\define@key{fams}{uby}{Abkhaz-Adyge}
\define@key{fams}{udi}{Nakh-Daghestanian}
\define@key{fams}{ude}{Tungusic}
\define@key{fams}{udm}{Uralic}
\define@key{fams}{ugn}{Sign Language}
\define@key{fams}{ukr}{Indo-European}
\define@key{fams}{ulc}{Tungusic}
\define@key{fams}{udl}{Afro-Asiatic}
\define@key{fams}{uli}{Austronesian}
\define@key{fams}{ppk}{Austronesian}
\define@key{fams}{cbd}{Cariban}
\define@key{fams}{ubu}{Nuclear Trans New Guinea}
\define@key{fams}{ump}{Pama-Nyungan}
\define@key{fams}{mtg}{Nuclear Trans New Guinea}
\define@key{fams}{unm}{Algic}
\define@key{fams}{ung}{Worrorran}
\define@key{fams}{kuu}{Athabaskan-Eyak-Tlingit}
\define@key{fams}{uur}{Austronesian}
\define@key{fams}{urf}{nan}
\define@key{fams}{urk}{Austronesian}
\define@key{fams}{ura}{Isolate}
\define@key{fams}{urt}{Nuclear Torricelli}
\define@key{fams}{urd}{Indo-European}
\define@key{fams}{urh}{Atlantic-Congo}
\define@key{fams}{uri}{Nuclear Torricelli}
\define@key{fams}{ure}{Uru-Chipaya}
\define@key{fams}{uks}{Sign Language}
\define@key{fams}{urb}{Tupian}
\define@key{fams}{uum}{Turkic}
\define@key{fams}{wnu}{Nuclear Trans New Guinea}
\define@key{fams}{usa}{Nuclear Trans New Guinea}
\define@key{fams}{ute}{Uto-Aztecan}
\define@key{fams}{uig}{Turkic}
\define@key{fams}{uzn}{Turkic}
\define@key{fams}{vaf}{Indo-European}
\define@key{fams}{vag}{Atlantic-Congo}
\define@key{fams}{vai}{Mande}
\define@key{fams}{vas}{Indo-European}
\define@key{fams}{dic}{Kru}
\define@key{fams}{ved}{Indo-European}
\define@key{fams}{ven}{Atlantic-Congo}
\define@key{fams}{vep}{Uralic}
\define@key{fams}{vie}{Austroasiatic}
\define@key{fams}{vif}{Atlantic-Congo}
\define@key{fams}{vnm}{Austronesian}
\define@key{fams}{vgt}{Sign Language}
\define@key{fams}{vot}{Uralic}
\define@key{fams}{wwa}{Atlantic-Congo}
\define@key{fams}{wkw}{Pama-Nyungan}
\define@key{fams}{waq}{Isolate}
\define@key{fams}{waw}{Cariban}
\define@key{fams}{wbk}{Indo-European}
\define@key{fams}{bao}{Tucanoan}
\define@key{fams}{wbl}{Indo-European}
\define@key{fams}{wls}{Austronesian}
\define@key{fams}{van}{Nuclear Torricelli}
\define@key{fams}{wmt}{Pama-Nyungan}
\define@key{fams}{wmb}{Mirndi}
\define@key{fams}{wms}{Nuclear Trans New Guinea}
\define@key{fams}{wme}{Sino-Tibetan}
\define@key{fams}{wan}{Mande}
\define@key{fams}{wgg}{Pama-Nyungan}
\define@key{fams}{xwk}{nan}
\define@key{fams}{wbt}{Pama-Nyungan}
\define@key{fams}{wnc}{Nuclear Trans New Guinea}
\define@key{fams}{auc}{Isolate}
\define@key{fams}{wap}{Arawakan}
\define@key{fams}{wao}{Yuki-Wappo}
\define@key{fams}{wba}{Isolate}
\define@key{fams}{wrz}{Gunwinyguan}
\define@key{fams}{war}{Austronesian}
\define@key{fams}{wrr}{Yangmanic}
\define@key{fams}{gae}{Arawakan}
\define@key{fams}{wsa}{Austronesian}
\define@key{fams}{pav}{Chapacuran}
\define@key{fams}{wrs}{Border}
\define@key{fams}{wbp}{Pama-Nyungan}
\define@key{fams}{wrb}{Pama-Nyungan}
\define@key{fams}{wnd}{Mangarrayi-Maran}
\define@key{fams}{wrp}{Austronesian}
\define@key{fams}{wgy}{Pama-Nyungan}
\define@key{fams}{gjm}{Pama-Nyungan}
\define@key{fams}{wrg}{Pama-Nyungan}
\define@key{fams}{wwr}{Nyulnyulan}
\define@key{fams}{wrm}{Pama-Nyungan}
\define@key{fams}{was}{Isolate}
\define@key{fams}{wsk}{Nuclear Trans New Guinea}
\define@key{fams}{wax}{Lower Sepik-Ramu}
\define@key{fams}{wth}{Pama-Nyungan}
\define@key{fams}{wbv}{Pama-Nyungan}
\define@key{fams}{noa}{Chocoan}
\define@key{fams}{wau}{Arawakan}
\define@key{fams}{oym}{Tupian}
\define@key{fams}{way}{Cariban}
\define@key{fams}{wed}{Austronesian}
\define@key{fams}{cym}{Indo-European}
\define@key{fams}{xww}{nan}
\define@key{fams}{wer}{Goilalan}
\define@key{fams}{mqs}{North Halmahera}
\define@key{fams}{lex}{Austronesian}
\define@key{fams}{wic}{Caddoan}
\define@key{fams}{mzh}{Matacoan}
\define@key{fams}{wim}{Pama-Nyungan}
\define@key{fams}{wig}{Pama-Nyungan}
\define@key{fams}{yok}{Yokutsan}
\define@key{fams}{win}{Siouan}
\define@key{fams}{wnw}{nan}
\define@key{fams}{wgu}{Pama-Nyungan}
\define@key{fams}{wiy}{Algic}
\define@key{fams}{wob}{Kru}
\define@key{fams}{wog}{Sepik}
\define@key{fams}{woi}{Timor-Alor-Pantar}
\define@key{fams}{wyu}{nan}
\define@key{fams}{wal}{Ta-Ne-Omotic}
\define@key{fams}{woe}{Austronesian}
\define@key{fams}{wlo}{Austronesian}
\define@key{fams}{wol}{Atlantic-Congo}
\define@key{fams}{wmx}{Sko}
\define@key{fams}{wro}{Worrorran}
\define@key{fams}{wuu}{Sino-Tibetan}
\define@key{fams}{wya}{Iroquoian}
\define@key{fams}{wem}{Atlantic-Congo}
\define@key{fams}{kao}{Mande}
\define@key{fams}{xav}{Nuclear-Macro-Je}
\define@key{fams}{xer}{Nuclear-Macro-Je}
\define@key{fams}{xho}{Atlantic-Congo}
\define@key{fams}{xir}{Arawakan}
\define@key{fams}{xok}{Nuclear-Macro-Je}
\define@key{fams}{ane}{Austronesian}
\define@key{fams}{yai}{Indo-European}
\define@key{fams}{yad}{Peba-Yagua}
\define@key{fams}{yag}{Isolate}
\define@key{fams}{yaf}{Atlantic-Congo}
\define@key{fams}{yka}{Austronesian}
\define@key{fams}{yky}{Atlantic-Congo}
\define@key{fams}{sah}{Turkic}
\define@key{fams}{ylr}{Pama-Nyungan}
\define@key{fams}{kkl}{Nuclear Trans New Guinea}
\define@key{fams}{yli}{Nuclear Trans New Guinea}
\define@key{fams}{yam}{Atlantic-Congo}
\define@key{fams}{jmd}{Austronesian}
\define@key{fams}{tao}{Austronesian}
\define@key{fams}{yaa}{Pano-Tacanan}
\define@key{fams}{ybi}{Sino-Tibetan}
\define@key{fams}{ynn}{Isolate}
\define@key{fams}{kdd}{Pama-Nyungan}
\define@key{fams}{wca}{Yanomamic}
\define@key{fams}{yns}{Atlantic-Congo}
\define@key{fams}{jao}{Pama-Nyungan}
\define@key{fams}{yao}{Atlantic-Congo}
\define@key{fams}{yap}{Austronesian}
\define@key{fams}{jaq}{Anim}
\define@key{fams}{yaq}{Uto-Aztecan}
\define@key{fams}{yrb}{Yareban}
\define@key{fams}{yae}{Isolate}
\define@key{fams}{yuf}{Cochimi-Yuman}
\define@key{fams}{yva}{Yawa-Saweru}
\define@key{fams}{ywr}{Nyulnyulan}
\define@key{fams}{pcc}{Tai-Kadai}
\define@key{fams}{xya}{Pama-Nyungan}
\define@key{fams}{yah}{Indo-European}
\define@key{fams}{kpv}{Uralic}
\define@key{fams}{jei}{Yam}
\define@key{fams}{jel}{Bulaka River}
\define@key{fams}{yle}{Isolate}
\define@key{fams}{ybb}{Atlantic-Congo}
\define@key{fams}{jnj}{Ta-Ne-Omotic}
\define@key{fams}{yss}{Sepik}
\define@key{fams}{yey}{Atlantic-Congo}
\define@key{fams}{ywq}{Sino-Tibetan}
\define@key{fams}{ydd}{Indo-European}
\define@key{fams}{yii}{Pama-Nyungan}
\define@key{fams}{yll}{Nuclear Torricelli}
\define@key{fams}{yee}{Lower Sepik-Ramu}
\define@key{fams}{yij}{Pama-Nyungan}
\define@key{fams}{yia}{Pama-Nyungan}
\define@key{fams}{yyr}{nan}
\define@key{fams}{xyy}{Pama-Nyungan}
\define@key{fams}{yor}{Atlantic-Congo}
\define@key{fams}{yua}{Mayan}
\define@key{fams}{yuc}{Isolate}
\define@key{fams}{ycn}{Arawakan}
\define@key{fams}{yug}{Yeniseian}
\define@key{fams}{yux}{Yukaghir}
\define@key{fams}{ykg}{Yukaghir}
\define@key{fams}{yuk}{Yuki-Wappo}
\define@key{fams}{yup}{Cariban}
\define@key{fams}{gcd}{Tangkic}
\define@key{fams}{mpj}{Pama-Nyungan}
\define@key{fams}{yul}{Central Sudanic}
\define@key{fams}{esu}{Eskimo-Aleut}
\define@key{fams}{ynk}{Eskimo-Aleut}
\define@key{fams}{ess}{Eskimo-Aleut}
\define@key{fams}{ysr}{Eskimo-Aleut}
\define@key{fams}{yuz}{Isolate}
\define@key{fams}{yur}{Algic}
\define@key{fams}{yui}{Tucanoan}
\define@key{fams}{zne}{Atlantic-Congo}
\define@key{fams}{zro}{Zaparoan}
\define@key{fams}{zai}{Otomanguean}
\define@key{fams}{zpd}{Otomanguean}
\define@key{fams}{zaa}{Otomanguean}
\define@key{fams}{zaw}{Otomanguean}
\define@key{fams}{zpm}{Otomanguean}
\define@key{fams}{zpi}{Otomanguean}
\define@key{fams}{zab}{Otomanguean}
\define@key{fams}{zpz}{Otomanguean}
\define@key{fams}{zav}{Otomanguean}
\define@key{fams}{zpq}{Otomanguean}
\define@key{fams}{dje}{Songhay}
\define@key{fams}{zay}{Ta-Ne-Omotic}
\define@key{fams}{diq}{Indo-European}
\define@key{fams}{zen}{Afro-Asiatic}
\define@key{fams}{zgb}{Tai-Kadai}
\define@key{fams}{zik}{Anim}
\define@key{fams}{zoh}{Mixe-Zoque}
\define@key{fams}{zos}{Mixe-Zoque}
\define@key{fams}{zoc}{Mixe-Zoque}
\define@key{fams}{zor}{Mixe-Zoque}
\define@key{fams}{zul}{Atlantic-Congo}
\define@key{fams}{zun}{Isolate}
\define@key{fams}{eme}{Tupian}
\define@key{fams}{aom}{Koiarian}
\define@key{fams}{aas}{Afro-Asiatic}
\define@key{fams}{kbt}{Austronesian}
\define@key{fams}{abg}{Nuclear Trans New Guinea}
\define@key{fams}{abf}{Austronesian}
\define@key{fams}{abm}{Atlantic-Congo}
\define@key{fams}{mij}{Atlantic-Congo}
\define@key{fams}{aba}{Atlantic-Congo}
\define@key{fams}{abp}{Austronesian}
\define@key{fams}{bsa}{Isolate}
\define@key{fams}{ash}{Isolate}
\define@key{fams}{aob}{Anim}
\define@key{fams}{abo}{Atlantic-Congo}
\define@key{fams}{abr}{Atlantic-Congo}
\define@key{fams}{abn}{Atlantic-Congo}
\define@key{fams}{abu}{Atlantic-Congo}
\define@key{fams}{mgj}{Atlantic-Congo}
\define@key{fams}{ado}{Lower Sepik-Ramu}
\define@key{fams}{tpx}{Otomanguean}
\define@key{fams}{yif}{Sino-Tibetan}
\define@key{fams}{acz}{Narrow Talodi}
\define@key{fams}{acs}{Nuclear-Macro-Je}
\define@key{fams}{xad}{Isolate}
\define@key{fams}{ada}{Atlantic-Congo}
\define@key{fams}{adq}{Atlantic-Congo}
\define@key{fams}{tiu}{Austronesian}
\define@key{fams}{ade}{Atlantic-Congo}
\define@key{fams}{adh}{Nilotic}
\define@key{fams}{gas}{Indo-European}
\define@key{fams}{adr}{Austronesian}
\define@key{fams}{aez}{Nuclear Trans New Guinea}
\define@key{fams}{aeq}{Indo-European}
\define@key{fams}{afg}{Sign Language}
\define@key{fams}{aft}{Nyimang}
\define@key{fams}{afh}{Artificial Language}
\define@key{fams}{afs}{Indo-European}
\define@key{fams}{agi}{Unattested}
\define@key{fams}{agc}{Atlantic-Congo}
\define@key{fams}{avo}{Unattested}
\define@key{fams}{ggr}{Pama-Nyungan}
\define@key{fams}{xag}{Nakh-Daghestanian}
\define@key{fams}{aif}{Nuclear Torricelli}
\define@key{fams}{kit}{Pahoturi}
\define@key{fams}{ibm}{Atlantic-Congo}
\define@key{fams}{apf}{Austronesian}
\define@key{fams}{aga}{Unattested}
\define@key{fams}{aug}{Atlantic-Congo}
\define@key{fams}{msm}{Austronesian}
\define@key{fams}{agn}{Austronesian}
\define@key{fams}{yay}{Atlantic-Congo}
\define@key{fams}{aha}{Atlantic-Congo}
\define@key{fams}{ahn}{Atlantic-Congo}
\define@key{fams}{esg}{Dravidian}
\define@key{fams}{thm}{Austroasiatic}
\define@key{fams}{kak}{Austronesian}
\define@key{fams}{aho}{Tai-Kadai}
\define@key{fams}{nfd}{Atlantic-Congo}
\define@key{fams}{aih}{Tai-Kadai}
\define@key{fams}{aix}{Austronesian}
\define@key{fams}{mwg}{Austronesian}
\define@key{fams}{aiq}{Indo-European}
\define@key{fams}{ail}{Bosavi}
\define@key{fams}{aim}{Sino-Tibetan}
\define@key{fams}{aic}{Border}
\define@key{fams}{aki}{Lower Sepik-Ramu}
\define@key{fams}{air}{Greater Kwerba}
\define@key{fams}{aio}{Tai-Kadai}
\define@key{fams}{ajw}{Afro-Asiatic}
\define@key{fams}{cpc}{Arawakan}
\define@key{fams}{soh}{Eastern Jebel}
\define@key{fams}{akm}{Great Andamanese}
\define@key{fams}{akj}{Great Andamanese}
\define@key{fams}{ack}{Great Andamanese}
\define@key{fams}{aky}{Great Andamanese}
\define@key{fams}{acl}{Great Andamanese}
\define@key{fams}{aks}{Atlantic-Congo}
\define@key{fams}{aik}{Atlantic-Congo}
\define@key{fams}{tsr}{Austronesian}
\define@key{fams}{aeu}{Sino-Tibetan}
\define@key{fams}{sia}{Uralic}
\define@key{fams}{akk}{Afro-Asiatic}
\define@key{fams}{akq}{Sepik}
\define@key{fams}{akt}{Austronesian}
\define@key{fams}{bss}{Atlantic-Congo}
\define@key{fams}{miw}{Angan}
\define@key{fams}{akf}{Atlantic-Congo}
\define@key{fams}{ibe}{Atlantic-Congo}
\define@key{fams}{afi}{Lower Sepik-Ramu}
\define@key{fams}{ayk}{Atlantic-Congo}
\define@key{fams}{aku}{Atlantic-Congo}
\define@key{fams}{aqz}{Tupian}
\define@key{fams}{ako}{Cariban}
\define@key{fams}{dul}{Austronesian}
\define@key{fams}{alw}{Afro-Asiatic}
\define@key{fams}{ala}{Atlantic-Congo}
\define@key{fams}{alk}{Austroasiatic}
\define@key{fams}{alj}{Austronesian}
\define@key{fams}{apv}{Unattested}
\define@key{fams}{bhk}{Austronesian}
\define@key{fams}{sqk}{Sign Language}
\define@key{fams}{lsc}{Sign Language}
\define@key{fams}{xta}{Otomanguean}
\define@key{fams}{alf}{Atlantic-Congo}
\define@key{fams}{asp}{Sign Language}
\define@key{fams}{arq}{Afro-Asiatic}
\define@key{fams}{aao}{Afro-Asiatic}
\define@key{fams}{aiy}{Atlantic-Congo}
\define@key{fams}{all}{Dravidian}
\define@key{fams}{aid}{Pama-Nyungan}
\define@key{fams}{zaq}{Otomanguean}
\define@key{fams}{ypo}{Sino-Tibetan}
\define@key{fams}{aol}{Austronesian}
\define@key{fams}{syy}{Sign Language}
\define@key{fams}{aub}{Sino-Tibetan}
\define@key{fams}{xua}{Dravidian}
\define@key{fams}{aab}{Atlantic-Congo}
\define@key{fams}{yna}{Sino-Tibetan}
\define@key{fams}{alz}{Nilotic}
\define@key{fams}{avd}{Indo-European}
\define@key{fams}{amq}{Austronesian}
\define@key{fams}{ali}{Nuclear Trans New Guinea}
\define@key{fams}{aad}{Sepik}
\define@key{fams}{jks}{Sign Language}
\define@key{fams}{ama}{Tupian}
\define@key{fams}{amg}{Iwaidjan Proper}
\define@key{fams}{aaz}{Austronesian}
\define@key{fams}{zpo}{Otomanguean}
\define@key{fams}{rwm}{Atlantic-Congo}
\define@key{fams}{utp}{Austronesian}
\define@key{fams}{abc}{Austronesian}
\define@key{fams}{aew}{Keram}
\define@key{fams}{ael}{Atlantic-Congo}
\define@key{fams}{amv}{Austronesian}
\define@key{fams}{alm}{Austronesian}
\define@key{fams}{amb}{Atlantic-Congo}
\define@key{fams}{abs}{Austronesian}
\define@key{fams}{qva}{Quechuan}
\define@key{fams}{aag}{Nuclear Torricelli}
\define@key{fams}{amj}{Furan}
\define@key{fams}{ifa}{Austronesian}
\define@key{fams}{alx}{Nuclear Torricelli}
\define@key{fams}{mbz}{Otomanguean}
\define@key{fams}{aqd}{Dogon}
\define@key{fams}{apg}{Austronesian}
\define@key{fams}{ajz}{Sino-Tibetan}
\define@key{fams}{amt}{Amto-Musan}
\define@key{fams}{adw}{Tupian}
\define@key{fams}{anw}{Atlantic-Congo}
\define@key{fams}{akg}{Austronesian}
\define@key{fams}{anm}{Sino-Tibetan}
\define@key{fams}{pda}{Nuclear Trans New Guinea}
\define@key{fams}{aan}{Tupian}
\define@key{fams}{dti}{Dogon}
\define@key{fams}{grc}{Indo-European}
\define@key{fams}{hbo}{Afro-Asiatic}
\define@key{fams}{xna}{Afro-Asiatic}
\define@key{fams}{xlg}{Unclassifiable}
\define@key{fams}{hca}{Indo-European}
\define@key{fams}{afd}{Arafundi}
\define@key{fams}{aod}{Lower Sepik-Ramu}
\define@key{fams}{ana}{Isolate}
\define@key{fams}{xaa}{Afro-Asiatic}
\define@key{fams}{adg}{Pama-Nyungan}
\define@key{fams}{bzb}{Austronesian}
\define@key{fams}{anb}{Zaparoan}
\define@key{fams}{anx}{Austronesian}
\define@key{fams}{aby}{Yareban}
\define@key{fams}{myo}{Ta-Ne-Omotic}
\define@key{fams}{akh}{Nuclear Trans New Guinea}
\define@key{fams}{age}{Nuclear Trans New Guinea}
\define@key{fams}{aoe}{Nuclear Trans New Guinea}
\define@key{fams}{aqt}{Lengua-Mascoy}
\define@key{fams}{avm}{Pama-Nyungan}
\define@key{fams}{anp}{Indo-European}
\define@key{fams}{rme}{Indo-European}
\define@key{fams}{aog}{Lower Sepik-Ramu}
\define@key{fams}{tnd}{Chibchan}
\define@key{fams}{blo}{Atlantic-Congo}
\define@key{fams}{anf}{Atlantic-Congo}
\define@key{fams}{aqk}{Atlantic-Congo}
\define@key{fams}{ypn}{Sino-Tibetan}
\define@key{fams}{boj}{Nuclear Trans New Guinea}
\define@key{fams}{aak}{Angan}
\define@key{fams}{amx}{Pama-Nyungan}
\define@key{fams}{anj}{Lower Sepik-Ramu}
\define@key{fams}{ans}{Chocoan}
\define@key{fams}{and}{Austronesian}
\define@key{fams}{ant}{Pama-Nyungan}
\define@key{fams}{xmv}{Austronesian}
\define@key{fams}{aig}{Indo-European}
\define@key{fams}{aui}{Austronesian}
\define@key{fams}{auq}{Austronesian}
\define@key{fams}{aud}{Austronesian}
\define@key{fams}{anl}{Sino-Tibetan}
\define@key{fams}{mtb}{Atlantic-Congo}
\define@key{fams}{pni}{Austronesian}
\define@key{fams}{aor}{Austronesian}
\define@key{fams}{aou}{Tai-Kadai}
\define@key{fams}{xap}{Muskogean}
\define@key{fams}{apo}{Austronesian}
\define@key{fams}{ena}{Nuclear Trans New Guinea}
\define@key{fams}{mip}{Otomanguean}
\define@key{fams}{api}{Tupian}
\define@key{fams}{app}{Austronesian}
\define@key{fams}{apx}{Austronesian}
\define@key{fams}{arg}{Indo-European}
\define@key{fams}{stk}{Yam}
\define@key{fams}{aaf}{Dravidian}
\define@key{fams}{xrt}{Unclassifiable}
\define@key{fams}{arj}{Tucanoan}
\define@key{fams}{awm}{Nuclear Trans New Guinea}
\define@key{fams}{awt}{Tupian}
\define@key{fams}{aae}{Indo-European}
\define@key{fams}{aea}{Pama-Nyungan}
\define@key{fams}{mwc}{Austronesian}
\define@key{fams}{aem}{Austroasiatic}
\define@key{fams}{qxu}{Quechuan}
\define@key{fams}{agj}{Afro-Asiatic}
\define@key{fams}{agf}{Austronesian}
\define@key{fams}{aqr}{Austronesian}
\define@key{fams}{aok}{Austronesian}
\define@key{fams}{ylu}{Austronesian}
\define@key{fams}{aai}{Austronesian}
\define@key{fams}{aqg}{Atlantic-Congo}
\define@key{fams}{aac}{Suki-Gogodala}
\define@key{fams}{ait}{Tupian}
\define@key{fams}{ark}{Nuclear-Macro-Je}
\define@key{fams}{xrn}{Yeniseian}
\define@key{fams}{luc}{Central Sudanic}
\define@key{fams}{dth}{Pama-Nyungan}
\define@key{fams}{aoh}{Unattested}
\define@key{fams}{aen}{Sign Language}
\define@key{fams}{rup}{Indo-European}
\define@key{fams}{aps}{Austronesian}
\define@key{fams}{atz}{Austronesian}
\define@key{fams}{arx}{Tupian}
\define@key{fams}{aru}{Arawan}
\define@key{fams}{aur}{Nuclear Torricelli}
\define@key{fams}{lsr}{Nuclear Torricelli}
\define@key{fams}{atx}{Isolate}
\define@key{fams}{aat}{Indo-European}
\define@key{fams}{mtv}{Nuclear Trans New Guinea}
\define@key{fams}{cni}{Arawakan}
\define@key{fams}{ahs}{Atlantic-Congo}
\define@key{fams}{prq}{Arawakan}
\define@key{fams}{ask}{Indo-European}
\define@key{fams}{atn}{Indo-European}
\define@key{fams}{asl}{Austronesian}
\define@key{fams}{eiv}{North Bougainville}
\define@key{fams}{asv}{Central Sudanic}
\define@key{fams}{asb}{Siouan}
\define@key{fams}{asz}{Austronesian}
\define@key{fams}{aua}{Austronesian}
\define@key{fams}{aum}{Atlantic-Congo}
\define@key{fams}{zoo}{Otomanguean}
\define@key{fams}{asr}{Austroasiatic}
\define@key{fams}{atm}{Austronesian}
\define@key{fams}{amz}{Pama-Nyungan}
\define@key{fams}{atd}{Austronesian}
\define@key{fams}{ate}{Nuclear Trans New Guinea}
\define@key{fams}{atk}{Austronesian}
\define@key{fams}{aqm}{Kayagaric}
\define@key{fams}{aot}{Sino-Tibetan}
\define@key{fams}{ato}{Atlantic-Congo}
\define@key{fams}{aox}{Arawakan}
\define@key{fams}{cch}{Atlantic-Congo}
\define@key{fams}{atc}{Pano-Tacanan}
\define@key{fams}{pkr}{Dravidian}
\define@key{fams}{ati}{Atlantic-Congo}
\define@key{fams}{kud}{Austronesian}
\define@key{fams}{aux}{Tupian}
\define@key{fams}{auh}{Atlantic-Congo}
\define@key{fams}{avs}{Zaparoan}
\define@key{fams}{asq}{Sign Language}
\define@key{fams}{asw}{Sign Language}
\define@key{fams}{aut}{Austronesian}
\define@key{fams}{smf}{Border}
\define@key{fams}{auu}{Nuclear Trans New Guinea}
\define@key{fams}{auo}{Afro-Asiatic}
\define@key{fams}{avv}{Tupian}
\define@key{fams}{avb}{Austronesian}
\define@key{fams}{ave}{Indo-European}
\define@key{fams}{awk}{Pama-Nyungan}
\define@key{fams}{vwa}{Austroasiatic}
\define@key{fams}{bcu}{Austronesian}
\define@key{fams}{awo}{Atlantic-Congo}
\define@key{fams}{awx}{Nuclear Trans New Guinea}
\define@key{fams}{aya}{Lower Sepik-Ramu}
\define@key{fams}{awh}{Bayono-Awbono}
\define@key{fams}{bob}{Afro-Asiatic}
\define@key{fams}{awr}{Lakes Plain}
\define@key{fams}{awe}{Tupian}
\define@key{fams}{azo}{Atlantic-Congo}
\define@key{fams}{auj}{Afro-Asiatic}
\define@key{fams}{aww}{Sepik}
\define@key{fams}{afu}{Atlantic-Congo}
\define@key{fams}{yiu}{Sino-Tibetan}
\define@key{fams}{ahb}{Austronesian}
\define@key{fams}{yix}{Sino-Tibetan}
\define@key{fams}{ayd}{Pama-Nyungan}
\define@key{fams}{vmy}{Otomanguean}
\define@key{fams}{aye}{Atlantic-Congo}
\define@key{fams}{ayq}{Sepik}
\define@key{fams}{yyz}{Sino-Tibetan}
\define@key{fams}{ayb}{Atlantic-Congo}
\define@key{fams}{zaf}{Otomanguean}
\define@key{fams}{ayu}{Atlantic-Congo}
\define@key{fams}{aza}{Sino-Tibetan}
\define@key{fams}{yiz}{Sino-Tibetan}
\define@key{fams}{tpc}{Otomanguean}
\define@key{fams}{bvj}{Atlantic-Congo}
\define@key{fams}{bqx}{Atlantic-Congo}
\define@key{fams}{bbm}{Atlantic-Congo}
\define@key{fams}{bbw}{Atlantic-Congo}
\define@key{fams}{bbk}{Atlantic-Congo}
\define@key{fams}{mbf}{Austronesian}
\define@key{fams}{bcr}{Athabaskan-Eyak-Tlingit}
\define@key{fams}{bzg}{Austronesian}
\define@key{fams}{btj}{Austronesian}
\define@key{fams}{bcy}{Afro-Asiatic}
\define@key{fams}{xbc}{Indo-European}
\define@key{fams}{bau}{Atlantic-Congo}
\define@key{fams}{bhz}{Austronesian}
\define@key{fams}{bdz}{Unattested}
\define@key{fams}{jbi}{Pama-Nyungan}
\define@key{fams}{bac}{Austronesian}
\define@key{fams}{pbp}{Atlantic-Congo}
\define@key{fams}{bvd}{Austronesian}
\define@key{fams}{bvc}{Austronesian}
\define@key{fams}{btr}{Austronesian}
\define@key{fams}{bwt}{Atlantic-Congo}
\define@key{fams}{bfj}{Atlantic-Congo}
\define@key{fams}{bmd}{Atlantic-Congo}
\define@key{fams}{bgo}{Atlantic-Congo}
\define@key{fams}{bcg}{Atlantic-Congo}
\define@key{fams}{bfy}{Indo-European}
\define@key{fams}{fui}{Atlantic-Congo}
\define@key{fams}{bqg}{Atlantic-Congo}
\define@key{fams}{bqb}{Greater Kwerba}
\define@key{fams}{bpi}{Nuclear Trans New Guinea}
\define@key{fams}{yha}{Tai-Kadai}
\define@key{fams}{bhv}{Austronesian}
\define@key{fams}{bah}{Indo-European}
\define@key{fams}{bhj}{Sino-Tibetan}
\define@key{fams}{bsu}{Austronesian}
\define@key{fams}{bbf}{Baibai-Fas}
\define@key{fams}{bdj}{Atlantic-Congo}
\define@key{fams}{bkx}{Austronesian}
\define@key{fams}{bqh}{Sino-Tibetan}
\define@key{fams}{bmx}{Nuclear Trans New Guinea}
\define@key{fams}{bab}{Atlantic-Congo}
\define@key{fams}{bcz}{Atlantic-Congo}
\define@key{fams}{fah}{Atlantic-Congo}
\define@key{fams}{bjs}{Indo-European}
\define@key{fams}{bjm}{Indo-European}
\define@key{fams}{bqz}{Atlantic-Congo}
\define@key{fams}{bqi}{Indo-European}
\define@key{fams}{bki}{Austronesian}
\define@key{fams}{bkh}{Atlantic-Congo}
\define@key{fams}{kme}{Atlantic-Congo}
\define@key{fams}{bbs}{Atlantic-Congo}
\define@key{fams}{bkr}{Austronesian}
\define@key{fams}{bjw}{Kru}
\define@key{fams}{ble}{Atlantic-Congo}
\define@key{fams}{bjt}{Atlantic-Congo}
\define@key{fams}{bls}{Austronesian}
\define@key{fams}{bdn}{Afro-Asiatic}
\define@key{fams}{bcn}{Atlantic-Congo}
\define@key{fams}{bcp}{Atlantic-Congo}
\define@key{fams}{mhp}{Austronesian}
\define@key{fams}{bgx}{Turkic}
\define@key{fams}{biz}{Atlantic-Congo}
\define@key{fams}{bqo}{Atlantic-Congo}
\define@key{fams}{blq}{Austronesian}
\define@key{fams}{bog}{Sign Language}
\define@key{fams}{bbq}{Atlantic-Congo}
\define@key{fams}{myf}{Blue Nile Mao}
\define@key{fams}{bmo}{Atlantic-Congo}
\define@key{fams}{bce}{Atlantic-Congo}
\define@key{fams}{bqt}{Atlantic-Congo}
\define@key{fams}{bvm}{Atlantic-Congo}
\define@key{fams}{bcf}{Kiwaian}
\define@key{fams}{bmg}{Atlantic-Congo}
\define@key{fams}{bjx}{Austronesian}
\define@key{fams}{byz}{Lower Sepik-Ramu}
\define@key{fams}{bqj}{Atlantic-Congo}
\define@key{fams}{bqk}{Atlantic-Congo}
\define@key{fams}{bpd}{Atlantic-Congo}
\define@key{fams}{bfl}{Atlantic-Congo}
\define@key{fams}{yaj}{Atlantic-Congo}
\define@key{fams}{bpq}{Austronesian}
\define@key{fams}{bnd}{Austronesian}
\define@key{fams}{bbe}{Atlantic-Congo}
\define@key{fams}{bgf}{Atlantic-Congo}
\define@key{fams}{bsj}{Atlantic-Congo}
\define@key{fams}{bnx}{Atlantic-Congo}
\define@key{fams}{bxg}{Atlantic-Congo}
\define@key{fams}{bgj}{Atlantic-Congo}
\define@key{fams}{mfb}{Austronesian}
\define@key{fams}{bjn}{Austronesian}
\define@key{fams}{bfk}{Sign Language}
\define@key{fams}{bxw}{Mande}
\define@key{fams}{dbw}{Dogon}
\define@key{fams}{bap}{Sino-Tibetan}
\define@key{fams}{bno}{Austronesian}
\define@key{fams}{bfx}{Austronesian}
\define@key{fams}{brd}{Sino-Tibetan}
\define@key{fams}{bbg}{Atlantic-Congo}
\define@key{fams}{baj}{Austronesian}
\define@key{fams}{bhr}{Austronesian}
\define@key{fams}{brs}{Austronesian}
\define@key{fams}{brp}{Geelvink Bay}
\define@key{fams}{bmz}{Anim}
\define@key{fams}{bpb}{Unattested}
\define@key{fams}{gry}{Kru}
\define@key{fams}{bva}{Afro-Asiatic}
\define@key{fams}{bxo}{Pidgin}
\define@key{fams}{bch}{Austronesian}
\define@key{fams}{bjc}{Yareban}
\define@key{fams}{jbk}{Turama-Kikori}
\define@key{fams}{bbi}{Atlantic-Congo}
\define@key{fams}{bjk}{Austronesian}
\define@key{fams}{bpt}{Pama-Nyungan}
\define@key{fams}{tbn}{Chibchan}
\define@key{fams}{bjz}{Nuclear Trans New Guinea}
\define@key{fams}{bwg}{Atlantic-Congo}
\define@key{fams}{bjf}{Afro-Asiatic}
\define@key{fams}{bsl}{Atlantic-Congo}
\define@key{fams}{buj}{Atlantic-Congo}
\define@key{fams}{bzw}{Atlantic-Congo}
\define@key{fams}{bdb}{Austronesian}
\define@key{fams}{byq}{Austronesian}
\define@key{fams}{bsg}{Indo-European}
\define@key{fams}{bst}{Ta-Ne-Omotic}
\define@key{fams}{bsr}{Atlantic-Congo}
\define@key{fams}{bsi}{Atlantic-Congo}
\define@key{fams}{bnm}{Atlantic-Congo}
\define@key{fams}{bts}{Austronesian}
\define@key{fams}{akb}{Austronesian}
\define@key{fams}{btm}{Austronesian}
\define@key{fams}{btd}{Austronesian}
\define@key{fams}{ayt}{Austronesian}
\define@key{fams}{bta}{Afro-Asiatic}
\define@key{fams}{btv}{Indo-European}
\define@key{fams}{btq}{Austroasiatic}
\define@key{fams}{btc}{Atlantic-Congo}
\define@key{fams}{bvt}{Austronesian}
\define@key{fams}{btu}{Atlantic-Congo}
\define@key{fams}{bay}{Austronesian}
\define@key{fams}{zbt}{Austronesian}
\define@key{fams}{sne}{Austronesian}
\define@key{fams}{bsf}{Atlantic-Congo}
\define@key{fams}{bge}{Indo-European}
\define@key{fams}{bxa}{Austronesian}
\define@key{fams}{bwk}{Mailuan}
\define@key{fams}{bjy}{Pama-Nyungan}
\define@key{fams}{bvy}{Austronesian}
\define@key{fams}{byg}{Dajuic}
\define@key{fams}{mkq}{Miwok-Costanoan}
\define@key{fams}{bda}{Atlantic-Congo}
\define@key{fams}{byl}{Bayono-Awbono}
\define@key{fams}{bfr}{Unclassifiable}
\define@key{fams}{beo}{Bosavi}
\define@key{fams}{bea}{Athabaskan-Eyak-Tlingit}
\define@key{fams}{bfp}{Atlantic-Congo}
\define@key{fams}{beb}{Atlantic-Congo}
\define@key{fams}{bzv}{Atlantic-Congo}
\define@key{fams}{bek}{Austronesian}
\define@key{fams}{bxp}{Atlantic-Congo}
\define@key{fams}{tnr}{Atlantic-Congo}
\define@key{fams}{bjv}{Central Sudanic}
\define@key{fams}{bed}{Austronesian}
\define@key{fams}{bkf}{Atlantic-Congo}
\define@key{fams}{bxq}{Afro-Asiatic}
\define@key{fams}{bnz}{Atlantic-Congo}
\define@key{fams}{bby}{Atlantic-Congo}
\define@key{fams}{bqv}{Atlantic-Congo}
\define@key{fams}{bei}{Austronesian}
\define@key{fams}{bkv}{Atlantic-Congo}
\define@key{fams}{bkw}{Atlantic-Congo}
\define@key{fams}{bvi}{Atlantic-Congo}
\define@key{fams}{bxb}{Nilotic}
\define@key{fams}{beg}{Austronesian}
\define@key{fams}{blm}{Central Sudanic}
\define@key{fams}{bey}{Nuclear Torricelli}
\define@key{fams}{bzj}{Indo-European}
\define@key{fams}{brw}{Dravidian}
\define@key{fams}{glb}{Afro-Asiatic}
\define@key{fams}{bmb}{Atlantic-Congo}
\define@key{fams}{yun}{Atlantic-Congo}
\define@key{fams}{bez}{Atlantic-Congo}
\define@key{fams}{bdp}{Atlantic-Congo}
\define@key{fams}{bct}{Central Sudanic}
\define@key{fams}{bgy}{Austronesian}
\define@key{fams}{bnu}{Austronesian}
\define@key{fams}{dbt}{Dogon}
\define@key{fams}{byd}{Austronesian}
\define@key{fams}{bie}{Nuclear Trans New Guinea}
\define@key{fams}{bxv}{Central Sudanic}
\define@key{fams}{bve}{Austronesian}
\define@key{fams}{bit}{Sepik}
\define@key{fams}{byt}{Saharan}
\define@key{fams}{bes}{Atlantic-Congo}
\define@key{fams}{bep}{Austronesian}
\define@key{fams}{bfe}{Tor-Orya}
\define@key{fams}{byf}{Atlantic-Congo}
\define@key{fams}{btt}{Atlantic-Congo}
\define@key{fams}{eot}{Atlantic-Congo}
\define@key{fams}{bhd}{Indo-European}
\define@key{fams}{bha}{Indo-European}
\define@key{fams}{bht}{Indo-European}
\define@key{fams}{bgw}{Indo-European}
\define@key{fams}{bhe}{Indo-European}
\define@key{fams}{bhy}{Atlantic-Congo}
\define@key{fams}{bhi}{Indo-European}
\define@key{fams}{nes}{Sino-Tibetan}
\define@key{fams}{bhu}{Indo-European}
\define@key{fams}{bdf}{Koiarian}
\define@key{fams}{beh}{Atlantic-Congo}
\define@key{fams}{bpv}{Anim}
\define@key{fams}{big}{Goilalan}
\define@key{fams}{byk}{Tai-Kadai}
\define@key{fams}{bje}{Hmong-Mien}
\define@key{fams}{bmt}{Hmong-Mien}
\define@key{fams}{bym}{Pama-Nyungan}
\define@key{fams}{bjg}{Atlantic-Congo}
\define@key{fams}{bmc}{Austronesian}
\define@key{fams}{bnk}{Austronesian}
\define@key{fams}{brj}{Austronesian}
\define@key{fams}{biu}{Sino-Tibetan}
\define@key{fams}{xbe}{Pama-Nyungan}
\define@key{fams}{bhc}{Austronesian}
\define@key{fams}{ibh}{Austronesian}
\define@key{fams}{jbm}{Atlantic-Congo}
\define@key{fams}{bix}{Austroasiatic}
\define@key{fams}{byb}{Atlantic-Congo}
\define@key{fams}{kfs}{Indo-European}
\define@key{fams}{bql}{Nuclear Trans New Guinea}
\define@key{fams}{brz}{Austronesian}
\define@key{fams}{bpz}{Austronesian}
\define@key{fams}{bil}{Atlantic-Congo}
\define@key{fams}{bms}{Saharan}
\define@key{fams}{bxf}{Austronesian}
\define@key{fams}{bhl}{Nuclear Trans New Guinea}
\define@key{fams}{byj}{Atlantic-Congo}
\define@key{fams}{bmn}{Austronesian}
\define@key{fams}{bxz}{Mailuan}
\define@key{fams}{bon}{Eastern Trans-Fly}
\define@key{fams}{bpj}{Atlantic-Congo}
\define@key{fams}{itb}{Austronesian}
\define@key{fams}{bne}{Austronesian}
\define@key{fams}{bny}{Austronesian}
\define@key{fams}{biq}{Austronesian}
\define@key{fams}{bxe}{Isolate}
\define@key{fams}{brr}{Austronesian}
\define@key{fams}{btf}{Afro-Asiatic}
\define@key{fams}{biy}{Austroasiatic}
\define@key{fams}{bqq}{Lakes Plain}
\define@key{fams}{brk}{Nubian}
\define@key{fams}{brl}{Atlantic-Congo}
\define@key{fams}{ije}{Ijoid}
\define@key{fams}{bpy}{Indo-European}
\define@key{fams}{bwh}{Atlantic-Congo}
\define@key{fams}{bnw}{Sepik}
\define@key{fams}{bir}{Nuclear Trans New Guinea}
\define@key{fams}{bzi}{Sino-Tibetan}
\define@key{fams}{brt}{Atlantic-Congo}
\define@key{fams}{bgk}{Austroasiatic}
\define@key{fams}{mcc}{Anim}
\define@key{fams}{bwm}{Yuat}
\define@key{fams}{byo}{Sino-Tibetan}
\define@key{fams}{bpm}{Nuclear Trans New Guinea}
\define@key{fams}{blp}{Austronesian}
\define@key{fams}{bfh}{Yam}
\define@key{fams}{beu}{Timor-Alor-Pantar}
\define@key{fams}{blr}{Austroasiatic}
\define@key{fams}{zbl}{Artificial Language}
\define@key{fams}{bzn}{Austronesian}
\define@key{fams}{bzl}{Austronesian}
\define@key{fams}{bty}{Austronesian}
\define@key{fams}{bgb}{Austronesian}
\define@key{fams}{bdv}{Indo-European}
\define@key{fams}{boy}{Atlantic-Congo}
\define@key{fams}{bff}{Atlantic-Congo}
\define@key{fams}{boq}{Isolate}
\define@key{fams}{bvw}{Afro-Asiatic}
\define@key{fams}{bux}{Afro-Asiatic}
\define@key{fams}{bqu}{Atlantic-Congo}
\define@key{fams}{bhn}{Afro-Asiatic}
\define@key{fams}{ybk}{Sino-Tibetan}
\define@key{fams}{bdt}{Atlantic-Congo}
\define@key{fams}{bkp}{Atlantic-Congo}
\define@key{fams}{bus}{Mande}
\define@key{fams}{bky}{Atlantic-Congo}
\define@key{fams}{bnp}{Austronesian}
\define@key{fams}{bld}{Austronesian}
\define@key{fams}{xbo}{Turkic}
\define@key{fams}{bvo}{Atlantic-Congo}
\define@key{fams}{bvl}{Sign Language}
\define@key{fams}{smk}{Austronesian}
\define@key{fams}{blv}{Atlantic-Congo}
\define@key{fams}{bkt}{Atlantic-Congo}
\define@key{fams}{bzm}{Atlantic-Congo}
\define@key{fams}{bof}{Mande}
\define@key{fams}{blj}{Austronesian}
\define@key{fams}{ply}{Austroasiatic}
\define@key{fams}{boh}{Atlantic-Congo}
\define@key{fams}{bml}{Atlantic-Congo}
\define@key{fams}{bws}{Atlantic-Congo}
\define@key{fams}{zmx}{Atlantic-Congo}
\define@key{fams}{bmf}{Atlantic-Congo}
\define@key{fams}{bmq}{Atlantic-Congo}
\define@key{fams}{bmw}{Atlantic-Congo}
\define@key{fams}{kzc}{Atlantic-Congo}
\define@key{fams}{bou}{Atlantic-Congo}
\define@key{fams}{dbu}{Dogon}
\define@key{fams}{bna}{Austronesian}
\define@key{fams}{bnv}{Tor-Orya}
\define@key{fams}{glc}{Atlantic-Congo}
\define@key{fams}{bui}{Atlantic-Congo}
\define@key{fams}{bpg}{Austronesian}
\define@key{fams}{bok}{Atlantic-Congo}
\define@key{fams}{bvg}{Atlantic-Congo}
\define@key{fams}{bop}{Nuclear Trans New Guinea}
\define@key{fams}{bnb}{Austronesian}
\define@key{fams}{bnl}{Afro-Asiatic}
\define@key{fams}{bvf}{Afro-Asiatic}
\define@key{fams}{bpw}{Left May}
\define@key{fams}{gai}{Lower Sepik-Ramu}
\define@key{fams}{fue}{Atlantic-Congo}
\define@key{fams}{ksr}{Nuclear Trans New Guinea}
\define@key{fams}{xxb}{Atlantic-Congo}
\define@key{fams}{mae}{Atlantic-Congo}
\define@key{fams}{bwf}{Austronesian}
\define@key{fams}{bqs}{Lower Sepik-Ramu}
\define@key{fams}{bmj}{Indo-European}
\define@key{fams}{bph}{Nakh-Daghestanian}
\define@key{fams}{sbl}{Austronesian}
\define@key{fams}{nku}{Atlantic-Congo}
\define@key{fams}{mux}{Nuclear Trans New Guinea}
\define@key{fams}{suo}{Sko}
\define@key{fams}{kxr}{Austronesian}
\define@key{fams}{aof}{Nuclear Torricelli}
\define@key{fams}{bra}{Indo-European}
\define@key{fams}{kvl}{Sino-Tibetan}
\define@key{fams}{buq}{Nuclear Trans New Guinea}
\define@key{fams}{brq}{Lower Sepik-Ramu}
\define@key{fams}{rib}{Sign Language}
\define@key{fams}{bzt}{Artificial Language}
\define@key{fams}{sgt}{Sino-Tibetan}
\define@key{fams}{bro}{Sino-Tibetan}
\define@key{fams}{bpl}{Pidgin}
\define@key{fams}{plw}{Austronesian}
\define@key{fams}{kxd}{Austronesian}
\define@key{fams}{bsb}{Austronesian}
\define@key{fams}{rnb}{Sign Language}
\define@key{fams}{bub}{Atlantic-Congo}
\define@key{fams}{cbl}{Sino-Tibetan}
\define@key{fams}{box}{Atlantic-Congo}
\define@key{fams}{buw}{Atlantic-Congo}
\define@key{fams}{stt}{Austroasiatic}
\define@key{fams}{btp}{Austronesian}
\define@key{fams}{bdx}{Austronesian}
\define@key{fams}{bja}{Atlantic-Congo}
\define@key{fams}{bbh}{Austroasiatic}
\define@key{fams}{buk}{Austronesian}
\define@key{fams}{bgt}{Austronesian}
\define@key{fams}{bku}{Austronesian}
\define@key{fams}{bxh}{Austronesian}
\define@key{fams}{byh}{Sino-Tibetan}
\define@key{fams}{bvk}{Austronesian}
\define@key{fams}{bhh}{Indo-European}
\define@key{fams}{bvu}{Austronesian}
\define@key{fams}{bkn}{Austronesian}
\define@key{fams}{tkb}{Indo-European}
\define@key{fams}{buz}{Atlantic-Congo}
\define@key{fams}{bqn}{Sign Language}
\define@key{fams}{bmp}{Nuclear Trans New Guinea}
\define@key{fams}{buy}{Atlantic-Congo}
\define@key{fams}{sti}{Austroasiatic}
\define@key{fams}{bjl}{Austronesian}
\define@key{fams}{byp}{Atlantic-Congo}
\define@key{fams}{aon}{Nuclear Torricelli}
\define@key{fams}{bmv}{Atlantic-Congo}
\define@key{fams}{kjz}{Sino-Tibetan}
\define@key{fams}{bwx}{Hmong-Mien}
\define@key{fams}{bdd}{Austronesian}
\define@key{fams}{bvn}{Nuclear Torricelli}
\define@key{fams}{bfn}{Timor-Alor-Pantar}
\define@key{fams}{bns}{Indo-European}
\define@key{fams}{bqd}{Atlantic-Congo}
\define@key{fams}{xbg}{Pama-Nyungan}
\define@key{fams}{wun}{Atlantic-Congo}
\define@key{fams}{bkz}{Austronesian}
\define@key{fams}{but}{Nuclear Torricelli}
\define@key{fams}{buv}{Yuat}
\define@key{fams}{dgb}{Dogon}
\define@key{fams}{bnn}{Austronesian}
\define@key{fams}{blf}{Austronesian}
\define@key{fams}{bys}{Atlantic-Congo}
\define@key{fams}{bti}{Geelvink Bay}
\define@key{fams}{bxn}{Pama-Nyungan}
\define@key{fams}{bvh}{Afro-Asiatic}
\define@key{fams}{pyx}{Sino-Tibetan}
\define@key{fams}{vrt}{Austronesian}
\define@key{fams}{bzu}{Isolate}
\define@key{fams}{bqw}{Atlantic-Congo}
\define@key{fams}{bdi}{Nilotic}
\define@key{fams}{bqr}{Austronesian}
\define@key{fams}{aip}{Nuclear Trans New Guinea}
\define@key{fams}{asi}{Nuclear Trans New Guinea}
\define@key{fams}{bry}{Ndu}
\define@key{fams}{bxs}{Atlantic-Congo}
\define@key{fams}{bsm}{Austronesian}
\define@key{fams}{bfg}{Austronesian}
\define@key{fams}{buc}{Austronesian}
\define@key{fams}{bup}{Austronesian}
\define@key{fams}{dox}{Afro-Asiatic}
\define@key{fams}{bju}{Atlantic-Congo}
\define@key{fams}{kyb}{Austronesian}
\define@key{fams}{bnr}{Austronesian}
\define@key{fams}{btw}{Austronesian}
\define@key{fams}{jid}{Atlantic-Congo}
\define@key{fams}{bhs}{Afro-Asiatic}
\define@key{fams}{jiy}{Sino-Tibetan}
\define@key{fams}{byi}{Atlantic-Congo}
\define@key{fams}{bww}{Atlantic-Congo}
\define@key{fams}{bwd}{Austronesian}
\define@key{fams}{tte}{Austronesian}
\define@key{fams}{bwa}{Austronesian}
\define@key{fams}{bwl}{Atlantic-Congo}
\define@key{fams}{bwc}{Atlantic-Congo}
\define@key{fams}{bwz}{Atlantic-Congo}
\define@key{fams}{mkk}{Atlantic-Congo}
\define@key{fams}{msq}{Austronesian}
\define@key{fams}{cbb}{Arawakan}
\define@key{fams}{ccr}{Misumalpan}
\define@key{fams}{miu}{Otomanguean}
\define@key{fams}{roc}{Austronesian}
\define@key{fams}{ccd}{Indo-European}
\define@key{fams}{cah}{Zaparoan}
\define@key{fams}{qvl}{Quechuan}
\define@key{fams}{zad}{Otomanguean}
\define@key{fams}{frc}{Indo-European}
\define@key{fams}{ckx}{Atlantic-Congo}
\define@key{fams}{ckz}{Mixed Language}
\define@key{fams}{cky}{Afro-Asiatic}
\define@key{fams}{tbk}{Austronesian}
\define@key{fams}{qud}{Quechuan}
\define@key{fams}{caw}{Speech Register}
\define@key{fams}{rmq}{Indo-European}
\define@key{fams}{clu}{Austronesian}
\define@key{fams}{abd}{Austronesian}
\define@key{fams}{csx}{Sign Language}
\define@key{fams}{mcu}{Atlantic-Congo}
\define@key{fams}{wes}{Indo-European}
\define@key{fams}{cml}{Austronesian}
\define@key{fams}{cmt}{Speech Register}
\define@key{fams}{xcc}{Unclassifiable}
\define@key{fams}{qxr}{Quechuan}
\define@key{fams}{caz}{Isolate}
\define@key{fams}{mlc}{Tai-Kadai}
\define@key{fams}{cov}{Tai-Kadai}
\define@key{fams}{cps}{Austronesian}
\define@key{fams}{cpg}{Indo-European}
\define@key{fams}{cot}{Arawakan}
\define@key{fams}{cby}{Unclassifiable}
\define@key{fams}{cfd}{Atlantic-Congo}
\define@key{fams}{crf}{Chocoan}
\define@key{fams}{xcr}{Indo-European}
\define@key{fams}{hns}{Indo-European}
\define@key{fams}{jvn}{Austronesian}
\define@key{fams}{crr}{Algic}
\define@key{fams}{rmc}{Indo-European}
\define@key{fams}{asc}{Nuclear Trans New Guinea}
\define@key{fams}{csc}{Sign Language}
\define@key{fams}{xcy}{Isolate}
\define@key{fams}{xce}{Indo-European}
\define@key{fams}{cen}{Atlantic-Congo}
\define@key{fams}{hmm}{Hmong-Mien}
\define@key{fams}{cmo}{Austroasiatic}
\define@key{fams}{zch}{Tai-Kadai}
\define@key{fams}{hmc}{Hmong-Mien}
\define@key{fams}{fuq}{Atlantic-Congo}
\define@key{fams}{grv}{Kru}
\define@key{fams}{cet}{Isolate}
\define@key{fams}{pse}{Austronesian}
\define@key{fams}{mwo}{Austronesian}
\define@key{fams}{mxz}{Austronesian}
\define@key{fams}{syb}{Austronesian}
\define@key{fams}{tgt}{Austronesian}
\define@key{fams}{plc}{Austronesian}
\define@key{fams}{sml}{Austronesian}
\define@key{fams}{zbc}{Austronesian}
\define@key{fams}{dtp}{Austronesian}
\define@key{fams}{awu}{Nuclear Trans New Guinea}
\define@key{fams}{ncx}{Uto-Aztecan}
\define@key{fams}{nch}{Uto-Aztecan}
\define@key{fams}{ojc}{Algic}
\define@key{fams}{pbs}{Otomanguean}
\define@key{fams}{quk}{Quechuan}
\define@key{fams}{cds}{Sign Language}
\define@key{fams}{cdy}{Tai-Kadai}
\define@key{fams}{chg}{Turkic}
\define@key{fams}{ciy}{Cariban}
\define@key{fams}{ccp}{Indo-European}
\define@key{fams}{ckh}{Sino-Tibetan}
\define@key{fams}{cli}{Atlantic-Congo}
\define@key{fams}{tgf}{Sino-Tibetan}
\define@key{fams}{cll}{Atlantic-Congo}
\define@key{fams}{cdh}{Indo-European}
\define@key{fams}{ceg}{Zamucoan}
\define@key{fams}{ccc}{Arawakan}
\define@key{fams}{cna}{Sino-Tibetan}
\define@key{fams}{cga}{Yuat}
\define@key{fams}{cra}{Ta-Ne-Omotic}
\define@key{fams}{crv}{Austroasiatic}
\define@key{fams}{xtb}{Otomanguean}
\define@key{fams}{ruk}{Atlantic-Congo}
\define@key{fams}{cde}{Dravidian}
\define@key{fams}{cjn}{Sepik}
\define@key{fams}{cnu}{Afro-Asiatic}
\define@key{fams}{ycp}{Sino-Tibetan}
\define@key{fams}{cpn}{Atlantic-Congo}
\define@key{fams}{ych}{Sino-Tibetan}
\define@key{fams}{cwg}{Austroasiatic}
\define@key{fams}{hne}{Indo-European}
\define@key{fams}{ctn}{Sino-Tibetan}
\define@key{fams}{cur}{Sino-Tibetan}
\define@key{fams}{csd}{Sign Language}
\define@key{fams}{cip}{Otomanguean}
\define@key{fams}{zpv}{Otomanguean}
\define@key{fams}{mii}{Otomanguean}
\define@key{fams}{csg}{Sign Language}
\define@key{fams}{clh}{Indo-European}
\define@key{fams}{clc}{Athabaskan-Eyak-Tlingit}
\define@key{fams}{csa}{Otomanguean}
\define@key{fams}{cpi}{Pidgin}
\define@key{fams}{chn}{Chinookan}
\define@key{fams}{cih}{Indo-European}
\define@key{fams}{bxu}{Mongolic-Khitan}
\define@key{fams}{cnb}{Sino-Tibetan}
\define@key{fams}{qxc}{Quechuan}
\define@key{fams}{cdf}{Sino-Tibetan}
\define@key{fams}{nhd}{Tupian}
\define@key{fams}{the}{Indo-European}
\define@key{fams}{cik}{Sino-Tibetan}
\define@key{fams}{zpc}{Otomanguean}
\define@key{fams}{cgk}{Sino-Tibetan}
\define@key{fams}{cdi}{Indo-European}
\define@key{fams}{nri}{Sino-Tibetan}
\define@key{fams}{cjk}{Atlantic-Congo}
\define@key{fams}{cda}{Sino-Tibetan}
\define@key{fams}{coh}{Atlantic-Congo}
\define@key{fams}{cce}{Atlantic-Congo}
\define@key{fams}{nct}{Sino-Tibetan}
\define@key{fams}{cvg}{Sino-Tibetan}
\define@key{fams}{cuw}{Sino-Tibetan}
\define@key{fams}{cuh}{Atlantic-Congo}
\define@key{fams}{chu}{Indo-European}
\define@key{fams}{cdj}{Indo-European}
\define@key{fams}{scb}{Austroasiatic}
\define@key{fams}{xcv}{Yukaghir}
\define@key{fams}{chw}{Atlantic-Congo}
\define@key{fams}{cia}{Austronesian}
\define@key{fams}{ckl}{Afro-Asiatic}
\define@key{fams}{awc}{Atlantic-Congo}
\define@key{fams}{cib}{Atlantic-Congo}
\define@key{fams}{cim}{Indo-European}
\define@key{fams}{mkx}{Austronesian}
\define@key{fams}{cdr}{Atlantic-Congo}
\define@key{fams}{cie}{Afro-Asiatic}
\define@key{fams}{cin}{Tupian}
\define@key{fams}{xcg}{Indo-European}
\define@key{fams}{asg}{Atlantic-Congo}
\define@key{fams}{txt}{Nuclear Trans New Guinea}
\define@key{fams}{tgd}{Afro-Asiatic}
\define@key{fams}{xcl}{Indo-European}
\define@key{fams}{nci}{Uto-Aztecan}
\define@key{fams}{qwc}{Quechuan}
\define@key{fams}{syc}{Afro-Asiatic}
\define@key{fams}{myz}{Afro-Asiatic}
\define@key{fams}{xct}{Sino-Tibetan}
\define@key{fams}{dri}{Atlantic-Congo}
\define@key{fams}{naz}{Uto-Aztecan}
\define@key{fams}{zps}{Otomanguean}
\define@key{fams}{zca}{Otomanguean}
\define@key{fams}{coj}{Cochimi-Yuman}
\define@key{fams}{coa}{Austronesian}
\define@key{fams}{liw}{Austronesian}
\define@key{fams}{csn}{Sign Language}
\define@key{fams}{gct}{Indo-European}
\define@key{fams}{cfg}{Atlantic-Congo}
\define@key{fams}{swc}{Atlantic-Congo}
\define@key{fams}{cnc}{Sino-Tibetan}
\define@key{fams}{coq}{Athabaskan-Eyak-Tlingit}
\define@key{fams}{cry}{Atlantic-Congo}
\define@key{fams}{qwa}{Quechuan}
\define@key{fams}{xxr}{Nuclear-Macro-Je}
\define@key{fams}{cos}{Indo-European}
\define@key{fams}{csr}{Sign Language}
\define@key{fams}{mta}{Austronesian}
\define@key{fams}{xcn}{Isolate}
\define@key{fams}{cow}{Salishan}
\define@key{fams}{toc}{Totonacan}
\define@key{fams}{gyn}{Indo-European}
\define@key{fams}{csq}{Sign Language}
\define@key{fams}{mfn}{Atlantic-Congo}
\define@key{fams}{crz}{Chumashan}
\define@key{fams}{csf}{Sign Language}
\define@key{fams}{cbq}{Atlantic-Congo}
\define@key{fams}{cuo}{Cariban}
\define@key{fams}{xlu}{Indo-European}
\define@key{fams}{cnq}{Atlantic-Congo}
\define@key{fams}{cuq}{Tai-Kadai}
\define@key{fams}{ccl}{Atlantic-Congo}
\define@key{fams}{cuv}{Afro-Asiatic}
\define@key{fams}{xtu}{Otomanguean}
\define@key{fams}{cyo}{Austronesian}
\define@key{fams}{bwy}{Atlantic-Congo}
\define@key{fams}{cse}{Sign Language}
\define@key{fams}{dao}{Sino-Tibetan}
\define@key{fams}{lni}{South Bougainville}
\define@key{fams}{dtn}{Gumuz}
\define@key{fams}{dbr}{Afro-Asiatic}
\define@key{fams}{dbe}{Tor-Orya}
\define@key{fams}{xdc}{Indo-European}
\define@key{fams}{dbd}{Atlantic-Congo}
\define@key{fams}{dgd}{Atlantic-Congo}
\define@key{fams}{dgk}{Central Sudanic}
\define@key{fams}{dec}{Narrow Talodi}
\define@key{fams}{dgn}{Yangmanic}
\define@key{fams}{dlk}{Afro-Asiatic}
\define@key{fams}{das}{Kru}
\define@key{fams}{dij}{Austronesian}
\define@key{fams}{drb}{Nubian}
\define@key{fams}{zhd}{Tai-Kadai}
\define@key{fams}{bpa}{Austronesian}
\define@key{fams}{dkk}{Austronesian}
\define@key{fams}{dka}{Sino-Tibetan}
\define@key{fams}{qer}{Indo-European}
\define@key{fams}{dlm}{Indo-European}
\define@key{fams}{dmm}{Atlantic-Congo}
\define@key{fams}{dam}{Atlantic-Congo}
\define@key{fams}{uhn}{Isolate}
\define@key{fams}{idb}{Indo-European}
\define@key{fams}{dac}{Austronesian}
\define@key{fams}{dml}{Indo-European}
\define@key{fams}{dms}{Austronesian}
\define@key{fams}{dnu}{Austroasiatic}
\define@key{fams}{dnr}{Nuclear Trans New Guinea}
\define@key{fams}{daq}{Dravidian}
\define@key{fams}{thl}{Indo-European}
\define@key{fams}{dsl}{Sign Language}
\define@key{fams}{daf}{Mande}
\define@key{fams}{aso}{Nuclear Trans New Guinea}
\define@key{fams}{gku}{Tuu}
\define@key{fams}{dnd}{Border}
\define@key{fams}{daz}{Nuclear Trans New Guinea}
\define@key{fams}{djc}{Dajuic}
\define@key{fams}{dln}{Sino-Tibetan}
\define@key{fams}{dro}{Austronesian}
\define@key{fams}{dot}{Afro-Asiatic}
\define@key{fams}{daw}{Austronesian}
\define@key{fams}{dww}{Austronesian}
\define@key{fams}{ddw}{Austronesian}
\define@key{fams}{dax}{Pama-Nyungan}
\define@key{fams}{dzg}{Saharan}
\define@key{fams}{dzd}{Unattested}
\define@key{fams}{ded}{Nuclear Trans New Guinea}
\define@key{fams}{gbh}{Atlantic-Congo}
\define@key{fams}{dge}{Nuclear Trans New Guinea}
\define@key{fams}{mzw}{Atlantic-Congo}
\define@key{fams}{deh}{Indo-European}
\define@key{fams}{dek}{Unattested}
\define@key{fams}{row}{Austronesian}
\define@key{fams}{ntr}{Atlantic-Congo}
\define@key{fams}{dmx}{Atlantic-Congo}
\define@key{fams}{dei}{Geelvink Bay}
\define@key{fams}{dem}{Isolate}
\define@key{fams}{dmy}{Sentanic}
\define@key{fams}{deq}{Atlantic-Congo}
\define@key{fams}{ddn}{Songhay}
\define@key{fams}{dez}{Atlantic-Congo}
\define@key{fams}{dnk}{Austronesian}
\define@key{fams}{dbb}{Afro-Asiatic}
\define@key{fams}{anv}{Atlantic-Congo}
\define@key{fams}{dee}{Kru}
\define@key{fams}{def}{Indo-European}
\define@key{fams}{dgh}{Afro-Asiatic}
\define@key{fams}{dhs}{Atlantic-Congo}
\define@key{fams}{dhn}{Indo-European}
\define@key{fams}{dwz}{Indo-European}
\define@key{fams}{nfa}{Austronesian}
\define@key{fams}{mki}{Indo-European}
\define@key{fams}{dho}{Indo-European}
\define@key{fams}{adf}{Afro-Asiatic}
\define@key{fams}{ddr}{Pama-Nyungan}
\define@key{fams}{dhd}{Indo-European}
\define@key{fams}{dia}{Nuclear Torricelli}
\define@key{fams}{mbd}{Austronesian}
\define@key{fams}{dby}{Isolate}
\define@key{fams}{dio}{Atlantic-Congo}
\define@key{fams}{duy}{Austronesian}
\define@key{fams}{dig}{Atlantic-Congo}
\define@key{fams}{cfa}{Atlantic-Congo}
\define@key{fams}{dil}{Nubian}
\define@key{fams}{jma}{Dagan}
\define@key{fams}{dii}{Atlantic-Congo}
\define@key{fams}{dmc}{Nuclear Trans New Guinea}
\define@key{fams}{ddi}{Austronesian}
\define@key{fams}{gdl}{Afro-Asiatic}
\define@key{fams}{diu}{Atlantic-Congo}
\define@key{fams}{dir}{Atlantic-Congo}
\define@key{fams}{dwa}{Afro-Asiatic}
\define@key{fams}{dsi}{Central Sudanic}
\define@key{fams}{tbz}{Atlantic-Congo}
\define@key{fams}{diy}{Nuclear Trans New Guinea}
\define@key{fams}{xtd}{Otomanguean}
\define@key{fams}{dix}{Austronesian}
\define@key{fams}{djf}{Pama-Nyungan}
\define@key{fams}{djn}{Gunwinyguan}
\define@key{fams}{djw}{Nyulnyulan}
\define@key{fams}{djb}{Pama-Nyungan}
\define@key{fams}{dze}{Pama-Nyungan}
\define@key{fams}{dob}{Austronesian}
\define@key{fams}{doe}{Atlantic-Congo}
\define@key{fams}{dgg}{Austronesian}
\define@key{fams}{dgx}{Nuclear Trans New Guinea}
\define@key{fams}{dgs}{Atlantic-Congo}
\define@key{fams}{dos}{Atlantic-Congo}
\define@key{fams}{dgr}{Athabaskan-Eyak-Tlingit}
\define@key{fams}{dbg}{Dogon}
\define@key{fams}{dbi}{Atlantic-Congo}
\define@key{fams}{uya}{Atlantic-Congo}
\define@key{fams}{dre}{Sino-Tibetan}
\define@key{fams}{dov}{Atlantic-Congo}
\define@key{fams}{doq}{Sign Language}
\define@key{fams}{doa}{Nuclear Trans New Guinea}
\define@key{fams}{doy}{Atlantic-Congo}
\define@key{fams}{dof}{Mailuan}
\define@key{fams}{dev}{Nuclear Trans New Guinea}
\define@key{fams}{dok}{Austronesian}
\define@key{fams}{yik}{Sino-Tibetan}
\define@key{fams}{doh}{Atlantic-Congo}
\define@key{fams}{ddd}{Nilotic}
\define@key{fams}{dde}{Atlantic-Congo}
\define@key{fams}{dor}{Austronesian}
\define@key{fams}{kqc}{Manubaran}
\define@key{fams}{doz}{Ta-Ne-Omotic}
\define@key{fams}{dol}{Doso-Turumsa}
\define@key{fams}{dty}{Indo-European}
\define@key{fams}{dup}{Austronesian}
\define@key{fams}{dva}{Austronesian}
\define@key{fams}{dub}{Indo-European}
\define@key{fams}{dmu}{Pauwasi}
\define@key{fams}{duk}{Nuclear Trans New Guinea}
\define@key{fams}{ndu}{Atlantic-Congo}
\define@key{fams}{dbm}{Atlantic-Congo}
\define@key{fams}{dme}{Afro-Asiatic}
\define@key{fams}{kbz}{Afro-Asiatic}
\define@key{fams}{nke}{Austronesian}
\define@key{fams}{dbo}{Atlantic-Congo}
\define@key{fams}{duz}{Atlantic-Congo}
\define@key{fams}{dmv}{Austronesian}
\define@key{fams}{wtf}{Nuclear Trans New Guinea}
\define@key{fams}{dui}{Nuclear Trans New Guinea}
\define@key{fams}{duh}{Indo-European}
\define@key{fams}{raa}{Sino-Tibetan}
\define@key{fams}{dng}{Sino-Tibetan}
\define@key{fams}{dbv}{Unattested}
\define@key{fams}{drq}{Sino-Tibetan}
\define@key{fams}{mvp}{Austronesian}
\define@key{fams}{dbn}{Inanwatan}
\define@key{fams}{dug}{Atlantic-Congo}
\define@key{fams}{dsn}{Austronesian}
\define@key{fams}{duw}{Austronesian}
\define@key{fams}{duq}{Austronesian}
\define@key{fams}{dun}{Austronesian}
\define@key{fams}{dws}{Artificial Language}
\define@key{fams}{dux}{Mande}
\define@key{fams}{dae}{Atlantic-Congo}
\define@key{fams}{duv}{Lakes Plain}
\define@key{fams}{dbp}{Afro-Asiatic}
\define@key{fams}{gve}{Austronesian}
\define@key{fams}{nnu}{Atlantic-Congo}
\define@key{fams}{dyb}{Nyulnyulan}
\define@key{fams}{dyn}{Pama-Nyungan}
\define@key{fams}{dya}{Atlantic-Congo}
\define@key{fams}{dyd}{Nyulnyulan}
\define@key{fams}{jen}{Atlantic-Congo}
\define@key{fams}{dzl}{Sino-Tibetan}
\define@key{fams}{dzn}{Atlantic-Congo}
\define@key{fams}{bpn}{Hmong-Mien}
\define@key{fams}{add}{Atlantic-Congo}
\define@key{fams}{dzo}{Sino-Tibetan}
\define@key{fams}{dnn}{Mande}
\define@key{fams}{ktv}{Austroasiatic}
\define@key{fams}{bgp}{Indo-European}
\define@key{fams}{lwl}{Austroasiatic}
\define@key{fams}{mng}{Austroasiatic}
\define@key{fams}{emu}{Dravidian}
\define@key{fams}{tge}{Sino-Tibetan}
\define@key{fams}{nos}{Sino-Tibetan}
\define@key{fams}{emq}{Sino-Tibetan}
\define@key{fams}{kif}{Sino-Tibetan}
\define@key{fams}{emg}{Sino-Tibetan}
\define@key{fams}{zeh}{Tai-Kadai}
\define@key{fams}{hmq}{Hmong-Mien}
\define@key{fams}{muq}{Hmong-Mien}
\define@key{fams}{hme}{Hmong-Mien}
\define@key{fams}{lma}{Atlantic-Congo}
\define@key{fams}{gbx}{Atlantic-Congo}
\define@key{fams}{xrb}{Atlantic-Congo}
\define@key{fams}{acp}{Atlantic-Congo}
\define@key{fams}{nle}{Atlantic-Congo}
\define@key{fams}{kqo}{Kru}
\define@key{fams}{vme}{Austronesian}
\define@key{fams}{tre}{Austronesian}
\define@key{fams}{dmr}{Austronesian}
\define@key{fams}{bnj}{Austronesian}
\define@key{fams}{pez}{Austronesian}
\define@key{fams}{zbe}{Austronesian}
\define@key{fams}{kjs}{Nuclear Trans New Guinea}
\define@key{fams}{nhe}{Uto-Aztecan}
\define@key{fams}{ojg}{Algic}
\define@key{fams}{aaq}{Algic}
\define@key{fams}{qve}{Quechuan}
\define@key{fams}{cly}{Otomanguean}
\define@key{fams}{avl}{Afro-Asiatic}
\define@key{fams}{sfe}{Austronesian}
\define@key{fams}{azd}{Uto-Aztecan}
\define@key{fams}{yit}{Sino-Tibetan}
\define@key{fams}{cek}{Sino-Tibetan}
\define@key{fams}{yol}{Indo-European}
\define@key{fams}{xeb}{Afro-Asiatic}
\define@key{fams}{ebr}{Atlantic-Congo}
\define@key{fams}{ebg}{Atlantic-Congo}
\define@key{fams}{ecs}{Sign Language}
\define@key{fams}{cbj}{Atlantic-Congo}
\define@key{fams}{idd}{Atlantic-Congo}
\define@key{fams}{ijj}{Atlantic-Congo}
\define@key{fams}{ica}{Atlantic-Congo}
\define@key{fams}{nqg}{Atlantic-Congo}
\define@key{fams}{awy}{Nuclear Trans New Guinea}
\define@key{fams}{dbf}{Lakes Plain}
\define@key{fams}{eee}{Tai-Kadai}
\define@key{fams}{efa}{Atlantic-Congo}
\define@key{fams}{efe}{Central Sudanic}
\define@key{fams}{ofu}{Atlantic-Congo}
\define@key{fams}{ego}{Atlantic-Congo}
\define@key{fams}{esl}{Sign Language}
\define@key{fams}{egy}{Afro-Asiatic}
\define@key{fams}{ehu}{Atlantic-Congo}
\define@key{fams}{eit}{Nuclear Torricelli}
\define@key{fams}{eja}{Atlantic-Congo}
\define@key{fams}{eka}{Atlantic-Congo}
\define@key{fams}{eki}{Atlantic-Congo}
\define@key{fams}{eke}{Atlantic-Congo}
\define@key{fams}{ekp}{Atlantic-Congo}
\define@key{fams}{zpp}{Otomanguean}
\define@key{fams}{elx}{Isolate}
\define@key{fams}{elm}{Atlantic-Congo}
\define@key{fams}{ele}{Nuclear Torricelli}
\define@key{fams}{elh}{Nubian}
\define@key{fams}{ekm}{Atlantic-Congo}
\define@key{fams}{elk}{Nuclear Torricelli}
\define@key{fams}{elo}{Afro-Asiatic}
\define@key{fams}{zte}{Otomanguean}
\define@key{fams}{afo}{Atlantic-Congo}
\define@key{fams}{elu}{Austronesian}
\define@key{fams}{xly}{Unclassifiable}
\define@key{fams}{yzg}{Tai-Kadai}
\define@key{fams}{emn}{Atlantic-Congo}
\define@key{fams}{bdc}{Chocoan}
\define@key{fams}{tdc}{Chocoan}
\define@key{fams}{ebu}{Atlantic-Congo}
\define@key{fams}{emw}{Austronesian}
\define@key{fams}{enr}{Pauwasi}
\define@key{fams}{unk}{Arawakan}
\define@key{fams}{end}{Austronesian}
\define@key{fams}{enc}{Tai-Kadai}
\define@key{fams}{ptt}{Austronesian}
\define@key{fams}{enu}{Sino-Tibetan}
\define@key{fams}{enw}{Atlantic-Congo}
\define@key{fams}{env}{Atlantic-Congo}
\define@key{fams}{epi}{Atlantic-Congo}
\define@key{fams}{emy}{Mayan}
\define@key{fams}{era}{Dravidian}
\define@key{fams}{kjy}{Nuclear Trans New Guinea}
\define@key{fams}{twp}{Austronesian}
\define@key{fams}{ert}{Lakes Plain}
\define@key{fams}{erw}{Austronesian}
\define@key{fams}{err}{Giimbiyu}
\define@key{fams}{emx}{Speech Register}
\define@key{fams}{ers}{Sino-Tibetan}
\define@key{fams}{erh}{Atlantic-Congo}
\define@key{fams}{ish}{Atlantic-Congo}
\define@key{fams}{mcq}{Koiarian}
\define@key{fams}{esh}{Indo-European}
\define@key{fams}{ags}{Atlantic-Congo}
\define@key{fams}{esy}{Artificial Language}
\define@key{fams}{epo}{Artificial Language}
\define@key{fams}{ots}{Otomanguean}
\define@key{fams}{eso}{Sign Language}
\define@key{fams}{esm}{Unattested}
\define@key{fams}{etb}{Atlantic-Congo}
\define@key{fams}{etx}{Atlantic-Congo}
\define@key{fams}{ecr}{Unclassifiable}
\define@key{fams}{ecy}{Unclassifiable}
\define@key{fams}{eth}{Sign Language}
\define@key{fams}{ich}{Atlantic-Congo}
\define@key{fams}{eto}{Atlantic-Congo}
\define@key{fams}{etn}{Austronesian}
\define@key{fams}{ett}{Isolate}
\define@key{fams}{utr}{Atlantic-Congo}
\define@key{fams}{bzz}{Atlantic-Congo}
\define@key{fams}{gev}{Atlantic-Congo}
\define@key{fams}{nou}{Nuclear Trans New Guinea}
\define@key{fams}{ext}{Indo-European}
\define@key{fams}{fab}{Indo-European}
\define@key{fams}{faf}{Austronesian}
\define@key{fams}{fif}{Afro-Asiatic}
\define@key{fams}{azt}{Austronesian}
\define@key{fams}{faj}{Nuclear Trans New Guinea}
\define@key{fams}{fai}{Nuclear Trans New Guinea}
\define@key{fams}{fax}{Indo-European}
\define@key{fams}{cfm}{Sino-Tibetan}
\define@key{fams}{fli}{Afro-Asiatic}
\define@key{fams}{xfa}{Indo-European}
\define@key{fams}{fam}{Atlantic-Congo}
\define@key{fams}{fng}{Pidgin}
\define@key{fams}{fan}{Atlantic-Congo}
\define@key{fams}{fak}{Atlantic-Congo}
\define@key{fams}{fni}{Atlantic-Congo}
\define@key{fams}{nsf}{Sino-Tibetan}
\define@key{fams}{fmu}{Dravidian}
\define@key{fams}{far}{Austronesian}
\define@key{fams}{ddg}{Timor-Alor-Pantar}
\define@key{fams}{fau}{Lakes Plain}
\define@key{fams}{agl}{East Strickland}
\define@key{fams}{fpe}{Indo-European}
\define@key{fams}{fer}{Atlantic-Congo}
\define@key{fams}{hif}{Indo-European}
\define@key{fams}{fil}{Austronesian}
\define@key{fams}{tlp}{Totonacan}
\define@key{fams}{bkb}{Austronesian}
\define@key{fams}{fss}{Sign Language}
\define@key{fams}{fag}{Nuclear Trans New Guinea}
\define@key{fams}{fip}{Atlantic-Congo}
\define@key{fams}{fir}{Atlantic-Congo}
\define@key{fams}{fiw}{East Kutubu}
\define@key{fams}{fln}{Pama-Nyungan}
\define@key{fams}{flh}{Lakes Plain}
\define@key{fams}{fod}{Atlantic-Congo}
\define@key{fams}{frq}{Nuclear Trans New Guinea}
\define@key{fams}{enf}{Uralic}
\define@key{fams}{frt}{Austronesian}
\define@key{fams}{frp}{Indo-European}
\define@key{fams}{fur}{Indo-European}
\define@key{fams}{flr}{Atlantic-Congo}
\define@key{fams}{ula}{Atlantic-Congo}
\define@key{fams}{fuy}{Goilalan}
\define@key{fams}{fwe}{Atlantic-Congo}
\define@key{fams}{fie}{Afro-Asiatic}
\define@key{fams}{ttb}{Atlantic-Congo}
\define@key{fams}{gie}{Kru}
\define@key{fams}{gab}{Afro-Asiatic}
\define@key{fams}{gdg}{Austronesian}
\define@key{fams}{gdk}{Afro-Asiatic}
\define@key{fams}{gbk}{Indo-European}
\define@key{fams}{gad}{Austronesian}
\define@key{fams}{gda}{Indo-European}
\define@key{fams}{gdh}{Jarrakan}
\define@key{fams}{gft}{Afro-Asiatic}
\define@key{fams}{btg}{Kru}
\define@key{fams}{ggu}{Mande}
\define@key{fams}{gbf}{Ndu}
\define@key{fams}{gic}{Unclassifiable}
\define@key{fams}{gcn}{Nuclear Trans New Guinea}
\define@key{fams}{xga}{Indo-European}
\define@key{fams}{glo}{Afro-Asiatic}
\define@key{fams}{gar}{Austronesian}
\define@key{fams}{gce}{Athabaskan-Eyak-Tlingit}
\define@key{fams}{sdn}{Indo-European}
\define@key{fams}{gap}{Nuclear Trans New Guinea}
\define@key{fams}{gal}{Austronesian}
\define@key{fams}{kgj}{Sino-Tibetan}
\define@key{fams}{gma}{Worrorran}
\define@key{fams}{wof}{Atlantic-Congo}
\define@key{fams}{gbl}{Indo-European}
\define@key{fams}{gak}{North Halmahera}
\define@key{fams}{bte}{Atlantic-Congo}
\define@key{fams}{ihw}{Pama-Nyungan}
\define@key{fams}{gne}{Atlantic-Congo}
\define@key{fams}{gnk}{Khoe-Kwadi}
\define@key{fams}{gnq}{Austronesian}
\define@key{fams}{unn}{Pama-Nyungan}
\define@key{fams}{gan}{Sino-Tibetan}
\define@key{fams}{pgd}{Indo-European}
\define@key{fams}{gzn}{Austronesian}
\define@key{fams}{gnb}{Sino-Tibetan}
\define@key{fams}{gnl}{Pama-Nyungan}
\define@key{fams}{ggl}{Nuclear Trans New Guinea}
\define@key{fams}{gao}{Nuclear Trans New Guinea}
\define@key{fams}{gza}{Blue Nile Mao}
\define@key{fams}{gnz}{Atlantic-Congo}
\define@key{fams}{gga}{Austronesian}
\define@key{fams}{gbm}{Indo-European}
\define@key{fams}{ilg}{Iwaidjan Proper}
\define@key{fams}{gex}{Afro-Asiatic}
\define@key{fams}{gaq}{Austroasiatic}
\define@key{fams}{gou}{Afro-Asiatic}
\define@key{fams}{gwt}{Indo-European}
\define@key{fams}{gyl}{South Omotic}
\define@key{fams}{gzi}{Indo-European}
\define@key{fams}{gbg}{Atlantic-Congo}
\define@key{fams}{gbv}{Atlantic-Congo}
\define@key{fams}{gby}{Atlantic-Congo}
\define@key{fams}{gyg}{Atlantic-Congo}
\define@key{fams}{gbq}{Atlantic-Congo}
\define@key{fams}{gbs}{Atlantic-Congo}
\define@key{fams}{ggb}{Kru}
\define@key{fams}{xgb}{Mande}
\define@key{fams}{grh}{Atlantic-Congo}
\define@key{fams}{gec}{Kru}
\define@key{fams}{kvq}{Sino-Tibetan}
\define@key{fams}{gei}{Austronesian}
\define@key{fams}{gdd}{Austronesian}
\define@key{fams}{drs}{Afro-Asiatic}
\define@key{fams}{hmj}{Hmong-Mien}
\define@key{fams}{gez}{Afro-Asiatic}
\define@key{fams}{ghk}{Sino-Tibetan}
\define@key{fams}{giu}{Tai-Kadai}
\define@key{fams}{geq}{Atlantic-Congo}
\define@key{fams}{gaf}{Nuclear Trans New Guinea}
\define@key{fams}{gej}{Atlantic-Congo}
\define@key{fams}{ygp}{Sino-Tibetan}
\define@key{fams}{gew}{Afro-Asiatic}
\define@key{fams}{gea}{Afro-Asiatic}
\define@key{fams}{ges}{Austronesian}
\define@key{fams}{gha}{Afro-Asiatic}
\define@key{fams}{gse}{Sign Language}
\define@key{fams}{ghn}{Austronesian}
\define@key{fams}{gpe}{Indo-European}
\define@key{fams}{gds}{Sign Language}
\define@key{fams}{gri}{Austronesian}
\define@key{fams}{ajs}{Sign Language}
\define@key{fams}{bmk}{Austronesian}
\define@key{fams}{aln}{Indo-European}
\define@key{fams}{ghr}{Indo-European}
\define@key{fams}{bbj}{Atlantic-Congo}
\define@key{fams}{gho}{Afro-Asiatic}
\define@key{fams}{bgi}{Austronesian}
\define@key{fams}{gib}{Pidgin}
\define@key{fams}{kks}{Afro-Asiatic}
\define@key{fams}{acd}{Atlantic-Congo}
\define@key{fams}{gix}{Atlantic-Congo}
\define@key{fams}{gip}{Austronesian}
\define@key{fams}{gim}{Nuclear Trans New Guinea}
\define@key{fams}{kmp}{Atlantic-Congo}
\define@key{fams}{gmn}{Atlantic-Congo}
\define@key{fams}{gnm}{Dagan}
\define@key{fams}{ayg}{Atlantic-Congo}
\define@key{fams}{bbr}{Nuclear Trans New Guinea}
\define@key{fams}{gii}{Afro-Asiatic}
\define@key{fams}{nyf}{Atlantic-Congo}
\define@key{fams}{toh}{Atlantic-Congo}
\define@key{fams}{ggt}{Austronesian}
\define@key{fams}{giy}{Unattested}
\define@key{fams}{tof}{Eastern Trans-Fly}
\define@key{fams}{glr}{Kru}
\define@key{fams}{glw}{Afro-Asiatic}
\define@key{fams}{oub}{Kru}
\define@key{fams}{gnu}{Nuclear Torricelli}
\define@key{fams}{gom}{Indo-European}
\define@key{fams}{gig}{Indo-European}
\define@key{fams}{goi}{East Strickland}
\define@key{fams}{gox}{Atlantic-Congo}
\define@key{fams}{gdx}{Indo-European}
\define@key{fams}{gof}{Ta-Ne-Omotic}
\define@key{fams}{gog}{Atlantic-Congo}
\define@key{fams}{goo}{Austronesian}
\define@key{fams}{goe}{Sino-Tibetan}
\define@key{fams}{gjn}{Atlantic-Congo}
\define@key{fams}{gov}{Mande}
\define@key{fams}{goq}{Austronesian}
\define@key{fams}{goc}{Austronesian}
\define@key{fams}{grq}{Lower Sepik-Ramu}
\define@key{fams}{gqr}{Central Sudanic}
\define@key{fams}{got}{Indo-European}
\define@key{fams}{goy}{Atlantic-Congo}
\define@key{fams}{gwf}{Indo-European}
\define@key{fams}{goz}{Indo-European}
\define@key{fams}{nli}{Indo-European}
\define@key{fams}{giq}{Tai-Kadai}
\define@key{fams}{gcl}{Indo-European}
\define@key{fams}{grs}{Nimboranic}
\define@key{fams}{gro}{Sino-Tibetan}
\define@key{fams}{gos}{Indo-European}
\define@key{fams}{ats}{Algic}
\define@key{fams}{gwx}{Atlantic-Congo}
\define@key{fams}{gvj}{Tupian}
\define@key{fams}{jiq}{Sino-Tibetan}
\define@key{fams}{gnc}{Afro-Asiatic}
\define@key{fams}{gyr}{Tupian}
\define@key{fams}{gsm}{Sign Language}
\define@key{fams}{xgd}{Pama-Nyungan}
\define@key{fams}{gdu}{Afro-Asiatic}
\define@key{fams}{zpg}{Otomanguean}
\define@key{fams}{gdc}{Pama-Nyungan}
\define@key{fams}{kkp}{Pama-Nyungan}
\define@key{fams}{wrw}{Pama-Nyungan}
\define@key{fams}{zgn}{Tai-Kadai}
\define@key{fams}{bet}{Kru}
\define@key{fams}{ztu}{Otomanguean}
\define@key{fams}{gus}{Sign Language}
\define@key{fams}{gkp}{Mande}
\define@key{fams}{gqi}{Sino-Tibetan}
\define@key{fams}{gvl}{Central Sudanic}
\define@key{fams}{glu}{Central Sudanic}
\define@key{fams}{gmb}{Austronesian}
\define@key{fams}{gly}{Isolate}
\define@key{fams}{gul}{Indo-European}
\define@key{fams}{gmu}{Nuclear Trans New Guinea}
\define@key{fams}{gdi}{Atlantic-Congo}
\define@key{fams}{gyf}{Pama-Nyungan}
\define@key{fams}{rub}{Atlantic-Congo}
\define@key{fams}{gnt}{Yam}
\define@key{fams}{gpa}{Atlantic-Congo}
\define@key{fams}{grz}{Austronesian}
\define@key{fams}{gdj}{Pama-Nyungan}
\define@key{fams}{ggg}{Indo-European}
\define@key{fams}{grx}{Isolate}
\define@key{fams}{gjr}{Mixed Language}
\define@key{fams}{gvm}{Atlantic-Congo}
\define@key{fams}{gvr}{Sino-Tibetan}
\define@key{fams}{grd}{Afro-Asiatic}
\define@key{fams}{gsn}{Nuclear Trans New Guinea}
\define@key{fams}{gsl}{Atlantic-Congo}
\define@key{fams}{xgw}{Pama-Nyungan}
\define@key{fams}{gwu}{Pama-Nyungan}
\define@key{fams}{gvy}{Pama-Nyungan}
\define@key{fams}{gka}{Nuclear Trans New Guinea}
\define@key{fams}{ngs}{Afro-Asiatic}
\define@key{fams}{gwb}{Atlantic-Congo}
\define@key{fams}{dah}{Nuclear Trans New Guinea}
\define@key{fams}{bga}{Atlantic-Congo}
\define@key{fams}{gwn}{Afro-Asiatic}
\define@key{fams}{grw}{Austronesian}
\define@key{fams}{gwe}{Atlantic-Congo}
\define@key{fams}{gwr}{Atlantic-Congo}
\define@key{fams}{gwj}{Khoe-Kwadi}
\define@key{fams}{gyi}{Atlantic-Congo}
\define@key{fams}{gye}{Atlantic-Congo}
\define@key{fams}{haq}{Atlantic-Congo}
\define@key{fams}{hbu}{Austronesian}
\define@key{fams}{hdy}{Afro-Asiatic}
\define@key{fams}{hoj}{Indo-European}
\define@key{fams}{xhd}{Afro-Asiatic}
\define@key{fams}{ayh}{Afro-Asiatic}
\define@key{fams}{aek}{Austronesian}
\define@key{fams}{hah}{Austronesian}
\define@key{fams}{hgw}{Austronesian}
\define@key{fams}{bzx}{Mande}
\define@key{fams}{hgm}{Khoe-Kwadi}
\define@key{fams}{haf}{Sign Language}
\define@key{fams}{hvc}{Unclassifiable}
\define@key{fams}{hji}{Austronesian}
\define@key{fams}{haj}{Indo-European}
\define@key{fams}{hao}{Austronesian}
\define@key{fams}{hld}{Austroasiatic}
\define@key{fams}{hmu}{Timor-Alor-Pantar}
\define@key{fams}{hba}{Atlantic-Congo}
\define@key{fams}{hag}{Atlantic-Congo}
\define@key{fams}{han}{Atlantic-Congo}
\define@key{fams}{haa}{Athabaskan-Eyak-Tlingit}
\define@key{fams}{hab}{Sign Language}
\define@key{fams}{xiv}{Unattested}
\define@key{fams}{kjo}{Indo-European}
\define@key{fams}{hro}{Austronesian}
\define@key{fams}{hrk}{Austronesian}
\define@key{fams}{bgc}{Indo-European}
\define@key{fams}{hrz}{Indo-European}
\define@key{fams}{ybj}{Atlantic-Congo}
\define@key{fams}{xht}{Isolate}
\define@key{fams}{hsl}{Sign Language}
\define@key{fams}{hvk}{Austronesian}
\define@key{fams}{hav}{Atlantic-Congo}
\define@key{fams}{hps}{Sign Language}
\define@key{fams}{xda}{Pama-Nyungan}
\define@key{fams}{haz}{Indo-European}
\define@key{fams}{hbn}{Heibanic}
\define@key{fams}{scp}{Sino-Tibetan}
\define@key{fams}{heg}{Austronesian}
\define@key{fams}{nix}{Atlantic-Congo}
\define@key{fams}{hed}{Afro-Asiatic}
\define@key{fams}{llf}{Austronesian}
\define@key{fams}{hrt}{Afro-Asiatic}
\define@key{fams}{ham}{Sepik}
\define@key{fams}{auk}{Nuclear Torricelli}
\define@key{fams}{hib}{Hibito-Cholon}
\define@key{fams}{hlu}{Indo-European}
\define@key{fams}{mba}{Austronesian}
\define@key{fams}{kjk}{Austronesian}
\define@key{fams}{hij}{Atlantic-Congo}
\define@key{fams}{hir}{Unattested}
\define@key{fams}{hii}{Indo-European}
\define@key{fams}{hmo}{Pidgin}
\define@key{fams}{hit}{Indo-European}
\define@key{fams}{htu}{Austronesian}
\define@key{fams}{hiw}{Austronesian}
\define@key{fams}{yhl}{Sino-Tibetan}
\define@key{fams}{hle}{Sino-Tibetan}
\define@key{fams}{hmf}{Hmong-Mien}
\define@key{fams}{hmz}{Hmong-Mien}
\define@key{fams}{hmv}{Hmong-Mien}
\define@key{fams}{mrk}{Austronesian}
\define@key{fams}{hoh}{Afro-Asiatic}
\define@key{fams}{hos}{Sign Language}
\define@key{fams}{hhi}{Anim}
\define@key{fams}{hoy}{Dravidian}
\define@key{fams}{hoi}{Athabaskan-Eyak-Tlingit}
\define@key{fams}{hod}{Afro-Asiatic}
\define@key{fams}{hol}{Atlantic-Congo}
\define@key{fams}{hom}{Atlantic-Congo}
\define@key{fams}{hds}{Sign Language}
\define@key{fams}{juh}{Atlantic-Congo}
\define@key{fams}{how}{Sino-Tibetan}
\define@key{fams}{hrm}{Hmong-Mien}
\define@key{fams}{hoe}{Atlantic-Congo}
\define@key{fams}{hor}{Central Sudanic}
\define@key{fams}{ero}{Sino-Tibetan}
\define@key{fams}{hot}{Austronesian}
\define@key{fams}{hti}{Austronesian}
\define@key{fams}{hov}{Austronesian}
\define@key{fams}{hhy}{Anim}
\define@key{fams}{hoz}{Blue Nile Mao}
\define@key{fams}{hpo}{Sino-Tibetan}
\define@key{fams}{hra}{Sino-Tibetan}
\define@key{fams}{hru}{Isolate}
\define@key{fams}{hug}{Harakmbut}
\define@key{fams}{qvh}{Quechuan}
\define@key{fams}{hud}{Austronesian}
\define@key{fams}{nhq}{Uto-Aztecan}
\define@key{fams}{qwh}{Quechuan}
\define@key{fams}{qvw}{Quechuan}
\define@key{fams}{huh}{Araucanian}
\define@key{fams}{mxs}{Otomanguean}
\define@key{fams}{czh}{Sino-Tibetan}
\define@key{fams}{huw}{Austronesian}
\define@key{fams}{hul}{Austronesian}
\define@key{fams}{huy}{Afro-Asiatic}
\define@key{fams}{hui}{Nuclear Trans New Guinea}
\define@key{fams}{huk}{Austronesian}
\define@key{fams}{hmb}{Songhay}
\define@key{fams}{huf}{Kwalean}
\define@key{fams}{hut}{Sino-Tibetan}
\define@key{fams}{hsh}{Sign Language}
\define@key{fams}{hnu}{Austroasiatic}
\define@key{fams}{nat}{Atlantic-Congo}
\define@key{fams}{hum}{Atlantic-Congo}
\define@key{fams}{hng}{Atlantic-Congo}
\define@key{fams}{hkk}{Nuclear Trans New Guinea}
\define@key{fams}{hap}{Nuclear Trans New Guinea}
\define@key{fams}{xhu}{Hurro-Urartian}
\define@key{fams}{geh}{Indo-European}
\define@key{fams}{huo}{Austroasiatic}
\define@key{fams}{hwo}{Afro-Asiatic}
\define@key{fams}{hya}{Afro-Asiatic}
\define@key{fams}{jab}{Atlantic-Congo}
\define@key{fams}{yml}{Austronesian}
\define@key{fams}{tek}{Atlantic-Congo}
\define@key{fams}{ibl}{Austronesian}
\define@key{fams}{iby}{Ijoid}
\define@key{fams}{xib}{Isolate}
\define@key{fams}{ibn}{Atlantic-Congo}
\define@key{fams}{ibr}{Atlantic-Congo}
\define@key{fams}{ibu}{North Halmahera}
\define@key{fams}{bec}{Atlantic-Congo}
\define@key{fams}{ida}{Atlantic-Congo}
\define@key{fams}{idt}{Austronesian}
\define@key{fams}{ide}{Atlantic-Congo}
\define@key{fams}{idi}{Pahoturi}
\define@key{fams}{idc}{Atlantic-Congo}
\define@key{fams}{ido}{Artificial Language}
\define@key{fams}{ldb}{Atlantic-Congo}
\define@key{fams}{ife}{Atlantic-Congo}
\define@key{fams}{iff}{Austronesian}
\define@key{fams}{igl}{Atlantic-Congo}
\define@key{fams}{igg}{Lower Sepik-Ramu}
\define@key{fams}{ahl}{Atlantic-Congo}
\define@key{fams}{nar}{Atlantic-Congo}
\define@key{fams}{igw}{Atlantic-Congo}
\define@key{fams}{ihb}{Pidgin}
\define@key{fams}{ikk}{Atlantic-Congo}
\define@key{fams}{ikr}{Pama-Nyungan}
\define@key{fams}{ikz}{Atlantic-Congo}
\define@key{fams}{meb}{Turama-Kikori}
\define@key{fams}{ntk}{Atlantic-Congo}
\define@key{fams}{iki}{Atlantic-Congo}
\define@key{fams}{ikp}{Atlantic-Congo}
\define@key{fams}{txi}{Cariban}
\define@key{fams}{ikv}{Atlantic-Congo}
\define@key{fams}{ikl}{Atlantic-Congo}
\define@key{fams}{ikw}{Atlantic-Congo}
\define@key{fams}{ila}{Austronesian}
\define@key{fams}{mbi}{Austronesian}
\define@key{fams}{ili}{Turkic}
\define@key{fams}{ilu}{Austronesian}
\define@key{fams}{xil}{Unclassifiable}
\define@key{fams}{ilk}{Austronesian}
\define@key{fams}{ilv}{Atlantic-Congo}
\define@key{fams}{mlk}{Atlantic-Congo}
\define@key{fams}{imo}{Nuclear Trans New Guinea}
\define@key{fams}{arc}{Afro-Asiatic}
\define@key{fams}{imr}{Austronesian}
\define@key{fams}{abx}{Austronesian}
\define@key{fams}{mzu}{Lower Sepik-Ramu}
\define@key{fams}{inp}{Arawakan}
\define@key{fams}{smn}{Uralic}
\define@key{fams}{inl}{Sign Language}
\define@key{fams}{idr}{Atlantic-Congo}
\define@key{fams}{mvy}{Indo-European}
\define@key{fams}{oin}{Nuclear Torricelli}
\define@key{fams}{iti}{Austronesian}
\define@key{fams}{ino}{Nuclear Trans New Guinea}
\define@key{fams}{loc}{Austronesian}
\define@key{fams}{ior}{Afro-Asiatic}
\define@key{fams}{ina}{Artificial Language}
\define@key{fams}{ile}{Artificial Language}
\define@key{fams}{igs}{Artificial Language}
\define@key{fams}{int}{Sino-Tibetan}
\define@key{fams}{iks}{Sign Language}
\define@key{fams}{azm}{Otomanguean}
\define@key{fams}{ipo}{Anim}
\define@key{fams}{ipi}{Nuclear Trans New Guinea}
\define@key{fams}{ass}{Atlantic-Congo}
\define@key{fams}{ill}{Austronesian}
\define@key{fams}{iry}{Austronesian}
\define@key{fams}{ire}{Austronesian}
\define@key{fams}{iri}{Atlantic-Congo}
\define@key{fams}{bto}{Austronesian}
\define@key{fams}{iru}{Dravidian}
\define@key{fams}{isa}{Nuclear Trans New Guinea}
\define@key{fams}{isn}{Atlantic-Congo}
\define@key{fams}{agk}{Austronesian}
\define@key{fams}{isc}{Pano-Tacanan}
\define@key{fams}{igo}{Nuclear Trans New Guinea}
\define@key{fams}{inn}{Austronesian}
\define@key{fams}{crb}{Arawakan}
\define@key{fams}{mir}{Mixe-Zoque}
\define@key{fams}{nhk}{Uto-Aztecan}
\define@key{fams}{ist}{Indo-European}
\define@key{fams}{ruo}{Indo-European}
\define@key{fams}{szv}{Atlantic-Congo}
\define@key{fams}{isu}{Atlantic-Congo}
\define@key{fams}{ite}{Chapacuran}
\define@key{fams}{itr}{Left May}
\define@key{fams}{itx}{Tor-Orya}
\define@key{fams}{itw}{Atlantic-Congo}
\define@key{fams}{itm}{Atlantic-Congo}
\define@key{fams}{mce}{Otomanguean}
\define@key{fams}{ivv}{Austronesian}
\define@key{fams}{atg}{Atlantic-Congo}
\define@key{fams}{iwk}{Austronesian}
\define@key{fams}{kbm}{Austronesian}
\define@key{fams}{iwo}{Nuclear Trans New Guinea}
\define@key{fams}{mzi}{Otomanguean}
\define@key{fams}{vmj}{Otomanguean}
\define@key{fams}{iya}{Atlantic-Congo}
\define@key{fams}{uiv}{Atlantic-Congo}
\define@key{fams}{crt}{Matacoan}
\define@key{fams}{nca}{Nuclear Trans New Guinea}
\define@key{fams}{crq}{Matacoan}
\define@key{fams}{izi}{Atlantic-Congo}
\define@key{fams}{cbo}{Atlantic-Congo}
\define@key{fams}{rzh}{Afro-Asiatic}
\define@key{fams}{jdg}{Indo-European}
\define@key{fams}{jad}{Mande}
\define@key{fams}{jah}{Austroasiatic}
\define@key{fams}{awv}{Nuclear Trans New Guinea}
\define@key{fams}{jat}{Indo-European}
\define@key{fams}{jak}{Austronesian}
\define@key{fams}{maj}{Otomanguean}
\define@key{fams}{bxl}{Mande}
\define@key{fams}{jcs}{Sign Language}
\define@key{fams}{jls}{Sign Language}
\define@key{fams}{jax}{Austronesian}
\define@key{fams}{jnd}{Indo-European}
\define@key{fams}{jna}{Sino-Tibetan}
\define@key{fams}{djo}{Austronesian}
\define@key{fams}{jni}{Atlantic-Congo}
\define@key{fams}{jar}{Atlantic-Congo}
\define@key{fams}{jra}{Austronesian}
\define@key{fams}{jaf}{Afro-Asiatic}
\define@key{fams}{qxw}{Quechuan}
\define@key{fams}{jns}{Indo-European}
\define@key{fams}{jvd}{Indo-European}
\define@key{fams}{jaz}{Austronesian}
\define@key{fams}{jyy}{Central Sudanic}
\define@key{fams}{jje}{Koreanic}
\define@key{fams}{bze}{Mande}
\define@key{fams}{xuj}{Dravidian}
\define@key{fams}{jer}{Atlantic-Congo}
\define@key{fams}{jee}{Sino-Tibetan}
\define@key{fams}{tmr}{Afro-Asiatic}
\define@key{fams}{jhs}{Sign Language}
\define@key{fams}{jio}{Tai-Kadai}
\define@key{fams}{juo}{Atlantic-Congo}
\define@key{fams}{jib}{Atlantic-Congo}
\define@key{fams}{jii}{Afro-Asiatic}
\define@key{fams}{jie}{Afro-Asiatic}
\define@key{fams}{jil}{Nuclear Trans New Guinea}
\define@key{fams}{jim}{Afro-Asiatic}
\define@key{fams}{jmi}{Afro-Asiatic}
\define@key{fams}{jia}{Afro-Asiatic}
\define@key{fams}{cjy}{Sino-Tibetan}
\define@key{fams}{pnu}{Hmong-Mien}
\define@key{fams}{jul}{Sino-Tibetan}
\define@key{fams}{jrr}{Atlantic-Congo}
\define@key{fams}{jit}{Atlantic-Congo}
\define@key{fams}{kaj}{Atlantic-Congo}
\define@key{fams}{job}{Atlantic-Congo}
\define@key{fams}{jbr}{Tor-Orya}
\define@key{fams}{jeu}{Afro-Asiatic}
\define@key{fams}{jor}{Tupian}
\define@key{fams}{jrt}{Afro-Asiatic}
\define@key{fams}{jow}{Mande}
\define@key{fams}{itk}{Indo-European}
\define@key{fams}{jdt}{Indo-European}
\define@key{fams}{jpr}{Indo-European}
\define@key{fams}{yud}{Afro-Asiatic}
\define@key{fams}{aju}{Afro-Asiatic}
\define@key{fams}{yhd}{Afro-Asiatic}
\define@key{fams}{jye}{Afro-Asiatic}
\define@key{fams}{jum}{Nilotic}
\define@key{fams}{jml}{Indo-European}
\define@key{fams}{jus}{Sign Language}
\define@key{fams}{mxq}{Mixe-Zoque}
\define@key{fams}{juy}{Austroasiatic}
\define@key{fams}{jut}{Indo-European}
\define@key{fams}{juu}{Afro-Asiatic}
\define@key{fams}{mwb}{Nuclear Torricelli}
\define@key{fams}{vmc}{Otomanguean}
\define@key{fams}{jwi}{Atlantic-Congo}
\define@key{fams}{xku}{Atlantic-Congo}
\define@key{fams}{gna}{Atlantic-Congo}
\define@key{fams}{ldl}{Atlantic-Congo}
\define@key{fams}{ckn}{Sino-Tibetan}
\define@key{fams}{ksp}{Central Sudanic}
\define@key{fams}{kvf}{Afro-Asiatic}
\define@key{fams}{gbw}{Pama-Nyungan}
\define@key{fams}{klz}{Timor-Alor-Pantar}
\define@key{fams}{onk}{Nuclear Torricelli}
\define@key{fams}{lkb}{Atlantic-Congo}
\define@key{fams}{uka}{South Bird's Head Family}
\define@key{fams}{kbu}{Indo-European}
\define@key{fams}{kea}{Indo-European}
\define@key{fams}{cwa}{Atlantic-Congo}
\define@key{fams}{kcw}{Atlantic-Congo}
\define@key{fams}{gjk}{Indo-European}
\define@key{fams}{kfr}{Indo-European}
\define@key{fams}{kcx}{Ta-Ne-Omotic}
\define@key{fams}{xkk}{Austroasiatic}
\define@key{fams}{kej}{Dravidian}
\define@key{fams}{kdu}{Nubian}
\define@key{fams}{kad}{Atlantic-Congo}
\define@key{fams}{kzd}{Austronesian}
\define@key{fams}{kdv}{Sino-Tibetan}
\define@key{fams}{ktp}{Sino-Tibetan}
\define@key{fams}{jka}{Timor-Alor-Pantar}
\define@key{fams}{kpu}{Timor-Alor-Pantar}
\define@key{fams}{sqx}{Sign Language}
\define@key{fams}{syw}{Sino-Tibetan}
\define@key{fams}{kll}{Austronesian}
\define@key{fams}{cgc}{Austronesian}
\define@key{fams}{gel}{Atlantic-Congo}
\define@key{fams}{xkg}{Mande}
\define@key{fams}{hka}{Atlantic-Congo}
\define@key{fams}{agw}{Austronesian}
\define@key{fams}{kzb}{Austronesian}
\define@key{fams}{kzp}{Austronesian}
\define@key{fams}{kbw}{Austronesian}
\define@key{fams}{kep}{Dravidian}
\define@key{fams}{kzq}{Sino-Tibetan}
\define@key{fams}{kkq}{Atlantic-Congo}
\define@key{fams}{xai}{Unclassifiable}
\define@key{fams}{zka}{Austronesian}
\define@key{fams}{krd}{Austronesian}
\define@key{fams}{ckr}{Baining}
\define@key{fams}{kzm}{South Bird's Head Family}
\define@key{fams}{kce}{Unattested}
\define@key{fams}{tcq}{Lakes Plain}
\define@key{fams}{xkj}{Indo-European}
\define@key{fams}{kag}{Austronesian}
\define@key{fams}{ckq}{Afro-Asiatic}
\define@key{fams}{kjv}{Indo-European}
\define@key{fams}{xdq}{Nakh-Daghestanian}
\define@key{fams}{kka}{Atlantic-Congo}
\define@key{fams}{kke}{Mande}
\define@key{fams}{kqf}{Austronesian}
\define@key{fams}{kkj}{Atlantic-Congo}
\define@key{fams}{keo}{Nilotic}
\define@key{fams}{wkl}{Dravidian}
\define@key{fams}{kzz}{West Bird's Head}
\define@key{fams}{kkf}{Sino-Tibetan}
\define@key{fams}{kba}{Pama-Nyungan}
\define@key{fams}{gll}{Pama-Nyungan}
\define@key{fams}{ijn}{Ijoid}
\define@key{fams}{knz}{Atlantic-Congo}
\define@key{fams}{kqe}{Austronesian}
\define@key{fams}{kve}{Austronesian}
\define@key{fams}{kly}{Austronesian}
\define@key{fams}{lkm}{Pama-Nyungan}
\define@key{fams}{xka}{Indo-European}
\define@key{fams}{rmf}{Indo-European}
\define@key{fams}{ywa}{Sepik}
\define@key{fams}{kli}{Austronesian}
\define@key{fams}{keq}{Indo-European}
\define@key{fams}{jmr}{Atlantic-Congo}
\define@key{fams}{kci}{Atlantic-Congo}
\define@key{fams}{klp}{Angan}
\define@key{fams}{kzx}{Austronesian}
\define@key{fams}{kyk}{Austronesian}
\define@key{fams}{kgx}{Austronesian}
\define@key{fams}{vkm}{Kamakanan}
\define@key{fams}{xbw}{Unclassifiable}
\define@key{fams}{irx}{Nuclear Trans New Guinea}
\define@key{fams}{kyy}{Nuclear Trans New Guinea}
\define@key{fams}{ktb}{Afro-Asiatic}
\define@key{fams}{kmi}{Atlantic-Congo}
\define@key{fams}{kdx}{Atlantic-Congo}
\define@key{fams}{kcq}{Atlantic-Congo}
\define@key{fams}{xla}{Kamula-Elevala}
\define@key{fams}{hig}{Afro-Asiatic}
\define@key{fams}{bjj}{Indo-European}
\define@key{fams}{xnb}{Austronesian}
\define@key{fams}{soq}{Dagan}
\define@key{fams}{kbs}{Atlantic-Congo}
\define@key{fams}{kqw}{Austronesian}
\define@key{fams}{gam}{Nuclear Trans New Guinea}
\define@key{fams}{xnr}{Indo-European}
\define@key{fams}{kxs}{Mongolic-Khitan}
\define@key{fams}{kzy}{Atlantic-Congo}
\define@key{fams}{kty}{Atlantic-Congo}
\define@key{fams}{kcp}{Kadugli-Krongo}
\define@key{fams}{kkv}{Austronesian}
\define@key{fams}{igm}{Lower Sepik-Ramu}
\define@key{fams}{kev}{Dravidian}
\define@key{fams}{kdp}{Atlantic-Congo}
\define@key{fams}{kzo}{Atlantic-Congo}
\define@key{fams}{wat}{Austronesian}
\define@key{fams}{ktk}{Austronesian}
\define@key{fams}{knr}{Sepik}
\define@key{fams}{kmu}{Nuclear Trans New Guinea}
\define@key{fams}{kft}{Indo-European}
\define@key{fams}{kbe}{Pama-Nyungan}
\define@key{fams}{kxn}{Austronesian}
\define@key{fams}{ksk}{Siouan}
\define@key{fams}{xkt}{Atlantic-Congo}
\define@key{fams}{kni}{Atlantic-Congo}
\define@key{fams}{khx}{Atlantic-Congo}
\define@key{fams}{kqn}{Atlantic-Congo}
\define@key{fams}{kax}{North Halmahera}
\define@key{fams}{xpn}{Unclassifiable}
\define@key{fams}{tbx}{Austronesian}
\define@key{fams}{khp}{Isolate}
\define@key{fams}{ykm}{Austronesian}
\define@key{fams}{kbi}{Austronesian}
\define@key{fams}{klo}{Atlantic-Congo}
\define@key{fams}{xkh}{Unattested}
\define@key{fams}{kzr}{Atlantic-Congo}
\define@key{fams}{reg}{Atlantic-Congo}
\define@key{fams}{kth}{Maban}
\define@key{fams}{mry}{Austronesian}
\define@key{fams}{xrw}{Sepik}
\define@key{fams}{xar}{Isolate}
\define@key{fams}{kgv}{West Bomberai}
\define@key{fams}{kbn}{Atlantic-Congo}
\define@key{fams}{kyd}{Austronesian}
\define@key{fams}{kmf}{Nuclear Trans New Guinea}
\define@key{fams}{kai}{Afro-Asiatic}
\define@key{fams}{kmv}{Indo-European}
\define@key{fams}{kgn}{Indo-European}
\define@key{fams}{kbj}{Atlantic-Congo}
\define@key{fams}{kil}{Afro-Asiatic}
\define@key{fams}{kuq}{Tupian}
\define@key{fams}{kko}{Nubian}
\define@key{fams}{krb}{Miwok-Costanoan}
\define@key{fams}{bbv}{Austronesian}
\define@key{fams}{krx}{Atlantic-Congo}
\define@key{fams}{kxh}{South Omotic}
\define@key{fams}{xkx}{Austronesian}
\define@key{fams}{kyn}{Austronesian}
\define@key{fams}{rxw}{Pama-Nyungan}
\define@key{fams}{ccj}{Atlantic-Congo}
\define@key{fams}{ksn}{Austronesian}
\define@key{fams}{kkz}{Athabaskan-Eyak-Tlingit}
\define@key{fams}{khs}{Bosavi}
\define@key{fams}{ktq}{Unclassifiable}
\define@key{fams}{xat}{Katukinan}
\define@key{fams}{tmb}{Austronesian}
\define@key{fams}{tkt}{Indo-European}
\define@key{fams}{ykt}{Sino-Tibetan}
\define@key{fams}{kfu}{Indo-European}
\define@key{fams}{kaf}{Sino-Tibetan}
\define@key{fams}{kta}{Austroasiatic}
\define@key{fams}{vkk}{Austronesian}
\define@key{fams}{xau}{Greater Kwerba}
\define@key{fams}{ckv}{Austronesian}
\define@key{fams}{kcb}{Angan}
\define@key{fams}{kgb}{Austronesian}
\define@key{fams}{kaw}{Austronesian}
\define@key{fams}{ktx}{Pano-Tacanan}
\define@key{fams}{kbb}{Cariban}
\define@key{fams}{pdu}{Sino-Tibetan}
\define@key{fams}{xay}{Austronesian}
\define@key{fams}{xkn}{Austronesian}
\define@key{fams}{kyt}{Kayagaric}
\define@key{fams}{kzl}{Austronesian}
\define@key{fams}{kxy}{Austroasiatic}
\define@key{fams}{kzu}{Austronesian}
\define@key{fams}{kzk}{Austronesian}
\define@key{fams}{keh}{Ndu}
\define@key{fams}{khz}{Austronesian}
\define@key{fams}{meo}{Austronesian}
\define@key{fams}{kdy}{Tor-Orya}
\define@key{fams}{khh}{Isolate}
\define@key{fams}{kec}{Kadugli-Krongo}
\define@key{fams}{bmh}{Nuclear Trans New Guinea}
\define@key{fams}{eyo}{Nilotic}
\define@key{fams}{khy}{Atlantic-Congo}
\define@key{fams}{keb}{Atlantic-Congo}
\define@key{fams}{ify}{Austronesian}
\define@key{fams}{kbo}{Central Sudanic}
\define@key{fams}{xel}{Eastern Jebel}
\define@key{fams}{kyo}{Timor-Alor-Pantar}
\define@key{fams}{kem}{Austronesian}
\define@key{fams}{bzp}{South Bird's Head Family}
\define@key{fams}{xem}{Austronesian}
\define@key{fams}{xkw}{Lepki-Murkim-Kembra}
\define@key{fams}{dmo}{Atlantic-Congo}
\define@key{fams}{sjk}{Uralic}
\define@key{fams}{xbn}{Isolate}
\define@key{fams}{gat}{Nuclear Trans New Guinea}
\define@key{fams}{kvm}{Atlantic-Congo}
\define@key{fams}{klf}{Maban}
\define@key{fams}{knx}{Austronesian}
\define@key{fams}{knl}{Austronesian}
\define@key{fams}{kxi}{Austronesian}
\define@key{fams}{kns}{Austroasiatic}
\define@key{fams}{ndb}{Atlantic-Congo}
\define@key{fams}{kzh}{Nubian}
\define@key{fams}{lke}{Atlantic-Congo}
\define@key{fams}{xeu}{Eleman}
\define@key{fams}{kpn}{Tupian}
\define@key{fams}{kuk}{Austronesian}
\define@key{fams}{hhr}{Atlantic-Congo}
\define@key{fams}{ked}{Atlantic-Congo}
\define@key{fams}{xke}{Austronesian}
\define@key{fams}{kxz}{Kiwaian}
\define@key{fams}{kvr}{Austronesian}
\define@key{fams}{xes}{Nuclear Trans New Guinea}
\define@key{fams}{kae}{Austronesian}
\define@key{fams}{ktt}{Nuclear Trans New Guinea}
\define@key{fams}{kyg}{Nuclear Trans New Guinea}
\define@key{fams}{xkv}{Atlantic-Congo}
\define@key{fams}{hkh}{Indo-European}
\define@key{fams}{kbg}{Sino-Tibetan}
\define@key{fams}{kht}{Tai-Kadai}
\define@key{fams}{ksu}{Tai-Kadai}
\define@key{fams}{khn}{Indo-European}
\define@key{fams}{kjm}{Austroasiatic}
\define@key{fams}{ksy}{Indo-European}
\define@key{fams}{kfw}{Sino-Tibetan}
\define@key{fams}{lko}{Atlantic-Congo}
\define@key{fams}{kqg}{Atlantic-Congo}
\define@key{fams}{tlx}{Austronesian}
\define@key{fams}{xkf}{Sino-Tibetan}
\define@key{fams}{xhe}{Indo-European}
\define@key{fams}{nkh}{Sino-Tibetan}
\define@key{fams}{kix}{Sino-Tibetan}
\define@key{fams}{kwx}{Dravidian}
\define@key{fams}{kqm}{Atlantic-Congo}
\define@key{fams}{ykl}{Sino-Tibetan}
\define@key{fams}{xkc}{Indo-European}
\define@key{fams}{nkb}{Sino-Tibetan}
\define@key{fams}{ktc}{Afro-Asiatic}
\define@key{fams}{kho}{Indo-European}
\define@key{fams}{khf}{Austroasiatic}
\define@key{fams}{kfm}{Indo-European}
\define@key{fams}{xco}{Indo-European}
\define@key{fams}{kie}{Maban}
\define@key{fams}{prm}{Isolate}
\define@key{fams}{kzg}{Japonic}
\define@key{fams}{kih}{Border}
\define@key{fams}{kqr}{Austronesian}
\define@key{fams}{kmb}{Atlantic-Congo}
\define@key{fams}{kiv}{Atlantic-Congo}
\define@key{fams}{sbt}{Isolate}
\define@key{fams}{kqp}{Afro-Asiatic}
\define@key{fams}{krj}{Austronesian}
\define@key{fams}{kco}{Nuclear Trans New Guinea}
\define@key{fams}{cbw}{Austronesian}
\define@key{fams}{knq}{Austroasiatic}
\define@key{fams}{kkd}{Atlantic-Congo}
\define@key{fams}{ues}{Austronesian}
\define@key{fams}{kkm}{Atlantic-Congo}
\define@key{fams}{apk}{Athabaskan-Eyak-Tlingit}
\define@key{fams}{sgc}{Nilotic}
\define@key{fams}{kyi}{Austronesian}
\define@key{fams}{kkr}{Afro-Asiatic}
\define@key{fams}{okr}{Ijoid}
\define@key{fams}{kiu}{Indo-European}
\define@key{fams}{fkk}{Afro-Asiatic}
\define@key{fams}{lks}{Atlantic-Congo}
\define@key{fams}{kiz}{Atlantic-Congo}
\define@key{fams}{kis}{Austronesian}
\define@key{fams}{zkt}{Mongolic-Khitan}
\define@key{fams}{mwk}{Mande}
\define@key{fams}{mkw}{Atlantic-Congo}
\define@key{fams}{kqt}{Austronesian}
\define@key{fams}{tlh}{Artificial Language}
\define@key{fams}{kib}{Heibanic}
\define@key{fams}{kpd}{Austronesian}
\define@key{fams}{kcj}{Atlantic-Congo}
\define@key{fams}{kgu}{Nuclear Trans New Guinea}
\define@key{fams}{thq}{Indo-European}
\define@key{fams}{kdq}{Sino-Tibetan}
\define@key{fams}{dhw}{Indo-European}
\define@key{fams}{cdz}{Austroasiatic}
\define@key{fams}{ksz}{Austroasiatic}
\define@key{fams}{vko}{Austronesian}
\define@key{fams}{kwp}{Kru}
\define@key{fams}{kod}{Austronesian}
\define@key{fams}{kcs}{Afro-Asiatic}
\define@key{fams}{kpi}{Geelvink Bay}
\define@key{fams}{kwl}{Afro-Asiatic}
\define@key{fams}{zkg}{Unclassifiable}
\define@key{fams}{plk}{Indo-European}
\define@key{fams}{kkx}{Austronesian}
\define@key{fams}{kkt}{Sino-Tibetan}
\define@key{fams}{nkd}{Sino-Tibetan}
\define@key{fams}{kxt}{Ndu}
\define@key{fams}{kou}{Atlantic-Congo}
\define@key{fams}{gko}{Pama-Nyungan}
\define@key{fams}{xod}{South Bird's Head Family}
\define@key{fams}{kzn}{Atlantic-Congo}
\define@key{fams}{klc}{Atlantic-Congo}
\define@key{fams}{ekl}{Austroasiatic}
\define@key{fams}{biw}{Atlantic-Congo}
\define@key{fams}{skn}{Austronesian}
\define@key{fams}{klm}{Nuclear Trans New Guinea}
\define@key{fams}{kol}{Isolate}
\define@key{fams}{klx}{Austronesian}
\define@key{fams}{kmy}{Atlantic-Congo}
\define@key{fams}{kpf}{Nuclear Trans New Guinea}
\define@key{fams}{tyn}{Nuclear Trans New Guinea}
\define@key{fams}{kmm}{Sino-Tibetan}
\define@key{fams}{xoi}{Lower Sepik-Ramu}
\define@key{fams}{kmw}{Atlantic-Congo}
\define@key{fams}{kvh}{Austronesian}
\define@key{fams}{kvp}{Austronesian}
\define@key{fams}{kzv}{Nuclear Trans New Guinea}
\define@key{fams}{kxw}{East Strickland}
\define@key{fams}{knd}{Konda-Yahadian}
\define@key{fams}{kdw}{Mombum-Koneraw}
\define@key{fams}{klk}{Atlantic-Congo}
\define@key{fams}{kcz}{Atlantic-Congo}
\define@key{fams}{knu}{Mande}
\define@key{fams}{kno}{Mande}
\define@key{fams}{koa}{Austronesian}
\define@key{fams}{kxc}{Afro-Asiatic}
\define@key{fams}{nbe}{Sino-Tibetan}
\define@key{fams}{mku}{Mande}
\define@key{fams}{koo}{Atlantic-Congo}
\define@key{fams}{ozm}{Atlantic-Congo}
\define@key{fams}{fuj}{Heibanic}
\define@key{fams}{xop}{Lower Sepik-Ramu}
\define@key{fams}{opk}{Nuclear Trans New Guinea}
\define@key{fams}{kcy}{Songhay}
\define@key{fams}{koz}{Nuclear Trans New Guinea}
\define@key{fams}{okh}{Indo-European}
\define@key{fams}{vkp}{Indo-European}
\define@key{fams}{ktl}{Indo-European}
\define@key{fams}{krp}{Atlantic-Congo}
\define@key{fams}{kfo}{Mande}
\define@key{fams}{krf}{Austronesian}
\define@key{fams}{xkq}{Austronesian}
\define@key{fams}{kqj}{South Bougainville}
\define@key{fams}{jkr}{Sino-Tibetan}
\define@key{fams}{vkn}{Atlantic-Congo}
\define@key{fams}{vkz}{Atlantic-Congo}
\define@key{fams}{kfd}{Dravidian}
\define@key{fams}{kpq}{Nuclear Trans New Guinea}
\define@key{fams}{xor}{Pano-Tacanan}
\define@key{fams}{kfp}{Austroasiatic}
\define@key{fams}{kiq}{Kaure-Kosare}
\define@key{fams}{kid}{Atlantic-Congo}
\define@key{fams}{kqk}{Atlantic-Congo}
\define@key{fams}{koq}{Atlantic-Congo}
\define@key{fams}{mqg}{Austronesian}
\define@key{fams}{grm}{Austronesian}
\define@key{fams}{avk}{Artificial Language}
\define@key{fams}{zko}{Yeniseian}
\define@key{fams}{kyf}{Kru}
\define@key{fams}{kqb}{Nuclear Trans New Guinea}
\define@key{fams}{kvc}{Austronesian}
\define@key{fams}{xow}{Nuclear Trans New Guinea}
\define@key{fams}{kwh}{Austronesian}
\define@key{fams}{kga}{Mande}
\define@key{fams}{koh}{Atlantic-Congo}
\define@key{fams}{kqd}{Afro-Asiatic}
\define@key{fams}{kuw}{Atlantic-Congo}
\define@key{fams}{kpl}{Atlantic-Congo}
\define@key{fams}{pbn}{Atlantic-Congo}
\define@key{fams}{koc}{Atlantic-Congo}
\define@key{fams}{cpo}{Mande}
\define@key{fams}{kef}{Atlantic-Congo}
\define@key{fams}{kph}{Atlantic-Congo}
\define@key{fams}{kye}{Atlantic-Congo}
\define@key{fams}{rka}{Austroasiatic}
\define@key{fams}{xre}{Nuclear-Macro-Je}
\define@key{fams}{kri}{Indo-European}
\define@key{fams}{kxb}{Atlantic-Congo}
\define@key{fams}{tyu}{Khoe-Kwadi}
\define@key{fams}{yku}{Sino-Tibetan}
\define@key{fams}{uan}{Tai-Kadai}
\define@key{fams}{kua}{Atlantic-Congo}
\define@key{fams}{ykn}{Sino-Tibetan}
\define@key{fams}{ugh}{Nakh-Daghestanian}
\define@key{fams}{kgf}{Nuclear Trans New Guinea}
\define@key{fams}{kof}{Afro-Asiatic}
\define@key{fams}{jko}{East Strickland}
\define@key{fams}{kvb}{Austronesian}
\define@key{fams}{lkc}{Sino-Tibetan}
\define@key{fams}{kfg}{Dravidian}
\define@key{fams}{kyw}{Indo-European}
\define@key{fams}{kov}{Atlantic-Congo}
\define@key{fams}{kow}{Atlantic-Congo}
\define@key{fams}{kes}{Atlantic-Congo}
\define@key{fams}{dkr}{Austronesian}
\define@key{fams}{vkj}{Isolate}
\define@key{fams}{kux}{Pama-Nyungan}
\define@key{fams}{kez}{Atlantic-Congo}
\define@key{fams}{kfn}{Atlantic-Congo}
\define@key{fams}{ugb}{Pama-Nyungan}
\define@key{fams}{xmp}{Pama-Nyungan}
\define@key{fams}{xmh}{Pama-Nyungan}
\define@key{fams}{ukv}{Nilotic}
\define@key{fams}{kul}{Afro-Asiatic}
\define@key{fams}{kxj}{Central Sudanic}
\define@key{fams}{vkl}{Austronesian}
\define@key{fams}{xpk}{Pano-Tacanan}
\define@key{fams}{kfx}{Indo-European}
\define@key{fams}{pzh}{Austronesian}
\define@key{fams}{uon}{Austronesian}
\define@key{fams}{bbu}{Atlantic-Congo}
\define@key{fams}{kdi}{Nilotic}
\define@key{fams}{ksl}{Austronesian}
\define@key{fams}{ksm}{Atlantic-Congo}
\define@key{fams}{xks}{Austronesian}
\define@key{fams}{kra}{Indo-European}
\define@key{fams}{kuo}{Nuclear Trans New Guinea}
\define@key{fams}{zum}{Indo-European}
\define@key{fams}{wku}{Dravidian}
\define@key{fams}{kdn}{Atlantic-Congo}
\define@key{fams}{shd}{Indo-European}
\define@key{fams}{kgl}{Pama-Nyungan}
\define@key{fams}{ggk}{Isolate}
\define@key{fams}{kfl}{Atlantic-Congo}
\define@key{fams}{kse}{Austronesian}
\define@key{fams}{xug}{Japonic}
\define@key{fams}{pep}{Yam}
\define@key{fams}{njx}{Atlantic-Congo}
\define@key{fams}{kug}{Atlantic-Congo}
\define@key{fams}{mkn}{Austronesian}
\define@key{fams}{key}{Indo-European}
\define@key{fams}{nqk}{Atlantic-Congo}
\define@key{fams}{krh}{Atlantic-Congo}
\define@key{fams}{kfh}{Dravidian}
\define@key{fams}{kuj}{Atlantic-Congo}
\define@key{fams}{nbn}{Austronesian}
\define@key{fams}{kfv}{Indo-European}
\define@key{fams}{vku}{Pama-Nyungan}
\define@key{fams}{kuv}{Austronesian}
\define@key{fams}{xkz}{Sino-Tibetan}
\define@key{fams}{ktm}{Austronesian}
\define@key{fams}{kjr}{Austronesian}
\define@key{fams}{kyr}{Tupian}
\define@key{fams}{kus}{Atlantic-Congo}
\define@key{fams}{ksg}{Austronesian}
\define@key{fams}{kuh}{Afro-Asiatic}
\define@key{fams}{ksv}{Atlantic-Congo}
\define@key{fams}{ght}{Sino-Tibetan}
\define@key{fams}{kub}{Atlantic-Congo}
\define@key{fams}{xut}{Pama-Nyungan}
\define@key{fams}{kpa}{Afro-Asiatic}
\define@key{fams}{khj}{Atlantic-Congo}
\define@key{fams}{kdc}{Atlantic-Congo}
\define@key{fams}{uky}{Pama-Nyungan}
\define@key{fams}{lku}{Pama-Nyungan}
\define@key{fams}{olu}{Atlantic-Congo}
\define@key{fams}{cwt}{Atlantic-Congo}
\define@key{fams}{blh}{Kru}
\define@key{fams}{kdt}{Austroasiatic}
\define@key{fams}{fkv}{Uralic}
\define@key{fams}{kwb}{Atlantic-Congo}
\define@key{fams}{bko}{Atlantic-Congo}
\define@key{fams}{kwz}{Khoe-Kwadi}
\define@key{fams}{wka}{Afro-Asiatic}
\define@key{fams}{kdz}{Atlantic-Congo}
\define@key{fams}{kwu}{Atlantic-Congo}
\define@key{fams}{qwt}{Athabaskan-Eyak-Tlingit}
\define@key{fams}{kmq}{Koman}
\define@key{fams}{ktf}{Atlantic-Congo}
\define@key{fams}{kwm}{Atlantic-Congo}
\define@key{fams}{okk}{Nuclear Torricelli}
\define@key{fams}{knp}{Atlantic-Congo}
\define@key{fams}{kwj}{Sepik}
\define@key{fams}{kvi}{Afro-Asiatic}
\define@key{fams}{xdo}{Atlantic-Congo}
\define@key{fams}{kwf}{Austronesian}
\define@key{fams}{kop}{Nuclear Trans New Guinea}
\define@key{fams}{kya}{Atlantic-Congo}
\define@key{fams}{cwe}{Atlantic-Congo}
\define@key{fams}{xwr}{Greater Kwerba}
\define@key{fams}{kkb}{Lakes Plain}
\define@key{fams}{kwr}{Nuclear Trans New Guinea}
\define@key{fams}{kws}{Atlantic-Congo}
\define@key{fams}{kwt}{Tor-Orya}
\define@key{fams}{kuc}{Tor-Orya}
\define@key{fams}{kww}{Indo-European}
\define@key{fams}{bka}{Atlantic-Congo}
\define@key{fams}{tye}{Mande}
\define@key{fams}{kql}{Yuat}
\define@key{fams}{ldn}{Artificial Language}
\define@key{fams}{bwj}{Atlantic-Congo}
\define@key{fams}{ldi}{Atlantic-Congo}
\define@key{fams}{lbb}{Austronesian}
\define@key{fams}{lbi}{Speech Register}
\define@key{fams}{jku}{Atlantic-Congo}
\define@key{fams}{ypb}{Sino-Tibetan}
\define@key{fams}{mwi}{Austronesian}
\define@key{fams}{dtb}{Austronesian}
\define@key{fams}{zpl}{Otomanguean}
\define@key{fams}{zpa}{Otomanguean}
\define@key{fams}{lkl}{Nuclear Torricelli}
\define@key{fams}{lgh}{Sino-Tibetan}
\define@key{fams}{lgb}{Austronesian}
\define@key{fams}{lhh}{Austronesian}
\define@key{fams}{lhn}{Austronesian}
\define@key{fams}{lhl}{Indo-European}
\define@key{fams}{lhi}{Sino-Tibetan}
\define@key{fams}{lmx}{Atlantic-Congo}
\define@key{fams}{lji}{Austronesian}
\define@key{fams}{lap}{Central Sudanic}
\define@key{fams}{lka}{Austronesian}
\define@key{fams}{lkh}{Sino-Tibetan}
\define@key{fams}{lki}{Indo-European}
\define@key{fams}{lkn}{Austronesian}
\define@key{fams}{lkd}{Nambiquaran}
\define@key{fams}{lxm}{Austronesian}
\define@key{fams}{lla}{Atlantic-Congo}
\define@key{fams}{leb}{Atlantic-Congo}
\define@key{fams}{cnl}{Otomanguean}
\define@key{fams}{las}{Atlantic-Congo}
\define@key{fams}{lmr}{Austronesian}
\define@key{fams}{lmq}{Austronesian}
\define@key{fams}{lai}{Atlantic-Congo}
\define@key{fams}{lmy}{Austronesian}
\define@key{fams}{quf}{Quechuan}
\define@key{fams}{lbn}{Austroasiatic}
\define@key{fams}{bma}{Atlantic-Congo}
\define@key{fams}{ldh}{Atlantic-Congo}
\define@key{fams}{lmk}{Sino-Tibetan}
\define@key{fams}{lev}{Timor-Alor-Pantar}
\define@key{fams}{lmg}{Austronesian}
\define@key{fams}{abl}{Austronesian}
\define@key{fams}{llh}{Sino-Tibetan}
\define@key{fams}{ruu}{Austronesian}
\define@key{fams}{ldm}{Atlantic-Congo}
\define@key{fams}{sfb}{Sign Language}
\define@key{fams}{yln}{Tai-Kadai}
\define@key{fams}{lna}{Atlantic-Congo}
\define@key{fams}{lno}{Nilotic}
\define@key{fams}{lnm}{Keram}
\define@key{fams}{lnh}{Austroasiatic}
\define@key{fams}{lwm}{Sino-Tibetan}
\define@key{fams}{ztl}{Otomanguean}
\define@key{fams}{laa}{Austronesian}
\define@key{fams}{lrt}{Austronesian}
\define@key{fams}{lrv}{Austronesian}
\define@key{fams}{hmd}{Hmong-Mien}
\define@key{fams}{lrl}{Indo-European}
\define@key{fams}{lro}{Heibanic}
\define@key{fams}{lar}{Atlantic-Congo}
\define@key{fams}{lan}{Atlantic-Congo}
\define@key{fams}{llm}{Austronesian}
\define@key{fams}{lsa}{Indo-European}
\define@key{fams}{lsi}{Sino-Tibetan}
\define@key{fams}{lss}{Indo-European}
\define@key{fams}{lat}{Indo-European}
\define@key{fams}{ltu}{Austronesian}
\define@key{fams}{ltn}{Nambiquaran}
\define@key{fams}{lsl}{Sign Language}
\define@key{fams}{llx}{Austronesian}
\define@key{fams}{luf}{Mailuan}
\define@key{fams}{lre}{Iroquoian}
\define@key{fams}{clt}{Sino-Tibetan}
\define@key{fams}{lbv}{Austronesian}
\define@key{fams}{lbx}{Austronesian}
\define@key{fams}{lvi}{Austroasiatic}
\define@key{fams}{tgi}{Austronesian}
\define@key{fams}{lwu}{Sino-Tibetan}
\define@key{fams}{lya}{Sino-Tibetan}
\define@key{fams}{ldk}{Atlantic-Congo}
\define@key{fams}{lfa}{Atlantic-Congo}
\define@key{fams}{lgm}{Atlantic-Congo}
\define@key{fams}{lcc}{Austronesian}
\define@key{fams}{cae}{Atlantic-Congo}
\define@key{fams}{tql}{Austronesian}
\define@key{fams}{urr}{Austronesian}
\define@key{fams}{lzn}{Sino-Tibetan}
\define@key{fams}{lek}{Austronesian}
\define@key{fams}{llk}{Austronesian}
\define@key{fams}{lel}{Atlantic-Congo}
\define@key{fams}{llc}{Mande}
\define@key{fams}{lpa}{Austronesian}
\define@key{fams}{lle}{Austronesian}
\define@key{fams}{leq}{Nuclear Trans New Guinea}
\define@key{fams}{lrz}{Austronesian}
\define@key{fams}{lei}{Nuclear Trans New Guinea}
\define@key{fams}{xle}{Unclassifiable}
\define@key{fams}{ldj}{Atlantic-Congo}
\define@key{fams}{ley}{Austronesian}
\define@key{fams}{lej}{Atlantic-Congo}
\define@key{fams}{lgr}{Austronesian}
\define@key{fams}{lgi}{Austronesian}
\define@key{fams}{leh}{Atlantic-Congo}
\define@key{fams}{ler}{Austronesian}
\define@key{fams}{ldg}{Atlantic-Congo}
\define@key{fams}{lpe}{Lepki-Murkim-Kembra}
\define@key{fams}{xlp}{Indo-European}
\define@key{fams}{gnh}{Atlantic-Congo}
\define@key{fams}{let}{Austronesian}
\define@key{fams}{nms}{Austronesian}
\define@key{fams}{leo}{Atlantic-Congo}
\define@key{fams}{lvu}{Austronesian}
\define@key{fams}{lwe}{Austronesian}
\define@key{fams}{lwt}{Austronesian}
\define@key{fams}{ayi}{Atlantic-Congo}
\define@key{fams}{lhp}{Sino-Tibetan}
\define@key{fams}{lix}{Austronesian}
\define@key{fams}{njn}{Sino-Tibetan}
\define@key{fams}{zln}{Tai-Kadai}
\define@key{fams}{ste}{Austronesian}
\define@key{fams}{lir}{Pidgin}
\define@key{fams}{liz}{Atlantic-Congo}
\define@key{fams}{liq}{Afro-Asiatic}
\define@key{fams}{lbs}{Sign Language}
\define@key{fams}{lig}{Mande}
\define@key{fams}{lgz}{Atlantic-Congo}
\define@key{fams}{lih}{Austronesian}
\define@key{fams}{mgi}{Atlantic-Congo}
\define@key{fams}{lik}{Atlantic-Congo}
\define@key{fams}{lie}{Atlantic-Congo}
\define@key{fams}{lio}{Austronesian}
\define@key{fams}{kxx}{Atlantic-Congo}
\define@key{fams}{lib}{Austronesian}
\define@key{fams}{kwc}{Atlantic-Congo}
\define@key{fams}{lll}{Bogia}
\define@key{fams}{bme}{Atlantic-Congo}
\define@key{fams}{lim}{Indo-European}
\define@key{fams}{lmp}{Atlantic-Congo}
\define@key{fams}{ylm}{Sino-Tibetan}
\define@key{fams}{kmk}{Austronesian}
\define@key{fams}{qlm}{Indo-European}
\define@key{fams}{klw}{Austronesian}
\define@key{fams}{pml}{Pidgin}
\define@key{fams}{onb}{Tai-Kadai}
\define@key{fams}{lgk}{Austronesian}
\define@key{fams}{lfn}{Artificial Language}
\define@key{fams}{ljl}{Austronesian}
\define@key{fams}{apl}{Athabaskan-Eyak-Tlingit}
\define@key{fams}{lpo}{Sino-Tibetan}
\define@key{fams}{lcs}{Austronesian}
\define@key{fams}{lcl}{Austronesian}
\define@key{fams}{lsh}{Sino-Tibetan}
\define@key{fams}{lsd}{Afro-Asiatic}
\define@key{fams}{lzh}{Sino-Tibetan}
\define@key{fams}{lls}{Sign Language}
\define@key{fams}{lzl}{Austronesian}
\define@key{fams}{zlj}{Tai-Kadai}
\define@key{fams}{zlq}{Tai-Kadai}
\define@key{fams}{olo}{Uralic}
\define@key{fams}{loq}{Atlantic-Congo}
\define@key{fams}{lbm}{Indo-European}
\define@key{fams}{lgq}{Atlantic-Congo}
\define@key{fams}{rag}{Atlantic-Congo}
\define@key{fams}{liu}{Dajuic}
\define@key{fams}{lof}{Heibanic}
\define@key{fams}{src}{Indo-European}
\define@key{fams}{qvj}{Quechuan}
\define@key{fams}{jbo}{Artificial Language}
\define@key{fams}{yaz}{Atlantic-Congo}
\define@key{fams}{lky}{Nilotic}
\define@key{fams}{lcd}{Austronesian}
\define@key{fams}{llq}{Austronesian}
\define@key{fams}{llg}{Austronesian}
\define@key{fams}{ycl}{Sino-Tibetan}
\define@key{fams}{llb}{Atlantic-Congo}
\define@key{fams}{loa}{North Halmahera}
\define@key{fams}{rmi}{Speech Register}
\define@key{fams}{loi}{Atlantic-Congo}
\define@key{fams}{lmv}{Austronesian}
\define@key{fams}{lmi}{Central Sudanic}
\define@key{fams}{lmo}{Indo-European}
\define@key{fams}{loo}{Atlantic-Congo}
\define@key{fams}{ngl}{Atlantic-Congo}
\define@key{fams}{lce}{Austronesian}
\define@key{fams}{lpn}{Sino-Tibetan}
\define@key{fams}{wok}{Atlantic-Congo}
\define@key{fams}{lnu}{Atlantic-Congo}
\define@key{fams}{ttw}{Austronesian}
\define@key{fams}{ldo}{Atlantic-Congo}
\define@key{fams}{lop}{Atlantic-Congo}
\define@key{fams}{lpx}{Nilotic}
\define@key{fams}{lrn}{Austronesian}
\define@key{fams}{spq}{Indo-European}
\define@key{fams}{lnn}{Austronesian}
\define@key{fams}{uvl}{Austronesian}
\define@key{fams}{lht}{Austronesian}
\define@key{fams}{dtr}{Austronesian}
\define@key{fams}{lou}{Indo-European}
\define@key{fams}{lox}{Austronesian}
\define@key{fams}{xlo}{Algic}
\define@key{fams}{sli}{Indo-European}
\define@key{fams}{tto}{Austroasiatic}
\define@key{fams}{nsb}{Tuu}
\define@key{fams}{kml}{Austronesian}
\define@key{fams}{cea}{Salishan}
\define@key{fams}{axl}{Pama-Nyungan}
\define@key{fams}{ztp}{Otomanguean}
\define@key{fams}{kcc}{Atlantic-Congo}
\define@key{fams}{lcf}{Austronesian}
\define@key{fams}{knb}{Austronesian}
\define@key{fams}{luq}{Atlantic-Congo}
\define@key{fams}{lud}{Uralic}
\define@key{fams}{ldq}{Atlantic-Congo}
\define@key{fams}{ruf}{Atlantic-Congo}
\define@key{fams}{lcq}{Austronesian}
\define@key{fams}{lum}{Atlantic-Congo}
\define@key{fams}{dop}{Atlantic-Congo}
\define@key{fams}{smj}{Uralic}
\define@key{fams}{lmz}{Unattested}
\define@key{fams}{lup}{Atlantic-Congo}
\define@key{fams}{lmd}{Narrow Talodi}
\define@key{fams}{luk}{Sino-Tibetan}
\define@key{fams}{luj}{Atlantic-Congo}
\define@key{fams}{lga}{Austronesian}
\define@key{fams}{luw}{Atlantic-Congo}
\define@key{fams}{hml}{Hmong-Mien}
\define@key{fams}{ldd}{Afro-Asiatic}
\define@key{fams}{lse}{Atlantic-Congo}
\define@key{fams}{xls}{Indo-European}
\define@key{fams}{ndy}{Central Sudanic}
\define@key{fams}{luv}{Indo-European}
\define@key{fams}{lyn}{Atlantic-Congo}
\define@key{fams}{lwa}{Atlantic-Congo}
\define@key{fams}{xlc}{Indo-European}
\define@key{fams}{xld}{Indo-European}
\define@key{fams}{lyg}{Austroasiatic}
\define@key{fams}{cma}{Austroasiatic}
\define@key{fams}{mew}{Afro-Asiatic}
\define@key{fams}{ymm}{Afro-Asiatic}
\define@key{fams}{mmz}{Atlantic-Congo}
\define@key{fams}{mfz}{Nilotic}
\define@key{fams}{mqa}{Austronesian}
\define@key{fams}{kkg}{Austronesian}
\define@key{fams}{muj}{Afro-Asiatic}
\define@key{fams}{mcl}{Tucanoan}
\define@key{fams}{mzs}{Indo-European}
\define@key{fams}{mvw}{Atlantic-Congo}
\define@key{fams}{jmc}{Atlantic-Congo}
\define@key{fams}{mpd}{Arawakan}
\define@key{fams}{wpc}{Saliban}
\define@key{fams}{mzc}{Sign Language}
\define@key{fams}{mmx}{Austronesian}
\define@key{fams}{xmx}{Austronesian}
\define@key{fams}{grg}{Nuclear Trans New Guinea}
\define@key{fams}{kmd}{Austronesian}
\define@key{fams}{mme}{Austronesian}
\define@key{fams}{itt}{Austronesian}
\define@key{fams}{maf}{Afro-Asiatic}
\define@key{fams}{mkv}{Austronesian}
\define@key{fams}{sgb}{Austronesian}
\define@key{fams}{mtw}{Austronesian}
\define@key{fams}{xtm}{Otomanguean}
\define@key{fams}{gmd}{Atlantic-Congo}
\define@key{fams}{blx}{Austronesian}
\define@key{fams}{gkd}{Nuclear Trans New Guinea}
\define@key{fams}{gmg}{Nuclear Trans New Guinea}
\define@key{fams}{gmx}{Atlantic-Congo}
\define@key{fams}{zgr}{Austronesian}
\define@key{fams}{bfz}{Indo-European}
\define@key{fams}{mjx}{Austroasiatic}
\define@key{fams}{pmh}{Indo-European}
\define@key{fams}{mjy}{Algic}
\define@key{fams}{mhb}{Atlantic-Congo}
\define@key{fams}{mzz}{Austronesian}
\define@key{fams}{tnh}{Nuclear Trans New Guinea}
\define@key{fams}{sks}{Nuclear Trans New Guinea}
\define@key{fams}{mmm}{Austronesian}
\define@key{fams}{vmf}{Indo-European}
\define@key{fams}{cwb}{Atlantic-Congo}
\define@key{fams}{xkl}{Austronesian}
\define@key{fams}{mum}{Austronesian}
\define@key{fams}{wmm}{Austronesian}
\define@key{fams}{mti}{Dagan}
\define@key{fams}{xmj}{Afro-Asiatic}
\define@key{fams}{mmj}{Austroasiatic}
\define@key{fams}{mjz}{Indo-European}
\define@key{fams}{mfp}{Austronesian}
\define@key{fams}{aup}{Anim}
\define@key{fams}{mkg}{Tai-Kadai}
\define@key{fams}{vmk}{Atlantic-Congo}
\define@key{fams}{xmc}{Atlantic-Congo}
\define@key{fams}{vmw}{Atlantic-Congo}
\define@key{fams}{mhm}{Atlantic-Congo}
\define@key{fams}{xsq}{Atlantic-Congo}
\define@key{fams}{pbl}{Atlantic-Congo}
\define@key{fams}{zmh}{Baining}
\define@key{fams}{jmn}{Sino-Tibetan}
\define@key{fams}{lva}{Austronesian}
\define@key{fams}{mpu}{Tupian}
\define@key{fams}{ymk}{Atlantic-Congo}
\define@key{fams}{umn}{Sino-Tibetan}
\define@key{fams}{lon}{Atlantic-Congo}
\define@key{fams}{xml}{Sign Language}
\define@key{fams}{ima}{Dravidian}
\define@key{fams}{ymr}{Dravidian}
\define@key{fams}{mjo}{Dravidian}
\define@key{fams}{mjr}{Dravidian}
\define@key{fams}{mjq}{Dravidian}
\define@key{fams}{mjp}{Dravidian}
\define@key{fams}{ruy}{Unattested}
\define@key{fams}{swk}{Atlantic-Congo}
\define@key{fams}{ccm}{Austronesian}
\define@key{fams}{mln}{Austronesian}
\define@key{fams}{mqz}{Austronesian}
\define@key{fams}{mmt}{Austronesian}
\define@key{fams}{ped}{Nuclear Trans New Guinea}
\define@key{fams}{mkr}{Nuclear Trans New Guinea}
\define@key{fams}{lws}{Artificial Language}
\define@key{fams}{bfo}{Atlantic-Congo}
\define@key{fams}{pkt}{Austroasiatic}
\define@key{fams}{mdc}{Nuclear Trans New Guinea}
\define@key{fams}{gut}{Chibchan}
\define@key{fams}{mlx}{Austronesian}
\define@key{fams}{vml}{Pama-Nyungan}
\define@key{fams}{mxf}{Afro-Asiatic}
\define@key{fams}{mgq}{Atlantic-Congo}
\define@key{fams}{mzd}{Atlantic-Congo}
\define@key{fams}{mli}{Austronesian}
\define@key{fams}{mlf}{Austroasiatic}
\define@key{fams}{mbk}{Austronesian}
\define@key{fams}{mkb}{Indo-European}
\define@key{fams}{mdl}{Sign Language}
\define@key{fams}{mll}{Austronesian}
\define@key{fams}{mup}{Indo-European}
\define@key{fams}{myk}{Atlantic-Congo}
\define@key{fams}{mma}{Atlantic-Congo}
\define@key{fams}{mhf}{Nuclear Trans New Guinea}
\define@key{fams}{wmd}{Nambiquaran}
\define@key{fams}{mvd}{Austronesian}
\define@key{fams}{mgm}{Austronesian}
\define@key{fams}{kdf}{Austronesian}
\define@key{fams}{mqx}{Austronesian}
\define@key{fams}{znk}{Unattested}
\define@key{fams}{mjl}{Indo-European}
\define@key{fams}{mha}{Dravidian}
\define@key{fams}{zma}{Western Daly}
\define@key{fams}{zmk}{Pama-Nyungan}
\define@key{fams}{mgs}{Atlantic-Congo}
\define@key{fams}{mqu}{Nilotic}
\define@key{fams}{tbf}{Austronesian}
\define@key{fams}{mqr}{Tor-Orya}
\define@key{fams}{aax}{Nuclear Trans New Guinea}
\define@key{fams}{bwp}{Nuclear Trans New Guinea}
\define@key{fams}{mht}{Arawakan}
\define@key{fams}{zng}{Austroasiatic}
\define@key{fams}{zme}{Giimbiyu}
\define@key{fams}{mem}{Pama-Nyungan}
\define@key{fams}{myj}{Atlantic-Congo}
\define@key{fams}{mdk}{Central Sudanic}
\define@key{fams}{kby}{Saharan}
\define@key{fams}{mrv}{Austronesian}
\define@key{fams}{mbh}{Austronesian}
\define@key{fams}{mmo}{Austronesian}
\define@key{fams}{zns}{Afro-Asiatic}
\define@key{fams}{xkb}{Atlantic-Congo}
\define@key{fams}{mqp}{Austronesian}
\define@key{fams}{nlm}{Indo-European}
\define@key{fams}{mml}{Austroasiatic}
\define@key{fams}{mjv}{Dravidian}
\define@key{fams}{woo}{Austronesian}
\define@key{fams}{msw}{Atlantic-Congo}
\define@key{fams}{msk}{Austronesian}
\define@key{fams}{nty}{Sino-Tibetan}
\define@key{fams}{myg}{Atlantic-Congo}
\define@key{fams}{kxf}{Sino-Tibetan}
\define@key{fams}{wha}{Austronesian}
\define@key{fams}{mxc}{Atlantic-Congo}
\define@key{fams}{mny}{Atlantic-Congo}
\define@key{fams}{mzj}{Mande}
\define@key{fams}{mzv}{Atlantic-Congo}
\define@key{fams}{mmd}{Tai-Kadai}
\define@key{fams}{mjn}{Nuclear Trans New Guinea}
\define@key{fams}{mlh}{Nuclear Trans New Guinea}
\define@key{fams}{mnm}{Dagan}
\define@key{fams}{mpy}{Austronesian}
\define@key{fams}{mpw}{Arawakan}
\define@key{fams}{bzh}{Austronesian}
\define@key{fams}{sjm}{Austronesian}
\define@key{fams}{vmh}{Indo-European}
\define@key{fams}{nma}{Sino-Tibetan}
\define@key{fams}{lrm}{Atlantic-Congo}
\define@key{fams}{lri}{Atlantic-Congo}
\define@key{fams}{mgb}{Tamaic}
\define@key{fams}{mvr}{Austronesian}
\define@key{fams}{mrs}{Austronesian}
\define@key{fams}{mpg}{Afro-Asiatic}
\define@key{fams}{dsz}{Sign Language}
\define@key{fams}{vmr}{Atlantic-Congo}
\define@key{fams}{mrx}{Tor-Orya}
\define@key{fams}{mvu}{Maban}
\define@key{fams}{mhg}{Marrku-Wurrugu}
\define@key{fams}{qvm}{Quechuan}
\define@key{fams}{mfm}{Afro-Asiatic}
\define@key{fams}{nsr}{Sign Language}
\define@key{fams}{mrr}{Dravidian}
\define@key{fams}{nng}{Sino-Tibetan}
\define@key{fams}{zmm}{Western Daly}
\define@key{fams}{zmj}{Western Daly}
\define@key{fams}{zmd}{Western Daly}
\define@key{fams}{zmy}{Western Daly}
\define@key{fams}{mrb}{Austronesian}
\define@key{fams}{dad}{Austronesian}
\define@key{fams}{hob}{Austronesian}
\define@key{fams}{mqi}{Austronesian}
\define@key{fams}{mbx}{Sepik}
\define@key{fams}{mds}{Manubaran}
\define@key{fams}{msp}{Tupian}
\define@key{fams}{enb}{Nilotic}
\define@key{fams}{rkm}{Mande}
\define@key{fams}{mvo}{Austronesian}
\define@key{fams}{xru}{Western Daly}
\define@key{fams}{mre}{Sign Language}
\define@key{fams}{zmg}{Western Daly}
\define@key{fams}{mzr}{Pano-Tacanan}
\define@key{fams}{mve}{Indo-European}
\define@key{fams}{rwr}{Indo-European}
\define@key{fams}{myx}{Atlantic-Congo}
\define@key{fams}{tis}{Austronesian}
\define@key{fams}{bks}{Austronesian}
\define@key{fams}{msb}{Austronesian}
\define@key{fams}{mho}{Atlantic-Congo}
\define@key{fams}{jms}{Atlantic-Congo}
\define@key{fams}{cuj}{Arawakan}
\define@key{fams}{ism}{Austronesian}
\define@key{fams}{bnf}{Austronesian}
\define@key{fams}{msh}{Austronesian}
\define@key{fams}{klv}{Austronesian}
\define@key{fams}{msv}{Afro-Asiatic}
\define@key{fams}{mes}{Afro-Asiatic}
\define@key{fams}{mdg}{Maban}
\define@key{fams}{mvs}{Isolate}
\define@key{fams}{mtn}{Misumalpan}
\define@key{fams}{mfh}{Afro-Asiatic}
\define@key{fams}{xmt}{Austronesian}
\define@key{fams}{mgv}{Atlantic-Congo}
\define@key{fams}{mqe}{Nuclear Trans New Guinea}
\define@key{fams}{mzo}{Cariban}
\define@key{fams}{mtm}{Uralic}
\define@key{fams}{met}{Austronesian}
\define@key{fams}{axg}{Isolate}
\define@key{fams}{stj}{Mande}
\define@key{fams}{cty}{Dravidian}
\define@key{fams}{lsy}{Sign Language}
\define@key{fams}{mhl}{Nuclear Trans New Guinea}
\define@key{fams}{wma}{Unattested}
\define@key{fams}{mjj}{Nuclear Trans New Guinea}
\define@key{fams}{mcz}{Nuclear Trans New Guinea}
\define@key{fams}{mcw}{Afro-Asiatic}
\define@key{fams}{mgk}{Isolate}
\define@key{fams}{mxl}{Atlantic-Congo}
\define@key{fams}{xmy}{Pama-Nyungan}
\define@key{fams}{sym}{Mande}
\define@key{fams}{mnt}{Pama-Nyungan}
\define@key{fams}{ifu}{Austronesian}
\define@key{fams}{mzl}{Mixe-Zoque}
\define@key{fams}{zpy}{Otomanguean}
\define@key{fams}{vmz}{Otomanguean}
\define@key{fams}{dkx}{Afro-Asiatic}
\define@key{fams}{mdp}{Atlantic-Congo}
\define@key{fams}{mgn}{Atlantic-Congo}
\define@key{fams}{zmz}{Atlantic-Congo}
\define@key{fams}{mxg}{Atlantic-Congo}
\define@key{fams}{zmn}{Atlantic-Congo}
\define@key{fams}{zmv}{Pama-Nyungan}
\define@key{fams}{mvl}{Pama-Nyungan}
\define@key{fams}{gwa}{Atlantic-Congo}
\define@key{fams}{mdn}{Atlantic-Congo}
\define@key{fams}{xmd}{Afro-Asiatic}
\define@key{fams}{mfo}{Atlantic-Congo}
\define@key{fams}{mql}{Atlantic-Congo}
\define@key{fams}{zms}{Atlantic-Congo}
\define@key{fams}{emz}{Atlantic-Congo}
\define@key{fams}{mbo}{Atlantic-Congo}
\define@key{fams}{zmw}{Atlantic-Congo}
\define@key{fams}{moi}{Atlantic-Congo}
\define@key{fams}{mdu}{Atlantic-Congo}
\define@key{fams}{xmb}{Atlantic-Congo}
\define@key{fams}{bgu}{Atlantic-Congo}
\define@key{fams}{mxo}{Atlantic-Congo}
\define@key{fams}{mka}{Atlantic-Congo}
\define@key{fams}{mgz}{Atlantic-Congo}
\define@key{fams}{mhw}{Atlantic-Congo}
\define@key{fams}{mqb}{Afro-Asiatic}
\define@key{fams}{bpc}{Atlantic-Congo}
\define@key{fams}{mbv}{Atlantic-Congo}
\define@key{fams}{mbu}{Atlantic-Congo}
\define@key{fams}{mlb}{Atlantic-Congo}
\define@key{fams}{mgy}{Atlantic-Congo}
\define@key{fams}{mck}{Atlantic-Congo}
\define@key{fams}{bbt}{Afro-Asiatic}
\define@key{fams}{muc}{Atlantic-Congo}
\define@key{fams}{mfu}{Atlantic-Congo}
\define@key{fams}{gun}{Tupian}
\define@key{fams}{mjm}{Austronesian}
\define@key{fams}{dmf}{Speech Register}
\define@key{fams}{mue}{Mixed Language}
\define@key{fams}{mud}{Eskimo-Aleut}
\define@key{fams}{byv}{Atlantic-Congo}
\define@key{fams}{mfj}{Afro-Asiatic}
\define@key{fams}{mef}{Austroasiatic}
\define@key{fams}{ruq}{Indo-European}
\define@key{fams}{mmh}{Arawakan}
\define@key{fams}{mvk}{Yuat}
\define@key{fams}{msf}{Nimboranic}
\define@key{fams}{hkn}{Austroasiatic}
\define@key{fams}{mfx}{Ta-Ne-Omotic}
\define@key{fams}{med}{Nuclear Trans New Guinea}
\define@key{fams}{mby}{Indo-European}
\define@key{fams}{mfd}{Atlantic-Congo}
\define@key{fams}{xkd}{Austronesian}
\define@key{fams}{sim}{Sepik}
\define@key{fams}{xmg}{Atlantic-Congo}
\define@key{fams}{mee}{Austronesian}
\define@key{fams}{mea}{Atlantic-Congo}
\define@key{fams}{mvx}{Austronesian}
\define@key{fams}{mxm}{Austronesian}
\define@key{fams}{lmb}{Austronesian}
\define@key{fams}{meq}{Afro-Asiatic}
\define@key{fams}{mrm}{Austronesian}
\define@key{fams}{xmr}{Isolate}
\define@key{fams}{mnu}{Mairasic}
\define@key{fams}{mer}{Atlantic-Congo}
\define@key{fams}{wry}{Indo-European}
\define@key{fams}{iyo}{Atlantic-Congo}
\define@key{fams}{mci}{Nuclear Trans New Guinea}
\define@key{fams}{zim}{Afro-Asiatic}
\define@key{fams}{mys}{Afro-Asiatic}
\define@key{fams}{mvz}{Afro-Asiatic}
\define@key{fams}{cms}{Indo-European}
\define@key{fams}{mgo}{Atlantic-Congo}
\define@key{fams}{mxv}{Otomanguean}
\define@key{fams}{mtr}{Indo-European}
\define@key{fams}{wtm}{Indo-European}
\define@key{fams}{mfs}{Sign Language}
\define@key{fams}{zmf}{Atlantic-Congo}
\define@key{fams}{nfu}{Atlantic-Congo}
\define@key{fams}{zam}{Otomanguean}
\define@key{fams}{pla}{Nuclear Trans New Guinea}
\define@key{fams}{xmi}{Unattested}
\define@key{fams}{mmc}{Otomanguean}
\define@key{fams}{enm}{Indo-European}
\define@key{fams}{gml}{Indo-European}
\define@key{fams}{dum}{Indo-European}
\define@key{fams}{mpl}{Austronesian}
\define@key{fams}{gmh}{Indo-European}
\define@key{fams}{ltc}{Sino-Tibetan}
\define@key{fams}{xng}{Mongolic-Khitan}
\define@key{fams}{dnt}{Nuclear Trans New Guinea}
\define@key{fams}{bjo}{Atlantic-Congo}
\define@key{fams}{mpp}{Nuclear Trans New Guinea}
\define@key{fams}{ymh}{Sino-Tibetan}
\define@key{fams}{mlj}{Afro-Asiatic}
\define@key{fams}{iml}{Coosan}
\define@key{fams}{imy}{Indo-European}
\define@key{fams}{mcv}{Anim}
\define@key{fams}{inm}{Afro-Asiatic}
\define@key{fams}{mnp}{Sino-Tibetan}
\define@key{fams}{mpn}{Austronesian}
\define@key{fams}{drc}{Indo-European}
\define@key{fams}{mko}{Atlantic-Congo}
\define@key{fams}{vmg}{Austronesian}
\define@key{fams}{wii}{Nuclear Torricelli}
\define@key{fams}{xxm}{Isolate}
\define@key{fams}{omn}{Unclassifiable}
\define@key{fams}{mqq}{Austronesian}
\define@key{fams}{mnq}{Austroasiatic}
\define@key{fams}{mzt}{Austroasiatic}
\define@key{fams}{czo}{Sino-Tibetan}
\define@key{fams}{zgm}{Tai-Kadai}
\define@key{fams}{yiq}{Sino-Tibetan}
\define@key{fams}{mwl}{Indo-European}
\define@key{fams}{mvh}{Afro-Asiatic}
\define@key{fams}{mmv}{Tucanoan}
\define@key{fams}{rsm}{Sign Language}
\define@key{fams}{mjs}{Afro-Asiatic}
\define@key{fams}{mpx}{Austronesian}
\define@key{fams}{vmm}{Otomanguean}
\define@key{fams}{mwu}{Central Sudanic}
\define@key{fams}{mpo}{Austronesian}
\define@key{fams}{vmi}{Worrorran}
\define@key{fams}{mfg}{Mande}
\define@key{fams}{mix}{Otomanguean}
\define@key{fams}{mvi}{Japonic}
\define@key{fams}{ehs}{Sign Language}
\define@key{fams}{soy}{Atlantic-Congo}
\define@key{fams}{lhs}{Afro-Asiatic}
\define@key{fams}{kja}{Nimboranic}
\define@key{fams}{mlo}{Atlantic-Congo}
\define@key{fams}{mmu}{Atlantic-Congo}
\define@key{fams}{bfm}{Atlantic-Congo}
\define@key{fams}{mfq}{Atlantic-Congo}
\define@key{fams}{mod}{Pidgin}
\define@key{fams}{ahm}{Kru}
\define@key{fams}{jkm}{Sino-Tibetan}
\define@key{fams}{mhn}{Indo-European}
\define@key{fams}{mhc}{Mayan}
\define@key{fams}{gbn}{Central Sudanic}
\define@key{fams}{mxd}{Austronesian}
\define@key{fams}{mqo}{North Halmahera}
\define@key{fams}{mvq}{Nuclear Trans New Guinea}
\define@key{fams}{mou}{Afro-Asiatic}
\define@key{fams}{mof}{Algic}
\define@key{fams}{mow}{Atlantic-Congo}
\define@key{fams}{mxn}{West Bird's Head}
\define@key{fams}{mkp}{Yareban}
\define@key{fams}{mwz}{Atlantic-Congo}
\define@key{fams}{ymi}{Sino-Tibetan}
\define@key{fams}{mft}{Austronesian}
\define@key{fams}{mwt}{Austronesian}
\define@key{fams}{mqt}{Austroasiatic}
\define@key{fams}{mkm}{Austronesian}
\define@key{fams}{mkl}{Atlantic-Congo}
\define@key{fams}{vms}{Unattested}
\define@key{fams}{pwm}{Austronesian}
\define@key{fams}{vsi}{Sign Language}
\define@key{fams}{bxc}{Atlantic-Congo}
\define@key{fams}{mox}{Austronesian}
\define@key{fams}{zmo}{Eastern Jebel}
\define@key{fams}{msl}{Isolate}
\define@key{fams}{mlw}{Afro-Asiatic}
\define@key{fams}{myl}{Austronesian}
\define@key{fams}{msz}{Nuclear Trans New Guinea}
\define@key{fams}{dmb}{Dogon}
\define@key{fams}{mmb}{Somahai}
\define@key{fams}{ver}{Atlantic-Congo}
\define@key{fams}{mzg}{Sign Language}
\define@key{fams}{npn}{Austronesian}
\define@key{fams}{msr}{Sign Language}
\define@key{fams}{mgt}{Keram}
\define@key{fams}{mom}{Otomanguean}
\define@key{fams}{moo}{Austroasiatic}
\define@key{fams}{mru}{Atlantic-Congo}
\define@key{fams}{mnh}{Atlantic-Congo}
\define@key{fams}{nmh}{Sino-Tibetan}
\define@key{fams}{mtl}{Afro-Asiatic}
\define@key{fams}{gwg}{Atlantic-Congo}
\define@key{fams}{crm}{Algic}
\define@key{fams}{msg}{West Bird's Head}
\define@key{fams}{mze}{Mailuan}
\define@key{fams}{moq}{Isolate}
\define@key{fams}{msx}{Nuclear Trans New Guinea}
\define@key{fams}{xmo}{Unattested}
\define@key{fams}{xmz}{Austronesian}
\define@key{fams}{mzq}{Austronesian}
\define@key{fams}{mdb}{Kiwaian}
\define@key{fams}{xms}{Sign Language}
\define@key{fams}{bdo}{Central Sudanic}
\define@key{fams}{mgc}{Central Sudanic}
\define@key{fams}{mrp}{Austronesian}
\define@key{fams}{mqn}{Austronesian}
\define@key{fams}{mrl}{Austronesian}
\define@key{fams}{mwy}{Nilotic}
\define@key{fams}{mqv}{Nuclear Trans New Guinea}
\define@key{fams}{mtj}{East Bird's Head}
\define@key{fams}{mtt}{Austronesian}
\define@key{fams}{mwh}{Austronesian}
\define@key{fams}{jmw}{Turama-Kikori}
\define@key{fams}{ity}{Austronesian}
\define@key{fams}{nmo}{Sino-Tibetan}
\define@key{fams}{mzy}{Sign Language}
\define@key{fams}{mxi}{Indo-European}
\define@key{fams}{xnq}{Atlantic-Congo}
\define@key{fams}{mpi}{Afro-Asiatic}
\define@key{fams}{mcx}{Atlantic-Congo}
\define@key{fams}{mpz}{Sino-Tibetan}
\define@key{fams}{pnd}{Atlantic-Congo}
\define@key{fams}{mgg}{Atlantic-Congo}
\define@key{fams}{mpa}{Atlantic-Congo}
\define@key{fams}{mvt}{Austronesian}
\define@key{fams}{zmp}{Atlantic-Congo}
\define@key{fams}{cmr}{Sino-Tibetan}
\define@key{fams}{mro}{Sino-Tibetan}
\define@key{fams}{kqx}{Afro-Asiatic}
\define@key{fams}{agz}{Austronesian}
\define@key{fams}{atl}{Austronesian}
\define@key{fams}{mtd}{Austronesian}
\define@key{fams}{tsx}{Anim}
\define@key{fams}{mub}{Afro-Asiatic}
\define@key{fams}{ymd}{Sino-Tibetan}
\define@key{fams}{gau}{Dravidian}
\define@key{fams}{udg}{Dravidian}
\define@key{fams}{vmd}{Dravidian}
\define@key{fams}{wiv}{Austronesian}
\define@key{fams}{muk}{Sino-Tibetan}
\define@key{fams}{mmk}{Dravidian}
\define@key{fams}{mfw}{Kwalean}
\define@key{fams}{kpb}{Dravidian}
\define@key{fams}{vmu}{Pama-Nyungan}
\define@key{fams}{kqa}{Nuclear Trans New Guinea}
\define@key{fams}{mwq}{Sino-Tibetan}
\define@key{fams}{boe}{Atlantic-Congo}
\define@key{fams}{mmf}{Afro-Asiatic}
\define@key{fams}{mth}{Austronesian}
\define@key{fams}{mpv}{Nuclear Trans New Guinea}
\define@key{fams}{mtc}{Nuclear Trans New Guinea}
\define@key{fams}{myr}{Isolate}
\define@key{fams}{mnj}{Indo-European}
\define@key{fams}{asx}{Nuclear Trans New Guinea}
\define@key{fams}{mxr}{Austronesian}
\define@key{fams}{rmh}{Lepki-Murkim-Kembra}
\define@key{fams}{tkv}{Austronesian}
\define@key{fams}{mqw}{Nuclear Trans New Guinea}
\define@key{fams}{smm}{Indo-European}
\define@key{fams}{mmi}{Nuclear Trans New Guinea}
\define@key{fams}{mmq}{Nuclear Trans New Guinea}
\define@key{fams}{mse}{Afro-Asiatic}
\define@key{fams}{mui}{Austronesian}
\define@key{fams}{mje}{Afro-Asiatic}
\define@key{fams}{muv}{Dravidian}
\define@key{fams}{tuc}{Austronesian}
\define@key{fams}{muy}{Afro-Asiatic}
\define@key{fams}{ymz}{Sino-Tibetan}
\define@key{fams}{mcj}{Atlantic-Congo}
\define@key{fams}{mxh}{Central Sudanic}
\define@key{fams}{wlc}{Atlantic-Congo}
\define@key{fams}{wmw}{Atlantic-Congo}
\define@key{fams}{moa}{Mande}
\define@key{fams}{mwa}{Austronesian}
\define@key{fams}{mjh}{Atlantic-Congo}
\define@key{fams}{mws}{Atlantic-Congo}
\define@key{fams}{gmy}{Indo-European}
\define@key{fams}{nme}{Sino-Tibetan}
\define@key{fams}{nbt}{Sino-Tibetan}
\define@key{fams}{nao}{Sino-Tibetan}
\define@key{fams}{mne}{Central Sudanic}
\define@key{fams}{mty}{Nuclear Torricelli}
\define@key{fams}{ncd}{Sino-Tibetan}
\define@key{fams}{srf}{Austronesian}
\define@key{fams}{nxx}{Sentanic}
\define@key{fams}{jbn}{Afro-Asiatic}
\define@key{fams}{nbg}{Unattested}
\define@key{fams}{nxe}{Austronesian}
\define@key{fams}{ngv}{Atlantic-Congo}
\define@key{fams}{nlx}{Indo-European}
\define@key{fams}{nhh}{Indo-European}
\define@key{fams}{ars}{Afro-Asiatic}
\define@key{fams}{nae}{Austronesian}
\define@key{fams}{nib}{Nuclear Trans New Guinea}
\define@key{fams}{nkj}{Nuclear Trans New Guinea}
\define@key{fams}{nbk}{Nuclear Trans New Guinea}
\define@key{fams}{mff}{Atlantic-Congo}
\define@key{fams}{nax}{Left May}
\define@key{fams}{nlc}{Nuclear Trans New Guinea}
\define@key{fams}{nss}{Austronesian}
\define@key{fams}{nlz}{Austronesian}
\define@key{fams}{ylo}{Sino-Tibetan}
\define@key{fams}{naj}{Atlantic-Congo}
\define@key{fams}{nmx}{Yam}
\define@key{fams}{nkm}{Yam}
\define@key{fams}{nmk}{Austronesian}
\define@key{fams}{nmq}{Atlantic-Congo}
\define@key{fams}{ncm}{Yam}
\define@key{fams}{neo}{Unclassifiable}
\define@key{fams}{nbs}{Sign Language}
\define@key{fams}{nvm}{Koiarian}
\define@key{fams}{naa}{Namla-Tofanma}
\define@key{fams}{mxw}{Yam}
\define@key{fams}{nmt}{Austronesian}
\define@key{fams}{bwb}{Austronesian}
\define@key{fams}{nmy}{Sino-Tibetan}
\define@key{fams}{nnc}{Afro-Asiatic}
\define@key{fams}{nzz}{Dogon}
\define@key{fams}{ngr}{Austronesian}
\define@key{fams}{cox}{Arawakan}
\define@key{fams}{afk}{Arafundi}
\define@key{fams}{qvo}{Quechuan}
\define@key{fams}{nrg}{Austronesian}
\define@key{fams}{nac}{Nuclear Trans New Guinea}
\define@key{fams}{loh}{Surmic}
\define@key{fams}{nnr}{Pama-Nyungan}
\define@key{fams}{nsy}{Austronesian}
\define@key{fams}{nvh}{Austronesian}
\define@key{fams}{ntz}{Indo-European}
\define@key{fams}{nte}{Atlantic-Congo}
\define@key{fams}{nti}{Atlantic-Congo}
\define@key{fams}{nxa}{Austronesian}
\define@key{fams}{ncn}{Austronesian}
\define@key{fams}{nwo}{Pama-Nyungan}
\define@key{fams}{nsw}{Austronesian}
\define@key{fams}{nwr}{Yareban}
\define@key{fams}{nwa}{Algic}
\define@key{fams}{nmz}{Atlantic-Congo}
\define@key{fams}{naw}{Atlantic-Congo}
\define@key{fams}{nyq}{Indo-European}
\define@key{fams}{noz}{Dizoid}
\define@key{fams}{ncr}{Atlantic-Congo}
\define@key{fams}{nlu}{Atlantic-Congo}
\define@key{fams}{gke}{Atlantic-Congo}
\define@key{fams}{ndk}{Atlantic-Congo}
\define@key{fams}{ndh}{Atlantic-Congo}
\define@key{fams}{ndj}{Atlantic-Congo}
\define@key{fams}{ndm}{Afro-Asiatic}
\define@key{fams}{nxo}{Atlantic-Congo}
\define@key{fams}{nnz}{Atlantic-Congo}
\define@key{fams}{nda}{Atlantic-Congo}
\define@key{fams}{ndc}{Atlantic-Congo}
\define@key{fams}{nml}{Atlantic-Congo}
\define@key{fams}{ndg}{Atlantic-Congo}
\define@key{fams}{dne}{Atlantic-Congo}
\define@key{fams}{ndd}{Atlantic-Congo}
\define@key{fams}{eli}{Narrow Talodi}
\define@key{fams}{ndw}{Atlantic-Congo}
\define@key{fams}{nbb}{Atlantic-Congo}
\define@key{fams}{ndl}{Atlantic-Congo}
\define@key{fams}{ndq}{Atlantic-Congo}
\define@key{fams}{nqm}{Kolopom}
\define@key{fams}{ndr}{Atlantic-Congo}
\define@key{fams}{ndp}{Central Sudanic}
\define@key{fams}{dno}{Central Sudanic}
\define@key{fams}{ndx}{Nuclear Trans New Guinea}
\define@key{fams}{nuh}{Atlantic-Congo}
\define@key{fams}{nww}{Atlantic-Congo}
\define@key{fams}{njt}{Pidgin}
\define@key{fams}{wni}{Atlantic-Congo}
\define@key{fams}{nec}{Timor-Alor-Pantar}
\define@key{fams}{nef}{Pidgin}
\define@key{fams}{dcr}{Indo-European}
\define@key{fams}{nkg}{Nuclear Trans New Guinea}
\define@key{fams}{nif}{Nuclear Trans New Guinea}
\define@key{fams}{nej}{Nuclear Trans New Guinea}
\define@key{fams}{nek}{Austronesian}
\define@key{fams}{nex}{Yam}
\define@key{fams}{nem}{Austronesian}
\define@key{fams}{nqn}{Yam}
\define@key{fams}{neu}{Artificial Language}
\define@key{fams}{nsp}{Sign Language}
\define@key{fams}{net}{Nuclear Trans New Guinea}
\define@key{fams}{jas}{Austronesian}
\define@key{fams}{jui}{Pama-Nyungan}
\define@key{fams}{nnf}{Nuclear Trans New Guinea}
\define@key{fams}{hlt}{Sino-Tibetan}
\define@key{fams}{szb}{Nuclear Trans New Guinea}
\define@key{fams}{nud}{Ndu}
\define@key{fams}{nmv}{Pama-Nyungan}
\define@key{fams}{nbv}{Atlantic-Congo}
\define@key{fams}{nmc}{Central Sudanic}
\define@key{fams}{nbh}{Afro-Asiatic}
\define@key{fams}{nyx}{Pama-Nyungan}
\define@key{fams}{gng}{Atlantic-Congo}
\define@key{fams}{nne}{Atlantic-Congo}
\define@key{fams}{nxd}{Atlantic-Congo}
\define@key{fams}{ngd}{Atlantic-Congo}
\define@key{fams}{nji}{Mirndi}
\define@key{fams}{rxd}{Pama-Nyungan}
\define@key{fams}{nsg}{Nilotic}
\define@key{fams}{ngm}{Speech Register}
\define@key{fams}{cnw}{Sino-Tibetan}
\define@key{fams}{zdj}{Atlantic-Congo}
\define@key{fams}{ngg}{Atlantic-Congo}
\define@key{fams}{jgb}{Atlantic-Congo}
\define@key{fams}{nbd}{Atlantic-Congo}
\define@key{fams}{nuu}{Atlantic-Congo}
\define@key{fams}{gnj}{Mande}
\define@key{fams}{nql}{Atlantic-Congo}
\define@key{fams}{ngt}{Austroasiatic}
\define@key{fams}{nnn}{Afro-Asiatic}
\define@key{fams}{nbq}{Nuclear Trans New Guinea}
\define@key{fams}{ngx}{Afro-Asiatic}
\define@key{fams}{nnh}{Atlantic-Congo}
\define@key{fams}{ngj}{Atlantic-Congo}
\define@key{fams}{nnq}{Atlantic-Congo}
\define@key{fams}{nra}{Atlantic-Congo}
\define@key{fams}{nla}{Atlantic-Congo}
\define@key{fams}{jgo}{Atlantic-Congo}
\define@key{fams}{noq}{Atlantic-Congo}
\define@key{fams}{nsh}{Atlantic-Congo}
\define@key{fams}{nuw}{Austronesian}
\define@key{fams}{ngp}{Atlantic-Congo}
\define@key{fams}{nlo}{Atlantic-Congo}
\define@key{fams}{xnm}{Nyulnyulan}
\define@key{fams}{nui}{Atlantic-Congo}
\define@key{fams}{nue}{Atlantic-Congo}
\define@key{fams}{ndn}{Atlantic-Congo}
\define@key{fams}{ngz}{Atlantic-Congo}
\define@key{fams}{nuo}{Austroasiatic}
\define@key{fams}{nrx}{Unattested}
\define@key{fams}{nbx}{Pama-Nyungan}
\define@key{fams}{ngq}{Atlantic-Congo}
\define@key{fams}{ngw}{Afro-Asiatic}
\define@key{fams}{nwe}{Atlantic-Congo}
\define@key{fams}{ngn}{Atlantic-Congo}
\define@key{fams}{yrl}{Tupian}
\define@key{fams}{nhf}{Pama-Nyungan}
\define@key{fams}{ncs}{Sign Language}
\define@key{fams}{nsi}{Sign Language}
\define@key{fams}{mzk}{Atlantic-Congo}
\define@key{fams}{nii}{Nuclear Trans New Guinea}
\define@key{fams}{xny}{Pama-Nyungan}
\define@key{fams}{gbe}{Sepik}
\define@key{fams}{nim}{Atlantic-Congo}
\define@key{fams}{nil}{Austronesian}
\define@key{fams}{noe}{Indo-European}
\define@key{fams}{nmp}{Nyulnyulan}
\define@key{fams}{nmr}{Atlantic-Congo}
\define@key{fams}{nis}{Nuclear Trans New Guinea}
\define@key{fams}{nmw}{Austronesian}
\define@key{fams}{niw}{Left May}
\define@key{fams}{nxi}{Atlantic-Congo}
\define@key{fams}{nxr}{Nuclear Trans New Guinea}
\define@key{fams}{nby}{Border}
\define@key{fams}{nlk}{Nuclear Trans New Guinea}
\define@key{fams}{nin}{Atlantic-Congo}
\define@key{fams}{nps}{Nuclear Trans New Guinea}
\define@key{fams}{njs}{Geelvink Bay}
\define@key{fams}{yso}{Sino-Tibetan}
\define@key{fams}{nkp}{Austronesian}
\define@key{fams}{njl}{Dajuic}
\define@key{fams}{nzb}{Atlantic-Congo}
\define@key{fams}{njj}{Atlantic-Congo}
\define@key{fams}{njr}{Atlantic-Congo}
\define@key{fams}{njy}{Atlantic-Congo}
\define@key{fams}{nkq}{Atlantic-Congo}
\define@key{fams}{nkn}{Atlantic-Congo}
\define@key{fams}{nkz}{Atlantic-Congo}
\define@key{fams}{khu}{Atlantic-Congo}
\define@key{fams}{nqo}{Artificial Language}
\define@key{fams}{nkc}{Atlantic-Congo}
\define@key{fams}{nkx}{Ijoid}
\define@key{fams}{nka}{Atlantic-Congo}
\define@key{fams}{nbo}{Atlantic-Congo}
\define@key{fams}{nkw}{Atlantic-Congo}
\define@key{fams}{nbp}{Atlantic-Congo}
\define@key{fams}{ngh}{Tuu}
\define@key{fams}{gaw}{Nuclear Trans New Guinea}
\define@key{fams}{noi}{Indo-European}
\define@key{fams}{nkk}{Austronesian}
\define@key{fams}{lem}{Atlantic-Congo}
\define@key{fams}{nof}{Nuclear Trans New Guinea}
\define@key{fams}{noh}{Nuclear Trans New Guinea}
\define@key{fams}{zhn}{Tai-Kadai}
\define@key{fams}{noj}{Huitotoan}
\define@key{fams}{nok}{Salishan}
\define@key{fams}{nrc}{Indo-European}
\define@key{fams}{nrp}{Unclassifiable}
\define@key{fams}{huj}{Hmong-Mien}
\define@key{fams}{hmp}{Hmong-Mien}
\define@key{fams}{crl}{Algic}
\define@key{fams}{pbu}{Indo-European}
\define@key{fams}{hno}{Indo-European}
\define@key{fams}{glh}{Indo-European}
\define@key{fams}{aee}{Indo-European}
\define@key{fams}{kxm}{Austroasiatic}
\define@key{fams}{atv}{Turkic}
\define@key{fams}{azj}{Turkic}
\define@key{fams}{ghh}{Sino-Tibetan}
\define@key{fams}{ymx}{Sino-Tibetan}
\define@key{fams}{yiv}{Sino-Tibetan}
\define@key{fams}{cng}{Sino-Tibetan}
\define@key{fams}{bfc}{Sino-Tibetan}
\define@key{fams}{nnl}{Sino-Tibetan}
\define@key{fams}{lbr}{Sino-Tibetan}
\define@key{fams}{tji}{Sino-Tibetan}
\define@key{fams}{doc}{Tai-Kadai}
\define@key{fams}{nod}{Tai-Kadai}
\define@key{fams}{tts}{Tai-Kadai}
\define@key{fams}{hea}{Hmong-Mien}
\define@key{fams}{hmi}{Hmong-Mien}
\define@key{fams}{kqs}{Atlantic-Congo}
\define@key{fams}{fll}{Atlantic-Congo}
\define@key{fams}{dgi}{Atlantic-Congo}
\define@key{fams}{tsp}{Atlantic-Congo}
\define@key{fams}{gbo}{Kru}
\define@key{fams}{dip}{Nilotic}
\define@key{fams}{diw}{Nilotic}
\define@key{fams}{max}{Austronesian}
\define@key{fams}{mmg}{Austronesian}
\define@key{fams}{mrq}{Austronesian}
\define@key{fams}{tnn}{Austronesian}
\define@key{fams}{una}{Austronesian}
\define@key{fams}{bcd}{Austronesian}
\define@key{fams}{weo}{Austronesian}
\define@key{fams}{nni}{Austronesian}
\define@key{fams}{aqn}{Austronesian}
\define@key{fams}{xnn}{Austronesian}
\define@key{fams}{cts}{Austronesian}
\define@key{fams}{stb}{Austronesian}
\define@key{fams}{bmm}{Austronesian}
\define@key{fams}{onr}{Nuclear Torricelli}
\define@key{fams}{kti}{Nuclear Trans New Guinea}
\define@key{fams}{nks}{Nuclear Trans New Guinea}
\define@key{fams}{yir}{Nuclear Trans New Guinea}
\define@key{fams}{whg}{Nuclear Trans New Guinea}
\define@key{fams}{kiw}{Kiwaian}
\define@key{fams}{ryn}{Japonic}
\define@key{fams}{neq}{Mixe-Zoque}
\define@key{fams}{scs}{Athabaskan-Eyak-Tlingit}
\define@key{fams}{esk}{Eskimo-Aleut}
\define@key{fams}{thh}{Uto-Aztecan}
\define@key{fams}{nhy}{Uto-Aztecan}
\define@key{fams}{ojb}{Algic}
\define@key{fams}{pef}{Pomoan}
\define@key{fams}{cst}{Miwok-Costanoan}
\define@key{fams}{enl}{Lengua-Mascoy}
\define@key{fams}{qvz}{Quechuan}
\define@key{fams}{qul}{Quechuan}
\define@key{fams}{qxn}{Quechuan}
\define@key{fams}{pmq}{Otomanguean}
\define@key{fams}{xtn}{Otomanguean}
\define@key{fams}{mxa}{Otomanguean}
\define@key{fams}{mfk}{Afro-Asiatic}
\define@key{fams}{ayp}{Afro-Asiatic}
\define@key{fams}{ntd}{Austronesian}
\define@key{fams}{cnp}{Sino-Tibetan}
\define@key{fams}{ncq}{Austroasiatic}
\define@key{fams}{bly}{Atlantic-Congo}
\define@key{fams}{ncf}{Austronesian}
\define@key{fams}{ntw}{Iroquoian}
\define@key{fams}{nov}{Artificial Language}
\define@key{fams}{noy}{Atlantic-Congo}
\define@key{fams}{asj}{Atlantic-Congo}
\define@key{fams}{nsc}{Unattested}
\define@key{fams}{nsx}{Atlantic-Congo}
\define@key{fams}{baf}{Atlantic-Congo}
\define@key{fams}{kte}{Sino-Tibetan}
\define@key{fams}{wbm}{Austroasiatic}
\define@key{fams}{bsq}{Kru}
\define@key{fams}{wla}{Walioic}
\define@key{fams}{wgi}{Nuclear Trans New Guinea}
\define@key{fams}{gyz}{Afro-Asiatic}
\define@key{fams}{nqt}{Afro-Asiatic}
\define@key{fams}{nnv}{Pama-Nyungan}
\define@key{fams}{noc}{Nuclear Trans New Guinea}
\define@key{fams}{klt}{Nuclear Trans New Guinea}
\define@key{fams}{nuq}{Austronesian}
\define@key{fams}{nur}{Austronesian}
\define@key{fams}{nuc}{Pano-Tacanan}
\define@key{fams}{nbr}{Atlantic-Congo}
\define@key{fams}{nop}{Nuclear Trans New Guinea}
\define@key{fams}{sij}{Austronesian}
\define@key{fams}{tgs}{Austronesian}
\define@key{fams}{kdk}{Austronesian}
\define@key{fams}{nxm}{Unclassifiable}
\define@key{fams}{nug}{Mirndi}
\define@key{fams}{rin}{Atlantic-Congo}
\define@key{fams}{nul}{Austronesian}
\define@key{fams}{nwb}{Kru}
\define@key{fams}{nev}{Austroasiatic}
\define@key{fams}{nyy}{Atlantic-Congo}
\define@key{fams}{nlj}{Atlantic-Congo}
\define@key{fams}{mwn}{Atlantic-Congo}
\define@key{fams}{nwm}{Central Sudanic}
\define@key{fams}{nmi}{Afro-Asiatic}
\define@key{fams}{nny}{Tangkic}
\define@key{fams}{nyb}{Atlantic-Congo}
\define@key{fams}{nyc}{Atlantic-Congo}
\define@key{fams}{nyk}{Atlantic-Congo}
\define@key{fams}{nnj}{Nilotic}
\define@key{fams}{sev}{Atlantic-Congo}
\define@key{fams}{nba}{Atlantic-Congo}
\define@key{fams}{neh}{Sino-Tibetan}
\define@key{fams}{nye}{Atlantic-Congo}
\define@key{fams}{nyl}{Austroasiatic}
\define@key{fams}{nyr}{Atlantic-Congo}
\define@key{fams}{nkv}{Atlantic-Congo}
\define@key{fams}{nkt}{Atlantic-Congo}
\define@key{fams}{nyg}{Atlantic-Congo}
\define@key{fams}{lid}{Austronesian}
\define@key{fams}{nvo}{Atlantic-Congo}
\define@key{fams}{nuj}{Atlantic-Congo}
\define@key{fams}{muo}{Atlantic-Congo}
\define@key{fams}{nyd}{Atlantic-Congo}
\define@key{fams}{nyu}{Atlantic-Congo}
\define@key{fams}{nzd}{Atlantic-Congo}
\define@key{fams}{nzy}{Atlantic-Congo}
\define@key{fams}{nja}{Afro-Asiatic}
\define@key{fams}{nzi}{Atlantic-Congo}
\define@key{fams}{bzy}{Atlantic-Congo}
\define@key{fams}{obi}{Chumashan}
\define@key{fams}{obl}{Atlantic-Congo}
\define@key{fams}{obo}{Austronesian}
\define@key{fams}{obu}{Atlantic-Congo}
\define@key{fams}{zac}{Otomanguean}
\define@key{fams}{odk}{Indo-European}
\define@key{fams}{bhf}{Isolate}
\define@key{fams}{kkc}{East Strickland}
\define@key{fams}{odu}{Atlantic-Congo}
\define@key{fams}{tyh}{Austroasiatic}
\define@key{fams}{opy}{Nuclear-Macro-Je}
\define@key{fams}{ofo}{Siouan}
\define@key{fams}{ogc}{Atlantic-Congo}
\define@key{fams}{ogg}{Atlantic-Congo}
\define@key{fams}{eri}{Nuclear Trans New Guinea}
\define@key{fams}{oia}{Timor-Alor-Pantar}
\define@key{fams}{chj}{Otomanguean}
\define@key{fams}{oki}{Nilotic}
\define@key{fams}{okn}{Japonic}
\define@key{fams}{okb}{Atlantic-Congo}
\define@key{fams}{okd}{Ijoid}
\define@key{fams}{oks}{Atlantic-Congo}
\define@key{fams}{okj}{Great Andamanese}
\define@key{fams}{kqv}{Austronesian}
\define@key{fams}{oie}{Nilotic}
\define@key{fams}{opa}{Atlantic-Congo}
\define@key{fams}{okx}{Atlantic-Congo}
\define@key{fams}{oke}{Atlantic-Congo}
\define@key{fams}{oar}{Afro-Asiatic}
\define@key{fams}{obr}{Sino-Tibetan}
\define@key{fams}{och}{Sino-Tibetan}
\define@key{fams}{odt}{Indo-European}
\define@key{fams}{ang}{Indo-European}
\define@key{fams}{fro}{Indo-European}
\define@key{fams}{ofs}{Indo-European}
\define@key{fams}{oge}{Kartvelian}
\define@key{fams}{goh}{Indo-European}
\define@key{fams}{sga}{Indo-European}
\define@key{fams}{ojp}{Japonic}
\define@key{fams}{okl}{Sign Language}
\define@key{fams}{qok}{Austroasiatic}
\define@key{fams}{qkn}{Dravidian}
\define@key{fams}{qbb}{Indo-European}
\define@key{fams}{omx}{Austroasiatic}
\define@key{fams}{omr}{Indo-European}
\define@key{fams}{non}{Indo-European}
\define@key{fams}{onw}{Nubian}
\define@key{fams}{oos}{Indo-European}
\define@key{fams}{pro}{Indo-European}
\define@key{fams}{peo}{Indo-European}
\define@key{fams}{orv}{Indo-European}
\define@key{fams}{osp}{Indo-European}
\define@key{fams}{osx}{Indo-European}
\define@key{fams}{oty}{Dravidian}
\define@key{fams}{oui}{Turkic}
\define@key{fams}{owl}{Indo-European}
\define@key{fams}{ole}{Sino-Tibetan}
\define@key{fams}{olm}{Atlantic-Congo}
\define@key{fams}{lul}{Central Sudanic}
\define@key{fams}{iko}{Atlantic-Congo}
\define@key{fams}{acx}{Afro-Asiatic}
\define@key{fams}{oml}{Atlantic-Congo}
\define@key{fams}{nht}{Uto-Aztecan}
\define@key{fams}{omi}{Central Sudanic}
\define@key{fams}{omt}{Nilotic}
\define@key{fams}{omu}{Isolate}
\define@key{fams}{oog}{Austroasiatic}
\define@key{fams}{onx}{Pidgin}
\define@key{fams}{oni}{Austronesian}
\define@key{fams}{onj}{Dagan}
\define@key{fams}{onn}{Bosavi}
\define@key{fams}{oor}{Indo-European}
\define@key{fams}{opo}{Eleman}
\define@key{fams}{opt}{Uto-Aztecan}
\define@key{fams}{lgn}{Koman}
\define@key{fams}{orn}{Austronesian}
\define@key{fams}{ors}{Austronesian}
\define@key{fams}{sdr}{Indo-European}
\define@key{fams}{org}{Atlantic-Congo}
\define@key{fams}{nlv}{Uto-Aztecan}
\define@key{fams}{fnb}{Austronesian}
\define@key{fams}{orc}{Afro-Asiatic}
\define@key{fams}{orz}{Austronesian}
\define@key{fams}{ora}{Austronesian}
\define@key{fams}{orx}{Atlantic-Congo}
\define@key{fams}{orh}{Tungusic}
\define@key{fams}{bpk}{Austronesian}
\define@key{fams}{orw}{Chapacuran}
\define@key{fams}{orr}{Ijoid}
\define@key{fams}{syx}{Atlantic-Congo}
\define@key{fams}{ost}{Atlantic-Congo}
\define@key{fams}{osc}{Indo-European}
\define@key{fams}{osi}{Austronesian}
\define@key{fams}{oso}{Atlantic-Congo}
\define@key{fams}{uta}{Atlantic-Congo}
\define@key{fams}{otd}{Austronesian}
\define@key{fams}{oti}{Isolate}
\define@key{fams}{otw}{Algic}
\define@key{fams}{lot}{Nilotic}
\define@key{fams}{otu}{Bororoan}
\define@key{fams}{oum}{Austronesian}
\define@key{fams}{oue}{South Bougainville}
\define@key{fams}{stn}{Austronesian}
\define@key{fams}{wsr}{Nuclear Trans New Guinea}
\define@key{fams}{oyy}{Austronesian}
\define@key{fams}{oyd}{Ta-Ne-Omotic}
\define@key{fams}{zao}{Otomanguean}
\define@key{fams}{chz}{Otomanguean}
\define@key{fams}{pfa}{Austronesian}
\define@key{fams}{sig}{Atlantic-Congo}
\define@key{fams}{qvp}{Quechuan}
\define@key{fams}{pcp}{Pano-Tacanan}
\define@key{fams}{pdi}{Tai-Kadai}
\define@key{fams}{pkc}{Unclassifiable}
\define@key{fams}{pae}{Atlantic-Congo}
\define@key{fams}{pgi}{Border}
\define@key{fams}{phr}{Indo-European}
\define@key{fams}{phj}{Sino-Tibetan}
\define@key{fams}{lgt}{Sepik}
\define@key{fams}{phv}{Indo-European}
\define@key{fams}{pal}{Indo-European}
\define@key{fams}{pha}{Hmong-Mien}
\define@key{fams}{pri}{Austronesian}
\define@key{fams}{ppi}{Cochimi-Yuman}
\define@key{fams}{qpp}{Indo-European}
\define@key{fams}{pta}{Tupian}
\define@key{fams}{pkg}{Austronesian}
\define@key{fams}{jkp}{Sino-Tibetan}
\define@key{fams}{pku}{Austronesian}
\define@key{fams}{pfl}{Indo-European}
\define@key{fams}{plq}{Indo-European}
\define@key{fams}{plr}{Atlantic-Congo}
\define@key{fams}{pln}{Indo-European}
\define@key{fams}{pnl}{Atlantic-Congo}
\define@key{fams}{pli}{Indo-European}
\define@key{fams}{pcf}{Dravidian}
\define@key{fams}{pmd}{Pama-Nyungan}
\define@key{fams}{abw}{Nuclear Trans New Guinea}
\define@key{fams}{pmc}{Unattested}
\define@key{fams}{ple}{Austronesian}
\define@key{fams}{plz}{Austronesian}
\define@key{fams}{bpx}{Indo-European}
\define@key{fams}{pmb}{Atlantic-Congo}
\define@key{fams}{pmn}{Atlantic-Congo}
\define@key{fams}{hih}{Nuclear Trans New Guinea}
\define@key{fams}{att}{Austronesian}
\define@key{fams}{pnz}{Atlantic-Congo}
\define@key{fams}{pnq}{Atlantic-Congo}
\define@key{fams}{pwb}{Atlantic-Congo}
\define@key{fams}{psn}{Austronesian}
\define@key{fams}{qxh}{Quechuan}
\define@key{fams}{lsp}{Sign Language}
\define@key{fams}{tdb}{Indo-European}
\define@key{fams}{pnp}{Austronesian}
\define@key{fams}{bkj}{Atlantic-Congo}
\define@key{fams}{pgg}{Indo-European}
\define@key{fams}{pgs}{Atlantic-Congo}
\define@key{fams}{slm}{Austronesian}
\define@key{fams}{pcg}{Dravidian}
\define@key{fams}{pnr}{Nuclear Trans New Guinea}
\define@key{fams}{pax}{Unattested}
\define@key{fams}{pkh}{Sino-Tibetan}
\define@key{fams}{paz}{Isolate}
\define@key{fams}{pnc}{Austronesian}
\define@key{fams}{knt}{Pano-Tacanan}
\define@key{fams}{pno}{Pano-Tacanan}
\define@key{fams}{blk}{Sino-Tibetan}
\define@key{fams}{ppv}{Unattested}
\define@key{fams}{ppn}{Austronesian}
\define@key{fams}{dpp}{Austronesian}
\define@key{fams}{pas}{Lakes Plain}
\define@key{fams}{pbo}{Atlantic-Congo}
\define@key{fams}{ppe}{Isolate}
\define@key{fams}{ppu}{Austronesian}
\define@key{fams}{ppm}{Austronesian}
\define@key{fams}{pgz}{Sign Language}
\define@key{fams}{prc}{Indo-European}
\define@key{fams}{pzn}{Sino-Tibetan}
\define@key{fams}{prf}{Austronesian}
\define@key{fams}{prw}{Nuclear Trans New Guinea}
\define@key{fams}{aap}{Cariban}
\define@key{fams}{pak}{Tupian}
\define@key{fams}{paf}{Tupian}
\define@key{fams}{gvp}{Nuclear-Macro-Je}
\define@key{fams}{pbg}{Arawakan}
\define@key{fams}{pys}{Sign Language}
\define@key{fams}{pcl}{Indo-European}
\define@key{fams}{pch}{Unattested}
\define@key{fams}{pcj}{Austroasiatic}
\define@key{fams}{ppt}{Kamula-Elevala}
\define@key{fams}{kvx}{Indo-European}
\define@key{fams}{xpr}{Indo-European}
\define@key{fams}{paq}{Indo-European}
\define@key{fams}{psq}{Sepik}
\define@key{fams}{yac}{Nuclear Trans New Guinea}
\define@key{fams}{ptn}{Austronesian}
\define@key{fams}{pth}{Nuclear-Macro-Je}
\define@key{fams}{pbc}{Cariban}
\define@key{fams}{pty}{Dravidian}
\define@key{fams}{ptq}{Dravidian}
\define@key{fams}{mfa}{Austronesian}
\define@key{fams}{pnk}{Arawakan}
\define@key{fams}{bfb}{Indo-European}
\define@key{fams}{psm}{Tupian}
\define@key{fams}{pmr}{Nuclear Trans New Guinea}
\define@key{fams}{pcb}{Austroasiatic}
\define@key{fams}{xpc}{Turkic}
\define@key{fams}{pai}{Atlantic-Congo}
\define@key{fams}{pfe}{Atlantic-Congo}
\define@key{fams}{ppq}{Walioic}
\define@key{fams}{pel}{Austronesian}
\define@key{fams}{bxd}{Sino-Tibetan}
\define@key{fams}{ata}{Isolate}
\define@key{fams}{pev}{Cariban}
\define@key{fams}{psg}{Sign Language}
\define@key{fams}{pek}{Austronesian}
\define@key{fams}{ums}{Austronesian}
\define@key{fams}{pdc}{Indo-European}
\define@key{fams}{pnh}{Austronesian}
\define@key{fams}{ptw}{Salishan}
\define@key{fams}{pea}{Austronesian}
\define@key{fams}{wet}{Austronesian}
\define@key{fams}{psc}{Sign Language}
\define@key{fams}{prl}{Sign Language}
\define@key{fams}{pex}{Austronesian}
\define@key{fams}{zpe}{Otomanguean}
\define@key{fams}{pey}{Indo-European}
\define@key{fams}{prt}{Austroasiatic}
\define@key{fams}{phk}{Tai-Kadai}
\define@key{fams}{phl}{Indo-European}
\define@key{fams}{ypa}{Sino-Tibetan}
\define@key{fams}{phq}{Sino-Tibetan}
\define@key{fams}{pem}{Atlantic-Congo}
\define@key{fams}{psp}{Sign Language}
\define@key{fams}{phm}{Atlantic-Congo}
\define@key{fams}{phn}{Afro-Asiatic}
\define@key{fams}{yip}{Sino-Tibetan}
\define@key{fams}{ypg}{Sino-Tibetan}
\define@key{fams}{nph}{Sino-Tibetan}
\define@key{fams}{pnx}{Austroasiatic}
\define@key{fams}{kjt}{Sino-Tibetan}
\define@key{fams}{xpg}{Indo-European}
\define@key{fams}{phu}{Tai-Kadai}
\define@key{fams}{phd}{Indo-European}
\define@key{fams}{pug}{Atlantic-Congo}
\define@key{fams}{phh}{Sino-Tibetan}
\define@key{fams}{ypm}{Sino-Tibetan}
\define@key{fams}{pho}{Sino-Tibetan}
\define@key{fams}{phg}{Austroasiatic}
\define@key{fams}{yph}{Sino-Tibetan}
\define@key{fams}{ypp}{Sino-Tibetan}
\define@key{fams}{pht}{Tai-Kadai}
\define@key{fams}{ypz}{Sino-Tibetan}
\define@key{fams}{ptr}{Austronesian}
\define@key{fams}{pin}{Sepik}
\define@key{fams}{pcd}{Indo-European}
\define@key{fams}{cpu}{Arawakan}
\define@key{fams}{xpi}{Unclassifiable}
\define@key{fams}{dep}{Pidgin}
\define@key{fams}{pij}{Unclassifiable}
\define@key{fams}{piz}{Austronesian}
\define@key{fams}{pis}{Indo-European}
\define@key{fams}{piw}{Atlantic-Congo}
\define@key{fams}{pnn}{Piawi}
\define@key{fams}{pnv}{Pama-Nyungan}
\define@key{fams}{tjp}{Pama-Nyungan}
\define@key{fams}{pic}{Atlantic-Congo}
\define@key{fams}{pti}{Pama-Nyungan}
\define@key{fams}{pny}{Atlantic-Congo}
\define@key{fams}{bxi}{Pama-Nyungan}
\define@key{fams}{pie}{Kiowa-Tanoan}
\define@key{fams}{xpa}{Pama-Nyungan}
\define@key{fams}{tpp}{Totonacan}
\define@key{fams}{pig}{Unattested}
\define@key{fams}{psy}{Algic}
\define@key{fams}{xps}{Indo-European}
\define@key{fams}{pih}{Indo-European}
\define@key{fams}{sje}{Uralic}
\define@key{fams}{pcn}{Atlantic-Congo}
\define@key{fams}{pix}{Austronesian}
\define@key{fams}{piy}{Afro-Asiatic}
\define@key{fams}{ktj}{Kru}
\define@key{fams}{pdt}{Indo-European}
\define@key{fams}{pbv}{Austroasiatic}
\define@key{fams}{npo}{Sino-Tibetan}
\define@key{fams}{pdn}{Austronesian}
\define@key{fams}{pof}{Atlantic-Congo}
\define@key{fams}{pkb}{Atlantic-Congo}
\define@key{fams}{pld}{Unclassifiable}
\define@key{fams}{plj}{Afro-Asiatic}
\define@key{fams}{pso}{Sign Language}
\define@key{fams}{plb}{Austronesian}
\define@key{fams}{pmo}{Austronesian}
\define@key{fams}{pmm}{Atlantic-Congo}
\define@key{fams}{ncc}{Austronesian}
\define@key{fams}{png}{Atlantic-Congo}
\define@key{fams}{pns}{Austronesian}
\define@key{fams}{pnt}{Indo-European}
\define@key{fams}{prh}{Austronesian}
\define@key{fams}{ptv}{Austronesian}
\define@key{fams}{pmx}{Sino-Tibetan}
\define@key{fams}{bye}{Sepik}
\define@key{fams}{pwr}{Indo-European}
\define@key{fams}{pyn}{Pano-Tacanan}
\define@key{fams}{prz}{Sign Language}
\define@key{fams}{prg}{Indo-European}
\define@key{fams}{kvj}{Afro-Asiatic}
\define@key{fams}{pux}{Sko}
\define@key{fams}{atp}{Austronesian}
\define@key{fams}{pbm}{Otomanguean}
\define@key{fams}{psl}{Sign Language}
\define@key{fams}{pkp}{Austronesian}
\define@key{fams}{pup}{Nuclear Trans New Guinea}
\define@key{fams}{pum}{Sino-Tibetan}
\define@key{fams}{xpm}{Yeniseian}
\define@key{fams}{puj}{Austronesian}
\define@key{fams}{pud}{Austronesian}
\define@key{fams}{puf}{Austronesian}
\define@key{fams}{pna}{Austronesian}
\define@key{fams}{pnm}{Austronesian}
\define@key{fams}{xpu}{Afro-Asiatic}
\define@key{fams}{qxp}{Quechuan}
\define@key{fams}{puu}{Atlantic-Congo}
\define@key{fams}{pru}{South Bird's Head Family}
\define@key{fams}{iar}{Isolate}
\define@key{fams}{puy}{Chumashan}
\define@key{fams}{prr}{Puri-Coroado}
\define@key{fams}{pur}{Tupian}
\define@key{fams}{pub}{Sino-Tibetan}
\define@key{fams}{mfl}{Afro-Asiatic}
\define@key{fams}{afe}{Atlantic-Congo}
\define@key{fams}{cpx}{Sino-Tibetan}
\define@key{fams}{pyu}{Austronesian}
\define@key{fams}{pme}{Austronesian}
\define@key{fams}{pop}{Austronesian}
\define@key{fams}{pwo}{Sino-Tibetan}
\define@key{fams}{pcw}{Afro-Asiatic}
\define@key{fams}{pye}{Kru}
\define@key{fams}{pyy}{Sino-Tibetan}
\define@key{fams}{pby}{Isolate}
\define@key{fams}{laq}{Tai-Kadai}
\define@key{fams}{qxq}{Turkic}
\define@key{fams}{xqt}{Afro-Asiatic}
\define@key{fams}{ymq}{Sino-Tibetan}
\define@key{fams}{zqe}{Tai-Kadai}
\define@key{fams}{qua}{Siouan}
\define@key{fams}{qya}{Artificial Language}
\define@key{fams}{qvy}{Sino-Tibetan}
\define@key{fams}{zpj}{Otomanguean}
\define@key{fams}{quq}{Unclassifiable}
\define@key{fams}{qun}{Salishan}
\define@key{fams}{ztq}{Otomanguean}
\define@key{fams}{rah}{Sino-Tibetan}
\define@key{fams}{xrr}{Unclassifiable}
\define@key{fams}{raz}{Austronesian}
\define@key{fams}{mqk}{Austronesian}
\define@key{fams}{rjs}{Indo-European}
\define@key{fams}{rjg}{Austronesian}
\define@key{fams}{gra}{Indo-European}
\define@key{fams}{rkh}{Austronesian}
\define@key{fams}{rki}{Sino-Tibetan}
\define@key{fams}{rai}{Austronesian}
\define@key{fams}{kjx}{North Bougainville}
\define@key{fams}{lje}{Austronesian}
\define@key{fams}{thr}{Indo-European}
\define@key{fams}{rkt}{Indo-European}
\define@key{fams}{rnl}{Sino-Tibetan}
\define@key{fams}{rax}{Atlantic-Congo}
\define@key{fams}{ray}{Austronesian}
\define@key{fams}{rpt}{Nuclear Trans New Guinea}
\define@key{fams}{lra}{Austronesian}
\define@key{fams}{rar}{Austronesian}
\define@key{fams}{rac}{Lakes Plain}
\define@key{fams}{btn}{Austronesian}
\define@key{fams}{bgd}{Indo-European}
\define@key{fams}{rtw}{Indo-European}
\define@key{fams}{rau}{Sino-Tibetan}
\define@key{fams}{yea}{Dravidian}
\define@key{fams}{jnl}{Sino-Tibetan}
\define@key{fams}{rat}{Indo-European}
\define@key{fams}{gir}{Tai-Kadai}
\define@key{fams}{atu}{Nilotic}
\define@key{fams}{ree}{Austronesian}
\define@key{fams}{rei}{Indo-European}
\define@key{fams}{bow}{Yam}
\define@key{fams}{reb}{Austronesian}
\define@key{fams}{agv}{Austronesian}
\define@key{fams}{rem}{Pano-Tacanan}
\define@key{fams}{rmp}{Nuclear Trans New Guinea}
\define@key{fams}{lkj}{Austronesian}
\define@key{fams}{rsi}{Artificial Language}
\define@key{fams}{rea}{Nuclear Trans New Guinea}
\define@key{fams}{rer}{Unattested}
\define@key{fams}{pgk}{Austronesian}
\define@key{fams}{res}{Atlantic-Congo}
\define@key{fams}{ret}{Timor-Alor-Pantar}
\define@key{fams}{rcf}{Indo-European}
\define@key{fams}{rey}{Pano-Tacanan}
\define@key{fams}{ril}{Austroasiatic}
\define@key{fams}{ria}{Sino-Tibetan}
\define@key{fams}{rir}{Austronesian}
\define@key{fams}{zar}{Otomanguean}
\define@key{fams}{rgu}{Austronesian}
\define@key{fams}{hrx}{Indo-European}
\define@key{fams}{rri}{Austronesian}
\define@key{fams}{riu}{Austronesian}
\define@key{fams}{snj}{Atlantic-Congo}
\define@key{fams}{rod}{Atlantic-Congo}
\define@key{fams}{rhg}{Indo-European}
\define@key{fams}{rge}{Indo-European}
\define@key{fams}{rms}{Sign Language}
\define@key{fams}{rgn}{Indo-European}
\define@key{fams}{rmx}{Austroasiatic}
\define@key{fams}{rmm}{Austronesian}
\define@key{fams}{rmv}{Artificial Language}
\define@key{fams}{rof}{Atlantic-Congo}
\define@key{fams}{rol}{Austronesian}
\define@key{fams}{rmk}{Lower Sepik-Ramu}
\define@key{fams}{ror}{Austronesian}
\define@key{fams}{roe}{Austronesian}
\define@key{fams}{rnn}{Austronesian}
\define@key{fams}{rga}{Austronesian}
\define@key{fams}{pce}{Austroasiatic}
\define@key{fams}{rdb}{Indo-European}
\define@key{fams}{ruh}{Sino-Tibetan}
\define@key{fams}{rbb}{Austroasiatic}
\define@key{fams}{ruz}{Unattested}
\define@key{fams}{rna}{Unattested}
\define@key{fams}{rnw}{Atlantic-Congo}
\define@key{fams}{drg}{Austronesian}
\define@key{fams}{bxr}{Mongolic-Khitan}
\define@key{fams}{rue}{Indo-European}
\define@key{fams}{ruc}{Atlantic-Congo}
\define@key{fams}{rnd}{Atlantic-Congo}
\define@key{fams}{rwk}{Atlantic-Congo}
\define@key{fams}{rsn}{Sign Language}
\define@key{fams}{sax}{Austronesian}
\define@key{fams}{sav}{Atlantic-Congo}
\define@key{fams}{raq}{Sino-Tibetan}
\define@key{fams}{lsm}{Atlantic-Congo}
\define@key{fams}{sxr}{Austronesian}
\define@key{fams}{spy}{Nilotic}
\define@key{fams}{msi}{Austronesian}
\define@key{fams}{bsy}{Austronesian}
\define@key{fams}{sae}{Nambiquaran}
\define@key{fams}{saa}{Afro-Asiatic}
\define@key{fams}{xsa}{Afro-Asiatic}
\define@key{fams}{qhr}{Indo-European}
\define@key{fams}{sbo}{Austroasiatic}
\define@key{fams}{quv}{Mayan}
\define@key{fams}{sck}{Indo-European}
\define@key{fams}{spd}{Nuclear Trans New Guinea}
\define@key{fams}{saf}{Atlantic-Congo}
\define@key{fams}{sbk}{Atlantic-Congo}
\define@key{fams}{sbm}{Atlantic-Congo}
\define@key{fams}{tga}{Atlantic-Congo}
\define@key{fams}{aec}{Afro-Asiatic}
\define@key{fams}{acf}{Indo-European}
\define@key{fams}{xsy}{Austronesian}
\define@key{fams}{sjl}{Sino-Tibetan}
\define@key{fams}{sjb}{Austronesian}
\define@key{fams}{sch}{Sino-Tibetan}
\define@key{fams}{skt}{Atlantic-Congo}
\define@key{fams}{skg}{Austronesian}
\define@key{fams}{skm}{Nuclear Trans New Guinea}
\define@key{fams}{sak}{Atlantic-Congo}
\define@key{fams}{szy}{Austronesian}
\define@key{fams}{shq}{Atlantic-Congo}
\define@key{fams}{slx}{Atlantic-Congo}
\define@key{fams}{sgu}{Austronesian}
\define@key{fams}{qxl}{Quechuan}
\define@key{fams}{mnd}{Tupian}
\define@key{fams}{slq}{Turkic}
\define@key{fams}{sau}{Austronesian}
\define@key{fams}{loe}{Austronesian}
\define@key{fams}{esn}{Sign Language}
\define@key{fams}{tmj}{Greater Kwerba}
\define@key{fams}{ysd}{Sino-Tibetan}
\define@key{fams}{smp}{Afro-Asiatic}
\define@key{fams}{xab}{Atlantic-Congo}
\define@key{fams}{smx}{Atlantic-Congo}
\define@key{fams}{ccg}{Atlantic-Congo}
\define@key{fams}{saq}{Nilotic}
\define@key{fams}{ssx}{Nuclear Trans New Guinea}
\define@key{fams}{spv}{Indo-European}
\define@key{fams}{smh}{Sino-Tibetan}
\define@key{fams}{snx}{Nuclear Trans New Guinea}
\define@key{fams}{swm}{Nuclear Trans New Guinea}
\define@key{fams}{rav}{Sino-Tibetan}
\define@key{fams}{stu}{Austroasiatic}
\define@key{fams}{smv}{Indo-European}
\define@key{fams}{ztm}{Otomanguean}
\define@key{fams}{icr}{Indo-European}
\define@key{fams}{spn}{Lengua-Mascoy}
\define@key{fams}{zpx}{Otomanguean}
\define@key{fams}{cuk}{Chibchan}
\define@key{fams}{hve}{Huavean}
\define@key{fams}{hue}{Huavean}
\define@key{fams}{mat}{Otomanguean}
\define@key{fams}{pow}{Otomanguean}
\define@key{fams}{xso}{Unclassifiable}
\define@key{fams}{sgr}{Indo-European}
\define@key{fams}{sgk}{Sino-Tibetan}
\define@key{fams}{nsa}{Sino-Tibetan}
\define@key{fams}{xsn}{Atlantic-Congo}
\define@key{fams}{sbp}{Atlantic-Congo}
\define@key{fams}{sng}{Atlantic-Congo}
\define@key{fams}{snl}{Austronesian}
\define@key{fams}{scg}{Austronesian}
\define@key{fams}{sgy}{Indo-European}
\define@key{fams}{ysy}{Sino-Tibetan}
\define@key{fams}{ysn}{Sino-Tibetan}
\define@key{fams}{sny}{Sepik}
\define@key{fams}{xtj}{Otomanguean}
\define@key{fams}{maa}{Otomanguean}
\define@key{fams}{msc}{Mande}
\define@key{fams}{pps}{Otomanguean}
\define@key{fams}{qvs}{Quechuan}
\define@key{fams}{xtp}{Otomanguean}
\define@key{fams}{trq}{Otomanguean}
\define@key{fams}{pls}{Otomanguean}
\define@key{fams}{azg}{Otomanguean}
\define@key{fams}{zpf}{Otomanguean}
\define@key{fams}{san}{Indo-European}
\define@key{fams}{ssi}{Indo-European}
\define@key{fams}{kwy}{Atlantic-Congo}
\define@key{fams}{hvv}{Huavean}
\define@key{fams}{nhz}{Uto-Aztecan}
\define@key{fams}{cok}{Uto-Aztecan}
\define@key{fams}{qus}{Quechuan}
\define@key{fams}{mza}{Otomanguean}
\define@key{fams}{mdv}{Otomanguean}
\define@key{fams}{zpn}{Otomanguean}
\define@key{fams}{ztn}{Otomanguean}
\define@key{fams}{zas}{Otomanguean}
\define@key{fams}{zpr}{Otomanguean}
\define@key{fams}{pca}{Otomanguean}
\define@key{fams}{zpt}{Otomanguean}
\define@key{fams}{scq}{Austroasiatic}
\define@key{fams}{zkp}{Nuclear-Macro-Je}
\define@key{fams}{cri}{Indo-European}
\define@key{fams}{spr}{Austronesian}
\define@key{fams}{spc}{Isolate}
\define@key{fams}{krn}{Kru}
\define@key{fams}{spi}{Lakes Plain}
\define@key{fams}{sbz}{Central Sudanic}
\define@key{fams}{kwv}{Central Sudanic}
\define@key{fams}{kwg}{Central Sudanic}
\define@key{fams}{zsa}{Austronesian}
\define@key{fams}{bps}{Austronesian}
\define@key{fams}{mbs}{Austronesian}
\define@key{fams}{sre}{Austronesian}
\define@key{fams}{sar}{Arawakan}
\define@key{fams}{srh}{Indo-European}
\define@key{fams}{mwm}{Central Sudanic}
\define@key{fams}{onp}{Sino-Tibetan}
\define@key{fams}{sdu}{Austronesian}
\define@key{fams}{sra}{Nuclear Trans New Guinea}
\define@key{fams}{swy}{Afro-Asiatic}
\define@key{fams}{sxs}{Atlantic-Congo}
\define@key{fams}{sas}{Austronesian}
\define@key{fams}{sdc}{Indo-European}
\define@key{fams}{stw}{Austronesian}
\define@key{fams}{stq}{Indo-European}
\define@key{fams}{mav}{Tupian}
\define@key{fams}{sdl}{Sign Language}
\define@key{fams}{skc}{Nuclear Trans New Guinea}
\define@key{fams}{saz}{Indo-European}
\define@key{fams}{mjt}{Dravidian}
\define@key{fams}{srt}{Geelvink Bay}
\define@key{fams}{psu}{Indo-European}
\define@key{fams}{ssj}{Nuclear Trans New Guinea}
\define@key{fams}{sao}{Isolate}
\define@key{fams}{swr}{Yawa-Saweru}
\define@key{fams}{swt}{Timor-Alor-Pantar}
\define@key{fams}{saw}{Nuclear Trans New Guinea}
\define@key{fams}{swn}{Afro-Asiatic}
\define@key{fams}{sxw}{Atlantic-Congo}
\define@key{fams}{say}{Afro-Asiatic}
\define@key{fams}{sco}{Indo-European}
\define@key{fams}{kdg}{Atlantic-Congo}
\define@key{fams}{sbx}{Austronesian}
\define@key{fams}{sib}{Austronesian}
\define@key{fams}{sec}{Salishan}
\define@key{fams}{tvw}{Austronesian}
\define@key{fams}{sos}{Mande}
\define@key{fams}{sge}{Austronesian}
\define@key{fams}{sbg}{West Bird's Head}
\define@key{fams}{seg}{Atlantic-Congo}
\define@key{fams}{sfw}{Atlantic-Congo}
\define@key{fams}{ssg}{Austronesian}
\define@key{fams}{hik}{Austronesian}
\define@key{fams}{skz}{Austronesian}
\define@key{fams}{skp}{Austronesian}
\define@key{fams}{sek}{Athabaskan-Eyak-Tlingit}
\define@key{fams}{ske}{Austronesian}
\define@key{fams}{syi}{Atlantic-Congo}
\define@key{fams}{sko}{Austronesian}
\define@key{fams}{skx}{Austronesian}
\define@key{fams}{lip}{Atlantic-Congo}
\define@key{fams}{kgi}{Sign Language}
\define@key{fams}{snw}{Atlantic-Congo}
\define@key{fams}{sws}{Austronesian}
\define@key{fams}{slg}{Austronesian}
\define@key{fams}{szc}{Austroasiatic}
\define@key{fams}{sbr}{Austronesian}
\define@key{fams}{etz}{Mairasic}
\define@key{fams}{smy}{Indo-European}
\define@key{fams}{ssm}{Austroasiatic}
\define@key{fams}{xse}{Nuclear Trans New Guinea}
\define@key{fams}{seq}{Atlantic-Congo}
\define@key{fams}{sej}{Nuclear Trans New Guinea}
\define@key{fams}{sds}{Afro-Asiatic}
\define@key{fams}{ssz}{Austronesian}
\define@key{fams}{spk}{Ndu}
\define@key{fams}{snu}{Border}
\define@key{fams}{sjs}{Afro-Asiatic}
\define@key{fams}{sni}{Pano-Tacanan}
\define@key{fams}{std}{Unattested}
\define@key{fams}{sez}{Sino-Tibetan}
\define@key{fams}{spe}{Austronesian}
\define@key{fams}{spb}{Austronesian}
\define@key{fams}{spm}{Lower Sepik-Ramu}
\define@key{fams}{iws}{Sepik}
\define@key{fams}{skr}{Indo-European}
\define@key{fams}{sry}{Austronesian}
\define@key{fams}{srr}{Atlantic-Congo}
\define@key{fams}{swf}{Atlantic-Congo}
\define@key{fams}{sve}{Austronesian}
\define@key{fams}{seu}{Austronesian}
\define@key{fams}{srw}{Austronesian}
\define@key{fams}{srk}{Austronesian}
\define@key{fams}{stf}{Nuclear Torricelli}
\define@key{fams}{stm}{Nuclear Trans New Guinea}
\define@key{fams}{sbi}{Nuclear Torricelli}
\define@key{fams}{sta}{Pidgin}
\define@key{fams}{sew}{Austronesian}
\define@key{fams}{lsw}{Sign Language}
\define@key{fams}{sze}{Blue Nile Mao}
\define@key{fams}{scw}{Afro-Asiatic}
\define@key{fams}{sdb}{Indo-European}
\define@key{fams}{srz}{Indo-European}
\define@key{fams}{sha}{Atlantic-Congo}
\define@key{fams}{xsh}{Atlantic-Congo}
\define@key{fams}{sqa}{Atlantic-Congo}
\define@key{fams}{jih}{Sino-Tibetan}
\define@key{fams}{sho}{Mande}
\define@key{fams}{swo}{Pano-Tacanan}
\define@key{fams}{ssv}{Austronesian}
\define@key{fams}{swq}{Afro-Asiatic}
\define@key{fams}{sqh}{Atlantic-Congo}
\define@key{fams}{shx}{Hmong-Mien}
\define@key{fams}{she}{Dizoid}
\define@key{fams}{sth}{Speech Register}
\define@key{fams}{shl}{Sino-Tibetan}
\define@key{fams}{scv}{Atlantic-Congo}
\define@key{fams}{bun}{Atlantic-Congo}
\define@key{fams}{kip}{Sino-Tibetan}
\define@key{fams}{ssh}{Afro-Asiatic}
\define@key{fams}{shr}{Atlantic-Congo}
\define@key{fams}{gua}{Atlantic-Congo}
\define@key{fams}{snh}{Unattested}
\define@key{fams}{sxg}{Sino-Tibetan}
\define@key{fams}{sle}{Dravidian}
\define@key{fams}{bcv}{Atlantic-Congo}
\define@key{fams}{suj}{Atlantic-Congo}
\define@key{fams}{sts}{Indo-European}
\define@key{fams}{scu}{Sino-Tibetan}
\define@key{fams}{ksa}{Unattested}
\define@key{fams}{shw}{Heibanic}
\define@key{fams}{slw}{Nuclear Trans New Guinea}
\define@key{fams}{sya}{Austronesian}
\define@key{fams}{spg}{Austronesian}
\define@key{fams}{mmp}{Amto-Musan}
\define@key{fams}{nco}{South Bougainville}
\define@key{fams}{sty}{Turkic}
\define@key{fams}{sdx}{Austronesian}
\define@key{fams}{sxc}{Unclassifiable}
\define@key{fams}{scn}{Indo-European}
\define@key{fams}{sep}{Atlantic-Congo}
\define@key{fams}{scx}{Unclassifiable}
\define@key{fams}{xsd}{Indo-European}
\define@key{fams}{sgx}{Sign Language}
\define@key{fams}{nsu}{Uto-Aztecan}
\define@key{fams}{sxe}{Atlantic-Congo}
\define@key{fams}{snr}{Nuclear Trans New Guinea}
\define@key{fams}{qws}{Quechuan}
\define@key{fams}{sky}{Austronesian}
\define@key{fams}{slt}{Sino-Tibetan}
\define@key{fams}{szl}{Indo-European}
\define@key{fams}{sbq}{Nuclear Trans New Guinea}
\define@key{fams}{mkc}{Nuclear Torricelli}
\define@key{fams}{wul}{Nuclear Trans New Guinea}
\define@key{fams}{xsp}{Nuclear Trans New Guinea}
\define@key{fams}{stv}{Afro-Asiatic}
\define@key{fams}{sie}{Atlantic-Congo}
\define@key{fams}{sbw}{Atlantic-Congo}
\define@key{fams}{smb}{Angan}
\define@key{fams}{sbb}{Austronesian}
\define@key{fams}{smg}{Baining}
\define@key{fams}{smz}{South Bougainville}
\define@key{fams}{smt}{Sino-Tibetan}
\define@key{fams}{siu}{Nuclear Torricelli}
\define@key{fams}{sbn}{Indo-European}
\define@key{fams}{xts}{Otomanguean}
\define@key{fams}{sjn}{Artificial Language}
\define@key{fams}{sgp}{Sino-Tibetan}
\define@key{fams}{sgm}{Atlantic-Congo}
\define@key{fams}{skq}{Mande}
\define@key{fams}{xti}{Otomanguean}
\define@key{fams}{snz}{Nuclear Trans New Guinea}
\define@key{fams}{sys}{Central Sudanic}
\define@key{fams}{swj}{Atlantic-Congo}
\define@key{fams}{sir}{Afro-Asiatic}
\define@key{fams}{srx}{Indo-European}
\define@key{fams}{sld}{Atlantic-Congo}
\define@key{fams}{sso}{Austronesian}
\define@key{fams}{siy}{Indo-European}
\define@key{fams}{lsv}{Sign Language}
\define@key{fams}{akp}{Atlantic-Congo}
\define@key{fams}{skw}{Indo-European}
\define@key{fams}{sms}{Uralic}
\define@key{fams}{svm}{Indo-European}
\define@key{fams}{svk}{Sign Language}
\define@key{fams}{sfm}{Hmong-Mien}
\define@key{fams}{kxq}{Yam}
\define@key{fams}{sox}{Atlantic-Congo}
\define@key{fams}{soc}{Atlantic-Congo}
\define@key{fams}{xog}{Atlantic-Congo}
\define@key{fams}{sog}{Indo-European}
\define@key{fams}{soj}{Indo-European}
\define@key{fams}{sok}{Afro-Asiatic}
\define@key{fams}{sby}{Atlantic-Congo}
\define@key{fams}{sol}{Austronesian}
\define@key{fams}{aaw}{Austronesian}
\define@key{fams}{szs}{Sign Language}
\define@key{fams}{smc}{Nuclear Trans New Guinea}
\define@key{fams}{smu}{Austroasiatic}
\define@key{fams}{sor}{Afro-Asiatic}
\define@key{fams}{kgt}{Atlantic-Congo}
\define@key{fams}{ysg}{Sino-Tibetan}
\define@key{fams}{shc}{Atlantic-Congo}
\define@key{fams}{soo}{Atlantic-Congo}
\define@key{fams}{sod}{Atlantic-Congo}
\define@key{fams}{soe}{Atlantic-Congo}
\define@key{fams}{soi}{Indo-European}
\define@key{fams}{siq}{Bosavi}
\define@key{fams}{sss}{Austroasiatic}
\define@key{fams}{urw}{Nuclear Trans New Guinea}
\define@key{fams}{sbh}{Austronesian}
\define@key{fams}{sqo}{Indo-European}
\define@key{fams}{ays}{Unattested}
\define@key{fams}{sdk}{Ndu}
\define@key{fams}{krz}{Yam}
\define@key{fams}{sfs}{Sign Language}
\define@key{fams}{nit}{Dravidian}
\define@key{fams}{hmy}{Hmong-Mien}
\define@key{fams}{hma}{Hmong-Mien}
\define@key{fams}{sdh}{Indo-European}
\define@key{fams}{bcc}{Indo-European}
\define@key{fams}{fay}{Indo-European}
\define@key{fams}{luz}{Indo-European}
\define@key{fams}{pbt}{Indo-European}
\define@key{fams}{hnd}{Indo-European}
\define@key{fams}{psh}{Indo-European}
\define@key{fams}{psi}{Indo-European}
\define@key{fams}{vro}{Uralic}
\define@key{fams}{nik}{Austroasiatic}
\define@key{fams}{mnn}{Austroasiatic}
\define@key{fams}{uzs}{Turkic}
\define@key{fams}{ghe}{Sino-Tibetan}
\define@key{fams}{ymc}{Sino-Tibetan}
\define@key{fams}{nsd}{Sino-Tibetan}
\define@key{fams}{qxs}{Sino-Tibetan}
\define@key{fams}{pmj}{Sino-Tibetan}
\define@key{fams}{bfs}{Sino-Tibetan}
\define@key{fams}{nre}{Sino-Tibetan}
\define@key{fams}{lrr}{Sino-Tibetan}
\define@key{fams}{tjs}{Sino-Tibetan}
\define@key{fams}{sou}{Tai-Kadai}
\define@key{fams}{hms}{Hmong-Mien}
\define@key{fams}{hmh}{Hmong-Mien}
\define@key{fams}{hmg}{Hmong-Mien}
\define@key{fams}{xtv}{Pama-Nyungan}
\define@key{fams}{ijs}{Ijoid}
\define@key{fams}{fal}{Atlantic-Congo}
\define@key{fams}{nbw}{Atlantic-Congo}
\define@key{fams}{lnl}{Atlantic-Congo}
\define@key{fams}{biv}{Atlantic-Congo}
\define@key{fams}{nnw}{Atlantic-Congo}
\define@key{fams}{snm}{Central Sudanic}
\define@key{fams}{dik}{Nilotic}
\define@key{fams}{dib}{Nilotic}
\define@key{fams}{dks}{Nilotic}
\define@key{fams}{bwq}{Mande}
\define@key{fams}{sbd}{Mande}
\define@key{fams}{sns}{Austronesian}
\define@key{fams}{mqm}{Austronesian}
\define@key{fams}{mcy}{Austronesian}
\define@key{fams}{vbb}{Austronesian}
\define@key{fams}{lmf}{Austronesian}
\define@key{fams}{agy}{Austronesian}
\define@key{fams}{ksc}{Austronesian}
\define@key{fams}{bln}{Austronesian}
\define@key{fams}{plv}{Austronesian}
\define@key{fams}{bzc}{Austronesian}
\define@key{fams}{osu}{Nuclear Torricelli}
\define@key{fams}{aws}{Nuclear Trans New Guinea}
\define@key{fams}{omw}{Nuclear Trans New Guinea}
\define@key{fams}{ams}{Japonic}
\define@key{fams}{hax}{Haida}
\define@key{fams}{tce}{Athabaskan-Eyak-Tlingit}
\define@key{fams}{caf}{Athabaskan-Eyak-Tlingit}
\define@key{fams}{twr}{Uto-Aztecan}
\define@key{fams}{tcu}{Uto-Aztecan}
\define@key{fams}{npl}{Uto-Aztecan}
\define@key{fams}{tla}{Uto-Aztecan}
\define@key{fams}{crj}{Algic}
\define@key{fams}{peq}{Pomoan}
\define@key{fams}{qup}{Quechuan}
\define@key{fams}{qxo}{Quechuan}
\define@key{fams}{ayc}{Aymaran}
\define@key{fams}{meh}{Otomanguean}
\define@key{fams}{mit}{Otomanguean}
\define@key{fams}{mxy}{Otomanguean}
\define@key{fams}{rgs}{Austronesian}
\define@key{fams}{giz}{Afro-Asiatic}
\define@key{fams}{cpy}{Arawakan}
\define@key{fams}{itd}{Austronesian}
\define@key{fams}{csp}{Sino-Tibetan}
\define@key{fams}{sct}{Austroasiatic}
\define@key{fams}{sqq}{Austroasiatic}
\define@key{fams}{sww}{Austronesian}
\define@key{fams}{sow}{Border}
\define@key{fams}{vmq}{Otomanguean}
\define@key{fams}{vmp}{Otomanguean}
\define@key{fams}{sqs}{Sign Language}
\define@key{fams}{sci}{Austronesian}
\define@key{fams}{seo}{Isolate}
\define@key{fams}{swp}{Austronesian}
\define@key{fams}{sxb}{Atlantic-Congo}
\define@key{fams}{ssc}{Atlantic-Congo}
\define@key{fams}{sut}{Otomanguean}
\define@key{fams}{apd}{Afro-Asiatic}
\define@key{fams}{pga}{Afro-Asiatic}
\define@key{fams}{sgi}{Atlantic-Congo}
\define@key{fams}{sug}{Nuclear Trans New Guinea}
\define@key{fams}{kzs}{Austronesian}
\define@key{fams}{zsu}{Austronesian}
\define@key{fams}{syk}{Afro-Asiatic}
\define@key{fams}{szn}{Austronesian}
\define@key{fams}{srg}{Austronesian}
\define@key{fams}{sqm}{Atlantic-Congo}
\define@key{fams}{siv}{Sepik}
\define@key{fams}{six}{Nuclear Trans New Guinea}
\define@key{fams}{suw}{Atlantic-Congo}
\define@key{fams}{smw}{Austronesian}
\define@key{fams}{sux}{Isolate}
\define@key{fams}{csv}{Sino-Tibetan}
\define@key{fams}{ssk}{Sino-Tibetan}
\define@key{fams}{suz}{Sino-Tibetan}
\define@key{fams}{syo}{Austroasiatic}
\define@key{fams}{sbj}{Maban}
\define@key{fams}{sgd}{Austronesian}
\define@key{fams}{sjp}{Indo-European}
\define@key{fams}{tdl}{Atlantic-Congo}
\define@key{fams}{sde}{Atlantic-Congo}
\define@key{fams}{mdz}{Tupian}
\define@key{fams}{sru}{Tupian}
\define@key{fams}{swx}{Arawan}
\define@key{fams}{sqn}{Iroquoian}
\define@key{fams}{ssu}{Angan}
\define@key{fams}{sdj}{Atlantic-Congo}
\define@key{fams}{swu}{Austronesian}
\define@key{fams}{suy}{Nuclear-Macro-Je}
\define@key{fams}{swg}{Indo-European}
\define@key{fams}{slf}{Sign Language}
\define@key{fams}{sgg}{Sign Language}
\define@key{fams}{ssr}{Sign Language}
\define@key{fams}{xdk}{Pama-Nyungan}
\define@key{fams}{syl}{Indo-European}
\define@key{fams}{zoq}{Mixe-Zoque}
\define@key{fams}{nhc}{Uto-Aztecan}
\define@key{fams}{zat}{Otomanguean}
\define@key{fams}{knv}{Isolate}
\define@key{fams}{tzx}{Lower Sepik-Ramu}
\define@key{fams}{xtt}{Otomanguean}
\define@key{fams}{lts}{Atlantic-Congo}
\define@key{fams}{dsq}{Songhay}
\define@key{fams}{tdy}{Austronesian}
\define@key{fams}{rob}{Austronesian}
\define@key{fams}{tcd}{Atlantic-Congo}
\define@key{fams}{klg}{Austronesian}
\define@key{fams}{bgs}{Austronesian}
\define@key{fams}{mvv}{Austronesian}
\define@key{fams}{tgz}{Pama-Nyungan}
\define@key{fams}{tbm}{Atlantic-Congo}
\define@key{fams}{tda}{Songhay}
\define@key{fams}{tgx}{Athabaskan-Eyak-Tlingit}
\define@key{fams}{tgj}{Sino-Tibetan}
\define@key{fams}{tgw}{Atlantic-Congo}
\define@key{fams}{tht}{Athabaskan-Eyak-Tlingit}
\define@key{fams}{blt}{Tai-Kadai}
\define@key{fams}{tyj}{Tai-Kadai}
\define@key{fams}{tyr}{Tai-Kadai}
\define@key{fams}{twh}{Tai-Kadai}
\define@key{fams}{tiz}{Tai-Kadai}
\define@key{fams}{taw}{Nuclear Trans New Guinea}
\define@key{fams}{aos}{Border}
\define@key{fams}{tlq}{Austroasiatic}
\define@key{fams}{thi}{Tai-Kadai}
\define@key{fams}{tjl}{Tai-Kadai}
\define@key{fams}{tdd}{Tai-Kadai}
\define@key{fams}{ago}{Angan}
\define@key{fams}{tnq}{Arawakan}
\define@key{fams}{tpo}{Tai-Kadai}
\define@key{fams}{uar}{Eleman}
\define@key{fams}{tmm}{Tai-Kadai}
\define@key{fams}{cuu}{Tai-Kadai}
\define@key{fams}{acq}{Afro-Asiatic}
\define@key{fams}{pee}{Austronesian}
\define@key{fams}{tdj}{Austronesian}
\define@key{fams}{abh}{Afro-Asiatic}
\define@key{fams}{tja}{Kru}
\define@key{fams}{tkz}{Austroasiatic}
\define@key{fams}{nho}{Austronesian}
\define@key{fams}{tke}{Atlantic-Congo}
\define@key{fams}{tak}{Afro-Asiatic}
\define@key{fams}{tdf}{Austroasiatic}
\define@key{fams}{tlr}{Austronesian}
\define@key{fams}{tlv}{Austronesian}
\define@key{fams}{tal}{Afro-Asiatic}
\define@key{fams}{tln}{Austronesian}
\define@key{fams}{tlk}{Austronesian}
\define@key{fams}{tzl}{Artificial Language}
\define@key{fams}{yta}{Sino-Tibetan}
\define@key{fams}{tcl}{Sino-Tibetan}
\define@key{fams}{tmn}{Austronesian}
\define@key{fams}{tmz}{Cariban}
\define@key{fams}{vmx}{Otomanguean}
\define@key{fams}{ten}{Tucanoan}
\define@key{fams}{tls}{Austronesian}
\define@key{fams}{xxt}{Isolate}
\define@key{fams}{tdk}{Afro-Asiatic}
\define@key{fams}{tmy}{Austronesian}
\define@key{fams}{tax}{Afro-Asiatic}
\define@key{fams}{tml}{Nuclear Trans New Guinea}
\define@key{fams}{tpu}{Austroasiatic}
\define@key{fams}{low}{Austronesian}
\define@key{fams}{tpv}{Austronesian}
\define@key{fams}{tcm}{Isolate}
\define@key{fams}{tni}{Austronesian}
\define@key{fams}{tdx}{Austronesian}
\define@key{fams}{tgn}{Austronesian}
\define@key{fams}{tnx}{Austronesian}
\define@key{fams}{tnv}{Indo-European}
\define@key{fams}{txg}{Sino-Tibetan}
\define@key{fams}{tgp}{Austronesian}
\define@key{fams}{tkx}{Nuclear Trans New Guinea}
\define@key{fams}{tgu}{Lower Sepik-Ramu}
\define@key{fams}{tbs}{Lower Sepik-Ramu}
\define@key{fams}{ytl}{Sino-Tibetan}
\define@key{fams}{tbe}{Austronesian}
\define@key{fams}{uji}{Atlantic-Congo}
\define@key{fams}{txy}{Austronesian}
\define@key{fams}{xnj}{Atlantic-Congo}
\define@key{fams}{qcs}{Mixe-Zoque}
\define@key{fams}{afp}{Arafundi}
\define@key{fams}{taf}{Tupian}
\define@key{fams}{txj}{Saharan}
\define@key{fams}{tpf}{Austronesian}
\define@key{fams}{txr}{Unclassifiable}
\define@key{fams}{tdm}{Isolate}
\define@key{fams}{twq}{Songhay}
\define@key{fams}{tmt}{Austronesian}
\define@key{fams}{ttd}{Goilalan}
\define@key{fams}{tco}{Sino-Tibetan}
\define@key{fams}{tpa}{Austronesian}
\define@key{fams}{tad}{Lakes Plain}
\define@key{fams}{tvs}{Atlantic-Congo}
\define@key{fams}{tvn}{Sino-Tibetan}
\define@key{fams}{rmu}{Speech Register}
\define@key{fams}{twl}{Atlantic-Congo}
\define@key{fams}{xtw}{Nambiquaran}
\define@key{fams}{ttq}{Afro-Asiatic}
\define@key{fams}{twy}{Austronesian}
\define@key{fams}{tbp}{Lakes Plain}
\define@key{fams}{tcp}{Sino-Tibetan}
\define@key{fams}{ayy}{Unattested}
\define@key{fams}{tas}{Pidgin}
\define@key{fams}{tnu}{Tai-Kadai}
\define@key{fams}{tys}{Tai-Kadai}
\define@key{fams}{tyt}{Tai-Kadai}
\define@key{fams}{tyz}{Tai-Kadai}
\define@key{fams}{tck}{Atlantic-Congo}
\define@key{fams}{bqa}{Atlantic-Congo}
\define@key{fams}{dtu}{Dogon}
\define@key{fams}{tsy}{Sign Language}
\define@key{fams}{tcw}{Totonacan}
\define@key{fams}{tuq}{Saharan}
\define@key{fams}{tkq}{Atlantic-Congo}
\define@key{fams}{lor}{Atlantic-Congo}
\define@key{fams}{tfo}{Geelvink Bay}
\define@key{fams}{twe}{Timor-Alor-Pantar}
\define@key{fams}{ztt}{Otomanguean}
\define@key{fams}{teg}{Atlantic-Congo}
\define@key{fams}{tyx}{Atlantic-Congo}
\define@key{fams}{lli}{Atlantic-Congo}
\define@key{fams}{ebo}{Atlantic-Congo}
\define@key{fams}{tyi}{Atlantic-Congo}
\define@key{fams}{tvm}{Austronesian}
\define@key{fams}{tlt}{Austronesian}
\define@key{fams}{nhv}{Uto-Aztecan}
\define@key{fams}{tjo}{Afro-Asiatic}
\define@key{fams}{tbt}{Atlantic-Congo}
\define@key{fams}{tmv}{Atlantic-Congo}
\define@key{fams}{tqb}{Tupian}
\define@key{fams}{tdo}{Atlantic-Congo}
\define@key{fams}{soz}{Atlantic-Congo}
\define@key{fams}{tmo}{Austroasiatic}
\define@key{fams}{ott}{Otomanguean}
\define@key{fams}{tmw}{Austronesian}
\define@key{fams}{quw}{Quechuan}
\define@key{fams}{otn}{Otomanguean}
\define@key{fams}{dtk}{Dogon}
\define@key{fams}{tes}{Austronesian}
\define@key{fams}{pah}{Tupian}
\define@key{fams}{tqn}{Sahaptian}
\define@key{fams}{tns}{Austronesian}
\define@key{fams}{tct}{Tai-Kadai}
\define@key{fams}{tev}{Austronesian}
\define@key{fams}{cux}{Otomanguean}
\define@key{fams}{cte}{Otomanguean}
\define@key{fams}{ted}{Kru}
\define@key{fams}{tef}{Austroasiatic}
\define@key{fams}{trb}{Austronesian}
\define@key{fams}{twg}{Timor-Alor-Pantar}
\define@key{fams}{tec}{Nilotic}
\define@key{fams}{tmg}{Indo-European}
\define@key{fams}{sjt}{Uralic}
\define@key{fams}{tkg}{Austronesian}
\define@key{fams}{keg}{Temeinic}
\define@key{fams}{twc}{Afro-Asiatic}
\define@key{fams}{tez}{Afro-Asiatic}
\define@key{fams}{tdt}{Austronesian}
\define@key{fams}{tve}{Austronesian}
\define@key{fams}{cut}{Otomanguean}
\define@key{fams}{twx}{Atlantic-Congo}
\define@key{fams}{otx}{Otomanguean}
\define@key{fams}{poq}{Mixe-Zoque}
\define@key{fams}{mxb}{Otomanguean}
\define@key{fams}{thy}{Atlantic-Congo}
\define@key{fams}{thn}{Dravidian}
\define@key{fams}{soa}{Tai-Kadai}
\define@key{fams}{nki}{Sino-Tibetan}
\define@key{fams}{thk}{Atlantic-Congo}
\define@key{fams}{iin}{Pama-Nyungan}
\define@key{fams}{tou}{Austroasiatic}
\define@key{fams}{ytp}{Sino-Tibetan}
\define@key{fams}{txh}{Indo-European}
\define@key{fams}{thu}{Nilotic}
\define@key{fams}{ahi}{Kru}
\define@key{fams}{mnl}{Austronesian}
\define@key{fams}{tbj}{Austronesian}
\define@key{fams}{ngy}{Atlantic-Congo}
\define@key{fams}{lsn}{Sign Language}
\define@key{fams}{tcn}{Sino-Tibetan}
\define@key{fams}{mtx}{Otomanguean}
\define@key{fams}{tia}{Afro-Asiatic}
\define@key{fams}{tiq}{Atlantic-Congo}
\define@key{fams}{boo}{Mande}
\define@key{fams}{tii}{Atlantic-Congo}
\define@key{fams}{nza}{Atlantic-Congo}
\define@key{fams}{txq}{Austronesian}
\define@key{fams}{xtl}{Otomanguean}
\define@key{fams}{tkp}{Austronesian}
\define@key{fams}{otl}{Otomanguean}
\define@key{fams}{zts}{Otomanguean}
\define@key{fams}{tij}{Sino-Tibetan}
\define@key{fams}{tim}{Nuclear Trans New Guinea}
\define@key{fams}{tvy}{Indo-European}
\define@key{fams}{xsb}{Austronesian}
\define@key{fams}{tit}{Isolate}
\define@key{fams}{tpz}{Austronesian}
\define@key{fams}{tpe}{Sino-Tibetan}
\define@key{fams}{tra}{Indo-European}
\define@key{fams}{tic}{Heibanic}
\define@key{fams}{tde}{Dogon}
\define@key{fams}{tdq}{Atlantic-Congo}
\define@key{fams}{ttv}{Austronesian}
\define@key{fams}{lax}{Sino-Tibetan}
\define@key{fams}{tju}{Pama-Nyungan}
\define@key{fams}{tpl}{Otomanguean}
\define@key{fams}{ctl}{Otomanguean}
\define@key{fams}{zpk}{Otomanguean}
\define@key{fams}{nuz}{Uto-Aztecan}
\define@key{fams}{mqh}{Otomanguean}
\define@key{fams}{tmf}{Lengua-Mascoy}
\define@key{fams}{tng}{Afro-Asiatic}
\define@key{fams}{tgh}{Indo-European}
\define@key{fams}{tox}{Austronesian}
\define@key{fams}{tgb}{Austronesian}
\define@key{fams}{taz}{Narrow Talodi}
\define@key{fams}{tdr}{Austroasiatic}
\define@key{fams}{tlg}{Namla-Tofanma}
\define@key{fams}{tfi}{Atlantic-Congo}
\define@key{fams}{tor}{Atlantic-Congo}
\define@key{fams}{tgy}{Atlantic-Congo}
\define@key{fams}{zuh}{Nuclear Trans New Guinea}
\define@key{fams}{xto}{Indo-European}
\define@key{fams}{txb}{Indo-European}
\define@key{fams}{tok}{Artificial Language}
\define@key{fams}{tkn}{Japonic}
\define@key{fams}{lbw}{Austronesian}
\define@key{fams}{tlm}{Austronesian}
\define@key{fams}{tol}{Athabaskan-Eyak-Tlingit}
\define@key{fams}{tod}{Mande}
\define@key{fams}{tdi}{Austronesian}
\define@key{fams}{tom}{Austronesian}
\define@key{fams}{txa}{Austronesian}
\define@key{fams}{ttp}{Austronesian}
\define@key{fams}{txm}{Austronesian}
\define@key{fams}{dtm}{Dogon}
\define@key{fams}{tqp}{Austronesian}
\define@key{fams}{tst}{Songhay}
\define@key{fams}{tnz}{Austroasiatic}
\define@key{fams}{tny}{Atlantic-Congo}
\define@key{fams}{tog}{Atlantic-Congo}
\define@key{fams}{xgf}{Uto-Aztecan}
\define@key{fams}{tjn}{Mande}
\define@key{fams}{tnw}{Austronesian}
\define@key{fams}{txs}{Austronesian}
\define@key{fams}{toz}{Atlantic-Congo}
\define@key{fams}{ttj}{Atlantic-Congo}
\define@key{fams}{toq}{Nilotic}
\define@key{fams}{toy}{Austronesian}
\define@key{fams}{ttu}{Austronesian}
\define@key{fams}{trz}{Chapacuran}
\define@key{fams}{trj}{Afro-Asiatic}
\define@key{fams}{fit}{Uralic}
\define@key{fams}{tdv}{Atlantic-Congo}
\define@key{fams}{tqr}{Narrow Talodi}
\define@key{fams}{dtt}{Dogon}
\define@key{fams}{tno}{Pano-Tacanan}
\define@key{fams}{tei}{Nuclear Torricelli}
\define@key{fams}{als}{Indo-European}
\define@key{fams}{ttl}{Atlantic-Congo}
\define@key{fams}{txo}{Sino-Tibetan}
\define@key{fams}{txe}{Austronesian}
\define@key{fams}{ttk}{Barbacoan}
\define@key{fams}{zph}{Otomanguean}
\define@key{fams}{tqu}{Isolate}
\define@key{fams}{neb}{Mande}
\define@key{fams}{don}{Austronesian}
\define@key{fams}{ttn}{Pauwasi}
\define@key{fams}{xtg}{Indo-European}
\define@key{fams}{trl}{Unclassifiable}
\define@key{fams}{rmg}{Speech Register}
\define@key{fams}{rmd}{Speech Register}
\define@key{fams}{trm}{Indo-European}
\define@key{fams}{tme}{Unattested}
\define@key{fams}{stg}{Austroasiatic}
\define@key{fams}{tip}{Greater Kwerba}
\define@key{fams}{trx}{Austronesian}
\define@key{fams}{tgq}{Austronesian}
\define@key{fams}{trn}{Arawakan}
\define@key{fams}{trf}{Indo-European}
\define@key{fams}{lst}{Sign Language}
\define@key{fams}{tka}{Unattested}
\define@key{fams}{tsa}{Atlantic-Congo}
\define@key{fams}{tsd}{Indo-European}
\define@key{fams}{kvz}{Nuclear Trans New Guinea}
\define@key{fams}{tsb}{Afro-Asiatic}
\define@key{fams}{tsk}{Sino-Tibetan}
\define@key{fams}{txc}{Athabaskan-Eyak-Tlingit}
\define@key{fams}{kdl}{Atlantic-Congo}
\define@key{fams}{xmw}{Austronesian}
\define@key{fams}{tsw}{Atlantic-Congo}
\define@key{fams}{hio}{Khoe-Kwadi}
\define@key{fams}{ldp}{Atlantic-Congo}
\define@key{fams}{lto}{Atlantic-Congo}
\define@key{fams}{fly}{Speech Register}
\define@key{fams}{ttz}{Sino-Tibetan}
\define@key{fams}{tsl}{Tai-Kadai}
\define@key{fams}{tvd}{Atlantic-Congo}
\define@key{fams}{tsh}{Afro-Asiatic}
\define@key{fams}{two}{Atlantic-Congo}
\define@key{fams}{tsc}{Atlantic-Congo}
\define@key{fams}{nrt}{Kalapuyan}
\define@key{fams}{tuy}{Nilotic}
\define@key{fams}{tuj}{North Halmahera}
\define@key{fams}{khc}{Austronesian}
\define@key{fams}{bhq}{Austronesian}
\define@key{fams}{tkf}{Unattested}
\define@key{fams}{tkd}{Austronesian}
\define@key{fams}{tul}{Atlantic-Congo}
\define@key{fams}{tlu}{Austronesian}
\define@key{fams}{tey}{Kadugli-Krongo}
\define@key{fams}{rak}{Austronesian}
\define@key{fams}{krt}{Saharan}
\define@key{fams}{iou}{Nuclear Trans New Guinea}
\define@key{fams}{tum}{Atlantic-Congo}
\define@key{fams}{kku}{Unattested}
\define@key{fams}{xtq}{Indo-European}
\define@key{fams}{tbr}{Kadugli-Krongo}
\define@key{fams}{enh}{Uralic}
\define@key{fams}{trt}{Geelvink Bay}
\define@key{fams}{tse}{Sign Language}
\define@key{fams}{tug}{Atlantic-Congo}
\define@key{fams}{tjg}{Austronesian}
\define@key{fams}{tqq}{Afro-Asiatic}
\define@key{fams}{dza}{Atlantic-Congo}
\define@key{fams}{ttf}{Atlantic-Congo}
\define@key{fams}{tpr}{Tupian}
\define@key{fams}{tpw}{Tupian}
\define@key{fams}{trh}{Dagan}
\define@key{fams}{trd}{Austroasiatic}
\define@key{fams}{twt}{Tupian}
\define@key{fams}{tuz}{Atlantic-Congo}
\define@key{fams}{tch}{Indo-European}
\define@key{fams}{tru}{Afro-Asiatic}
\define@key{fams}{try}{Tai-Kadai}
\define@key{fams}{tqm}{Doso-Turumsa}
\define@key{fams}{ttg}{Austronesian}
\define@key{fams}{tmi}{Austronesian}
\define@key{fams}{mtu}{Otomanguean}
\define@key{fams}{tww}{Walioic}
\define@key{fams}{ifk}{Austronesian}
\define@key{fams}{bov}{Atlantic-Congo}
\define@key{fams}{tud}{Isolate}
\define@key{fams}{tux}{Pano-Tacanan}
\define@key{fams}{xjb}{Pama-Nyungan}
\define@key{fams}{twn}{Atlantic-Congo}
\define@key{fams}{uam}{Unclassifiable}
\define@key{fams}{ksj}{Kwalean}
\define@key{fams}{byc}{Atlantic-Congo}
\define@key{fams}{uba}{Atlantic-Congo}
\define@key{fams}{ubi}{Afro-Asiatic}
\define@key{fams}{ubr}{Austronesian}
\define@key{fams}{cpb}{Arawakan}
\define@key{fams}{uda}{Atlantic-Congo}
\define@key{fams}{udu}{Koman}
\define@key{fams}{ufi}{Nuclear Trans New Guinea}
\define@key{fams}{uga}{Afro-Asiatic}
\define@key{fams}{uge}{Austronesian}
\define@key{fams}{ugo}{Sino-Tibetan}
\define@key{fams}{uha}{Atlantic-Congo}
\define@key{fams}{uis}{South Bougainville}
\define@key{fams}{udj}{Austronesian}
\define@key{fams}{kcf}{Atlantic-Congo}
\define@key{fams}{ukh}{Atlantic-Congo}
\define@key{fams}{umi}{Austronesian}
\define@key{fams}{ukp}{Atlantic-Congo}
\define@key{fams}{akd}{Atlantic-Congo}
\define@key{fams}{ukl}{Sign Language}
\define@key{fams}{uku}{Atlantic-Congo}
\define@key{fams}{ukg}{Nuclear Trans New Guinea}
\define@key{fams}{ukq}{Atlantic-Congo}
\define@key{fams}{ukw}{Atlantic-Congo}
\define@key{fams}{svb}{Austronesian}
\define@key{fams}{ull}{Dravidian}
\define@key{fams}{ulb}{Atlantic-Congo}
\define@key{fams}{ulm}{Austronesian}
\define@key{fams}{ulw}{Misumalpan}
\define@key{fams}{ulu}{Austronesian}
\define@key{fams}{xky}{Austronesian}
\define@key{fams}{gdn}{Dagan}
\define@key{fams}{umd}{Pama-Nyungan}
\define@key{fams}{xum}{Indo-European}
\define@key{fams}{umr}{Isolate}
\define@key{fams}{umg}{Pama-Nyungan}
\define@key{fams}{upi}{Border}
\define@key{fams}{sju}{Uralic}
\define@key{fams}{due}{Austronesian}
\define@key{fams}{umm}{Atlantic-Congo}
\define@key{fams}{umo}{Bororoan}
\define@key{fams}{unz}{Austronesian}
\define@key{fams}{bbn}{Austronesian}
\define@key{fams}{une}{Atlantic-Congo}
\define@key{fams}{xgu}{Worrorran}
\define@key{fams}{uni}{Sko}
\define@key{fams}{uln}{Indo-European}
\define@key{fams}{onu}{Austronesian}
\define@key{fams}{unu}{Austronesian}
\define@key{fams}{tov}{Indo-European}
\define@key{fams}{tku}{Totonacan}
\define@key{fams}{sxu}{Indo-European}
\define@key{fams}{tth}{Austroasiatic}
\define@key{fams}{dmg}{Austronesian}
\define@key{fams}{dna}{Nuclear Trans New Guinea}
\define@key{fams}{xup}{Athabaskan-Eyak-Tlingit}
\define@key{fams}{tau}{Athabaskan-Eyak-Tlingit}
\define@key{fams}{url}{Dravidian}
\define@key{fams}{urm}{Nuclear Trans New Guinea}
\define@key{fams}{uro}{Baining}
\define@key{fams}{xur}{Hurro-Urartian}
\define@key{fams}{urg}{Nuclear Trans New Guinea}
\define@key{fams}{uvh}{Nuclear Trans New Guinea}
\define@key{fams}{urx}{Nuclear Torricelli}
\define@key{fams}{urc}{Giimbiyu}
\define@key{fams}{urv}{Austronesian}
\define@key{fams}{urn}{Austronesian}
\define@key{fams}{urz}{Tupian}
\define@key{fams}{ugy}{Sign Language}
\define@key{fams}{uru}{Tupian}
\define@key{fams}{urp}{Unclassifiable}
\define@key{fams}{usk}{Atlantic-Congo}
\define@key{fams}{ush}{Indo-European}
\define@key{fams}{ulf}{Isolate}
\define@key{fams}{usp}{Mayan}
\define@key{fams}{usi}{Sino-Tibetan}
\define@key{fams}{omo}{Nuclear Trans New Guinea}
\define@key{fams}{wsg}{Dravidian}
\define@key{fams}{utu}{Nuclear Trans New Guinea}
\define@key{fams}{uuu}{Austroasiatic}
\define@key{fams}{evh}{Atlantic-Congo}
\define@key{fams}{usu}{Nuclear Trans New Guinea}
\define@key{fams}{auz}{Afro-Asiatic}
\define@key{fams}{eze}{Atlantic-Congo}
\define@key{fams}{vaa}{Indo-European}
\define@key{fams}{kqu}{Tuu}
\define@key{fams}{vgr}{Indo-European}
\define@key{fams}{dkg}{Atlantic-Congo}
\define@key{fams}{tva}{Austronesian}
\define@key{fams}{vap}{Sino-Tibetan}
\define@key{fams}{vae}{Central Sudanic}
\define@key{fams}{vsv}{Sign Language}
\define@key{fams}{vmv}{Maiduan}
\define@key{fams}{cvn}{Otomanguean}
\define@key{fams}{vlp}{Austronesian}
\define@key{fams}{mkt}{Austronesian}
\define@key{fams}{mlr}{Afro-Asiatic}
\define@key{fams}{mpr}{Austronesian}
\define@key{fams}{vnk}{Austronesian}
\define@key{fams}{vau}{Atlantic-Congo}
\define@key{fams}{vao}{Austronesian}
\define@key{fams}{vah}{Indo-European}
\define@key{fams}{vrs}{Austronesian}
\define@key{fams}{vav}{Indo-European}
\define@key{fams}{vaj}{Kxa}
\define@key{fams}{val}{Austronesian}
\define@key{fams}{vem}{Afro-Asiatic}
\define@key{fams}{vsl}{Sign Language}
\define@key{fams}{xve}{Indo-European}
\define@key{fams}{vec}{Indo-European}
\define@key{fams}{veo}{Chumashan}
\define@key{fams}{vra}{Austronesian}
\define@key{fams}{vid}{Atlantic-Congo}
\define@key{fams}{vig}{Atlantic-Congo}
\define@key{fams}{vil}{Isolate}
\define@key{fams}{dyg}{Unattested}
\define@key{fams}{svc}{Indo-European}
\define@key{fams}{vin}{Atlantic-Congo}
\define@key{fams}{vic}{Indo-European}
\define@key{fams}{vis}{Dravidian}
\define@key{fams}{vit}{Atlantic-Congo}
\define@key{fams}{vto}{Tor-Orya}
\define@key{fams}{vls}{Indo-European}
\define@key{fams}{vol}{Artificial Language}
\define@key{fams}{kch}{Unattested}
\define@key{fams}{vor}{Atlantic-Congo}
\define@key{fams}{vum}{Atlantic-Congo}
\define@key{fams}{vnp}{Austronesian}
\define@key{fams}{vun}{Atlantic-Congo}
\define@key{fams}{msn}{Austronesian}
\define@key{fams}{vut}{Atlantic-Congo}
\define@key{fams}{wbi}{Atlantic-Congo}
\define@key{fams}{wmn}{Austronesian}
\define@key{fams}{wab}{Austronesian}
\define@key{fams}{wbb}{Austronesian}
\define@key{fams}{kmx}{Kiwaian}
\define@key{fams}{wci}{Atlantic-Congo}
\define@key{fams}{wdg}{Nuclear Trans New Guinea}
\define@key{fams}{wbq}{Dravidian}
\define@key{fams}{kxp}{Indo-European}
\define@key{fams}{wdu}{Pama-Nyungan}
\define@key{fams}{wag}{Austronesian}
\define@key{fams}{wrx}{Austronesian}
\define@key{fams}{waj}{Nuclear Trans New Guinea}
\define@key{fams}{wga}{Pama-Nyungan}
\define@key{fams}{wgb}{Austronesian}
\define@key{fams}{wbr}{Indo-European}
\define@key{fams}{fad}{Nuclear Trans New Guinea}
\define@key{fams}{whk}{Austronesian}
\define@key{fams}{wgo}{Austronesian}
\define@key{fams}{wlr}{Austronesian}
\define@key{fams}{wlk}{Athabaskan-Eyak-Tlingit}
\define@key{fams}{wmh}{Austronesian}
\define@key{fams}{atr}{Cariban}
\define@key{fams}{wli}{North Halmahera}
\define@key{fams}{wja}{Atlantic-Congo}
\define@key{fams}{wav}{Atlantic-Congo}
\define@key{fams}{wwb}{Unclassifiable}
\define@key{fams}{wkd}{Austronesian}
\define@key{fams}{waf}{Unattested}
\define@key{fams}{lgl}{Austronesian}
\define@key{fams}{wlw}{Nuclear Trans New Guinea}
\define@key{fams}{wly}{Sino-Tibetan}
\define@key{fams}{wll}{Nubian}
\define@key{fams}{wlx}{Atlantic-Congo}
\define@key{fams}{waa}{Sahaptian}
\define@key{fams}{wln}{Indo-European}
\define@key{fams}{wae}{Indo-European}
\define@key{fams}{ola}{Sino-Tibetan}
\define@key{fams}{wmc}{Nuclear Trans New Guinea}
\define@key{fams}{wmi}{Pama-Nyungan}
\define@key{fams}{lbq}{Austronesian}
\define@key{fams}{waz}{Austronesian}
\define@key{fams}{qyp}{Algic}
\define@key{fams}{wnp}{Nuclear Torricelli}
\define@key{fams}{wnb}{Nuclear Trans New Guinea}
\define@key{fams}{nnp}{Sino-Tibetan}
\define@key{fams}{wbh}{Atlantic-Congo}
\define@key{fams}{wdd}{Atlantic-Congo}
\define@key{fams}{wad}{Austronesian}
\define@key{fams}{mfi}{Afro-Asiatic}
\define@key{fams}{wne}{Indo-European}
\define@key{fams}{hwa}{Kru}
\define@key{fams}{wnm}{Pama-Nyungan}
\define@key{fams}{lwg}{Atlantic-Congo}
\define@key{fams}{wng}{Nuclear Trans New Guinea}
\define@key{fams}{jub}{Atlantic-Congo}
\define@key{fams}{wno}{Nuclear Trans New Guinea}
\define@key{fams}{wnk}{Austronesian}
\define@key{fams}{wny}{Garrwan}
\define@key{fams}{juk}{Atlantic-Congo}
\define@key{fams}{juw}{Atlantic-Congo}
\define@key{fams}{wbf}{Atlantic-Congo}
\define@key{fams}{tci}{Yam}
\define@key{fams}{srv}{Austronesian}
\define@key{fams}{bpe}{Sko}
\define@key{fams}{wre}{Unattested}
\define@key{fams}{wai}{Unattested}
\define@key{fams}{wri}{Pama-Nyungan}
\define@key{fams}{wbe}{Lakes Plain}
\define@key{fams}{aml}{Austroasiatic}
\define@key{fams}{wji}{Afro-Asiatic}
\define@key{fams}{bgv}{Anim}
\define@key{fams}{wrl}{Pama-Nyungan}
\define@key{fams}{wrn}{Heibanic}
\define@key{fams}{wru}{Austronesian}
\define@key{fams}{wrv}{Suki-Gogodala}
\define@key{fams}{wss}{Atlantic-Congo}
\define@key{fams}{gsp}{Nuclear Trans New Guinea}
\define@key{fams}{wsu}{Unattested}
\define@key{fams}{wtk}{Sepik}
\define@key{fams}{wah}{Austronesian}
\define@key{fams}{wuy}{Austronesian}
\define@key{fams}{www}{Atlantic-Congo}
\define@key{fams}{wow}{Austronesian}
\define@key{fams}{wxa}{Sino-Tibetan}
\define@key{fams}{ctt}{Dravidian}
\define@key{fams}{wyr}{Tupian}
\define@key{fams}{weh}{Atlantic-Congo}
\define@key{fams}{wew}{Austronesian}
\define@key{fams}{wlh}{Austronesian}
\define@key{fams}{klh}{Nuclear Trans New Guinea}
\define@key{fams}{wei}{Anim}
\define@key{fams}{gxx}{Kru}
\define@key{fams}{ywl}{Sino-Tibetan}
\define@key{fams}{hmw}{Hmong-Mien}
\define@key{fams}{ojw}{Algic}
\define@key{fams}{tqt}{Totonacan}
\define@key{fams}{yih}{Indo-European}
\define@key{fams}{pnb}{Indo-European}
\define@key{fams}{lcp}{Austroasiatic}
\define@key{fams}{kuf}{Austroasiatic}
\define@key{fams}{mut}{Dravidian}
\define@key{fams}{kyu}{Sino-Tibetan}
\define@key{fams}{tdg}{Sino-Tibetan}
\define@key{fams}{wmg}{Sino-Tibetan}
\define@key{fams}{raf}{Sino-Tibetan}
\define@key{fams}{mmr}{Hmong-Mien}
\define@key{fams}{lia}{Atlantic-Congo}
\define@key{fams}{xwl}{Atlantic-Congo}
\define@key{fams}{bbp}{Atlantic-Congo}
\define@key{fams}{ssl}{Atlantic-Congo}
\define@key{fams}{krw}{Kru}
\define@key{fams}{nnd}{Austronesian}
\define@key{fams}{uve}{Austronesian}
\define@key{fams}{mss}{Austronesian}
\define@key{fams}{lmj}{Austronesian}
\define@key{fams}{drn}{Austronesian}
\define@key{fams}{suc}{Austronesian}
\define@key{fams}{twb}{Austronesian}
\define@key{fams}{pne}{Austronesian}
\define@key{fams}{zbw}{Austronesian}
\define@key{fams}{dnw}{Nuclear Trans New Guinea}
\define@key{fams}{nhw}{Uto-Aztecan}
\define@key{fams}{pua}{Tarascan}
\define@key{fams}{gnw}{Tupian}
\define@key{fams}{jmx}{Otomanguean}
\define@key{fams}{tnb}{Chibchan}
\define@key{fams}{amw}{Afro-Asiatic}
\define@key{fams}{azn}{Uto-Aztecan}
\define@key{fams}{wwo}{Austronesian}
\define@key{fams}{wea}{Sino-Tibetan}
\define@key{fams}{wec}{Kru}
\define@key{fams}{woy}{Unattested}
\define@key{fams}{lwh}{Tai-Kadai}
\define@key{fams}{giw}{Tai-Kadai}
\define@key{fams}{tnp}{Austronesian}
\define@key{fams}{tua}{Nuclear Torricelli}
\define@key{fams}{mtp}{Matacoan}
\define@key{fams}{wlv}{Matacoan}
\define@key{fams}{wik}{Pama-Nyungan}
\define@key{fams}{wie}{Pama-Nyungan}
\define@key{fams}{wij}{Pama-Nyungan}
\define@key{fams}{wif}{Unattested}
\define@key{fams}{wih}{Pama-Nyungan}
\define@key{fams}{wua}{Pama-Nyungan}
\define@key{fams}{wil}{Worrorran}
\define@key{fams}{wit}{Wintuan}
\define@key{fams}{gdr}{Eastern Trans-Fly}
\define@key{fams}{wrh}{Pama-Nyungan}
\define@key{fams}{wir}{Tupian}
\define@key{fams}{wiu}{Isolate}
\define@key{fams}{xwc}{Siouan}
\define@key{fams}{woc}{Austronesian}
\define@key{fams}{wbw}{Austronesian}
\define@key{fams}{wyi}{Pama-Nyungan}
\define@key{fams}{jod}{Mande}
\define@key{fams}{wod}{Nuclear Trans New Guinea}
\define@key{fams}{wle}{Afro-Asiatic}
\define@key{fams}{wom}{Atlantic-Congo}
\define@key{fams}{wmo}{Nuclear Torricelli}
\define@key{fams}{won}{Atlantic-Congo}
\define@key{fams}{cwd}{Algic}
\define@key{fams}{kda}{Pama-Nyungan}
\define@key{fams}{wor}{Geelvink Bay}
\define@key{fams}{jud}{Mande}
\define@key{fams}{wsv}{Indo-European}
\define@key{fams}{wtw}{Austronesian}
\define@key{fams}{wud}{Atlantic-Congo}
\define@key{fams}{qgu}{Pama-Nyungan}
\define@key{fams}{wlu}{Pama-Nyungan}
\define@key{fams}{wux}{Limilngan-Wulna}
\define@key{fams}{bqm}{Atlantic-Congo}
\define@key{fams}{wum}{Atlantic-Congo}
\define@key{fams}{ywu}{Sino-Tibetan}
\define@key{fams}{bwn}{Hmong-Mien}
\define@key{fams}{wub}{Worrorran}
\define@key{fams}{wur}{Marrku-Wurrugu}
\define@key{fams}{yig}{Sino-Tibetan}
\define@key{fams}{bse}{Atlantic-Congo}
\define@key{fams}{wsi}{Austronesian}
\define@key{fams}{wuh}{Sino-Tibetan}
\define@key{fams}{wut}{Sko}
\define@key{fams}{wuv}{Austronesian}
\define@key{fams}{wym}{Indo-European}
\define@key{fams}{zax}{Otomanguean}
\define@key{fams}{xkr}{Nuclear-Macro-Je}
\define@key{fams}{xan}{Afro-Asiatic}
\define@key{fams}{ztg}{Otomanguean}
\define@key{fams}{axx}{Austronesian}
\define@key{fams}{xeg}{Tuu}
\define@key{fams}{xet}{Tupian}
\define@key{fams}{hsn}{Sino-Tibetan}
\define@key{fams}{sjo}{Tungusic}
\define@key{fams}{asn}{Tupian}
\define@key{fams}{xiy}{Tupian}
\define@key{fams}{xip}{Unattested}
\define@key{fams}{xii}{Khoe-Kwadi}
\define@key{fams}{xoo}{Isolate}
\define@key{fams}{xwe}{Atlantic-Congo}
\define@key{fams}{tyy}{Atlantic-Congo}
\define@key{fams}{muu}{Afro-Asiatic}
\define@key{fams}{yar}{Cariban}
\define@key{fams}{ybn}{Arawakan}
\define@key{fams}{ybm}{Nuclear Trans New Guinea}
\define@key{fams}{ybo}{Nuclear Trans New Guinea}
\define@key{fams}{ekr}{Atlantic-Congo}
\define@key{fams}{rys}{Japonic}
\define@key{fams}{wfg}{Pauwasi}
\define@key{fams}{ygm}{Nuclear Trans New Guinea}
\define@key{fams}{ygw}{Angan}
\define@key{fams}{rhp}{Nuclear Torricelli}
\define@key{fams}{ner}{Konda-Yahadian}
\define@key{fams}{ynu}{Tucanoan}
\define@key{fams}{iyx}{Atlantic-Congo}
\define@key{fams}{ykk}{Austronesian}
\define@key{fams}{ybh}{Sino-Tibetan}
\define@key{fams}{xyl}{Unattested}
\define@key{fams}{yba}{Atlantic-Congo}
\define@key{fams}{jal}{Austronesian}
\define@key{fams}{zpu}{Otomanguean}
\define@key{fams}{yal}{Mande}
\define@key{fams}{ymp}{Austronesian}
\define@key{fams}{yat}{Atlantic-Congo}
\define@key{fams}{ymb}{Nuclear Torricelli}
\define@key{fams}{yme}{Peba-Yagua}
\define@key{fams}{ymn}{Austronesian}
\define@key{fams}{qur}{Quechuan}
\define@key{fams}{yda}{Pama-Nyungan}
\define@key{fams}{dym}{Dogon}
\define@key{fams}{xyb}{Pama-Nyungan}
\define@key{fams}{zyg}{Tai-Kadai}
\define@key{fams}{jng}{Yangmanic}
\define@key{fams}{yng}{Atlantic-Congo}
\define@key{fams}{bsx}{Atlantic-Congo}
\define@key{fams}{yav}{Atlantic-Congo}
\define@key{fams}{ygl}{Nuclear Torricelli}
\define@key{fams}{ymo}{Nuclear Torricelli}
\define@key{fams}{yde}{Nuclear Torricelli}
\define@key{fams}{ynl}{Nuclear Trans New Guinea}
\define@key{fams}{tjj}{Pama-Nyungan}
\define@key{fams}{ysm}{Sign Language}
\define@key{fams}{jay}{Pama-Nyungan}
\define@key{fams}{guu}{Yanomamic}
\define@key{fams}{asy}{Nuclear Trans New Guinea}
\define@key{fams}{yre}{Mande}
\define@key{fams}{yev}{Nuclear Torricelli}
\define@key{fams}{yrw}{Nuclear Trans New Guinea}
\define@key{fams}{zae}{Otomanguean}
\define@key{fams}{yro}{Yanomamic}
\define@key{fams}{yko}{Atlantic-Congo}
\define@key{fams}{zty}{Otomanguean}
\define@key{fams}{yla}{Keram}
\define@key{fams}{yuw}{Nuclear Trans New Guinea}
\define@key{fams}{jau}{Austronesian}
\define@key{fams}{yyu}{Nuclear Torricelli}
\define@key{fams}{zpb}{Otomanguean}
\define@key{fams}{qux}{Quechuan}
\define@key{fams}{yvt}{Arawakan}
\define@key{fams}{yww}{Pama-Nyungan}
\define@key{fams}{ywn}{Pano-Tacanan}
\define@key{fams}{yaw}{Arawakan}
\define@key{fams}{yby}{Nuclear Trans New Guinea}
\define@key{fams}{ybx}{Walioic}
\define@key{fams}{ykr}{Nuclear Trans New Guinea}
\define@key{fams}{yel}{Atlantic-Congo}
\define@key{fams}{ylg}{Ndu}
\define@key{fams}{ynq}{Atlantic-Congo}
\define@key{fams}{yec}{Mixed Language}
\define@key{fams}{yei}{Atlantic-Congo}
\define@key{fams}{yra}{Isolate}
\define@key{fams}{gop}{Austronesian}
\define@key{fams}{yrn}{Tai-Kadai}
\define@key{fams}{yeu}{Dravidian}
\define@key{fams}{yes}{Atlantic-Congo}
\define@key{fams}{yet}{Isolate}
\define@key{fams}{yej}{Indo-European}
\define@key{fams}{ydg}{Indo-European}
\define@key{fams}{yim}{Sino-Tibetan}
\define@key{fams}{kvu}{Sino-Tibetan}
\define@key{fams}{yin}{Austroasiatic}
\define@key{fams}{yil}{Pama-Nyungan}
\define@key{fams}{ywg}{Pama-Nyungan}
\define@key{fams}{kvy}{Sino-Tibetan}
\define@key{fams}{yxm}{Pama-Nyungan}
\define@key{fams}{ljw}{Pama-Nyungan}
\define@key{fams}{yiy}{Pama-Nyungan}
\define@key{fams}{yis}{Nuclear Torricelli}
\define@key{fams}{gek}{Afro-Asiatic}
\define@key{fams}{yob}{Austronesian}
\define@key{fams}{gud}{Kru}
\define@key{fams}{yog}{Austronesian}
\define@key{fams}{ydk}{Nuclear Trans New Guinea}
\define@key{fams}{yki}{Austronesian}
\define@key{fams}{ygs}{Sign Language}
\define@key{fams}{xty}{Otomanguean}
\define@key{fams}{pil}{Atlantic-Congo}
\define@key{fams}{yoi}{Japonic}
\define@key{fams}{sxk}{Kalapuyan}
\define@key{fams}{nru}{Sino-Tibetan}
\define@key{fams}{zyn}{Tai-Kadai}
\define@key{fams}{zyb}{Tai-Kadai}
\define@key{fams}{yno}{Tai-Kadai}
\define@key{fams}{yon}{Nuclear Trans New Guinea}
\define@key{fams}{yut}{Nuclear Trans New Guinea}
\define@key{fams}{mts}{Pano-Tacanan}
\define@key{fams}{yox}{Japonic}
\define@key{fams}{yot}{Atlantic-Congo}
\define@key{fams}{zyj}{Tai-Kadai}
\define@key{fams}{ytw}{Nuclear Trans New Guinea}
\define@key{fams}{yoy}{Tai-Kadai}
\define@key{fams}{nua}{Austronesian}
\define@key{fams}{msd}{Sign Language}
\define@key{fams}{mvg}{Otomanguean}
\define@key{fams}{yub}{Pama-Nyungan}
\define@key{fams}{ysl}{Sign Language}
\define@key{fams}{ygu}{Unattested}
\define@key{fams}{yab}{Naduhup}
\define@key{fams}{omk}{Yukaghir}
\define@key{fams}{ybl}{Atlantic-Congo}
\define@key{fams}{yuq}{Tupian}
\define@key{fams}{ljx}{Pama-Nyungan}
\define@key{fams}{mab}{Otomanguean}
\define@key{fams}{yau}{Isolate}
\define@key{fams}{ztx}{Otomanguean}
\define@key{fams}{kji}{Austronesian}
\define@key{fams}{nhi}{Uto-Aztecan}
\define@key{fams}{ctz}{Otomanguean}
\define@key{fams}{atb}{Sino-Tibetan}
\define@key{fams}{zkr}{Sino-Tibetan}
\define@key{fams}{zsl}{Sign Language}
\define@key{fams}{zak}{Atlantic-Congo}
\define@key{fams}{zau}{Sino-Tibetan}
\define@key{fams}{zna}{Atlantic-Congo}
\define@key{fams}{zah}{Afro-Asiatic}
\define@key{fams}{zpw}{Otomanguean}
\define@key{fams}{zaj}{Atlantic-Congo}
\define@key{fams}{zbu}{Afro-Asiatic}
\define@key{fams}{zaz}{Afro-Asiatic}
\define@key{fams}{zal}{Sino-Tibetan}
\define@key{fams}{kxk}{Sino-Tibetan}
\define@key{fams}{zwa}{Afro-Asiatic}
\define@key{fams}{jaj}{Austronesian}
\define@key{fams}{zua}{Afro-Asiatic}
\define@key{fams}{dhm}{Atlantic-Congo}
\define@key{fams}{zeg}{Austronesian}
\define@key{fams}{czn}{Otomanguean}
\define@key{fams}{zhb}{Sino-Tibetan}
\define@key{fams}{xzh}{Sino-Tibetan}
\define@key{fams}{zhi}{Atlantic-Congo}
\define@key{fams}{zhw}{Atlantic-Congo}
\define@key{fams}{zia}{Nuclear Trans New Guinea}
\define@key{fams}{zil}{Mande}
\define@key{fams}{ziw}{Atlantic-Congo}
\define@key{fams}{zib}{Sign Language}
\define@key{fams}{zmb}{Atlantic-Congo}
\define@key{fams}{zin}{Atlantic-Congo}
\define@key{fams}{sih}{Austronesian}
\define@key{fams}{zrn}{Afro-Asiatic}
\define@key{fams}{ziz}{Afro-Asiatic}
\define@key{fams}{pto}{Tupian}
\define@key{fams}{yzk}{Sino-Tibetan}
\define@key{fams}{gbz}{Indo-European}
\define@key{fams}{czt}{Sino-Tibetan}
\define@key{fams}{zom}{Sino-Tibetan}
\define@key{fams}{zla}{Atlantic-Congo}
\define@key{fams}{gnd}{Afro-Asiatic}
\define@key{fams}{zuy}{Afro-Asiatic}
\define@key{fams}{jmb}{Afro-Asiatic}
\define@key{fams}{zzj}{Tai-Kadai}
\define@key{fams}{zyp}{Sino-Tibetan}
}
\DeclareOption{none}{}
%    \end{macrocode}
% Language codes, names and families are set with the |\define@key|\marg{family} \marg{key} \marg{value} macro from the \textsf{xkeyval} package(see the \textsf{langs\_.tex} and \textsf{fams\_.tex} files in the package folder). \marg{family} is either \textsf{names} or \textsf{fams}. Thus, each language has two key-value pairs that refer to it, one defining its name and the other one defining its family, both using the ISO 639-3 code as their key.
% \end{macro}
% \subsection{Macro definitions}
% \begin{macro}{lname}
% The |\lname| macro takes the key specified in its mandatory argument to call its corresponding key value pair from the \textsf{names} family, and prints it. This is achieved through the use of the |\csname| and |\endcsname| macros. The |\unskip| macro is used in all the macro definitions to avoid the adding of an extra space after the macro has been printed.
%    \begin{macrocode}
\newcommand{\lname}[1]{%
  \csname KV@names@#1\endcsname\unskip
}
%    \end{macrocode}
% \end{macro}
% \begin{macro}{liso}
% The |\liso| macro takes, like |\lname|, the value from the \textsf{names} family from the argument input, and prints the name as well as the ISO 639-3 code (which is the argument verbatim) between parenthesis.
%    \begin{macrocode}
\newcommand{\liso}[1]{%
  \csname KV@names@#1\endcsname{} (ISO 639-3: #1)\unskip
}
%    \end{macrocode}
% \end{macro}
% \begin{macro}{lfam}
% The |\lfam| macro, like |\lname| and |\liso|, calls the key-value pair from the \textsf{names} family corresponding to the input of the mandatory argument, plus the key-value pair from the \textsf{fams} family which gives it the genetic affiliation, which is printed between parenthesis.
%    \begin{macrocode}
\newcommand{\lfam}[1]{%
  \csname KV@names@#1\endcsname{} (\csname KV@fams@#1\endcsname{})\unskip
}
%    \end{macrocode}
% \end{macro}
% \begin{macro}{newlang}
% The macro |\newlang| defines new keys for a language from the three mandatory arguments with the |\define@key| \marg{family} \marg{key} \marg{value} macro from \textsf{xkeyval}. The first argument of |\newlang| \marg{code} defines the code which serves as identifier (the ISO code in the case of pre-defined key-value pairs). The second argument \marg{name} defines the printed name of the language. The third argument \marg{family} defines the family to which the language belongs.
%    \begin{macrocode}
\newcommand{\newlang}[3]{%
  \define@key{names}{#1}{#2}
  \define@key{fams}{#1}{#3}
}
%    \end{macrocode}
% \end{macro}
% \begin{macro}{ProcessOptions}
% This line of code simply tells the package to set the options specified above.
%    \begin{macrocode}
\ProcessOptions \relax
%    \end{macrocode}
% \end{macro}
% \Finale
\endinput
