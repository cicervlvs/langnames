% \iffalse meta-comment
%
% Copyright (C) 2022 by Alejandro García Matarredona
%
% This file may be distributed and/or modified under the
% conditions of the LaTeX Project Public License, either
% version 1.3 of this license or (at your option) any later
% version. The latest version of this license is in:
%
% http://www.latex-project.org/lppl.txt
%
% and version 1.3 or later is part of all distributions of
% LaTeX version 2005/12/01 or later.
%
% \fi
% \iffalse
%<package>\NeedsTeXFormat{LaTeX2e}[2005/12/01]
%<package>\ProvidesPackage{langnames}
%<package> [2022/08/25 v1.0 langnames package for naming and classification of languages]
%
%<*driver>
\documentclass{ltxdoc}
\usepackage{langnames}
\usepackage{expex}
\gathertags
\usepackage{xkeyval}
\usepackage{hyperref}
\hypersetup{
  colorlinks=true,
  urlcolor=blue,
}
\EnableCrossrefs
\CodelineIndex
\RecordChanges
\begin{document}
\DocInput{langnames.dtx}
\end{document}
%</driver>
% \fi
% \CheckSum{31}
% \changes{v1.0}{2022/08/25}{Initial version}
% \GetFileInfo{langnames.sty}
% \DoNotIndex{\unskip}
% \title{The \textsf{langnames} package\thanks{This document
% corresponds to \textsf{langnames}~\fileversion,
% dated~\filedate.}}
% \author{Alejandro García Matarredona\\ \texttt{alejandrogarciaag41@gmail.com}}
% \maketitle
% \begin{abstract}
% \noindent The \textsf{langnames} package provides a set of macros for formatting names of languages, as well as their identification (in the form of ISO 639-3 codes) and their classification (in the form of its top-level family). The datasets from \href{https://wals.info}{WALS} and \href{https://glottolog.org}{Glottolog} are included in the package. The package also allows users to rename and add new languages.
% \end{abstract}
% \section{Introduction}
% The typing out of language names in academic papers, especially those in language typology or related fields where many names have to be typed many times, is often inconvenient and inconsistent. This package attempts to be a small help to writers, especially of large projects or of collaborative ones, to have a slightly easier time with names of languages. It does so by defining three main commands: |\lname|, |\liso|, and |\lfam|, which respectively print out the name, name and ISO 639-3 code, and name and family of the specified language. While the package comes with about 7500 pre-defined languages, with code, name, and family, the user may also define new ones through the |\newlang| command. The basic use of all four of these commands is explained below.
% \section{Usage}
% \subsection{Installation}
% Download the package from wherever it was found to a place where \LaTeX may see it, typically in \$TEXMFHOME/tex/latex. \textsf{langnames} should automatically load the \textsf{xkeyval} package.
% \subsection{Package options}
% When calling |\usepackage{langnames}|, the user must specify one of three options: \textsf{glottolog}, \textsf{wals}, or \textsf{none}.
% The first option, \textsf{glottolog}, selects the naming conventions from the \href{https://glottolog.org}{Glottolog} database. The second option, \textsf{wals}, predictably selects the naming conventions of the \href{https://wals.info}{WALS} database. The names and the genetic classification differ in some languages, so the user may choose what convention to follow.
% During the preparation of the dataset, there were instances of languages which appeared in WALS but not in Glottolog, and vice-versa. In such cases, the missing information was added from the other database. For more details on how I built the dataset, one may consult the Python script made for it in the \href{https://github.com/cicervlvs/langnames}{Github repository}.
% The third option, \textsf{none}, tells the package not to load either of the datasets, and instead start off from an empty canvas. If one specifies this option, one will have to fill in the details of each language with the macro |\newlang| (see explanation in Section 2.3 below).
% \subsection{Macros}\label{sec:mac}
% When referring to a language, the author may use one of three macros to print out different information about it. Languages are identified by their ISO 639-3 code.
% \DescribeMacro{\lname}
% The simplest macro is |\lname|, which prints out the name of the specified language according to the code provided. The basic syntax is thus|\newlang| \marg{ISO code}. This can be seen in example (\getfullref{lname}).\\
% \ex<lname>
% \textsf{My native language is} |\lname{cat}|.\\ \\
% My native language is Catalan. \\
% \xe
% \DescribeMacro{\liso}
% One may also use the |\liso| macro to print out both the name and the ISO 639-3 code of the language specified in the macro in parenthesis , again according to its ISO code(|\liso| \marg{ISO code}). Example (\getfullref{liso}) shows its behavior.\\
% \ex<liso>
% \textsf{I have recently taken up} |\liso{brg}|.\\ \\
% I have recently taken up Baure (Arawakan). \\
% \xe
% \DescribeMacro{\lfam}
% A third macro for use is the |\lfam| command, which prints the name of the language and its family in parenthesis. Once again, the language is identified by its ISO 639-3 code. Example (\getfullref{lfam}) shows how it works.\\
% \ex<lfam>
% \textsf{The tone system of} |\liso{ptk}| is fascinating.\\ \\
% The tone system of Maleng (Austroasiatic) is fascinating. \\
% \xe
% \DescribeMacro{\newlang}
% Finally, users may add their own languages (or change the code, name, or genetic affiliation of a language already in the dataset) via the use of the |\newlang| command, which takes the three arguments \marg{code}, \marg{name}, and \marg{family}. This command may be used in the preamble, before |\begin{document}|. Example (\getfullref{newlang}) shows its usage.\\
% \ex<newlang>
% |\newlang{boo}{Ameli}{Amelian}| \\
% |\begin{document}|\\
% \textsf{My new made up language is} |\lname{boo}|.\\
% \textsf{My new made up language is} |\liso{boo}|.\\
% \textsf{My new made up language is} |\lfam{boo}|.\\
% |\end{document}|\\ \\
% My new made up language is Ameli.\\
% My new made up language is Ameli (ISO 639-3: boo).\\
% My new made up language is Ameli (Amelian).\\
% \xe
% Be aware that setting a new language overwrites any other language with the same code, as the package only listens to the language that is defined last.
% \StopEventually{\PrintIndex\PrintChanges}
% \section{Implementation}
% \subsection{Dependencies}
% The \textsf{langnames} package needs to load the \textsf{xkeyval} package for its key-value pair setting functionality.
%    \begin{macrocode}
\usepackage{xkeyval}
%    \end{macrocode}

% \subsection{Option setting}
% Options are set for what dataset to use. \textsf{glottolog} use Glottolog data;  \textsf{wals} uses WALS data; \textsf{none} selects neither dataset and all languages are defined by the user. See \textsf{langnames.py} in the \href{https://github.com/cicervlvs/langnames}{Github repository} to see how I gathered and handled the data.
% \begin{macro}{Options}
%    \begin{macrocode}
\DeclareOption{glottolog}{\input{langs_glot.tex}
                          \define@key{fams}{knw}{Kxa}
\define@key{fams}{nmn}{Tu}
\define@key{fams}{alu}{Austronesian}
\define@key{fams}{hnh}{Khoe-Kwadi}
\define@key{fams}{xam}{Tu}
\define@key{fams}{huc}{Kxa}
\define@key{fams}{apq}{Great Andamanese}
\define@key{fams}{aiw}{Afro-Asiatic}
\define@key{fams}{aau}{Sepik}
\define@key{fams}{abq}{Northwest Caucasian}
\define@key{fams}{abe}{Algic}
\define@key{fams}{abi}{Niger-Congo}
\define@key{fams}{axb}{Guaicuruan}
\define@key{fams}{abk}{Northwest Caucasian}
\define@key{fams}{abz}{Greater West Bomberai}
\define@key{fams}{kgr}{Isolate}
\define@key{fams}{ace}{Austronesian}
\define@key{fams}{aca}{Arawakan}
\define@key{fams}{acn}{Sino-Tibetan}
\define@key{fams}{ach}{Eastern Sudanic}
\define@key{fams}{acu}{Jivaroan}
\define@key{fams}{acv}{Hokan}
\define@key{fams}{guq}{Tupian}
\define@key{fams}{acr}{Mayan}
\define@key{fams}{kjq}{Keresan}
\define@key{fams}{ads}{other}
\define@key{fams}{adn}{Greater West Bomberai}
\define@key{fams}{adj}{Niger-Congo}
\define@key{fams}{ady}{Northwest Caucasian}
\define@key{fams}{adt}{Pama-Nyungan}
\define@key{fams}{adz}{Austronesian}
\define@key{fams}{awi}{Kamula-Elevala}
\define@key{fams}{afr}{Indo-European}
\define@key{fams}{agd}{Trans-New Guinea}
\define@key{fams}{agq}{Niger-Congo}
\define@key{fams}{ahh}{Trans-New Guinea}
\define@key{fams}{agx}{Nakh-Daghestanian}
\define@key{fams}{agt}{Austronesian}
\define@key{fams}{duo}{Austronesian}
\define@key{fams}{agu}{Mayan}
\define@key{fams}{agr}{Jivaroan}
\define@key{fams}{aht}{Na-Dene}
\define@key{fams}{tba}{Isolate}
\define@key{fams}{ain}{Isolate}
\define@key{fams}{ahp}{Niger-Congo}
\define@key{fams}{aja}{Central Sudanic}
\define@key{fams}{ajg}{Niger-Congo}
\define@key{fams}{aji}{Austronesian}
\define@key{fams}{axk}{Niger-Congo}
\define@key{fams}{abj}{Great Andamanese}
\define@key{fams}{aci}{Great Andamanese}
\define@key{fams}{akx}{Great Andamanese}
\define@key{fams}{aka}{Niger-Congo}
\define@key{fams}{ake}{Cariban}
\define@key{fams}{ahk}{Sino-Tibetan}
\define@key{fams}{akv}{Nakh-Daghestanian}
\define@key{fams}{akl}{Austronesian}
\define@key{fams}{akw}{Niger-Congo}
\define@key{fams}{nrz}{Austronesian}
\define@key{fams}{akz}{Muskogean}
\define@key{fams}{wbj}{Afro-Asiatic}
\define@key{fams}{amp}{Sepik}
\define@key{fams}{btz}{Austronesian}
\define@key{fams}{alh}{Mangarrayi-Maran}
\define@key{fams}{sqi}{Indo-European}
\define@key{fams}{ale}{Eskimo-Aleut}
\define@key{fams}{alq}{Algic}
\define@key{fams}{ald}{Niger-Congo}
\define@key{fams}{gsw}{Indo-European}
\define@key{fams}{aes}{Oregon Coast}
\define@key{fams}{alt}{Altaic}
\define@key{fams}{alp}{Austronesian}
\define@key{fams}{ems}{Eskimo-Aleut}
\define@key{fams}{alr}{Chukotko-Kamchatkan}
\define@key{fams}{aly}{Pama-Nyungan}
\define@key{fams}{amm}{Left May}
\define@key{fams}{amc}{Pano-Tacanan}
\define@key{fams}{amn}{Border}
\define@key{fams}{aie}{Austronesian}
\define@key{fams}{amr}{Harakmbet}
\define@key{fams}{omb}{Austronesian}
\define@key{fams}{amk}{Austronesian}
\define@key{fams}{abt}{Sepik}
\define@key{fams}{adx}{Sino-Tibetan}
\define@key{fams}{aey}{Trans-New Guinea}
\define@key{fams}{ase}{other}
\define@key{fams}{amh}{Afro-Asiatic}
\define@key{fams}{ami}{Austronesian}
\define@key{fams}{amo}{Niger-Congo}
\define@key{fams}{apz}{Trans-New Guinea}
\define@key{fams}{ame}{Arawakan}
\define@key{fams}{amu}{Oto-Manguean}
\define@key{fams}{imi}{Trans-New Guinea}
\define@key{fams}{ani}{Nakh-Daghestanian}
\define@key{fams}{ano}{Isolate}
\define@key{fams}{aty}{Austronesian}
\define@key{fams}{agm}{Trans-New Guinea}
\define@key{fams}{njm}{Sino-Tibetan}
\define@key{fams}{anc}{Afro-Asiatic}
\define@key{fams}{agg}{Senagi}
\define@key{fams}{aoa}{other}
\define@key{fams}{awg}{Pama-Nyungan}
\define@key{fams}{aoi}{Gunwinyguan}
\define@key{fams}{nun}{Sino-Tibetan}
\define@key{fams}{cko}{Niger-Congo}
\define@key{fams}{any}{Niger-Congo}
\define@key{fams}{anu}{Eastern Sudanic}
\define@key{fams}{anz}{Isolate}
\define@key{fams}{njo}{Sino-Tibetan}
\define@key{fams}{apm}{Na-Dene}
\define@key{fams}{apj}{Na-Dene}
\define@key{fams}{apw}{Na-Dene}
\define@key{fams}{apy}{Cariban}
\define@key{fams}{apt}{Sino-Tibetan}
\define@key{fams}{apn}{Macro-Ge}
\define@key{fams}{apu}{Arawakan}
\define@key{fams}{ard}{Pama-Nyungan}
\define@key{fams}{arl}{Zaparoan}
\define@key{fams}{abv}{Afro-Asiatic}
\define@key{fams}{mey}{Afro-Asiatic}
\define@key{fams}{shu}{Afro-Asiatic}
\define@key{fams}{ayl}{Afro-Asiatic}
\define@key{fams}{arz}{Afro-Asiatic}
\define@key{fams}{afb}{Afro-Asiatic}
\define@key{fams}{acw}{Afro-Asiatic}
\define@key{fams}{acm}{Afro-Asiatic}
\define@key{fams}{acy}{Afro-Asiatic}
\define@key{fams}{arb}{Afro-Asiatic}
\define@key{fams}{ary}{Afro-Asiatic}
\define@key{fams}{ajp}{Afro-Asiatic}
\define@key{fams}{ayn}{Afro-Asiatic}
\define@key{fams}{apc}{Afro-Asiatic}
\define@key{fams}{aeb}{Afro-Asiatic}
\define@key{fams}{rmz}{Sino-Tibetan}
\define@key{fams}{akr}{Austronesian}
\define@key{fams}{atq}{Austronesian}
\define@key{fams}{jbj}{South Bird's Head}
\define@key{fams}{aro}{Pano-Tacanan}
\define@key{fams}{arp}{Algic}
\define@key{fams}{aah}{Torricelli}
\define@key{fams}{ape}{Torricelli}
\define@key{fams}{arv}{Afro-Asiatic}
\define@key{fams}{aqc}{Nakh-Daghestanian}
\define@key{fams}{laz}{Austronesian}
\define@key{fams}{ari}{Caddoan}
\define@key{fams}{hye}{Indo-European}
\define@key{fams}{hyw}{Indo-European}
\define@key{fams}{apr}{Austronesian}
\define@key{fams}{aia}{Austronesian}
\define@key{fams}{aer}{Pama-Nyungan}
\define@key{fams}{are}{Pama-Nyungan}
\define@key{fams}{cns}{Asmat-Kamrau Bay}
\define@key{fams}{asm}{Indo-European}
\define@key{fams}{ast}{Indo-European}
\define@key{fams}{asu}{Tupian}
\define@key{fams}{kuz}{Kunza}
\define@key{fams}{aqp}{Isolate}
\define@key{fams}{tay}{Austronesian}
\define@key{fams}{upv}{Austronesian}
\define@key{fams}{aph}{Sino-Tibetan}
\define@key{fams}{atj}{Algic}
\define@key{fams}{atw}{Hokan}
\define@key{fams}{avt}{Torricelli}
\define@key{fams}{aul}{Austronesian}
\define@key{fams}{asf}{other}
\define@key{fams}{auy}{Trans-New Guinea}
\define@key{fams}{ava}{Nakh-Daghestanian}
\define@key{fams}{avn}{Niger-Congo}
\define@key{fams}{avi}{Niger-Congo}
\define@key{fams}{avu}{Central Sudanic}
\define@key{fams}{awb}{Trans-New Guinea}
\define@key{fams}{kwi}{Barbacoan}
\define@key{fams}{awa}{Indo-European}
\define@key{fams}{awn}{Afro-Asiatic}
\define@key{fams}{kmn}{Sepik}
\define@key{fams}{auw}{Border}
\define@key{fams}{nfl}{Austronesian}
\define@key{fams}{ayr}{Aymaran}
\define@key{fams}{aib}{Altaic}
\define@key{fams}{ayo}{Zamucoan}
\define@key{fams}{azb}{Altaic}
\define@key{fams}{koe}{Eastern Sudanic}
\define@key{fams}{bvx}{Niger-Congo}
\define@key{fams}{bav}{Niger-Congo}
\define@key{fams}{wdj}{Wandjiginy}
\define@key{fams}{bfq}{Dravidian}
\define@key{fams}{bde}{Afro-Asiatic}
\define@key{fams}{bia}{Pama-Nyungan}
\define@key{fams}{ksf}{Niger-Congo}
\define@key{fams}{bfd}{Niger-Congo}
\define@key{fams}{bsp}{Niger-Congo}
\define@key{fams}{bmi}{Central Sudanic}
\define@key{fams}{fuu}{Central Sudanic}
\define@key{fams}{bgq}{Indo-European}
\define@key{fams}{kva}{Nakh-Daghestanian}
\define@key{fams}{bdw}{Greater West Bomberai}
\define@key{fams}{bjh}{Sepik}
\define@key{fams}{bdq}{Austro-Asiatic}
\define@key{fams}{bca}{Sino-Tibetan}
\define@key{fams}{bdl}{Austronesian}
\define@key{fams}{bdr}{Austronesian}
\define@key{fams}{bkc}{Niger-Congo}
\define@key{fams}{bdh}{Central Sudanic}
\define@key{fams}{bkq}{Cariban}
\define@key{fams}{bri}{Niger-Congo}
\define@key{fams}{blw}{Austronesian}
\define@key{fams}{blz}{Austronesian}
\define@key{fams}{ban}{Austronesian}
\define@key{fams}{bft}{Sino-Tibetan}
\define@key{fams}{bgn}{Indo-European}
\define@key{fams}{ptu}{Austronesian}
\define@key{fams}{bam}{Mande}
\define@key{fams}{bax}{Niger-Congo}
\define@key{fams}{bcw}{Afro-Asiatic}
\define@key{fams}{jaa}{Arauan}
\define@key{fams}{bza}{Mande}
\define@key{fams}{bdy}{Pama-Nyungan}
\define@key{fams}{bgz}{Austronesian}
\define@key{fams}{bjb}{Pama-Nyungan}
\define@key{fams}{bdg}{Austronesian}
\define@key{fams}{dba}{Isolate}
\define@key{fams}{bvv}{Arawakan}
\define@key{fams}{bwi}{Arawakan}
\define@key{fams}{abb}{Niger-Congo}
\define@key{fams}{bcm}{Austronesian}
\define@key{fams}{bnq}{Austronesian}
\define@key{fams}{peh}{Altaic}
\define@key{fams}{bci}{Niger-Congo}
\define@key{fams}{loy}{Sino-Tibetan}
\define@key{fams}{bbb}{Trans-New Guinea}
\define@key{fams}{brm}{Niger-Congo}
\define@key{fams}{bsn}{Tucanoan}
\define@key{fams}{bcj}{Nyulnyulan}
\define@key{fams}{mlp}{Trans-New Guinea}
\define@key{fams}{bfa}{Eastern Sudanic}
\define@key{fams}{bba}{Niger-Congo}
\define@key{fams}{wra}{Skou}
\define@key{fams}{byr}{Trans-New Guinea}
\define@key{fams}{bae}{Arawakan}
\define@key{fams}{mot}{Chibchan}
\define@key{fams}{bsc}{Niger-Congo}
\define@key{fams}{bas}{Niger-Congo}
\define@key{fams}{bak}{Altaic}
\define@key{fams}{eus}{Isolate}
\define@key{fams}{bya}{Austronesian}
\define@key{fams}{btx}{Austronesian}
\define@key{fams}{bbc}{Austronesian}
\define@key{fams}{bhm}{Afro-Asiatic}
\define@key{fams}{bbd}{Trans-New Guinea}
\define@key{fams}{brg}{Arawakan}
\define@key{fams}{bvz}{Geelvink Bay}
\define@key{fams}{bgr}{Sino-Tibetan}
\define@key{fams}{bsw}{Afro-Asiatic}
\define@key{fams}{bxj}{Pama-Nyungan}
\define@key{fams}{beq}{Niger-Congo}
\define@key{fams}{dbj}{Austronesian}
\define@key{fams}{bej}{Afro-Asiatic}
\define@key{fams}{byw}{Sino-Tibetan}
\define@key{fams}{blc}{Salishan}
\define@key{fams}{bel}{Indo-European}
\define@key{fams}{bem}{Niger-Congo}
\define@key{fams}{bef}{Trans-New Guinea}
\define@key{fams}{nhb}{Mande}
\define@key{fams}{bng}{Niger-Congo}
\define@key{fams}{ben}{Indo-European}
\define@key{fams}{ctg}{Indo-European}
\define@key{fams}{bue}{Isolate}
\define@key{fams}{brf}{Niger-Congo}
\define@key{fams}{shy}{Afro-Asiatic}
\define@key{fams}{grr}{Afro-Asiatic}
\define@key{fams}{tzm}{Afro-Asiatic}
\define@key{fams}{mzb}{Afro-Asiatic}
\define@key{fams}{rif}{Afro-Asiatic}
\define@key{fams}{siz}{Afro-Asiatic}
\define@key{fams}{oua}{Afro-Asiatic}
\define@key{fams}{brc}{other}
\define@key{fams}{zag}{Saharan}
\define@key{fams}{bkl}{Tor-Kwerba}
\define@key{fams}{wti}{Isolate}
\define@key{fams}{xub}{Dravidian}
\define@key{fams}{kap}{Nakh-Daghestanian}
\define@key{fams}{bhb}{Indo-European}
\define@key{fams}{bho}{Indo-European}
\define@key{fams}{unr}{Austro-Asiatic}
\define@key{fams}{bif}{Niger-Congo}
\define@key{fams}{bhw}{Austronesian}
\define@key{fams}{bth}{Austronesian}
\define@key{fams}{bid}{Afro-Asiatic}
\define@key{fams}{bcl}{Austronesian}
\define@key{fams}{bip}{Niger-Congo}
\define@key{fams}{bpr}{Austronesian}
\define@key{fams}{byn}{Afro-Asiatic}
\define@key{fams}{nbj}{Pama-Nyungan}
\define@key{fams}{bll}{Siouan}
\define@key{fams}{blb}{Solomons East Papuan}
\define@key{fams}{bhp}{Austronesian}
\define@key{fams}{bim}{Niger-Congo}
\define@key{fams}{bhg}{Trans-New Guinea}
\define@key{fams}{bin}{Niger-Congo}
\define@key{fams}{gup}{Gunwinyguan}
\define@key{fams}{bkd}{Austronesian}
\define@key{fams}{bjr}{Trans-New Guinea}
\define@key{fams}{bzr}{Pama-Nyungan}
\define@key{fams}{bom}{Niger-Congo}
\define@key{fams}{bvq}{Central Sudanic}
\define@key{fams}{bib}{Mande}
\define@key{fams}{bis}{other}
\define@key{fams}{bla}{Algic}
\define@key{fams}{kvg}{Trans-New Guinea}
\define@key{fams}{bni}{Niger-Congo}
\define@key{fams}{bbo}{Mande}
\define@key{fams}{brx}{Sino-Tibetan}
\define@key{fams}{bzf}{Sepik}
\define@key{fams}{bqc}{Mande}
\define@key{fams}{bol}{Afro-Asiatic}
\define@key{fams}{bli}{Niger-Congo}
\define@key{fams}{bot}{Central Sudanic}
\define@key{fams}{bpu}{Trans-New Guinea}
\define@key{fams}{lbk}{Austronesian}
\define@key{fams}{boa}{Boran}
\define@key{fams}{adi}{Sino-Tibetan}
\define@key{fams}{bor}{Bororoan}
\define@key{fams}{brn}{Chibchan}
\define@key{fams}{bos}{Indo-European}
\define@key{fams}{boz}{Mande}
\define@key{fams}{brh}{Dravidian}
\define@key{fams}{brb}{Austro-Asiatic}
\define@key{fams}{bre}{Indo-European}
\define@key{fams}{bzd}{Chibchan}
\define@key{fams}{bfi}{other}
\define@key{fams}{tcs}{other}
\define@key{fams}{bkk}{Indo-European}
\define@key{fams}{bru}{Austro-Asiatic}
\define@key{fams}{brv}{Austro-Asiatic}
\define@key{fams}{bvb}{Niger-Congo}
\define@key{fams}{buu}{Niger-Congo}
\define@key{fams}{bdk}{Nakh-Daghestanian}
\define@key{fams}{bdm}{Afro-Asiatic}
\define@key{fams}{bug}{Austronesian}
\define@key{fams}{sab}{Chibchan}
\define@key{fams}{bgg}{Sino-Tibetan}
\define@key{fams}{buo}{South Bougainville}
\define@key{fams}{nmg}{Niger-Congo}
\define@key{fams}{bxk}{Niger-Congo}
\define@key{fams}{bul}{Indo-European}
\define@key{fams}{bwu}{Niger-Congo}
\define@key{fams}{bzq}{Austronesian}
\define@key{fams}{bum}{Niger-Congo}
\define@key{fams}{tkw}{Austronesian}
\define@key{fams}{bfu}{Sino-Tibetan}
\define@key{fams}{buh}{Hmong-Mien}
\define@key{fams}{bck}{Bunuban}
\define@key{fams}{bwr}{Afro-Asiatic}
\define@key{fams}{bvr}{Mangrida}
\define@key{fams}{bxm}{Altaic}
\define@key{fams}{bji}{Afro-Asiatic}
\define@key{fams}{mya}{Sino-Tibetan}
\define@key{fams}{mhs}{Austronesian}
\define@key{fams}{bmu}{Trans-New Guinea}
\define@key{fams}{bds}{Afro-Asiatic}
\define@key{fams}{bsk}{Isolate}
\define@key{fams}{bqp}{Mande}
\define@key{fams}{buf}{Niger-Congo}
\define@key{fams}{ngc}{Niger-Congo}
\define@key{fams}{bee}{Sino-Tibetan}
\define@key{fams}{bev}{Niger-Congo}
\define@key{fams}{cjp}{Chibchan}
\define@key{fams}{cbv}{Cacua-Nukak}
\define@key{fams}{cad}{Caddoan}
\define@key{fams}{chl}{Uto-Aztecan}
\define@key{fams}{cak}{Mayan}
\define@key{fams}{rab}{Sino-Tibetan}
\define@key{fams}{cjo}{Arawakan}
\define@key{fams}{kbh}{Isolate}
\define@key{fams}{knm}{Katukinan}
\define@key{fams}{cbu}{Isolate}
\define@key{fams}{ram}{Macro-Ge}
\define@key{fams}{yue}{Sino-Tibetan}
\define@key{fams}{kaq}{Pano-Tacanan}
\define@key{fams}{cbc}{Tucanoan}
\define@key{fams}{car}{Cariban}
\define@key{fams}{mch}{Cariban}
\define@key{fams}{cal}{Austronesian}
\define@key{fams}{crx}{Na-Dene}
\define@key{fams}{cbr}{Pano-Tacanan}
\define@key{fams}{cbs}{Pano-Tacanan}
\define@key{fams}{cat}{Indo-European}
\define@key{fams}{chc}{Siouan}
\define@key{fams}{cto}{Choco}
\define@key{fams}{cav}{Pano-Tacanan}
\define@key{fams}{cbi}{Barbacoan}
\define@key{fams}{cay}{Iroquoian}
\define@key{fams}{cyb}{Isolate}
\define@key{fams}{ceb}{Austronesian}
\define@key{fams}{old}{Niger-Congo}
\define@key{fams}{suq}{Eastern Sudanic}
\define@key{fams}{cld}{Afro-Asiatic}
\define@key{fams}{cjm}{Austronesian}
\define@key{fams}{cja}{Austronesian}
\define@key{fams}{cji}{Nakh-Daghestanian}
\define@key{fams}{can}{Lower Sepik-Ramu}
\define@key{fams}{cha}{Austronesian}
\define@key{fams}{nbc}{Sino-Tibetan}
\define@key{fams}{chx}{Sino-Tibetan}
\define@key{fams}{tuu}{Na-Dene}
\define@key{fams}{cya}{Oto-Manguean}
\define@key{fams}{cta}{Oto-Manguean}
\define@key{fams}{ctp}{Oto-Manguean}
\define@key{fams}{cdn}{Sino-Tibetan}
\define@key{fams}{cbk}{other}
\define@key{fams}{cbt}{Cahuapanan}
\define@key{fams}{che}{Nakh-Daghestanian}
\define@key{fams}{cjh}{Salishan}
\define@key{fams}{mrn}{Austronesian}
\define@key{fams}{xch}{Chimakuan}
\define@key{fams}{cdm}{Sino-Tibetan}
\define@key{fams}{chr}{Iroquoian}
\define@key{fams}{chy}{Algic}
\define@key{fams}{nya}{Niger-Congo}
\define@key{fams}{pei}{Oto-Manguean}
\define@key{fams}{cic}{Muskogean}
\define@key{fams}{cob}{Mayan}
\define@key{fams}{cid}{Hokan}
\define@key{fams}{cbg}{Chibchan}
\define@key{fams}{mrh}{Sino-Tibetan}
\define@key{fams}{csy}{Sino-Tibetan}
\define@key{fams}{ctd}{Sino-Tibetan}
\define@key{fams}{cco}{Oto-Manguean}
\define@key{fams}{cle}{Oto-Manguean}
\define@key{fams}{cpa}{Oto-Manguean}
\define@key{fams}{chq}{Oto-Manguean}
\define@key{fams}{cuc}{Oto-Manguean}
\define@key{fams}{cso}{Oto-Manguean}
\define@key{fams}{cnt}{Oto-Manguean}
\define@key{fams}{csl}{other}
\define@key{fams}{chh}{Penutian}
\define@key{fams}{wac}{Penutian}
\define@key{fams}{cap}{Uru-Chipaya}
\define@key{fams}{chp}{Na-Dene}
\define@key{fams}{cax}{Isolate}
\define@key{fams}{gui}{Tupian}
\define@key{fams}{ctm}{Isolate}
\define@key{fams}{coz}{Oto-Manguean}
\define@key{fams}{cho}{Muskogean}
\define@key{fams}{ctu}{Mayan}
\define@key{fams}{cht}{Hobitu-Cholon}
\define@key{fams}{chd}{Hokan}
\define@key{fams}{clo}{Hokan}
\define@key{fams}{chf}{Mayan}
\define@key{fams}{caa}{Mayan}
\define@key{fams}{crw}{Austro-Asiatic}
\define@key{fams}{cje}{Austronesian}
\define@key{fams}{cjv}{Trans-New Guinea}
\define@key{fams}{cac}{Mayan}
\define@key{fams}{ckt}{Chukotko-Kamchatkan}
\define@key{fams}{clw}{Altaic}
\define@key{fams}{boi}{Chumash}
\define@key{fams}{inz}{Chumash}
\define@key{fams}{ncu}{Niger-Congo}
\define@key{fams}{chk}{Austronesian}
\define@key{fams}{chv}{Altaic}
\define@key{fams}{cao}{Pano-Tacanan}
\define@key{fams}{lua}{Niger-Congo}
\define@key{fams}{clm}{Salishan}
\define@key{fams}{xcw}{Coahuiltecan}
\define@key{fams}{cod}{Tupian}
\define@key{fams}{coc}{Hokan}
\define@key{fams}{crd}{Salishan}
\define@key{fams}{con}{Isolate}
\define@key{fams}{kog}{Chibchan}
\define@key{fams}{col}{Salishan}
\define@key{fams}{com}{Uto-Aztecan}
\define@key{fams}{xcm}{Hokan}
\define@key{fams}{swb}{Niger-Congo}
\define@key{fams}{coo}{Salishan}
\define@key{fams}{csz}{Oregon Coast}
\define@key{fams}{cop}{Afro-Asiatic}
\define@key{fams}{crn}{Uto-Aztecan}
\define@key{fams}{cor}{Indo-European}
\define@key{fams}{crk}{Algic}
\define@key{fams}{csw}{Algic}
\define@key{fams}{mus}{Muskogean}
\define@key{fams}{crh}{Altaic}
\define@key{fams}{cro}{Siouan}
\define@key{fams}{cua}{Austro-Asiatic}
\define@key{fams}{cub}{Tucanoan}
\define@key{fams}{cui}{Guahiban}
\define@key{fams}{cuy}{Isolate}
\define@key{fams}{cul}{Arauan}
\define@key{fams}{cup}{Uto-Aztecan}
\define@key{fams}{kpc}{Arawakan}
\define@key{fams}{ces}{Indo-European}
\define@key{fams}{cam}{Austronesian}
\define@key{fams}{kzf}{Austronesian}
\define@key{fams}{dbq}{Afro-Asiatic}
\define@key{fams}{dav}{Niger-Congo}
\define@key{fams}{mps}{Teberan-Pawaian}
\define@key{fams}{dgz}{Trans-New Guinea}
\define@key{fams}{dga}{Niger-Congo}
\define@key{fams}{dag}{Niger-Congo}
\define@key{fams}{dta}{Altaic}
\define@key{fams}{dal}{Afro-Asiatic}
\define@key{fams}{daj}{Eastern Sudanic}
\define@key{fams}{dak}{Siouan}
\define@key{fams}{mbp}{Chibchan}
\define@key{fams}{dnj}{Mande}
\define@key{fams}{daa}{Afro-Asiatic}
\define@key{fams}{dni}{Trans-New Guinea}
\define@key{fams}{dan}{Indo-European}
\define@key{fams}{dry}{Indo-European}
\define@key{fams}{dar}{Nakh-Daghestanian}
\define@key{fams}{prs}{Indo-European}
\define@key{fams}{drd}{Sino-Tibetan}
\define@key{fams}{tcc}{Eastern Sudanic}
\define@key{fams}{dai}{Niger-Congo}
\define@key{fams}{afn}{Ijoid}
\define@key{fams}{deg}{Niger-Congo}
\define@key{fams}{ing}{Na-Dene}
\define@key{fams}{dny}{Arauan}
\define@key{fams}{des}{Tucanoan}
\define@key{fams}{shg}{Khoe-Kwadi}
\define@key{fams}{der}{Sino-Tibetan}
\define@key{fams}{gsg}{other}
\define@key{fams}{dsh}{Afro-Asiatic}
\define@key{fams}{dhl}{Pama-Nyungan}
\define@key{fams}{tbh}{Pama-Nyungan}
\define@key{fams}{dhr}{Pama-Nyungan}
\define@key{fams}{xgm}{Pama-Nyungan}
\define@key{fams}{dhi}{Sino-Tibetan}
\define@key{fams}{div}{Indo-European}
\define@key{fams}{dhu}{Pama-Nyungan}
\define@key{fams}{did}{Eastern Sudanic}
\define@key{fams}{mhu}{Sino-Tibetan}
\define@key{fams}{dur}{Niger-Congo}
\define@key{fams}{dis}{Sino-Tibetan}
\define@key{fams}{dim}{Afro-Asiatic}
\define@key{fams}{diz}{Niger-Congo}
\define@key{fams}{din}{Eastern Sudanic}
\define@key{fams}{dyo}{Niger-Congo}
\define@key{fams}{csk}{Niger-Congo}
\define@key{fams}{dif}{Pama-Nyungan}
\define@key{fams}{mdx}{Afro-Asiatic}
\define@key{fams}{dyy}{Pama-Nyungan}
\define@key{fams}{djr}{Pama-Nyungan}
\define@key{fams}{duj}{Pama-Nyungan}
\define@key{fams}{ddj}{Pama-Nyungan}
\define@key{fams}{dji}{Pama-Nyungan}
\define@key{fams}{jig}{Mirndi}
\define@key{fams}{kbv}{Senagi}
\define@key{fams}{kvo}{Austronesian}
\define@key{fams}{dgo}{Indo-European}
\define@key{fams}{dlg}{Altaic}
\define@key{fams}{dmk}{Indo-European}
\define@key{fams}{rmt}{Indo-European}
\define@key{fams}{kmc}{Tai-Kadai}
\define@key{fams}{doo}{Niger-Congo}
\define@key{fams}{dds}{Dogon}
\define@key{fams}{tds}{Lakes Plain}
\define@key{fams}{dow}{Niger-Congo}
\define@key{fams}{dhv}{Austronesian}
\define@key{fams}{dua}{Niger-Congo}
\define@key{fams}{dud}{Niger-Congo}
\define@key{fams}{gwd}{Afro-Asiatic}
\define@key{fams}{duu}{Sino-Tibetan}
\define@key{fams}{dma}{Niger-Congo}
\define@key{fams}{dgc}{Austronesian}
\define@key{fams}{dus}{Sino-Tibetan}
\define@key{fams}{vam}{Skou}
\define@key{fams}{duc}{Duna-Bogaya}
\define@key{fams}{nld}{Indo-European}
\define@key{fams}{zea}{Indo-European}
\define@key{fams}{dyi}{Niger-Congo}
\define@key{fams}{dbl}{Pama-Nyungan}
\define@key{fams}{dyu}{Mande}
\define@key{fams}{kwa}{Nadahup}
\define@key{fams}{igb}{Niger-Congo}
\define@key{fams}{etr}{Trans-New Guinea}
\define@key{fams}{erk}{Austronesian}
\define@key{fams}{efi}{Niger-Congo}
\define@key{fams}{ega}{Niger-Congo}
\define@key{fams}{eip}{Trans-New Guinea}
\define@key{fams}{etu}{Niger-Congo}
\define@key{fams}{ekg}{Trans-New Guinea}
\define@key{fams}{eko}{Niger-Congo}
\define@key{fams}{mrf}{Morwap}
\define@key{fams}{ema}{Niger-Congo}
\define@key{fams}{emb}{Austronesian}
\define@key{fams}{cmi}{Choco}
\define@key{fams}{emp}{Choco}
\define@key{fams}{amy}{Western Daly}
\define@key{fams}{enq}{Trans-New Guinea}
\define@key{fams}{enn}{Niger-Congo}
\define@key{fams}{eno}{Austronesian}
\define@key{fams}{eng}{Indo-European}
\define@key{fams}{gey}{Niger-Congo}
\define@key{fams}{sja}{Choco}
\define@key{fams}{erg}{Austronesian}
\define@key{fams}{ese}{Pano-Tacanan}
\define@key{fams}{esq}{Isolate}
\define@key{fams}{ekk}{Uralic}
\define@key{fams}{ets}{Niger-Congo}
\define@key{fams}{eve}{Altaic}
\define@key{fams}{ewe}{Niger-Congo}
\define@key{fams}{ewo}{Niger-Congo}
\define@key{fams}{eya}{Na-Dene}
\define@key{fams}{fao}{Indo-European}
\define@key{fams}{faa}{Trans-New Guinea}
\define@key{fams}{fmp}{Niger-Congo}
\define@key{fams}{fij}{Austronesian}
\define@key{fams}{fin}{Uralic}
\define@key{fams}{fse}{other}
\define@key{fams}{foi}{Trans-New Guinea}
\define@key{fams}{ppo}{Teberan-Pawaian}
\define@key{fams}{fon}{Niger-Congo}
\define@key{fams}{frd}{Austronesian}
\define@key{fams}{for}{Trans-New Guinea}
\define@key{fams}{sac}{Algic}
\define@key{fams}{fra}{Indo-European}
\define@key{fams}{fry}{Indo-European}
\define@key{fams}{frs}{Indo-European}
\define@key{fams}{frr}{Indo-European}
\define@key{fams}{fuh}{Niger-Congo}
\define@key{fams}{fuf}{Niger-Congo}
\define@key{fams}{fub}{Niger-Congo}
\define@key{fams}{ffm}{Niger-Congo}
\define@key{fams}{fuv}{Niger-Congo}
\define@key{fams}{fun}{Isolate}
\define@key{fams}{fvr}{Isolate}
\define@key{fams}{fud}{Austronesian}
\define@key{fams}{fut}{Austronesian}
\define@key{fams}{cdo}{Sino-Tibetan}
\define@key{fams}{pym}{Niger-Congo}
\define@key{fams}{gqa}{Afro-Asiatic}
\define@key{fams}{gbu}{Isolate}
\define@key{fams}{dhg}{Pama-Nyungan}
\define@key{fams}{gdb}{Dravidian}
\define@key{fams}{ged}{Niger-Congo}
\define@key{fams}{gaj}{Trans-New Guinea}
\define@key{fams}{gla}{Indo-European}
\define@key{fams}{gag}{Altaic}
\define@key{fams}{gah}{Trans-New Guinea}
\define@key{fams}{gbi}{North Halmaheran}
\define@key{fams}{glg}{Indo-European}
\define@key{fams}{adl}{Sino-Tibetan}
\define@key{fams}{kld}{Pama-Nyungan}
\define@key{fams}{gmv}{Afro-Asiatic}
\define@key{fams}{pwg}{Austronesian}
\define@key{fams}{grt}{Sino-Tibetan}
\define@key{fams}{wrk}{Garrwan}
\define@key{fams}{gyb}{Trans-New Guinea}
\define@key{fams}{cab}{Arawakan}
\define@key{fams}{gvo}{Tupian}
\define@key{fams}{gay}{Austronesian}
\define@key{fams}{gya}{Niger-Congo}
\define@key{fams}{gso}{Niger-Congo}
\define@key{fams}{gbp}{Niger-Congo}
\define@key{fams}{nlg}{Austronesian}
\define@key{fams}{gqu}{Tai-Kadai}
\define@key{fams}{kat}{Kartvelian}
\define@key{fams}{deu}{Indo-European}
\define@key{fams}{bar}{Indo-European}
\define@key{fams}{ksh}{Indo-European}
\define@key{fams}{wep}{Indo-European}
\define@key{fams}{aaa}{Niger-Congo}
\define@key{fams}{ghl}{Eastern Sudanic}
\define@key{fams}{gih}{Pama-Nyungan}
\define@key{fams}{gid}{Afro-Asiatic}
\define@key{fams}{glk}{Indo-European}
\define@key{fams}{bcq}{Afro-Asiatic}
\define@key{fams}{git}{Tsimshianic}
\define@key{fams}{gis}{Afro-Asiatic}
\define@key{fams}{guc}{Arawakan}
\define@key{fams}{god}{Niger-Congo}
\define@key{fams}{gdo}{Nakh-Daghestanian}
\define@key{fams}{ank}{Afro-Asiatic}
\define@key{fams}{ggw}{Trans-New Guinea}
\define@key{fams}{gju}{Indo-European}
\define@key{fams}{gkn}{Niger-Congo}
\define@key{fams}{gol}{Niger-Congo}
\define@key{fams}{gvf}{Trans-New Guinea}
\define@key{fams}{gno}{Dravidian}
\define@key{fams}{gni}{Bunuban}
\define@key{fams}{gor}{Austronesian}
\define@key{fams}{gow}{Afro-Asiatic}
\define@key{fams}{grj}{Niger-Congo}
\define@key{fams}{ell}{Indo-European}
\define@key{fams}{gss}{other}
\define@key{fams}{kal}{Eskimo-Aleut}
\define@key{fams}{guh}{Guahiban}
\define@key{fams}{gub}{Tupian}
\define@key{fams}{gum}{Barbacoan}
\define@key{fams}{gva}{Mascoian}
\define@key{fams}{gvc}{Tucanoan}
\define@key{fams}{gug}{Tupian}
\define@key{fams}{var}{Uto-Aztecan}
\define@key{fams}{gta}{Isolate}
\define@key{fams}{guo}{Guahiban}
\define@key{fams}{gde}{Afro-Asiatic}
\define@key{fams}{gdf}{Afro-Asiatic}
\define@key{fams}{ktd}{Pama-Nyungan}
\define@key{fams}{ggd}{Pama-Nyungan}
\define@key{fams}{ghs}{Trans-New Guinea}
\define@key{fams}{gcr}{other}
\define@key{fams}{pov}{other}
\define@key{fams}{guj}{Indo-European}
\define@key{fams}{kcm}{Central Sudanic}
\define@key{fams}{glj}{Niger-Congo}
\define@key{fams}{gnn}{Pama-Nyungan}
\define@key{fams}{gvs}{Austronesian}
\define@key{fams}{kgs}{Pama-Nyungan}
\define@key{fams}{guk}{Isolate}
\define@key{fams}{wlg}{Gunwinyguan}
\define@key{fams}{guw}{Niger-Congo}
\define@key{fams}{gww}{Worrorran}
\define@key{fams}{yas}{Niger-Congo}
\define@key{fams}{gyy}{Pama-Nyungan}
\define@key{fams}{guf}{Pama-Nyungan}
\define@key{fams}{gnr}{Pama-Nyungan}
\define@key{fams}{gur}{Niger-Congo}
\define@key{fams}{gue}{Pama-Nyungan}
\define@key{fams}{gux}{Niger-Congo}
\define@key{fams}{goa}{Mande}
\define@key{fams}{gge}{Mangrida}
\define@key{fams}{guz}{Niger-Congo}
\define@key{fams}{gbj}{Austro-Asiatic}
\define@key{fams}{kky}{Pama-Nyungan}
\define@key{fams}{gbr}{Niger-Congo}
\define@key{fams}{kcg}{Niger-Congo}
\define@key{fams}{gaa}{Niger-Congo}
\define@key{fams}{pue}{Chonan}
\define@key{fams}{hts}{Isolate}
\define@key{fams}{hai}{Isolate}
\define@key{fams}{hdn}{Haida}
\define@key{fams}{has}{Wakashan}
\define@key{fams}{hat}{other}
\define@key{fams}{hak}{Sino-Tibetan}
\define@key{fams}{hal}{Austro-Asiatic}
\define@key{fams}{hlb}{Indo-European}
\define@key{fams}{hla}{Austronesian}
\define@key{fams}{amf}{Afro-Asiatic}
\define@key{fams}{hmt}{Trans-New Guinea}
\define@key{fams}{wos}{Sepik}
\define@key{fams}{hni}{Sino-Tibetan}
\define@key{fams}{hnn}{Austronesian}
\define@key{fams}{har}{Afro-Asiatic}
\define@key{fams}{hss}{Afro-Asiatic}
\define@key{fams}{tmd}{Piawi}
\define@key{fams}{had}{Hatim-Mansim}
\define@key{fams}{hau}{Afro-Asiatic}
\define@key{fams}{haw}{Austronesian}
\define@key{fams}{hwc}{other}
\define@key{fams}{hac}{Indo-European}
\define@key{fams}{hay}{Niger-Congo}
\define@key{fams}{vay}{Sino-Tibetan}
\define@key{fams}{xed}{Afro-Asiatic}
\define@key{fams}{heb}{Afro-Asiatic}
\define@key{fams}{heh}{Niger-Congo}
\define@key{fams}{hei}{Wakashan}
\define@key{fams}{hem}{Niger-Congo}
\define@key{fams}{her}{Niger-Congo}
\define@key{fams}{hid}{Siouan}
\define@key{fams}{hil}{Austronesian}
\define@key{fams}{hin}{Indo-European}
\define@key{fams}{gin}{Nakh-Daghestanian}
\define@key{fams}{hix}{Cariban}
\define@key{fams}{lic}{Tai-Kadai}
\define@key{fams}{hmr}{Sino-Tibetan}
\define@key{fams}{mww}{Hmong-Mien}
\define@key{fams}{hnj}{Hmong-Mien}
\define@key{fams}{hoc}{Austro-Asiatic}
\define@key{fams}{hoa}{Austronesian}
\define@key{fams}{hoo}{Niger-Congo}
\define@key{fams}{hks}{other}
\define@key{fams}{hop}{Uto-Aztecan}
\define@key{fams}{hre}{Austro-Asiatic}
\define@key{fams}{ygr}{Trans-New Guinea}
\define@key{fams}{hub}{Jivaroan}
\define@key{fams}{hus}{Mayan}
\define@key{fams}{huv}{Huavean}
\define@key{fams}{hch}{Uto-Aztecan}
\define@key{fams}{hto}{Witotoan}
\define@key{fams}{hux}{Witotoan}
\define@key{fams}{huu}{Witotoan}
\define@key{fams}{hke}{Niger-Congo}
\define@key{fams}{hun}{Uralic}
\define@key{fams}{huz}{Nakh-Daghestanian}
\define@key{fams}{jup}{Nadahup}
\define@key{fams}{hup}{Na-Dene}
\define@key{fams}{csh}{Sino-Tibetan}
\define@key{fams}{ksi}{Skou}
\define@key{fams}{iai}{Austronesian}
\define@key{fams}{ian}{Sepik}
\define@key{fams}{tmu}{Lakes Plain}
\define@key{fams}{iba}{Austronesian}
\define@key{fams}{ibg}{Austronesian}
\define@key{fams}{ibb}{Niger-Congo}
\define@key{fams}{isl}{Indo-European}
\define@key{fams}{icl}{other}
\define@key{fams}{idu}{Niger-Congo}
\define@key{fams}{clk}{Sino-Tibetan}
\define@key{fams}{viv}{Austronesian}
\define@key{fams}{mxe}{Austronesian}
\define@key{fams}{ifb}{Austronesian}
\define@key{fams}{ifm}{Niger-Congo}
\define@key{fams}{ibo}{Niger-Congo}
\define@key{fams}{ige}{Niger-Congo}
\define@key{fams}{ign}{Arawakan}
\define@key{fams}{ihp}{Greater West Bomberai}
\define@key{fams}{ijc}{Ijoid}
\define@key{fams}{ikx}{Eastern Sudanic}
\define@key{fams}{arh}{Chibchan}
\define@key{fams}{ilb}{Niger-Congo}
\define@key{fams}{mia}{Algic}
\define@key{fams}{ilo}{Austronesian}
\define@key{fams}{imn}{Border}
\define@key{fams}{szp}{South Bird's Head}
\define@key{fams}{ins}{other}
\define@key{fams}{pks}{other}
\define@key{fams}{ind}{Austronesian}
\define@key{fams}{pmy}{Austronesian}
\define@key{fams}{inb}{Quechuan}
\define@key{fams}{tbi}{Eastern Sudanic}
\define@key{fams}{inh}{Nakh-Daghestanian}
\define@key{fams}{ynd}{Pama-Nyungan}
\define@key{fams}{ils}{other}
\define@key{fams}{ike}{Eskimo-Aleut}
\define@key{fams}{iqu}{Zaparoan}
\define@key{fams}{irn}{Isolate}
\define@key{fams}{irk}{Afro-Asiatic}
\define@key{fams}{irh}{Austronesian}
\define@key{fams}{gle}{Indo-European}
\define@key{fams}{isg}{other}
\define@key{fams}{its}{Niger-Congo}
\define@key{fams}{isk}{Indo-European}
\define@key{fams}{srl}{Tor-Kwerba}
\define@key{fams}{isd}{Austronesian}
\define@key{fams}{iso}{Niger-Congo}
\define@key{fams}{isr}{other}
\define@key{fams}{ita}{Indo-European}
\define@key{fams}{egl}{Indo-European}
\define@key{fams}{lij}{Indo-European}
\define@key{fams}{nap}{Indo-European}
\define@key{fams}{pms}{Indo-European}
\define@key{fams}{itv}{Austronesian}
\define@key{fams}{itl}{Chukotko-Kamchatkan}
\define@key{fams}{ito}{Isolate}
\define@key{fams}{itz}{Mayan}
\define@key{fams}{ivb}{Austronesian}
\define@key{fams}{ibd}{Iwaidjan}
\define@key{fams}{iwm}{Sepik}
\define@key{fams}{yom}{Niger-Congo}
\define@key{fams}{ixc}{Oto-Manguean}
\define@key{fams}{ixl}{Mayan}
\define@key{fams}{izr}{Niger-Congo}
\define@key{fams}{izh}{Uralic}
\define@key{fams}{izz}{Niger-Congo}
\define@key{fams}{esi}{Eskimo-Aleut}
\define@key{fams}{jbt}{Macro-Ge}
\define@key{fams}{jae}{Austronesian}
\define@key{fams}{jda}{Sino-Tibetan}
\define@key{fams}{jhi}{Austro-Asiatic}
\define@key{fams}{jac}{Mayan}
\define@key{fams}{jam}{other}
\define@key{fams}{djd}{Mirndi}
\define@key{fams}{djm}{Dogon}
\define@key{fams}{jpn}{Isolate}
\define@key{fams}{jru}{Cariban}
\define@key{fams}{jqr}{Aymaran}
\define@key{fams}{anq}{South Andamanese}
\define@key{fams}{jav}{Austronesian}
\define@key{fams}{jeb}{Cahuapanan}
\define@key{fams}{jeh}{Austro-Asiatic}
\define@key{fams}{jek}{Mande}
\define@key{fams}{tow}{Kiowa-Tanoan}
\define@key{fams}{jya}{Sino-Tibetan}
\define@key{fams}{shv}{Afro-Asiatic}
\define@key{fams}{kac}{Sino-Tibetan}
\define@key{fams}{jiu}{Sino-Tibetan}
\define@key{fams}{jiv}{Jivaroan}
\define@key{fams}{rgk}{Sino-Tibetan}
\define@key{fams}{tlo}{Kordofanian}
\define@key{fams}{jun}{Austro-Asiatic}
\define@key{fams}{nst}{Sino-Tibetan}
\define@key{fams}{jbu}{Niger-Congo}
\define@key{fams}{bex}{Central Sudanic}
\define@key{fams}{juc}{Altaic}
\define@key{fams}{jur}{Tupian}
\define@key{fams}{ktz}{Kxa}
\define@key{fams}{jua}{Tupian}
\define@key{fams}{kek}{Mayan}
\define@key{fams}{kbd}{Northwest Caucasian}
\define@key{fams}{xkp}{Indo-European}
\define@key{fams}{kbp}{Niger-Congo}
\define@key{fams}{nbu}{Sino-Tibetan}
\define@key{fams}{kab}{Afro-Asiatic}
\define@key{fams}{xac}{Sino-Tibetan}
\define@key{fams}{kzj}{Austronesian}
\define@key{fams}{kbc}{Guaicuruan}
\define@key{fams}{kdm}{Niger-Congo}
\define@key{fams}{kki}{Niger-Congo}
\define@key{fams}{kct}{Lower Sepik-Ramu}
\define@key{fams}{lew}{Austronesian}
\define@key{fams}{kgp}{Macro-Ge}
\define@key{fams}{kxa}{Austronesian}
\define@key{fams}{kgk}{Tupian}
\define@key{fams}{tbd}{Tate}
\define@key{fams}{mwp}{Pama-Nyungan}
\define@key{fams}{kmh}{Trans-New Guinea}
\define@key{fams}{gwc}{Indo-European}
\define@key{fams}{kck}{Niger-Congo}
\define@key{fams}{kyl}{Kalapuyan}
\define@key{fams}{kls}{Indo-European}
\define@key{fams}{fla}{Salishan}
\define@key{fams}{ktg}{Pama-Nyungan}
\define@key{fams}{bco}{Trans-New Guinea}
\define@key{fams}{kay}{Tupian}
\define@key{fams}{kbq}{Trans-New Guinea}
\define@key{fams}{kms}{Torricelli}
\define@key{fams}{xas}{Uralic}
\define@key{fams}{kam}{Niger-Congo}
\define@key{fams}{xbr}{Austronesian}
\define@key{fams}{kbx}{Lower Sepik-Ramu}
\define@key{fams}{kcu}{Niger-Congo}
\define@key{fams}{kgq}{Asmat-Kamrau Bay}
\define@key{fams}{xmu}{Eastern Daly}
\define@key{fams}{ogo}{Niger-Congo}
\define@key{fams}{kna}{Afro-Asiatic}
\define@key{fams}{xns}{Sino-Tibetan}
\define@key{fams}{kbl}{Saharan}
\define@key{fams}{ikt}{Eskimo-Aleut}
\define@key{fams}{kjb}{Mayan}
\define@key{fams}{knj}{Mayan}
\define@key{fams}{kne}{Austronesian}
\define@key{fams}{kan}{Dravidian}
\define@key{fams}{kxo}{Kapixana}
\define@key{fams}{khd}{Yam}
\define@key{fams}{kcd}{Yam}
\define@key{fams}{knc}{Saharan}
\define@key{fams}{kny}{Niger-Congo}
\define@key{fams}{pam}{Austronesian}
\define@key{fams}{kpg}{Austronesian}
\define@key{fams}{kah}{Central Sudanic}
\define@key{fams}{leu}{Austronesian}
\define@key{fams}{krc}{Altaic}
\define@key{fams}{gbd}{Pama-Nyungan}
\define@key{fams}{kdr}{Altaic}
\define@key{fams}{kpj}{Macro-Ge}
\define@key{fams}{kaa}{Altaic}
\define@key{fams}{zkk}{Isolate}
\define@key{fams}{kyj}{Austronesian}
\define@key{fams}{kpt}{Nakh-Daghestanian}
\define@key{fams}{krl}{Uralic}
\define@key{fams}{bwe}{Sino-Tibetan}
\define@key{fams}{kjp}{Sino-Tibetan}
\define@key{fams}{ksw}{Sino-Tibetan}
\define@key{fams}{vka}{Pama-Nyungan}
\define@key{fams}{kdj}{Eastern Sudanic}
\define@key{fams}{ktn}{Tupian}
\define@key{fams}{yuj}{Pauwasi}
\define@key{fams}{kyh}{Hokan}
\define@key{fams}{arr}{Tupian}
\define@key{fams}{xsm}{Niger-Congo}
\define@key{fams}{kju}{Hokan}
\define@key{fams}{kas}{Indo-European}
\define@key{fams}{csb}{Indo-European}
\define@key{fams}{cog}{Austro-Asiatic}
\define@key{fams}{bqy}{other}
\define@key{fams}{xtc}{Kadu}
\define@key{fams}{bsh}{Indo-European}
\define@key{fams}{kts}{Trans-New Guinea}
\define@key{fams}{kcr}{Kordofanian}
\define@key{fams}{ktw}{Na-Dene}
\define@key{fams}{pss}{Austronesian}
\define@key{fams}{bpp}{Isolate}
\define@key{fams}{zku}{Pama-Nyungan}
\define@key{fams}{xaw}{Uto-Aztecan}
\define@key{fams}{kyz}{Tupian}
\define@key{fams}{eky}{Sino-Tibetan}
\define@key{fams}{kys}{Austronesian}
\define@key{fams}{txu}{Macro-Ge}
\define@key{fams}{gyd}{Tangkic}
\define@key{fams}{gbb}{Pama-Nyungan}
\define@key{fams}{kaz}{Altaic}
\define@key{fams}{ksx}{Austronesian}
\define@key{fams}{kbr}{Afro-Asiatic}
\define@key{fams}{kei}{Austronesian}
\define@key{fams}{kcl}{Austronesian}
\define@key{fams}{kzi}{Austronesian}
\define@key{fams}{sbc}{Austronesian}
\define@key{fams}{ahg}{Afro-Asiatic}
\define@key{fams}{kmt}{Nimboran}
\define@key{fams}{kyq}{Central Sudanic}
\define@key{fams}{keu}{Austronesian}
\define@key{fams}{xki}{other}
\define@key{fams}{ken}{Niger-Congo}
\define@key{fams}{xxk}{Austronesian}
\define@key{fams}{ker}{Afro-Asiatic}
\define@key{fams}{krk}{Chukotko-Kamchatkan}
\define@key{fams}{kee}{Keresan}
\define@key{fams}{ket}{Yeniseian}
\define@key{fams}{xdy}{Austronesian}
\define@key{fams}{kcv}{Niger-Congo}
\define@key{fams}{xte}{Trans-New Guinea}
\define@key{fams}{kew}{Trans-New Guinea}
\define@key{fams}{kjh}{Altaic}
\define@key{fams}{klj}{Altaic}
\define@key{fams}{klr}{Sino-Tibetan}
\define@key{fams}{khk}{Altaic}
\define@key{fams}{kjl}{Sino-Tibetan}
\define@key{fams}{khg}{Sino-Tibetan}
\define@key{fams}{kca}{Uralic}
\define@key{fams}{khr}{Austro-Asiatic}
\define@key{fams}{kha}{Austro-Asiatic}
\define@key{fams}{kjj}{Nakh-Daghestanian}
\define@key{fams}{khm}{Austro-Asiatic}
\define@key{fams}{kjg}{Austro-Asiatic}
\define@key{fams}{khw}{Indo-European}
\define@key{fams}{cnk}{Sino-Tibetan}
\define@key{fams}{khv}{Nakh-Daghestanian}
\define@key{fams}{kkh}{Tai-Kadai}
\define@key{fams}{kic}{Algic}
\define@key{fams}{kik}{Niger-Congo}
\define@key{fams}{hbb}{Afro-Asiatic}
\define@key{fams}{kij}{Austronesian}
\define@key{fams}{klb}{Hokan}
\define@key{fams}{lub}{Niger-Congo}
\define@key{fams}{kig}{Kolopom}
\define@key{fams}{zga}{Niger-Congo}
\define@key{fams}{kfk}{Sino-Tibetan}
\define@key{fams}{kin}{Niger-Congo}
\define@key{fams}{kio}{Kiowa-Tanoan}
\define@key{fams}{kzw}{Kariri}
\define@key{fams}{geb}{Lower Sepik-Ramu}
\define@key{fams}{kir}{Altaic}
\define@key{fams}{gil}{Austronesian}
\define@key{fams}{kiy}{Lakes Plain}
\define@key{fams}{cme}{Niger-Congo}
\define@key{fams}{kje}{Austronesian}
\define@key{fams}{kss}{Niger-Congo}
\define@key{fams}{gia}{Jarrakan}
\define@key{fams}{kii}{Caddoan}
\define@key{fams}{ktu}{other}
\define@key{fams}{kjd}{Trans-New Guinea}
\define@key{fams}{kla}{Penutian}
\define@key{fams}{klu}{Niger-Congo}
\define@key{fams}{yak}{Penutian}
\define@key{fams}{kst}{Niger-Congo}
\define@key{fams}{cku}{Muskogean}
\define@key{fams}{kpw}{Trans-New Guinea}
\define@key{fams}{kfa}{Dravidian}
\define@key{fams}{xwg}{Eastern Sudanic}
\define@key{fams}{xuo}{Niger-Congo}
\define@key{fams}{bcs}{Niger-Congo}
\define@key{fams}{kpx}{Trans-New Guinea}
\define@key{fams}{kbk}{Trans-New Guinea}
\define@key{fams}{kqi}{Trans-New Guinea}
\define@key{fams}{trp}{Sino-Tibetan}
\define@key{fams}{kex}{Indo-European}
\define@key{fams}{kkk}{Austronesian}
\define@key{fams}{kvv}{Austronesian}
\define@key{fams}{kfb}{Dravidian}
\define@key{fams}{kvw}{Greater West Bomberai}
\define@key{fams}{shm}{Indo-European}
\define@key{fams}{bkm}{Niger-Congo}
\define@key{fams}{xbi}{Torricelli}
\define@key{fams}{kge}{Austronesian}
\define@key{fams}{koi}{Uralic}
\define@key{fams}{xom}{Koman}
\define@key{fams}{kfc}{Dravidian}
\define@key{fams}{kng}{Niger-Congo}
\define@key{fams}{kjc}{Austronesian}
\define@key{fams}{knn}{Indo-European}
\define@key{fams}{xon}{Niger-Congo}
\define@key{fams}{mjd}{Penutian}
\define@key{fams}{kma}{Niger-Congo}
\define@key{fams}{kyx}{West Bougainville}
\define@key{fams}{cou}{Niger-Congo}
\define@key{fams}{kqy}{Afro-Asiatic}
\define@key{fams}{kpr}{Trans-New Guinea}
\define@key{fams}{kqz}{Khoe-Kwadi}
\define@key{fams}{knk}{Mande}
\define@key{fams}{kor}{Isolate}
\define@key{fams}{coe}{Tucanoan}
\define@key{fams}{kfq}{Austro-Asiatic}
\define@key{fams}{kfz}{Niger-Congo}
\define@key{fams}{khe}{Trans-New Guinea}
\define@key{fams}{kpy}{Chukotko-Kamchatkan}
\define@key{fams}{kia}{Niger-Congo}
\define@key{fams}{kos}{Austronesian}
\define@key{fams}{kfe}{Dravidian}
\define@key{fams}{aal}{Afro-Asiatic}
\define@key{fams}{kff}{Dravidian}
\define@key{fams}{khq}{Songhay}
\define@key{fams}{ses}{Songhay}
\define@key{fams}{koy}{Na-Dene}
\define@key{fams}{kpk}{Niger-Congo}
\define@key{fams}{xpe}{Mande}
\define@key{fams}{kpo}{Niger-Congo}
\define@key{fams}{xra}{Macro-Ge}
\define@key{fams}{kqq}{Macro-Ge}
\define@key{fams}{krs}{Central Sudanic}
\define@key{fams}{rop}{other}
\define@key{fams}{kgo}{Kadu}
\define@key{fams}{jct}{Altaic}
\define@key{fams}{kry}{Nakh-Daghestanian}
\define@key{fams}{puo}{Austro-Asiatic}
\define@key{fams}{sdm}{Austronesian}
\define@key{fams}{uwa}{Pama-Nyungan}
\define@key{fams}{kxu}{Dravidian}
\define@key{fams}{kvd}{Greater West Bomberai}
\define@key{fams}{kui}{Cariban}
\define@key{fams}{gvn}{Pama-Nyungan}
\define@key{fams}{mbt}{Austronesian}
\define@key{fams}{dwr}{Afro-Asiatic}
\define@key{fams}{kle}{Sino-Tibetan}
\define@key{fams}{kue}{Trans-New Guinea}
\define@key{fams}{kfy}{Indo-European}
\define@key{fams}{kum}{Altaic}
\define@key{fams}{kvn}{Chibchan}
\define@key{fams}{kun}{Isolate}
\define@key{fams}{kup}{Trans-New Guinea}
\define@key{fams}{kjn}{Pama-Nyungan}
\define@key{fams}{cmn}{Sino-Tibetan}
\define@key{fams}{kto}{Isolate}
\define@key{fams}{ckb}{Indo-European}
\define@key{fams}{kmr}{Indo-European}
\define@key{fams}{kru}{Dravidian}
\define@key{fams}{kgg}{Isolate}
\define@key{fams}{vkt}{Austronesian}
\define@key{fams}{gwi}{Na-Dene}
\define@key{fams}{kut}{Isolate}
\define@key{fams}{thd}{Pama-Nyungan}
\define@key{fams}{kuy}{Pama-Nyungan}
\define@key{fams}{kxv}{Dravidian}
\define@key{fams}{kwd}{Austronesian}
\define@key{fams}{kwk}{Wakashan}
\define@key{fams}{tnk}{Austronesian}
\define@key{fams}{ksq}{Afro-Asiatic}
\define@key{fams}{kwn}{Niger-Congo}
\define@key{fams}{xwa}{Isolate}
\define@key{fams}{kwe}{Tor-Kwerba}
\define@key{fams}{kmo}{Sepik}
\define@key{fams}{kwo}{Isolate}
\define@key{fams}{xuu}{Khoe-Kwadi}
\define@key{fams}{kyc}{Trans-New Guinea}
\define@key{fams}{kgy}{Sino-Tibetan}
\define@key{fams}{nuk}{Wakashan}
\define@key{fams}{kmg}{Trans-New Guinea}
\define@key{fams}{gdm}{Isolate}
\define@key{fams}{lbu}{Austronesian}
\define@key{fams}{lac}{Mayan}
\define@key{fams}{lbt}{Tai-Kadai}
\define@key{fams}{lbj}{Sino-Tibetan}
\define@key{fams}{lld}{Indo-European}
\define@key{fams}{lad}{Indo-European}
\define@key{fams}{laf}{Kordofanian}
\define@key{fams}{kot}{Afro-Asiatic}
\define@key{fams}{lha}{Tai-Kadai}
\define@key{fams}{lhu}{Sino-Tibetan}
\define@key{fams}{cnh}{Sino-Tibetan}
\define@key{fams}{lbe}{Nakh-Daghestanian}
\define@key{fams}{lkt}{Siouan}
\define@key{fams}{lbc}{Tai-Kadai}
\define@key{fams}{ywt}{Sino-Tibetan}
\define@key{fams}{slp}{Austronesian}
\define@key{fams}{hia}{Afro-Asiatic}
\define@key{fams}{lmn}{Indo-European}
\define@key{fams}{lam}{Niger-Congo}
\define@key{fams}{lmu}{Austronesian}
\define@key{fams}{lns}{Niger-Congo}
\define@key{fams}{ljp}{Austronesian}
\define@key{fams}{lby}{Pama-Nyungan}
\define@key{fams}{lme}{Afro-Asiatic}
\define@key{fams}{lag}{Niger-Congo}
\define@key{fams}{laj}{Eastern Sudanic}
\define@key{fams}{fsl}{other}
\define@key{fams}{fcs}{other}
\define@key{fams}{lao}{Tai-Kadai}
\define@key{fams}{lrg}{Darwin Region}
\define@key{fams}{lbz}{Tangkic}
\define@key{fams}{alo}{Austronesian}
\define@key{fams}{lav}{Indo-European}
\define@key{fams}{llu}{Austronesian}
\define@key{fams}{law}{Austronesian}
\define@key{fams}{lvk}{Solomons East Papuan}
\define@key{fams}{lzz}{Kartvelian}
\define@key{fams}{agh}{Niger-Congo}
\define@key{fams}{lea}{Niger-Congo}
\define@key{fams}{agb}{Niger-Congo}
\define@key{fams}{lec}{Isolate}
\define@key{fams}{lln}{Afro-Asiatic}
\define@key{fams}{lef}{Niger-Congo}
\define@key{fams}{tnl}{Austronesian}
\define@key{fams}{led}{Central Sudanic}
\define@key{fams}{enx}{Mascoian}
\define@key{fams}{aed}{other}
\define@key{fams}{ssp}{other}
\define@key{fams}{lep}{Sino-Tibetan}
\define@key{fams}{les}{Central Sudanic}
\define@key{fams}{lti}{Austronesian}
\define@key{fams}{lww}{Austronesian}
\define@key{fams}{lez}{Nakh-Daghestanian}
\define@key{fams}{lhm}{Sino-Tibetan}
\define@key{fams}{lil}{Salishan}
\define@key{fams}{lif}{Sino-Tibetan}
\define@key{fams}{lmc}{Darwin Region}
\define@key{fams}{liy}{Niger-Congo}
\define@key{fams}{lin}{Niger-Congo}
\define@key{fams}{ise}{other}
\define@key{fams}{lnj}{Pama-Nyungan}
\define@key{fams}{lis}{Sino-Tibetan}
\define@key{fams}{lit}{Indo-European}
\define@key{fams}{liv}{Uralic}
\define@key{fams}{lob}{Niger-Congo}
\define@key{fams}{log}{Central Sudanic}
\define@key{fams}{lok}{Mande}
\define@key{fams}{arw}{Arawakan}
\define@key{fams}{lom}{Mande}
\define@key{fams}{bdu}{Niger-Congo}
\define@key{fams}{lgu}{Austronesian}
\define@key{fams}{los}{Austronesian}
\define@key{fams}{crc}{Austronesian}
\define@key{fams}{njh}{Sino-Tibetan}
\define@key{fams}{loj}{Austronesian}
\define@key{fams}{lbo}{Austro-Asiatic}
\define@key{fams}{nds}{Indo-European}
\define@key{fams}{loz}{Niger-Congo}
\define@key{fams}{nie}{Niger-Congo}
\define@key{fams}{ojv}{Austronesian}
\define@key{fams}{lch}{Niger-Congo}
\define@key{fams}{lug}{Niger-Congo}
\define@key{fams}{lgg}{Central Sudanic}
\define@key{fams}{jos}{other}
\define@key{fams}{lui}{Uto-Aztecan}
\define@key{fams}{ule}{Isolate}
\define@key{fams}{str}{Salishan}
\define@key{fams}{lnd}{Austronesian}
\define@key{fams}{lun}{Niger-Congo}
\define@key{fams}{luo}{Eastern Sudanic}
\define@key{fams}{lrc}{Indo-European}
\define@key{fams}{lut}{Salishan}
\define@key{fams}{khl}{Austronesian}
\define@key{fams}{lue}{Niger-Congo}
\define@key{fams}{lwo}{Eastern Sudanic}
\define@key{fams}{ltz}{Indo-European}
\define@key{fams}{luy}{Niger-Congo}
\define@key{fams}{lee}{Niger-Congo}
\define@key{fams}{psr}{other}
\define@key{fams}{bzs}{other}
\define@key{fams}{khb}{Tai-Kadai}
\define@key{fams}{msj}{Niger-Congo}
\define@key{fams}{mhy}{Austronesian}
\define@key{fams}{mhi}{Central Sudanic}
\define@key{fams}{slz}{Austronesian}
\define@key{fams}{mdy}{Afro-Asiatic}
\define@key{fams}{mas}{Eastern Sudanic}
\define@key{fams}{mde}{Maban}
\define@key{fams}{mca}{Matacoan}
\define@key{fams}{mbn}{Guahiban}
\define@key{fams}{mkd}{Indo-European}
\define@key{fams}{mcb}{Arawakan}
\define@key{fams}{myy}{Tucanoan}
\define@key{fams}{mbc}{Cariban}
\define@key{fams}{mxu}{Afro-Asiatic}
\define@key{fams}{mda}{Niger-Congo}
\define@key{fams}{dmd}{Pama-Nyungan}
\define@key{fams}{mad}{Austronesian}
\define@key{fams}{mmw}{Austronesian}
\define@key{fams}{mag}{Indo-European}
\define@key{fams}{mgp}{Sino-Tibetan}
\define@key{fams}{mrd}{Sino-Tibetan}
\define@key{fams}{mgu}{Trans-New Guinea}
\define@key{fams}{mdh}{Austronesian}
\define@key{fams}{mhe}{Austro-Asiatic}
\define@key{fams}{xpq}{Algic}
\define@key{fams}{nmu}{Penutian}
\define@key{fams}{zrs}{Mairasic}
\define@key{fams}{mbq}{Austronesian}
\define@key{fams}{mai}{Indo-European}
\define@key{fams}{mpe}{Eastern Sudanic}
\define@key{fams}{mcp}{Niger-Congo}
\define@key{fams}{myh}{Wakashan}
\define@key{fams}{mkz}{Greater West Bomberai}
\define@key{fams}{mak}{Austronesian}
\define@key{fams}{mgf}{Bulaka River}
\define@key{fams}{kde}{Niger-Congo}
\define@key{fams}{mgh}{Niger-Congo}
\define@key{fams}{mcm}{other}
\define@key{fams}{plt}{Austronesian}
\define@key{fams}{mpb}{Northern Daly}
\define@key{fams}{zsm}{Austronesian}
\define@key{fams}{zlm}{Austronesian}
\define@key{fams}{zmi}{Austronesian}
\define@key{fams}{mal}{Dravidian}
\define@key{fams}{mgl}{Austronesian}
\define@key{fams}{gcc}{Baining}
\define@key{fams}{mlt}{Afro-Asiatic}
\define@key{fams}{kmj}{Dravidian}
\define@key{fams}{mam}{Mayan}
\define@key{fams}{mmn}{Austronesian}
\define@key{fams}{mqj}{Austronesian}
\define@key{fams}{mcs}{Niger-Congo}
\define@key{fams}{mgr}{Niger-Congo}
\define@key{fams}{maw}{Niger-Congo}
\define@key{fams}{mdi}{Central Sudanic}
\define@key{fams}{xmm}{Austronesian}
\define@key{fams}{mva}{Austronesian}
\define@key{fams}{mle}{Sepik}
\define@key{fams}{nmm}{Sino-Tibetan}
\define@key{fams}{mnc}{Altaic}
\define@key{fams}{mid}{Afro-Asiatic}
\define@key{fams}{mhq}{Siouan}
\define@key{fams}{mdr}{Austronesian}
\define@key{fams}{mnk}{Mande}
\define@key{fams}{jet}{Border}
\define@key{fams}{mna}{Austronesian}
\define@key{fams}{mpc}{Mangarrayi-Maran}
\define@key{fams}{mdj}{Central Sudanic}
\define@key{fams}{mqy}{Austronesian}
\define@key{fams}{mjg}{Altaic}
\define@key{fams}{mge}{Central Sudanic}
\define@key{fams}{emk}{Mande}
\define@key{fams}{mlq}{Mande}
\define@key{fams}{mfv}{Niger-Congo}
\define@key{fams}{knf}{Niger-Congo}
\define@key{fams}{nge}{Niger-Congo}
\define@key{fams}{mev}{Mande}
\define@key{fams}{mbb}{Austronesian}
\define@key{fams}{mns}{Uralic}
\define@key{fams}{glv}{Indo-European}
\define@key{fams}{mri}{Austronesian}
\define@key{fams}{mcg}{Cariban}
\define@key{fams}{arn}{Araucanian}
\define@key{fams}{mec}{Mangarrayi-Maran}
\define@key{fams}{mrw}{Austronesian}
\define@key{fams}{zmr}{Western Daly}
\define@key{fams}{mar}{Indo-European}
\define@key{fams}{rnp}{Sino-Tibetan}
\define@key{fams}{zmc}{Pama-Nyungan}
\define@key{fams}{mrt}{Afro-Asiatic}
\define@key{fams}{mrj}{Uralic}
\define@key{fams}{mhr}{Uralic}
\define@key{fams}{mrc}{Hokan}
\define@key{fams}{mrz}{Trans-New Guinea}
\define@key{fams}{mbw}{Trans-New Guinea}
\define@key{fams}{zmt}{Western Daly}
\define@key{fams}{mfr}{Western Daly}
\define@key{fams}{mah}{Austronesian}
\define@key{fams}{gcf}{other}
\define@key{fams}{vma}{Pama-Nyungan}
\define@key{fams}{mhx}{Sino-Tibetan}
\define@key{fams}{mcn}{Afro-Asiatic}
\define@key{fams}{jle}{Kordofanian}
\define@key{fams}{mls}{Maban}
\define@key{fams}{wam}{Algic}
\define@key{fams}{mpq}{Pano-Tacanan}
\define@key{fams}{zml}{Eastern Daly}
\define@key{fams}{mcf}{Pano-Tacanan}
\define@key{fams}{mvb}{Na-Dene}
\define@key{fams}{mjk}{Austronesian}
\define@key{fams}{mgw}{Niger-Congo}
\define@key{fams}{mxx}{Mande}
\define@key{fams}{mph}{Iwaidjan}
\define@key{fams}{mfe}{other}
\define@key{fams}{mke}{Indo-European}
\define@key{fams}{mbl}{Macro-Ge}
\define@key{fams}{yan}{Misumalpan}
\define@key{fams}{ayz}{Isolate}
\define@key{fams}{xyj}{Pama-Nyungan}
\define@key{fams}{mfy}{Uto-Aztecan}
\define@key{fams}{mdm}{Niger-Congo}
\define@key{fams}{maz}{Oto-Manguean}
\define@key{fams}{mzn}{Indo-European}
\define@key{fams}{maq}{Oto-Manguean}
\define@key{fams}{mau}{Oto-Manguean}
\define@key{fams}{mfc}{Niger-Congo}
\define@key{fams}{vmb}{Pama-Nyungan}
\define@key{fams}{lnb}{Niger-Congo}
\define@key{fams}{mpk}{Afro-Asiatic}
\define@key{fams}{myb}{Central Sudanic}
\define@key{fams}{mtk}{Niger-Congo}
\define@key{fams}{mdt}{Niger-Congo}
\define@key{fams}{baw}{Niger-Congo}
\define@key{fams}{gmm}{Niger-Congo}
\define@key{fams}{mdq}{Niger-Congo}
\define@key{fams}{mdw}{Niger-Congo}
\define@key{fams}{mhd}{Afro-Asiatic}
\define@key{fams}{mdd}{Niger-Congo}
\define@key{fams}{mym}{Eastern Sudanic}
\define@key{fams}{nux}{Sepik}
\define@key{fams}{gdq}{Afro-Asiatic}
\define@key{fams}{mni}{Sino-Tibetan}
\define@key{fams}{skf}{Tupian}
\define@key{fams}{mek}{Austronesian}
\define@key{fams}{mel}{Austronesian}
\define@key{fams}{bew}{other}
\define@key{fams}{men}{Mande}
\define@key{fams}{mez}{Algic}
\define@key{fams}{mwv}{Austronesian}
\define@key{fams}{sdo}{Austronesian}
\define@key{fams}{mcr}{Trans-New Guinea}
\define@key{fams}{ulk}{Eastern Trans-Fly}
\define@key{fams}{mej}{East Bird's Head}
\define@key{fams}{mpt}{Trans-New Guinea}
\define@key{fams}{crg}{Algic}
\define@key{fams}{mic}{Algic}
\define@key{fams}{mei}{Eastern Sudanic}
\define@key{fams}{ium}{Hmong-Mien}
\define@key{fams}{mmy}{Afro-Asiatic}
\define@key{fams}{mxj}{Sino-Tibetan}
\define@key{fams}{msy}{Lower Sepik-Ramu}
\define@key{fams}{mik}{Muskogean}
\define@key{fams}{mjw}{Sino-Tibetan}
\define@key{fams}{hna}{Afro-Asiatic}
\define@key{fams}{min}{Austronesian}
\define@key{fams}{mvn}{Austronesian}
\define@key{fams}{xmf}{Kartvelian}
\define@key{fams}{mep}{Jarrakan}
\define@key{fams}{nju}{Pama-Nyungan}
\define@key{fams}{mrg}{Sino-Tibetan}
\define@key{fams}{miq}{Misumalpan}
\define@key{fams}{zmq}{Niger-Congo}
\define@key{fams}{csi}{Penutian}
\define@key{fams}{csm}{Penutian}
\define@key{fams}{lmw}{Penutian}
\define@key{fams}{nsq}{Penutian}
\define@key{fams}{pmw}{Penutian}
\define@key{fams}{skd}{Penutian}
\define@key{fams}{mxp}{Mixe-Zoque}
\define@key{fams}{mco}{Mixe-Zoque}
\define@key{fams}{mto}{Mixe-Zoque}
\define@key{fams}{mim}{Oto-Manguean}
\define@key{fams}{mib}{Oto-Manguean}
\define@key{fams}{miy}{Oto-Manguean}
\define@key{fams}{mih}{Oto-Manguean}
\define@key{fams}{miz}{Oto-Manguean}
\define@key{fams}{mxt}{Oto-Manguean}
\define@key{fams}{mio}{Oto-Manguean}
\define@key{fams}{mig}{Oto-Manguean}
\define@key{fams}{mie}{Oto-Manguean}
\define@key{fams}{mil}{Oto-Manguean}
\define@key{fams}{mjc}{Oto-Manguean}
\define@key{fams}{mks}{Oto-Manguean}
\define@key{fams}{mpm}{Oto-Manguean}
\define@key{fams}{mkf}{Afro-Asiatic}
\define@key{fams}{lus}{Sino-Tibetan}
\define@key{fams}{mra}{Austro-Asiatic}
\define@key{fams}{moy}{Afro-Asiatic}
\define@key{fams}{omc}{Isolate}
\define@key{fams}{moc}{Guaicuruan}
\define@key{fams}{mif}{Afro-Asiatic}
\define@key{fams}{mhj}{Altaic}
\define@key{fams}{moh}{Iroquoian}
\define@key{fams}{mov}{Hokan}
\define@key{fams}{mkj}{Austronesian}
\define@key{fams}{moz}{Afro-Asiatic}
\define@key{fams}{mbe}{Penutian}
\define@key{fams}{mso}{Isolate}
\define@key{fams}{fqs}{Baibai-Fas}
\define@key{fams}{mqf}{Trans-New Guinea}
\define@key{fams}{mnw}{Austro-Asiatic}
\define@key{fams}{ndt}{Niger-Congo}
\define@key{fams}{lol}{Niger-Congo}
\define@key{fams}{mog}{Austronesian}
\define@key{fams}{mnz}{Trans-New Guinea}
\define@key{fams}{mnr}{Uto-Aztecan}
\define@key{fams}{mte}{Austronesian}
\define@key{fams}{moe}{Algic}
\define@key{fams}{mxk}{Bogia}
\define@key{fams}{mos}{Niger-Congo}
\define@key{fams}{mop}{Mayan}
\define@key{fams}{mhz}{Austronesian}
\define@key{fams}{mok}{Isolate}
\define@key{fams}{myv}{Uralic}
\define@key{fams}{mdf}{Uralic}
\define@key{fams}{mor}{Kordofanian}
\define@key{fams}{mgd}{Central Sudanic}
\define@key{fams}{cas}{Mosetenan}
\define@key{fams}{meu}{Austronesian}
\define@key{fams}{siw}{South Bougainville}
\define@key{fams}{mzp}{Isolate}
\define@key{fams}{mye}{Niger-Congo}
\define@key{fams}{akc}{Isolate}
\define@key{fams}{dmw}{Pama-Nyungan}
\define@key{fams}{aoj}{Torricelli}
\define@key{fams}{sgw}{Afro-Asiatic}
\define@key{fams}{bmr}{Boran}
\define@key{fams}{chb}{Chibchan}
\define@key{fams}{mlm}{Tai-Kadai}
\define@key{fams}{mzm}{Niger-Congo}
\define@key{fams}{mji}{Hmong-Mien}
\define@key{fams}{mnb}{Austronesian}
\define@key{fams}{mua}{Niger-Congo}
\define@key{fams}{mnf}{Niger-Congo}
\define@key{fams}{myu}{Tupian}
\define@key{fams}{mhk}{Niger-Congo}
\define@key{fams}{umu}{Algic}
\define@key{fams}{moj}{Niger-Congo}
\define@key{fams}{mtq}{Austro-Asiatic}
\define@key{fams}{sur}{Afro-Asiatic}
\define@key{fams}{mtf}{Lower Sepik-Ramu}
\define@key{fams}{mur}{Eastern Sudanic}
\define@key{fams}{mwf}{Southern Daly}
\define@key{fams}{muz}{Eastern Sudanic}
\define@key{fams}{zmu}{Pama-Nyungan}
\define@key{fams}{mug}{Afro-Asiatic}
\define@key{fams}{msu}{Austronesian}
\define@key{fams}{hur}{Salishan}
\define@key{fams}{emi}{Austronesian}
\define@key{fams}{css}{Penutian}
\define@key{fams}{myw}{Austronesian}
\define@key{fams}{mwe}{Niger-Congo}
\define@key{fams}{mlv}{Austronesian}
\define@key{fams}{xak}{Isolate}
\define@key{fams}{bzk}{other}
\define@key{fams}{muh}{Niger-Congo}
\define@key{fams}{naf}{Trans-New Guinea}
\define@key{fams}{wyy}{Austronesian}
\define@key{fams}{mbj}{Nadahup}
\define@key{fams}{nfr}{Niger-Congo}
\define@key{fams}{nbi}{Sino-Tibetan}
\define@key{fams}{nmf}{Sino-Tibetan}
\define@key{fams}{nzm}{Sino-Tibetan}
\define@key{fams}{nag}{other}
\define@key{fams}{nce}{Yale}
\define@key{fams}{nll}{Isolate}
\define@key{fams}{nhn}{Uto-Aztecan}
\define@key{fams}{ncj}{Uto-Aztecan}
\define@key{fams}{nhx}{Uto-Aztecan}
\define@key{fams}{ncl}{Uto-Aztecan}
\define@key{fams}{nhm}{Uto-Aztecan}
\define@key{fams}{nhp}{Uto-Aztecan}
\define@key{fams}{xpo}{Uto-Aztecan}
\define@key{fams}{azz}{Uto-Aztecan}
\define@key{fams}{nhg}{Uto-Aztecan}
\define@key{fams}{ngu}{Uto-Aztecan}
\define@key{fams}{bio}{Kwomtari}
\define@key{fams}{nak}{Austronesian}
\define@key{fams}{nck}{Mangrida}
\define@key{fams}{nal}{Austronesian}
\define@key{fams}{naq}{Khoe-Kwadi}
\define@key{fams}{nmb}{Austronesian}
\define@key{fams}{nab}{Nambikuaran}
\define@key{fams}{nnm}{Sepik}
\define@key{fams}{gld}{Altaic}
\define@key{fams}{ncb}{Austro-Asiatic}
\define@key{fams}{nnb}{Niger-Congo}
\define@key{fams}{niq}{Eastern Sudanic}
\define@key{fams}{sen}{Niger-Congo}
\define@key{fams}{nnk}{Trans-New Guinea}
\define@key{fams}{nnt}{Algic}
\define@key{fams}{tvl}{Austronesian}
\define@key{fams}{npy}{Austronesian}
\define@key{fams}{npa}{Sino-Tibetan}
\define@key{fams}{nrb}{Eastern Sudanic}
\define@key{fams}{nrm}{Austronesian}
\define@key{fams}{nas}{South Bougainville}
\define@key{fams}{nsk}{Algic}
\define@key{fams}{ncz}{Isolate}
\define@key{fams}{ntm}{Niger-Congo}
\define@key{fams}{ntu}{Austronesian}
\define@key{fams}{nau}{Austronesian}
\define@key{fams}{nav}{Na-Dene}
\define@key{fams}{nxq}{Sino-Tibetan}
\define@key{fams}{bud}{Niger-Congo}
\define@key{fams}{nde}{Niger-Congo}
\define@key{fams}{djj}{Mangrida}
\define@key{fams}{ndz}{Niger-Congo}
\define@key{fams}{ndo}{Niger-Congo}
\define@key{fams}{nmd}{Niger-Congo}
\define@key{fams}{ndv}{Niger-Congo}
\define@key{fams}{djk}{other}
\define@key{fams}{dse}{other}
\define@key{fams}{neg}{Altaic}
\define@key{fams}{nsn}{Austronesian}
\define@key{fams}{nee}{Austronesian}
\define@key{fams}{anh}{Trans-New Guinea}
\define@key{fams}{yrk}{Uralic}
\define@key{fams}{nen}{Austronesian}
\define@key{fams}{aij}{Afro-Asiatic}
\define@key{fams}{aii}{Afro-Asiatic}
\define@key{fams}{trg}{Afro-Asiatic}
\define@key{fams}{npi}{Indo-European}
\define@key{fams}{pia}{Uto-Aztecan}
\define@key{fams}{nzs}{other}
\define@key{fams}{new}{Sino-Tibetan}
\define@key{fams}{ney}{Niger-Congo}
\define@key{fams}{nez}{Penutian}
\define@key{fams}{ntj}{Pama-Nyungan}
\define@key{fams}{nxg}{Austronesian}
\define@key{fams}{nig}{Gunwinyguan}
\define@key{fams}{ngk}{Gunwinyguan}
\define@key{fams}{sba}{Central Sudanic}
\define@key{fams}{nam}{Southern Daly}
\define@key{fams}{nio}{Uralic}
\define@key{fams}{nid}{Gunwinyguan}
\define@key{fams}{nay}{Pama-Nyungan}
\define@key{fams}{nrk}{Pama-Nyungan}
\define@key{fams}{nrl}{Pama-Nyungan}
\define@key{fams}{nxn}{Pama-Nyungan}
\define@key{fams}{nbm}{Niger-Congo}
\define@key{fams}{nga}{Niger-Congo}
\define@key{fams}{ngb}{Niger-Congo}
\define@key{fams}{niy}{Central Sudanic}
\define@key{fams}{wyb}{Pama-Nyungan}
\define@key{fams}{ngi}{Afro-Asiatic}
\define@key{fams}{ngo}{Niger-Congo}
\define@key{fams}{llp}{Austronesian}
\define@key{fams}{gym}{Chibchan}
\define@key{fams}{nha}{Pama-Nyungan}
\define@key{fams}{nhr}{Khoe-Kwadi}
\define@key{fams}{nia}{Austronesian}
\define@key{fams}{caq}{Austro-Asiatic}
\define@key{fams}{pcm}{other}
\define@key{fams}{jsl}{other}
\define@key{fams}{nir}{Isolate}
\define@key{fams}{niz}{Torricelli}
\define@key{fams}{nsz}{Penutian}
\define@key{fams}{ncg}{Tsimshianic}
\define@key{fams}{dtd}{Wakashan}
\define@key{fams}{num}{Austronesian}
\define@key{fams}{niu}{Austronesian}
\define@key{fams}{cag}{Matacoan}
\define@key{fams}{niv}{Isolate}
\define@key{fams}{isi}{Niger-Congo}
\define@key{fams}{nko}{Niger-Congo}
\define@key{fams}{cgg}{Niger-Congo}
\define@key{fams}{fia}{Eastern Sudanic}
\define@key{fams}{njb}{Sino-Tibetan}
\define@key{fams}{nog}{Altaic}
\define@key{fams}{not}{Arawakan}
\define@key{fams}{nhu}{Niger-Congo}
\define@key{fams}{snf}{Niger-Congo}
\define@key{fams}{nsl}{other}
\define@key{fams}{nor}{Indo-European}
\define@key{fams}{nse}{Niger-Congo}
\define@key{fams}{nto}{Niger-Congo}
\define@key{fams}{nxl}{Austronesian}
\define@key{fams}{kcn}{other}
\define@key{fams}{dgl}{Eastern Sudanic}
\define@key{fams}{xnz}{Eastern Sudanic}
\define@key{fams}{nus}{Eastern Sudanic}
\define@key{fams}{mbr}{Cacua-Nukak}
\define@key{fams}{nkr}{Austronesian}
\define@key{fams}{nut}{Tai-Kadai}
\define@key{fams}{nuy}{Gunwinyguan}
\define@key{fams}{nuv}{Niger-Congo}
\define@key{fams}{iii}{Sino-Tibetan}
\define@key{fams}{nup}{Niger-Congo}
\define@key{fams}{nuf}{Sino-Tibetan}
\define@key{fams}{cbn}{Austro-Asiatic}
\define@key{fams}{nly}{Pama-Nyungan}
\define@key{fams}{now}{Niger-Congo}
\define@key{fams}{tpq}{Sino-Tibetan}
\define@key{fams}{nym}{Niger-Congo}
\define@key{fams}{nyj}{Niger-Congo}
\define@key{fams}{nyp}{Eastern Sudanic}
\define@key{fams}{nna}{Pama-Nyungan}
\define@key{fams}{nyt}{Pama-Nyungan}
\define@key{fams}{yly}{Austronesian}
\define@key{fams}{nyh}{Nyulnyulan}
\define@key{fams}{nih}{Niger-Congo}
\define@key{fams}{nyi}{Eastern Sudanic}
\define@key{fams}{njz}{Sino-Tibetan}
\define@key{fams}{nyv}{Nyulnyulan}
\define@key{fams}{nys}{Pama-Nyungan}
\define@key{fams}{nzk}{Niger-Congo}
\define@key{fams}{ood}{Uto-Aztecan}
\define@key{fams}{afz}{Lakes Plain}
\define@key{fams}{ann}{Niger-Congo}
\define@key{fams}{oca}{Witotoan}
\define@key{fams}{oci}{Indo-European}
\define@key{fams}{ocu}{Oto-Manguean}
\define@key{fams}{ogb}{Niger-Congo}
\define@key{fams}{ogu}{Niger-Congo}
\define@key{fams}{oyb}{Austro-Asiatic}
\define@key{fams}{xal}{Altaic}
\define@key{fams}{ojs}{Algic}
\define@key{fams}{ciw}{Algic}
\define@key{fams}{oka}{Salishan}
\define@key{fams}{opm}{Trans-New Guinea}
\define@key{fams}{oku}{Niger-Congo}
\define@key{fams}{ong}{Torricelli}
\define@key{fams}{plo}{Mixe-Zoque}
\define@key{fams}{omg}{Tupian}
\define@key{fams}{oma}{Siouan}
\define@key{fams}{aun}{Torricelli}
\define@key{fams}{one}{Iroquoian}
\define@key{fams}{oon}{South Andamanese}
\define@key{fams}{ons}{Trans-New Guinea}
\define@key{fams}{ono}{Iroquoian}
\define@key{fams}{mvf}{Altaic}
\define@key{fams}{ore}{Tucanoan}
\define@key{fams}{tag}{Kordofanian}
\define@key{fams}{ory}{Indo-European}
\define@key{fams}{ort}{Indo-European}
\define@key{fams}{oru}{Indo-European}
\define@key{fams}{oac}{Altaic}
\define@key{fams}{oaa}{Altaic}
\define@key{fams}{okv}{Trans-New Guinea}
\define@key{fams}{oro}{Eleman}
\define@key{fams}{gax}{Afro-Asiatic}
\define@key{fams}{hae}{Afro-Asiatic}
\define@key{fams}{ssn}{Afro-Asiatic}
\define@key{fams}{gaz}{Afro-Asiatic}
\define@key{fams}{ury}{Tor-Kwerba}
\define@key{fams}{osa}{Siouan}
\define@key{fams}{oss}{Indo-European}
\define@key{fams}{iow}{Siouan}
\define@key{fams}{otz}{Oto-Manguean}
\define@key{fams}{ote}{Oto-Manguean}
\define@key{fams}{otq}{Oto-Manguean}
\define@key{fams}{otm}{Oto-Manguean}
\define@key{fams}{otr}{Kordofanian}
\define@key{fams}{owi}{Left May}
\define@key{fams}{pqa}{Afro-Asiatic}
\define@key{fams}{drl}{Pama-Nyungan}
\define@key{fams}{pma}{Austronesian}
\define@key{fams}{pac}{Austro-Asiatic}
\define@key{fams}{pdo}{Austronesian}
\define@key{fams}{pgu}{North Halmaheran}
\define@key{fams}{duf}{Austronesian}
\define@key{fams}{pck}{Sino-Tibetan}
\define@key{fams}{pao}{Uto-Aztecan}
\define@key{fams}{pwn}{Austronesian}
\define@key{fams}{pkn}{Pama-Nyungan}
\define@key{fams}{pau}{Austronesian}
\define@key{fams}{pll}{Austro-Asiatic}
\define@key{fams}{plu}{Arawakan}
\define@key{fams}{fap}{Niger-Congo}
\define@key{fams}{nad}{Pama-Nyungan}
\define@key{fams}{pmz}{Oto-Manguean}
\define@key{fams}{pmf}{Austronesian}
\define@key{fams}{pbh}{Cariban}
\define@key{fams}{kre}{Macro-Ge}
\define@key{fams}{pag}{Austronesian}
\define@key{fams}{pbr}{Niger-Congo}
\define@key{fams}{pan}{Indo-European}
\define@key{fams}{pnw}{Pama-Nyungan}
\define@key{fams}{pap}{other}
\define@key{fams}{prk}{Austro-Asiatic}
\define@key{fams}{asa}{Niger-Congo}
\define@key{fams}{pab}{Arawakan}
\define@key{fams}{pci}{Dravidian}
\define@key{fams}{pst}{Indo-European}
\define@key{fams}{pqm}{Algic}
\define@key{fams}{ptp}{Austronesian}
\define@key{fams}{gfk}{Austronesian}
\define@key{fams}{lae}{Sino-Tibetan}
\define@key{fams}{pwi}{Penutian}
\define@key{fams}{plh}{Austronesian}
\define@key{fams}{pad}{Arauan}
\define@key{fams}{pwa}{Teberan-Pawaian}
\define@key{fams}{paw}{Caddoan}
\define@key{fams}{pay}{Chibchan}
\define@key{fams}{aoc}{Cariban}
\define@key{fams}{peg}{Dravidian}
\define@key{fams}{pip}{Afro-Asiatic}
\define@key{fams}{pes}{Indo-European}
\define@key{fams}{pww}{Sino-Tibetan}
\define@key{fams}{pio}{Arawakan}
\define@key{fams}{pid}{Sáliban}
\define@key{fams}{plg}{Guaicuruan}
\define@key{fams}{piv}{Austronesian}
\define@key{fams}{pif}{Austronesian}
\define@key{fams}{piu}{Pama-Nyungan}
\define@key{fams}{ppl}{Uto-Aztecan}
\define@key{fams}{myp}{Mura}
\define@key{fams}{pir}{Tucanoan}
\define@key{fams}{pib}{Arawakan}
\define@key{fams}{psa}{Trans-New Guinea}
\define@key{fams}{pjt}{Pama-Nyungan}
\define@key{fams}{pit}{Pama-Nyungan}
\define@key{fams}{psd}{other}
\define@key{fams}{gob}{Guahiban}
\define@key{fams}{fwa}{Austronesian}
\define@key{fams}{pbi}{Afro-Asiatic}
\define@key{fams}{poy}{Niger-Congo}
\define@key{fams}{pon}{Austronesian}
\define@key{fams}{rwa}{Skou}
\define@key{fams}{poh}{Mayan}
\define@key{fams}{pko}{Eastern Sudanic}
\define@key{fams}{pox}{Indo-European}
\define@key{fams}{pol}{Indo-European}
\define@key{fams}{poo}{Hokan}
\define@key{fams}{peb}{Hokan}
\define@key{fams}{pej}{Hokan}
\define@key{fams}{pom}{Hokan}
\define@key{fams}{pbe}{Oto-Manguean}
\define@key{fams}{poe}{Oto-Manguean}
\define@key{fams}{pbf}{Oto-Manguean}
\define@key{fams}{poi}{Mixe-Zoque}
\define@key{fams}{poc}{Mayan}
\define@key{fams}{psw}{Austronesian}
\define@key{fams}{por}{Indo-European}
\define@key{fams}{pot}{Algic}
\define@key{fams}{pim}{Algic}
\define@key{fams}{prn}{Indo-European}
\define@key{fams}{pre}{other}
\define@key{fams}{pui}{Isolate}
\define@key{fams}{fuc}{Niger-Congo}
\define@key{fams}{nij}{Austronesian}
\define@key{fams}{puw}{Austronesian}
\define@key{fams}{pmi}{Sino-Tibetan}
\define@key{fams}{puq}{Isolate}
\define@key{fams}{prx}{Sino-Tibetan}
\define@key{fams}{tsz}{Tarascan}
\define@key{fams}{pbb}{Páezan}
\define@key{fams}{lkr}{Eastern Sudanic}
\define@key{fams}{aar}{Afro-Asiatic}
\define@key{fams}{byx}{Baining}
\define@key{fams}{alc}{Alacalufan}
\define@key{fams}{yum}{Hokan}
\define@key{fams}{qxa}{Quechuan}
\define@key{fams}{quy}{Quechuan}
\define@key{fams}{qvc}{Quechuan}
\define@key{fams}{quh}{Quechuan}
\define@key{fams}{quz}{Quechuan}
\define@key{fams}{qug}{Quechuan}
\define@key{fams}{qub}{Quechuan}
\define@key{fams}{qvi}{Quechuan}
\define@key{fams}{qvn}{Quechuan}
\define@key{fams}{quc}{Mayan}
\define@key{fams}{qui}{Chimakuan}
\define@key{fams}{rad}{Austronesian}
\define@key{fams}{lml}{Austronesian}
\define@key{fams}{rji}{Sino-Tibetan}
\define@key{fams}{ral}{Sino-Tibetan}
\define@key{fams}{rma}{Chibchan}
\define@key{fams}{bod}{Sino-Tibetan}
\define@key{fams}{rao}{Lower Sepik-Ramu}
\define@key{fams}{rap}{Austronesian}
\define@key{fams}{ras}{Kordofanian}
\define@key{fams}{rwo}{Trans-New Guinea}
\define@key{fams}{raw}{Sino-Tibetan}
\define@key{fams}{rej}{Austronesian}
\define@key{fams}{rmb}{Gunwinyguan}
\define@key{fams}{bfw}{Austro-Asiatic}
\define@key{fams}{rel}{Afro-Asiatic}
\define@key{fams}{ren}{Austro-Asiatic}
\define@key{fams}{mnv}{Austronesian}
\define@key{fams}{rgr}{Arawakan}
\define@key{fams}{tnc}{Tucanoan}
\define@key{fams}{ran}{Kolopom}
\define@key{fams}{rkb}{Macro-Ge}
\define@key{fams}{rim}{Niger-Congo}
\define@key{fams}{rit}{Pama-Nyungan}
\define@key{fams}{rog}{Austronesian}
\define@key{fams}{rmn}{Indo-European}
\define@key{fams}{rmo}{Indo-European}
\define@key{fams}{rmy}{Indo-European}
\define@key{fams}{rml}{Indo-European}
\define@key{fams}{rmw}{Indo-European}
\define@key{fams}{ron}{Indo-European}
\define@key{fams}{roh}{Indo-European}
\define@key{fams}{cla}{Afro-Asiatic}
\define@key{fams}{rng}{Niger-Congo}
\define@key{fams}{rro}{Austronesian}
\define@key{fams}{twu}{Austronesian}
\define@key{fams}{roo}{West Bougainville}
\define@key{fams}{rtm}{Austronesian}
\define@key{fams}{rug}{Austronesian}
\define@key{fams}{dru}{Austronesian}
\define@key{fams}{klq}{Trans-New Guinea}
\define@key{fams}{run}{Niger-Congo}
\define@key{fams}{rou}{Maban}
\define@key{fams}{nyn}{Niger-Congo}
\define@key{fams}{nyo}{Niger-Congo}
\define@key{fams}{rus}{Indo-European}
\define@key{fams}{rsl}{other}
\define@key{fams}{rut}{Nakh-Daghestanian}
\define@key{fams}{apb}{Austronesian}
\define@key{fams}{snv}{Austronesian}
\define@key{fams}{sma}{Uralic}
\define@key{fams}{sjd}{Uralic}
\define@key{fams}{sme}{Uralic}
\define@key{fams}{skb}{Tai-Kadai}
\define@key{fams}{uma}{Penutian}
\define@key{fams}{ssy}{Afro-Asiatic}
\define@key{fams}{saj}{North Halmaheran}
\define@key{fams}{sku}{Austronesian}
\define@key{fams}{slr}{Altaic}
\define@key{fams}{sbe}{Austronesian}
\define@key{fams}{sln}{Hokan}
\define@key{fams}{slh}{Salishan}
\define@key{fams}{sll}{Trans-New Guinea}
\define@key{fams}{sse}{Austronesian}
\define@key{fams}{ssb}{Austronesian}
\define@key{fams}{ndi}{Niger-Congo}
\define@key{fams}{smq}{Trans-New Guinea}
\define@key{fams}{smo}{Austronesian}
\define@key{fams}{sad}{Isolate}
\define@key{fams}{sxn}{Austronesian}
\define@key{fams}{sag}{Niger-Congo}
\define@key{fams}{snq}{Niger-Congo}
\define@key{fams}{sce}{Altaic}
\define@key{fams}{sat}{Austro-Asiatic}
\define@key{fams}{xsu}{Yanomam}
\define@key{fams}{spu}{Austro-Asiatic}
\define@key{fams}{srm}{other}
\define@key{fams}{srs}{Na-Dene}
\define@key{fams}{sro}{Indo-European}
\define@key{fams}{dju}{Sepik}
\define@key{fams}{ybe}{Altaic}
\define@key{fams}{sdg}{Indo-European}
\define@key{fams}{svs}{Solomons East Papuan}
\define@key{fams}{szw}{Austronesian}
\define@key{fams}{hvn}{Austronesian}
\define@key{fams}{pos}{Mixe-Zoque}
\define@key{fams}{kpz}{Eastern Sudanic}
\define@key{fams}{sey}{Tucanoan}
\define@key{fams}{sed}{Austro-Asiatic}
\define@key{fams}{trv}{Austronesian}
\define@key{fams}{slu}{Austronesian}
\define@key{fams}{sly}{Austronesian}
\define@key{fams}{spl}{Trans-New Guinea}
\define@key{fams}{ona}{Chonan}
\define@key{fams}{sel}{Uralic}
\define@key{fams}{nsm}{Sino-Tibetan}
\define@key{fams}{sea}{Austro-Asiatic}
\define@key{fams}{sif}{Niger-Congo}
\define@key{fams}{sza}{Austro-Asiatic}
\define@key{fams}{seh}{Niger-Congo}
\define@key{fams}{sef}{Niger-Congo}
\define@key{fams}{see}{Iroquoian}
\define@key{fams}{szg}{Niger-Congo}
\define@key{fams}{set}{Isolate}
\define@key{fams}{hbs}{Indo-European}
\define@key{fams}{sei}{Hokan}
\define@key{fams}{ser}{Uto-Aztecan}
\define@key{fams}{sot}{Niger-Congo}
\define@key{fams}{crs}{other}
\define@key{fams}{sbf}{Isolate}
\define@key{fams}{ksb}{Niger-Congo}
\define@key{fams}{shn}{Tai-Kadai}
\define@key{fams}{mcd}{Pano-Tacanan}
\define@key{fams}{sht}{Hokan}
\define@key{fams}{shj}{Eastern Sudanic}
\define@key{fams}{sjw}{Algic}
\define@key{fams}{swv}{Indo-European}
\define@key{fams}{sdp}{Sino-Tibetan}
\define@key{fams}{xsr}{Sino-Tibetan}
\define@key{fams}{shk}{Eastern Sudanic}
\define@key{fams}{scl}{Indo-European}
\define@key{fams}{bwo}{Afro-Asiatic}
\define@key{fams}{shp}{Pano-Tacanan}
\define@key{fams}{yuy}{Altaic}
\define@key{fams}{shb}{Yanomam}
\define@key{fams}{sii}{Isolate}
\define@key{fams}{sna}{Niger-Congo}
\define@key{fams}{cjs}{Altaic}
\define@key{fams}{shh}{Uto-Aztecan}
\define@key{fams}{sgh}{Indo-European}
\define@key{fams}{ryu}{Japanese}
\define@key{fams}{shs}{Salishan}
\define@key{fams}{snp}{Trans-New Guinea}
\define@key{fams}{sjr}{Austronesian}
\define@key{fams}{sid}{Afro-Asiatic}
\define@key{fams}{ski}{Austronesian}
\define@key{fams}{tty}{Lakes Plain}
\define@key{fams}{sip}{Sino-Tibetan}
\define@key{fams}{skh}{Austronesian}
\define@key{fams}{dau}{Eastern Sudanic}
\define@key{fams}{smr}{Austronesian}
\define@key{fams}{snc}{Austronesian}
\define@key{fams}{snd}{Indo-European}
\define@key{fams}{sin}{Indo-European}
\define@key{fams}{xsi}{Austronesian}
\define@key{fams}{snn}{Tucanoan}
\define@key{fams}{qum}{Mayan}
\define@key{fams}{fos}{Austronesian}
\define@key{fams}{sri}{Tucanoan}
\define@key{fams}{srq}{Tupian}
\define@key{fams}{ssd}{Trans-New Guinea}
\define@key{fams}{sil}{Niger-Congo}
\define@key{fams}{baa}{Austronesian}
\define@key{fams}{sis}{Oregon Coast}
\define@key{fams}{skv}{Skou}
\define@key{fams}{den}{Na-Dene}
\define@key{fams}{xsl}{Na-Dene}
\define@key{fams}{slk}{Indo-European}
\define@key{fams}{slv}{Indo-European}
\define@key{fams}{teu}{Eastern Sudanic}
\define@key{fams}{sob}{Austronesian}
\define@key{fams}{gru}{Afro-Asiatic}
\define@key{fams}{evn}{Altaic}
\define@key{fams}{som}{Afro-Asiatic}
\define@key{fams}{sop}{Niger-Congo}
\define@key{fams}{snk}{Mande}
\define@key{fams}{sov}{Austronesian}
\define@key{fams}{sqt}{Afro-Asiatic}
\define@key{fams}{srb}{Austro-Asiatic}
\define@key{fams}{dsb}{Indo-European}
\define@key{fams}{hsb}{Indo-European}
\define@key{fams}{nso}{Niger-Congo}
\define@key{fams}{mnx}{East Bird's Head}
\define@key{fams}{kvk}{other}
\define@key{fams}{tvk}{Austronesian}
\define@key{fams}{wib}{Niger-Congo}
\define@key{fams}{spa}{Indo-European}
\define@key{fams}{spt}{Sino-Tibetan}
\define@key{fams}{spo}{Salishan}
\define@key{fams}{squ}{Salishan}
\define@key{fams}{srn}{other}
\define@key{fams}{kpm}{Austro-Asiatic}
\define@key{fams}{sto}{Siouan}
\define@key{fams}{sbs}{Niger-Congo}
\define@key{fams}{tgo}{Austronesian}
\define@key{fams}{sue}{Trans-New Guinea}
\define@key{fams}{swi}{Tai-Kadai}
\define@key{fams}{sui}{Trans-New Guinea}
\define@key{fams}{sub}{Niger-Congo}
\define@key{fams}{suk}{Niger-Congo}
\define@key{fams}{sua}{Isolate}
\define@key{fams}{suv}{Isolate}
\define@key{fams}{sun}{Austronesian}
\define@key{fams}{sjg}{Eastern Sudanic}
\define@key{fams}{spp}{Niger-Congo}
\define@key{fams}{sgz}{Austronesian}
\define@key{fams}{sus}{Mande}
\define@key{fams}{sva}{Kartvelian}
\define@key{fams}{swl}{other}
\define@key{fams}{swh}{Niger-Congo}
\define@key{fams}{ssw}{Niger-Congo}
\define@key{fams}{swe}{Indo-European}
\define@key{fams}{slc}{Sáliban}
\define@key{fams}{mky}{Austronesian}
\define@key{fams}{sst}{Trans-New Guinea}
\define@key{fams}{tby}{North Halmaheran}
\define@key{fams}{tab}{Nakh-Daghestanian}
\define@key{fams}{tnm}{Sentani}
\define@key{fams}{tap}{Niger-Congo}
\define@key{fams}{tna}{Pano-Tacanan}
\define@key{fams}{tgl}{Austronesian}
\define@key{fams}{tbw}{Austronesian}
\define@key{fams}{tah}{Austronesian}
\define@key{fams}{gpn}{Gapun}
\define@key{fams}{sps}{Austronesian}
\define@key{fams}{tbg}{Trans-New Guinea}
\define@key{fams}{tss}{other}
\define@key{fams}{tgk}{Indo-European}
\define@key{fams}{tkm}{Isolate}
\define@key{fams}{tbc}{Austronesian}
\define@key{fams}{tld}{Austronesian}
\define@key{fams}{tlj}{Niger-Congo}
\define@key{fams}{tly}{Indo-European}
\define@key{fams}{tma}{Eastern Sudanic}
\define@key{fams}{mla}{Austronesian}
\define@key{fams}{tcg}{Kayagar}
\define@key{fams}{taj}{Sino-Tibetan}
\define@key{fams}{taq}{Afro-Asiatic}
\define@key{fams}{tam}{Dravidian}
\define@key{fams}{tpm}{Niger-Congo}
\define@key{fams}{tcb}{Na-Dene}
\define@key{fams}{tfn}{Na-Dene}
\define@key{fams}{taa}{Na-Dene}
\define@key{fams}{tan}{Afro-Asiatic}
\define@key{fams}{skj}{Sino-Tibetan}
\define@key{fams}{tgg}{Austronesian}
\define@key{fams}{tpg}{Greater West Bomberai}
\define@key{fams}{nwi}{Austronesian}
\define@key{fams}{tza}{other}
\define@key{fams}{tpj}{Tupian}
\define@key{fams}{tar}{Uto-Aztecan}
\define@key{fams}{tac}{Uto-Aztecan}
\define@key{fams}{txn}{Austronesian}
\define@key{fams}{tro}{Sino-Tibetan}
\define@key{fams}{tae}{Arawakan}
\define@key{fams}{yer}{Niger-Congo}
\define@key{fams}{shi}{Afro-Asiatic}
\define@key{fams}{ttt}{Indo-European}
\define@key{fams}{txx}{Austronesian}
\define@key{fams}{tat}{Altaic}
\define@key{fams}{tks}{Indo-European}
\define@key{fams}{tav}{Tucanoan}
\define@key{fams}{tuh}{Isolate}
\define@key{fams}{trr}{Isolate}
\define@key{fams}{tsg}{Austronesian}
\define@key{fams}{tya}{Trans-New Guinea}
\define@key{fams}{tbo}{Austronesian}
\define@key{fams}{cks}{other}
\define@key{fams}{tbl}{Austronesian}
\define@key{fams}{ttc}{Mayan}
\define@key{fams}{kps}{West Bird's Head}
\define@key{fams}{teh}{Chonan}
\define@key{fams}{kkw}{Niger-Congo}
\define@key{fams}{tlf}{Trans-New Guinea}
\define@key{fams}{tel}{Dravidian}
\define@key{fams}{kdh}{Niger-Congo}
\define@key{fams}{teq}{Eastern Sudanic}
\define@key{fams}{tea}{Austro-Asiatic}
\define@key{fams}{tem}{Niger-Congo}
\define@key{fams}{tex}{Eastern Sudanic}
\define@key{fams}{kza}{Niger-Congo}
\define@key{fams}{tio}{Austronesian}
\define@key{fams}{tep}{Uto-Aztecan}
\define@key{fams}{tee}{Totonacan}
\define@key{fams}{tpt}{Totonacan}
\define@key{fams}{ntp}{Uto-Aztecan}
\define@key{fams}{stp}{Uto-Aztecan}
\define@key{fams}{ttr}{Afro-Asiatic}
\define@key{fams}{tfr}{Chibchan}
\define@key{fams}{tft}{North Halmaheran}
\define@key{fams}{ter}{Arawakan}
\define@key{fams}{teo}{Eastern Sudanic}
\define@key{fams}{tll}{Niger-Congo}
\define@key{fams}{tet}{Austronesian}
\define@key{fams}{tew}{Kiowa-Tanoan}
\define@key{fams}{tcz}{Sino-Tibetan}
\define@key{fams}{tha}{Tai-Kadai}
\define@key{fams}{tsq}{other}
\define@key{fams}{ths}{Sino-Tibetan}
\define@key{fams}{thf}{Sino-Tibetan}
\define@key{fams}{ssf}{Austronesian}
\define@key{fams}{typ}{Pama-Nyungan}
\define@key{fams}{thp}{Salishan}
\define@key{fams}{tdh}{Sino-Tibetan}
\define@key{fams}{tca}{Isolate}
\define@key{fams}{tvo}{North Halmaheran}
\define@key{fams}{tif}{Trans-New Guinea}
\define@key{fams}{tgc}{Austronesian}
\define@key{fams}{tir}{Afro-Asiatic}
\define@key{fams}{tig}{Afro-Asiatic}
\define@key{fams}{dih}{Hokan}
\define@key{fams}{tik}{Niger-Congo}
\define@key{fams}{til}{Salishan}
\define@key{fams}{tms}{Kordofanian}
\define@key{fams}{aoz}{Austronesian}
\define@key{fams}{tjm}{Isolate}
\define@key{fams}{tih}{Austronesian}
\define@key{fams}{lbf}{Sino-Tibetan}
\define@key{fams}{tin}{Nakh-Daghestanian}
\define@key{fams}{cir}{Austronesian}
\define@key{fams}{tri}{Cariban}
\define@key{fams}{tiy}{Austronesian}
\define@key{fams}{tiv}{Niger-Congo}
\define@key{fams}{twf}{Kiowa-Tanoan}
\define@key{fams}{tix}{Kiowa-Tanoan}
\define@key{fams}{tiw}{Tiwian}
\define@key{fams}{tcf}{Oto-Manguean}
\define@key{fams}{tli}{Na-Dene}
\define@key{fams}{tqo}{Eleman}
\define@key{fams}{tob}{Guaicuruan}
\define@key{fams}{tti}{Austronesian}
\define@key{fams}{tlb}{North Halmaheran}
\define@key{fams}{sbu}{Sino-Tibetan}
\define@key{fams}{tcx}{Dravidian}
\define@key{fams}{kim}{Altaic}
\define@key{fams}{toj}{Mayan}
\define@key{fams}{tpi}{other}
\define@key{fams}{tkl}{Austronesian}
\define@key{fams}{jic}{Isolate}
\define@key{fams}{ksd}{Austronesian}
\define@key{fams}{dto}{Dogon}
\define@key{fams}{tdn}{Austronesian}
\define@key{fams}{toi}{Niger-Congo}
\define@key{fams}{ton}{Austronesian}
\define@key{fams}{tqw}{Isolate}
\define@key{fams}{tnt}{Austronesian}
\define@key{fams}{mlu}{Austronesian}
\define@key{fams}{sda}{Austronesian}
\define@key{fams}{rth}{Austronesian}
\define@key{fams}{dts}{Dogon}
\define@key{fams}{trw}{Indo-European}
\define@key{fams}{tlc}{Totonacan}
\define@key{fams}{top}{Totonacan}
\define@key{fams}{tos}{Totonacan}
\define@key{fams}{too}{Totonacan}
\define@key{fams}{trs}{Oto-Manguean}
\define@key{fams}{trc}{Oto-Manguean}
\define@key{fams}{tpy}{Isolate}
\define@key{fams}{cof}{Barbacoan}
\define@key{fams}{tkr}{Nakh-Daghestanian}
\define@key{fams}{huq}{Austronesian}
\define@key{fams}{ddo}{Nakh-Daghestanian}
\define@key{fams}{tsj}{Sino-Tibetan}
\define@key{fams}{tsi}{Tsimshianic}
\define@key{fams}{tsv}{Niger-Congo}
\define@key{fams}{tso}{Niger-Congo}
\define@key{fams}{tsu}{Austronesian}
\define@key{fams}{bbl}{Nakh-Daghestanian}
\define@key{fams}{tsn}{Niger-Congo}
\define@key{fams}{pmt}{Austronesian}
\define@key{fams}{thz}{Afro-Asiatic}
\define@key{fams}{thv}{Afro-Asiatic}
\define@key{fams}{tbu}{Uto-Aztecan}
\define@key{fams}{tuo}{Tucanoan}
\define@key{fams}{tzn}{Austronesian}
\define@key{fams}{bag}{Niger-Congo}
\define@key{fams}{tcy}{Dravidian}
\define@key{fams}{tmc}{Afro-Asiatic}
\define@key{fams}{tmq}{Austronesian}
\define@key{fams}{tuf}{Chibchan}
\define@key{fams}{tvu}{Niger-Congo}
\define@key{fams}{lcm}{Austronesian}
\define@key{fams}{tun}{Isolate}
\define@key{fams}{tpn}{Tupian}
\define@key{fams}{tui}{Niger-Congo}
\define@key{fams}{tuv}{Eastern Sudanic}
\define@key{fams}{kmz}{Altaic}
\define@key{fams}{tur}{Altaic}
\define@key{fams}{tuk}{Altaic}
\define@key{fams}{tus}{Iroquoian}
\define@key{fams}{ttm}{Na-Dene}
\define@key{fams}{tta}{Siouan}
\define@key{fams}{tvt}{Sino-Tibetan}
\define@key{fams}{tyv}{Altaic}
\define@key{fams}{tue}{Tucanoan}
\define@key{fams}{twa}{Salishan}
\define@key{fams}{woa}{Northern Daly}
\define@key{fams}{tzh}{Mayan}
\define@key{fams}{tzo}{Mayan}
\define@key{fams}{tzj}{Mayan}
\define@key{fams}{tub}{Uto-Aztecan}
\define@key{fams}{par}{Uto-Aztecan}
\define@key{fams}{tsm}{other}
\define@key{fams}{umb}{Niger-Congo}
\define@key{fams}{uby}{Northwest Caucasian}
\define@key{fams}{udi}{Nakh-Daghestanian}
\define@key{fams}{ude}{Altaic}
\define@key{fams}{udm}{Uralic}
\define@key{fams}{ugn}{other}
\define@key{fams}{ukr}{Indo-European}
\define@key{fams}{ulc}{Altaic}
\define@key{fams}{udl}{Afro-Asiatic}
\define@key{fams}{uli}{Austronesian}
\define@key{fams}{ppk}{Austronesian}
\define@key{fams}{cbd}{Cariban}
\define@key{fams}{ubu}{Trans-New Guinea}
\define@key{fams}{ump}{Pama-Nyungan}
\define@key{fams}{mtg}{Trans-New Guinea}
\define@key{fams}{unm}{Algic}
\define@key{fams}{ung}{Worrorran}
\define@key{fams}{kuu}{Na-Dene}
\define@key{fams}{uur}{Austronesian}
\define@key{fams}{urf}{Pama-Nyungan}
\define@key{fams}{urk}{Austronesian}
\define@key{fams}{ura}{Isolate}
\define@key{fams}{urt}{Torricelli}
\define@key{fams}{urd}{Indo-European}
\define@key{fams}{urh}{Niger-Congo}
\define@key{fams}{uri}{Torricelli}
\define@key{fams}{ure}{Uru-Chipaya}
\define@key{fams}{uks}{other}
\define@key{fams}{urb}{Tupian}
\define@key{fams}{uum}{Altaic}
\define@key{fams}{wnu}{Trans-New Guinea}
\define@key{fams}{usa}{Trans-New Guinea}
\define@key{fams}{ute}{Uto-Aztecan}
\define@key{fams}{uig}{Altaic}
\define@key{fams}{uzn}{Altaic}
\define@key{fams}{vaf}{Indo-European}
\define@key{fams}{vag}{Niger-Congo}
\define@key{fams}{vai}{Mande}
\define@key{fams}{vas}{Indo-European}
\define@key{fams}{dic}{Niger-Congo}
\define@key{fams}{ved}{Indo-European}
\define@key{fams}{ven}{Niger-Congo}
\define@key{fams}{vep}{Uralic}
\define@key{fams}{vie}{Austro-Asiatic}
\define@key{fams}{vif}{Niger-Congo}
\define@key{fams}{vnm}{Austronesian}
\define@key{fams}{vgt}{other}
\define@key{fams}{vot}{Uralic}
\define@key{fams}{wwa}{Niger-Congo}
\define@key{fams}{wkw}{Pama-Nyungan}
\define@key{fams}{waq}{Isolate}
\define@key{fams}{waw}{Cariban}
\define@key{fams}{wbk}{Indo-European}
\define@key{fams}{bao}{Tucanoan}
\define@key{fams}{wbl}{Indo-European}
\define@key{fams}{wls}{Austronesian}
\define@key{fams}{van}{Torricelli}
\define@key{fams}{wmt}{Pama-Nyungan}
\define@key{fams}{wmb}{Mirndi}
\define@key{fams}{wms}{Trans-New Guinea}
\define@key{fams}{wme}{Sino-Tibetan}
\define@key{fams}{wan}{Mande}
\define@key{fams}{wgg}{Pama-Nyungan}
\define@key{fams}{xwk}{Pama-Nyungan}
\define@key{fams}{wbt}{Pama-Nyungan}
\define@key{fams}{wnc}{Trans-New Guinea}
\define@key{fams}{auc}{Isolate}
\define@key{fams}{wap}{Arawakan}
\define@key{fams}{wao}{Wappo-Yukian}
\define@key{fams}{wba}{Isolate}
\define@key{fams}{wrz}{Gunwinyguan}
\define@key{fams}{war}{Austronesian}
\define@key{fams}{wrr}{Yangmanic}
\define@key{fams}{gae}{Arawakan}
\define@key{fams}{wsa}{Austronesian}
\define@key{fams}{pav}{Chapacura-Wanham}
\define@key{fams}{wrs}{Border}
\define@key{fams}{wbp}{Pama-Nyungan}
\define@key{fams}{wrb}{Pama-Nyungan}
\define@key{fams}{wnd}{Mangarrayi-Maran}
\define@key{fams}{wrp}{Austronesian}
\define@key{fams}{wgy}{Pama-Nyungan}
\define@key{fams}{gjm}{Pama-Nyungan}
\define@key{fams}{wrg}{Pama-Nyungan}
\define@key{fams}{wwr}{Nyulnyulan}
\define@key{fams}{wrm}{Pama-Nyungan}
\define@key{fams}{was}{Isolate}
\define@key{fams}{wsk}{Trans-New Guinea}
\define@key{fams}{wax}{Lower Sepik-Ramu}
\define@key{fams}{wth}{Pama-Nyungan}
\define@key{fams}{wbv}{Pama-Nyungan}
\define@key{fams}{noa}{Choco}
\define@key{fams}{wau}{Arawakan}
\define@key{fams}{oym}{Tupian}
\define@key{fams}{way}{Cariban}
\define@key{fams}{wed}{Austronesian}
\define@key{fams}{cym}{Indo-European}
\define@key{fams}{xww}{Pama-Nyungan}
\define@key{fams}{wer}{Trans-New Guinea}
\define@key{fams}{mqs}{North Halmaheran}
\define@key{fams}{lex}{Austronesian}
\define@key{fams}{wic}{Caddoan}
\define@key{fams}{mzh}{Matacoan}
\define@key{fams}{wim}{Pama-Nyungan}
\define@key{fams}{wig}{Pama-Nyungan}
\define@key{fams}{yok}{Penutian}
\define@key{fams}{win}{Siouan}
\define@key{fams}{wnw}{Penutian}
\define@key{fams}{wgu}{Pama-Nyungan}
\define@key{fams}{wiy}{Algic}
\define@key{fams}{wob}{Niger-Congo}
\define@key{fams}{wog}{Sepik}
\define@key{fams}{woi}{Greater West Bomberai}
\define@key{fams}{wyu}{Pama-Nyungan}
\define@key{fams}{wal}{Afro-Asiatic}
\define@key{fams}{woe}{Austronesian}
\define@key{fams}{wlo}{Austronesian}
\define@key{fams}{wol}{Niger-Congo}
\define@key{fams}{wmx}{Skou}
\define@key{fams}{wro}{Worrorran}
\define@key{fams}{wuu}{Sino-Tibetan}
\define@key{fams}{wya}{Iroquoian}
\define@key{fams}{wem}{Niger-Congo}
\define@key{fams}{kao}{Mande}
\define@key{fams}{xav}{Macro-Ge}
\define@key{fams}{xer}{Macro-Ge}
\define@key{fams}{xho}{Niger-Congo}
\define@key{fams}{xir}{Arawakan}
\define@key{fams}{xok}{Macro-Ge}
\define@key{fams}{ane}{Austronesian}
\define@key{fams}{yai}{Indo-European}
\define@key{fams}{yad}{Peba-Yaguan}
\define@key{fams}{yag}{Yámana}
\define@key{fams}{yaf}{Niger-Congo}
\define@key{fams}{yka}{Austronesian}
\define@key{fams}{yky}{Niger-Congo}
\define@key{fams}{sah}{Altaic}
\define@key{fams}{ylr}{Pama-Nyungan}
\define@key{fams}{kkl}{Trans-New Guinea}
\define@key{fams}{yli}{Trans-New Guinea}
\define@key{fams}{yam}{Niger-Congo}
\define@key{fams}{jmd}{Austronesian}
\define@key{fams}{tao}{Austronesian}
\define@key{fams}{yaa}{Pano-Tacanan}
\define@key{fams}{ybi}{Sino-Tibetan}
\define@key{fams}{ynn}{Hokan}
\define@key{fams}{kdd}{Pama-Nyungan}
\define@key{fams}{wca}{Yanomam}
\define@key{fams}{yns}{Niger-Congo}
\define@key{fams}{jao}{Pama-Nyungan}
\define@key{fams}{yao}{Niger-Congo}
\define@key{fams}{yap}{Austronesian}
\define@key{fams}{jaq}{Trans-New Guinea}
\define@key{fams}{yaq}{Uto-Aztecan}
\define@key{fams}{yrb}{Trans-New Guinea}
\define@key{fams}{yae}{Isolate}
\define@key{fams}{yuf}{Hokan}
\define@key{fams}{yva}{Isolate}
\define@key{fams}{ywr}{Nyulnyulan}
\define@key{fams}{pcc}{Tai-Kadai}
\define@key{fams}{xya}{Pama-Nyungan}
\define@key{fams}{yah}{Indo-European}
\define@key{fams}{kpv}{Uralic}
\define@key{fams}{jei}{Yam}
\define@key{fams}{jel}{Bulaka River}
\define@key{fams}{yle}{Yele}
\define@key{fams}{ybb}{Niger-Congo}
\define@key{fams}{jnj}{Afro-Asiatic}
\define@key{fams}{yss}{Sepik}
\define@key{fams}{yey}{Niger-Congo}
\define@key{fams}{ywq}{Sino-Tibetan}
\define@key{fams}{ydd}{Indo-European}
\define@key{fams}{yii}{Pama-Nyungan}
\define@key{fams}{yll}{Torricelli}
\define@key{fams}{yee}{Lower Sepik-Ramu}
\define@key{fams}{yij}{Pama-Nyungan}
\define@key{fams}{yia}{Pama-Nyungan}
\define@key{fams}{yyr}{Pama-Nyungan}
\define@key{fams}{xyy}{Pama-Nyungan}
\define@key{fams}{yor}{Niger-Congo}
\define@key{fams}{yua}{Mayan}
\define@key{fams}{yuc}{Isolate}
\define@key{fams}{ycn}{Arawakan}
\define@key{fams}{yug}{Yeniseian}
\define@key{fams}{yux}{Yukaghir}
\define@key{fams}{ykg}{Yukaghir}
\define@key{fams}{yuk}{Wappo-Yukian}
\define@key{fams}{yup}{Cariban}
\define@key{fams}{gcd}{Tangkic}
\define@key{fams}{mpj}{Pama-Nyungan}
\define@key{fams}{yul}{Central Sudanic}
\define@key{fams}{esu}{Eskimo-Aleut}
\define@key{fams}{ynk}{Eskimo-Aleut}
\define@key{fams}{ess}{Eskimo-Aleut}
\define@key{fams}{ysr}{Eskimo-Aleut}
\define@key{fams}{yuz}{Isolate}
\define@key{fams}{yur}{Algic}
\define@key{fams}{yui}{Tucanoan}
\define@key{fams}{zne}{Niger-Congo}
\define@key{fams}{zro}{Zaparoan}
\define@key{fams}{zai}{Oto-Manguean}
\define@key{fams}{zpd}{Oto-Manguean}
\define@key{fams}{zaa}{Oto-Manguean}
\define@key{fams}{zaw}{Oto-Manguean}
\define@key{fams}{zpm}{Oto-Manguean}
\define@key{fams}{zpi}{Oto-Manguean}
\define@key{fams}{zab}{Oto-Manguean}
\define@key{fams}{zpz}{Oto-Manguean}
\define@key{fams}{zav}{Oto-Manguean}
\define@key{fams}{zpq}{Oto-Manguean}
\define@key{fams}{dje}{Songhay}
\define@key{fams}{zay}{Afro-Asiatic}
\define@key{fams}{diq}{Indo-European}
\define@key{fams}{zen}{Afro-Asiatic}
\define@key{fams}{zgb}{Tai-Kadai}
\define@key{fams}{zik}{Trans-New Guinea}
\define@key{fams}{zoh}{Mixe-Zoque}
\define@key{fams}{zos}{Mixe-Zoque}
\define@key{fams}{zoc}{Mixe-Zoque}
\define@key{fams}{zor}{Mixe-Zoque}
\define@key{fams}{zul}{Niger-Congo}
\define@key{fams}{zun}{Isolate}
\define@key{fams}{eme}{Tupian}
\define@key{fams}{aom}{Trans-New Guinea}
\define@key{fams}{aas}{nan}
\define@key{fams}{kbt}{nan}
\define@key{fams}{abg}{nan}
\define@key{fams}{abf}{nan}
\define@key{fams}{abm}{nan}
\define@key{fams}{mij}{nan}
\define@key{fams}{aba}{nan}
\define@key{fams}{abp}{nan}
\define@key{fams}{bsa}{nan}
\define@key{fams}{ash}{nan}
\define@key{fams}{aob}{nan}
\define@key{fams}{abo}{nan}
\define@key{fams}{abr}{nan}
\define@key{fams}{abn}{nan}
\define@key{fams}{abu}{nan}
\define@key{fams}{mgj}{nan}
\define@key{fams}{ado}{nan}
\define@key{fams}{tpx}{nan}
\define@key{fams}{yif}{nan}
\define@key{fams}{acz}{nan}
\define@key{fams}{acs}{nan}
\define@key{fams}{xad}{nan}
\define@key{fams}{ada}{nan}
\define@key{fams}{adq}{nan}
\define@key{fams}{tiu}{nan}
\define@key{fams}{ade}{nan}
\define@key{fams}{adh}{nan}
\define@key{fams}{gas}{nan}
\define@key{fams}{adr}{nan}
\define@key{fams}{aez}{nan}
\define@key{fams}{aeq}{nan}
\define@key{fams}{afg}{nan}
\define@key{fams}{aft}{nan}
\define@key{fams}{afh}{nan}
\define@key{fams}{afs}{nan}
\define@key{fams}{agi}{nan}
\define@key{fams}{agc}{nan}
\define@key{fams}{avo}{nan}
\define@key{fams}{ggr}{nan}
\define@key{fams}{xag}{nan}
\define@key{fams}{aif}{nan}
\define@key{fams}{kit}{nan}
\define@key{fams}{ibm}{nan}
\define@key{fams}{apf}{nan}
\define@key{fams}{aga}{nan}
\define@key{fams}{aug}{nan}
\define@key{fams}{msm}{nan}
\define@key{fams}{agn}{nan}
\define@key{fams}{yay}{nan}
\define@key{fams}{aha}{nan}
\define@key{fams}{ahn}{nan}
\define@key{fams}{esg}{nan}
\define@key{fams}{thm}{nan}
\define@key{fams}{kak}{nan}
\define@key{fams}{aho}{nan}
\define@key{fams}{nfd}{nan}
\define@key{fams}{aih}{nan}
\define@key{fams}{aix}{nan}
\define@key{fams}{mwg}{nan}
\define@key{fams}{aiq}{nan}
\define@key{fams}{ail}{nan}
\define@key{fams}{aim}{nan}
\define@key{fams}{aic}{nan}
\define@key{fams}{aki}{nan}
\define@key{fams}{air}{nan}
\define@key{fams}{aio}{nan}
\define@key{fams}{ajw}{nan}
\define@key{fams}{cpc}{nan}
\define@key{fams}{soh}{nan}
\define@key{fams}{akm}{nan}
\define@key{fams}{akj}{nan}
\define@key{fams}{ack}{nan}
\define@key{fams}{aky}{nan}
\define@key{fams}{acl}{nan}
\define@key{fams}{aks}{nan}
\define@key{fams}{aik}{nan}
\define@key{fams}{tsr}{nan}
\define@key{fams}{aeu}{nan}
\define@key{fams}{sia}{nan}
\define@key{fams}{akk}{nan}
\define@key{fams}{akq}{nan}
\define@key{fams}{akt}{nan}
\define@key{fams}{bss}{nan}
\define@key{fams}{miw}{nan}
\define@key{fams}{akf}{nan}
\define@key{fams}{ibe}{nan}
\define@key{fams}{afi}{nan}
\define@key{fams}{ayk}{nan}
\define@key{fams}{aku}{nan}
\define@key{fams}{aqz}{nan}
\define@key{fams}{ako}{nan}
\define@key{fams}{dul}{nan}
\define@key{fams}{alw}{nan}
\define@key{fams}{ala}{nan}
\define@key{fams}{alk}{nan}
\define@key{fams}{alj}{nan}
\define@key{fams}{apv}{nan}
\define@key{fams}{bhk}{nan}
\define@key{fams}{sqk}{nan}
\define@key{fams}{lsc}{nan}
\define@key{fams}{xta}{nan}
\define@key{fams}{alf}{nan}
\define@key{fams}{asp}{nan}
\define@key{fams}{arq}{nan}
\define@key{fams}{aao}{nan}
\define@key{fams}{aiy}{nan}
\define@key{fams}{all}{nan}
\define@key{fams}{aid}{nan}
\define@key{fams}{zaq}{nan}
\define@key{fams}{ypo}{nan}
\define@key{fams}{aol}{nan}
\define@key{fams}{syy}{nan}
\define@key{fams}{aub}{nan}
\define@key{fams}{xua}{nan}
\define@key{fams}{aab}{nan}
\define@key{fams}{yna}{nan}
\define@key{fams}{alz}{nan}
\define@key{fams}{avd}{nan}
\define@key{fams}{amq}{nan}
\define@key{fams}{ali}{nan}
\define@key{fams}{aad}{nan}
\define@key{fams}{jks}{nan}
\define@key{fams}{ama}{nan}
\define@key{fams}{amg}{nan}
\define@key{fams}{aaz}{nan}
\define@key{fams}{zpo}{nan}
\define@key{fams}{rwm}{nan}
\define@key{fams}{utp}{nan}
\define@key{fams}{abc}{nan}
\define@key{fams}{aew}{nan}
\define@key{fams}{ael}{nan}
\define@key{fams}{amv}{nan}
\define@key{fams}{alm}{nan}
\define@key{fams}{amb}{nan}
\define@key{fams}{abs}{nan}
\define@key{fams}{qva}{nan}
\define@key{fams}{aag}{nan}
\define@key{fams}{amj}{nan}
\define@key{fams}{ifa}{nan}
\define@key{fams}{alx}{nan}
\define@key{fams}{mbz}{nan}
\define@key{fams}{aqd}{nan}
\define@key{fams}{apg}{nan}
\define@key{fams}{ajz}{nan}
\define@key{fams}{amt}{nan}
\define@key{fams}{adw}{nan}
\define@key{fams}{anw}{nan}
\define@key{fams}{akg}{nan}
\define@key{fams}{anm}{nan}
\define@key{fams}{pda}{nan}
\define@key{fams}{aan}{nan}
\define@key{fams}{dti}{nan}
\define@key{fams}{grc}{nan}
\define@key{fams}{hbo}{nan}
\define@key{fams}{xna}{nan}
\define@key{fams}{xlg}{nan}
\define@key{fams}{hca}{nan}
\define@key{fams}{afd}{nan}
\define@key{fams}{aod}{nan}
\define@key{fams}{ana}{nan}
\define@key{fams}{xaa}{nan}
\define@key{fams}{adg}{nan}
\define@key{fams}{bzb}{nan}
\define@key{fams}{anb}{nan}
\define@key{fams}{anx}{nan}
\define@key{fams}{aby}{nan}
\define@key{fams}{myo}{nan}
\define@key{fams}{akh}{nan}
\define@key{fams}{age}{nan}
\define@key{fams}{aoe}{nan}
\define@key{fams}{aqt}{nan}
\define@key{fams}{avm}{nan}
\define@key{fams}{anp}{nan}
\define@key{fams}{rme}{nan}
\define@key{fams}{aog}{nan}
\define@key{fams}{tnd}{nan}
\define@key{fams}{blo}{nan}
\define@key{fams}{anf}{nan}
\define@key{fams}{aqk}{nan}
\define@key{fams}{ypn}{nan}
\define@key{fams}{boj}{nan}
\define@key{fams}{aak}{nan}
\define@key{fams}{amx}{nan}
\define@key{fams}{anj}{nan}
\define@key{fams}{ans}{nan}
\define@key{fams}{and}{nan}
\define@key{fams}{ant}{nan}
\define@key{fams}{xmv}{nan}
\define@key{fams}{aig}{nan}
\define@key{fams}{aui}{nan}
\define@key{fams}{auq}{nan}
\define@key{fams}{aud}{nan}
\define@key{fams}{anl}{nan}
\define@key{fams}{mtb}{nan}
\define@key{fams}{pni}{nan}
\define@key{fams}{aor}{nan}
\define@key{fams}{aou}{nan}
\define@key{fams}{xap}{nan}
\define@key{fams}{apo}{nan}
\define@key{fams}{ena}{nan}
\define@key{fams}{mip}{nan}
\define@key{fams}{api}{nan}
\define@key{fams}{app}{nan}
\define@key{fams}{apx}{nan}
\define@key{fams}{arg}{nan}
\define@key{fams}{stk}{nan}
\define@key{fams}{aaf}{nan}
\define@key{fams}{xrt}{nan}
\define@key{fams}{arj}{nan}
\define@key{fams}{awm}{nan}
\define@key{fams}{awt}{nan}
\define@key{fams}{aae}{nan}
\define@key{fams}{aea}{nan}
\define@key{fams}{mwc}{nan}
\define@key{fams}{aem}{nan}
\define@key{fams}{qxu}{nan}
\define@key{fams}{agj}{nan}
\define@key{fams}{agf}{nan}
\define@key{fams}{aqr}{nan}
\define@key{fams}{aok}{nan}
\define@key{fams}{ylu}{nan}
\define@key{fams}{aai}{nan}
\define@key{fams}{aqg}{nan}
\define@key{fams}{aac}{nan}
\define@key{fams}{ait}{nan}
\define@key{fams}{ark}{nan}
\define@key{fams}{xrn}{nan}
\define@key{fams}{luc}{nan}
\define@key{fams}{dth}{nan}
\define@key{fams}{aoh}{nan}
\define@key{fams}{aen}{nan}
\define@key{fams}{rup}{nan}
\define@key{fams}{aps}{nan}
\define@key{fams}{atz}{nan}
\define@key{fams}{arx}{nan}
\define@key{fams}{aru}{nan}
\define@key{fams}{aur}{nan}
\define@key{fams}{lsr}{nan}
\define@key{fams}{atx}{nan}
\define@key{fams}{aat}{nan}
\define@key{fams}{mtv}{nan}
\define@key{fams}{cni}{nan}
\define@key{fams}{ahs}{nan}
\define@key{fams}{prq}{nan}
\define@key{fams}{ask}{nan}
\define@key{fams}{atn}{nan}
\define@key{fams}{asl}{nan}
\define@key{fams}{eiv}{nan}
\define@key{fams}{asv}{nan}
\define@key{fams}{asb}{nan}
\define@key{fams}{asz}{nan}
\define@key{fams}{aua}{nan}
\define@key{fams}{aum}{nan}
\define@key{fams}{zoo}{nan}
\define@key{fams}{asr}{nan}
\define@key{fams}{atm}{nan}
\define@key{fams}{amz}{nan}
\define@key{fams}{atd}{nan}
\define@key{fams}{ate}{nan}
\define@key{fams}{atk}{nan}
\define@key{fams}{aqm}{nan}
\define@key{fams}{aot}{nan}
\define@key{fams}{ato}{nan}
\define@key{fams}{aox}{nan}
\define@key{fams}{cch}{nan}
\define@key{fams}{atc}{nan}
\define@key{fams}{pkr}{nan}
\define@key{fams}{ati}{nan}
\define@key{fams}{kud}{nan}
\define@key{fams}{aux}{nan}
\define@key{fams}{auh}{nan}
\define@key{fams}{avs}{nan}
\define@key{fams}{asq}{nan}
\define@key{fams}{asw}{nan}
\define@key{fams}{aut}{nan}
\define@key{fams}{smf}{nan}
\define@key{fams}{auu}{nan}
\define@key{fams}{auo}{nan}
\define@key{fams}{avv}{nan}
\define@key{fams}{avb}{nan}
\define@key{fams}{ave}{nan}
\define@key{fams}{awk}{nan}
\define@key{fams}{vwa}{nan}
\define@key{fams}{bcu}{nan}
\define@key{fams}{awo}{nan}
\define@key{fams}{awx}{nan}
\define@key{fams}{aya}{nan}
\define@key{fams}{awh}{nan}
\define@key{fams}{bob}{nan}
\define@key{fams}{awr}{nan}
\define@key{fams}{awe}{nan}
\define@key{fams}{azo}{nan}
\define@key{fams}{auj}{nan}
\define@key{fams}{aww}{nan}
\define@key{fams}{afu}{nan}
\define@key{fams}{yiu}{nan}
\define@key{fams}{ahb}{nan}
\define@key{fams}{yix}{nan}
\define@key{fams}{ayd}{nan}
\define@key{fams}{vmy}{nan}
\define@key{fams}{aye}{nan}
\define@key{fams}{ayq}{nan}
\define@key{fams}{yyz}{nan}
\define@key{fams}{ayb}{nan}
\define@key{fams}{zaf}{nan}
\define@key{fams}{ayu}{nan}
\define@key{fams}{aza}{nan}
\define@key{fams}{yiz}{nan}
\define@key{fams}{tpc}{nan}
\define@key{fams}{bvj}{nan}
\define@key{fams}{bqx}{nan}
\define@key{fams}{bbm}{nan}
\define@key{fams}{bbw}{nan}
\define@key{fams}{bbk}{nan}
\define@key{fams}{mbf}{nan}
\define@key{fams}{bcr}{nan}
\define@key{fams}{bzg}{nan}
\define@key{fams}{btj}{nan}
\define@key{fams}{bcy}{nan}
\define@key{fams}{xbc}{nan}
\define@key{fams}{bau}{nan}
\define@key{fams}{bhz}{nan}
\define@key{fams}{bdz}{nan}
\define@key{fams}{jbi}{nan}
\define@key{fams}{bac}{nan}
\define@key{fams}{pbp}{nan}
\define@key{fams}{bvd}{nan}
\define@key{fams}{bvc}{nan}
\define@key{fams}{btr}{nan}
\define@key{fams}{bwt}{nan}
\define@key{fams}{bfj}{nan}
\define@key{fams}{bmd}{nan}
\define@key{fams}{bgo}{nan}
\define@key{fams}{bcg}{nan}
\define@key{fams}{bfy}{nan}
\define@key{fams}{fui}{nan}
\define@key{fams}{bqg}{nan}
\define@key{fams}{bqb}{nan}
\define@key{fams}{bpi}{nan}
\define@key{fams}{yha}{nan}
\define@key{fams}{bhv}{nan}
\define@key{fams}{bah}{nan}
\define@key{fams}{bhj}{nan}
\define@key{fams}{bsu}{nan}
\define@key{fams}{bbf}{nan}
\define@key{fams}{bdj}{nan}
\define@key{fams}{bkx}{nan}
\define@key{fams}{bqh}{nan}
\define@key{fams}{bmx}{nan}
\define@key{fams}{bab}{nan}
\define@key{fams}{bcz}{nan}
\define@key{fams}{fah}{nan}
\define@key{fams}{bjs}{nan}
\define@key{fams}{bjm}{nan}
\define@key{fams}{bqz}{nan}
\define@key{fams}{bqi}{nan}
\define@key{fams}{bki}{nan}
\define@key{fams}{bkh}{nan}
\define@key{fams}{kme}{nan}
\define@key{fams}{bbs}{nan}
\define@key{fams}{bkr}{nan}
\define@key{fams}{bjw}{nan}
\define@key{fams}{ble}{nan}
\define@key{fams}{bjt}{nan}
\define@key{fams}{bls}{nan}
\define@key{fams}{bdn}{nan}
\define@key{fams}{bcn}{nan}
\define@key{fams}{bcp}{nan}
\define@key{fams}{mhp}{nan}
\define@key{fams}{bgx}{nan}
\define@key{fams}{biz}{nan}
\define@key{fams}{bqo}{nan}
\define@key{fams}{blq}{nan}
\define@key{fams}{bog}{nan}
\define@key{fams}{bbq}{nan}
\define@key{fams}{myf}{nan}
\define@key{fams}{bmo}{nan}
\define@key{fams}{bce}{nan}
\define@key{fams}{bqt}{nan}
\define@key{fams}{bvm}{nan}
\define@key{fams}{bcf}{nan}
\define@key{fams}{bmg}{nan}
\define@key{fams}{bjx}{nan}
\define@key{fams}{byz}{nan}
\define@key{fams}{bqj}{nan}
\define@key{fams}{bqk}{nan}
\define@key{fams}{bpd}{nan}
\define@key{fams}{bfl}{nan}
\define@key{fams}{yaj}{nan}
\define@key{fams}{bpq}{nan}
\define@key{fams}{bnd}{nan}
\define@key{fams}{bbe}{nan}
\define@key{fams}{bgf}{nan}
\define@key{fams}{bsj}{nan}
\define@key{fams}{bnx}{nan}
\define@key{fams}{bxg}{nan}
\define@key{fams}{bgj}{nan}
\define@key{fams}{mfb}{nan}
\define@key{fams}{bjn}{nan}
\define@key{fams}{bfk}{nan}
\define@key{fams}{bxw}{nan}
\define@key{fams}{dbw}{nan}
\define@key{fams}{bap}{nan}
\define@key{fams}{bno}{nan}
\define@key{fams}{bfx}{nan}
\define@key{fams}{brd}{nan}
\define@key{fams}{bbg}{nan}
\define@key{fams}{baj}{nan}
\define@key{fams}{bhr}{nan}
\define@key{fams}{brs}{nan}
\define@key{fams}{brp}{nan}
\define@key{fams}{bmz}{nan}
\define@key{fams}{bpb}{nan}
\define@key{fams}{gry}{nan}
\define@key{fams}{bva}{nan}
\define@key{fams}{bxo}{nan}
\define@key{fams}{bch}{nan}
\define@key{fams}{bjc}{nan}
\define@key{fams}{jbk}{nan}
\define@key{fams}{bbi}{nan}
\define@key{fams}{bjk}{nan}
\define@key{fams}{bpt}{nan}
\define@key{fams}{tbn}{nan}
\define@key{fams}{bjz}{nan}
\define@key{fams}{bwg}{nan}
\define@key{fams}{bjf}{nan}
\define@key{fams}{bsl}{nan}
\define@key{fams}{buj}{nan}
\define@key{fams}{bzw}{nan}
\define@key{fams}{bdb}{nan}
\define@key{fams}{byq}{nan}
\define@key{fams}{bsg}{nan}
\define@key{fams}{bst}{nan}
\define@key{fams}{bsr}{nan}
\define@key{fams}{bsi}{nan}
\define@key{fams}{bnm}{nan}
\define@key{fams}{bts}{nan}
\define@key{fams}{akb}{nan}
\define@key{fams}{btm}{nan}
\define@key{fams}{btd}{nan}
\define@key{fams}{ayt}{nan}
\define@key{fams}{bta}{nan}
\define@key{fams}{btv}{nan}
\define@key{fams}{btq}{nan}
\define@key{fams}{btc}{nan}
\define@key{fams}{bvt}{nan}
\define@key{fams}{btu}{nan}
\define@key{fams}{bay}{nan}
\define@key{fams}{zbt}{nan}
\define@key{fams}{sne}{nan}
\define@key{fams}{bsf}{nan}
\define@key{fams}{bge}{nan}
\define@key{fams}{bxa}{nan}
\define@key{fams}{bwk}{nan}
\define@key{fams}{bjy}{nan}
\define@key{fams}{bvy}{nan}
\define@key{fams}{byg}{nan}
\define@key{fams}{mkq}{nan}
\define@key{fams}{bda}{nan}
\define@key{fams}{byl}{nan}
\define@key{fams}{bfr}{nan}
\define@key{fams}{beo}{nan}
\define@key{fams}{bea}{nan}
\define@key{fams}{bfp}{nan}
\define@key{fams}{beb}{nan}
\define@key{fams}{bzv}{nan}
\define@key{fams}{bek}{nan}
\define@key{fams}{bxp}{nan}
\define@key{fams}{tnr}{nan}
\define@key{fams}{bjv}{nan}
\define@key{fams}{bed}{nan}
\define@key{fams}{bkf}{nan}
\define@key{fams}{bxq}{nan}
\define@key{fams}{bnz}{nan}
\define@key{fams}{bby}{nan}
\define@key{fams}{bqv}{nan}
\define@key{fams}{bei}{nan}
\define@key{fams}{bkv}{nan}
\define@key{fams}{bkw}{nan}
\define@key{fams}{bvi}{nan}
\define@key{fams}{bxb}{nan}
\define@key{fams}{beg}{nan}
\define@key{fams}{blm}{nan}
\define@key{fams}{bey}{nan}
\define@key{fams}{bzj}{nan}
\define@key{fams}{brw}{nan}
\define@key{fams}{glb}{nan}
\define@key{fams}{bmb}{nan}
\define@key{fams}{yun}{nan}
\define@key{fams}{bez}{nan}
\define@key{fams}{bdp}{nan}
\define@key{fams}{bct}{nan}
\define@key{fams}{bgy}{nan}
\define@key{fams}{bnu}{nan}
\define@key{fams}{dbt}{nan}
\define@key{fams}{byd}{nan}
\define@key{fams}{bie}{nan}
\define@key{fams}{bxv}{nan}
\define@key{fams}{bve}{nan}
\define@key{fams}{bit}{nan}
\define@key{fams}{byt}{nan}
\define@key{fams}{bes}{nan}
\define@key{fams}{bep}{nan}
\define@key{fams}{bfe}{nan}
\define@key{fams}{byf}{nan}
\define@key{fams}{btt}{nan}
\define@key{fams}{eot}{nan}
\define@key{fams}{bhd}{nan}
\define@key{fams}{bha}{nan}
\define@key{fams}{bht}{nan}
\define@key{fams}{bgw}{nan}
\define@key{fams}{bhe}{nan}
\define@key{fams}{bhy}{nan}
\define@key{fams}{bhi}{nan}
\define@key{fams}{nes}{nan}
\define@key{fams}{bhu}{nan}
\define@key{fams}{bdf}{nan}
\define@key{fams}{beh}{nan}
\define@key{fams}{bpv}{nan}
\define@key{fams}{big}{nan}
\define@key{fams}{byk}{nan}
\define@key{fams}{bje}{nan}
\define@key{fams}{bmt}{nan}
\define@key{fams}{bym}{nan}
\define@key{fams}{bjg}{nan}
\define@key{fams}{bmc}{nan}
\define@key{fams}{bnk}{nan}
\define@key{fams}{brj}{nan}
\define@key{fams}{biu}{nan}
\define@key{fams}{xbe}{nan}
\define@key{fams}{bhc}{nan}
\define@key{fams}{ibh}{nan}
\define@key{fams}{jbm}{nan}
\define@key{fams}{bix}{nan}
\define@key{fams}{byb}{nan}
\define@key{fams}{kfs}{nan}
\define@key{fams}{bql}{nan}
\define@key{fams}{brz}{nan}
\define@key{fams}{bpz}{nan}
\define@key{fams}{bil}{nan}
\define@key{fams}{bms}{nan}
\define@key{fams}{bxf}{nan}
\define@key{fams}{bhl}{nan}
\define@key{fams}{byj}{nan}
\define@key{fams}{bmn}{nan}
\define@key{fams}{bxz}{nan}
\define@key{fams}{bon}{nan}
\define@key{fams}{bpj}{nan}
\define@key{fams}{itb}{nan}
\define@key{fams}{bne}{nan}
\define@key{fams}{bny}{nan}
\define@key{fams}{biq}{nan}
\define@key{fams}{bxe}{nan}
\define@key{fams}{brr}{nan}
\define@key{fams}{btf}{nan}
\define@key{fams}{biy}{nan}
\define@key{fams}{bqq}{nan}
\define@key{fams}{brk}{nan}
\define@key{fams}{brl}{nan}
\define@key{fams}{ije}{nan}
\define@key{fams}{bpy}{nan}
\define@key{fams}{bwh}{nan}
\define@key{fams}{bnw}{nan}
\define@key{fams}{bir}{nan}
\define@key{fams}{bzi}{nan}
\define@key{fams}{brt}{nan}
\define@key{fams}{bgk}{nan}
\define@key{fams}{mcc}{nan}
\define@key{fams}{bwm}{nan}
\define@key{fams}{byo}{nan}
\define@key{fams}{bpm}{nan}
\define@key{fams}{blp}{nan}
\define@key{fams}{bfh}{nan}
\define@key{fams}{beu}{nan}
\define@key{fams}{blr}{nan}
\define@key{fams}{zbl}{nan}
\define@key{fams}{bzn}{nan}
\define@key{fams}{bzl}{nan}
\define@key{fams}{bty}{nan}
\define@key{fams}{bgb}{nan}
\define@key{fams}{bdv}{nan}
\define@key{fams}{boy}{nan}
\define@key{fams}{bff}{nan}
\define@key{fams}{boq}{nan}
\define@key{fams}{bvw}{nan}
\define@key{fams}{bux}{nan}
\define@key{fams}{bqu}{nan}
\define@key{fams}{bhn}{nan}
\define@key{fams}{ybk}{nan}
\define@key{fams}{bdt}{nan}
\define@key{fams}{bkp}{nan}
\define@key{fams}{bus}{nan}
\define@key{fams}{bky}{nan}
\define@key{fams}{bnp}{nan}
\define@key{fams}{bld}{nan}
\define@key{fams}{xbo}{nan}
\define@key{fams}{bvo}{nan}
\define@key{fams}{bvl}{nan}
\define@key{fams}{smk}{nan}
\define@key{fams}{blv}{nan}
\define@key{fams}{bkt}{nan}
\define@key{fams}{bzm}{nan}
\define@key{fams}{bof}{nan}
\define@key{fams}{blj}{nan}
\define@key{fams}{ply}{nan}
\define@key{fams}{boh}{nan}
\define@key{fams}{bml}{nan}
\define@key{fams}{bws}{nan}
\define@key{fams}{zmx}{nan}
\define@key{fams}{bmf}{nan}
\define@key{fams}{bmq}{nan}
\define@key{fams}{bmw}{nan}
\define@key{fams}{kzc}{nan}
\define@key{fams}{bou}{nan}
\define@key{fams}{dbu}{nan}
\define@key{fams}{bna}{nan}
\define@key{fams}{bnv}{nan}
\define@key{fams}{glc}{nan}
\define@key{fams}{bui}{nan}
\define@key{fams}{bpg}{nan}
\define@key{fams}{bok}{nan}
\define@key{fams}{bvg}{nan}
\define@key{fams}{bop}{nan}
\define@key{fams}{bnb}{nan}
\define@key{fams}{bnl}{nan}
\define@key{fams}{bvf}{nan}
\define@key{fams}{bpw}{nan}
\define@key{fams}{gai}{nan}
\define@key{fams}{fue}{nan}
\define@key{fams}{ksr}{nan}
\define@key{fams}{xxb}{nan}
\define@key{fams}{mae}{nan}
\define@key{fams}{bwf}{nan}
\define@key{fams}{bqs}{nan}
\define@key{fams}{bmj}{nan}
\define@key{fams}{bph}{nan}
\define@key{fams}{sbl}{nan}
\define@key{fams}{nku}{nan}
\define@key{fams}{mux}{nan}
\define@key{fams}{suo}{nan}
\define@key{fams}{kxr}{nan}
\define@key{fams}{aof}{nan}
\define@key{fams}{bra}{nan}
\define@key{fams}{kvl}{nan}
\define@key{fams}{buq}{nan}
\define@key{fams}{brq}{nan}
\define@key{fams}{rib}{nan}
\define@key{fams}{bzt}{nan}
\define@key{fams}{sgt}{nan}
\define@key{fams}{bro}{nan}
\define@key{fams}{bpl}{nan}
\define@key{fams}{plw}{nan}
\define@key{fams}{kxd}{nan}
\define@key{fams}{bsb}{nan}
\define@key{fams}{rnb}{nan}
\define@key{fams}{bub}{nan}
\define@key{fams}{cbl}{nan}
\define@key{fams}{box}{nan}
\define@key{fams}{buw}{nan}
\define@key{fams}{stt}{nan}
\define@key{fams}{btp}{nan}
\define@key{fams}{bdx}{nan}
\define@key{fams}{bja}{nan}
\define@key{fams}{bbh}{nan}
\define@key{fams}{buk}{nan}
\define@key{fams}{bgt}{nan}
\define@key{fams}{bku}{nan}
\define@key{fams}{bxh}{nan}
\define@key{fams}{byh}{nan}
\define@key{fams}{bvk}{nan}
\define@key{fams}{bhh}{nan}
\define@key{fams}{bvu}{nan}
\define@key{fams}{bkn}{nan}
\define@key{fams}{tkb}{nan}
\define@key{fams}{buz}{nan}
\define@key{fams}{bqn}{nan}
\define@key{fams}{bmp}{nan}
\define@key{fams}{buy}{nan}
\define@key{fams}{sti}{nan}
\define@key{fams}{bjl}{nan}
\define@key{fams}{byp}{nan}
\define@key{fams}{aon}{nan}
\define@key{fams}{bmv}{nan}
\define@key{fams}{kjz}{nan}
\define@key{fams}{bwx}{nan}
\define@key{fams}{bdd}{nan}
\define@key{fams}{bvn}{nan}
\define@key{fams}{bfn}{nan}
\define@key{fams}{bns}{nan}
\define@key{fams}{bqd}{nan}
\define@key{fams}{xbg}{nan}
\define@key{fams}{wun}{nan}
\define@key{fams}{bkz}{nan}
\define@key{fams}{but}{nan}
\define@key{fams}{buv}{nan}
\define@key{fams}{dgb}{nan}
\define@key{fams}{bnn}{nan}
\define@key{fams}{blf}{nan}
\define@key{fams}{bys}{nan}
\define@key{fams}{bti}{nan}
\define@key{fams}{bxn}{nan}
\define@key{fams}{bvh}{nan}
\define@key{fams}{pyx}{nan}
\define@key{fams}{vrt}{nan}
\define@key{fams}{bzu}{nan}
\define@key{fams}{bqw}{nan}
\define@key{fams}{bdi}{nan}
\define@key{fams}{bqr}{nan}
\define@key{fams}{aip}{nan}
\define@key{fams}{asi}{nan}
\define@key{fams}{bry}{nan}
\define@key{fams}{bxs}{nan}
\define@key{fams}{bsm}{nan}
\define@key{fams}{bfg}{nan}
\define@key{fams}{buc}{nan}
\define@key{fams}{bup}{nan}
\define@key{fams}{dox}{nan}
\define@key{fams}{bju}{nan}
\define@key{fams}{kyb}{nan}
\define@key{fams}{bnr}{nan}
\define@key{fams}{btw}{nan}
\define@key{fams}{jid}{nan}
\define@key{fams}{bhs}{nan}
\define@key{fams}{jiy}{nan}
\define@key{fams}{byi}{nan}
\define@key{fams}{bww}{nan}
\define@key{fams}{bwd}{nan}
\define@key{fams}{tte}{nan}
\define@key{fams}{bwa}{nan}
\define@key{fams}{bwl}{nan}
\define@key{fams}{bwc}{nan}
\define@key{fams}{bwz}{nan}
\define@key{fams}{mkk}{nan}
\define@key{fams}{msq}{nan}
\define@key{fams}{cbb}{nan}
\define@key{fams}{ccr}{nan}
\define@key{fams}{miu}{nan}
\define@key{fams}{roc}{nan}
\define@key{fams}{ccd}{nan}
\define@key{fams}{cah}{nan}
\define@key{fams}{qvl}{nan}
\define@key{fams}{zad}{nan}
\define@key{fams}{frc}{nan}
\define@key{fams}{ckx}{nan}
\define@key{fams}{ckz}{nan}
\define@key{fams}{cky}{nan}
\define@key{fams}{tbk}{nan}
\define@key{fams}{qud}{nan}
\define@key{fams}{caw}{nan}
\define@key{fams}{rmq}{nan}
\define@key{fams}{clu}{nan}
\define@key{fams}{abd}{nan}
\define@key{fams}{csx}{nan}
\define@key{fams}{mcu}{nan}
\define@key{fams}{wes}{nan}
\define@key{fams}{cml}{nan}
\define@key{fams}{cmt}{nan}
\define@key{fams}{xcc}{nan}
\define@key{fams}{qxr}{nan}
\define@key{fams}{caz}{nan}
\define@key{fams}{mlc}{nan}
\define@key{fams}{cov}{nan}
\define@key{fams}{cps}{nan}
\define@key{fams}{cpg}{nan}
\define@key{fams}{cot}{nan}
\define@key{fams}{cby}{nan}
\define@key{fams}{cfd}{nan}
\define@key{fams}{crf}{nan}
\define@key{fams}{xcr}{nan}
\define@key{fams}{hns}{nan}
\define@key{fams}{jvn}{nan}
\define@key{fams}{crr}{nan}
\define@key{fams}{rmc}{nan}
\define@key{fams}{asc}{nan}
\define@key{fams}{csc}{nan}
\define@key{fams}{xcy}{nan}
\define@key{fams}{xce}{nan}
\define@key{fams}{cen}{nan}
\define@key{fams}{hmm}{nan}
\define@key{fams}{cmo}{nan}
\define@key{fams}{zch}{nan}
\define@key{fams}{hmc}{nan}
\define@key{fams}{fuq}{nan}
\define@key{fams}{grv}{nan}
\define@key{fams}{cet}{nan}
\define@key{fams}{pse}{nan}
\define@key{fams}{mwo}{nan}
\define@key{fams}{mxz}{nan}
\define@key{fams}{syb}{nan}
\define@key{fams}{tgt}{nan}
\define@key{fams}{plc}{nan}
\define@key{fams}{sml}{nan}
\define@key{fams}{zbc}{nan}
\define@key{fams}{dtp}{nan}
\define@key{fams}{awu}{nan}
\define@key{fams}{ncx}{nan}
\define@key{fams}{nch}{nan}
\define@key{fams}{ojc}{nan}
\define@key{fams}{pbs}{nan}
\define@key{fams}{quk}{nan}
\define@key{fams}{cds}{nan}
\define@key{fams}{cdy}{nan}
\define@key{fams}{chg}{nan}
\define@key{fams}{ciy}{nan}
\define@key{fams}{ccp}{nan}
\define@key{fams}{ckh}{nan}
\define@key{fams}{cli}{nan}
\define@key{fams}{tgf}{nan}
\define@key{fams}{cll}{nan}
\define@key{fams}{cdh}{nan}
\define@key{fams}{ceg}{nan}
\define@key{fams}{ccc}{nan}
\define@key{fams}{cna}{nan}
\define@key{fams}{cga}{nan}
\define@key{fams}{cra}{nan}
\define@key{fams}{crv}{nan}
\define@key{fams}{xtb}{nan}
\define@key{fams}{ruk}{nan}
\define@key{fams}{cde}{nan}
\define@key{fams}{cjn}{nan}
\define@key{fams}{cnu}{nan}
\define@key{fams}{ycp}{nan}
\define@key{fams}{cpn}{nan}
\define@key{fams}{ych}{nan}
\define@key{fams}{cwg}{nan}
\define@key{fams}{hne}{nan}
\define@key{fams}{ctn}{nan}
\define@key{fams}{cur}{nan}
\define@key{fams}{csd}{nan}
\define@key{fams}{cip}{nan}
\define@key{fams}{zpv}{nan}
\define@key{fams}{mii}{nan}
\define@key{fams}{csg}{nan}
\define@key{fams}{clh}{nan}
\define@key{fams}{clc}{nan}
\define@key{fams}{csa}{nan}
\define@key{fams}{cpi}{nan}
\define@key{fams}{chn}{nan}
\define@key{fams}{cih}{nan}
\define@key{fams}{bxu}{nan}
\define@key{fams}{cnb}{nan}
\define@key{fams}{qxc}{nan}
\define@key{fams}{cdf}{nan}
\define@key{fams}{nhd}{nan}
\define@key{fams}{the}{nan}
\define@key{fams}{cik}{nan}
\define@key{fams}{zpc}{nan}
\define@key{fams}{cgk}{nan}
\define@key{fams}{cdi}{nan}
\define@key{fams}{nri}{nan}
\define@key{fams}{cjk}{nan}
\define@key{fams}{cda}{nan}
\define@key{fams}{coh}{nan}
\define@key{fams}{cce}{nan}
\define@key{fams}{nct}{nan}
\define@key{fams}{cvg}{nan}
\define@key{fams}{cuw}{nan}
\define@key{fams}{cuh}{nan}
\define@key{fams}{chu}{nan}
\define@key{fams}{cdj}{nan}
\define@key{fams}{scb}{nan}
\define@key{fams}{xcv}{nan}
\define@key{fams}{chw}{nan}
\define@key{fams}{cia}{nan}
\define@key{fams}{ckl}{nan}
\define@key{fams}{awc}{nan}
\define@key{fams}{cib}{nan}
\define@key{fams}{cim}{nan}
\define@key{fams}{mkx}{nan}
\define@key{fams}{cdr}{nan}
\define@key{fams}{cie}{nan}
\define@key{fams}{cin}{nan}
\define@key{fams}{xcg}{nan}
\define@key{fams}{asg}{nan}
\define@key{fams}{txt}{nan}
\define@key{fams}{tgd}{nan}
\define@key{fams}{xcl}{nan}
\define@key{fams}{nci}{nan}
\define@key{fams}{qwc}{nan}
\define@key{fams}{syc}{nan}
\define@key{fams}{myz}{nan}
\define@key{fams}{xct}{nan}
\define@key{fams}{dri}{nan}
\define@key{fams}{naz}{nan}
\define@key{fams}{zps}{nan}
\define@key{fams}{zca}{nan}
\define@key{fams}{coj}{nan}
\define@key{fams}{coa}{nan}
\define@key{fams}{liw}{nan}
\define@key{fams}{csn}{nan}
\define@key{fams}{gct}{nan}
\define@key{fams}{cfg}{nan}
\define@key{fams}{swc}{nan}
\define@key{fams}{cnc}{nan}
\define@key{fams}{coq}{nan}
\define@key{fams}{cry}{nan}
\define@key{fams}{qwa}{nan}
\define@key{fams}{xxr}{nan}
\define@key{fams}{cos}{nan}
\define@key{fams}{csr}{nan}
\define@key{fams}{mta}{nan}
\define@key{fams}{xcn}{nan}
\define@key{fams}{cow}{nan}
\define@key{fams}{toc}{nan}
\define@key{fams}{gyn}{nan}
\define@key{fams}{csq}{nan}
\define@key{fams}{mfn}{nan}
\define@key{fams}{crz}{nan}
\define@key{fams}{csf}{nan}
\define@key{fams}{cbq}{nan}
\define@key{fams}{cuo}{nan}
\define@key{fams}{xlu}{nan}
\define@key{fams}{cnq}{nan}
\define@key{fams}{cuq}{nan}
\define@key{fams}{ccl}{nan}
\define@key{fams}{cuv}{nan}
\define@key{fams}{xtu}{nan}
\define@key{fams}{cyo}{nan}
\define@key{fams}{bwy}{nan}
\define@key{fams}{cse}{nan}
\define@key{fams}{dao}{nan}
\define@key{fams}{lni}{nan}
\define@key{fams}{dtn}{nan}
\define@key{fams}{dbr}{nan}
\define@key{fams}{dbe}{nan}
\define@key{fams}{xdc}{nan}
\define@key{fams}{dbd}{nan}
\define@key{fams}{dgd}{nan}
\define@key{fams}{dgk}{nan}
\define@key{fams}{dec}{nan}
\define@key{fams}{dgn}{nan}
\define@key{fams}{dlk}{nan}
\define@key{fams}{das}{nan}
\define@key{fams}{dij}{nan}
\define@key{fams}{drb}{nan}
\define@key{fams}{zhd}{nan}
\define@key{fams}{bpa}{nan}
\define@key{fams}{dkk}{nan}
\define@key{fams}{dka}{nan}
\define@key{fams}{qer}{nan}
\define@key{fams}{dlm}{nan}
\define@key{fams}{dmm}{nan}
\define@key{fams}{dam}{nan}
\define@key{fams}{uhn}{nan}
\define@key{fams}{idb}{nan}
\define@key{fams}{dac}{nan}
\define@key{fams}{dml}{nan}
\define@key{fams}{dms}{nan}
\define@key{fams}{dnu}{nan}
\define@key{fams}{dnr}{nan}
\define@key{fams}{daq}{nan}
\define@key{fams}{thl}{nan}
\define@key{fams}{dsl}{nan}
\define@key{fams}{daf}{nan}
\define@key{fams}{aso}{nan}
\define@key{fams}{gku}{nan}
\define@key{fams}{dnd}{nan}
\define@key{fams}{daz}{nan}
\define@key{fams}{djc}{nan}
\define@key{fams}{dln}{nan}
\define@key{fams}{dro}{nan}
\define@key{fams}{dot}{nan}
\define@key{fams}{daw}{nan}
\define@key{fams}{dww}{nan}
\define@key{fams}{ddw}{nan}
\define@key{fams}{dax}{nan}
\define@key{fams}{dzg}{nan}
\define@key{fams}{dzd}{nan}
\define@key{fams}{ded}{nan}
\define@key{fams}{gbh}{nan}
\define@key{fams}{dge}{nan}
\define@key{fams}{mzw}{nan}
\define@key{fams}{deh}{nan}
\define@key{fams}{dek}{nan}
\define@key{fams}{row}{nan}
\define@key{fams}{ntr}{nan}
\define@key{fams}{dmx}{nan}
\define@key{fams}{dei}{nan}
\define@key{fams}{dem}{nan}
\define@key{fams}{dmy}{nan}
\define@key{fams}{deq}{nan}
\define@key{fams}{ddn}{nan}
\define@key{fams}{dez}{nan}
\define@key{fams}{dnk}{nan}
\define@key{fams}{dbb}{nan}
\define@key{fams}{anv}{nan}
\define@key{fams}{dee}{nan}
\define@key{fams}{def}{nan}
\define@key{fams}{dgh}{nan}
\define@key{fams}{dhs}{nan}
\define@key{fams}{dhn}{nan}
\define@key{fams}{dwz}{nan}
\define@key{fams}{nfa}{nan}
\define@key{fams}{mki}{nan}
\define@key{fams}{dho}{nan}
\define@key{fams}{adf}{nan}
\define@key{fams}{ddr}{nan}
\define@key{fams}{dhd}{nan}
\define@key{fams}{dia}{nan}
\define@key{fams}{mbd}{nan}
\define@key{fams}{dby}{nan}
\define@key{fams}{dio}{nan}
\define@key{fams}{duy}{nan}
\define@key{fams}{dig}{nan}
\define@key{fams}{cfa}{nan}
\define@key{fams}{dil}{nan}
\define@key{fams}{jma}{nan}
\define@key{fams}{dii}{nan}
\define@key{fams}{dmc}{nan}
\define@key{fams}{ddi}{nan}
\define@key{fams}{gdl}{nan}
\define@key{fams}{diu}{nan}
\define@key{fams}{dir}{nan}
\define@key{fams}{dwa}{nan}
\define@key{fams}{dsi}{nan}
\define@key{fams}{tbz}{nan}
\define@key{fams}{diy}{nan}
\define@key{fams}{xtd}{nan}
\define@key{fams}{dix}{nan}
\define@key{fams}{djf}{nan}
\define@key{fams}{djn}{nan}
\define@key{fams}{djw}{nan}
\define@key{fams}{djb}{nan}
\define@key{fams}{dze}{nan}
\define@key{fams}{dob}{nan}
\define@key{fams}{doe}{nan}
\define@key{fams}{dgg}{nan}
\define@key{fams}{dgx}{nan}
\define@key{fams}{dgs}{nan}
\define@key{fams}{dos}{nan}
\define@key{fams}{dgr}{nan}
\define@key{fams}{dbg}{nan}
\define@key{fams}{dbi}{nan}
\define@key{fams}{uya}{nan}
\define@key{fams}{dre}{nan}
\define@key{fams}{dov}{nan}
\define@key{fams}{doq}{nan}
\define@key{fams}{doa}{nan}
\define@key{fams}{doy}{nan}
\define@key{fams}{dof}{nan}
\define@key{fams}{dev}{nan}
\define@key{fams}{dok}{nan}
\define@key{fams}{yik}{nan}
\define@key{fams}{doh}{nan}
\define@key{fams}{ddd}{nan}
\define@key{fams}{dde}{nan}
\define@key{fams}{dor}{nan}
\define@key{fams}{kqc}{nan}
\define@key{fams}{doz}{nan}
\define@key{fams}{dol}{nan}
\define@key{fams}{dty}{nan}
\define@key{fams}{dup}{nan}
\define@key{fams}{dva}{nan}
\define@key{fams}{dub}{nan}
\define@key{fams}{dmu}{nan}
\define@key{fams}{duk}{nan}
\define@key{fams}{ndu}{nan}
\define@key{fams}{dbm}{nan}
\define@key{fams}{dme}{nan}
\define@key{fams}{kbz}{nan}
\define@key{fams}{nke}{nan}
\define@key{fams}{dbo}{nan}
\define@key{fams}{duz}{nan}
\define@key{fams}{dmv}{nan}
\define@key{fams}{wtf}{nan}
\define@key{fams}{dui}{nan}
\define@key{fams}{duh}{nan}
\define@key{fams}{raa}{nan}
\define@key{fams}{dng}{nan}
\define@key{fams}{dbv}{nan}
\define@key{fams}{drq}{nan}
\define@key{fams}{mvp}{nan}
\define@key{fams}{dbn}{nan}
\define@key{fams}{dug}{nan}
\define@key{fams}{dsn}{nan}
\define@key{fams}{duw}{nan}
\define@key{fams}{duq}{nan}
\define@key{fams}{dun}{nan}
\define@key{fams}{dws}{nan}
\define@key{fams}{dux}{nan}
\define@key{fams}{dae}{nan}
\define@key{fams}{duv}{nan}
\define@key{fams}{dbp}{nan}
\define@key{fams}{gve}{nan}
\define@key{fams}{nnu}{nan}
\define@key{fams}{dyb}{nan}
\define@key{fams}{dyn}{nan}
\define@key{fams}{dya}{nan}
\define@key{fams}{dyd}{nan}
\define@key{fams}{jen}{nan}
\define@key{fams}{dzl}{nan}
\define@key{fams}{dzn}{nan}
\define@key{fams}{bpn}{nan}
\define@key{fams}{add}{nan}
\define@key{fams}{dzo}{nan}
\define@key{fams}{dnn}{nan}
\define@key{fams}{ktv}{nan}
\define@key{fams}{bgp}{nan}
\define@key{fams}{lwl}{nan}
\define@key{fams}{mng}{nan}
\define@key{fams}{emu}{nan}
\define@key{fams}{tge}{nan}
\define@key{fams}{nos}{nan}
\define@key{fams}{emq}{nan}
\define@key{fams}{kif}{nan}
\define@key{fams}{emg}{nan}
\define@key{fams}{zeh}{nan}
\define@key{fams}{hmq}{nan}
\define@key{fams}{muq}{nan}
\define@key{fams}{hme}{nan}
\define@key{fams}{lma}{nan}
\define@key{fams}{gbx}{nan}
\define@key{fams}{xrb}{nan}
\define@key{fams}{acp}{nan}
\define@key{fams}{nle}{nan}
\define@key{fams}{kqo}{nan}
\define@key{fams}{vme}{nan}
\define@key{fams}{tre}{nan}
\define@key{fams}{dmr}{nan}
\define@key{fams}{bnj}{nan}
\define@key{fams}{pez}{nan}
\define@key{fams}{zbe}{nan}
\define@key{fams}{kjs}{nan}
\define@key{fams}{nhe}{nan}
\define@key{fams}{ojg}{nan}
\define@key{fams}{aaq}{nan}
\define@key{fams}{qve}{nan}
\define@key{fams}{cly}{nan}
\define@key{fams}{avl}{nan}
\define@key{fams}{sfe}{nan}
\define@key{fams}{azd}{nan}
\define@key{fams}{yit}{nan}
\define@key{fams}{cek}{nan}
\define@key{fams}{yol}{nan}
\define@key{fams}{xeb}{nan}
\define@key{fams}{ebr}{nan}
\define@key{fams}{ebg}{nan}
\define@key{fams}{ecs}{nan}
\define@key{fams}{cbj}{nan}
\define@key{fams}{idd}{nan}
\define@key{fams}{ijj}{nan}
\define@key{fams}{ica}{nan}
\define@key{fams}{nqg}{nan}
\define@key{fams}{awy}{nan}
\define@key{fams}{dbf}{nan}
\define@key{fams}{eee}{nan}
\define@key{fams}{efa}{nan}
\define@key{fams}{efe}{nan}
\define@key{fams}{ofu}{nan}
\define@key{fams}{ego}{nan}
\define@key{fams}{esl}{nan}
\define@key{fams}{egy}{nan}
\define@key{fams}{ehu}{nan}
\define@key{fams}{eit}{nan}
\define@key{fams}{eja}{nan}
\define@key{fams}{eka}{nan}
\define@key{fams}{eki}{nan}
\define@key{fams}{eke}{nan}
\define@key{fams}{ekp}{nan}
\define@key{fams}{zpp}{nan}
\define@key{fams}{elx}{nan}
\define@key{fams}{elm}{nan}
\define@key{fams}{ele}{nan}
\define@key{fams}{elh}{nan}
\define@key{fams}{ekm}{nan}
\define@key{fams}{elk}{nan}
\define@key{fams}{elo}{nan}
\define@key{fams}{zte}{nan}
\define@key{fams}{afo}{nan}
\define@key{fams}{elu}{nan}
\define@key{fams}{xly}{nan}
\define@key{fams}{yzg}{nan}
\define@key{fams}{emn}{nan}
\define@key{fams}{bdc}{nan}
\define@key{fams}{tdc}{nan}
\define@key{fams}{ebu}{nan}
\define@key{fams}{emw}{nan}
\define@key{fams}{enr}{nan}
\define@key{fams}{unk}{nan}
\define@key{fams}{end}{nan}
\define@key{fams}{enc}{nan}
\define@key{fams}{ptt}{nan}
\define@key{fams}{enu}{nan}
\define@key{fams}{enw}{nan}
\define@key{fams}{env}{nan}
\define@key{fams}{epi}{nan}
\define@key{fams}{emy}{nan}
\define@key{fams}{era}{nan}
\define@key{fams}{kjy}{nan}
\define@key{fams}{twp}{nan}
\define@key{fams}{ert}{nan}
\define@key{fams}{erw}{nan}
\define@key{fams}{err}{nan}
\define@key{fams}{emx}{nan}
\define@key{fams}{ers}{nan}
\define@key{fams}{erh}{nan}
\define@key{fams}{ish}{nan}
\define@key{fams}{mcq}{nan}
\define@key{fams}{esh}{nan}
\define@key{fams}{ags}{nan}
\define@key{fams}{esy}{nan}
\define@key{fams}{epo}{nan}
\define@key{fams}{ots}{nan}
\define@key{fams}{eso}{nan}
\define@key{fams}{esm}{nan}
\define@key{fams}{etb}{nan}
\define@key{fams}{etx}{nan}
\define@key{fams}{ecr}{nan}
\define@key{fams}{ecy}{nan}
\define@key{fams}{eth}{nan}
\define@key{fams}{ich}{nan}
\define@key{fams}{eto}{nan}
\define@key{fams}{etn}{nan}
\define@key{fams}{ett}{nan}
\define@key{fams}{utr}{nan}
\define@key{fams}{bzz}{nan}
\define@key{fams}{gev}{nan}
\define@key{fams}{nou}{nan}
\define@key{fams}{ext}{nan}
\define@key{fams}{fab}{nan}
\define@key{fams}{faf}{nan}
\define@key{fams}{fif}{nan}
\define@key{fams}{azt}{nan}
\define@key{fams}{faj}{nan}
\define@key{fams}{fai}{nan}
\define@key{fams}{fax}{nan}
\define@key{fams}{cfm}{nan}
\define@key{fams}{fli}{nan}
\define@key{fams}{xfa}{nan}
\define@key{fams}{fam}{nan}
\define@key{fams}{fng}{nan}
\define@key{fams}{fan}{nan}
\define@key{fams}{fak}{nan}
\define@key{fams}{fni}{nan}
\define@key{fams}{nsf}{nan}
\define@key{fams}{fmu}{nan}
\define@key{fams}{far}{nan}
\define@key{fams}{ddg}{nan}
\define@key{fams}{fau}{nan}
\define@key{fams}{agl}{nan}
\define@key{fams}{fpe}{nan}
\define@key{fams}{fer}{nan}
\define@key{fams}{hif}{nan}
\define@key{fams}{fil}{nan}
\define@key{fams}{tlp}{nan}
\define@key{fams}{bkb}{nan}
\define@key{fams}{fss}{nan}
\define@key{fams}{fag}{nan}
\define@key{fams}{fip}{nan}
\define@key{fams}{fir}{nan}
\define@key{fams}{fiw}{nan}
\define@key{fams}{fln}{nan}
\define@key{fams}{flh}{nan}
\define@key{fams}{fod}{nan}
\define@key{fams}{frq}{nan}
\define@key{fams}{enf}{nan}
\define@key{fams}{frt}{nan}
\define@key{fams}{frp}{nan}
\define@key{fams}{fur}{nan}
\define@key{fams}{flr}{nan}
\define@key{fams}{ula}{nan}
\define@key{fams}{fuy}{nan}
\define@key{fams}{fwe}{nan}
\define@key{fams}{fie}{nan}
\define@key{fams}{ttb}{nan}
\define@key{fams}{gie}{nan}
\define@key{fams}{gab}{nan}
\define@key{fams}{gdg}{nan}
\define@key{fams}{gdk}{nan}
\define@key{fams}{gbk}{nan}
\define@key{fams}{gad}{nan}
\define@key{fams}{gda}{nan}
\define@key{fams}{gdh}{nan}
\define@key{fams}{gft}{nan}
\define@key{fams}{btg}{nan}
\define@key{fams}{ggu}{nan}
\define@key{fams}{gbf}{nan}
\define@key{fams}{gic}{nan}
\define@key{fams}{gcn}{nan}
\define@key{fams}{xga}{nan}
\define@key{fams}{glo}{nan}
\define@key{fams}{gar}{nan}
\define@key{fams}{gce}{nan}
\define@key{fams}{sdn}{nan}
\define@key{fams}{gap}{nan}
\define@key{fams}{gal}{nan}
\define@key{fams}{kgj}{nan}
\define@key{fams}{gma}{nan}
\define@key{fams}{wof}{nan}
\define@key{fams}{gbl}{nan}
\define@key{fams}{gak}{nan}
\define@key{fams}{bte}{nan}
\define@key{fams}{ihw}{nan}
\define@key{fams}{gne}{nan}
\define@key{fams}{gnk}{nan}
\define@key{fams}{gnq}{nan}
\define@key{fams}{unn}{nan}
\define@key{fams}{gan}{nan}
\define@key{fams}{pgd}{nan}
\define@key{fams}{gzn}{nan}
\define@key{fams}{gnb}{nan}
\define@key{fams}{gnl}{nan}
\define@key{fams}{ggl}{nan}
\define@key{fams}{gao}{nan}
\define@key{fams}{gza}{nan}
\define@key{fams}{gnz}{nan}
\define@key{fams}{gga}{nan}
\define@key{fams}{gbm}{nan}
\define@key{fams}{ilg}{nan}
\define@key{fams}{gex}{nan}
\define@key{fams}{gaq}{nan}
\define@key{fams}{gou}{nan}
\define@key{fams}{gwt}{nan}
\define@key{fams}{gyl}{nan}
\define@key{fams}{gzi}{nan}
\define@key{fams}{gbg}{nan}
\define@key{fams}{gbv}{nan}
\define@key{fams}{gby}{nan}
\define@key{fams}{gyg}{nan}
\define@key{fams}{gbq}{nan}
\define@key{fams}{gbs}{nan}
\define@key{fams}{ggb}{nan}
\define@key{fams}{xgb}{nan}
\define@key{fams}{grh}{nan}
\define@key{fams}{gec}{nan}
\define@key{fams}{kvq}{nan}
\define@key{fams}{gei}{nan}
\define@key{fams}{gdd}{nan}
\define@key{fams}{drs}{nan}
\define@key{fams}{hmj}{nan}
\define@key{fams}{gez}{nan}
\define@key{fams}{ghk}{nan}
\define@key{fams}{giu}{nan}
\define@key{fams}{geq}{nan}
\define@key{fams}{gaf}{nan}
\define@key{fams}{gej}{nan}
\define@key{fams}{ygp}{nan}
\define@key{fams}{gew}{nan}
\define@key{fams}{gea}{nan}
\define@key{fams}{ges}{nan}
\define@key{fams}{gha}{nan}
\define@key{fams}{gse}{nan}
\define@key{fams}{ghn}{nan}
\define@key{fams}{gpe}{nan}
\define@key{fams}{gds}{nan}
\define@key{fams}{gri}{nan}
\define@key{fams}{ajs}{nan}
\define@key{fams}{bmk}{nan}
\define@key{fams}{aln}{nan}
\define@key{fams}{ghr}{nan}
\define@key{fams}{bbj}{nan}
\define@key{fams}{gho}{nan}
\define@key{fams}{bgi}{nan}
\define@key{fams}{gib}{nan}
\define@key{fams}{kks}{nan}
\define@key{fams}{acd}{nan}
\define@key{fams}{gix}{nan}
\define@key{fams}{gip}{nan}
\define@key{fams}{gim}{nan}
\define@key{fams}{kmp}{nan}
\define@key{fams}{gmn}{nan}
\define@key{fams}{gnm}{nan}
\define@key{fams}{ayg}{nan}
\define@key{fams}{bbr}{nan}
\define@key{fams}{gii}{nan}
\define@key{fams}{nyf}{nan}
\define@key{fams}{toh}{nan}
\define@key{fams}{ggt}{nan}
\define@key{fams}{giy}{nan}
\define@key{fams}{tof}{nan}
\define@key{fams}{glr}{nan}
\define@key{fams}{glw}{nan}
\define@key{fams}{oub}{nan}
\define@key{fams}{gnu}{nan}
\define@key{fams}{gom}{nan}
\define@key{fams}{gig}{nan}
\define@key{fams}{goi}{nan}
\define@key{fams}{gox}{nan}
\define@key{fams}{gdx}{nan}
\define@key{fams}{gof}{nan}
\define@key{fams}{gog}{nan}
\define@key{fams}{goo}{nan}
\define@key{fams}{goe}{nan}
\define@key{fams}{gjn}{nan}
\define@key{fams}{gov}{nan}
\define@key{fams}{goq}{nan}
\define@key{fams}{goc}{nan}
\define@key{fams}{grq}{nan}
\define@key{fams}{gqr}{nan}
\define@key{fams}{got}{nan}
\define@key{fams}{goy}{nan}
\define@key{fams}{gwf}{nan}
\define@key{fams}{goz}{nan}
\define@key{fams}{nli}{nan}
\define@key{fams}{giq}{nan}
\define@key{fams}{gcl}{nan}
\define@key{fams}{grs}{nan}
\define@key{fams}{gro}{nan}
\define@key{fams}{gos}{nan}
\define@key{fams}{ats}{nan}
\define@key{fams}{gwx}{nan}
\define@key{fams}{gvj}{nan}
\define@key{fams}{jiq}{nan}
\define@key{fams}{gnc}{nan}
\define@key{fams}{gyr}{nan}
\define@key{fams}{gsm}{nan}
\define@key{fams}{xgd}{nan}
\define@key{fams}{gdu}{nan}
\define@key{fams}{zpg}{nan}
\define@key{fams}{gdc}{nan}
\define@key{fams}{kkp}{nan}
\define@key{fams}{wrw}{nan}
\define@key{fams}{zgn}{nan}
\define@key{fams}{bet}{nan}
\define@key{fams}{ztu}{nan}
\define@key{fams}{gus}{nan}
\define@key{fams}{gkp}{nan}
\define@key{fams}{gqi}{nan}
\define@key{fams}{gvl}{nan}
\define@key{fams}{glu}{nan}
\define@key{fams}{gmb}{nan}
\define@key{fams}{gly}{nan}
\define@key{fams}{gul}{nan}
\define@key{fams}{gmu}{nan}
\define@key{fams}{gdi}{nan}
\define@key{fams}{gyf}{nan}
\define@key{fams}{rub}{nan}
\define@key{fams}{gnt}{nan}
\define@key{fams}{gpa}{nan}
\define@key{fams}{grz}{nan}
\define@key{fams}{gdj}{nan}
\define@key{fams}{ggg}{nan}
\define@key{fams}{grx}{nan}
\define@key{fams}{gjr}{nan}
\define@key{fams}{gvm}{nan}
\define@key{fams}{gvr}{nan}
\define@key{fams}{grd}{nan}
\define@key{fams}{gsn}{nan}
\define@key{fams}{gsl}{nan}
\define@key{fams}{xgw}{nan}
\define@key{fams}{gwu}{nan}
\define@key{fams}{gvy}{nan}
\define@key{fams}{gka}{nan}
\define@key{fams}{ngs}{nan}
\define@key{fams}{gwb}{nan}
\define@key{fams}{dah}{nan}
\define@key{fams}{bga}{nan}
\define@key{fams}{gwn}{nan}
\define@key{fams}{grw}{nan}
\define@key{fams}{gwe}{nan}
\define@key{fams}{gwr}{nan}
\define@key{fams}{gwj}{nan}
\define@key{fams}{gyi}{nan}
\define@key{fams}{gye}{nan}
\define@key{fams}{haq}{nan}
\define@key{fams}{hbu}{nan}
\define@key{fams}{hdy}{nan}
\define@key{fams}{hoj}{nan}
\define@key{fams}{xhd}{nan}
\define@key{fams}{ayh}{nan}
\define@key{fams}{aek}{nan}
\define@key{fams}{hah}{nan}
\define@key{fams}{hgw}{nan}
\define@key{fams}{bzx}{nan}
\define@key{fams}{hgm}{nan}
\define@key{fams}{haf}{nan}
\define@key{fams}{hvc}{nan}
\define@key{fams}{hji}{nan}
\define@key{fams}{haj}{nan}
\define@key{fams}{hao}{nan}
\define@key{fams}{hld}{nan}
\define@key{fams}{hmu}{nan}
\define@key{fams}{hba}{nan}
\define@key{fams}{hag}{nan}
\define@key{fams}{han}{nan}
\define@key{fams}{haa}{nan}
\define@key{fams}{hab}{nan}
\define@key{fams}{xiv}{nan}
\define@key{fams}{kjo}{nan}
\define@key{fams}{hro}{nan}
\define@key{fams}{hrk}{nan}
\define@key{fams}{bgc}{nan}
\define@key{fams}{hrz}{nan}
\define@key{fams}{ybj}{nan}
\define@key{fams}{xht}{nan}
\define@key{fams}{hsl}{nan}
\define@key{fams}{hvk}{nan}
\define@key{fams}{hav}{nan}
\define@key{fams}{hps}{nan}
\define@key{fams}{xda}{nan}
\define@key{fams}{haz}{nan}
\define@key{fams}{hbn}{nan}
\define@key{fams}{scp}{nan}
\define@key{fams}{heg}{nan}
\define@key{fams}{nix}{nan}
\define@key{fams}{hed}{nan}
\define@key{fams}{llf}{nan}
\define@key{fams}{hrt}{nan}
\define@key{fams}{ham}{nan}
\define@key{fams}{auk}{nan}
\define@key{fams}{hib}{nan}
\define@key{fams}{hlu}{nan}
\define@key{fams}{mba}{nan}
\define@key{fams}{kjk}{nan}
\define@key{fams}{hij}{nan}
\define@key{fams}{hir}{nan}
\define@key{fams}{hii}{nan}
\define@key{fams}{hmo}{nan}
\define@key{fams}{hit}{nan}
\define@key{fams}{htu}{nan}
\define@key{fams}{hiw}{nan}
\define@key{fams}{yhl}{nan}
\define@key{fams}{hle}{nan}
\define@key{fams}{hmf}{nan}
\define@key{fams}{hmz}{nan}
\define@key{fams}{hmv}{nan}
\define@key{fams}{mrk}{nan}
\define@key{fams}{hoh}{nan}
\define@key{fams}{hos}{nan}
\define@key{fams}{hhi}{nan}
\define@key{fams}{hoy}{nan}
\define@key{fams}{hoi}{nan}
\define@key{fams}{hod}{nan}
\define@key{fams}{hol}{nan}
\define@key{fams}{hom}{nan}
\define@key{fams}{hds}{nan}
\define@key{fams}{juh}{nan}
\define@key{fams}{how}{nan}
\define@key{fams}{hrm}{nan}
\define@key{fams}{hoe}{nan}
\define@key{fams}{hor}{nan}
\define@key{fams}{ero}{nan}
\define@key{fams}{hot}{nan}
\define@key{fams}{hti}{nan}
\define@key{fams}{hov}{nan}
\define@key{fams}{hhy}{nan}
\define@key{fams}{hoz}{nan}
\define@key{fams}{hpo}{nan}
\define@key{fams}{hra}{nan}
\define@key{fams}{hru}{nan}
\define@key{fams}{hug}{nan}
\define@key{fams}{qvh}{nan}
\define@key{fams}{hud}{nan}
\define@key{fams}{nhq}{nan}
\define@key{fams}{qwh}{nan}
\define@key{fams}{qvw}{nan}
\define@key{fams}{huh}{nan}
\define@key{fams}{mxs}{nan}
\define@key{fams}{czh}{nan}
\define@key{fams}{huw}{nan}
\define@key{fams}{hul}{nan}
\define@key{fams}{huy}{nan}
\define@key{fams}{hui}{nan}
\define@key{fams}{huk}{nan}
\define@key{fams}{hmb}{nan}
\define@key{fams}{huf}{nan}
\define@key{fams}{hut}{nan}
\define@key{fams}{hsh}{nan}
\define@key{fams}{hnu}{nan}
\define@key{fams}{nat}{nan}
\define@key{fams}{hum}{nan}
\define@key{fams}{hng}{nan}
\define@key{fams}{hkk}{nan}
\define@key{fams}{hap}{nan}
\define@key{fams}{xhu}{nan}
\define@key{fams}{geh}{nan}
\define@key{fams}{huo}{nan}
\define@key{fams}{hwo}{nan}
\define@key{fams}{hya}{nan}
\define@key{fams}{jab}{nan}
\define@key{fams}{yml}{nan}
\define@key{fams}{tek}{nan}
\define@key{fams}{ibl}{nan}
\define@key{fams}{iby}{nan}
\define@key{fams}{xib}{nan}
\define@key{fams}{ibn}{nan}
\define@key{fams}{ibr}{nan}
\define@key{fams}{ibu}{nan}
\define@key{fams}{bec}{nan}
\define@key{fams}{ida}{nan}
\define@key{fams}{idt}{nan}
\define@key{fams}{ide}{nan}
\define@key{fams}{idi}{nan}
\define@key{fams}{idc}{nan}
\define@key{fams}{ido}{nan}
\define@key{fams}{ldb}{nan}
\define@key{fams}{ife}{nan}
\define@key{fams}{iff}{nan}
\define@key{fams}{igl}{nan}
\define@key{fams}{igg}{nan}
\define@key{fams}{ahl}{nan}
\define@key{fams}{nar}{nan}
\define@key{fams}{igw}{nan}
\define@key{fams}{ihb}{nan}
\define@key{fams}{ikk}{nan}
\define@key{fams}{ikr}{nan}
\define@key{fams}{ikz}{nan}
\define@key{fams}{meb}{nan}
\define@key{fams}{ntk}{nan}
\define@key{fams}{iki}{nan}
\define@key{fams}{ikp}{nan}
\define@key{fams}{txi}{nan}
\define@key{fams}{ikv}{nan}
\define@key{fams}{ikl}{nan}
\define@key{fams}{ikw}{nan}
\define@key{fams}{ila}{nan}
\define@key{fams}{mbi}{nan}
\define@key{fams}{ili}{nan}
\define@key{fams}{ilu}{nan}
\define@key{fams}{xil}{nan}
\define@key{fams}{ilk}{nan}
\define@key{fams}{ilv}{nan}
\define@key{fams}{mlk}{nan}
\define@key{fams}{imo}{nan}
\define@key{fams}{arc}{nan}
\define@key{fams}{imr}{nan}
\define@key{fams}{abx}{nan}
\define@key{fams}{mzu}{nan}
\define@key{fams}{inp}{nan}
\define@key{fams}{smn}{nan}
\define@key{fams}{inl}{nan}
\define@key{fams}{idr}{nan}
\define@key{fams}{mvy}{nan}
\define@key{fams}{oin}{nan}
\define@key{fams}{iti}{nan}
\define@key{fams}{ino}{nan}
\define@key{fams}{loc}{nan}
\define@key{fams}{ior}{nan}
\define@key{fams}{ina}{nan}
\define@key{fams}{ile}{nan}
\define@key{fams}{igs}{nan}
\define@key{fams}{int}{nan}
\define@key{fams}{iks}{nan}
\define@key{fams}{azm}{nan}
\define@key{fams}{ipo}{nan}
\define@key{fams}{ipi}{nan}
\define@key{fams}{ass}{nan}
\define@key{fams}{ill}{nan}
\define@key{fams}{iry}{nan}
\define@key{fams}{ire}{nan}
\define@key{fams}{iri}{nan}
\define@key{fams}{bto}{nan}
\define@key{fams}{iru}{nan}
\define@key{fams}{isa}{nan}
\define@key{fams}{isn}{nan}
\define@key{fams}{agk}{nan}
\define@key{fams}{isc}{nan}
\define@key{fams}{igo}{nan}
\define@key{fams}{inn}{nan}
\define@key{fams}{crb}{nan}
\define@key{fams}{mir}{nan}
\define@key{fams}{nhk}{nan}
\define@key{fams}{ist}{nan}
\define@key{fams}{ruo}{nan}
\define@key{fams}{szv}{nan}
\define@key{fams}{isu}{nan}
\define@key{fams}{ite}{nan}
\define@key{fams}{itr}{nan}
\define@key{fams}{itx}{nan}
\define@key{fams}{itw}{nan}
\define@key{fams}{itm}{nan}
\define@key{fams}{mce}{nan}
\define@key{fams}{ivv}{nan}
\define@key{fams}{atg}{nan}
\define@key{fams}{iwk}{nan}
\define@key{fams}{kbm}{nan}
\define@key{fams}{iwo}{nan}
\define@key{fams}{mzi}{nan}
\define@key{fams}{vmj}{nan}
\define@key{fams}{iya}{nan}
\define@key{fams}{uiv}{nan}
\define@key{fams}{crt}{nan}
\define@key{fams}{nca}{nan}
\define@key{fams}{crq}{nan}
\define@key{fams}{izi}{nan}
\define@key{fams}{cbo}{nan}
\define@key{fams}{rzh}{nan}
\define@key{fams}{jdg}{nan}
\define@key{fams}{jad}{nan}
\define@key{fams}{jah}{nan}
\define@key{fams}{awv}{nan}
\define@key{fams}{jat}{nan}
\define@key{fams}{jak}{nan}
\define@key{fams}{maj}{nan}
\define@key{fams}{bxl}{nan}
\define@key{fams}{jcs}{nan}
\define@key{fams}{jls}{nan}
\define@key{fams}{jax}{nan}
\define@key{fams}{jnd}{nan}
\define@key{fams}{jna}{nan}
\define@key{fams}{djo}{nan}
\define@key{fams}{jni}{nan}
\define@key{fams}{jar}{nan}
\define@key{fams}{jra}{nan}
\define@key{fams}{jaf}{nan}
\define@key{fams}{qxw}{nan}
\define@key{fams}{jns}{nan}
\define@key{fams}{jvd}{nan}
\define@key{fams}{jaz}{nan}
\define@key{fams}{jyy}{nan}
\define@key{fams}{jje}{nan}
\define@key{fams}{bze}{nan}
\define@key{fams}{xuj}{nan}
\define@key{fams}{jer}{nan}
\define@key{fams}{jee}{nan}
\define@key{fams}{tmr}{nan}
\define@key{fams}{jhs}{nan}
\define@key{fams}{jio}{nan}
\define@key{fams}{juo}{nan}
\define@key{fams}{jib}{nan}
\define@key{fams}{jii}{nan}
\define@key{fams}{jie}{nan}
\define@key{fams}{jil}{nan}
\define@key{fams}{jim}{nan}
\define@key{fams}{jmi}{nan}
\define@key{fams}{jia}{nan}
\define@key{fams}{cjy}{nan}
\define@key{fams}{pnu}{nan}
\define@key{fams}{jul}{nan}
\define@key{fams}{jrr}{nan}
\define@key{fams}{jit}{nan}
\define@key{fams}{kaj}{nan}
\define@key{fams}{job}{nan}
\define@key{fams}{jbr}{nan}
\define@key{fams}{jeu}{nan}
\define@key{fams}{jor}{nan}
\define@key{fams}{jrt}{nan}
\define@key{fams}{jow}{nan}
\define@key{fams}{itk}{nan}
\define@key{fams}{jdt}{nan}
\define@key{fams}{jpr}{nan}
\define@key{fams}{yud}{nan}
\define@key{fams}{aju}{nan}
\define@key{fams}{yhd}{nan}
\define@key{fams}{jye}{nan}
\define@key{fams}{jum}{nan}
\define@key{fams}{jml}{nan}
\define@key{fams}{jus}{nan}
\define@key{fams}{mxq}{nan}
\define@key{fams}{juy}{nan}
\define@key{fams}{jut}{nan}
\define@key{fams}{juu}{nan}
\define@key{fams}{mwb}{nan}
\define@key{fams}{vmc}{nan}
\define@key{fams}{jwi}{nan}
\define@key{fams}{xku}{nan}
\define@key{fams}{gna}{nan}
\define@key{fams}{ldl}{nan}
\define@key{fams}{ckn}{nan}
\define@key{fams}{ksp}{nan}
\define@key{fams}{kvf}{nan}
\define@key{fams}{gbw}{nan}
\define@key{fams}{klz}{nan}
\define@key{fams}{onk}{nan}
\define@key{fams}{lkb}{nan}
\define@key{fams}{uka}{nan}
\define@key{fams}{kbu}{nan}
\define@key{fams}{kea}{nan}
\define@key{fams}{cwa}{nan}
\define@key{fams}{kcw}{nan}
\define@key{fams}{gjk}{nan}
\define@key{fams}{kfr}{nan}
\define@key{fams}{kcx}{nan}
\define@key{fams}{xkk}{nan}
\define@key{fams}{kej}{nan}
\define@key{fams}{kdu}{nan}
\define@key{fams}{kad}{nan}
\define@key{fams}{kzd}{nan}
\define@key{fams}{kdv}{nan}
\define@key{fams}{ktp}{nan}
\define@key{fams}{jka}{nan}
\define@key{fams}{kpu}{nan}
\define@key{fams}{sqx}{nan}
\define@key{fams}{syw}{nan}
\define@key{fams}{kll}{nan}
\define@key{fams}{cgc}{nan}
\define@key{fams}{gel}{nan}
\define@key{fams}{xkg}{nan}
\define@key{fams}{hka}{nan}
\define@key{fams}{agw}{nan}
\define@key{fams}{kzb}{nan}
\define@key{fams}{kzp}{nan}
\define@key{fams}{kbw}{nan}
\define@key{fams}{kep}{nan}
\define@key{fams}{kzq}{nan}
\define@key{fams}{kkq}{nan}
\define@key{fams}{xai}{nan}
\define@key{fams}{zka}{nan}
\define@key{fams}{krd}{nan}
\define@key{fams}{ckr}{nan}
\define@key{fams}{kzm}{nan}
\define@key{fams}{kce}{nan}
\define@key{fams}{tcq}{nan}
\define@key{fams}{xkj}{nan}
\define@key{fams}{kag}{nan}
\define@key{fams}{ckq}{nan}
\define@key{fams}{kjv}{nan}
\define@key{fams}{xdq}{nan}
\define@key{fams}{kka}{nan}
\define@key{fams}{kke}{nan}
\define@key{fams}{kqf}{nan}
\define@key{fams}{kkj}{nan}
\define@key{fams}{keo}{nan}
\define@key{fams}{wkl}{nan}
\define@key{fams}{kzz}{nan}
\define@key{fams}{kkf}{nan}
\define@key{fams}{kba}{nan}
\define@key{fams}{gll}{nan}
\define@key{fams}{ijn}{nan}
\define@key{fams}{knz}{nan}
\define@key{fams}{kqe}{nan}
\define@key{fams}{kve}{nan}
\define@key{fams}{kly}{nan}
\define@key{fams}{lkm}{nan}
\define@key{fams}{xka}{nan}
\define@key{fams}{rmf}{nan}
\define@key{fams}{ywa}{nan}
\define@key{fams}{kli}{nan}
\define@key{fams}{keq}{nan}
\define@key{fams}{jmr}{nan}
\define@key{fams}{kci}{nan}
\define@key{fams}{klp}{nan}
\define@key{fams}{kzx}{nan}
\define@key{fams}{kyk}{nan}
\define@key{fams}{kgx}{nan}
\define@key{fams}{vkm}{nan}
\define@key{fams}{xbw}{nan}
\define@key{fams}{irx}{nan}
\define@key{fams}{kyy}{nan}
\define@key{fams}{ktb}{nan}
\define@key{fams}{kmi}{nan}
\define@key{fams}{kdx}{nan}
\define@key{fams}{kcq}{nan}
\define@key{fams}{xla}{nan}
\define@key{fams}{hig}{nan}
\define@key{fams}{bjj}{nan}
\define@key{fams}{xnb}{nan}
\define@key{fams}{soq}{nan}
\define@key{fams}{kbs}{nan}
\define@key{fams}{kqw}{nan}
\define@key{fams}{gam}{nan}
\define@key{fams}{xnr}{nan}
\define@key{fams}{kxs}{nan}
\define@key{fams}{kzy}{nan}
\define@key{fams}{kty}{nan}
\define@key{fams}{kcp}{nan}
\define@key{fams}{kkv}{nan}
\define@key{fams}{igm}{nan}
\define@key{fams}{kev}{nan}
\define@key{fams}{kdp}{nan}
\define@key{fams}{kzo}{nan}
\define@key{fams}{wat}{nan}
\define@key{fams}{ktk}{nan}
\define@key{fams}{knr}{nan}
\define@key{fams}{kmu}{nan}
\define@key{fams}{kft}{nan}
\define@key{fams}{kbe}{nan}
\define@key{fams}{kxn}{nan}
\define@key{fams}{ksk}{nan}
\define@key{fams}{xkt}{nan}
\define@key{fams}{kni}{nan}
\define@key{fams}{khx}{nan}
\define@key{fams}{kqn}{nan}
\define@key{fams}{kax}{nan}
\define@key{fams}{xpn}{nan}
\define@key{fams}{tbx}{nan}
\define@key{fams}{khp}{nan}
\define@key{fams}{ykm}{nan}
\define@key{fams}{kbi}{nan}
\define@key{fams}{klo}{nan}
\define@key{fams}{xkh}{nan}
\define@key{fams}{kzr}{nan}
\define@key{fams}{reg}{nan}
\define@key{fams}{kth}{nan}
\define@key{fams}{mry}{nan}
\define@key{fams}{xrw}{nan}
\define@key{fams}{xar}{nan}
\define@key{fams}{kgv}{nan}
\define@key{fams}{kbn}{nan}
\define@key{fams}{kyd}{nan}
\define@key{fams}{kmf}{nan}
\define@key{fams}{kai}{nan}
\define@key{fams}{kmv}{nan}
\define@key{fams}{kgn}{nan}
\define@key{fams}{kbj}{nan}
\define@key{fams}{kil}{nan}
\define@key{fams}{kuq}{nan}
\define@key{fams}{kko}{nan}
\define@key{fams}{krb}{nan}
\define@key{fams}{bbv}{nan}
\define@key{fams}{krx}{nan}
\define@key{fams}{kxh}{nan}
\define@key{fams}{xkx}{nan}
\define@key{fams}{kyn}{nan}
\define@key{fams}{rxw}{nan}
\define@key{fams}{ccj}{nan}
\define@key{fams}{ksn}{nan}
\define@key{fams}{kkz}{nan}
\define@key{fams}{khs}{nan}
\define@key{fams}{ktq}{nan}
\define@key{fams}{xat}{nan}
\define@key{fams}{tmb}{nan}
\define@key{fams}{tkt}{nan}
\define@key{fams}{ykt}{nan}
\define@key{fams}{kfu}{nan}
\define@key{fams}{kaf}{nan}
\define@key{fams}{kta}{nan}
\define@key{fams}{vkk}{nan}
\define@key{fams}{xau}{nan}
\define@key{fams}{ckv}{nan}
\define@key{fams}{kcb}{nan}
\define@key{fams}{kgb}{nan}
\define@key{fams}{kaw}{nan}
\define@key{fams}{ktx}{nan}
\define@key{fams}{kbb}{nan}
\define@key{fams}{pdu}{nan}
\define@key{fams}{xay}{nan}
\define@key{fams}{xkn}{nan}
\define@key{fams}{kyt}{nan}
\define@key{fams}{kzl}{nan}
\define@key{fams}{kxy}{nan}
\define@key{fams}{kzu}{nan}
\define@key{fams}{kzk}{nan}
\define@key{fams}{keh}{nan}
\define@key{fams}{khz}{nan}
\define@key{fams}{meo}{nan}
\define@key{fams}{kdy}{nan}
\define@key{fams}{khh}{nan}
\define@key{fams}{kec}{nan}
\define@key{fams}{bmh}{nan}
\define@key{fams}{eyo}{nan}
\define@key{fams}{khy}{nan}
\define@key{fams}{keb}{nan}
\define@key{fams}{ify}{nan}
\define@key{fams}{kbo}{nan}
\define@key{fams}{xel}{nan}
\define@key{fams}{kyo}{nan}
\define@key{fams}{kem}{nan}
\define@key{fams}{bzp}{nan}
\define@key{fams}{xem}{nan}
\define@key{fams}{xkw}{nan}
\define@key{fams}{dmo}{nan}
\define@key{fams}{sjk}{nan}
\define@key{fams}{xbn}{nan}
\define@key{fams}{gat}{nan}
\define@key{fams}{kvm}{nan}
\define@key{fams}{klf}{nan}
\define@key{fams}{knx}{nan}
\define@key{fams}{knl}{nan}
\define@key{fams}{kxi}{nan}
\define@key{fams}{kns}{nan}
\define@key{fams}{ndb}{nan}
\define@key{fams}{kzh}{nan}
\define@key{fams}{lke}{nan}
\define@key{fams}{xeu}{nan}
\define@key{fams}{kpn}{nan}
\define@key{fams}{kuk}{nan}
\define@key{fams}{hhr}{nan}
\define@key{fams}{ked}{nan}
\define@key{fams}{xke}{nan}
\define@key{fams}{kxz}{nan}
\define@key{fams}{kvr}{nan}
\define@key{fams}{xes}{nan}
\define@key{fams}{kae}{nan}
\define@key{fams}{ktt}{nan}
\define@key{fams}{kyg}{nan}
\define@key{fams}{xkv}{nan}
\define@key{fams}{hkh}{nan}
\define@key{fams}{kbg}{nan}
\define@key{fams}{kht}{nan}
\define@key{fams}{ksu}{nan}
\define@key{fams}{khn}{nan}
\define@key{fams}{kjm}{nan}
\define@key{fams}{ksy}{nan}
\define@key{fams}{kfw}{nan}
\define@key{fams}{lko}{nan}
\define@key{fams}{kqg}{nan}
\define@key{fams}{tlx}{nan}
\define@key{fams}{xkf}{nan}
\define@key{fams}{xhe}{nan}
\define@key{fams}{nkh}{nan}
\define@key{fams}{kix}{nan}
\define@key{fams}{kwx}{nan}
\define@key{fams}{kqm}{nan}
\define@key{fams}{ykl}{nan}
\define@key{fams}{xkc}{nan}
\define@key{fams}{nkb}{nan}
\define@key{fams}{ktc}{nan}
\define@key{fams}{kho}{nan}
\define@key{fams}{khf}{nan}
\define@key{fams}{kfm}{nan}
\define@key{fams}{xco}{nan}
\define@key{fams}{kie}{nan}
\define@key{fams}{prm}{nan}
\define@key{fams}{kzg}{nan}
\define@key{fams}{kih}{nan}
\define@key{fams}{kqr}{nan}
\define@key{fams}{kmb}{nan}
\define@key{fams}{kiv}{nan}
\define@key{fams}{sbt}{nan}
\define@key{fams}{kqp}{nan}
\define@key{fams}{krj}{nan}
\define@key{fams}{kco}{nan}
\define@key{fams}{cbw}{nan}
\define@key{fams}{knq}{nan}
\define@key{fams}{kkd}{nan}
\define@key{fams}{ues}{nan}
\define@key{fams}{kkm}{nan}
\define@key{fams}{apk}{nan}
\define@key{fams}{sgc}{nan}
\define@key{fams}{kyi}{nan}
\define@key{fams}{kkr}{nan}
\define@key{fams}{okr}{nan}
\define@key{fams}{kiu}{nan}
\define@key{fams}{fkk}{nan}
\define@key{fams}{lks}{nan}
\define@key{fams}{kiz}{nan}
\define@key{fams}{kis}{nan}
\define@key{fams}{zkt}{nan}
\define@key{fams}{mwk}{nan}
\define@key{fams}{mkw}{nan}
\define@key{fams}{kqt}{nan}
\define@key{fams}{tlh}{nan}
\define@key{fams}{kib}{nan}
\define@key{fams}{kpd}{nan}
\define@key{fams}{kcj}{nan}
\define@key{fams}{kgu}{nan}
\define@key{fams}{thq}{nan}
\define@key{fams}{kdq}{nan}
\define@key{fams}{dhw}{nan}
\define@key{fams}{cdz}{nan}
\define@key{fams}{ksz}{nan}
\define@key{fams}{vko}{nan}
\define@key{fams}{kwp}{nan}
\define@key{fams}{kod}{nan}
\define@key{fams}{kcs}{nan}
\define@key{fams}{kpi}{nan}
\define@key{fams}{kwl}{nan}
\define@key{fams}{zkg}{nan}
\define@key{fams}{plk}{nan}
\define@key{fams}{kkx}{nan}
\define@key{fams}{kkt}{nan}
\define@key{fams}{nkd}{nan}
\define@key{fams}{kxt}{nan}
\define@key{fams}{kou}{nan}
\define@key{fams}{gko}{nan}
\define@key{fams}{xod}{nan}
\define@key{fams}{kzn}{nan}
\define@key{fams}{klc}{nan}
\define@key{fams}{ekl}{nan}
\define@key{fams}{biw}{nan}
\define@key{fams}{skn}{nan}
\define@key{fams}{klm}{nan}
\define@key{fams}{kol}{nan}
\define@key{fams}{klx}{nan}
\define@key{fams}{kmy}{nan}
\define@key{fams}{kpf}{nan}
\define@key{fams}{tyn}{nan}
\define@key{fams}{kmm}{nan}
\define@key{fams}{xoi}{nan}
\define@key{fams}{kmw}{nan}
\define@key{fams}{kvh}{nan}
\define@key{fams}{kvp}{nan}
\define@key{fams}{kzv}{nan}
\define@key{fams}{kxw}{nan}
\define@key{fams}{knd}{nan}
\define@key{fams}{kdw}{nan}
\define@key{fams}{klk}{nan}
\define@key{fams}{kcz}{nan}
\define@key{fams}{knu}{nan}
\define@key{fams}{kno}{nan}
\define@key{fams}{koa}{nan}
\define@key{fams}{kxc}{nan}
\define@key{fams}{nbe}{nan}
\define@key{fams}{mku}{nan}
\define@key{fams}{koo}{nan}
\define@key{fams}{ozm}{nan}
\define@key{fams}{fuj}{nan}
\define@key{fams}{xop}{nan}
\define@key{fams}{opk}{nan}
\define@key{fams}{kcy}{nan}
\define@key{fams}{koz}{nan}
\define@key{fams}{okh}{nan}
\define@key{fams}{vkp}{nan}
\define@key{fams}{ktl}{nan}
\define@key{fams}{krp}{nan}
\define@key{fams}{kfo}{nan}
\define@key{fams}{krf}{nan}
\define@key{fams}{xkq}{nan}
\define@key{fams}{kqj}{nan}
\define@key{fams}{jkr}{nan}
\define@key{fams}{vkn}{nan}
\define@key{fams}{vkz}{nan}
\define@key{fams}{kfd}{nan}
\define@key{fams}{kpq}{nan}
\define@key{fams}{xor}{nan}
\define@key{fams}{kfp}{nan}
\define@key{fams}{kiq}{nan}
\define@key{fams}{kid}{nan}
\define@key{fams}{kqk}{nan}
\define@key{fams}{koq}{nan}
\define@key{fams}{mqg}{nan}
\define@key{fams}{grm}{nan}
\define@key{fams}{avk}{nan}
\define@key{fams}{zko}{nan}
\define@key{fams}{kyf}{nan}
\define@key{fams}{kqb}{nan}
\define@key{fams}{kvc}{nan}
\define@key{fams}{xow}{nan}
\define@key{fams}{kwh}{nan}
\define@key{fams}{kga}{nan}
\define@key{fams}{koh}{nan}
\define@key{fams}{kqd}{nan}
\define@key{fams}{kuw}{nan}
\define@key{fams}{kpl}{nan}
\define@key{fams}{pbn}{nan}
\define@key{fams}{koc}{nan}
\define@key{fams}{cpo}{nan}
\define@key{fams}{kef}{nan}
\define@key{fams}{kph}{nan}
\define@key{fams}{kye}{nan}
\define@key{fams}{rka}{nan}
\define@key{fams}{xre}{nan}
\define@key{fams}{kri}{nan}
\define@key{fams}{kxb}{nan}
\define@key{fams}{tyu}{nan}
\define@key{fams}{yku}{nan}
\define@key{fams}{uan}{nan}
\define@key{fams}{kua}{nan}
\define@key{fams}{ykn}{nan}
\define@key{fams}{ugh}{nan}
\define@key{fams}{kgf}{nan}
\define@key{fams}{kof}{nan}
\define@key{fams}{jko}{nan}
\define@key{fams}{kvb}{nan}
\define@key{fams}{lkc}{nan}
\define@key{fams}{kfg}{nan}
\define@key{fams}{kyw}{nan}
\define@key{fams}{kov}{nan}
\define@key{fams}{kow}{nan}
\define@key{fams}{kes}{nan}
\define@key{fams}{dkr}{nan}
\define@key{fams}{vkj}{nan}
\define@key{fams}{kux}{nan}
\define@key{fams}{kez}{nan}
\define@key{fams}{kfn}{nan}
\define@key{fams}{ugb}{nan}
\define@key{fams}{xmp}{nan}
\define@key{fams}{xmh}{nan}
\define@key{fams}{ukv}{nan}
\define@key{fams}{kul}{nan}
\define@key{fams}{kxj}{nan}
\define@key{fams}{vkl}{nan}
\define@key{fams}{xpk}{nan}
\define@key{fams}{kfx}{nan}
\define@key{fams}{pzh}{nan}
\define@key{fams}{uon}{nan}
\define@key{fams}{bbu}{nan}
\define@key{fams}{kdi}{nan}
\define@key{fams}{ksl}{nan}
\define@key{fams}{ksm}{nan}
\define@key{fams}{xks}{nan}
\define@key{fams}{kra}{nan}
\define@key{fams}{kuo}{nan}
\define@key{fams}{zum}{nan}
\define@key{fams}{wku}{nan}
\define@key{fams}{kdn}{nan}
\define@key{fams}{shd}{nan}
\define@key{fams}{kgl}{nan}
\define@key{fams}{ggk}{nan}
\define@key{fams}{kfl}{nan}
\define@key{fams}{kse}{nan}
\define@key{fams}{xug}{nan}
\define@key{fams}{pep}{nan}
\define@key{fams}{njx}{nan}
\define@key{fams}{kug}{nan}
\define@key{fams}{mkn}{nan}
\define@key{fams}{key}{nan}
\define@key{fams}{nqk}{nan}
\define@key{fams}{krh}{nan}
\define@key{fams}{kfh}{nan}
\define@key{fams}{kuj}{nan}
\define@key{fams}{nbn}{nan}
\define@key{fams}{kfv}{nan}
\define@key{fams}{vku}{nan}
\define@key{fams}{kuv}{nan}
\define@key{fams}{xkz}{nan}
\define@key{fams}{ktm}{nan}
\define@key{fams}{kjr}{nan}
\define@key{fams}{kyr}{nan}
\define@key{fams}{kus}{nan}
\define@key{fams}{ksg}{nan}
\define@key{fams}{kuh}{nan}
\define@key{fams}{ksv}{nan}
\define@key{fams}{ght}{nan}
\define@key{fams}{kub}{nan}
\define@key{fams}{xut}{nan}
\define@key{fams}{kpa}{nan}
\define@key{fams}{khj}{nan}
\define@key{fams}{kdc}{nan}
\define@key{fams}{uky}{nan}
\define@key{fams}{lku}{nan}
\define@key{fams}{olu}{nan}
\define@key{fams}{cwt}{nan}
\define@key{fams}{blh}{nan}
\define@key{fams}{kdt}{nan}
\define@key{fams}{fkv}{nan}
\define@key{fams}{kwb}{nan}
\define@key{fams}{bko}{nan}
\define@key{fams}{kwz}{nan}
\define@key{fams}{wka}{nan}
\define@key{fams}{kdz}{nan}
\define@key{fams}{kwu}{nan}
\define@key{fams}{qwt}{nan}
\define@key{fams}{kmq}{nan}
\define@key{fams}{ktf}{nan}
\define@key{fams}{kwm}{nan}
\define@key{fams}{okk}{nan}
\define@key{fams}{knp}{nan}
\define@key{fams}{kwj}{nan}
\define@key{fams}{kvi}{nan}
\define@key{fams}{xdo}{nan}
\define@key{fams}{kwf}{nan}
\define@key{fams}{kop}{nan}
\define@key{fams}{kya}{nan}
\define@key{fams}{cwe}{nan}
\define@key{fams}{xwr}{nan}
\define@key{fams}{kkb}{nan}
\define@key{fams}{kwr}{nan}
\define@key{fams}{kws}{nan}
\define@key{fams}{kwt}{nan}
\define@key{fams}{kuc}{nan}
\define@key{fams}{kww}{nan}
\define@key{fams}{bka}{nan}
\define@key{fams}{tye}{nan}
\define@key{fams}{kql}{nan}
\define@key{fams}{ldn}{nan}
\define@key{fams}{bwj}{nan}
\define@key{fams}{ldi}{nan}
\define@key{fams}{lbb}{nan}
\define@key{fams}{lbi}{nan}
\define@key{fams}{jku}{nan}
\define@key{fams}{ypb}{nan}
\define@key{fams}{mwi}{nan}
\define@key{fams}{dtb}{nan}
\define@key{fams}{zpl}{nan}
\define@key{fams}{zpa}{nan}
\define@key{fams}{lkl}{nan}
\define@key{fams}{lgh}{nan}
\define@key{fams}{lgb}{nan}
\define@key{fams}{lhh}{nan}
\define@key{fams}{lhn}{nan}
\define@key{fams}{lhl}{nan}
\define@key{fams}{lhi}{nan}
\define@key{fams}{lmx}{nan}
\define@key{fams}{lji}{nan}
\define@key{fams}{lap}{nan}
\define@key{fams}{lka}{nan}
\define@key{fams}{lkh}{nan}
\define@key{fams}{lki}{nan}
\define@key{fams}{lkn}{nan}
\define@key{fams}{lkd}{nan}
\define@key{fams}{lxm}{nan}
\define@key{fams}{lla}{nan}
\define@key{fams}{leb}{nan}
\define@key{fams}{cnl}{nan}
\define@key{fams}{las}{nan}
\define@key{fams}{lmr}{nan}
\define@key{fams}{lmq}{nan}
\define@key{fams}{lai}{nan}
\define@key{fams}{lmy}{nan}
\define@key{fams}{quf}{nan}
\define@key{fams}{lbn}{nan}
\define@key{fams}{bma}{nan}
\define@key{fams}{ldh}{nan}
\define@key{fams}{lmk}{nan}
\define@key{fams}{lev}{nan}
\define@key{fams}{lmg}{nan}
\define@key{fams}{abl}{nan}
\define@key{fams}{llh}{nan}
\define@key{fams}{ruu}{nan}
\define@key{fams}{ldm}{nan}
\define@key{fams}{sfb}{nan}
\define@key{fams}{yln}{nan}
\define@key{fams}{lna}{nan}
\define@key{fams}{lno}{nan}
\define@key{fams}{lnm}{nan}
\define@key{fams}{lnh}{nan}
\define@key{fams}{lwm}{nan}
\define@key{fams}{ztl}{nan}
\define@key{fams}{laa}{nan}
\define@key{fams}{lrt}{nan}
\define@key{fams}{lrv}{nan}
\define@key{fams}{hmd}{nan}
\define@key{fams}{lrl}{nan}
\define@key{fams}{lro}{nan}
\define@key{fams}{lar}{nan}
\define@key{fams}{lan}{nan}
\define@key{fams}{llm}{nan}
\define@key{fams}{lsa}{nan}
\define@key{fams}{lsi}{nan}
\define@key{fams}{lss}{nan}
\define@key{fams}{lat}{nan}
\define@key{fams}{ltu}{nan}
\define@key{fams}{ltn}{nan}
\define@key{fams}{lsl}{nan}
\define@key{fams}{llx}{nan}
\define@key{fams}{luf}{nan}
\define@key{fams}{lre}{nan}
\define@key{fams}{clt}{nan}
\define@key{fams}{lbv}{nan}
\define@key{fams}{lbx}{nan}
\define@key{fams}{lvi}{nan}
\define@key{fams}{tgi}{nan}
\define@key{fams}{lwu}{nan}
\define@key{fams}{lya}{nan}
\define@key{fams}{ldk}{nan}
\define@key{fams}{lfa}{nan}
\define@key{fams}{lgm}{nan}
\define@key{fams}{lcc}{nan}
\define@key{fams}{cae}{nan}
\define@key{fams}{tql}{nan}
\define@key{fams}{urr}{nan}
\define@key{fams}{lzn}{nan}
\define@key{fams}{lek}{nan}
\define@key{fams}{llk}{nan}
\define@key{fams}{lel}{nan}
\define@key{fams}{llc}{nan}
\define@key{fams}{lpa}{nan}
\define@key{fams}{lle}{nan}
\define@key{fams}{leq}{nan}
\define@key{fams}{lrz}{nan}
\define@key{fams}{lei}{nan}
\define@key{fams}{xle}{nan}
\define@key{fams}{ldj}{nan}
\define@key{fams}{ley}{nan}
\define@key{fams}{lej}{nan}
\define@key{fams}{lgr}{nan}
\define@key{fams}{lgi}{nan}
\define@key{fams}{leh}{nan}
\define@key{fams}{ler}{nan}
\define@key{fams}{ldg}{nan}
\define@key{fams}{lpe}{nan}
\define@key{fams}{xlp}{nan}
\define@key{fams}{gnh}{nan}
\define@key{fams}{let}{nan}
\define@key{fams}{nms}{nan}
\define@key{fams}{leo}{nan}
\define@key{fams}{lvu}{nan}
\define@key{fams}{lwe}{nan}
\define@key{fams}{lwt}{nan}
\define@key{fams}{ayi}{nan}
\define@key{fams}{lhp}{nan}
\define@key{fams}{lix}{nan}
\define@key{fams}{njn}{nan}
\define@key{fams}{zln}{nan}
\define@key{fams}{ste}{nan}
\define@key{fams}{lir}{nan}
\define@key{fams}{liz}{nan}
\define@key{fams}{liq}{nan}
\define@key{fams}{lbs}{nan}
\define@key{fams}{lig}{nan}
\define@key{fams}{lgz}{nan}
\define@key{fams}{lih}{nan}
\define@key{fams}{mgi}{nan}
\define@key{fams}{lik}{nan}
\define@key{fams}{lie}{nan}
\define@key{fams}{lio}{nan}
\define@key{fams}{kxx}{nan}
\define@key{fams}{lib}{nan}
\define@key{fams}{kwc}{nan}
\define@key{fams}{lll}{nan}
\define@key{fams}{bme}{nan}
\define@key{fams}{lim}{nan}
\define@key{fams}{lmp}{nan}
\define@key{fams}{ylm}{nan}
\define@key{fams}{kmk}{nan}
\define@key{fams}{qlm}{nan}
\define@key{fams}{klw}{nan}
\define@key{fams}{pml}{nan}
\define@key{fams}{onb}{nan}
\define@key{fams}{lgk}{nan}
\define@key{fams}{lfn}{nan}
\define@key{fams}{ljl}{nan}
\define@key{fams}{apl}{nan}
\define@key{fams}{lpo}{nan}
\define@key{fams}{lcs}{nan}
\define@key{fams}{lcl}{nan}
\define@key{fams}{lsh}{nan}
\define@key{fams}{lsd}{nan}
\define@key{fams}{lzh}{nan}
\define@key{fams}{lls}{nan}
\define@key{fams}{lzl}{nan}
\define@key{fams}{zlj}{nan}
\define@key{fams}{zlq}{nan}
\define@key{fams}{olo}{nan}
\define@key{fams}{loq}{nan}
\define@key{fams}{lbm}{nan}
\define@key{fams}{lgq}{nan}
\define@key{fams}{rag}{nan}
\define@key{fams}{liu}{nan}
\define@key{fams}{lof}{nan}
\define@key{fams}{src}{nan}
\define@key{fams}{qvj}{nan}
\define@key{fams}{jbo}{nan}
\define@key{fams}{yaz}{nan}
\define@key{fams}{lky}{nan}
\define@key{fams}{lcd}{nan}
\define@key{fams}{llq}{nan}
\define@key{fams}{llg}{nan}
\define@key{fams}{ycl}{nan}
\define@key{fams}{llb}{nan}
\define@key{fams}{loa}{nan}
\define@key{fams}{rmi}{nan}
\define@key{fams}{loi}{nan}
\define@key{fams}{lmv}{nan}
\define@key{fams}{lmi}{nan}
\define@key{fams}{lmo}{nan}
\define@key{fams}{loo}{nan}
\define@key{fams}{ngl}{nan}
\define@key{fams}{lce}{nan}
\define@key{fams}{lpn}{nan}
\define@key{fams}{wok}{nan}
\define@key{fams}{lnu}{nan}
\define@key{fams}{ttw}{nan}
\define@key{fams}{ldo}{nan}
\define@key{fams}{lop}{nan}
\define@key{fams}{lpx}{nan}
\define@key{fams}{lrn}{nan}
\define@key{fams}{spq}{nan}
\define@key{fams}{lnn}{nan}
\define@key{fams}{uvl}{nan}
\define@key{fams}{lht}{nan}
\define@key{fams}{dtr}{nan}
\define@key{fams}{lou}{nan}
\define@key{fams}{lox}{nan}
\define@key{fams}{xlo}{nan}
\define@key{fams}{sli}{nan}
\define@key{fams}{tto}{nan}
\define@key{fams}{nsb}{nan}
\define@key{fams}{kml}{nan}
\define@key{fams}{cea}{nan}
\define@key{fams}{axl}{nan}
\define@key{fams}{ztp}{nan}
\define@key{fams}{kcc}{nan}
\define@key{fams}{lcf}{nan}
\define@key{fams}{knb}{nan}
\define@key{fams}{luq}{nan}
\define@key{fams}{lud}{nan}
\define@key{fams}{ldq}{nan}
\define@key{fams}{ruf}{nan}
\define@key{fams}{lcq}{nan}
\define@key{fams}{lum}{nan}
\define@key{fams}{dop}{nan}
\define@key{fams}{smj}{nan}
\define@key{fams}{lmz}{nan}
\define@key{fams}{lup}{nan}
\define@key{fams}{lmd}{nan}
\define@key{fams}{luk}{nan}
\define@key{fams}{luj}{nan}
\define@key{fams}{lga}{nan}
\define@key{fams}{luw}{nan}
\define@key{fams}{hml}{nan}
\define@key{fams}{ldd}{nan}
\define@key{fams}{lse}{nan}
\define@key{fams}{xls}{nan}
\define@key{fams}{ndy}{nan}
\define@key{fams}{luv}{nan}
\define@key{fams}{lyn}{nan}
\define@key{fams}{lwa}{nan}
\define@key{fams}{xlc}{nan}
\define@key{fams}{xld}{nan}
\define@key{fams}{lyg}{nan}
\define@key{fams}{cma}{nan}
\define@key{fams}{mew}{nan}
\define@key{fams}{ymm}{nan}
\define@key{fams}{mmz}{nan}
\define@key{fams}{mfz}{nan}
\define@key{fams}{mqa}{nan}
\define@key{fams}{kkg}{nan}
\define@key{fams}{muj}{nan}
\define@key{fams}{mcl}{nan}
\define@key{fams}{mzs}{nan}
\define@key{fams}{mvw}{nan}
\define@key{fams}{jmc}{nan}
\define@key{fams}{mpd}{nan}
\define@key{fams}{wpc}{nan}
\define@key{fams}{mzc}{nan}
\define@key{fams}{mmx}{nan}
\define@key{fams}{xmx}{nan}
\define@key{fams}{grg}{nan}
\define@key{fams}{kmd}{nan}
\define@key{fams}{mme}{nan}
\define@key{fams}{itt}{nan}
\define@key{fams}{maf}{nan}
\define@key{fams}{mkv}{nan}
\define@key{fams}{sgb}{nan}
\define@key{fams}{mtw}{nan}
\define@key{fams}{xtm}{nan}
\define@key{fams}{gmd}{nan}
\define@key{fams}{blx}{nan}
\define@key{fams}{gkd}{nan}
\define@key{fams}{gmg}{nan}
\define@key{fams}{gmx}{nan}
\define@key{fams}{zgr}{nan}
\define@key{fams}{bfz}{nan}
\define@key{fams}{mjx}{nan}
\define@key{fams}{pmh}{nan}
\define@key{fams}{mjy}{nan}
\define@key{fams}{mhb}{nan}
\define@key{fams}{mzz}{nan}
\define@key{fams}{tnh}{nan}
\define@key{fams}{sks}{nan}
\define@key{fams}{mmm}{nan}
\define@key{fams}{vmf}{nan}
\define@key{fams}{cwb}{nan}
\define@key{fams}{xkl}{nan}
\define@key{fams}{mum}{nan}
\define@key{fams}{wmm}{nan}
\define@key{fams}{mti}{nan}
\define@key{fams}{xmj}{nan}
\define@key{fams}{mmj}{nan}
\define@key{fams}{mjz}{nan}
\define@key{fams}{mfp}{nan}
\define@key{fams}{aup}{nan}
\define@key{fams}{mkg}{nan}
\define@key{fams}{vmk}{nan}
\define@key{fams}{xmc}{nan}
\define@key{fams}{vmw}{nan}
\define@key{fams}{mhm}{nan}
\define@key{fams}{xsq}{nan}
\define@key{fams}{pbl}{nan}
\define@key{fams}{zmh}{nan}
\define@key{fams}{jmn}{nan}
\define@key{fams}{lva}{nan}
\define@key{fams}{mpu}{nan}
\define@key{fams}{ymk}{nan}
\define@key{fams}{umn}{nan}
\define@key{fams}{lon}{nan}
\define@key{fams}{xml}{nan}
\define@key{fams}{ima}{nan}
\define@key{fams}{ymr}{nan}
\define@key{fams}{mjo}{nan}
\define@key{fams}{mjr}{nan}
\define@key{fams}{mjq}{nan}
\define@key{fams}{mjp}{nan}
\define@key{fams}{ruy}{nan}
\define@key{fams}{swk}{nan}
\define@key{fams}{ccm}{nan}
\define@key{fams}{mln}{nan}
\define@key{fams}{mqz}{nan}
\define@key{fams}{mmt}{nan}
\define@key{fams}{ped}{nan}
\define@key{fams}{mkr}{nan}
\define@key{fams}{lws}{nan}
\define@key{fams}{bfo}{nan}
\define@key{fams}{pkt}{nan}
\define@key{fams}{mdc}{nan}
\define@key{fams}{gut}{nan}
\define@key{fams}{mlx}{nan}
\define@key{fams}{vml}{nan}
\define@key{fams}{mxf}{nan}
\define@key{fams}{mgq}{nan}
\define@key{fams}{mzd}{nan}
\define@key{fams}{mli}{nan}
\define@key{fams}{mlf}{nan}
\define@key{fams}{mbk}{nan}
\define@key{fams}{mkb}{nan}
\define@key{fams}{mdl}{nan}
\define@key{fams}{mll}{nan}
\define@key{fams}{mup}{nan}
\define@key{fams}{myk}{nan}
\define@key{fams}{mma}{nan}
\define@key{fams}{mhf}{nan}
\define@key{fams}{wmd}{nan}
\define@key{fams}{mvd}{nan}
\define@key{fams}{mgm}{nan}
\define@key{fams}{kdf}{nan}
\define@key{fams}{mqx}{nan}
\define@key{fams}{znk}{nan}
\define@key{fams}{mjl}{nan}
\define@key{fams}{mha}{nan}
\define@key{fams}{zma}{nan}
\define@key{fams}{zmk}{nan}
\define@key{fams}{mgs}{nan}
\define@key{fams}{mqu}{nan}
\define@key{fams}{tbf}{nan}
\define@key{fams}{mqr}{nan}
\define@key{fams}{aax}{nan}
\define@key{fams}{bwp}{nan}
\define@key{fams}{mht}{nan}
\define@key{fams}{zng}{nan}
\define@key{fams}{zme}{nan}
\define@key{fams}{mem}{nan}
\define@key{fams}{myj}{nan}
\define@key{fams}{mdk}{nan}
\define@key{fams}{kby}{nan}
\define@key{fams}{mrv}{nan}
\define@key{fams}{mbh}{nan}
\define@key{fams}{mmo}{nan}
\define@key{fams}{zns}{nan}
\define@key{fams}{xkb}{nan}
\define@key{fams}{mqp}{nan}
\define@key{fams}{nlm}{nan}
\define@key{fams}{mml}{nan}
\define@key{fams}{mjv}{nan}
\define@key{fams}{woo}{nan}
\define@key{fams}{msw}{nan}
\define@key{fams}{msk}{nan}
\define@key{fams}{nty}{nan}
\define@key{fams}{myg}{nan}
\define@key{fams}{kxf}{nan}
\define@key{fams}{wha}{nan}
\define@key{fams}{mxc}{nan}
\define@key{fams}{mny}{nan}
\define@key{fams}{mzj}{nan}
\define@key{fams}{mzv}{nan}
\define@key{fams}{mmd}{nan}
\define@key{fams}{mjn}{nan}
\define@key{fams}{mlh}{nan}
\define@key{fams}{mnm}{nan}
\define@key{fams}{mpy}{nan}
\define@key{fams}{mpw}{nan}
\define@key{fams}{bzh}{nan}
\define@key{fams}{sjm}{nan}
\define@key{fams}{vmh}{nan}
\define@key{fams}{nma}{nan}
\define@key{fams}{lrm}{nan}
\define@key{fams}{lri}{nan}
\define@key{fams}{mgb}{nan}
\define@key{fams}{mvr}{nan}
\define@key{fams}{mrs}{nan}
\define@key{fams}{mpg}{nan}
\define@key{fams}{dsz}{nan}
\define@key{fams}{vmr}{nan}
\define@key{fams}{mrx}{nan}
\define@key{fams}{mvu}{nan}
\define@key{fams}{mhg}{nan}
\define@key{fams}{qvm}{nan}
\define@key{fams}{mfm}{nan}
\define@key{fams}{nsr}{nan}
\define@key{fams}{mrr}{nan}
\define@key{fams}{nng}{nan}
\define@key{fams}{zmm}{nan}
\define@key{fams}{zmj}{nan}
\define@key{fams}{zmd}{nan}
\define@key{fams}{zmy}{nan}
\define@key{fams}{mrb}{nan}
\define@key{fams}{dad}{nan}
\define@key{fams}{hob}{nan}
\define@key{fams}{mqi}{nan}
\define@key{fams}{mbx}{nan}
\define@key{fams}{mds}{nan}
\define@key{fams}{msp}{nan}
\define@key{fams}{enb}{nan}
\define@key{fams}{rkm}{nan}
\define@key{fams}{mvo}{nan}
\define@key{fams}{xru}{nan}
\define@key{fams}{mre}{nan}
\define@key{fams}{zmg}{nan}
\define@key{fams}{mzr}{nan}
\define@key{fams}{mve}{nan}
\define@key{fams}{rwr}{nan}
\define@key{fams}{myx}{nan}
\define@key{fams}{tis}{nan}
\define@key{fams}{bks}{nan}
\define@key{fams}{msb}{nan}
\define@key{fams}{mho}{nan}
\define@key{fams}{jms}{nan}
\define@key{fams}{cuj}{nan}
\define@key{fams}{ism}{nan}
\define@key{fams}{bnf}{nan}
\define@key{fams}{msh}{nan}
\define@key{fams}{klv}{nan}
\define@key{fams}{msv}{nan}
\define@key{fams}{mes}{nan}
\define@key{fams}{mdg}{nan}
\define@key{fams}{mvs}{nan}
\define@key{fams}{mtn}{nan}
\define@key{fams}{mfh}{nan}
\define@key{fams}{xmt}{nan}
\define@key{fams}{mgv}{nan}
\define@key{fams}{mqe}{nan}
\define@key{fams}{mzo}{nan}
\define@key{fams}{mtm}{nan}
\define@key{fams}{met}{nan}
\define@key{fams}{axg}{nan}
\define@key{fams}{stj}{nan}
\define@key{fams}{cty}{nan}
\define@key{fams}{lsy}{nan}
\define@key{fams}{mhl}{nan}
\define@key{fams}{wma}{nan}
\define@key{fams}{mjj}{nan}
\define@key{fams}{mcz}{nan}
\define@key{fams}{mcw}{nan}
\define@key{fams}{mgk}{nan}
\define@key{fams}{mxl}{nan}
\define@key{fams}{xmy}{nan}
\define@key{fams}{sym}{nan}
\define@key{fams}{mnt}{nan}
\define@key{fams}{ifu}{nan}
\define@key{fams}{mzl}{nan}
\define@key{fams}{zpy}{nan}
\define@key{fams}{vmz}{nan}
\define@key{fams}{dkx}{nan}
\define@key{fams}{mdp}{nan}
\define@key{fams}{mgn}{nan}
\define@key{fams}{zmz}{nan}
\define@key{fams}{mxg}{nan}
\define@key{fams}{zmn}{nan}
\define@key{fams}{zmv}{nan}
\define@key{fams}{mvl}{nan}
\define@key{fams}{gwa}{nan}
\define@key{fams}{mdn}{nan}
\define@key{fams}{xmd}{nan}
\define@key{fams}{mfo}{nan}
\define@key{fams}{mql}{nan}
\define@key{fams}{zms}{nan}
\define@key{fams}{emz}{nan}
\define@key{fams}{mbo}{nan}
\define@key{fams}{zmw}{nan}
\define@key{fams}{moi}{nan}
\define@key{fams}{mdu}{nan}
\define@key{fams}{xmb}{nan}
\define@key{fams}{bgu}{nan}
\define@key{fams}{mxo}{nan}
\define@key{fams}{mka}{nan}
\define@key{fams}{mgz}{nan}
\define@key{fams}{mhw}{nan}
\define@key{fams}{mqb}{nan}
\define@key{fams}{bpc}{nan}
\define@key{fams}{mbv}{nan}
\define@key{fams}{mbu}{nan}
\define@key{fams}{mlb}{nan}
\define@key{fams}{mgy}{nan}
\define@key{fams}{mck}{nan}
\define@key{fams}{bbt}{nan}
\define@key{fams}{muc}{nan}
\define@key{fams}{mfu}{nan}
\define@key{fams}{gun}{nan}
\define@key{fams}{mjm}{nan}
\define@key{fams}{dmf}{nan}
\define@key{fams}{mue}{nan}
\define@key{fams}{mud}{nan}
\define@key{fams}{byv}{nan}
\define@key{fams}{mfj}{nan}
\define@key{fams}{mef}{nan}
\define@key{fams}{ruq}{nan}
\define@key{fams}{mmh}{nan}
\define@key{fams}{mvk}{nan}
\define@key{fams}{msf}{nan}
\define@key{fams}{hkn}{nan}
\define@key{fams}{mfx}{nan}
\define@key{fams}{med}{nan}
\define@key{fams}{mby}{nan}
\define@key{fams}{mfd}{nan}
\define@key{fams}{xkd}{nan}
\define@key{fams}{sim}{nan}
\define@key{fams}{xmg}{nan}
\define@key{fams}{mee}{nan}
\define@key{fams}{mea}{nan}
\define@key{fams}{mvx}{nan}
\define@key{fams}{mxm}{nan}
\define@key{fams}{lmb}{nan}
\define@key{fams}{meq}{nan}
\define@key{fams}{mrm}{nan}
\define@key{fams}{xmr}{nan}
\define@key{fams}{mnu}{nan}
\define@key{fams}{mer}{nan}
\define@key{fams}{wry}{nan}
\define@key{fams}{iyo}{nan}
\define@key{fams}{mci}{nan}
\define@key{fams}{zim}{nan}
\define@key{fams}{mys}{nan}
\define@key{fams}{mvz}{nan}
\define@key{fams}{cms}{nan}
\define@key{fams}{mgo}{nan}
\define@key{fams}{mxv}{nan}
\define@key{fams}{mtr}{nan}
\define@key{fams}{wtm}{nan}
\define@key{fams}{mfs}{nan}
\define@key{fams}{zmf}{nan}
\define@key{fams}{nfu}{nan}
\define@key{fams}{zam}{nan}
\define@key{fams}{pla}{nan}
\define@key{fams}{xmi}{nan}
\define@key{fams}{mmc}{nan}
\define@key{fams}{enm}{nan}
\define@key{fams}{gml}{nan}
\define@key{fams}{dum}{nan}
\define@key{fams}{mpl}{nan}
\define@key{fams}{gmh}{nan}
\define@key{fams}{ltc}{nan}
\define@key{fams}{xng}{nan}
\define@key{fams}{dnt}{nan}
\define@key{fams}{bjo}{nan}
\define@key{fams}{mpp}{nan}
\define@key{fams}{ymh}{nan}
\define@key{fams}{mlj}{nan}
\define@key{fams}{iml}{nan}
\define@key{fams}{imy}{nan}
\define@key{fams}{mcv}{nan}
\define@key{fams}{inm}{nan}
\define@key{fams}{mnp}{nan}
\define@key{fams}{mpn}{nan}
\define@key{fams}{drc}{nan}
\define@key{fams}{mko}{nan}
\define@key{fams}{vmg}{nan}
\define@key{fams}{wii}{nan}
\define@key{fams}{xxm}{nan}
\define@key{fams}{omn}{nan}
\define@key{fams}{mqq}{nan}
\define@key{fams}{mnq}{nan}
\define@key{fams}{mzt}{nan}
\define@key{fams}{czo}{nan}
\define@key{fams}{zgm}{nan}
\define@key{fams}{yiq}{nan}
\define@key{fams}{mwl}{nan}
\define@key{fams}{mvh}{nan}
\define@key{fams}{mmv}{nan}
\define@key{fams}{rsm}{nan}
\define@key{fams}{mjs}{nan}
\define@key{fams}{mpx}{nan}
\define@key{fams}{vmm}{nan}
\define@key{fams}{mwu}{nan}
\define@key{fams}{mpo}{nan}
\define@key{fams}{vmi}{nan}
\define@key{fams}{mfg}{nan}
\define@key{fams}{mix}{nan}
\define@key{fams}{mvi}{nan}
\define@key{fams}{ehs}{nan}
\define@key{fams}{soy}{nan}
\define@key{fams}{lhs}{nan}
\define@key{fams}{kja}{nan}
\define@key{fams}{mlo}{nan}
\define@key{fams}{mmu}{nan}
\define@key{fams}{bfm}{nan}
\define@key{fams}{mfq}{nan}
\define@key{fams}{mod}{nan}
\define@key{fams}{ahm}{nan}
\define@key{fams}{jkm}{nan}
\define@key{fams}{mhn}{nan}
\define@key{fams}{mhc}{nan}
\define@key{fams}{gbn}{nan}
\define@key{fams}{mxd}{nan}
\define@key{fams}{mqo}{nan}
\define@key{fams}{mvq}{nan}
\define@key{fams}{mou}{nan}
\define@key{fams}{mof}{nan}
\define@key{fams}{mow}{nan}
\define@key{fams}{mxn}{nan}
\define@key{fams}{mkp}{nan}
\define@key{fams}{mwz}{nan}
\define@key{fams}{ymi}{nan}
\define@key{fams}{mft}{nan}
\define@key{fams}{mwt}{nan}
\define@key{fams}{mqt}{nan}
\define@key{fams}{mkm}{nan}
\define@key{fams}{mkl}{nan}
\define@key{fams}{vms}{nan}
\define@key{fams}{pwm}{nan}
\define@key{fams}{vsi}{nan}
\define@key{fams}{bxc}{nan}
\define@key{fams}{mox}{nan}
\define@key{fams}{zmo}{nan}
\define@key{fams}{msl}{nan}
\define@key{fams}{mlw}{nan}
\define@key{fams}{myl}{nan}
\define@key{fams}{msz}{nan}
\define@key{fams}{dmb}{nan}
\define@key{fams}{mmb}{nan}
\define@key{fams}{ver}{nan}
\define@key{fams}{mzg}{nan}
\define@key{fams}{npn}{nan}
\define@key{fams}{msr}{nan}
\define@key{fams}{mgt}{nan}
\define@key{fams}{mom}{nan}
\define@key{fams}{moo}{nan}
\define@key{fams}{mru}{nan}
\define@key{fams}{mnh}{nan}
\define@key{fams}{nmh}{nan}
\define@key{fams}{mtl}{nan}
\define@key{fams}{gwg}{nan}
\define@key{fams}{crm}{nan}
\define@key{fams}{msg}{nan}
\define@key{fams}{mze}{nan}
\define@key{fams}{moq}{nan}
\define@key{fams}{msx}{nan}
\define@key{fams}{xmo}{nan}
\define@key{fams}{xmz}{nan}
\define@key{fams}{mzq}{nan}
\define@key{fams}{mdb}{nan}
\define@key{fams}{xms}{nan}
\define@key{fams}{bdo}{nan}
\define@key{fams}{mgc}{nan}
\define@key{fams}{mrp}{nan}
\define@key{fams}{mqn}{nan}
\define@key{fams}{mrl}{nan}
\define@key{fams}{mwy}{nan}
\define@key{fams}{mqv}{nan}
\define@key{fams}{mtj}{nan}
\define@key{fams}{mtt}{nan}
\define@key{fams}{mwh}{nan}
\define@key{fams}{jmw}{nan}
\define@key{fams}{ity}{nan}
\define@key{fams}{nmo}{nan}
\define@key{fams}{mzy}{nan}
\define@key{fams}{mxi}{nan}
\define@key{fams}{xnq}{nan}
\define@key{fams}{mpi}{nan}
\define@key{fams}{mcx}{nan}
\define@key{fams}{mpz}{nan}
\define@key{fams}{pnd}{nan}
\define@key{fams}{mgg}{nan}
\define@key{fams}{mpa}{nan}
\define@key{fams}{mvt}{nan}
\define@key{fams}{zmp}{nan}
\define@key{fams}{cmr}{nan}
\define@key{fams}{mro}{nan}
\define@key{fams}{kqx}{nan}
\define@key{fams}{agz}{nan}
\define@key{fams}{atl}{nan}
\define@key{fams}{mtd}{nan}
\define@key{fams}{tsx}{nan}
\define@key{fams}{mub}{nan}
\define@key{fams}{ymd}{nan}
\define@key{fams}{gau}{nan}
\define@key{fams}{udg}{nan}
\define@key{fams}{vmd}{nan}
\define@key{fams}{wiv}{nan}
\define@key{fams}{muk}{nan}
\define@key{fams}{mmk}{nan}
\define@key{fams}{mfw}{nan}
\define@key{fams}{kpb}{nan}
\define@key{fams}{vmu}{nan}
\define@key{fams}{kqa}{nan}
\define@key{fams}{mwq}{nan}
\define@key{fams}{boe}{nan}
\define@key{fams}{mmf}{nan}
\define@key{fams}{mth}{nan}
\define@key{fams}{mpv}{nan}
\define@key{fams}{mtc}{nan}
\define@key{fams}{myr}{nan}
\define@key{fams}{mnj}{nan}
\define@key{fams}{asx}{nan}
\define@key{fams}{mxr}{nan}
\define@key{fams}{rmh}{nan}
\define@key{fams}{tkv}{nan}
\define@key{fams}{mqw}{nan}
\define@key{fams}{smm}{nan}
\define@key{fams}{mmi}{nan}
\define@key{fams}{mmq}{nan}
\define@key{fams}{mse}{nan}
\define@key{fams}{mui}{nan}
\define@key{fams}{mje}{nan}
\define@key{fams}{muv}{nan}
\define@key{fams}{tuc}{nan}
\define@key{fams}{muy}{nan}
\define@key{fams}{ymz}{nan}
\define@key{fams}{mcj}{nan}
\define@key{fams}{mxh}{nan}
\define@key{fams}{wlc}{nan}
\define@key{fams}{wmw}{nan}
\define@key{fams}{moa}{nan}
\define@key{fams}{mwa}{nan}
\define@key{fams}{mjh}{nan}
\define@key{fams}{mws}{nan}
\define@key{fams}{gmy}{nan}
\define@key{fams}{nme}{nan}
\define@key{fams}{nbt}{nan}
\define@key{fams}{nao}{nan}
\define@key{fams}{mne}{nan}
\define@key{fams}{mty}{nan}
\define@key{fams}{ncd}{nan}
\define@key{fams}{srf}{nan}
\define@key{fams}{nxx}{nan}
\define@key{fams}{jbn}{nan}
\define@key{fams}{nbg}{nan}
\define@key{fams}{nxe}{nan}
\define@key{fams}{ngv}{nan}
\define@key{fams}{nlx}{nan}
\define@key{fams}{nhh}{nan}
\define@key{fams}{ars}{nan}
\define@key{fams}{nae}{nan}
\define@key{fams}{nib}{nan}
\define@key{fams}{nkj}{nan}
\define@key{fams}{nbk}{nan}
\define@key{fams}{mff}{nan}
\define@key{fams}{nax}{nan}
\define@key{fams}{nlc}{nan}
\define@key{fams}{nss}{nan}
\define@key{fams}{nlz}{nan}
\define@key{fams}{ylo}{nan}
\define@key{fams}{naj}{nan}
\define@key{fams}{nmx}{nan}
\define@key{fams}{nkm}{nan}
\define@key{fams}{nmk}{nan}
\define@key{fams}{nmq}{nan}
\define@key{fams}{ncm}{nan}
\define@key{fams}{neo}{nan}
\define@key{fams}{nbs}{nan}
\define@key{fams}{nvm}{nan}
\define@key{fams}{naa}{nan}
\define@key{fams}{mxw}{nan}
\define@key{fams}{nmt}{nan}
\define@key{fams}{bwb}{nan}
\define@key{fams}{nmy}{nan}
\define@key{fams}{nnc}{nan}
\define@key{fams}{nzz}{nan}
\define@key{fams}{ngr}{nan}
\define@key{fams}{cox}{nan}
\define@key{fams}{afk}{nan}
\define@key{fams}{qvo}{nan}
\define@key{fams}{nrg}{nan}
\define@key{fams}{nac}{nan}
\define@key{fams}{loh}{nan}
\define@key{fams}{nnr}{nan}
\define@key{fams}{nsy}{nan}
\define@key{fams}{nvh}{nan}
\define@key{fams}{ntz}{nan}
\define@key{fams}{nte}{nan}
\define@key{fams}{nti}{nan}
\define@key{fams}{nxa}{nan}
\define@key{fams}{ncn}{nan}
\define@key{fams}{nwo}{nan}
\define@key{fams}{nsw}{nan}
\define@key{fams}{nwr}{nan}
\define@key{fams}{nwa}{nan}
\define@key{fams}{nmz}{nan}
\define@key{fams}{naw}{nan}
\define@key{fams}{nyq}{nan}
\define@key{fams}{noz}{nan}
\define@key{fams}{ncr}{nan}
\define@key{fams}{nlu}{nan}
\define@key{fams}{gke}{nan}
\define@key{fams}{ndk}{nan}
\define@key{fams}{ndh}{nan}
\define@key{fams}{ndj}{nan}
\define@key{fams}{ndm}{nan}
\define@key{fams}{nxo}{nan}
\define@key{fams}{nnz}{nan}
\define@key{fams}{nda}{nan}
\define@key{fams}{ndc}{nan}
\define@key{fams}{nml}{nan}
\define@key{fams}{ndg}{nan}
\define@key{fams}{dne}{nan}
\define@key{fams}{ndd}{nan}
\define@key{fams}{eli}{nan}
\define@key{fams}{ndw}{nan}
\define@key{fams}{nbb}{nan}
\define@key{fams}{ndl}{nan}
\define@key{fams}{ndq}{nan}
\define@key{fams}{nqm}{nan}
\define@key{fams}{ndr}{nan}
\define@key{fams}{ndp}{nan}
\define@key{fams}{dno}{nan}
\define@key{fams}{ndx}{nan}
\define@key{fams}{nuh}{nan}
\define@key{fams}{nww}{nan}
\define@key{fams}{njt}{nan}
\define@key{fams}{wni}{nan}
\define@key{fams}{nec}{nan}
\define@key{fams}{nef}{nan}
\define@key{fams}{dcr}{nan}
\define@key{fams}{nkg}{nan}
\define@key{fams}{nif}{nan}
\define@key{fams}{nej}{nan}
\define@key{fams}{nek}{nan}
\define@key{fams}{nex}{nan}
\define@key{fams}{nem}{nan}
\define@key{fams}{nqn}{nan}
\define@key{fams}{neu}{nan}
\define@key{fams}{nsp}{nan}
\define@key{fams}{net}{nan}
\define@key{fams}{jas}{nan}
\define@key{fams}{jui}{nan}
\define@key{fams}{nnf}{nan}
\define@key{fams}{hlt}{nan}
\define@key{fams}{szb}{nan}
\define@key{fams}{nud}{nan}
\define@key{fams}{nmv}{nan}
\define@key{fams}{nbv}{nan}
\define@key{fams}{nmc}{nan}
\define@key{fams}{nbh}{nan}
\define@key{fams}{nyx}{nan}
\define@key{fams}{gng}{nan}
\define@key{fams}{nne}{nan}
\define@key{fams}{nxd}{nan}
\define@key{fams}{ngd}{nan}
\define@key{fams}{nji}{nan}
\define@key{fams}{rxd}{nan}
\define@key{fams}{nsg}{nan}
\define@key{fams}{ngm}{nan}
\define@key{fams}{cnw}{nan}
\define@key{fams}{zdj}{nan}
\define@key{fams}{ngg}{nan}
\define@key{fams}{jgb}{nan}
\define@key{fams}{nbd}{nan}
\define@key{fams}{nuu}{nan}
\define@key{fams}{gnj}{nan}
\define@key{fams}{nql}{nan}
\define@key{fams}{ngt}{nan}
\define@key{fams}{nnn}{nan}
\define@key{fams}{nbq}{nan}
\define@key{fams}{ngx}{nan}
\define@key{fams}{nnh}{nan}
\define@key{fams}{ngj}{nan}
\define@key{fams}{nnq}{nan}
\define@key{fams}{nra}{nan}
\define@key{fams}{nla}{nan}
\define@key{fams}{jgo}{nan}
\define@key{fams}{noq}{nan}
\define@key{fams}{nsh}{nan}
\define@key{fams}{nuw}{nan}
\define@key{fams}{ngp}{nan}
\define@key{fams}{nlo}{nan}
\define@key{fams}{xnm}{nan}
\define@key{fams}{nui}{nan}
\define@key{fams}{nue}{nan}
\define@key{fams}{ndn}{nan}
\define@key{fams}{ngz}{nan}
\define@key{fams}{nuo}{nan}
\define@key{fams}{nrx}{nan}
\define@key{fams}{nbx}{nan}
\define@key{fams}{ngq}{nan}
\define@key{fams}{ngw}{nan}
\define@key{fams}{nwe}{nan}
\define@key{fams}{ngn}{nan}
\define@key{fams}{yrl}{nan}
\define@key{fams}{nhf}{nan}
\define@key{fams}{ncs}{nan}
\define@key{fams}{nsi}{nan}
\define@key{fams}{mzk}{nan}
\define@key{fams}{nii}{nan}
\define@key{fams}{xny}{nan}
\define@key{fams}{gbe}{nan}
\define@key{fams}{nim}{nan}
\define@key{fams}{nil}{nan}
\define@key{fams}{noe}{nan}
\define@key{fams}{nmp}{nan}
\define@key{fams}{nmr}{nan}
\define@key{fams}{nis}{nan}
\define@key{fams}{nmw}{nan}
\define@key{fams}{niw}{nan}
\define@key{fams}{nxi}{nan}
\define@key{fams}{nxr}{nan}
\define@key{fams}{nby}{nan}
\define@key{fams}{nlk}{nan}
\define@key{fams}{nin}{nan}
\define@key{fams}{nps}{nan}
\define@key{fams}{njs}{nan}
\define@key{fams}{yso}{nan}
\define@key{fams}{nkp}{nan}
\define@key{fams}{njl}{nan}
\define@key{fams}{nzb}{nan}
\define@key{fams}{njj}{nan}
\define@key{fams}{njr}{nan}
\define@key{fams}{njy}{nan}
\define@key{fams}{nkq}{nan}
\define@key{fams}{nkn}{nan}
\define@key{fams}{nkz}{nan}
\define@key{fams}{khu}{nan}
\define@key{fams}{nqo}{nan}
\define@key{fams}{nkc}{nan}
\define@key{fams}{nkx}{nan}
\define@key{fams}{nka}{nan}
\define@key{fams}{nbo}{nan}
\define@key{fams}{nkw}{nan}
\define@key{fams}{nbp}{nan}
\define@key{fams}{ngh}{nan}
\define@key{fams}{gaw}{nan}
\define@key{fams}{noi}{nan}
\define@key{fams}{nkk}{nan}
\define@key{fams}{lem}{nan}
\define@key{fams}{nof}{nan}
\define@key{fams}{noh}{nan}
\define@key{fams}{zhn}{nan}
\define@key{fams}{noj}{nan}
\define@key{fams}{nok}{nan}
\define@key{fams}{nrc}{nan}
\define@key{fams}{nrp}{nan}
\define@key{fams}{huj}{nan}
\define@key{fams}{hmp}{nan}
\define@key{fams}{crl}{nan}
\define@key{fams}{pbu}{nan}
\define@key{fams}{hno}{nan}
\define@key{fams}{glh}{nan}
\define@key{fams}{aee}{nan}
\define@key{fams}{kxm}{nan}
\define@key{fams}{atv}{nan}
\define@key{fams}{azj}{nan}
\define@key{fams}{ghh}{nan}
\define@key{fams}{ymx}{nan}
\define@key{fams}{yiv}{nan}
\define@key{fams}{cng}{nan}
\define@key{fams}{bfc}{nan}
\define@key{fams}{nnl}{nan}
\define@key{fams}{lbr}{nan}
\define@key{fams}{tji}{nan}
\define@key{fams}{doc}{nan}
\define@key{fams}{nod}{nan}
\define@key{fams}{tts}{nan}
\define@key{fams}{hea}{nan}
\define@key{fams}{hmi}{nan}
\define@key{fams}{kqs}{nan}
\define@key{fams}{fll}{nan}
\define@key{fams}{dgi}{nan}
\define@key{fams}{tsp}{nan}
\define@key{fams}{gbo}{nan}
\define@key{fams}{dip}{nan}
\define@key{fams}{diw}{nan}
\define@key{fams}{max}{nan}
\define@key{fams}{mmg}{nan}
\define@key{fams}{mrq}{nan}
\define@key{fams}{tnn}{nan}
\define@key{fams}{una}{nan}
\define@key{fams}{bcd}{nan}
\define@key{fams}{weo}{nan}
\define@key{fams}{nni}{nan}
\define@key{fams}{aqn}{nan}
\define@key{fams}{xnn}{nan}
\define@key{fams}{cts}{nan}
\define@key{fams}{stb}{nan}
\define@key{fams}{bmm}{nan}
\define@key{fams}{onr}{nan}
\define@key{fams}{kti}{nan}
\define@key{fams}{nks}{nan}
\define@key{fams}{yir}{nan}
\define@key{fams}{whg}{nan}
\define@key{fams}{kiw}{nan}
\define@key{fams}{ryn}{nan}
\define@key{fams}{neq}{nan}
\define@key{fams}{scs}{nan}
\define@key{fams}{esk}{nan}
\define@key{fams}{thh}{nan}
\define@key{fams}{nhy}{nan}
\define@key{fams}{ojb}{nan}
\define@key{fams}{pef}{nan}
\define@key{fams}{cst}{nan}
\define@key{fams}{enl}{nan}
\define@key{fams}{qvz}{nan}
\define@key{fams}{qul}{nan}
\define@key{fams}{qxn}{nan}
\define@key{fams}{pmq}{nan}
\define@key{fams}{xtn}{nan}
\define@key{fams}{mxa}{nan}
\define@key{fams}{mfk}{nan}
\define@key{fams}{ayp}{nan}
\define@key{fams}{ntd}{nan}
\define@key{fams}{cnp}{nan}
\define@key{fams}{ncq}{nan}
\define@key{fams}{bly}{nan}
\define@key{fams}{ncf}{nan}
\define@key{fams}{ntw}{nan}
\define@key{fams}{nov}{nan}
\define@key{fams}{noy}{nan}
\define@key{fams}{asj}{nan}
\define@key{fams}{nsc}{nan}
\define@key{fams}{nsx}{nan}
\define@key{fams}{baf}{nan}
\define@key{fams}{kte}{nan}
\define@key{fams}{wbm}{nan}
\define@key{fams}{bsq}{nan}
\define@key{fams}{wla}{nan}
\define@key{fams}{wgi}{nan}
\define@key{fams}{gyz}{nan}
\define@key{fams}{nqt}{nan}
\define@key{fams}{nnv}{nan}
\define@key{fams}{noc}{nan}
\define@key{fams}{klt}{nan}
\define@key{fams}{nuq}{nan}
\define@key{fams}{nur}{nan}
\define@key{fams}{nuc}{nan}
\define@key{fams}{nbr}{nan}
\define@key{fams}{nop}{nan}
\define@key{fams}{sij}{nan}
\define@key{fams}{tgs}{nan}
\define@key{fams}{kdk}{nan}
\define@key{fams}{nxm}{nan}
\define@key{fams}{nug}{nan}
\define@key{fams}{rin}{nan}
\define@key{fams}{nul}{nan}
\define@key{fams}{nwb}{nan}
\define@key{fams}{nev}{nan}
\define@key{fams}{nyy}{nan}
\define@key{fams}{nlj}{nan}
\define@key{fams}{mwn}{nan}
\define@key{fams}{nwm}{nan}
\define@key{fams}{nmi}{nan}
\define@key{fams}{nny}{nan}
\define@key{fams}{nyb}{nan}
\define@key{fams}{nyc}{nan}
\define@key{fams}{nyk}{nan}
\define@key{fams}{nnj}{nan}
\define@key{fams}{sev}{nan}
\define@key{fams}{nba}{nan}
\define@key{fams}{neh}{nan}
\define@key{fams}{nye}{nan}
\define@key{fams}{nyl}{nan}
\define@key{fams}{nyr}{nan}
\define@key{fams}{nkv}{nan}
\define@key{fams}{nkt}{nan}
\define@key{fams}{nyg}{nan}
\define@key{fams}{lid}{nan}
\define@key{fams}{nvo}{nan}
\define@key{fams}{nuj}{nan}
\define@key{fams}{muo}{nan}
\define@key{fams}{nyd}{nan}
\define@key{fams}{nyu}{nan}
\define@key{fams}{nzd}{nan}
\define@key{fams}{nzy}{nan}
\define@key{fams}{nja}{nan}
\define@key{fams}{nzi}{nan}
\define@key{fams}{bzy}{nan}
\define@key{fams}{obi}{nan}
\define@key{fams}{obl}{nan}
\define@key{fams}{obo}{nan}
\define@key{fams}{obu}{nan}
\define@key{fams}{zac}{nan}
\define@key{fams}{odk}{nan}
\define@key{fams}{bhf}{nan}
\define@key{fams}{kkc}{nan}
\define@key{fams}{odu}{nan}
\define@key{fams}{tyh}{nan}
\define@key{fams}{opy}{nan}
\define@key{fams}{ofo}{nan}
\define@key{fams}{ogc}{nan}
\define@key{fams}{ogg}{nan}
\define@key{fams}{eri}{nan}
\define@key{fams}{oia}{nan}
\define@key{fams}{chj}{nan}
\define@key{fams}{oki}{nan}
\define@key{fams}{okn}{nan}
\define@key{fams}{okb}{nan}
\define@key{fams}{okd}{nan}
\define@key{fams}{oks}{nan}
\define@key{fams}{okj}{nan}
\define@key{fams}{kqv}{nan}
\define@key{fams}{oie}{nan}
\define@key{fams}{opa}{nan}
\define@key{fams}{okx}{nan}
\define@key{fams}{oke}{nan}
\define@key{fams}{oar}{nan}
\define@key{fams}{obr}{nan}
\define@key{fams}{och}{nan}
\define@key{fams}{odt}{nan}
\define@key{fams}{ang}{nan}
\define@key{fams}{fro}{nan}
\define@key{fams}{ofs}{nan}
\define@key{fams}{oge}{nan}
\define@key{fams}{goh}{nan}
\define@key{fams}{sga}{nan}
\define@key{fams}{ojp}{nan}
\define@key{fams}{okl}{nan}
\define@key{fams}{qok}{nan}
\define@key{fams}{qkn}{nan}
\define@key{fams}{qbb}{nan}
\define@key{fams}{omx}{nan}
\define@key{fams}{omr}{nan}
\define@key{fams}{non}{nan}
\define@key{fams}{onw}{nan}
\define@key{fams}{oos}{nan}
\define@key{fams}{pro}{nan}
\define@key{fams}{peo}{nan}
\define@key{fams}{orv}{nan}
\define@key{fams}{osp}{nan}
\define@key{fams}{osx}{nan}
\define@key{fams}{oty}{nan}
\define@key{fams}{oui}{nan}
\define@key{fams}{owl}{nan}
\define@key{fams}{ole}{nan}
\define@key{fams}{olm}{nan}
\define@key{fams}{lul}{nan}
\define@key{fams}{iko}{nan}
\define@key{fams}{acx}{nan}
\define@key{fams}{oml}{nan}
\define@key{fams}{nht}{nan}
\define@key{fams}{omi}{nan}
\define@key{fams}{omt}{nan}
\define@key{fams}{omu}{nan}
\define@key{fams}{oog}{nan}
\define@key{fams}{onx}{nan}
\define@key{fams}{oni}{nan}
\define@key{fams}{onj}{nan}
\define@key{fams}{onn}{nan}
\define@key{fams}{oor}{nan}
\define@key{fams}{opo}{nan}
\define@key{fams}{opt}{nan}
\define@key{fams}{lgn}{nan}
\define@key{fams}{orn}{nan}
\define@key{fams}{ors}{nan}
\define@key{fams}{sdr}{nan}
\define@key{fams}{org}{nan}
\define@key{fams}{nlv}{nan}
\define@key{fams}{fnb}{nan}
\define@key{fams}{orc}{nan}
\define@key{fams}{orz}{nan}
\define@key{fams}{ora}{nan}
\define@key{fams}{orx}{nan}
\define@key{fams}{orh}{nan}
\define@key{fams}{bpk}{nan}
\define@key{fams}{orw}{nan}
\define@key{fams}{orr}{nan}
\define@key{fams}{syx}{nan}
\define@key{fams}{ost}{nan}
\define@key{fams}{osc}{nan}
\define@key{fams}{osi}{nan}
\define@key{fams}{oso}{nan}
\define@key{fams}{uta}{nan}
\define@key{fams}{otd}{nan}
\define@key{fams}{oti}{nan}
\define@key{fams}{otw}{nan}
\define@key{fams}{lot}{nan}
\define@key{fams}{otu}{nan}
\define@key{fams}{oum}{nan}
\define@key{fams}{oue}{nan}
\define@key{fams}{stn}{nan}
\define@key{fams}{wsr}{nan}
\define@key{fams}{oyy}{nan}
\define@key{fams}{oyd}{nan}
\define@key{fams}{zao}{nan}
\define@key{fams}{chz}{nan}
\define@key{fams}{pfa}{nan}
\define@key{fams}{sig}{nan}
\define@key{fams}{qvp}{nan}
\define@key{fams}{pcp}{nan}
\define@key{fams}{pdi}{nan}
\define@key{fams}{pkc}{nan}
\define@key{fams}{pae}{nan}
\define@key{fams}{pgi}{nan}
\define@key{fams}{phr}{nan}
\define@key{fams}{phj}{nan}
\define@key{fams}{lgt}{nan}
\define@key{fams}{phv}{nan}
\define@key{fams}{pal}{nan}
\define@key{fams}{pha}{nan}
\define@key{fams}{pri}{nan}
\define@key{fams}{ppi}{nan}
\define@key{fams}{qpp}{nan}
\define@key{fams}{pta}{nan}
\define@key{fams}{pkg}{nan}
\define@key{fams}{jkp}{nan}
\define@key{fams}{pku}{nan}
\define@key{fams}{pfl}{nan}
\define@key{fams}{plq}{nan}
\define@key{fams}{plr}{nan}
\define@key{fams}{pln}{nan}
\define@key{fams}{pnl}{nan}
\define@key{fams}{pli}{nan}
\define@key{fams}{pcf}{nan}
\define@key{fams}{pmd}{nan}
\define@key{fams}{abw}{nan}
\define@key{fams}{pmc}{nan}
\define@key{fams}{ple}{nan}
\define@key{fams}{plz}{nan}
\define@key{fams}{bpx}{nan}
\define@key{fams}{pmb}{nan}
\define@key{fams}{pmn}{nan}
\define@key{fams}{hih}{nan}
\define@key{fams}{att}{nan}
\define@key{fams}{pnz}{nan}
\define@key{fams}{pnq}{nan}
\define@key{fams}{pwb}{nan}
\define@key{fams}{psn}{nan}
\define@key{fams}{qxh}{nan}
\define@key{fams}{lsp}{nan}
\define@key{fams}{tdb}{nan}
\define@key{fams}{pnp}{nan}
\define@key{fams}{bkj}{nan}
\define@key{fams}{pgg}{nan}
\define@key{fams}{pgs}{nan}
\define@key{fams}{slm}{nan}
\define@key{fams}{pcg}{nan}
\define@key{fams}{pnr}{nan}
\define@key{fams}{pax}{nan}
\define@key{fams}{pkh}{nan}
\define@key{fams}{paz}{nan}
\define@key{fams}{pnc}{nan}
\define@key{fams}{knt}{nan}
\define@key{fams}{pno}{nan}
\define@key{fams}{blk}{nan}
\define@key{fams}{ppv}{nan}
\define@key{fams}{ppn}{nan}
\define@key{fams}{dpp}{nan}
\define@key{fams}{pas}{nan}
\define@key{fams}{pbo}{nan}
\define@key{fams}{ppe}{nan}
\define@key{fams}{ppu}{nan}
\define@key{fams}{ppm}{nan}
\define@key{fams}{pgz}{nan}
\define@key{fams}{prc}{nan}
\define@key{fams}{pzn}{nan}
\define@key{fams}{prf}{nan}
\define@key{fams}{prw}{nan}
\define@key{fams}{aap}{nan}
\define@key{fams}{pak}{nan}
\define@key{fams}{paf}{nan}
\define@key{fams}{gvp}{nan}
\define@key{fams}{pbg}{nan}
\define@key{fams}{pys}{nan}
\define@key{fams}{pcl}{nan}
\define@key{fams}{pch}{nan}
\define@key{fams}{pcj}{nan}
\define@key{fams}{ppt}{nan}
\define@key{fams}{kvx}{nan}
\define@key{fams}{xpr}{nan}
\define@key{fams}{paq}{nan}
\define@key{fams}{psq}{nan}
\define@key{fams}{yac}{nan}
\define@key{fams}{ptn}{nan}
\define@key{fams}{pth}{nan}
\define@key{fams}{pbc}{nan}
\define@key{fams}{pty}{nan}
\define@key{fams}{ptq}{nan}
\define@key{fams}{mfa}{nan}
\define@key{fams}{pnk}{nan}
\define@key{fams}{bfb}{nan}
\define@key{fams}{psm}{nan}
\define@key{fams}{pmr}{nan}
\define@key{fams}{pcb}{nan}
\define@key{fams}{xpc}{nan}
\define@key{fams}{pai}{nan}
\define@key{fams}{pfe}{nan}
\define@key{fams}{ppq}{nan}
\define@key{fams}{pel}{nan}
\define@key{fams}{bxd}{nan}
\define@key{fams}{ata}{nan}
\define@key{fams}{pev}{nan}
\define@key{fams}{psg}{nan}
\define@key{fams}{pek}{nan}
\define@key{fams}{ums}{nan}
\define@key{fams}{pdc}{nan}
\define@key{fams}{pnh}{nan}
\define@key{fams}{ptw}{nan}
\define@key{fams}{pea}{nan}
\define@key{fams}{wet}{nan}
\define@key{fams}{psc}{nan}
\define@key{fams}{prl}{nan}
\define@key{fams}{pex}{nan}
\define@key{fams}{zpe}{nan}
\define@key{fams}{pey}{nan}
\define@key{fams}{prt}{nan}
\define@key{fams}{phk}{nan}
\define@key{fams}{phl}{nan}
\define@key{fams}{ypa}{nan}
\define@key{fams}{phq}{nan}
\define@key{fams}{pem}{nan}
\define@key{fams}{psp}{nan}
\define@key{fams}{phm}{nan}
\define@key{fams}{phn}{nan}
\define@key{fams}{yip}{nan}
\define@key{fams}{ypg}{nan}
\define@key{fams}{nph}{nan}
\define@key{fams}{pnx}{nan}
\define@key{fams}{kjt}{nan}
\define@key{fams}{xpg}{nan}
\define@key{fams}{phu}{nan}
\define@key{fams}{phd}{nan}
\define@key{fams}{pug}{nan}
\define@key{fams}{phh}{nan}
\define@key{fams}{ypm}{nan}
\define@key{fams}{pho}{nan}
\define@key{fams}{phg}{nan}
\define@key{fams}{yph}{nan}
\define@key{fams}{ypp}{nan}
\define@key{fams}{pht}{nan}
\define@key{fams}{ypz}{nan}
\define@key{fams}{ptr}{nan}
\define@key{fams}{pin}{nan}
\define@key{fams}{pcd}{nan}
\define@key{fams}{cpu}{nan}
\define@key{fams}{xpi}{nan}
\define@key{fams}{dep}{nan}
\define@key{fams}{pij}{nan}
\define@key{fams}{piz}{nan}
\define@key{fams}{pis}{nan}
\define@key{fams}{piw}{nan}
\define@key{fams}{pnn}{nan}
\define@key{fams}{pnv}{nan}
\define@key{fams}{tjp}{nan}
\define@key{fams}{pic}{nan}
\define@key{fams}{pti}{nan}
\define@key{fams}{pny}{nan}
\define@key{fams}{bxi}{nan}
\define@key{fams}{pie}{nan}
\define@key{fams}{xpa}{nan}
\define@key{fams}{tpp}{nan}
\define@key{fams}{pig}{nan}
\define@key{fams}{psy}{nan}
\define@key{fams}{xps}{nan}
\define@key{fams}{pih}{nan}
\define@key{fams}{sje}{nan}
\define@key{fams}{pcn}{nan}
\define@key{fams}{pix}{nan}
\define@key{fams}{piy}{nan}
\define@key{fams}{ktj}{nan}
\define@key{fams}{pdt}{nan}
\define@key{fams}{pbv}{nan}
\define@key{fams}{npo}{nan}
\define@key{fams}{pdn}{nan}
\define@key{fams}{pof}{nan}
\define@key{fams}{pkb}{nan}
\define@key{fams}{pld}{nan}
\define@key{fams}{plj}{nan}
\define@key{fams}{pso}{nan}
\define@key{fams}{plb}{nan}
\define@key{fams}{pmo}{nan}
\define@key{fams}{pmm}{nan}
\define@key{fams}{ncc}{nan}
\define@key{fams}{png}{nan}
\define@key{fams}{pns}{nan}
\define@key{fams}{pnt}{nan}
\define@key{fams}{prh}{nan}
\define@key{fams}{ptv}{nan}
\define@key{fams}{pmx}{nan}
\define@key{fams}{bye}{nan}
\define@key{fams}{pwr}{nan}
\define@key{fams}{pyn}{nan}
\define@key{fams}{prz}{nan}
\define@key{fams}{prg}{nan}
\define@key{fams}{kvj}{nan}
\define@key{fams}{pux}{nan}
\define@key{fams}{atp}{nan}
\define@key{fams}{pbm}{nan}
\define@key{fams}{psl}{nan}
\define@key{fams}{pkp}{nan}
\define@key{fams}{pup}{nan}
\define@key{fams}{pum}{nan}
\define@key{fams}{xpm}{nan}
\define@key{fams}{puj}{nan}
\define@key{fams}{pud}{nan}
\define@key{fams}{puf}{nan}
\define@key{fams}{pna}{nan}
\define@key{fams}{pnm}{nan}
\define@key{fams}{xpu}{nan}
\define@key{fams}{qxp}{nan}
\define@key{fams}{puu}{nan}
\define@key{fams}{pru}{nan}
\define@key{fams}{iar}{nan}
\define@key{fams}{puy}{nan}
\define@key{fams}{prr}{nan}
\define@key{fams}{pur}{nan}
\define@key{fams}{pub}{nan}
\define@key{fams}{mfl}{nan}
\define@key{fams}{afe}{nan}
\define@key{fams}{cpx}{nan}
\define@key{fams}{pyu}{nan}
\define@key{fams}{pme}{nan}
\define@key{fams}{pop}{nan}
\define@key{fams}{pwo}{nan}
\define@key{fams}{pcw}{nan}
\define@key{fams}{pye}{nan}
\define@key{fams}{pyy}{nan}
\define@key{fams}{pby}{nan}
\define@key{fams}{laq}{nan}
\define@key{fams}{qxq}{nan}
\define@key{fams}{xqt}{nan}
\define@key{fams}{ymq}{nan}
\define@key{fams}{zqe}{nan}
\define@key{fams}{qua}{nan}
\define@key{fams}{qya}{nan}
\define@key{fams}{qvy}{nan}
\define@key{fams}{zpj}{nan}
\define@key{fams}{quq}{nan}
\define@key{fams}{qun}{nan}
\define@key{fams}{ztq}{nan}
\define@key{fams}{rah}{nan}
\define@key{fams}{xrr}{nan}
\define@key{fams}{raz}{nan}
\define@key{fams}{mqk}{nan}
\define@key{fams}{rjs}{nan}
\define@key{fams}{rjg}{nan}
\define@key{fams}{gra}{nan}
\define@key{fams}{rkh}{nan}
\define@key{fams}{rki}{nan}
\define@key{fams}{rai}{nan}
\define@key{fams}{kjx}{nan}
\define@key{fams}{lje}{nan}
\define@key{fams}{thr}{nan}
\define@key{fams}{rkt}{nan}
\define@key{fams}{rnl}{nan}
\define@key{fams}{rax}{nan}
\define@key{fams}{ray}{nan}
\define@key{fams}{rpt}{nan}
\define@key{fams}{lra}{nan}
\define@key{fams}{rar}{nan}
\define@key{fams}{rac}{nan}
\define@key{fams}{btn}{nan}
\define@key{fams}{bgd}{nan}
\define@key{fams}{rtw}{nan}
\define@key{fams}{rau}{nan}
\define@key{fams}{yea}{nan}
\define@key{fams}{jnl}{nan}
\define@key{fams}{rat}{nan}
\define@key{fams}{gir}{nan}
\define@key{fams}{atu}{nan}
\define@key{fams}{ree}{nan}
\define@key{fams}{rei}{nan}
\define@key{fams}{bow}{nan}
\define@key{fams}{reb}{nan}
\define@key{fams}{agv}{nan}
\define@key{fams}{rem}{nan}
\define@key{fams}{rmp}{nan}
\define@key{fams}{lkj}{nan}
\define@key{fams}{rsi}{nan}
\define@key{fams}{rea}{nan}
\define@key{fams}{rer}{nan}
\define@key{fams}{pgk}{nan}
\define@key{fams}{res}{nan}
\define@key{fams}{ret}{nan}
\define@key{fams}{rcf}{nan}
\define@key{fams}{rey}{nan}
\define@key{fams}{ril}{nan}
\define@key{fams}{ria}{nan}
\define@key{fams}{rir}{nan}
\define@key{fams}{zar}{nan}
\define@key{fams}{rgu}{nan}
\define@key{fams}{hrx}{nan}
\define@key{fams}{rri}{nan}
\define@key{fams}{riu}{nan}
\define@key{fams}{snj}{nan}
\define@key{fams}{rod}{nan}
\define@key{fams}{rhg}{nan}
\define@key{fams}{rge}{nan}
\define@key{fams}{rms}{nan}
\define@key{fams}{rgn}{nan}
\define@key{fams}{rmx}{nan}
\define@key{fams}{rmm}{nan}
\define@key{fams}{rmv}{nan}
\define@key{fams}{rof}{nan}
\define@key{fams}{rol}{nan}
\define@key{fams}{rmk}{nan}
\define@key{fams}{ror}{nan}
\define@key{fams}{roe}{nan}
\define@key{fams}{rnn}{nan}
\define@key{fams}{rga}{nan}
\define@key{fams}{pce}{nan}
\define@key{fams}{rdb}{nan}
\define@key{fams}{ruh}{nan}
\define@key{fams}{rbb}{nan}
\define@key{fams}{ruz}{nan}
\define@key{fams}{rna}{nan}
\define@key{fams}{rnw}{nan}
\define@key{fams}{drg}{nan}
\define@key{fams}{bxr}{nan}
\define@key{fams}{rue}{nan}
\define@key{fams}{ruc}{nan}
\define@key{fams}{rnd}{nan}
\define@key{fams}{rwk}{nan}
\define@key{fams}{rsn}{nan}
\define@key{fams}{sax}{nan}
\define@key{fams}{sav}{nan}
\define@key{fams}{raq}{nan}
\define@key{fams}{lsm}{nan}
\define@key{fams}{sxr}{nan}
\define@key{fams}{spy}{nan}
\define@key{fams}{msi}{nan}
\define@key{fams}{bsy}{nan}
\define@key{fams}{sae}{nan}
\define@key{fams}{saa}{nan}
\define@key{fams}{xsa}{nan}
\define@key{fams}{qhr}{nan}
\define@key{fams}{sbo}{nan}
\define@key{fams}{quv}{nan}
\define@key{fams}{sck}{nan}
\define@key{fams}{spd}{nan}
\define@key{fams}{saf}{nan}
\define@key{fams}{sbk}{nan}
\define@key{fams}{sbm}{nan}
\define@key{fams}{tga}{nan}
\define@key{fams}{aec}{nan}
\define@key{fams}{acf}{nan}
\define@key{fams}{xsy}{nan}
\define@key{fams}{sjl}{nan}
\define@key{fams}{sjb}{nan}
\define@key{fams}{sch}{nan}
\define@key{fams}{skt}{nan}
\define@key{fams}{skg}{nan}
\define@key{fams}{skm}{nan}
\define@key{fams}{sak}{nan}
\define@key{fams}{szy}{nan}
\define@key{fams}{shq}{nan}
\define@key{fams}{slx}{nan}
\define@key{fams}{sgu}{nan}
\define@key{fams}{qxl}{nan}
\define@key{fams}{mnd}{nan}
\define@key{fams}{slq}{nan}
\define@key{fams}{sau}{nan}
\define@key{fams}{loe}{nan}
\define@key{fams}{esn}{nan}
\define@key{fams}{tmj}{nan}
\define@key{fams}{ysd}{nan}
\define@key{fams}{smp}{nan}
\define@key{fams}{xab}{nan}
\define@key{fams}{smx}{nan}
\define@key{fams}{ccg}{nan}
\define@key{fams}{saq}{nan}
\define@key{fams}{ssx}{nan}
\define@key{fams}{spv}{nan}
\define@key{fams}{smh}{nan}
\define@key{fams}{snx}{nan}
\define@key{fams}{swm}{nan}
\define@key{fams}{rav}{nan}
\define@key{fams}{stu}{nan}
\define@key{fams}{smv}{nan}
\define@key{fams}{ztm}{nan}
\define@key{fams}{icr}{nan}
\define@key{fams}{spn}{nan}
\define@key{fams}{zpx}{nan}
\define@key{fams}{cuk}{nan}
\define@key{fams}{hve}{nan}
\define@key{fams}{hue}{nan}
\define@key{fams}{mat}{nan}
\define@key{fams}{pow}{nan}
\define@key{fams}{xso}{nan}
\define@key{fams}{sgr}{nan}
\define@key{fams}{sgk}{nan}
\define@key{fams}{nsa}{nan}
\define@key{fams}{xsn}{nan}
\define@key{fams}{sbp}{nan}
\define@key{fams}{sng}{nan}
\define@key{fams}{snl}{nan}
\define@key{fams}{scg}{nan}
\define@key{fams}{sgy}{nan}
\define@key{fams}{ysy}{nan}
\define@key{fams}{ysn}{nan}
\define@key{fams}{sny}{nan}
\define@key{fams}{xtj}{nan}
\define@key{fams}{maa}{nan}
\define@key{fams}{msc}{nan}
\define@key{fams}{pps}{nan}
\define@key{fams}{qvs}{nan}
\define@key{fams}{xtp}{nan}
\define@key{fams}{trq}{nan}
\define@key{fams}{pls}{nan}
\define@key{fams}{azg}{nan}
\define@key{fams}{zpf}{nan}
\define@key{fams}{san}{nan}
\define@key{fams}{ssi}{nan}
\define@key{fams}{kwy}{nan}
\define@key{fams}{hvv}{nan}
\define@key{fams}{nhz}{nan}
\define@key{fams}{cok}{nan}
\define@key{fams}{qus}{nan}
\define@key{fams}{mza}{nan}
\define@key{fams}{mdv}{nan}
\define@key{fams}{zpn}{nan}
\define@key{fams}{ztn}{nan}
\define@key{fams}{zas}{nan}
\define@key{fams}{zpr}{nan}
\define@key{fams}{pca}{nan}
\define@key{fams}{zpt}{nan}
\define@key{fams}{scq}{nan}
\define@key{fams}{zkp}{nan}
\define@key{fams}{cri}{nan}
\define@key{fams}{spr}{nan}
\define@key{fams}{spc}{nan}
\define@key{fams}{krn}{nan}
\define@key{fams}{spi}{nan}
\define@key{fams}{sbz}{nan}
\define@key{fams}{kwv}{nan}
\define@key{fams}{kwg}{nan}
\define@key{fams}{zsa}{nan}
\define@key{fams}{bps}{nan}
\define@key{fams}{mbs}{nan}
\define@key{fams}{sre}{nan}
\define@key{fams}{sar}{nan}
\define@key{fams}{srh}{nan}
\define@key{fams}{mwm}{nan}
\define@key{fams}{onp}{nan}
\define@key{fams}{sdu}{nan}
\define@key{fams}{sra}{nan}
\define@key{fams}{swy}{nan}
\define@key{fams}{sxs}{nan}
\define@key{fams}{sas}{nan}
\define@key{fams}{sdc}{nan}
\define@key{fams}{stw}{nan}
\define@key{fams}{stq}{nan}
\define@key{fams}{mav}{nan}
\define@key{fams}{sdl}{nan}
\define@key{fams}{skc}{nan}
\define@key{fams}{saz}{nan}
\define@key{fams}{mjt}{nan}
\define@key{fams}{srt}{nan}
\define@key{fams}{psu}{nan}
\define@key{fams}{ssj}{nan}
\define@key{fams}{sao}{nan}
\define@key{fams}{swr}{nan}
\define@key{fams}{swt}{nan}
\define@key{fams}{saw}{nan}
\define@key{fams}{swn}{nan}
\define@key{fams}{sxw}{nan}
\define@key{fams}{say}{nan}
\define@key{fams}{sco}{nan}
\define@key{fams}{kdg}{nan}
\define@key{fams}{sbx}{nan}
\define@key{fams}{sib}{nan}
\define@key{fams}{sec}{nan}
\define@key{fams}{tvw}{nan}
\define@key{fams}{sos}{nan}
\define@key{fams}{sge}{nan}
\define@key{fams}{sbg}{nan}
\define@key{fams}{seg}{nan}
\define@key{fams}{sfw}{nan}
\define@key{fams}{ssg}{nan}
\define@key{fams}{hik}{nan}
\define@key{fams}{skz}{nan}
\define@key{fams}{skp}{nan}
\define@key{fams}{sek}{nan}
\define@key{fams}{ske}{nan}
\define@key{fams}{syi}{nan}
\define@key{fams}{sko}{nan}
\define@key{fams}{skx}{nan}
\define@key{fams}{lip}{nan}
\define@key{fams}{kgi}{nan}
\define@key{fams}{snw}{nan}
\define@key{fams}{sws}{nan}
\define@key{fams}{slg}{nan}
\define@key{fams}{szc}{nan}
\define@key{fams}{sbr}{nan}
\define@key{fams}{etz}{nan}
\define@key{fams}{smy}{nan}
\define@key{fams}{ssm}{nan}
\define@key{fams}{xse}{nan}
\define@key{fams}{seq}{nan}
\define@key{fams}{sej}{nan}
\define@key{fams}{sds}{nan}
\define@key{fams}{ssz}{nan}
\define@key{fams}{spk}{nan}
\define@key{fams}{snu}{nan}
\define@key{fams}{sjs}{nan}
\define@key{fams}{sni}{nan}
\define@key{fams}{std}{nan}
\define@key{fams}{sez}{nan}
\define@key{fams}{spe}{nan}
\define@key{fams}{spb}{nan}
\define@key{fams}{spm}{nan}
\define@key{fams}{iws}{nan}
\define@key{fams}{skr}{nan}
\define@key{fams}{sry}{nan}
\define@key{fams}{srr}{nan}
\define@key{fams}{swf}{nan}
\define@key{fams}{sve}{nan}
\define@key{fams}{seu}{nan}
\define@key{fams}{srw}{nan}
\define@key{fams}{srk}{nan}
\define@key{fams}{stf}{nan}
\define@key{fams}{stm}{nan}
\define@key{fams}{sbi}{nan}
\define@key{fams}{sta}{nan}
\define@key{fams}{sew}{nan}
\define@key{fams}{lsw}{nan}
\define@key{fams}{sze}{nan}
\define@key{fams}{scw}{nan}
\define@key{fams}{sdb}{nan}
\define@key{fams}{srz}{nan}
\define@key{fams}{sha}{nan}
\define@key{fams}{xsh}{nan}
\define@key{fams}{sqa}{nan}
\define@key{fams}{jih}{nan}
\define@key{fams}{sho}{nan}
\define@key{fams}{swo}{nan}
\define@key{fams}{ssv}{nan}
\define@key{fams}{swq}{nan}
\define@key{fams}{sqh}{nan}
\define@key{fams}{shx}{nan}
\define@key{fams}{she}{nan}
\define@key{fams}{sth}{nan}
\define@key{fams}{shl}{nan}
\define@key{fams}{scv}{nan}
\define@key{fams}{bun}{nan}
\define@key{fams}{kip}{nan}
\define@key{fams}{ssh}{nan}
\define@key{fams}{shr}{nan}
\define@key{fams}{gua}{nan}
\define@key{fams}{snh}{nan}
\define@key{fams}{sxg}{nan}
\define@key{fams}{sle}{nan}
\define@key{fams}{bcv}{nan}
\define@key{fams}{suj}{nan}
\define@key{fams}{sts}{nan}
\define@key{fams}{scu}{nan}
\define@key{fams}{ksa}{nan}
\define@key{fams}{shw}{nan}
\define@key{fams}{slw}{nan}
\define@key{fams}{sya}{nan}
\define@key{fams}{spg}{nan}
\define@key{fams}{mmp}{nan}
\define@key{fams}{nco}{nan}
\define@key{fams}{sty}{nan}
\define@key{fams}{sdx}{nan}
\define@key{fams}{sxc}{nan}
\define@key{fams}{scn}{nan}
\define@key{fams}{sep}{nan}
\define@key{fams}{scx}{nan}
\define@key{fams}{xsd}{nan}
\define@key{fams}{sgx}{nan}
\define@key{fams}{nsu}{nan}
\define@key{fams}{sxe}{nan}
\define@key{fams}{snr}{nan}
\define@key{fams}{qws}{nan}
\define@key{fams}{sky}{nan}
\define@key{fams}{slt}{nan}
\define@key{fams}{szl}{nan}
\define@key{fams}{sbq}{nan}
\define@key{fams}{mkc}{nan}
\define@key{fams}{wul}{nan}
\define@key{fams}{xsp}{nan}
\define@key{fams}{stv}{nan}
\define@key{fams}{sie}{nan}
\define@key{fams}{sbw}{nan}
\define@key{fams}{smb}{nan}
\define@key{fams}{sbb}{nan}
\define@key{fams}{smg}{nan}
\define@key{fams}{smz}{nan}
\define@key{fams}{smt}{nan}
\define@key{fams}{siu}{nan}
\define@key{fams}{sbn}{nan}
\define@key{fams}{xts}{nan}
\define@key{fams}{sjn}{nan}
\define@key{fams}{sgp}{nan}
\define@key{fams}{sgm}{nan}
\define@key{fams}{skq}{nan}
\define@key{fams}{xti}{nan}
\define@key{fams}{snz}{nan}
\define@key{fams}{sys}{nan}
\define@key{fams}{swj}{nan}
\define@key{fams}{sir}{nan}
\define@key{fams}{srx}{nan}
\define@key{fams}{sld}{nan}
\define@key{fams}{sso}{nan}
\define@key{fams}{siy}{nan}
\define@key{fams}{lsv}{nan}
\define@key{fams}{akp}{nan}
\define@key{fams}{skw}{nan}
\define@key{fams}{sms}{nan}
\define@key{fams}{svm}{nan}
\define@key{fams}{svk}{nan}
\define@key{fams}{sfm}{nan}
\define@key{fams}{kxq}{nan}
\define@key{fams}{sox}{nan}
\define@key{fams}{soc}{nan}
\define@key{fams}{xog}{nan}
\define@key{fams}{sog}{nan}
\define@key{fams}{soj}{nan}
\define@key{fams}{sok}{nan}
\define@key{fams}{sby}{nan}
\define@key{fams}{sol}{nan}
\define@key{fams}{aaw}{nan}
\define@key{fams}{szs}{nan}
\define@key{fams}{smc}{nan}
\define@key{fams}{smu}{nan}
\define@key{fams}{sor}{nan}
\define@key{fams}{kgt}{nan}
\define@key{fams}{ysg}{nan}
\define@key{fams}{shc}{nan}
\define@key{fams}{soo}{nan}
\define@key{fams}{sod}{nan}
\define@key{fams}{soe}{nan}
\define@key{fams}{soi}{nan}
\define@key{fams}{siq}{nan}
\define@key{fams}{sss}{nan}
\define@key{fams}{urw}{nan}
\define@key{fams}{sbh}{nan}
\define@key{fams}{sqo}{nan}
\define@key{fams}{ays}{nan}
\define@key{fams}{sdk}{nan}
\define@key{fams}{krz}{nan}
\define@key{fams}{sfs}{nan}
\define@key{fams}{nit}{nan}
\define@key{fams}{hmy}{nan}
\define@key{fams}{hma}{nan}
\define@key{fams}{sdh}{nan}
\define@key{fams}{bcc}{nan}
\define@key{fams}{fay}{nan}
\define@key{fams}{luz}{nan}
\define@key{fams}{pbt}{nan}
\define@key{fams}{hnd}{nan}
\define@key{fams}{psh}{nan}
\define@key{fams}{psi}{nan}
\define@key{fams}{vro}{nan}
\define@key{fams}{nik}{nan}
\define@key{fams}{mnn}{nan}
\define@key{fams}{uzs}{nan}
\define@key{fams}{ghe}{nan}
\define@key{fams}{ymc}{nan}
\define@key{fams}{nsd}{nan}
\define@key{fams}{qxs}{nan}
\define@key{fams}{pmj}{nan}
\define@key{fams}{bfs}{nan}
\define@key{fams}{nre}{nan}
\define@key{fams}{lrr}{nan}
\define@key{fams}{tjs}{nan}
\define@key{fams}{sou}{nan}
\define@key{fams}{hms}{nan}
\define@key{fams}{hmh}{nan}
\define@key{fams}{hmg}{nan}
\define@key{fams}{xtv}{nan}
\define@key{fams}{ijs}{nan}
\define@key{fams}{fal}{nan}
\define@key{fams}{nbw}{nan}
\define@key{fams}{lnl}{nan}
\define@key{fams}{biv}{nan}
\define@key{fams}{nnw}{nan}
\define@key{fams}{snm}{nan}
\define@key{fams}{dik}{nan}
\define@key{fams}{dib}{nan}
\define@key{fams}{dks}{nan}
\define@key{fams}{bwq}{nan}
\define@key{fams}{sbd}{nan}
\define@key{fams}{sns}{nan}
\define@key{fams}{mqm}{nan}
\define@key{fams}{mcy}{nan}
\define@key{fams}{vbb}{nan}
\define@key{fams}{lmf}{nan}
\define@key{fams}{agy}{nan}
\define@key{fams}{ksc}{nan}
\define@key{fams}{bln}{nan}
\define@key{fams}{plv}{nan}
\define@key{fams}{bzc}{nan}
\define@key{fams}{osu}{nan}
\define@key{fams}{aws}{nan}
\define@key{fams}{omw}{nan}
\define@key{fams}{ams}{nan}
\define@key{fams}{hax}{nan}
\define@key{fams}{tce}{nan}
\define@key{fams}{caf}{nan}
\define@key{fams}{twr}{nan}
\define@key{fams}{tcu}{nan}
\define@key{fams}{npl}{nan}
\define@key{fams}{tla}{nan}
\define@key{fams}{crj}{nan}
\define@key{fams}{peq}{nan}
\define@key{fams}{qup}{nan}
\define@key{fams}{qxo}{nan}
\define@key{fams}{ayc}{nan}
\define@key{fams}{meh}{nan}
\define@key{fams}{mit}{nan}
\define@key{fams}{mxy}{nan}
\define@key{fams}{rgs}{nan}
\define@key{fams}{giz}{nan}
\define@key{fams}{cpy}{nan}
\define@key{fams}{itd}{nan}
\define@key{fams}{csp}{nan}
\define@key{fams}{sct}{nan}
\define@key{fams}{sqq}{nan}
\define@key{fams}{sww}{nan}
\define@key{fams}{sow}{nan}
\define@key{fams}{vmq}{nan}
\define@key{fams}{vmp}{nan}
\define@key{fams}{sqs}{nan}
\define@key{fams}{sci}{nan}
\define@key{fams}{seo}{nan}
\define@key{fams}{swp}{nan}
\define@key{fams}{sxb}{nan}
\define@key{fams}{ssc}{nan}
\define@key{fams}{sut}{nan}
\define@key{fams}{apd}{nan}
\define@key{fams}{pga}{nan}
\define@key{fams}{sgi}{nan}
\define@key{fams}{sug}{nan}
\define@key{fams}{kzs}{nan}
\define@key{fams}{zsu}{nan}
\define@key{fams}{syk}{nan}
\define@key{fams}{szn}{nan}
\define@key{fams}{srg}{nan}
\define@key{fams}{sqm}{nan}
\define@key{fams}{siv}{nan}
\define@key{fams}{six}{nan}
\define@key{fams}{suw}{nan}
\define@key{fams}{smw}{nan}
\define@key{fams}{sux}{nan}
\define@key{fams}{csv}{nan}
\define@key{fams}{ssk}{nan}
\define@key{fams}{suz}{nan}
\define@key{fams}{syo}{nan}
\define@key{fams}{sbj}{nan}
\define@key{fams}{sgd}{nan}
\define@key{fams}{sjp}{nan}
\define@key{fams}{tdl}{nan}
\define@key{fams}{sde}{nan}
\define@key{fams}{mdz}{nan}
\define@key{fams}{sru}{nan}
\define@key{fams}{swx}{nan}
\define@key{fams}{sqn}{nan}
\define@key{fams}{ssu}{nan}
\define@key{fams}{sdj}{nan}
\define@key{fams}{swu}{nan}
\define@key{fams}{suy}{nan}
\define@key{fams}{swg}{nan}
\define@key{fams}{slf}{nan}
\define@key{fams}{sgg}{nan}
\define@key{fams}{ssr}{nan}
\define@key{fams}{xdk}{nan}
\define@key{fams}{syl}{nan}
\define@key{fams}{zoq}{nan}
\define@key{fams}{nhc}{nan}
\define@key{fams}{zat}{nan}
\define@key{fams}{knv}{nan}
\define@key{fams}{tzx}{nan}
\define@key{fams}{xtt}{nan}
\define@key{fams}{lts}{nan}
\define@key{fams}{dsq}{nan}
\define@key{fams}{tdy}{nan}
\define@key{fams}{rob}{nan}
\define@key{fams}{tcd}{nan}
\define@key{fams}{klg}{nan}
\define@key{fams}{bgs}{nan}
\define@key{fams}{mvv}{nan}
\define@key{fams}{tgz}{nan}
\define@key{fams}{tbm}{nan}
\define@key{fams}{tda}{nan}
\define@key{fams}{tgx}{nan}
\define@key{fams}{tgj}{nan}
\define@key{fams}{tgw}{nan}
\define@key{fams}{tht}{nan}
\define@key{fams}{blt}{nan}
\define@key{fams}{tyj}{nan}
\define@key{fams}{tyr}{nan}
\define@key{fams}{twh}{nan}
\define@key{fams}{tiz}{nan}
\define@key{fams}{taw}{nan}
\define@key{fams}{aos}{nan}
\define@key{fams}{tlq}{nan}
\define@key{fams}{thi}{nan}
\define@key{fams}{tjl}{nan}
\define@key{fams}{tdd}{nan}
\define@key{fams}{ago}{nan}
\define@key{fams}{tnq}{nan}
\define@key{fams}{tpo}{nan}
\define@key{fams}{uar}{nan}
\define@key{fams}{tmm}{nan}
\define@key{fams}{cuu}{nan}
\define@key{fams}{acq}{nan}
\define@key{fams}{pee}{nan}
\define@key{fams}{tdj}{nan}
\define@key{fams}{abh}{nan}
\define@key{fams}{tja}{nan}
\define@key{fams}{tkz}{nan}
\define@key{fams}{nho}{nan}
\define@key{fams}{tke}{nan}
\define@key{fams}{tak}{nan}
\define@key{fams}{tdf}{nan}
\define@key{fams}{tlr}{nan}
\define@key{fams}{tlv}{nan}
\define@key{fams}{tal}{nan}
\define@key{fams}{tln}{nan}
\define@key{fams}{tlk}{nan}
\define@key{fams}{tzl}{nan}
\define@key{fams}{yta}{nan}
\define@key{fams}{tcl}{nan}
\define@key{fams}{tmn}{nan}
\define@key{fams}{tmz}{nan}
\define@key{fams}{vmx}{nan}
\define@key{fams}{ten}{nan}
\define@key{fams}{tls}{nan}
\define@key{fams}{xxt}{nan}
\define@key{fams}{tdk}{nan}
\define@key{fams}{tmy}{nan}
\define@key{fams}{tax}{nan}
\define@key{fams}{tml}{nan}
\define@key{fams}{tpu}{nan}
\define@key{fams}{low}{nan}
\define@key{fams}{tpv}{nan}
\define@key{fams}{tcm}{nan}
\define@key{fams}{tni}{nan}
\define@key{fams}{tdx}{nan}
\define@key{fams}{tgn}{nan}
\define@key{fams}{tnx}{nan}
\define@key{fams}{tnv}{nan}
\define@key{fams}{txg}{nan}
\define@key{fams}{tgp}{nan}
\define@key{fams}{tkx}{nan}
\define@key{fams}{tgu}{nan}
\define@key{fams}{tbs}{nan}
\define@key{fams}{ytl}{nan}
\define@key{fams}{tbe}{nan}
\define@key{fams}{uji}{nan}
\define@key{fams}{txy}{nan}
\define@key{fams}{xnj}{nan}
\define@key{fams}{qcs}{nan}
\define@key{fams}{afp}{nan}
\define@key{fams}{taf}{nan}
\define@key{fams}{txj}{nan}
\define@key{fams}{tpf}{nan}
\define@key{fams}{txr}{nan}
\define@key{fams}{tdm}{nan}
\define@key{fams}{twq}{nan}
\define@key{fams}{tmt}{nan}
\define@key{fams}{ttd}{nan}
\define@key{fams}{tco}{nan}
\define@key{fams}{tpa}{nan}
\define@key{fams}{tad}{nan}
\define@key{fams}{tvs}{nan}
\define@key{fams}{tvn}{nan}
\define@key{fams}{rmu}{nan}
\define@key{fams}{twl}{nan}
\define@key{fams}{xtw}{nan}
\define@key{fams}{ttq}{nan}
\define@key{fams}{twy}{nan}
\define@key{fams}{tbp}{nan}
\define@key{fams}{tcp}{nan}
\define@key{fams}{ayy}{nan}
\define@key{fams}{tas}{nan}
\define@key{fams}{tnu}{nan}
\define@key{fams}{tys}{nan}
\define@key{fams}{tyt}{nan}
\define@key{fams}{tyz}{nan}
\define@key{fams}{tck}{nan}
\define@key{fams}{bqa}{nan}
\define@key{fams}{dtu}{nan}
\define@key{fams}{tsy}{nan}
\define@key{fams}{tcw}{nan}
\define@key{fams}{tuq}{nan}
\define@key{fams}{tkq}{nan}
\define@key{fams}{lor}{nan}
\define@key{fams}{tfo}{nan}
\define@key{fams}{twe}{nan}
\define@key{fams}{ztt}{nan}
\define@key{fams}{teg}{nan}
\define@key{fams}{tyx}{nan}
\define@key{fams}{lli}{nan}
\define@key{fams}{ebo}{nan}
\define@key{fams}{tyi}{nan}
\define@key{fams}{tvm}{nan}
\define@key{fams}{tlt}{nan}
\define@key{fams}{nhv}{nan}
\define@key{fams}{tjo}{nan}
\define@key{fams}{tbt}{nan}
\define@key{fams}{tmv}{nan}
\define@key{fams}{tqb}{nan}
\define@key{fams}{tdo}{nan}
\define@key{fams}{soz}{nan}
\define@key{fams}{tmo}{nan}
\define@key{fams}{ott}{nan}
\define@key{fams}{tmw}{nan}
\define@key{fams}{quw}{nan}
\define@key{fams}{otn}{nan}
\define@key{fams}{dtk}{nan}
\define@key{fams}{tes}{nan}
\define@key{fams}{pah}{nan}
\define@key{fams}{tqn}{nan}
\define@key{fams}{tns}{nan}
\define@key{fams}{tct}{nan}
\define@key{fams}{tev}{nan}
\define@key{fams}{cux}{nan}
\define@key{fams}{cte}{nan}
\define@key{fams}{ted}{nan}
\define@key{fams}{tef}{nan}
\define@key{fams}{trb}{nan}
\define@key{fams}{twg}{nan}
\define@key{fams}{tec}{nan}
\define@key{fams}{tmg}{nan}
\define@key{fams}{sjt}{nan}
\define@key{fams}{tkg}{nan}
\define@key{fams}{keg}{nan}
\define@key{fams}{twc}{nan}
\define@key{fams}{tez}{nan}
\define@key{fams}{tdt}{nan}
\define@key{fams}{tve}{nan}
\define@key{fams}{cut}{nan}
\define@key{fams}{twx}{nan}
\define@key{fams}{otx}{nan}
\define@key{fams}{poq}{nan}
\define@key{fams}{mxb}{nan}
\define@key{fams}{thy}{nan}
\define@key{fams}{thn}{nan}
\define@key{fams}{soa}{nan}
\define@key{fams}{nki}{nan}
\define@key{fams}{thk}{nan}
\define@key{fams}{iin}{nan}
\define@key{fams}{tou}{nan}
\define@key{fams}{ytp}{nan}
\define@key{fams}{txh}{nan}
\define@key{fams}{thu}{nan}
\define@key{fams}{ahi}{nan}
\define@key{fams}{mnl}{nan}
\define@key{fams}{tbj}{nan}
\define@key{fams}{ngy}{nan}
\define@key{fams}{lsn}{nan}
\define@key{fams}{tcn}{nan}
\define@key{fams}{mtx}{nan}
\define@key{fams}{tia}{nan}
\define@key{fams}{tiq}{nan}
\define@key{fams}{boo}{nan}
\define@key{fams}{tii}{nan}
\define@key{fams}{nza}{nan}
\define@key{fams}{txq}{nan}
\define@key{fams}{xtl}{nan}
\define@key{fams}{tkp}{nan}
\define@key{fams}{otl}{nan}
\define@key{fams}{zts}{nan}
\define@key{fams}{tij}{nan}
\define@key{fams}{tim}{nan}
\define@key{fams}{tvy}{nan}
\define@key{fams}{xsb}{nan}
\define@key{fams}{tit}{nan}
\define@key{fams}{tpz}{nan}
\define@key{fams}{tpe}{nan}
\define@key{fams}{tra}{nan}
\define@key{fams}{tic}{nan}
\define@key{fams}{tde}{nan}
\define@key{fams}{tdq}{nan}
\define@key{fams}{ttv}{nan}
\define@key{fams}{lax}{nan}
\define@key{fams}{tju}{nan}
\define@key{fams}{tpl}{nan}
\define@key{fams}{ctl}{nan}
\define@key{fams}{zpk}{nan}
\define@key{fams}{nuz}{nan}
\define@key{fams}{mqh}{nan}
\define@key{fams}{tmf}{nan}
\define@key{fams}{tng}{nan}
\define@key{fams}{tgh}{nan}
\define@key{fams}{tox}{nan}
\define@key{fams}{tgb}{nan}
\define@key{fams}{taz}{nan}
\define@key{fams}{tdr}{nan}
\define@key{fams}{tlg}{nan}
\define@key{fams}{tfi}{nan}
\define@key{fams}{tor}{nan}
\define@key{fams}{tgy}{nan}
\define@key{fams}{zuh}{nan}
\define@key{fams}{xto}{nan}
\define@key{fams}{txb}{nan}
\define@key{fams}{tok}{nan}
\define@key{fams}{tkn}{nan}
\define@key{fams}{lbw}{nan}
\define@key{fams}{tlm}{nan}
\define@key{fams}{tol}{nan}
\define@key{fams}{tod}{nan}
\define@key{fams}{tdi}{nan}
\define@key{fams}{tom}{nan}
\define@key{fams}{txa}{nan}
\define@key{fams}{ttp}{nan}
\define@key{fams}{txm}{nan}
\define@key{fams}{dtm}{nan}
\define@key{fams}{tqp}{nan}
\define@key{fams}{tst}{nan}
\define@key{fams}{tnz}{nan}
\define@key{fams}{tny}{nan}
\define@key{fams}{tog}{nan}
\define@key{fams}{xgf}{nan}
\define@key{fams}{tjn}{nan}
\define@key{fams}{tnw}{nan}
\define@key{fams}{txs}{nan}
\define@key{fams}{toz}{nan}
\define@key{fams}{ttj}{nan}
\define@key{fams}{toq}{nan}
\define@key{fams}{toy}{nan}
\define@key{fams}{ttu}{nan}
\define@key{fams}{trz}{nan}
\define@key{fams}{trj}{nan}
\define@key{fams}{fit}{nan}
\define@key{fams}{tdv}{nan}
\define@key{fams}{tqr}{nan}
\define@key{fams}{dtt}{nan}
\define@key{fams}{tno}{nan}
\define@key{fams}{tei}{nan}
\define@key{fams}{als}{nan}
\define@key{fams}{ttl}{nan}
\define@key{fams}{txo}{nan}
\define@key{fams}{txe}{nan}
\define@key{fams}{ttk}{nan}
\define@key{fams}{zph}{nan}
\define@key{fams}{tqu}{nan}
\define@key{fams}{neb}{nan}
\define@key{fams}{don}{nan}
\define@key{fams}{ttn}{nan}
\define@key{fams}{xtg}{nan}
\define@key{fams}{trl}{nan}
\define@key{fams}{rmg}{nan}
\define@key{fams}{rmd}{nan}
\define@key{fams}{trm}{nan}
\define@key{fams}{tme}{nan}
\define@key{fams}{stg}{nan}
\define@key{fams}{tip}{nan}
\define@key{fams}{trx}{nan}
\define@key{fams}{tgq}{nan}
\define@key{fams}{trn}{nan}
\define@key{fams}{trf}{nan}
\define@key{fams}{lst}{nan}
\define@key{fams}{tka}{nan}
\define@key{fams}{tsa}{nan}
\define@key{fams}{tsd}{nan}
\define@key{fams}{kvz}{nan}
\define@key{fams}{tsb}{nan}
\define@key{fams}{tsk}{nan}
\define@key{fams}{txc}{nan}
\define@key{fams}{kdl}{nan}
\define@key{fams}{xmw}{nan}
\define@key{fams}{tsw}{nan}
\define@key{fams}{hio}{nan}
\define@key{fams}{ldp}{nan}
\define@key{fams}{lto}{nan}
\define@key{fams}{fly}{nan}
\define@key{fams}{ttz}{nan}
\define@key{fams}{tsl}{nan}
\define@key{fams}{tvd}{nan}
\define@key{fams}{tsh}{nan}
\define@key{fams}{two}{nan}
\define@key{fams}{tsc}{nan}
\define@key{fams}{nrt}{nan}
\define@key{fams}{tuy}{nan}
\define@key{fams}{tuj}{nan}
\define@key{fams}{khc}{nan}
\define@key{fams}{bhq}{nan}
\define@key{fams}{tkf}{nan}
\define@key{fams}{tkd}{nan}
\define@key{fams}{tul}{nan}
\define@key{fams}{tlu}{nan}
\define@key{fams}{tey}{nan}
\define@key{fams}{rak}{nan}
\define@key{fams}{krt}{nan}
\define@key{fams}{iou}{nan}
\define@key{fams}{tum}{nan}
\define@key{fams}{kku}{nan}
\define@key{fams}{xtq}{nan}
\define@key{fams}{tbr}{nan}
\define@key{fams}{enh}{nan}
\define@key{fams}{trt}{nan}
\define@key{fams}{tse}{nan}
\define@key{fams}{tug}{nan}
\define@key{fams}{tjg}{nan}
\define@key{fams}{tqq}{nan}
\define@key{fams}{dza}{nan}
\define@key{fams}{ttf}{nan}
\define@key{fams}{tpr}{nan}
\define@key{fams}{tpw}{nan}
\define@key{fams}{trh}{nan}
\define@key{fams}{trd}{nan}
\define@key{fams}{twt}{nan}
\define@key{fams}{tuz}{nan}
\define@key{fams}{tch}{nan}
\define@key{fams}{tru}{nan}
\define@key{fams}{try}{nan}
\define@key{fams}{tqm}{nan}
\define@key{fams}{ttg}{nan}
\define@key{fams}{tmi}{nan}
\define@key{fams}{mtu}{nan}
\define@key{fams}{tww}{nan}
\define@key{fams}{ifk}{nan}
\define@key{fams}{bov}{nan}
\define@key{fams}{tud}{nan}
\define@key{fams}{tux}{nan}
\define@key{fams}{xjb}{nan}
\define@key{fams}{twn}{nan}
\define@key{fams}{uam}{nan}
\define@key{fams}{ksj}{nan}
\define@key{fams}{byc}{nan}
\define@key{fams}{uba}{nan}
\define@key{fams}{ubi}{nan}
\define@key{fams}{ubr}{nan}
\define@key{fams}{cpb}{nan}
\define@key{fams}{uda}{nan}
\define@key{fams}{udu}{nan}
\define@key{fams}{ufi}{nan}
\define@key{fams}{uga}{nan}
\define@key{fams}{uge}{nan}
\define@key{fams}{ugo}{nan}
\define@key{fams}{uha}{nan}
\define@key{fams}{uis}{nan}
\define@key{fams}{udj}{nan}
\define@key{fams}{kcf}{nan}
\define@key{fams}{ukh}{nan}
\define@key{fams}{umi}{nan}
\define@key{fams}{ukp}{nan}
\define@key{fams}{akd}{nan}
\define@key{fams}{ukl}{nan}
\define@key{fams}{uku}{nan}
\define@key{fams}{ukg}{nan}
\define@key{fams}{ukq}{nan}
\define@key{fams}{ukw}{nan}
\define@key{fams}{svb}{nan}
\define@key{fams}{ull}{nan}
\define@key{fams}{ulb}{nan}
\define@key{fams}{ulm}{nan}
\define@key{fams}{ulw}{nan}
\define@key{fams}{ulu}{nan}
\define@key{fams}{xky}{nan}
\define@key{fams}{gdn}{nan}
\define@key{fams}{umd}{nan}
\define@key{fams}{xum}{nan}
\define@key{fams}{umr}{nan}
\define@key{fams}{umg}{nan}
\define@key{fams}{upi}{nan}
\define@key{fams}{sju}{nan}
\define@key{fams}{due}{nan}
\define@key{fams}{umm}{nan}
\define@key{fams}{umo}{nan}
\define@key{fams}{unz}{nan}
\define@key{fams}{bbn}{nan}
\define@key{fams}{une}{nan}
\define@key{fams}{xgu}{nan}
\define@key{fams}{uni}{nan}
\define@key{fams}{uln}{nan}
\define@key{fams}{onu}{nan}
\define@key{fams}{unu}{nan}
\define@key{fams}{tov}{nan}
\define@key{fams}{tku}{nan}
\define@key{fams}{sxu}{nan}
\define@key{fams}{tth}{nan}
\define@key{fams}{dmg}{nan}
\define@key{fams}{dna}{nan}
\define@key{fams}{xup}{nan}
\define@key{fams}{tau}{nan}
\define@key{fams}{url}{nan}
\define@key{fams}{urm}{nan}
\define@key{fams}{uro}{nan}
\define@key{fams}{xur}{nan}
\define@key{fams}{urg}{nan}
\define@key{fams}{uvh}{nan}
\define@key{fams}{urx}{nan}
\define@key{fams}{urc}{nan}
\define@key{fams}{urv}{nan}
\define@key{fams}{urn}{nan}
\define@key{fams}{urz}{nan}
\define@key{fams}{ugy}{nan}
\define@key{fams}{uru}{nan}
\define@key{fams}{urp}{nan}
\define@key{fams}{usk}{nan}
\define@key{fams}{ush}{nan}
\define@key{fams}{ulf}{nan}
\define@key{fams}{usp}{nan}
\define@key{fams}{usi}{nan}
\define@key{fams}{omo}{nan}
\define@key{fams}{wsg}{nan}
\define@key{fams}{utu}{nan}
\define@key{fams}{uuu}{nan}
\define@key{fams}{evh}{nan}
\define@key{fams}{usu}{nan}
\define@key{fams}{auz}{nan}
\define@key{fams}{eze}{nan}
\define@key{fams}{vaa}{nan}
\define@key{fams}{kqu}{nan}
\define@key{fams}{vgr}{nan}
\define@key{fams}{dkg}{nan}
\define@key{fams}{tva}{nan}
\define@key{fams}{vap}{nan}
\define@key{fams}{vae}{nan}
\define@key{fams}{vsv}{nan}
\define@key{fams}{vmv}{nan}
\define@key{fams}{cvn}{nan}
\define@key{fams}{vlp}{nan}
\define@key{fams}{mkt}{nan}
\define@key{fams}{mlr}{nan}
\define@key{fams}{mpr}{nan}
\define@key{fams}{vnk}{nan}
\define@key{fams}{vau}{nan}
\define@key{fams}{vao}{nan}
\define@key{fams}{vah}{nan}
\define@key{fams}{vrs}{nan}
\define@key{fams}{vav}{nan}
\define@key{fams}{vaj}{nan}
\define@key{fams}{val}{nan}
\define@key{fams}{vem}{nan}
\define@key{fams}{vsl}{nan}
\define@key{fams}{xve}{nan}
\define@key{fams}{vec}{nan}
\define@key{fams}{veo}{nan}
\define@key{fams}{vra}{nan}
\define@key{fams}{vid}{nan}
\define@key{fams}{vig}{nan}
\define@key{fams}{vil}{nan}
\define@key{fams}{dyg}{nan}
\define@key{fams}{svc}{nan}
\define@key{fams}{vin}{nan}
\define@key{fams}{vic}{nan}
\define@key{fams}{vis}{nan}
\define@key{fams}{vit}{nan}
\define@key{fams}{vto}{nan}
\define@key{fams}{vls}{nan}
\define@key{fams}{vol}{nan}
\define@key{fams}{kch}{nan}
\define@key{fams}{vor}{nan}
\define@key{fams}{vum}{nan}
\define@key{fams}{vnp}{nan}
\define@key{fams}{vun}{nan}
\define@key{fams}{msn}{nan}
\define@key{fams}{vut}{nan}
\define@key{fams}{wbi}{nan}
\define@key{fams}{wmn}{nan}
\define@key{fams}{wab}{nan}
\define@key{fams}{wbb}{nan}
\define@key{fams}{kmx}{nan}
\define@key{fams}{wci}{nan}
\define@key{fams}{wdg}{nan}
\define@key{fams}{wbq}{nan}
\define@key{fams}{kxp}{nan}
\define@key{fams}{wdu}{nan}
\define@key{fams}{wag}{nan}
\define@key{fams}{wrx}{nan}
\define@key{fams}{waj}{nan}
\define@key{fams}{wga}{nan}
\define@key{fams}{wgb}{nan}
\define@key{fams}{wbr}{nan}
\define@key{fams}{fad}{nan}
\define@key{fams}{whk}{nan}
\define@key{fams}{wgo}{nan}
\define@key{fams}{wlr}{nan}
\define@key{fams}{wlk}{nan}
\define@key{fams}{wmh}{nan}
\define@key{fams}{atr}{nan}
\define@key{fams}{wli}{nan}
\define@key{fams}{wja}{nan}
\define@key{fams}{wav}{nan}
\define@key{fams}{wwb}{nan}
\define@key{fams}{wkd}{nan}
\define@key{fams}{waf}{nan}
\define@key{fams}{lgl}{nan}
\define@key{fams}{wlw}{nan}
\define@key{fams}{wly}{nan}
\define@key{fams}{wll}{nan}
\define@key{fams}{wlx}{nan}
\define@key{fams}{waa}{nan}
\define@key{fams}{wln}{nan}
\define@key{fams}{wae}{nan}
\define@key{fams}{ola}{nan}
\define@key{fams}{wmc}{nan}
\define@key{fams}{wmi}{nan}
\define@key{fams}{lbq}{nan}
\define@key{fams}{waz}{nan}
\define@key{fams}{qyp}{nan}
\define@key{fams}{wnp}{nan}
\define@key{fams}{wnb}{nan}
\define@key{fams}{nnp}{nan}
\define@key{fams}{wbh}{nan}
\define@key{fams}{wdd}{nan}
\define@key{fams}{wad}{nan}
\define@key{fams}{mfi}{nan}
\define@key{fams}{wne}{nan}
\define@key{fams}{hwa}{nan}
\define@key{fams}{wnm}{nan}
\define@key{fams}{lwg}{nan}
\define@key{fams}{wng}{nan}
\define@key{fams}{jub}{nan}
\define@key{fams}{wno}{nan}
\define@key{fams}{wnk}{nan}
\define@key{fams}{wny}{nan}
\define@key{fams}{juk}{nan}
\define@key{fams}{juw}{nan}
\define@key{fams}{wbf}{nan}
\define@key{fams}{tci}{nan}
\define@key{fams}{srv}{nan}
\define@key{fams}{bpe}{nan}
\define@key{fams}{wre}{nan}
\define@key{fams}{wai}{nan}
\define@key{fams}{wri}{nan}
\define@key{fams}{wbe}{nan}
\define@key{fams}{aml}{nan}
\define@key{fams}{wji}{nan}
\define@key{fams}{bgv}{nan}
\define@key{fams}{wrl}{nan}
\define@key{fams}{wrn}{nan}
\define@key{fams}{wru}{nan}
\define@key{fams}{wrv}{nan}
\define@key{fams}{wss}{nan}
\define@key{fams}{gsp}{nan}
\define@key{fams}{wsu}{nan}
\define@key{fams}{wtk}{nan}
\define@key{fams}{wah}{nan}
\define@key{fams}{wuy}{nan}
\define@key{fams}{www}{nan}
\define@key{fams}{wow}{nan}
\define@key{fams}{wxa}{nan}
\define@key{fams}{ctt}{nan}
\define@key{fams}{wyr}{nan}
\define@key{fams}{weh}{nan}
\define@key{fams}{wew}{nan}
\define@key{fams}{wlh}{nan}
\define@key{fams}{klh}{nan}
\define@key{fams}{wei}{nan}
\define@key{fams}{gxx}{nan}
\define@key{fams}{ywl}{nan}
\define@key{fams}{hmw}{nan}
\define@key{fams}{ojw}{nan}
\define@key{fams}{tqt}{nan}
\define@key{fams}{yih}{nan}
\define@key{fams}{pnb}{nan}
\define@key{fams}{lcp}{nan}
\define@key{fams}{kuf}{nan}
\define@key{fams}{mut}{nan}
\define@key{fams}{kyu}{nan}
\define@key{fams}{tdg}{nan}
\define@key{fams}{wmg}{nan}
\define@key{fams}{raf}{nan}
\define@key{fams}{mmr}{nan}
\define@key{fams}{lia}{nan}
\define@key{fams}{xwl}{nan}
\define@key{fams}{bbp}{nan}
\define@key{fams}{ssl}{nan}
\define@key{fams}{krw}{nan}
\define@key{fams}{nnd}{nan}
\define@key{fams}{uve}{nan}
\define@key{fams}{mss}{nan}
\define@key{fams}{lmj}{nan}
\define@key{fams}{drn}{nan}
\define@key{fams}{suc}{nan}
\define@key{fams}{twb}{nan}
\define@key{fams}{pne}{nan}
\define@key{fams}{zbw}{nan}
\define@key{fams}{dnw}{nan}
\define@key{fams}{nhw}{nan}
\define@key{fams}{pua}{nan}
\define@key{fams}{gnw}{nan}
\define@key{fams}{jmx}{nan}
\define@key{fams}{tnb}{nan}
\define@key{fams}{amw}{nan}
\define@key{fams}{azn}{nan}
\define@key{fams}{wwo}{nan}
\define@key{fams}{wea}{nan}
\define@key{fams}{wec}{nan}
\define@key{fams}{woy}{nan}
\define@key{fams}{lwh}{nan}
\define@key{fams}{giw}{nan}
\define@key{fams}{tnp}{nan}
\define@key{fams}{tua}{nan}
\define@key{fams}{mtp}{nan}
\define@key{fams}{wlv}{nan}
\define@key{fams}{wik}{nan}
\define@key{fams}{wie}{nan}
\define@key{fams}{wij}{nan}
\define@key{fams}{wif}{nan}
\define@key{fams}{wih}{nan}
\define@key{fams}{wua}{nan}
\define@key{fams}{wil}{nan}
\define@key{fams}{wit}{nan}
\define@key{fams}{gdr}{nan}
\define@key{fams}{wrh}{nan}
\define@key{fams}{wir}{nan}
\define@key{fams}{wiu}{nan}
\define@key{fams}{xwc}{nan}
\define@key{fams}{woc}{nan}
\define@key{fams}{wbw}{nan}
\define@key{fams}{wyi}{nan}
\define@key{fams}{jod}{nan}
\define@key{fams}{wod}{nan}
\define@key{fams}{wle}{nan}
\define@key{fams}{wom}{nan}
\define@key{fams}{wmo}{nan}
\define@key{fams}{won}{nan}
\define@key{fams}{cwd}{nan}
\define@key{fams}{kda}{nan}
\define@key{fams}{wor}{nan}
\define@key{fams}{jud}{nan}
\define@key{fams}{wsv}{nan}
\define@key{fams}{wtw}{nan}
\define@key{fams}{wud}{nan}
\define@key{fams}{qgu}{nan}
\define@key{fams}{wlu}{nan}
\define@key{fams}{wux}{nan}
\define@key{fams}{bqm}{nan}
\define@key{fams}{wum}{nan}
\define@key{fams}{ywu}{nan}
\define@key{fams}{bwn}{nan}
\define@key{fams}{wub}{nan}
\define@key{fams}{wur}{nan}
\define@key{fams}{yig}{nan}
\define@key{fams}{bse}{nan}
\define@key{fams}{wsi}{nan}
\define@key{fams}{wuh}{nan}
\define@key{fams}{wut}{nan}
\define@key{fams}{wuv}{nan}
\define@key{fams}{wym}{nan}
\define@key{fams}{zax}{nan}
\define@key{fams}{xkr}{nan}
\define@key{fams}{xan}{nan}
\define@key{fams}{ztg}{nan}
\define@key{fams}{axx}{nan}
\define@key{fams}{xeg}{nan}
\define@key{fams}{xet}{nan}
\define@key{fams}{hsn}{nan}
\define@key{fams}{sjo}{nan}
\define@key{fams}{asn}{nan}
\define@key{fams}{xiy}{nan}
\define@key{fams}{xip}{nan}
\define@key{fams}{xii}{nan}
\define@key{fams}{xoo}{nan}
\define@key{fams}{xwe}{nan}
\define@key{fams}{tyy}{nan}
\define@key{fams}{muu}{nan}
\define@key{fams}{yar}{nan}
\define@key{fams}{ybn}{nan}
\define@key{fams}{ybm}{nan}
\define@key{fams}{ybo}{nan}
\define@key{fams}{ekr}{nan}
\define@key{fams}{rys}{nan}
\define@key{fams}{wfg}{nan}
\define@key{fams}{ygm}{nan}
\define@key{fams}{ygw}{nan}
\define@key{fams}{rhp}{nan}
\define@key{fams}{ner}{nan}
\define@key{fams}{ynu}{nan}
\define@key{fams}{iyx}{nan}
\define@key{fams}{ykk}{nan}
\define@key{fams}{ybh}{nan}
\define@key{fams}{xyl}{nan}
\define@key{fams}{yba}{nan}
\define@key{fams}{jal}{nan}
\define@key{fams}{zpu}{nan}
\define@key{fams}{yal}{nan}
\define@key{fams}{ymp}{nan}
\define@key{fams}{yat}{nan}
\define@key{fams}{ymb}{nan}
\define@key{fams}{yme}{nan}
\define@key{fams}{ymn}{nan}
\define@key{fams}{qur}{nan}
\define@key{fams}{yda}{nan}
\define@key{fams}{dym}{nan}
\define@key{fams}{xyb}{nan}
\define@key{fams}{zyg}{nan}
\define@key{fams}{jng}{nan}
\define@key{fams}{yng}{nan}
\define@key{fams}{bsx}{nan}
\define@key{fams}{yav}{nan}
\define@key{fams}{ygl}{nan}
\define@key{fams}{ymo}{nan}
\define@key{fams}{yde}{nan}
\define@key{fams}{ynl}{nan}
\define@key{fams}{tjj}{nan}
\define@key{fams}{ysm}{nan}
\define@key{fams}{jay}{nan}
\define@key{fams}{guu}{nan}
\define@key{fams}{asy}{nan}
\define@key{fams}{yre}{nan}
\define@key{fams}{yev}{nan}
\define@key{fams}{yrw}{nan}
\define@key{fams}{zae}{nan}
\define@key{fams}{yro}{nan}
\define@key{fams}{yko}{nan}
\define@key{fams}{zty}{nan}
\define@key{fams}{yla}{nan}
\define@key{fams}{yuw}{nan}
\define@key{fams}{jau}{nan}
\define@key{fams}{yyu}{nan}
\define@key{fams}{zpb}{nan}
\define@key{fams}{qux}{nan}
\define@key{fams}{yvt}{nan}
\define@key{fams}{yww}{nan}
\define@key{fams}{ywn}{nan}
\define@key{fams}{yaw}{nan}
\define@key{fams}{yby}{nan}
\define@key{fams}{ybx}{nan}
\define@key{fams}{ykr}{nan}
\define@key{fams}{yel}{nan}
\define@key{fams}{ylg}{nan}
\define@key{fams}{ynq}{nan}
\define@key{fams}{yec}{nan}
\define@key{fams}{yei}{nan}
\define@key{fams}{yra}{nan}
\define@key{fams}{gop}{nan}
\define@key{fams}{yrn}{nan}
\define@key{fams}{yeu}{nan}
\define@key{fams}{yes}{nan}
\define@key{fams}{yet}{nan}
\define@key{fams}{yej}{nan}
\define@key{fams}{ydg}{nan}
\define@key{fams}{yim}{nan}
\define@key{fams}{kvu}{nan}
\define@key{fams}{yin}{nan}
\define@key{fams}{yil}{nan}
\define@key{fams}{ywg}{nan}
\define@key{fams}{kvy}{nan}
\define@key{fams}{yxm}{nan}
\define@key{fams}{ljw}{nan}
\define@key{fams}{yiy}{nan}
\define@key{fams}{yis}{nan}
\define@key{fams}{gek}{nan}
\define@key{fams}{yob}{nan}
\define@key{fams}{gud}{nan}
\define@key{fams}{yog}{nan}
\define@key{fams}{ydk}{nan}
\define@key{fams}{yki}{nan}
\define@key{fams}{ygs}{nan}
\define@key{fams}{xty}{nan}
\define@key{fams}{pil}{nan}
\define@key{fams}{yoi}{nan}
\define@key{fams}{sxk}{nan}
\define@key{fams}{nru}{nan}
\define@key{fams}{zyn}{nan}
\define@key{fams}{zyb}{nan}
\define@key{fams}{yno}{nan}
\define@key{fams}{yon}{nan}
\define@key{fams}{yut}{nan}
\define@key{fams}{mts}{nan}
\define@key{fams}{yox}{nan}
\define@key{fams}{yot}{nan}
\define@key{fams}{zyj}{nan}
\define@key{fams}{ytw}{nan}
\define@key{fams}{yoy}{nan}
\define@key{fams}{nua}{nan}
\define@key{fams}{msd}{nan}
\define@key{fams}{mvg}{nan}
\define@key{fams}{yub}{nan}
\define@key{fams}{ysl}{nan}
\define@key{fams}{ygu}{nan}
\define@key{fams}{yab}{nan}
\define@key{fams}{omk}{nan}
\define@key{fams}{ybl}{nan}
\define@key{fams}{yuq}{nan}
\define@key{fams}{ljx}{nan}
\define@key{fams}{mab}{nan}
\define@key{fams}{yau}{nan}
\define@key{fams}{ztx}{nan}
\define@key{fams}{kji}{nan}
\define@key{fams}{nhi}{nan}
\define@key{fams}{ctz}{nan}
\define@key{fams}{atb}{nan}
\define@key{fams}{zkr}{nan}
\define@key{fams}{zsl}{nan}
\define@key{fams}{zak}{nan}
\define@key{fams}{zau}{nan}
\define@key{fams}{zna}{nan}
\define@key{fams}{zah}{nan}
\define@key{fams}{zpw}{nan}
\define@key{fams}{zaj}{nan}
\define@key{fams}{zbu}{nan}
\define@key{fams}{zaz}{nan}
\define@key{fams}{zal}{nan}
\define@key{fams}{kxk}{nan}
\define@key{fams}{zwa}{nan}
\define@key{fams}{jaj}{nan}
\define@key{fams}{zua}{nan}
\define@key{fams}{dhm}{nan}
\define@key{fams}{zeg}{nan}
\define@key{fams}{czn}{nan}
\define@key{fams}{zhb}{nan}
\define@key{fams}{xzh}{nan}
\define@key{fams}{zhi}{nan}
\define@key{fams}{zhw}{nan}
\define@key{fams}{zia}{nan}
\define@key{fams}{zil}{nan}
\define@key{fams}{ziw}{nan}
\define@key{fams}{zib}{nan}
\define@key{fams}{zmb}{nan}
\define@key{fams}{zin}{nan}
\define@key{fams}{sih}{nan}
\define@key{fams}{zrn}{nan}
\define@key{fams}{ziz}{nan}
\define@key{fams}{pto}{nan}
\define@key{fams}{yzk}{nan}
\define@key{fams}{gbz}{nan}
\define@key{fams}{czt}{nan}
\define@key{fams}{zom}{nan}
\define@key{fams}{zla}{nan}
\define@key{fams}{gnd}{nan}
\define@key{fams}{zuy}{nan}
\define@key{fams}{jmb}{nan}
\define@key{fams}{zzj}{nan}
\define@key{fams}{zyp}{nan}
}
\DeclareOption{wals}{\input{langs_wals.tex}
                     \input{fams_wals.tex}}
\DeclareOption{none}{}
%    \end{macrocode}
% Language codes, names and families are set with the |\define@key|\marg{family} \marg{key} \marg{value} macro from the \textsf{xkeyval} package(see the \textsf{langs\_.tex} and \textsf{fams\_.tex} files in the package folder). \marg{family} is either \textsf{names} or \textsf{fams}. Thus, each language has two key-value pairs that refer to it, one defining its name and the other one defining its family, both using the ISO 639-3 code as their key.
% \end{macro}
% \subsection{Macro definitions}
% \begin{macro}{lname}
% The |\lname| macro takes the key specified in its mandatory argument to call its corresponding key value pair from the \textsf{names} family, and prints it. This is achieved through the use of the |\csname| and |\endcsname| macros. The |\unskip| macro is used in all the macro definitions to avoid the adding of an extra space after the macro has been printed.
%    \begin{macrocode}
\newcommand{\lname}[1]{%
  \csname KV@names@#1\endcsname\unskip
}
%    \end{macrocode}
% \end{macro}
% \begin{macro}{liso}
% The |\liso| macro takes, like |\lname|, the value from the \textsf{names} family from the argument input, and prints the name as well as the ISO 639-3 code (which is the argument verbatim) between parenthesis.
%    \begin{macrocode}
\newcommand{\liso}[1]{%
  \csname KV@names@#1\endcsname{} (ISO 639-3: #1)\unskip
}
%    \end{macrocode}
% \end{macro}
% \begin{macro}{lfam}
% The |\lfam| macro, like |\lname| and |\liso|, calls the key-value pair from the \textsf{names} family corresponding to the input of the mandatory argument, plus the key-value pair from the \textsf{fams} family which gives it the genetic affiliation, which is printed between parenthesis.
%    \begin{macrocode}
\newcommand{\lfam}[1]{%
  \csname KV@names@#1\endcsname{} (\csname KV@fams@#1\endcsname{})\unskip
}
%    \end{macrocode}
% \end{macro}
% \begin{macro}{newlang}
% The macro |\newlang| defines new keys for a language from the three mandatory arguments with the |\define@key| \marg{family} \marg{key} \marg{value} macro from \textsf{xkeyval}. The first argument of |\newlang| \marg{code} defines the code which serves as identifier (the ISO code in the case of pre-defined key-value pairs). The second argument \marg{name} defines the printed name of the language. The third argument \marg{family} defines the family to which the language belongs.
%    \begin{macrocode}
\newcommand{\newlang}[3]{%
  \define@key{names}{#1}{#2}
  \define@key{fams}{#1}{#3}
}
%    \end{macrocode}
% \end{macro}
% \begin{macro}{ProcessOptions}
% This line of code simply tells the package to set the options specified above.
%    \begin{macrocode}
\ProcessOptions \relax
%    \end{macrocode}
% \end{macro}
% \Finale
\endinput
