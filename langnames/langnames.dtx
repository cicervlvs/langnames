% \iffalse meta-comment
%
% Copyright (C) 2022, 2023 Alejandro García Matarredona, निरंजन
%
% This file may be distributed and/or modified under the
% conditions of the LaTeX Project Public License, either
% version 1.3 of this license or (at your option) any later
% version. The latest version of this license is in:
%
% http://www.latex-project.org/lppl.txt
%
% and version 1.3 or later is part of all distributions of
% LaTeX version 2005/12/01 or later.
%
% \fi
% \iffalse
%<package>\ProvidesPackage{langnames}
%<package> [2023/12/03 v3.0.0 langnames package for naming and classification of languages]
%
%<*driver>
\documentclass{l3doc}
\usepackage[glottolog]{langnames}
\usepackage{hologo}
\DeclareRobustCommand\XeLaTeX{\hologo{XeLaTeX}}
\usepackage{fontspec}
\newfontfamily\mrtxt[%
  Scale={0.8},%
  Script={Devanagari},%
  Renderer={Harfbuzz}%
]{Shobhika}
\usepackage{expex}
\gathertags
\usepackage{hyperref,hyperxmp}
\hypersetup{%
  unicode,%
  colorlinks=true,%
  urlcolor=blue,%
  pdftitle={The langnames package},%
  pdfauthor={Alejandro García Matarredona, निरंजन},%
  pdfsubject={The documentation of the LaTeX package `langnames'},%
  pdfcreator={निरंजन},%
  pdfkeywords={Linguistics, typology, languages, LaTeX},%
  pdfcopyright={%
    The LaTeX package langnames\textLF
    This file may be distributed and/or modified under the
    conditions of the LaTeX Project Public License, either
    version 1.3 of this license or (at your option) any later
    version.%
  },%
  pdflicenseurl={http://www.latex-project.org/lppl.txt}%
}
\EnableCrossrefs
\CodelineIndex
\RecordChanges
\begin{document}
\DocInput{langnames.dtx}
\end{document}
%</driver>
% \fi
% \CheckSum{103}
% \changes{v1.0}{2022/08/25}{Initial version}
% \changes{v2.0}{2022/09/05}{Optimizing and debugging the code}
% \changes{v2.1}{2023/01/04}{Removing all the dependencies}
% \changes{v3.0.0}{2023/12/03}{Updated the database}
% \GetFileInfo{langnames.sty}
% \DoNotIndex{\AddToHook,\def,\ekvdefinekeys,\ekvoProcessLocalOptions,\else,\expandafter,\fi,\ifdefined,\csname,\newcommand,\input,\usepackage,\PackageError,\space,\ssp,\MessageBreak,\langnames@cs@prefix,\endcsname}
% \title{The \textsf{langnames} package\thanks{This document
% corresponds to \textsf{langnames}~\fileversion,
% dated~\filedate.}}
% \author{^^A
%   Alejandro García Matarredona\\^^A
%   \texttt{alejandrogarciaag41@gmail.com}^^A
%   \and
%   {\mrtxt निरंजन}\\^^A
%   \texttt{hi.niranjan@pm.me}^^A
% }
% \maketitle
% \begin{abstract}
% \noindent The \textsf{langnames} package provides a set of macros for formatting names of languages, as well as their identification (in the form of ISO 639-3 codes) and their classification (in the form of its top-level family). The datasets from \href{https://wals.info}{WALS} and \href{https://glottolog.org}{Glottolog} are included in the package. The package also allows users to rename and add new languages.
% \end{abstract}
% \tableofcontents
% \section{Introduction}
% The typing out of language names in academic papers, especially those in language typology or related fields where many names have to be typed many times, is often inconvenient and inconsistent. This package attempts to be a small help to writers, especially of large projects or of collaborative ones, to have a slightly easier time with names of languages. It does so by defining three main commands: |\lname|, |\liso|, and |\lfam|, which respectively print out the name, name and ISO 639-3 code, and name and family of the specified language. While the package comes with about 7500 pre-defined languages, with code, name, and family, the user may also define new ones through the |\newlang| and |\renewlang| commands. The basic use of all five of these commands is explained below.
% \section{Usage}
% \subsection{Installation}
% Download the package from wherever it was found to a place where \LaTeX\ may see it, typically in \texttt{\$TEXMFHOME/tex/latex}.
% \subsection{Package options}
% When calling |\usepackage{langnames}|, the user must specify one of three options: |glottolog|, |wals|, or |none|.
% The first option, \textsf{glottolog}, selects the naming conventions from the \href{https://glottolog.org}{Glottolog} database. The second option, \textsf{wals}, predictably selects the naming conventions of the \href{https://wals.info}{WALS} database. The names and the genetic classification differ in some languages, so the user may choose what convention to follow.
% During the preparation of the dataset, there were instances of languages which appeared in WALS but not in Glottolog, and vice-versa. In such cases, the missing information was added from the other database. For more details on how I built the dataset, one may consult the Python script made for it in the \href{https://github.com/cicervlvs/langnames}{Github repository}.
% The third option, \textsf{none}, tells the package not to load either of the datasets, and instead start off from an empty canvas. If one specifies this option, one will have to fill in the details of each language with the macro |\newlang| (see explanation in Section 2.3 below).
% \subsection{Macros}\label{sec:mac}
% When referring to a language, the author may use one of three macros to print out different information about it. Languages are identified by their ISO 639-3 code.
% \begin{function}{\lname}
%   \begin{syntax}
%     \cs{lname}\marg{ISO code}
%   \end{syntax}
%   The simplest macro is |\lname|, which prints out the name of the specified language according to the code provided. The basic syntax is thus|\newlang| \marg{ISO code}. This can be seen in example (\nextx).
%   
%   \ex
%   |My native language is \lname{cat}.|\par\noindent
%   My native language is \lname{cat}.
%   \xe
% \end{function}
% \begin{function}{\liso}
%   \begin{syntax}
%     \cs{liso}\marg{ISO code}
%   \end{syntax}
%   The |\liso| macro prints out both the name and the ISO 639-3 code of the language specified in the argument (\cs{liso}\marg{ISO code}). Example (\nextx) shows its behavior.
%   \ex
%   |I have recently taken up \liso{brg}.|\par\noindent
%   I have recently taken up \liso{brg}.
%   \xe
% \end{function}
% \begin{function}{\lfam}
%   \begin{syntax}
%     \cs{lfam}\marg{ISO code}
%   \end{syntax}
%   The |\lfam| command prints the name of the language and its family in parenthesis. Once again, the language is identified by its ISO 639-3 code. Example (\nextx) shows how it works.
%   \ex
%   |The tone system of \lfam{pkt} is fascinating.|\par\noindent
%   The tone system of \lfam{pkt} is fascinating.
%   \xe
% \end{function}
% \begin{function}{\langnative}
%   \begin{syntax}
%     \cs{langnative}\marg{code}
%   \end{syntax}
%   This macro may be used to print the native name of the language in its own script. This might require the user to use, for example, Lua\LaTeX{} or \XeLaTeX. Currently, the package provides the native names of a few languages with both their |glottolog| and |wals| code. The concerned files are named |ln_lang_glot_native.tex| and |ln_lang_wals_native.tex| respectively. Unless a user defines their own languages with \cs{newlangnative}, this functionality is only available with the package option |native|. As package authors we are unable to provide the native versions of all names in the dataset; if you cannot see your language in the output, we are open to \href{https://github.com/cicervlvs/langnames/pulls}{pull requests} or \href{https://github.com/cicervlvs/langnames/issues}{issues}.
% \begin{verbatim}
% \documentclass{article}
% \usepackage[glottolog,native]{langnames}
% \usepackage{fontspec}
% \newfontfamily\martxt[%
%   Script=Devanagari,%
%   Scale=0.9,%
%   Renderer=Harfbuzz% Only required with LuaLaTeX
% ]{Shobhika}
% % ctan.org/pkg/shobhika
% \newfontfamily\jpntxt[Scale=0.9]{HaranoAjiMincho-Regular.otf}
% % ctan.org/pkg/haranoaji
% 
% \begin{document}
% The endonym of \lname{mar} is {\martxt\langnative{mar}}.\par
% The endonym of \lname{jpn} is {\jpntxt\langnative{jpn}}.\par
% The endonym of \lname{deu} is {\jpntxt\langnative{deu}}.
% \end{document}
% \end{verbatim}
% \end{function}
% \begin{function}{\newlang}
%   \begin{syntax}
%     \cs{newlang}\marg{pseudo code}\marg{name}\marg{family}
%   \end{syntax}
%   Users may add their own languages via the use of the |\newlang| and |\renewlang| commands. |\newlang| takes three arguments as shown above. Example (\nextx) shows its usage.
% 
% \ex\par\noindent
%
%   \vspace*{-0.8\baselineskip}
%   \begin{minipage}[h]{0.8\linewidth}
% \begin{verbatim}
% \newlang{boo}{Ameli}{Amelian}
% \begin{document}
% My new made up language is \lname{boo}.\par\noindent
% My new made up language is \liso{boo}.\par\noindent
% My new made up language is \lfam{boo}.
% \end{document}
% \end{verbatim}
%   \end{minipage}
%   \medskip
%   
%   \changetonone
%   \newlang{boo}{Ameli}{Amelian}
%   My new made up language is \lname{boo}.\par\noindent
%   My new made up language is \liso{boo}.\par\noindent
%   My new made up language is \lfam{boo}.
%   \xe
%   Note that adding new macros with |\newlang| will not overwrite any from the two other datasets as all three of them have different prefixes. Therefore, if the package was loaded with any option but |none|, none of the \cs{newlang}s defined will work, because `\newlang`s are meant to be used with one's own dataset.
% \end{function}
% \begin{function}{\renewlang}
%   \begin{syntax}
%     \cs{renewlang}\marg{dataset}\marg{code/pseudo code}\marg{name}\marg{family}
%   \end{syntax}
%   Unlike \cs{newlang}, this command actually renews a definition from the specified \meta{dataset}. This macro therefore has one more argument than |\newlang|, the first argument \meta{dataset}.
% \end{function}
% \begin{function}{\newlangnative}
%   \begin{syntax}
%     \cs{langnative}\marg{dataset}\marg{code}\marg{name}
%   \end{syntax}
% This macro is the |native| counterpart to the |\newlang| macro, adding a new language to the |native| dataset. Here the user may define new languages to print in their endonymous form.
% \end{function}
% \subsubsection{Local changes}
% The package offers another set of macros to change the language dataset locally, for example if the |wals| option was passed while loading the package and in one section one needs to use the language name from the |glottolog| set, the following commands may be used. These commands do not take any arguments.
% \begin{function}{\changetoglottolog,\changetowals,\changetonone}
%   As the names suggest, these will change your dataset for the current local group, i.e., the running environment, or the current pair of |{|, |}|, |\begingroup|, |\endgroup| pair, or |\bgroup|, |\egroup| pair.
% \end{function}
% \subsection{A miscellaneous example}
% \emph{The following code}:
% \begin{verbatim}
% \documentclass{article}
% \usepackage[glottolog]{langnames}
% 
% \begin{document}
% \noindent
% My language is \lname{cat}.\par
% {%
%   \changetonone
%   \newlang{cat}{Meow}{Meowian}%
%   \noindent
%   My language is \lname{cat}.\par
% }\noindent
% My language is \lname{cat}.\par
% \renewlang{glottolog}{cat}{Meow}{Meowian}\noindent
% My language is \lname{cat}.
% \end{document}
% \end{verbatim}
% \smallskip
% 
% \noindent
% \emph{Produces}:\par\noindent
% My language is \lname{cat}.\par
% {%
%   \changetonone
%   \newlang{cat}{Meow}{Meowian}%
%   \noindent
%   My language is \lname{cat}.\par
% }\noindent
% My language is \lname{cat}.\par
% \renewlang{glottolog}{cat}{Meow}{Meowian}\noindent
% My language is \lname{cat}.
% 
% \StopEventually{\PrintIndex\PrintChanges}
% \section{Implementation}
% Language codes, names and families are set with simple |\newcommand|s. These commands have a four part structure as follows:
% \begin{description}
% \item[Internalization:] As these commands are for internal use, they should be inaccessible to the users and hence the |@| symbol is used in the command name. The command name starts with the package name, in order to ensure safety. Thus, the first part of macros look like so: |\langnames@|
% \item[Name or family:] The package presents two different sets of names, namely language names and language family names. Internally they are named |name| and |fams| respectively. Thus, combining the first part with this one one gets |\langnames@langs| or |\langname@fams|.
% \item[Prefix:] As there exist three options, three prefixes need to be defined in order to allow for conditional selection. For this there exist three prefixes: |none|, |wals| and |glottolog|.
% \end{description}
% Thus, each language has two internal macros, one defining its name and the other one defining its family, both using the ISO 639-3 code as their key. For more information see |ln_langs_*| and |ln_fams_*| files in the package folder.
% \subsection{Option setting}
% Options are set for what dataset to use. |glottolog| uses Glottolog data; |wals| uses WALS data; |none| selects neither datasets and all languages are defined by the user. See |langnames/langnames.py| in the \href{https://github.com/cicervlvs/langnames}{Github repository} to see how I gathered and handled the data.
%    \begin{macrocode}
\DeclareOption{glottolog}{%
  \def\langnames@cs@prefix{glottolog}%
  \def\langnames@langs@glot@knw{!Xun (Ekoka)}
\def\langnames@langs@glot@nmn{!Xóõ}
\def\langnames@langs@glot@alu{'Are'are}
\def\langnames@langs@glot@hnh{//Ani}
\def\langnames@langs@glot@xam{/Xam}
\def\langnames@langs@glot@huc{=|Hoan}
\def\langnames@langs@glot@apq{A-Pucikwar}
\def\langnames@langs@glot@aiw{Aari}
\def\langnames@langs@glot@aau{Abau}
\def\langnames@langs@glot@abq{Abaza}
\def\langnames@langs@glot@abe{Abenaki (Western)}
\def\langnames@langs@glot@abi{Abidji}
\def\langnames@langs@glot@axb{Abipón}
\def\langnames@langs@glot@abk{Abkhaz}
\def\langnames@langs@glot@abz{Abui}
\def\langnames@langs@glot@kgr{Abun}
\def\langnames@langs@glot@ace{Acehnese}
\def\langnames@langs@glot@aca{Achagua}
\def\langnames@langs@glot@acn{Achang}
\def\langnames@langs@glot@ach{Acholi}
\def\langnames@langs@glot@acu{Achuar}
\def\langnames@langs@glot@acv{Achumawi}
\def\langnames@langs@glot@guq{Aché}
\def\langnames@langs@glot@acr{Achí}
\def\langnames@langs@glot@kjq{Acoma}
\def\langnames@langs@glot@ads{Adamorobe Sign Language}
\def\langnames@langs@glot@adn{Adang}
\def\langnames@langs@glot@adj{Adioukrou}
\def\langnames@langs@glot@ady{Adyghe (Abzakh)}
\def\langnames@langs@glot@adt{Adynyamathanha}
\def\langnames@langs@glot@adz{Adzera}
\def\langnames@langs@glot@awi{Aekyom}
\def\langnames@langs@glot@afr{Afrikaans}
\def\langnames@langs@glot@agd{Agarabi}
\def\langnames@langs@glot@agq{Aghem}
\def\langnames@langs@glot@ahh{Aghu}
\def\langnames@langs@glot@agx{Aghul}
\def\langnames@langs@glot@agt{Agta (Central)}
\def\langnames@langs@glot@duo{Agta (Dupaningan)}
\def\langnames@langs@glot@agu{Aguacatec}
\def\langnames@langs@glot@agr{Aguaruna}
\def\langnames@langs@glot@aht{Ahtna}
\def\langnames@langs@glot@tba{Aikaná}
\def\langnames@langs@glot@ain{Ainu}
\def\langnames@langs@glot@ahp{Aizi}
\def\langnames@langs@glot@aja{Aja}
\def\langnames@langs@glot@ajg{Ajagbe}
\def\langnames@langs@glot@aji{Ajië}
\def\langnames@langs@glot@axk{Aka}
\def\langnames@langs@glot@abj{Aka-Biada}
\def\langnames@langs@glot@aci{Aka-Cari}
\def\langnames@langs@glot@akx{Aka-Kede}
\def\langnames@langs@glot@aka{Akan}
\def\langnames@langs@glot@ake{Akawaio}
\def\langnames@langs@glot@ahk{Akha}
\def\langnames@langs@glot@akv{Akhvakh}
\def\langnames@langs@glot@akl{Aklanon}
\def\langnames@langs@glot@akw{Akwa}
\def\langnames@langs@glot@nrz{Ala'ala}
\def\langnames@langs@glot@akz{Alabama}
\def\langnames@langs@glot@wbj{Alagwa}
\def\langnames@langs@glot@amp{Alamblak}
\def\langnames@langs@glot@btz{Alas}
\def\langnames@langs@glot@alh{Alawa}
\def\langnames@langs@glot@sqi{Albanian}
\def\langnames@langs@glot@ale{Aleut}
\def\langnames@langs@glot@alq{Algonquin}
\def\langnames@langs@glot@ald{Alladian}
\def\langnames@langs@glot@gsw{Alsatian}
\def\langnames@langs@glot@aes{Alsea}
\def\langnames@langs@glot@alt{Altai (Southern)}
\def\langnames@langs@glot@alp{Alune}
\def\langnames@langs@glot@ems{Alutiiq}
\def\langnames@langs@glot@alr{Alutor}
\def\langnames@langs@glot@aly{Alyawarra}
\def\langnames@langs@glot@amm{Ama}
\def\langnames@langs@glot@amc{Amahuaca}
\def\langnames@langs@glot@amn{Amanab}
\def\langnames@langs@glot@aie{Amara}
\def\langnames@langs@glot@amr{Amarakaeri}
\def\langnames@langs@glot@omb{Ambae (Lolovoli Northeast)}
\def\langnames@langs@glot@amk{Ambai}
\def\langnames@langs@glot@abt{Ambulas}
\def\langnames@langs@glot@adx{Amdo}
\def\langnames@langs@glot@aey{Amele}
\def\langnames@langs@glot@ase{American Sign Language}
\def\langnames@langs@glot@amh{Amharic}
\def\langnames@langs@glot@ami{Amis}
\def\langnames@langs@glot@amo{Amo}
\def\langnames@langs@glot@apz{Ampeeli}
\def\langnames@langs@glot@ame{Amuesha}
\def\langnames@langs@glot@amu{Amuzgo}
\def\langnames@langs@glot@imi{Anamuxra}
\def\langnames@langs@glot@ani{Andi}
\def\langnames@langs@glot@ano{Andoke}
\def\langnames@langs@glot@aty{Anejom}
\def\langnames@langs@glot@agm{Angaataha}
\def\langnames@langs@glot@njm{Angami}
\def\langnames@langs@glot@anc{Angas}
\def\langnames@langs@glot@agg{Anggor}
\def\langnames@langs@glot@aoa{Angolar}
\def\langnames@langs@glot@awg{Anguthimri}
\def\langnames@langs@glot@aoi{Anindilyakwa}
\def\langnames@langs@glot@nun{Anong}
\def\langnames@langs@glot@cko{Anufo}
\def\langnames@langs@glot@any{Anyi}
\def\langnames@langs@glot@anu{Anywa}
\def\langnames@langs@glot@anz{Anêm}
\def\langnames@langs@glot@njo{Ao}
\def\langnames@langs@glot@apm{Apache (Chiricahua)}
\def\langnames@langs@glot@apj{Apache (Jicarilla)}
\def\langnames@langs@glot@apw{Apache (Western)}
\def\langnames@langs@glot@apy{Apalaí}
\def\langnames@langs@glot@apt{Apatani}
\def\langnames@langs@glot@apn{Apinayé}
\def\langnames@langs@glot@apu{Apurinã}
\def\langnames@langs@glot@ard{Arabana}
\def\langnames@langs@glot@arl{Arabela}
\def\langnames@langs@glot@abv{Arabic (Bahrain)}
\def\langnames@langs@glot@mey{Arabic (Bani-Hassan)}
\def\langnames@langs@glot@shu{Arabic (Chadian)}
\def\langnames@langs@glot@ayl{Arabic (Eastern Libyan)}
\def\langnames@langs@glot@arz{Arabic (Egyptian)}
\def\langnames@langs@glot@afb{Arabic (Gulf)}
\def\langnames@langs@glot@acw{Arabic (Hijazi)}
\def\langnames@langs@glot@acm{Arabic (Iraqi)}
\def\langnames@langs@glot@acy{Arabic (Kormakiti)}
\def\langnames@langs@glot@arb{Arabic (Modern Standard)}
\def\langnames@langs@glot@ary{Arabic (Moroccan)}
\def\langnames@langs@glot@ajp{Arabic (Negev)}
\def\langnames@langs@glot@ayn{Arabic (San'ani)}
\def\langnames@langs@glot@apc{Arabic (Syrian)}
\def\langnames@langs@glot@aeb{Arabic (Tunisian)}
\def\langnames@langs@glot@rmz{Arakanese (Marma)}
\def\langnames@langs@glot@akr{Araki}
\def\langnames@langs@glot@atq{Aralle-Tabulahan}
\def\langnames@langs@glot@jbj{Arandai}
\def\langnames@langs@glot@aro{Araona}
\def\langnames@langs@glot@arp{Arapaho}
\def\langnames@langs@glot@aah{Arapesh (Abu)}
\def\langnames@langs@glot@ape{Arapesh (Mountain)}
\def\langnames@langs@glot@arv{Arbore}
\def\langnames@langs@glot@aqc{Archi}
\def\langnames@langs@glot@laz{Aribwatsa}
\def\langnames@langs@glot@ari{Arikara}
\def\langnames@langs@glot@hye{Armenian (Iranian)}
\def\langnames@langs@glot@hyw{Armenian (Western)}
\def\langnames@langs@glot@apr{Arop-Lokep}
\def\langnames@langs@glot@aia{Arosi}
\def\langnames@langs@glot@aer{Arrernte (Mparntwe)}
\def\langnames@langs@glot@are{Arrernte (Western)}
\def\langnames@langs@glot@cns{Asmat}
\def\langnames@langs@glot@asm{Assamese}
\def\langnames@langs@glot@ast{Asturian}
\def\langnames@langs@glot@asu{Asuriní}
\def\langnames@langs@glot@kuz{Atacameño}
\def\langnames@langs@glot@aqp{Atakapa}
\def\langnames@langs@glot@tay{Atayal}
\def\langnames@langs@glot@upv{Atchin}
\def\langnames@langs@glot@aph{Athpare}
\def\langnames@langs@glot@atj{Atikamekw}
\def\langnames@langs@glot@atw{Atsugewi}
\def\langnames@langs@glot@avt{Au}
\def\langnames@langs@glot@aul{Aulua}
\def\langnames@langs@glot@asf{Auslan}
\def\langnames@langs@glot@auy{Auyana}
\def\langnames@langs@glot@ava{Avar}
\def\langnames@langs@glot@avn{Avatime}
\def\langnames@langs@glot@avi{Avikam}
\def\langnames@langs@glot@avu{Avokaya}
\def\langnames@langs@glot@awb{Awa}
\def\langnames@langs@glot@kwi{Awa Pit}
\def\langnames@langs@glot@awa{Awadhi}
\def\langnames@langs@glot@awn{Awngi}
\def\langnames@langs@glot@kmn{Awtuw}
\def\langnames@langs@glot@auw{Awyi}
\def\langnames@langs@glot@nfl{Ayiwo}
\def\langnames@langs@glot@ayr{Aymara (Central)}
\def\langnames@langs@glot@aib{Aynu}
\def\langnames@langs@glot@ayo{Ayoreo}
\def\langnames@langs@glot@azb{Azari (Iranian)}
\def\langnames@langs@glot@koe{Baale}
\def\langnames@langs@glot@bvx{Babole}
\def\langnames@langs@glot@bav{Babungo}
\def\langnames@langs@glot@wdj{Bachamal}
\def\langnames@langs@glot@bfq{Badaga}
\def\langnames@langs@glot@bde{Bade}
\def\langnames@langs@glot@bia{Badimaya}
\def\langnames@langs@glot@ksf{Bafia}
\def\langnames@langs@glot@bfd{Bafut}
\def\langnames@langs@glot@bsp{Baga Sitemu}
\def\langnames@langs@glot@bmi{Bagirmi}
\def\langnames@langs@glot@fuu{Bagiro}
\def\langnames@langs@glot@bgq{Bagri}
\def\langnames@langs@glot@kva{Bagvalal}
\def\langnames@langs@glot@bdw{Baham}
\def\langnames@langs@glot@bjh{Bahinemo}
\def\langnames@langs@glot@bdq{Bahnar}
\def\langnames@langs@glot@bca{Bai}
\def\langnames@langs@glot@bdl{Bajau (Sama)}
\def\langnames@langs@glot@bdr{Bajau (West Coast)}
\def\langnames@langs@glot@bkc{Baka (in Cameroon)}
\def\langnames@langs@glot@bdh{Baka (in South Sudan)}
\def\langnames@langs@glot@bkq{Bakairí}
\def\langnames@langs@glot@bri{Bakueri}
\def\langnames@langs@glot@blw{Balangao}
\def\langnames@langs@glot@blz{Balantak}
\def\langnames@langs@glot@ban{Balinese}
\def\langnames@langs@glot@bft{Balti}
\def\langnames@langs@glot@bgn{Baluchi}
\def\langnames@langs@glot@ptu{Bambam}
\def\langnames@langs@glot@bam{Bambara}
\def\langnames@langs@glot@bax{Bamun}
\def\langnames@langs@glot@bcw{Bana}
\def\langnames@langs@glot@jaa{Banawá}
\def\langnames@langs@glot@bza{Bandi}
\def\langnames@langs@glot@bdy{Bandjalang}
\def\langnames@langs@glot@bgz{Banggai}
\def\langnames@langs@glot@bjb{Banggarla}
\def\langnames@langs@glot@bdg{Banggi}
\def\langnames@langs@glot@dba{Bangime}
\def\langnames@langs@glot@bvv{Baniva}
\def\langnames@langs@glot@bwi{Baniwa}
\def\langnames@langs@glot@abb{Bankon}
\def\langnames@langs@glot@bcm{Banoni}
\def\langnames@langs@glot@bnq{Bantik}
\def\langnames@langs@glot@peh{Bao'an}
\def\langnames@langs@glot@bci{Baoulé}
\def\langnames@langs@glot@loy{Baragaunle}
\def\langnames@langs@glot@bbb{Barai}
\def\langnames@langs@glot@brm{Barambu}
\def\langnames@langs@glot@bsn{Barasano}
\def\langnames@langs@glot@bcj{Bardi}
\def\langnames@langs@glot@mlp{Bargam}
\def\langnames@langs@glot@bfa{Bari}
\def\langnames@langs@glot@bba{Bariba}
\def\langnames@langs@glot@wra{Barupu}
\def\langnames@langs@glot@byr{Baruya}
\def\langnames@langs@glot@bae{Baré}
\def\langnames@langs@glot@mot{Barí}
\def\langnames@langs@glot@bsc{Basari}
\def\langnames@langs@glot@bas{Basaá}
\def\langnames@langs@glot@bak{Bashkir}
\def\langnames@langs@glot@eus{Basque}
\def\langnames@langs@glot@bya{Batak}
\def\langnames@langs@glot@btx{Batak (Karo)}
\def\langnames@langs@glot@bbc{Batak (Toba)}
\def\langnames@langs@glot@bhm{Bathari}
\def\langnames@langs@glot@bbd{Bau}
\def\langnames@langs@glot@brg{Baure}
\def\langnames@langs@glot@bvz{Bauzi}
\def\langnames@langs@glot@bgr{Bawm}
\def\langnames@langs@glot@bsw{Bayso}
\def\langnames@langs@glot@bxj{Bayungu}
\def\langnames@langs@glot@beq{Beembe}
\def\langnames@langs@glot@dbj{Begak-Ida'an}
\def\langnames@langs@glot@bej{Beja}
\def\langnames@langs@glot@byw{Belhare}
\def\langnames@langs@glot@blc{Bella Coola}
\def\langnames@langs@glot@bel{Belorussian}
\def\langnames@langs@glot@bem{Bemba}
\def\langnames@langs@glot@bef{Benabena}
\def\langnames@langs@glot@nhb{Beng}
\def\langnames@langs@glot@bng{Benga}
\def\langnames@langs@glot@ben{Bengali}
\def\langnames@langs@glot@ctg{Bengali (Chittagong)}
\def\langnames@langs@glot@bue{Beothuk}
\def\langnames@langs@glot@brf{Bera}
\def\langnames@langs@glot@shy{Berber (Chaouia)}
\def\langnames@langs@glot@grr{Berber (Figuig)}
\def\langnames@langs@glot@tzm{Berber (Middle Atlas)}
\def\langnames@langs@glot@mzb{Berber (Mzab)}
\def\langnames@langs@glot@rif{Berber (Rif)}
\def\langnames@langs@glot@siz{Berber (Siwa)}
\def\langnames@langs@glot@oua{Berber (Wargla)}
\def\langnames@langs@glot@brc{Berbice Dutch Creole}
\def\langnames@langs@glot@zag{Beria}
\def\langnames@langs@glot@bkl{Berik}
\def\langnames@langs@glot@wti{Berta}
\def\langnames@langs@glot@xub{Betta Kurumba}
\def\langnames@langs@glot@kap{Bezhta}
\def\langnames@langs@glot@bhb{Bhili}
\def\langnames@langs@glot@bho{Bhojpuri}
\def\langnames@langs@glot@unr{Bhumij}
\def\langnames@langs@glot@bif{Biafada}
\def\langnames@langs@glot@bhw{Biak}
\def\langnames@langs@glot@bth{Biatah}
\def\langnames@langs@glot@bid{Bidiya}
\def\langnames@langs@glot@bcl{Bikol}
\def\langnames@langs@glot@bip{Bila}
\def\langnames@langs@glot@bpr{Bilaan (Koronadal)}
\def\langnames@langs@glot@byn{Bilin}
\def\langnames@langs@glot@nbj{Bilinarra}
\def\langnames@langs@glot@bll{Biloxi}
\def\langnames@langs@glot@blb{Bilua}
\def\langnames@langs@glot@bhp{Bima}
\def\langnames@langs@glot@bim{Bimoba}
\def\langnames@langs@glot@bhg{Binandere}
\def\langnames@langs@glot@bin{Bini}
\def\langnames@langs@glot@gup{Bininj Gun-Wok}
\def\langnames@langs@glot@bkd{Binukid}
\def\langnames@langs@glot@bjr{Binumarien}
\def\langnames@langs@glot@bzr{Biri}
\def\langnames@langs@glot@bom{Birom}
\def\langnames@langs@glot@bvq{Birri}
\def\langnames@langs@glot@bib{Bisa}
\def\langnames@langs@glot@bis{Bislama}
\def\langnames@langs@glot@bla{Blackfoot}
\def\langnames@langs@glot@kvg{Boazi (Kuni)}
\def\langnames@langs@glot@bni{Bobangi}
\def\langnames@langs@glot@bbo{Bobo Madaré (Northern)}
\def\langnames@langs@glot@brx{Bodo}
\def\langnames@langs@glot@bzf{Boiken}
\def\langnames@langs@glot@bqc{Boko}
\def\langnames@langs@glot@bol{Bole}
\def\langnames@langs@glot@bli{Bolia}
\def\langnames@langs@glot@bot{Bongo}
\def\langnames@langs@glot@bpu{Bongu}
\def\langnames@langs@glot@lbk{Bontok}
\def\langnames@langs@glot@boa{Bora}
\def\langnames@langs@glot@adi{Bori}
\def\langnames@langs@glot@bor{Bororo}
\def\langnames@langs@glot@brn{Boruca}
\def\langnames@langs@glot@bos{Bosnian}
\def\langnames@langs@glot@boz{Bozo (Tigemaxo)}
\def\langnames@langs@glot@brh{Brahui}
\def\langnames@langs@glot@brb{Brao}
\def\langnames@langs@glot@bre{Breton}
\def\langnames@langs@glot@bzd{Bribri}
\def\langnames@langs@glot@bfi{British Sign Language}
\def\langnames@langs@glot@tcs{Broken}
\def\langnames@langs@glot@bkk{Brokskat}
\def\langnames@langs@glot@bru{Bru (Eastern)}
\def\langnames@langs@glot@brv{Bru (Western)}
\def\langnames@langs@glot@bvb{Bubi}
\def\langnames@langs@glot@buu{Budu}
\def\langnames@langs@glot@bdk{Budukh}
\def\langnames@langs@glot@bdm{Buduma}
\def\langnames@langs@glot@bug{Bugis}
\def\langnames@langs@glot@sab{Buglere}
\def\langnames@langs@glot@bgg{Bugun}
\def\langnames@langs@glot@buo{Buin}
\def\langnames@langs@glot@nmg{Bujeba}
\def\langnames@langs@glot@bxk{Bukusu}
\def\langnames@langs@glot@bul{Bulgarian}
\def\langnames@langs@glot@bwu{Buli (in Ghana)}
\def\langnames@langs@glot@bzq{Buli (in Indonesia)}
\def\langnames@langs@glot@bum{Bulu}
\def\langnames@langs@glot@tkw{Buma}
\def\langnames@langs@glot@bfu{Bunan}
\def\langnames@langs@glot@buh{Bunu (Younuo)}
\def\langnames@langs@glot@bck{Bunuba}
\def\langnames@langs@glot@bwr{Bura-Pabir}
\def\langnames@langs@glot@bvr{Burarra}
\def\langnames@langs@glot@bxm{Buriat}
\def\langnames@langs@glot@bji{Burji}
\def\langnames@langs@glot@mya{Burmese}
\def\langnames@langs@glot@mhs{Buru}
\def\langnames@langs@glot@bmu{Burum}
\def\langnames@langs@glot@bds{Burunge}
\def\langnames@langs@glot@bsk{Burushaski}
\def\langnames@langs@glot@bqp{Busa}
\def\langnames@langs@glot@buf{Bushoong}
\def\langnames@langs@glot@ngc{Bwele}
\def\langnames@langs@glot@bee{Byansi}
\def\langnames@langs@glot@bev{Bété}
\def\langnames@langs@glot@cjp{Cabécar}
\def\langnames@langs@glot@cbv{Cacua}
\def\langnames@langs@glot@cad{Caddo}
\def\langnames@langs@glot@chl{Cahuilla}
\def\langnames@langs@glot@cak{Cakchiquel}
\def\langnames@langs@glot@rab{Camling}
\def\langnames@langs@glot@cjo{Campa Pajonal Asheninca}
\def\langnames@langs@glot@kbh{Camsá}
\def\langnames@langs@glot@knm{Canamarí}
\def\langnames@langs@glot@cbu{Candoshi}
\def\langnames@langs@glot@ram{Canela}
\def\langnames@langs@glot@yue{Cantonese}
\def\langnames@langs@glot@kaq{Capanahua}
\def\langnames@langs@glot@cbc{Carapana}
\def\langnames@langs@glot@car{Carib}
\def\langnames@langs@glot@mch{Carib (De'kwana)}
\def\langnames@langs@glot@cal{Carolinian}
\def\langnames@langs@glot@crx{Carrier}
\def\langnames@langs@glot@cbr{Cashibo}
\def\langnames@langs@glot@cbs{Cashinahua}
\def\langnames@langs@glot@cat{Catalan}
\def\langnames@langs@glot@chc{Catawba}
\def\langnames@langs@glot@cto{Catio}
\def\langnames@langs@glot@cav{Cavineña}
\def\langnames@langs@glot@cbi{Cayapa}
\def\langnames@langs@glot@cay{Cayuga}
\def\langnames@langs@glot@cyb{Cayuvava}
\def\langnames@langs@glot@ceb{Cebuano}
\def\langnames@langs@glot@old{Chaga}
\def\langnames@langs@glot@suq{Chai}
\def\langnames@langs@glot@cld{Chaldean (Modern)}
\def\langnames@langs@glot@cjm{Cham (Eastern)}
\def\langnames@langs@glot@cja{Cham (Western)}
\def\langnames@langs@glot@cji{Chamalal}
\def\langnames@langs@glot@can{Chambri}
\def\langnames@langs@glot@cha{Chamorro}
\def\langnames@langs@glot@nbc{Chang}
\def\langnames@langs@glot@chx{Chantyal}
\def\langnames@langs@glot@tuu{Chasta Costa}
\def\langnames@langs@glot@cya{Chatino (Nopala)}
\def\langnames@langs@glot@cta{Chatino (Tataltepec)}
\def\langnames@langs@glot@ctp{Chatino (Yaitepec)}
\def\langnames@langs@glot@cdn{Chaudangsi}
\def\langnames@langs@glot@cbk{Chavacano}
\def\langnames@langs@glot@cbt{Chayahuita}
\def\langnames@langs@glot@che{Chechen}
\def\langnames@langs@glot@cjh{Chehalis (Upper)}
\def\langnames@langs@glot@mrn{Cheke Holo}
\def\langnames@langs@glot@xch{Chemakum}
\def\langnames@langs@glot@cdm{Chepang}
\def\langnames@langs@glot@chr{Cherokee}
\def\langnames@langs@glot@chy{Cheyenne}
\def\langnames@langs@glot@nya{Chichewa}
\def\langnames@langs@glot@pei{Chichimeca-Jonaz}
\def\langnames@langs@glot@cic{Chickasaw}
\def\langnames@langs@glot@cob{Chicomuceltec}
\def\langnames@langs@glot@cid{Chimariko}
\def\langnames@langs@glot@cbg{Chimila}
\def\langnames@langs@glot@mrh{Chin (Mara)}
\def\langnames@langs@glot@csy{Chin (Siyin)}
\def\langnames@langs@glot@ctd{Chin (Tiddim)}
\def\langnames@langs@glot@cco{Chinantec (Comaltepec)}
\def\langnames@langs@glot@cle{Chinantec (Lealao)}
\def\langnames@langs@glot@cpa{Chinantec (Palantla)}
\def\langnames@langs@glot@chq{Chinantec (Quiotepec)}
\def\langnames@langs@glot@cuc{Chinantec (San Felipe Usila)}
\def\langnames@langs@glot@cso{Chinantec (Sochiapan)}
\def\langnames@langs@glot@cnt{Chinantec (Tepetotutla)}
\def\langnames@langs@glot@csl{Chinese Sign Language}
\def\langnames@langs@glot@chh{Chinook (Lower)}
\def\langnames@langs@glot@wac{Chinook (Upper)}
\def\langnames@langs@glot@cap{Chipaya}
\def\langnames@langs@glot@chp{Chipewyan}
\def\langnames@langs@glot@cax{Chiquitano}
\def\langnames@langs@glot@gui{Chiriguano}
\def\langnames@langs@glot@ctm{Chitimacha}
\def\langnames@langs@glot@coz{Chocho}
\def\langnames@langs@glot@cho{Choctaw}
\def\langnames@langs@glot@ctu{Chol}
\def\langnames@langs@glot@cht{Cholón}
\def\langnames@langs@glot@chd{Chontal (Highland)}
\def\langnames@langs@glot@clo{Chontal (Huamelultec Oaxaca)}
\def\langnames@langs@glot@chf{Chontal Maya}
\def\langnames@langs@glot@caa{Chortí}
\def\langnames@langs@glot@crw{Chrau}
\def\langnames@langs@glot@cje{Chru}
\def\langnames@langs@glot@cjv{Chuave}
\def\langnames@langs@glot@cac{Chuj}
\def\langnames@langs@glot@ckt{Chukchi}
\def\langnames@langs@glot@clw{Chulym}
\def\langnames@langs@glot@boi{Chumash (Barbareño)}
\def\langnames@langs@glot@inz{Chumash (Ineseño)}
\def\langnames@langs@glot@ncu{Chumburung}
\def\langnames@langs@glot@chk{Chuukese}
\def\langnames@langs@glot@chv{Chuvash}
\def\langnames@langs@glot@cao{Chácobo}
\def\langnames@langs@glot@lua{CiLuba}
\def\langnames@langs@glot@clm{Clallam}
\def\langnames@langs@glot@xcw{Coahuilteco}
\def\langnames@langs@glot@cod{Cocama}
\def\langnames@langs@glot@coc{Cocopa}
\def\langnames@langs@glot@crd{Coeur d'Alene}
\def\langnames@langs@glot@con{Cofán}
\def\langnames@langs@glot@kog{Cogui}
\def\langnames@langs@glot@col{Columbia-Wenatchi}
\def\langnames@langs@glot@com{Comanche}
\def\langnames@langs@glot@xcm{Comecrudo}
\def\langnames@langs@glot@swb{Comorian}
\def\langnames@langs@glot@coo{Comox}
\def\langnames@langs@glot@csz{Coos (Hanis)}
\def\langnames@langs@glot@cop{Coptic}
\def\langnames@langs@glot@crn{Cora}
\def\langnames@langs@glot@cor{Cornish}
\def\langnames@langs@glot@crk{Cree (Plains)}
\def\langnames@langs@glot@csw{Cree (Swampy)}
\def\langnames@langs@glot@mus{Creek}
\def\langnames@langs@glot@crh{Crimean Tatar}
\def\langnames@langs@glot@cro{Crow}
\def\langnames@langs@glot@cua{Cua}
\def\langnames@langs@glot@cub{Cubeo}
\def\langnames@langs@glot@cui{Cuiba}
\def\langnames@langs@glot@cuy{Cuitlatec}
\def\langnames@langs@glot@cul{Culina}
\def\langnames@langs@glot@cup{Cupeño}
\def\langnames@langs@glot@kpc{Curripaco}
\def\langnames@langs@glot@ces{Czech}
\def\langnames@langs@glot@cam{Cèmuhî}
\def\langnames@langs@glot@kzf{Da'a}
\def\langnames@langs@glot@dbq{Daba}
\def\langnames@langs@glot@dav{Dabida}
\def\langnames@langs@glot@mps{Dadibi}
\def\langnames@langs@glot@dgz{Daga}
\def\langnames@langs@glot@dga{Dagaare}
\def\langnames@langs@glot@dag{Dagbani}
\def\langnames@langs@glot@dta{Dagur}
\def\langnames@langs@glot@dal{Dahalo}
\def\langnames@langs@glot@daj{Daju (Dar Fur)}
\def\langnames@langs@glot@dak{Dakota}
\def\langnames@langs@glot@mbp{Damana}
\def\langnames@langs@glot@dnj{Dan}
\def\langnames@langs@glot@daa{Dangaléat (Western)}
\def\langnames@langs@glot@dni{Dani (Lower Grand Valley)}
\def\langnames@langs@glot@dan{Danish}
\def\langnames@langs@glot@dry{Darai}
\def\langnames@langs@glot@dar{Dargwa}
\def\langnames@langs@glot@prs{Dari}
\def\langnames@langs@glot@drd{Darma}
\def\langnames@langs@glot@tcc{Datooga}
\def\langnames@langs@glot@dai{Day}
\def\langnames@langs@glot@afn{Defaka}
\def\langnames@langs@glot@deg{Degema}
\def\langnames@langs@glot@ing{Degexit'an}
\def\langnames@langs@glot@dny{Dení}
\def\langnames@langs@glot@des{Desano}
\def\langnames@langs@glot@shg{Deti}
\def\langnames@langs@glot@der{Deuri}
\def\langnames@langs@glot@gsg{Deutsche Gebärdensprache}
\def\langnames@langs@glot@dsh{Dhaasanac}
\def\langnames@langs@glot@dhl{Dhalandji}
\def\langnames@langs@glot@tbh{Dharawal}
\def\langnames@langs@glot@dhr{Dhargari}
\def\langnames@langs@glot@xgm{Dharumbal}
\def\langnames@langs@glot@dhi{Dhimal}
\def\langnames@langs@glot@div{Dhivehi}
\def\langnames@langs@glot@dhu{Dhurga}
\def\langnames@langs@glot@did{Didinga}
\def\langnames@langs@glot@mhu{Digaro}
\def\langnames@langs@glot@dur{Dii}
\def\langnames@langs@glot@dis{Dimasa}
\def\langnames@langs@glot@dim{Dime}
\def\langnames@langs@glot@diz{Ding}
\def\langnames@langs@glot@din{Dinka}
\def\langnames@langs@glot@dyo{Diola-Fogny}
\def\langnames@langs@glot@csk{Diola-Kasa}
\def\langnames@langs@glot@dif{Diyari}
\def\langnames@langs@glot@mdx{Dizi}
\def\langnames@langs@glot@dyy{Djabugay}
\def\langnames@langs@glot@djr{Djambarrpuyngu}
\def\langnames@langs@glot@duj{Djapu}
\def\langnames@langs@glot@ddj{Djaru}
\def\langnames@langs@glot@dji{Djinang}
\def\langnames@langs@glot@jig{Djingili}
\def\langnames@langs@glot@kbv{Dla (Proper)}
\def\langnames@langs@glot@kvo{Dobel}
\def\langnames@langs@glot@dgo{Dogri}
\def\langnames@langs@glot@dlg{Dolgan}
\def\langnames@langs@glot@dmk{Domaaki}
\def\langnames@langs@glot@rmt{Domari}
\def\langnames@langs@glot@kmc{Dong (Southern)}
\def\langnames@langs@glot@doo{Dongo}
\def\langnames@langs@glot@dds{Donno So}
\def\langnames@langs@glot@tds{Doutai}
\def\langnames@langs@glot@dow{Doyayo}
\def\langnames@langs@glot@dhv{Drehu}
\def\langnames@langs@glot@dua{Duala}
\def\langnames@langs@glot@dud{Duka}
\def\langnames@langs@glot@gwd{Dullay (Gollango)}
\def\langnames@langs@glot@duu{Dulong}
\def\langnames@langs@glot@dma{Duma}
\def\langnames@langs@glot@dgc{Dumagat (Casiguran)}
\def\langnames@langs@glot@dus{Dumi}
\def\langnames@langs@glot@vam{Dumo}
\def\langnames@langs@glot@duc{Duna}
\def\langnames@langs@glot@nld{Dutch}
\def\langnames@langs@glot@zea{Dutch (Zeeuws)}
\def\langnames@langs@glot@dyi{Dyimini}
\def\langnames@langs@glot@dbl{Dyirbal}
\def\langnames@langs@glot@dyu{Dyula}
\def\langnames@langs@glot@kwa{Dâw}
\def\langnames@langs@glot@igb{Ebira}
\def\langnames@langs@glot@etr{Edolo}
\def\langnames@langs@glot@erk{Efate (South)}
\def\langnames@langs@glot@efi{Efik}
\def\langnames@langs@glot@ega{Ega}
\def\langnames@langs@glot@eip{Eipo}
\def\langnames@langs@glot@etu{Ejagham}
\def\langnames@langs@glot@ekg{Ekari}
\def\langnames@langs@glot@eko{Ekoti}
\def\langnames@langs@glot@mrf{Elseng}
\def\langnames@langs@glot@ema{Emai}
\def\langnames@langs@glot@emb{Embaloh}
\def\langnames@langs@glot@cmi{Embera Chami}
\def\langnames@langs@glot@emp{Emberá (Northern)}
\def\langnames@langs@glot@amy{Emmi}
\def\langnames@langs@glot@enq{Enga}
\def\langnames@langs@glot@enn{Engenni}
\def\langnames@langs@glot@eno{Enggano}
\def\langnames@langs@glot@eng{English}
\def\langnames@langs@glot@gey{Enya}
\def\langnames@langs@glot@sja{Epena Pedee}
\def\langnames@langs@glot@erg{Erromangan}
\def\langnames@langs@glot@ese{Ese Ejja}
\def\langnames@langs@glot@esq{Esselen}
\def\langnames@langs@glot@ekk{Estonian}
\def\langnames@langs@glot@ets{Etsako}
\def\langnames@langs@glot@eve{Even}
\def\langnames@langs@glot@ewe{Ewe}
\def\langnames@langs@glot@ewo{Ewondo}
\def\langnames@langs@glot@eya{Eyak}
\def\langnames@langs@glot@fao{Faroese}
\def\langnames@langs@glot@faa{Fasu}
\def\langnames@langs@glot@fmp{Fe'fe'}
\def\langnames@langs@glot@fij{Fijian}
\def\langnames@langs@glot@fin{Finnish}
\def\langnames@langs@glot@fse{Finnish Sign Language}
\def\langnames@langs@glot@foi{Foe}
\def\langnames@langs@glot@ppo{Folopa}
\def\langnames@langs@glot@fon{Fongbe}
\def\langnames@langs@glot@frd{Fordata}
\def\langnames@langs@glot@for{Fore}
\def\langnames@langs@glot@sac{Fox}
\def\langnames@langs@glot@fra{French}
\def\langnames@langs@glot@fry{Frisian}
\def\langnames@langs@glot@frs{Frisian (Eastern)}
\def\langnames@langs@glot@frr{Frisian (North)}
\def\langnames@langs@glot@fuh{Ful (Liptako)}
\def\langnames@langs@glot@fuf{Fula (Guinean)}
\def\langnames@langs@glot@fub{Fulfulde (Adamawa)}
\def\langnames@langs@glot@ffm{Fulfulde (Maasina)}
\def\langnames@langs@glot@fuv{Fulfulde (Nigerian)}
\def\langnames@langs@glot@fun{Fulniô}
\def\langnames@langs@glot@fvr{Fur}
\def\langnames@langs@glot@fud{Futuna (East)}
\def\langnames@langs@glot@fut{Futuna-Aniwa}
\def\langnames@langs@glot@cdo{Fuzhou}
\def\langnames@langs@glot@pym{Fyem}
\def\langnames@langs@glot@gqa{Ga'anda}
\def\langnames@langs@glot@gbu{Gaagudju}
\def\langnames@langs@glot@dhg{Gaalpu}
\def\langnames@langs@glot@gdb{Gadaba (Kondekor)}
\def\langnames@langs@glot@ged{Gade}
\def\langnames@langs@glot@gaj{Gadsup}
\def\langnames@langs@glot@gla{Gaelic (Scots)}
\def\langnames@langs@glot@gag{Gagauz}
\def\langnames@langs@glot@gah{Gahuku}
\def\langnames@langs@glot@gbi{Galela}
\def\langnames@langs@glot@glg{Galician}
\def\langnames@langs@glot@adl{Galo}
\def\langnames@langs@glot@kld{Gamilaraay}
\def\langnames@langs@glot@gmv{Gamo}
\def\langnames@langs@glot@pwg{Gapapaiwa}
\def\langnames@langs@glot@grt{Garo}
\def\langnames@langs@glot@wrk{Garrwa}
\def\langnames@langs@glot@gyb{Garus}
\def\langnames@langs@glot@cab{Garífuna}
\def\langnames@langs@glot@gvo{Gavião}
\def\langnames@langs@glot@gay{Gayo}
\def\langnames@langs@glot@gya{Gbaya (Northwest)}
\def\langnames@langs@glot@gso{Gbaya (Southwest)}
\def\langnames@langs@glot@gbp{Gbeya Bossangoa}
\def\langnames@langs@glot@nlg{Gela}
\def\langnames@langs@glot@gqu{Gelao}
\def\langnames@langs@glot@kat{Georgian}
\def\langnames@langs@glot@deu{German}
\def\langnames@langs@glot@bar{German (Bavarian)}
\def\langnames@langs@glot@ksh{German (Ripuarian)}
\def\langnames@langs@glot@wep{German (Westphalian)}
\def\langnames@langs@glot@aaa{Ghotuo}
\def\langnames@langs@glot@ghl{Ghulfan}
\def\langnames@langs@glot@gih{Gidabal}
\def\langnames@langs@glot@gid{Gidar}
\def\langnames@langs@glot@glk{Gilaki}
\def\langnames@langs@glot@bcq{Gimira}
\def\langnames@langs@glot@git{Gitksan}
\def\langnames@langs@glot@gis{Giziga}
\def\langnames@langs@glot@guc{Goajiro}
\def\langnames@langs@glot@god{Godié}
\def\langnames@langs@glot@gdo{Godoberi}
\def\langnames@langs@glot@ank{Goemai}
\def\langnames@langs@glot@ggw{Gogodala}
\def\langnames@langs@glot@gju{Gojri}
\def\langnames@langs@glot@gkn{Gokana}
\def\langnames@langs@glot@gol{Gola}
\def\langnames@langs@glot@gvf{Golin}
\def\langnames@langs@glot@gno{Gondi}
\def\langnames@langs@glot@gni{Gooniyandi}
\def\langnames@langs@glot@gor{Gorontalo}
\def\langnames@langs@glot@gow{Gorowa}
\def\langnames@langs@glot@grj{Grebo}
\def\langnames@langs@glot@ell{Greek (Modern)}
\def\langnames@langs@glot@gss{Greek Sign Language}
\def\langnames@langs@glot@kal{Greenlandic (West)}
\def\langnames@langs@glot@guh{Guahibo}
\def\langnames@langs@glot@gub{Guajajara}
\def\langnames@langs@glot@gum{Guambiano}
\def\langnames@langs@glot@gva{Guana}
\def\langnames@langs@glot@gvc{Guanano}
\def\langnames@langs@glot@gug{Guaraní}
\def\langnames@langs@glot@var{Guarijío}
\def\langnames@langs@glot@gta{Guató}
\def\langnames@langs@glot@guo{Guayabero}
\def\langnames@langs@glot@gde{Gude}
\def\langnames@langs@glot@gdf{Guduf}
\def\langnames@langs@glot@ktd{Gugada}
\def\langnames@langs@glot@ggd{Gugadj}
\def\langnames@langs@glot@ghs{Guhu-Samane}
\def\langnames@langs@glot@gcr{Guianese French Creole}
\def\langnames@langs@glot@pov{Guinea Bissau Crioulo}
\def\langnames@langs@glot@guj{Gujarati}
\def\langnames@langs@glot@kcm{Gula (in Central African Republic)}
\def\langnames@langs@glot@glj{Gula Iro}
\def\langnames@langs@glot@gnn{Gumatj}
\def\langnames@langs@glot@gvs{Gumawana}
\def\langnames@langs@glot@kgs{Gumbaynggir}
\def\langnames@langs@glot@guk{Gumuz}
\def\langnames@langs@glot@wlg{Gunbalang}
\def\langnames@langs@glot@guw{Gungbe}
\def\langnames@langs@glot@gww{Gunin}
\def\langnames@langs@glot@yas{Gunu}
\def\langnames@langs@glot@gyy{Gunya}
\def\langnames@langs@glot@guf{Gupapuyngu}
\def\langnames@langs@glot@gnr{Gureng Gureng}
\def\langnames@langs@glot@gur{Gurenne}
\def\langnames@langs@glot@gue{Gurindji}
\def\langnames@langs@glot@gux{Gurma}
\def\langnames@langs@glot@goa{Guro}
\def\langnames@langs@glot@gge{Gurr-goni}
\def\langnames@langs@glot@guz{Gusii}
\def\langnames@langs@glot@gbj{Gutob}
\def\langnames@langs@glot@kky{Guugu Yimidhirr}
\def\langnames@langs@glot@gbr{Gwari}
\def\langnames@langs@glot@kcg{Gworok}
\def\langnames@langs@glot@gaa{Gã}
\def\langnames@langs@glot@pue{Gününa Küne}
\def\langnames@langs@glot@hts{Hadza}
\def\langnames@langs@glot@hai{Haida}
\def\langnames@langs@glot@hdn{Haida (Northern)}
\def\langnames@langs@glot@has{Haisla}
\def\langnames@langs@glot@hat{Haitian Creole}
\def\langnames@langs@glot@hak{Hakka}
\def\langnames@langs@glot@hal{Halang}
\def\langnames@langs@glot@hlb{Halbi}
\def\langnames@langs@glot@hla{Halia}
\def\langnames@langs@glot@amf{Hamer}
\def\langnames@langs@glot@hmt{Hamtai}
\def\langnames@langs@glot@wos{Hanga Hundi}
\def\langnames@langs@glot@hni{Hani}
\def\langnames@langs@glot@hnn{Hanunóo}
\def\langnames@langs@glot@har{Harari}
\def\langnames@langs@glot@hss{Harsusi}
\def\langnames@langs@glot@tmd{Haruai}
\def\langnames@langs@glot@had{Hatam}
\def\langnames@langs@glot@hau{Hausa}
\def\langnames@langs@glot@haw{Hawaiian}
\def\langnames@langs@glot@hwc{Hawaiian Creole}
\def\langnames@langs@glot@hac{Hawrami}
\def\langnames@langs@glot@hay{Haya}
\def\langnames@langs@glot@vay{Hayu}
\def\langnames@langs@glot@xed{Hdi}
\def\langnames@langs@glot@heb{Hebrew (Modern)}
\def\langnames@langs@glot@heh{Hehe}
\def\langnames@langs@glot@hei{Heiltsuk}
\def\langnames@langs@glot@hem{Hemba}
\def\langnames@langs@glot@her{Herero}
\def\langnames@langs@glot@hid{Hidatsa}
\def\langnames@langs@glot@hil{Hiligaynon}
\def\langnames@langs@glot@hin{Hindi}
\def\langnames@langs@glot@gin{Hinuq}
\def\langnames@langs@glot@hix{Hixkaryana}
\def\langnames@langs@glot@lic{Hlai (Baoding)}
\def\langnames@langs@glot@hmr{Hmar}
\def\langnames@langs@glot@mww{Hmong Daw}
\def\langnames@langs@glot@hnj{Hmong Njua}
\def\langnames@langs@glot@hoc{Ho}
\def\langnames@langs@glot@hoa{Hoava}
\def\langnames@langs@glot@hoo{Holoholo}
\def\langnames@langs@glot@hks{Hong Kong Sign Language}
\def\langnames@langs@glot@hop{Hopi}
\def\langnames@langs@glot@hre{Hre}
\def\langnames@langs@glot@ygr{Hua}
\def\langnames@langs@glot@hub{Huambisa}
\def\langnames@langs@glot@hus{Huastec}
\def\langnames@langs@glot@huv{Huave (San Mateo del Mar)}
\def\langnames@langs@glot@hch{Huichol}
\def\langnames@langs@glot@hto{Huitoto (Minica)}
\def\langnames@langs@glot@hux{Huitoto (Muinane)}
\def\langnames@langs@glot@huu{Huitoto (Murui)}
\def\langnames@langs@glot@hke{Hunde}
\def\langnames@langs@glot@hun{Hungarian}
\def\langnames@langs@glot@huz{Hunzib}
\def\langnames@langs@glot@jup{Hup}
\def\langnames@langs@glot@hup{Hupa}
\def\langnames@langs@glot@csh{Hyow}
\def\langnames@langs@glot@ksi{I'saka}
\def\langnames@langs@glot@iai{Iaai}
\def\langnames@langs@glot@ian{Iatmul}
\def\langnames@langs@glot@tmu{Iau}
\def\langnames@langs@glot@iba{Iban}
\def\langnames@langs@glot@ibg{Ibanag}
\def\langnames@langs@glot@ibb{Ibibio}
\def\langnames@langs@glot@isl{Icelandic}
\def\langnames@langs@glot@icl{Icelandic Sign Language}
\def\langnames@langs@glot@idu{Idoma}
\def\langnames@langs@glot@clk{Idu}
\def\langnames@langs@glot@viv{Iduna}
\def\langnames@langs@glot@mxe{Ifira-Mele}
\def\langnames@langs@glot@ifb{Ifugao (Batad)}
\def\langnames@langs@glot@ifm{Ifumu}
\def\langnames@langs@glot@ibo{Igbo}
\def\langnames@langs@glot@ige{Igede}
\def\langnames@langs@glot@ign{Ignaciano}
\def\langnames@langs@glot@ihp{Iha}
\def\langnames@langs@glot@ijc{Ijo (Kolokuma)}
\def\langnames@langs@glot@ikx{Ik}
\def\langnames@langs@glot@arh{Ika}
\def\langnames@langs@glot@ilb{Ila}
\def\langnames@langs@glot@mia{Illinois}
\def\langnames@langs@glot@ilo{Ilocano}
\def\langnames@langs@glot@imn{Imonda}
\def\langnames@langs@glot@szp{Inanwatan}
\def\langnames@langs@glot@ins{Indo-Pakistani Sign Language (Indian dialects)}
\def\langnames@langs@glot@pks{Indo-Pakistani Sign Language (Karachi dialect)}
\def\langnames@langs@glot@ind{Indonesian}
\def\langnames@langs@glot@pmy{Indonesian (Papuan)}
\def\langnames@langs@glot@inb{Inga}
\def\langnames@langs@glot@tbi{Ingessana}
\def\langnames@langs@glot@inh{Ingush}
\def\langnames@langs@glot@ynd{Innamincka}
\def\langnames@langs@glot@ils{International Sign}
\def\langnames@langs@glot@ike{Inuktitut (Salluit)}
\def\langnames@langs@glot@iqu{Iquito}
\def\langnames@langs@glot@irn{Iranxe}
\def\langnames@langs@glot@irk{Iraqw}
\def\langnames@langs@glot@irh{Irarutu}
\def\langnames@langs@glot@gle{Irish}
\def\langnames@langs@glot@isg{Irish Sign Language}
\def\langnames@langs@glot@its{Isekiri}
\def\langnames@langs@glot@isk{Ishkashimi}
\def\langnames@langs@glot@srl{Isirawa}
\def\langnames@langs@glot@isd{Isnag}
\def\langnames@langs@glot@iso{Isoko}
\def\langnames@langs@glot@isr{Israeli Sign Language}
\def\langnames@langs@glot@ita{Italian}
\def\langnames@langs@glot@egl{Italian (Bologna)}
\def\langnames@langs@glot@lij{Italian (Genoa)}
\def\langnames@langs@glot@nap{Italian (Napolitanian)}
\def\langnames@langs@glot@pms{Italian (Turinese)}
\def\langnames@langs@glot@itv{Itawis}
\def\langnames@langs@glot@itl{Itelmen}
\def\langnames@langs@glot@ito{Itonama}
\def\langnames@langs@glot@itz{Itzaj}
\def\langnames@langs@glot@ivb{Ivatan}
\def\langnames@langs@glot@ibd{Iwaidja}
\def\langnames@langs@glot@iwm{Iwam}
\def\langnames@langs@glot@yom{Iwoyo}
\def\langnames@langs@glot@ixc{Ixcatec}
\def\langnames@langs@glot@ixl{Ixil}
\def\langnames@langs@glot@izr{Izere}
\def\langnames@langs@glot@izh{Izhor}
\def\langnames@langs@glot@izz{Izi}
\def\langnames@langs@glot@esi{Iñupiaq}
\def\langnames@langs@glot@jbt{Jabutí}
\def\langnames@langs@glot@jae{Jabêm}
\def\langnames@langs@glot@jda{Jad}
\def\langnames@langs@glot@jhi{Jahai}
\def\langnames@langs@glot@jac{Jakaltek}
\def\langnames@langs@glot@jam{Jamaican Creole}
\def\langnames@langs@glot@djd{Jaminjung}
\def\langnames@langs@glot@djm{Jamsay}
\def\langnames@langs@glot@jpn{Japanese}
\def\langnames@langs@glot@jru{Japreria}
\def\langnames@langs@glot@jqr{Jaqaru}
\def\langnames@langs@glot@anq{Jarawa (in Andamans)}
\def\langnames@langs@glot@jav{Javanese}
\def\langnames@langs@glot@jeb{Jebero}
\def\langnames@langs@glot@jeh{Jeh}
\def\langnames@langs@glot@jek{Jeli}
\def\langnames@langs@glot@tow{Jemez}
\def\langnames@langs@glot@jya{Jiarong}
\def\langnames@langs@glot@shv{Jibbali}
\def\langnames@langs@glot@kac{Jingpho}
\def\langnames@langs@glot@jiu{Jino}
\def\langnames@langs@glot@jiv{Jivaro}
\def\langnames@langs@glot@rgk{Johari}
\def\langnames@langs@glot@tlo{Jomang}
\def\langnames@langs@glot@jun{Juang}
\def\langnames@langs@glot@nst{Jugli}
\def\langnames@langs@glot@jbu{Jukun}
\def\langnames@langs@glot@bex{Jur Mödö}
\def\langnames@langs@glot@juc{Jurchen}
\def\langnames@langs@glot@jur{Juruna}
\def\langnames@langs@glot@ktz{Ju|'hoan}
\def\langnames@langs@glot@jua{Júma}
\def\langnames@langs@glot@kek{K'ekchí}
\def\langnames@langs@glot@kbd{Kabardian}
\def\langnames@langs@glot@xkp{Kabatei}
\def\langnames@langs@glot@kbp{Kabiyé}
\def\langnames@langs@glot@nbu{Kabui}
\def\langnames@langs@glot@kab{Kabyle}
\def\langnames@langs@glot@xac{Kachari}
\def\langnames@langs@glot@kzj{Kadazan}
\def\langnames@langs@glot@kbc{Kadiwéu}
\def\langnames@langs@glot@kdm{Kagoma}
\def\langnames@langs@glot@kki{Kagulu}
\def\langnames@langs@glot@kct{Kaian}
\def\langnames@langs@glot@lew{Kaili}
\def\langnames@langs@glot@kgp{Kaingang}
\def\langnames@langs@glot@kxa{Kairiru}
\def\langnames@langs@glot@kgk{Kaiwá}
\def\langnames@langs@glot@tbd{Kaki Ae}
\def\langnames@langs@glot@mwp{Kala Lagaw Ya}
\def\langnames@langs@glot@kmh{Kalam}
\def\langnames@langs@glot@gwc{Kalami}
\def\langnames@langs@glot@kck{Kalanga}
\def\langnames@langs@glot@kyl{Kalapuya}
\def\langnames@langs@glot@kls{Kalasha}
\def\langnames@langs@glot@fla{Kalispel}
\def\langnames@langs@glot@ktg{Kalkatungu}
\def\langnames@langs@glot@bco{Kaluli}
\def\langnames@langs@glot@kay{Kamaiurá}
\def\langnames@langs@glot@kbq{Kamano-Kafe}
\def\langnames@langs@glot@kms{Kamasau}
\def\langnames@langs@glot@xas{Kamass}
\def\langnames@langs@glot@kam{Kamba}
\def\langnames@langs@glot@xbr{Kambera}
\def\langnames@langs@glot@kbx{Kambot}
\def\langnames@langs@glot@kcu{Kami}
\def\langnames@langs@glot@kgq{Kamoro}
\def\langnames@langs@glot@xmu{Kamu}
\def\langnames@langs@glot@ogo{Kana}
\def\langnames@langs@glot@kna{Kanakuru}
\def\langnames@langs@glot@xns{Kanashi}
\def\langnames@langs@glot@kbl{Kanembu}
\def\langnames@langs@glot@ikt{Kangiryuarmiut}
\def\langnames@langs@glot@kjb{Kanjobal (Eastern)}
\def\langnames@langs@glot@knj{Kanjobal (Western)}
\def\langnames@langs@glot@kne{Kankanay}
\def\langnames@langs@glot@kan{Kannada}
\def\langnames@langs@glot@kxo{Kanoê}
\def\langnames@langs@glot@khd{Kanum (Bädi)}
\def\langnames@langs@glot@kcd{Kanum (Ngkâlmpw)}
\def\langnames@langs@glot@knc{Kanuri}
\def\langnames@langs@glot@kny{Kanyok}
\def\langnames@langs@glot@pam{Kapampangan}
\def\langnames@langs@glot@kpg{Kapingamarangi}
\def\langnames@langs@glot@kah{Kara (in Central African Republic)}
\def\langnames@langs@glot@leu{Kara (in Papua New Guinea)}
\def\langnames@langs@glot@krc{Karachay-Balkar}
\def\langnames@langs@glot@gbd{Karadjeri}
\def\langnames@langs@glot@kdr{Karaim}
\def\langnames@langs@glot@kpj{Karajá}
\def\langnames@langs@glot@kaa{Karakalpak}
\def\langnames@langs@glot@zkk{Karankawa}
\def\langnames@langs@glot@kyj{Karao}
\def\langnames@langs@glot@kpt{Karata}
\def\langnames@langs@glot@krl{Karelian}
\def\langnames@langs@glot@bwe{Karen (Bwe)}
\def\langnames@langs@glot@kjp{Karen (Pwo)}
\def\langnames@langs@glot@ksw{Karen (Sgaw)}
\def\langnames@langs@glot@vka{Kariera}
\def\langnames@langs@glot@kdj{Karimojong}
\def\langnames@langs@glot@ktn{Karitiâna}
\def\langnames@langs@glot@yuj{Karkar-Yuri}
\def\langnames@langs@glot@kyh{Karok}
\def\langnames@langs@glot@arr{Karó (Arára)}
\def\langnames@langs@glot@xsm{Kasem}
\def\langnames@langs@glot@kju{Kashaya}
\def\langnames@langs@glot@kas{Kashmiri}
\def\langnames@langs@glot@csb{Kashubian}
\def\langnames@langs@glot@cog{Kasong}
\def\langnames@langs@glot@bqy{Kata Kolok}
\def\langnames@langs@glot@xtc{Katcha}
\def\langnames@langs@glot@bsh{Kati (in Afghanistan)}
\def\langnames@langs@glot@kts{Kati (in West Papua, Indonesia)}
\def\langnames@langs@glot@kcr{Katla}
\def\langnames@langs@glot@ktw{Kato}
\def\langnames@langs@glot@pss{Kaulong}
\def\langnames@langs@glot@bpp{Kaure}
\def\langnames@langs@glot@zku{Kaurna}
\def\langnames@langs@glot@xaw{Kawaiisu}
\def\langnames@langs@glot@kyz{Kayabí}
\def\langnames@langs@glot@eky{Kayah Li (Eastern)}
\def\langnames@langs@glot@kys{Kayan (Baram)}
\def\langnames@langs@glot@txu{Kayapó}
\def\langnames@langs@glot@gyd{Kayardild}
\def\langnames@langs@glot@gbb{Kaytej}
\def\langnames@langs@glot@kaz{Kazakh}
\def\langnames@langs@glot@ksx{Kedang}
\def\langnames@langs@glot@kbr{Kefa}
\def\langnames@langs@glot@kei{Kei}
\def\langnames@langs@glot@kcl{Kela (Apoze)}
\def\langnames@langs@glot@kzi{Kelabit}
\def\langnames@langs@glot@sbc{Kele}
\def\langnames@langs@glot@ahg{Kemant}
\def\langnames@langs@glot@kmt{Kemtuik}
\def\langnames@langs@glot@kyq{Kenga}
\def\langnames@langs@glot@keu{Kenyah (Uma' Lung)}
\def\langnames@langs@glot@xki{Kenyan Sign Language}
\def\langnames@langs@glot@ken{Kenyang}
\def\langnames@langs@glot@xxk{Keo}
\def\langnames@langs@glot@ker{Kera}
\def\langnames@langs@glot@krk{Kerek}
\def\langnames@langs@glot@kee{Keresan (Santa Ana)}
\def\langnames@langs@glot@ket{Ket}
\def\langnames@langs@glot@xdy{Ketapang}
\def\langnames@langs@glot@kcv{Kete}
\def\langnames@langs@glot@xte{Ketengban}
\def\langnames@langs@glot@kew{Kewa}
\def\langnames@langs@glot@kjh{Khakas}
\def\langnames@langs@glot@klj{Khalaj}
\def\langnames@langs@glot@klr{Khaling}
\def\langnames@langs@glot@khk{Khalkha}
\def\langnames@langs@glot@kjl{Kham}
\def\langnames@langs@glot@khg{Kham (Dege)}
\def\langnames@langs@glot@kca{Khanty}
\def\langnames@langs@glot@khr{Kharia}
\def\langnames@langs@glot@kha{Khasi}
\def\langnames@langs@glot@kjj{Khinalug}
\def\langnames@langs@glot@khm{Khmer}
\def\langnames@langs@glot@kjg{Khmu'}
\def\langnames@langs@glot@khw{Khowar}
\def\langnames@langs@glot@cnk{Khumi}
\def\langnames@langs@glot@khv{Khwarshi}
\def\langnames@langs@glot@kkh{Khün}
\def\langnames@langs@glot@kic{Kickapoo}
\def\langnames@langs@glot@kik{Kikuyu}
\def\langnames@langs@glot@hbb{Kilba}
\def\langnames@langs@glot@kij{Kilivila}
\def\langnames@langs@glot@klb{Kiliwa}
\def\langnames@langs@glot@lub{Kiluba}
\def\langnames@langs@glot@kig{Kimaghama}
\def\langnames@langs@glot@zga{Kinga}
\def\langnames@langs@glot@kfk{Kinnauri}
\def\langnames@langs@glot@kin{Kinyarwanda}
\def\langnames@langs@glot@kio{Kiowa}
\def\langnames@langs@glot@kzw{Kipea}
\def\langnames@langs@glot@geb{Kire}
\def\langnames@langs@glot@kir{Kirghiz}
\def\langnames@langs@glot@gil{Kiribati}
\def\langnames@langs@glot@kiy{Kirikiri}
\def\langnames@langs@glot@cme{Kirma}
\def\langnames@langs@glot@kje{Kisar}
\def\langnames@langs@glot@kss{Kisi}
\def\langnames@langs@glot@gia{Kitja}
\def\langnames@langs@glot@kii{Kitsai}
\def\langnames@langs@glot@ktu{Kituba}
\def\langnames@langs@glot@kjd{Kiwai (Southern)}
\def\langnames@langs@glot@kla{Klamath}
\def\langnames@langs@glot@klu{Klao}
\def\langnames@langs@glot@yak{Klikitat}
\def\langnames@langs@glot@kst{Ko (Winye)}
\def\langnames@langs@glot@cku{Koasati}
\def\langnames@langs@glot@kpw{Kobon}
\def\langnames@langs@glot@kfa{Kodava}
\def\langnames@langs@glot@xwg{Koegu}
\def\langnames@langs@glot@xuo{Koh}
\def\langnames@langs@glot@bcs{Kohumono}
\def\langnames@langs@glot@kpx{Koiali (Mountain)}
\def\langnames@langs@glot@kbk{Koiari}
\def\langnames@langs@glot@kqi{Koita}
\def\langnames@langs@glot@trp{Kokborok}
\def\langnames@langs@glot@kex{Kokni}
\def\langnames@langs@glot@kkk{Kokota}
\def\langnames@langs@glot@kvv{Kola}
\def\langnames@langs@glot@kfb{Kolami}
\def\langnames@langs@glot@kvw{Kolana}
\def\langnames@langs@glot@shm{Koluri}
\def\langnames@langs@glot@bkm{Kom}
\def\langnames@langs@glot@xbi{Kombio}
\def\langnames@langs@glot@kge{Komering}
\def\langnames@langs@glot@koi{Komi-Permyak}
\def\langnames@langs@glot@xom{Komo}
\def\langnames@langs@glot@kfc{Konda}
\def\langnames@langs@glot@kng{Kongo}
\def\langnames@langs@glot@kjc{Konjo}
\def\langnames@langs@glot@knn{Konkani}
\def\langnames@langs@glot@xon{Konkomba}
\def\langnames@langs@glot@mjd{Konkow}
\def\langnames@langs@glot@kma{Konni}
\def\langnames@langs@glot@kyx{Konua}
\def\langnames@langs@glot@cou{Konyagi}
\def\langnames@langs@glot@kqy{Koorete}
\def\langnames@langs@glot@kpr{Korafe}
\def\langnames@langs@glot@kqz{Korana}
\def\langnames@langs@glot@knk{Koranko}
\def\langnames@langs@glot@kor{Korean}
\def\langnames@langs@glot@coe{Koreguaje}
\def\langnames@langs@glot@kfq{Korku}
\def\langnames@langs@glot@kfz{Koromfe}
\def\langnames@langs@glot@khe{Korowai}
\def\langnames@langs@glot@kpy{Koryak}
\def\langnames@langs@glot@kia{Kosop}
\def\langnames@langs@glot@kos{Kosraean}
\def\langnames@langs@glot@kfe{Kota}
\def\langnames@langs@glot@aal{Kotoko}
\def\langnames@langs@glot@kff{Koya}
\def\langnames@langs@glot@khq{Koyra Chiini}
\def\langnames@langs@glot@ses{Koyraboro Senni}
\def\langnames@langs@glot@koy{Koyukon}
\def\langnames@langs@glot@kpk{Kpan}
\def\langnames@langs@glot@xpe{Kpelle}
\def\langnames@langs@glot@kpo{Kposo}
\def\langnames@langs@glot@xra{Krahô}
\def\langnames@langs@glot@kqq{Krenak}
\def\langnames@langs@glot@krs{Kresh}
\def\langnames@langs@glot@rop{Kriol (Ngukurr)}
\def\langnames@langs@glot@kgo{Krongo}
\def\langnames@langs@glot@jct{Krymchak}
\def\langnames@langs@glot@kry{Kryz}
\def\langnames@langs@glot@puo{Ksingmul}
\def\langnames@langs@glot@sdm{Kualan}
\def\langnames@langs@glot@uwa{Kugu Nganhcara}
\def\langnames@langs@glot@kxu{Kui (in India)}
\def\langnames@langs@glot@kvd{Kui (in Indonesia)}
\def\langnames@langs@glot@kui{Kuikúro}
\def\langnames@langs@glot@gvn{Kuku-Yalanji}
\def\langnames@langs@glot@mbt{Kulamanen}
\def\langnames@langs@glot@dwr{Kullo}
\def\langnames@langs@glot@kle{Kulung}
\def\langnames@langs@glot@kue{Kuman}
\def\langnames@langs@glot@kfy{Kumauni}
\def\langnames@langs@glot@kum{Kumyk}
\def\langnames@langs@glot@kvn{Kuna}
\def\langnames@langs@glot@kun{Kunama}
\def\langnames@langs@glot@kup{Kunimaipa}
\def\langnames@langs@glot@kjn{Kunjen}
\def\langnames@langs@glot@cmn{Kunming}
\def\langnames@langs@glot@kto{Kuot}
\def\langnames@langs@glot@ckb{Kurdish (Central)}
\def\langnames@langs@glot@kmr{Kurmanji}
\def\langnames@langs@glot@kru{Kurukh}
\def\langnames@langs@glot@kgg{Kusunda}
\def\langnames@langs@glot@vkt{Kutai}
\def\langnames@langs@glot@gwi{Kutchin}
\def\langnames@langs@glot@kut{Kutenai}
\def\langnames@langs@glot@thd{Kuuk Thaayorre}
\def\langnames@langs@glot@kuy{Kuuku Ya'u}
\def\langnames@langs@glot@kxv{Kuvi}
\def\langnames@langs@glot@kwd{Kwaio}
\def\langnames@langs@glot@kwk{Kwakw'ala}
\def\langnames@langs@glot@tnk{Kwamera}
\def\langnames@langs@glot@ksq{Kwami}
\def\langnames@langs@glot@kwn{Kwangali}
\def\langnames@langs@glot@xwa{Kwaza}
\def\langnames@langs@glot@kwe{Kwerba}
\def\langnames@langs@glot@kmo{Kwoma}
\def\langnames@langs@glot@kwo{Kwomtari}
\def\langnames@langs@glot@xuu{Kxoe}
\def\langnames@langs@glot@kyc{Kyaka}
\def\langnames@langs@glot@kgy{Kyirong}
\def\langnames@langs@glot@nuk{Kyuquot}
\def\langnames@langs@glot@kmg{Kâte}
\def\langnames@langs@glot@gdm{Laal}
\def\langnames@langs@glot@lbu{Labu}
\def\langnames@langs@glot@lac{Lacandón}
\def\langnames@langs@glot@lbt{Lachi}
\def\langnames@langs@glot@lbj{Ladakhi}
\def\langnames@langs@glot@lld{Ladin}
\def\langnames@langs@glot@lad{Ladino}
\def\langnames@langs@glot@laf{Lafofa}
\def\langnames@langs@glot@kot{Lagwan}
\def\langnames@langs@glot@lha{Laha}
\def\langnames@langs@glot@lhu{Lahu}
\def\langnames@langs@glot@cnh{Lai}
\def\langnames@langs@glot@lbe{Lak}
\def\langnames@langs@glot@lkt{Lakhota}
\def\langnames@langs@glot@lbc{Lakkia}
\def\langnames@langs@glot@ywt{Lalo}
\def\langnames@langs@glot@slp{Lamaholot}
\def\langnames@langs@glot@hia{Lamang}
\def\langnames@langs@glot@lmn{Lamani}
\def\langnames@langs@glot@lam{Lamba}
\def\langnames@langs@glot@lmu{Lamen}
\def\langnames@langs@glot@lns{Lamnso'}
\def\langnames@langs@glot@ljp{Lampung}
\def\langnames@langs@glot@lby{Lamu-Lamu}
\def\langnames@langs@glot@lme{Lamé}
\def\langnames@langs@glot@lag{Langi}
\def\langnames@langs@glot@laj{Lango}
\def\langnames@langs@glot@fsl{Langue des Signes Française}
\def\langnames@langs@glot@fcs{Langue des Signes Québecoise}
\def\langnames@langs@glot@lao{Lao}
\def\langnames@langs@glot@lrg{Laragia}
\def\langnames@langs@glot@lbz{Lardil}
\def\langnames@langs@glot@alo{Larike}
\def\langnames@langs@glot@lav{Latvian}
\def\langnames@langs@glot@llu{Lau}
\def\langnames@langs@glot@law{Lauje}
\def\langnames@langs@glot@lvk{Lavukaleve}
\def\langnames@langs@glot@lzz{Laz}
\def\langnames@langs@glot@agh{Lebeo}
\def\langnames@langs@glot@lea{Lega}
\def\langnames@langs@glot@agb{Leggbó}
\def\langnames@langs@glot@lec{Leko}
\def\langnames@langs@glot@lln{Lele}
\def\langnames@langs@glot@lef{Lelemi}
\def\langnames@langs@glot@tnl{Lenakel}
\def\langnames@langs@glot@led{Lendu}
\def\langnames@langs@glot@enx{Lengua}
\def\langnames@langs@glot@aed{Lengua de Señas Argentina}
\def\langnames@langs@glot@ssp{Lengua de Señas Española}
\def\langnames@langs@glot@lep{Lepcha}
\def\langnames@langs@glot@les{Lese}
\def\langnames@langs@glot@lti{Leti}
\def\langnames@langs@glot@lww{Lewo}
\def\langnames@langs@glot@lez{Lezgian}
\def\langnames@langs@glot@lhm{Lhomi}
\def\langnames@langs@glot@lil{Lillooet}
\def\langnames@langs@glot@lif{Limbu}
\def\langnames@langs@glot@lmc{Limilngan}
\def\langnames@langs@glot@liy{Linda}
\def\langnames@langs@glot@lin{Lingala}
\def\langnames@langs@glot@ise{Lingua Italiana dei Segni}
\def\langnames@langs@glot@lnj{Linngithig}
\def\langnames@langs@glot@lis{Lisu}
\def\langnames@langs@glot@lit{Lithuanian}
\def\langnames@langs@glot@liv{Liv}
\def\langnames@langs@glot@lob{Lobi}
\def\langnames@langs@glot@log{Logoti}
\def\langnames@langs@glot@lok{Loko}
\def\langnames@langs@glot@arw{Lokono}
\def\langnames@langs@glot@lom{Loma}
\def\langnames@langs@glot@bdu{Londo}
\def\langnames@langs@glot@lgu{Longgu}
\def\langnames@langs@glot@los{Loniu}
\def\langnames@langs@glot@crc{Lonwolwol}
\def\langnames@langs@glot@njh{Lotha}
\def\langnames@langs@glot@loj{Lou}
\def\langnames@langs@glot@lbo{Loven}
\def\langnames@langs@glot@nds{Low German}
\def\langnames@langs@glot@loz{Lozi}
\def\langnames@langs@glot@nie{Lua}
\def\langnames@langs@glot@ojv{Luangiua}
\def\langnames@langs@glot@lch{Lucazi}
\def\langnames@langs@glot@lug{Luganda}
\def\langnames@langs@glot@lgg{Lugbara}
\def\langnames@langs@glot@jos{Lughat al-Isharat al-Lubnaniya}
\def\langnames@langs@glot@lui{Luiseño}
\def\langnames@langs@glot@ule{Lule}
\def\langnames@langs@glot@str{Lummi}
\def\langnames@langs@glot@lnd{Lun Dayeh}
\def\langnames@langs@glot@lun{Lunda}
\def\langnames@langs@glot@luo{Luo}
\def\langnames@langs@glot@lrc{Luri}
\def\langnames@langs@glot@lut{Lushootseed}
\def\langnames@langs@glot@khl{Lusi}
\def\langnames@langs@glot@lue{Luvale}
\def\langnames@langs@glot@lwo{Luwo}
\def\langnames@langs@glot@ltz{Luxemburgeois}
\def\langnames@langs@glot@luy{Luyia}
\def\langnames@langs@glot@lee{Lyele}
\def\langnames@langs@glot@psr{Língua Gestual Portuguesa}
\def\langnames@langs@glot@bzs{Língua de Sinais Brasileira}
\def\langnames@langs@glot@khb{Lü}
\def\langnames@langs@glot@msj{Ma}
\def\langnames@langs@glot@mhy{Ma'anyan}
\def\langnames@langs@glot@mhi{Ma'di}
\def\langnames@langs@glot@slz{Ma'ya}
\def\langnames@langs@glot@mdy{Maale}
\def\langnames@langs@glot@mas{Maasai}
\def\langnames@langs@glot@mde{Maba}
\def\langnames@langs@glot@mca{Maca}
\def\langnames@langs@glot@mbn{Macaguán}
\def\langnames@langs@glot@mkd{Macedonian}
\def\langnames@langs@glot@mcb{Machiguenga}
\def\langnames@langs@glot@myy{Macuna}
\def\langnames@langs@glot@mbc{Macushi}
\def\langnames@langs@glot@mxu{Mada (in Cameroon)}
\def\langnames@langs@glot@mda{Mada (in Nigeria)}
\def\langnames@langs@glot@dmd{Madimadi}
\def\langnames@langs@glot@mad{Madurese}
\def\langnames@langs@glot@mmw{Mae}
\def\langnames@langs@glot@mag{Magahi}
\def\langnames@langs@glot@mgp{Magar}
\def\langnames@langs@glot@mrd{Magar (Syangja)}
\def\langnames@langs@glot@mgu{Magi}
\def\langnames@langs@glot@mdh{Magindanao}
\def\langnames@langs@glot@mhe{Mah Meri}
\def\langnames@langs@glot@xpq{Mahican}
\def\langnames@langs@glot@nmu{Maidu (Northeast)}
\def\langnames@langs@glot@zrs{Mairasi}
\def\langnames@langs@glot@mbq{Maisin}
\def\langnames@langs@glot@mai{Maithili}
\def\langnames@langs@glot@mpe{Majang}
\def\langnames@langs@glot@mcp{Makaa}
\def\langnames@langs@glot@myh{Makah}
\def\langnames@langs@glot@mkz{Makasae}
\def\langnames@langs@glot@mak{Makassar}
\def\langnames@langs@glot@mgf{Maklew}
\def\langnames@langs@glot@kde{Makonde}
\def\langnames@langs@glot@mgh{Makua}
\def\langnames@langs@glot@mcm{Malacca Creole}
\def\langnames@langs@glot@plt{Malagasy}
\def\langnames@langs@glot@mpb{Malakmalak}
\def\langnames@langs@glot@zsm{Malay}
\def\langnames@langs@glot@zlm{Malay (Kuala Lumpur)}
\def\langnames@langs@glot@zmi{Malay (Ulu Muar)}
\def\langnames@langs@glot@mal{Malayalam}
\def\langnames@langs@glot@mgl{Maleu}
\def\langnames@langs@glot@gcc{Mali}
\def\langnames@langs@glot@mlt{Maltese}
\def\langnames@langs@glot@kmj{Malto}
\def\langnames@langs@glot@mam{Mam}
\def\langnames@langs@glot@mmn{Mamanwa}
\def\langnames@langs@glot@mqj{Mamasa}
\def\langnames@langs@glot@mcs{Mambai}
\def\langnames@langs@glot@mgr{Mambwe}
\def\langnames@langs@glot@maw{Mampruli}
\def\langnames@langs@glot@mdi{Mamvu}
\def\langnames@langs@glot@xmm{Manadonese}
\def\langnames@langs@glot@mva{Manam}
\def\langnames@langs@glot@mle{Manambu}
\def\langnames@langs@glot@nmm{Manange}
\def\langnames@langs@glot@mnc{Manchu}
\def\langnames@langs@glot@mid{Mandaic (Modern)}
\def\langnames@langs@glot@mhq{Mandan}
\def\langnames@langs@glot@mdr{Mandar}
\def\langnames@langs@glot@mnk{Mandinka}
\def\langnames@langs@glot@jet{Manem}
\def\langnames@langs@glot@mna{Mangap-Mbula}
\def\langnames@langs@glot@mpc{Mangarrayi}
\def\langnames@langs@glot@mdj{Mangbetu}
\def\langnames@langs@glot@mqy{Manggarai}
\def\langnames@langs@glot@mjg{Mangghuer}
\def\langnames@langs@glot@mge{Mango}
\def\langnames@langs@glot@emk{Maninka}
\def\langnames@langs@glot@mlq{Maninka (Western)}
\def\langnames@langs@glot@mfv{Manjaku}
\def\langnames@langs@glot@knf{Mankanya}
\def\langnames@langs@glot@nge{Mankon}
\def\langnames@langs@glot@mev{Mano}
\def\langnames@langs@glot@mbb{Manobo (Western Bukidnon)}
\def\langnames@langs@glot@mns{Mansi}
\def\langnames@langs@glot@glv{Manx}
\def\langnames@langs@glot@mri{Maori}
\def\langnames@langs@glot@mcg{Mapoyo}
\def\langnames@langs@glot@arn{Mapudungun}
\def\langnames@langs@glot@mec{Mara}
\def\langnames@langs@glot@mrw{Maranao}
\def\langnames@langs@glot@zmr{Maranungku}
\def\langnames@langs@glot@mar{Marathi}
\def\langnames@langs@glot@rnp{Marchha}
\def\langnames@langs@glot@zmc{Margany}
\def\langnames@langs@glot@mrt{Margi}
\def\langnames@langs@glot@mrj{Mari (Hill)}
\def\langnames@langs@glot@mhr{Mari (Meadow)}
\def\langnames@langs@glot@mrc{Maricopa}
\def\langnames@langs@glot@mrz{Marind}
\def\langnames@langs@glot@mbw{Maring}
\def\langnames@langs@glot@zmt{Maringarr}
\def\langnames@langs@glot@mfr{Marrithiyel}
\def\langnames@langs@glot@mah{Marshallese}
\def\langnames@langs@glot@gcf{Martinique Creole}
\def\langnames@langs@glot@vma{Martuthunira}
\def\langnames@langs@glot@mhx{Maru}
\def\langnames@langs@glot@mcn{Masa}
\def\langnames@langs@glot@jle{Masakin}
\def\langnames@langs@glot@mls{Masalit}
\def\langnames@langs@glot@wam{Massachusett}
\def\langnames@langs@glot@mpq{Matis}
\def\langnames@langs@glot@zml{Matngele}
\def\langnames@langs@glot@mcf{Matsés}
\def\langnames@langs@glot@mvb{Mattole}
\def\langnames@langs@glot@mjk{Matukar}
\def\langnames@langs@glot@mgw{Matuumbi}
\def\langnames@langs@glot@mxx{Mauka}
\def\langnames@langs@glot@mph{Maung}
\def\langnames@langs@glot@mfe{Mauritian Creole}
\def\langnames@langs@glot@mke{Mawchi}
\def\langnames@langs@glot@mbl{Maxakalí}
\def\langnames@langs@glot@yan{Mayangna}
\def\langnames@langs@glot@ayz{Maybrat}
\def\langnames@langs@glot@xyj{Mayi-Yapi}
\def\langnames@langs@glot@mfy{Mayo}
\def\langnames@langs@glot@mdm{Mayogo}
\def\langnames@langs@glot@maz{Mazahua}
\def\langnames@langs@glot@mzn{Mazanderani}
\def\langnames@langs@glot@maq{Mazatec (Chiquihuitlán)}
\def\langnames@langs@glot@mau{Mazatec (Huautla)}
\def\langnames@langs@glot@mfc{Mba}
\def\langnames@langs@glot@vmb{Mbabaram}
\def\langnames@langs@glot@lnb{Mbalanhu}
\def\langnames@langs@glot@mpk{Mbara}
\def\langnames@langs@glot@myb{Mbay}
\def\langnames@langs@glot@mtk{Mbe'}
\def\langnames@langs@glot@mdt{Mbere}
\def\langnames@langs@glot@baw{Mbili}
\def\langnames@langs@glot@gmm{Mbodomo}
\def\langnames@langs@glot@mdq{Mbole}
\def\langnames@langs@glot@mdw{Mbosi}
\def\langnames@langs@glot@mhd{Mbugu}
\def\langnames@langs@glot@mdd{Mbum}
\def\langnames@langs@glot@mym{Me'en}
\def\langnames@langs@glot@nux{Mehek}
\def\langnames@langs@glot@gdq{Mehri}
\def\langnames@langs@glot@mni{Meithei}
\def\langnames@langs@glot@skf{Mekens}
\def\langnames@langs@glot@mek{Mekeo}
\def\langnames@langs@glot@mel{Melanau}
\def\langnames@langs@glot@bew{Melayu Betawi}
\def\langnames@langs@glot@men{Mende}
\def\langnames@langs@glot@mez{Menomini}
\def\langnames@langs@glot@mwv{Mentawai}
\def\langnames@langs@glot@sdo{Mentuh Tapuh}
\def\langnames@langs@glot@mcr{Menya}
\def\langnames@langs@glot@ulk{Meryam Mir}
\def\langnames@langs@glot@mej{Meyah}
\def\langnames@langs@glot@mpt{Mian}
\def\langnames@langs@glot@crg{Michif}
\def\langnames@langs@glot@mic{Micmac}
\def\langnames@langs@glot@mei{Midob}
\def\langnames@langs@glot@ium{Mien}
\def\langnames@langs@glot@mmy{Migama}
\def\langnames@langs@glot@mxj{Miju}
\def\langnames@langs@glot@msy{Mikarew}
\def\langnames@langs@glot@mik{Mikasuki}
\def\langnames@langs@glot@mjw{Mikir}
\def\langnames@langs@glot@hna{Mina}
\def\langnames@langs@glot@min{Minangkabau}
\def\langnames@langs@glot@mvn{Minaveha}
\def\langnames@langs@glot@xmf{Mingrelian}
\def\langnames@langs@glot@mep{Miriwung}
\def\langnames@langs@glot@nju{Mirniny}
\def\langnames@langs@glot@mrg{Mising}
\def\langnames@langs@glot@miq{Miskito}
\def\langnames@langs@glot@zmq{Mituku}
\def\langnames@langs@glot@csi{Miwok (Bodega)}
\def\langnames@langs@glot@csm{Miwok (Central Sierra)}
\def\langnames@langs@glot@lmw{Miwok (Lake)}
\def\langnames@langs@glot@nsq{Miwok (Northern Sierra)}
\def\langnames@langs@glot@pmw{Miwok (Plains)}
\def\langnames@langs@glot@skd{Miwok (Southern Sierra)}
\def\langnames@langs@glot@mxp{Mixe (Ayutla)}
\def\langnames@langs@glot@mco{Mixe (Coatlán)}
\def\langnames@langs@glot@mto{Mixe (Totontepec)}
\def\langnames@langs@glot@mim{Mixtec (Alacatlatzala)}
\def\langnames@langs@glot@mib{Mixtec (Atatlahuca)}
\def\langnames@langs@glot@miy{Mixtec (Ayutla)}
\def\langnames@langs@glot@mih{Mixtec (Chayuco)}
\def\langnames@langs@glot@miz{Mixtec (Coatzospan)}
\def\langnames@langs@glot@mxt{Mixtec (Jamiltepec)}
\def\langnames@langs@glot@mio{Mixtec (Jicaltepec)}
\def\langnames@langs@glot@mig{Mixtec (Molinos)}
\def\langnames@langs@glot@mie{Mixtec (Ocotepec)}
\def\langnames@langs@glot@mil{Mixtec (Peñoles)}
\def\langnames@langs@glot@mjc{Mixtec (San Juan Colorado)}
\def\langnames@langs@glot@mks{Mixtec (Silacayoapan)}
\def\langnames@langs@glot@mpm{Mixtec (Yosondúa)}
\def\langnames@langs@glot@mkf{Miya}
\def\langnames@langs@glot@lus{Mizo}
\def\langnames@langs@glot@mra{Mlabri (Minor)}
\def\langnames@langs@glot@moy{Moca}
\def\langnames@langs@glot@omc{Mochica}
\def\langnames@langs@glot@moc{Mocoví}
\def\langnames@langs@glot@mif{Mofu-Gudur}
\def\langnames@langs@glot@mhj{Moghol}
\def\langnames@langs@glot@moh{Mohawk}
\def\langnames@langs@glot@mov{Mojave}
\def\langnames@langs@glot@mkj{Mokilese}
\def\langnames@langs@glot@moz{Mokilko}
\def\langnames@langs@glot@mbe{Molala}
\def\langnames@langs@glot@mso{Mombum}
\def\langnames@langs@glot@fqs{Momu}
\def\langnames@langs@glot@mqf{Momuna}
\def\langnames@langs@glot@mnw{Mon}
\def\langnames@langs@glot@ndt{Mondunga}
\def\langnames@langs@glot@lol{Mongo}
\def\langnames@langs@glot@mog{Mongondow}
\def\langnames@langs@glot@mnz{Moni}
\def\langnames@langs@glot@mnr{Mono (in United States)}
\def\langnames@langs@glot@mte{Mono-Alu}
\def\langnames@langs@glot@moe{Montagnais}
\def\langnames@langs@glot@mxk{Monumbo}
\def\langnames@langs@glot@mos{Mooré}
\def\langnames@langs@glot@mop{Mopan}
\def\langnames@langs@glot@mhz{Mor}
\def\langnames@langs@glot@mok{Moraori}
\def\langnames@langs@glot@myv{Mordvin (Erzya)}
\def\langnames@langs@glot@mdf{Mordvin (Moksha)}
\def\langnames@langs@glot@mor{Moro}
\def\langnames@langs@glot@mgd{Moru}
\def\langnames@langs@glot@cas{Mosetén}
\def\langnames@langs@glot@meu{Motu}
\def\langnames@langs@glot@siw{Motuna}
\def\langnames@langs@glot@mzp{Movima}
\def\langnames@langs@glot@mye{Mpongwe}
\def\langnames@langs@glot@akc{Mpur}
\def\langnames@langs@glot@dmw{Mudburra}
\def\langnames@langs@glot@aoj{Mufian}
\def\langnames@langs@glot@sgw{Muher}
\def\langnames@langs@glot@bmr{Muinane}
\def\langnames@langs@glot@chb{Muisca}
\def\langnames@langs@glot@mlm{Mulao}
\def\langnames@langs@glot@mzm{Mumuye}
\def\langnames@langs@glot@mji{Mun}
\def\langnames@langs@glot@mnb{Muna}
\def\langnames@langs@glot@mua{Mundang}
\def\langnames@langs@glot@mnf{Mundani}
\def\langnames@langs@glot@myu{Mundurukú}
\def\langnames@langs@glot@mhk{Mungaka}
\def\langnames@langs@glot@umu{Munsee}
\def\langnames@langs@glot@moj{Munzombo}
\def\langnames@langs@glot@mtq{Muong}
\def\langnames@langs@glot@sur{Mupun}
\def\langnames@langs@glot@mtf{Murik}
\def\langnames@langs@glot@mur{Murle}
\def\langnames@langs@glot@mwf{Murrinh-Patha}
\def\langnames@langs@glot@muz{Mursi}
\def\langnames@langs@glot@zmu{Muruwari}
\def\langnames@langs@glot@mug{Musgu}
\def\langnames@langs@glot@msu{Musom}
\def\langnames@langs@glot@hur{Musqueam}
\def\langnames@langs@glot@emi{Mussau}
\def\langnames@langs@glot@css{Mutsun}
\def\langnames@langs@glot@myw{Muyuw}
\def\langnames@langs@glot@mwe{Mwera}
\def\langnames@langs@glot@mlv{Mwotlap}
\def\langnames@langs@glot@xak{Máku}
\def\langnames@langs@glot@bzk{Mískito Coast English Creole}
\def\langnames@langs@glot@muh{Mündü}
\def\langnames@langs@glot@naf{Nabak}
\def\langnames@langs@glot@wyy{Nadroga}
\def\langnames@langs@glot@mbj{Nadëb}
\def\langnames@langs@glot@nfr{Nafaanra}
\def\langnames@langs@glot@nbi{Naga (Mao)}
\def\langnames@langs@glot@nmf{Naga (Tangkhul)}
\def\langnames@langs@glot@nzm{Naga (Zeme)}
\def\langnames@langs@glot@nag{Naga Pidgin}
\def\langnames@langs@glot@nce{Nagatman}
\def\langnames@langs@glot@nll{Nahali}
\def\langnames@langs@glot@nhn{Nahuatl (Central)}
\def\langnames@langs@glot@ncj{Nahuatl (Huauchinango)}
\def\langnames@langs@glot@nhx{Nahuatl (Mecayapan Isthmus)}
\def\langnames@langs@glot@ncl{Nahuatl (Michoacán)}
\def\langnames@langs@glot@nhm{Nahuatl (Milpa Alta)}
\def\langnames@langs@glot@nhp{Nahuatl (Pajapan)}
\def\langnames@langs@glot@xpo{Nahuatl (Pochutla)}
\def\langnames@langs@glot@azz{Nahuatl (Sierra de Zacapoaxtla)}
\def\langnames@langs@glot@nhg{Nahuatl (Tetelcingo)}
\def\langnames@langs@glot@ngu{Nahuatl (Xalitla)}
\def\langnames@langs@glot@bio{Nai}
\def\langnames@langs@glot@nak{Nakanai}
\def\langnames@langs@glot@nck{Nakkara}
\def\langnames@langs@glot@nal{Nalik}
\def\langnames@langs@glot@naq{Nama}
\def\langnames@langs@glot@nmb{Nambas (Big)}
\def\langnames@langs@glot@nab{Nambikuára (Southern)}
\def\langnames@langs@glot@nnm{Namia}
\def\langnames@langs@glot@gld{Nanai}
\def\langnames@langs@glot@ncb{Nancowry}
\def\langnames@langs@glot@nnb{Nande}
\def\langnames@langs@glot@niq{Nandi}
\def\langnames@langs@glot@sen{Nanerge}
\def\langnames@langs@glot@nnk{Nankina}
\def\langnames@langs@glot@nnt{Nanticoke}
\def\langnames@langs@glot@tvl{Nanumea}
\def\langnames@langs@glot@npy{Napu}
\def\langnames@langs@glot@npa{Nar-Phu}
\def\langnames@langs@glot@nrb{Nara (in Ethiopia)}
\def\langnames@langs@glot@nrm{Narom}
\def\langnames@langs@glot@nas{Nasioi}
\def\langnames@langs@glot@nsk{Naskapi}
\def\langnames@langs@glot@ncz{Natchez}
\def\langnames@langs@glot@ntm{Nateni}
\def\langnames@langs@glot@ntu{Natügu}
\def\langnames@langs@glot@nau{Nauruan}
\def\langnames@langs@glot@nav{Navajo}
\def\langnames@langs@glot@nxq{Naxi}
\def\langnames@langs@glot@bud{Ncàm}
\def\langnames@langs@glot@nde{Ndebele}
\def\langnames@langs@glot@djj{Ndjébbana}
\def\langnames@langs@glot@ndz{Ndogo}
\def\langnames@langs@glot@ndo{Ndonga}
\def\langnames@langs@glot@nmd{Ndumu}
\def\langnames@langs@glot@ndv{Ndut}
\def\langnames@langs@glot@djk{Ndyuka}
\def\langnames@langs@glot@dse{Nederlandse Gebarentaal}
\def\langnames@langs@glot@neg{Negidal}
\def\langnames@langs@glot@nsn{Nehan}
\def\langnames@langs@glot@nee{Nelemwa}
\def\langnames@langs@glot@anh{Nend}
\def\langnames@langs@glot@yrk{Nenets}
\def\langnames@langs@glot@nen{Nengone}
\def\langnames@langs@glot@aij{Neo-Aramaic (Arbel Jewish)}
\def\langnames@langs@glot@aii{Neo-Aramaic (Assyrian)}
\def\langnames@langs@glot@trg{Neo-Aramaic (Persian Azerbaijan)}
\def\langnames@langs@glot@npi{Nepali}
\def\langnames@langs@glot@pia{Nevome}
\def\langnames@langs@glot@nzs{New Zealand Sign Language}
\def\langnames@langs@glot@new{Newar (Dolakha)}
\def\langnames@langs@glot@ney{Neyo}
\def\langnames@langs@glot@nez{Nez Perce}
\def\langnames@langs@glot@ntj{Ngaanyatjarra}
\def\langnames@langs@glot@nxg{Ngad'a}
\def\langnames@langs@glot@nig{Ngalakan}
\def\langnames@langs@glot@ngk{Ngalkbun}
\def\langnames@langs@glot@sba{Ngambay}
\def\langnames@langs@glot@nam{Ngan'gityemerri}
\def\langnames@langs@glot@nio{Nganasan}
\def\langnames@langs@glot@nid{Ngandi}
\def\langnames@langs@glot@nay{Ngarinyeri}
\def\langnames@langs@glot@nrk{Ngarla}
\def\langnames@langs@glot@nrl{Ngarluma}
\def\langnames@langs@glot@nxn{Ngawun}
\def\langnames@langs@glot@nbm{Ngbaka (Ma'bo)}
\def\langnames@langs@glot@nga{Ngbaka (Minagende)}
\def\langnames@langs@glot@ngb{Ngbandi}
\def\langnames@langs@glot@niy{Ngiti}
\def\langnames@langs@glot@wyb{Ngiyambaa}
\def\langnames@langs@glot@ngi{Ngizim}
\def\langnames@langs@glot@ngo{Ngoni}
\def\langnames@langs@glot@llp{Nguna}
\def\langnames@langs@glot@gym{Ngäbere}
\def\langnames@langs@glot@nha{Nhanda}
\def\langnames@langs@glot@nhr{Nharo}
\def\langnames@langs@glot@nia{Nias}
\def\langnames@langs@glot@caq{Nicobarese (Car)}
\def\langnames@langs@glot@pcm{Nigerian Pidgin}
\def\langnames@langs@glot@jsl{Nihon Shuwa (Japanese Sign Language)}
\def\langnames@langs@glot@nir{Nimboran}
\def\langnames@langs@glot@niz{Ningil}
\def\langnames@langs@glot@nsz{Nisenan}
\def\langnames@langs@glot@ncg{Nisgha}
\def\langnames@langs@glot@dtd{Nitinaht}
\def\langnames@langs@glot@num{Niuafo'ou}
\def\langnames@langs@glot@niu{Niuean}
\def\langnames@langs@glot@cag{Nivacle}
\def\langnames@langs@glot@niv{Nivkh}
\def\langnames@langs@glot@isi{Nkem}
\def\langnames@langs@glot@nko{Nkonya}
\def\langnames@langs@glot@cgg{Nkore-Kiga}
\def\langnames@langs@glot@fia{Nobiin}
\def\langnames@langs@glot@njb{Nocte}
\def\langnames@langs@glot@nog{Noghay}
\def\langnames@langs@glot@not{Nomatsiguenga}
\def\langnames@langs@glot@nhu{Noni}
\def\langnames@langs@glot@snf{Noon}
\def\langnames@langs@glot@nsl{Norsk Tegnspråk}
\def\langnames@langs@glot@nor{Norwegian}
\def\langnames@langs@glot@nse{Nsenga}
\def\langnames@langs@glot@nto{Ntomba}
\def\langnames@langs@glot@nxl{Nuaulu}
\def\langnames@langs@glot@kcn{Nubi}
\def\langnames@langs@glot@dgl{Nubian (Dongolese)}
\def\langnames@langs@glot@xnz{Nubian (Kunuz)}
\def\langnames@langs@glot@nus{Nuer}
\def\langnames@langs@glot@mbr{Nukak}
\def\langnames@langs@glot@nkr{Nukuoro}
\def\langnames@langs@glot@nut{Nung (in Vietnam)}
\def\langnames@langs@glot@nuy{Nunggubuyu}
\def\langnames@langs@glot@nuv{Nuni (Northern)}
\def\langnames@langs@glot@iii{Nuosu}
\def\langnames@langs@glot@nup{Nupe}
\def\langnames@langs@glot@nuf{Nusu}
\def\langnames@langs@glot@cbn{Nyah Kur (Tha Pong)}
\def\langnames@langs@glot@nly{Nyamal}
\def\langnames@langs@glot@now{Nyambo}
\def\langnames@langs@glot@tpq{Nyamkad}
\def\langnames@langs@glot@nym{Nyamwezi}
\def\langnames@langs@glot@nyj{Nyanga}
\def\langnames@langs@glot@nyp{Nyangi}
\def\langnames@langs@glot@nna{Nyangumarda}
\def\langnames@langs@glot@nyt{Nyawaygi}
\def\langnames@langs@glot@yly{Nyelâyu}
\def\langnames@langs@glot@nyh{Nyigina}
\def\langnames@langs@glot@nih{Nyiha}
\def\langnames@langs@glot@nyi{Nyimang}
\def\langnames@langs@glot@njz{Nyishi}
\def\langnames@langs@glot@nyv{Nyulnyul}
\def\langnames@langs@glot@nys{Nyungar}
\def\langnames@langs@glot@nzk{Nzakara}
\def\langnames@langs@glot@ood{O'odham}
\def\langnames@langs@glot@afz{Obokuitai}
\def\langnames@langs@glot@ann{Obolo}
\def\langnames@langs@glot@oca{Ocaina}
\def\langnames@langs@glot@oci{Occitan}
\def\langnames@langs@glot@ocu{Ocuilteco}
\def\langnames@langs@glot@ogb{Ogbia}
\def\langnames@langs@glot@ogu{Ogbronuagum}
\def\langnames@langs@glot@oyb{Oi}
\def\langnames@langs@glot@xal{Oirat}
\def\langnames@langs@glot@ojs{Ojibwa (Severn)}
\def\langnames@langs@glot@ciw{Ojibwe (Minnesota)}
\def\langnames@langs@glot@oka{Okanagan}
\def\langnames@langs@glot@opm{Oksapmin}
\def\langnames@langs@glot@oku{Oku}
\def\langnames@langs@glot@ong{Olo}
\def\langnames@langs@glot@plo{Olutec}
\def\langnames@langs@glot@omg{Omagua}
\def\langnames@langs@glot@oma{Omaha}
\def\langnames@langs@glot@aun{One}
\def\langnames@langs@glot@one{Oneida}
\def\langnames@langs@glot@oon{Onge}
\def\langnames@langs@glot@ons{Ono}
\def\langnames@langs@glot@ono{Onondaga}
\def\langnames@langs@glot@mvf{Ordos}
\def\langnames@langs@glot@ore{Orejón}
\def\langnames@langs@glot@tag{Orig}
\def\langnames@langs@glot@ory{Oriya}
\def\langnames@langs@glot@ort{Oriya (Kotia)}
\def\langnames@langs@glot@oru{Ormuri}
\def\langnames@langs@glot@oac{Oroch}
\def\langnames@langs@glot@oaa{Orok}
\def\langnames@langs@glot@okv{Orokaiva}
\def\langnames@langs@glot@oro{Orokolo}
\def\langnames@langs@glot@gax{Oromo (Boraana)}
\def\langnames@langs@glot@hae{Oromo (Harar)}
\def\langnames@langs@glot@ssn{Oromo (Waata)}
\def\langnames@langs@glot@gaz{Oromo (West-Central)}
\def\langnames@langs@glot@ury{Orya}
\def\langnames@langs@glot@osa{Osage}
\def\langnames@langs@glot@oss{Ossetic}
\def\langnames@langs@glot@iow{Oto}
\def\langnames@langs@glot@otz{Otomí (Ixtenco)}
\def\langnames@langs@glot@ote{Otomí (Mezquital)}
\def\langnames@langs@glot@otq{Otomí (Santiago Mexquititlan)}
\def\langnames@langs@glot@otm{Otomí (Sierra)}
\def\langnames@langs@glot@otr{Otoro}
\def\langnames@langs@glot@owi{Owininga}
\def\langnames@langs@glot@pqa{Pa'a}
\def\langnames@langs@glot@drl{Paakantyi}
\def\langnames@langs@glot@pma{Paamese}
\def\langnames@langs@glot@pac{Pacoh}
\def\langnames@langs@glot@pdo{Padoe}
\def\langnames@langs@glot@pgu{Pagu}
\def\langnames@langs@glot@duf{Paita}
\def\langnames@langs@glot@pck{Paite}
\def\langnames@langs@glot@pao{Paiute (Northern)}
\def\langnames@langs@glot@pwn{Paiwan}
\def\langnames@langs@glot@pkn{Pakanha}
\def\langnames@langs@glot@pau{Palauan}
\def\langnames@langs@glot@pll{Palaung}
\def\langnames@langs@glot@plu{Palikur}
\def\langnames@langs@glot@fap{Palor}
\def\langnames@langs@glot@nad{Palyku}
\def\langnames@langs@glot@pmz{Pame}
\def\langnames@langs@glot@pmf{Pamona}
\def\langnames@langs@glot@pbh{Panare}
\def\langnames@langs@glot@kre{Panará}
\def\langnames@langs@glot@pag{Pangasinan}
\def\langnames@langs@glot@pbr{Pangwa}
\def\langnames@langs@glot@pan{Panjabi}
\def\langnames@langs@glot@pnw{Panyjima}
\def\langnames@langs@glot@pap{Papiamentu}
\def\langnames@langs@glot@prk{Parauk}
\def\langnames@langs@glot@asa{Pare}
\def\langnames@langs@glot@pab{Paresi}
\def\langnames@langs@glot@pci{Parji (Dravidian)}
\def\langnames@langs@glot@pst{Pashto}
\def\langnames@langs@glot@pqm{Passamaquoddy-Maliseet}
\def\langnames@langs@glot@ptp{Patep}
\def\langnames@langs@glot@gfk{Patpatar}
\def\langnames@langs@glot@lae{Pattani}
\def\langnames@langs@glot@pwi{Patwin}
\def\langnames@langs@glot@plh{Paulohi}
\def\langnames@langs@glot@pad{Paumarí}
\def\langnames@langs@glot@pwa{Pawaian}
\def\langnames@langs@glot@paw{Pawnee}
\def\langnames@langs@glot@pay{Pech}
\def\langnames@langs@glot@aoc{Pemon}
\def\langnames@langs@glot@peg{Pengo}
\def\langnames@langs@glot@pip{Pero}
\def\langnames@langs@glot@pes{Persian}
\def\langnames@langs@glot@pww{Phlong}
\def\langnames@langs@glot@pio{Piapoco}
\def\langnames@langs@glot@pid{Piaroa}
\def\langnames@langs@glot@plg{Pilagá}
\def\langnames@langs@glot@piv{Pileni}
\def\langnames@langs@glot@pif{Pingilapese}
\def\langnames@langs@glot@piu{Pintupi}
\def\langnames@langs@glot@ppl{Pipil}
\def\langnames@langs@glot@myp{Pirahã}
\def\langnames@langs@glot@pir{Piratapuyo}
\def\langnames@langs@glot@pib{Piro}
\def\langnames@langs@glot@psa{Pisa}
\def\langnames@langs@glot@pjt{Pitjantjatjara}
\def\langnames@langs@glot@pit{Pitta Pitta}
\def\langnames@langs@glot@psd{Plains-Indians Sign Language}
\def\langnames@langs@glot@gob{Playero}
\def\langnames@langs@glot@fwa{Po-Ai}
\def\langnames@langs@glot@pbi{Podoko}
\def\langnames@langs@glot@poy{Pogoro}
\def\langnames@langs@glot@pon{Pohnpeian}
\def\langnames@langs@glot@rwa{Poko-Rawo}
\def\langnames@langs@glot@poh{Pokomchí}
\def\langnames@langs@glot@pko{Pokot}
\def\langnames@langs@glot@pox{Polabian}
\def\langnames@langs@glot@pol{Polish}
\def\langnames@langs@glot@poo{Pomo (Central)}
\def\langnames@langs@glot@peb{Pomo (Eastern)}
\def\langnames@langs@glot@pej{Pomo (Northern)}
\def\langnames@langs@glot@pom{Pomo (Southeastern)}
\def\langnames@langs@glot@pbe{Popoloca (Metzontla)}
\def\langnames@langs@glot@poe{Popoloca (San Juan Atzingo)}
\def\langnames@langs@glot@pbf{Popoloca (San Vicente Coyotepec)}
\def\langnames@langs@glot@poi{Popoluca (Sierra)}
\def\langnames@langs@glot@poc{Poqomam}
\def\langnames@langs@glot@psw{Port Sandwich}
\def\langnames@langs@glot@por{Portuguese}
\def\langnames@langs@glot@pot{Potawatomi}
\def\langnames@langs@glot@pim{Powhatan}
\def\langnames@langs@glot@prn{Prasuni}
\def\langnames@langs@glot@pre{Príncipense}
\def\langnames@langs@glot@pui{Puinave}
\def\langnames@langs@glot@fuc{Pulaar}
\def\langnames@langs@glot@nij{Pulopetak}
\def\langnames@langs@glot@puw{Puluwat}
\def\langnames@langs@glot@pmi{Pumi}
\def\langnames@langs@glot@puq{Puquina}
\def\langnames@langs@glot@prx{Purki}
\def\langnames@langs@glot@tsz{Purépecha}
\def\langnames@langs@glot@pbb{Páez}
\def\langnames@langs@glot@lkr{Päri}
\def\langnames@langs@glot@aar{Qafar}
\def\langnames@langs@glot@byx{Qaqet}
\def\langnames@langs@glot@alc{Qawasqar}
\def\langnames@langs@glot@yum{Quechan}
\def\langnames@langs@glot@qxa{Quechua (Ancash)}
\def\langnames@langs@glot@quy{Quechua (Ayacucho)}
\def\langnames@langs@glot@qvc{Quechua (Cajamarca)}
\def\langnames@langs@glot@quh{Quechua (Cochabamba)}
\def\langnames@langs@glot@quz{Quechua (Cuzco)}
\def\langnames@langs@glot@qug{Quechua (Ecuadorean)}
\def\langnames@langs@glot@qub{Quechua (Huallaga)}
\def\langnames@langs@glot@qvi{Quechua (Imbabura)}
\def\langnames@langs@glot@qvn{Quechua (Tarma)}
\def\langnames@langs@glot@quc{Quiché}
\def\langnames@langs@glot@qui{Quileute}
\def\langnames@langs@glot@rad{Rade}
\def\langnames@langs@glot@lml{Raga}
\def\langnames@langs@glot@rji{Raji}
\def\langnames@langs@glot@ral{Ralte}
\def\langnames@langs@glot@rma{Rama}
\def\langnames@langs@glot@bod{Rang Pas}
\def\langnames@langs@glot@rao{Rao}
\def\langnames@langs@glot@rap{Rapanui}
\def\langnames@langs@glot@ras{Rashad}
\def\langnames@langs@glot@rwo{Rawa}
\def\langnames@langs@glot@raw{Rawang}
\def\langnames@langs@glot@rej{Rejang}
\def\langnames@langs@glot@rmb{Rembarnga}
\def\langnames@langs@glot@bfw{Remo}
\def\langnames@langs@glot@rel{Rendille}
\def\langnames@langs@glot@ren{Rengao}
\def\langnames@langs@glot@mnv{Rennellese}
\def\langnames@langs@glot@rgr{Resígaro}
\def\langnames@langs@glot@tnc{Retuarã}
\def\langnames@langs@glot@ran{Riantana}
\def\langnames@langs@glot@rkb{Rikbaktsa}
\def\langnames@langs@glot@rim{Rimi}
\def\langnames@langs@glot@rit{Ritharngu}
\def\langnames@langs@glot@rog{Roglai (Northern)}
\def\langnames@langs@glot@rmn{Romani (Bugurdzi)}
\def\langnames@langs@glot@rmo{Romani (Burgenland)}
\def\langnames@langs@glot@rmy{Romani (Lovari)}
\def\langnames@langs@glot@rml{Romani (North Russian)}
\def\langnames@langs@glot@rmw{Romani (Welsh)}
\def\langnames@langs@glot@ron{Romanian}
\def\langnames@langs@glot@roh{Romansch}
\def\langnames@langs@glot@cla{Ron}
\def\langnames@langs@glot@rng{Ronga}
\def\langnames@langs@glot@rro{Roro}
\def\langnames@langs@glot@twu{Roti}
\def\langnames@langs@glot@roo{Rotokas}
\def\langnames@langs@glot@rtm{Rotuman}
\def\langnames@langs@glot@rug{Roviana}
\def\langnames@langs@glot@dru{Rukai (Tanan)}
\def\langnames@langs@glot@klq{Rumu}
\def\langnames@langs@glot@run{Rundi}
\def\langnames@langs@glot@rou{Runga}
\def\langnames@langs@glot@nyn{Runyankore}
\def\langnames@langs@glot@nyo{Runyoro-Rutooro}
\def\langnames@langs@glot@rus{Russian}
\def\langnames@langs@glot@rsl{Russian Sign Language}
\def\langnames@langs@glot@rut{Rutul}
\def\langnames@langs@glot@apb{Sa'a}
\def\langnames@langs@glot@snv{Sa'ban}
\def\langnames@langs@glot@sma{Saami (Central-South)}
\def\langnames@langs@glot@sjd{Saami (Kildin)}
\def\langnames@langs@glot@sme{Saami (Northern)}
\def\langnames@langs@glot@skb{Saek}
\def\langnames@langs@glot@uma{Sahaptin (Umatilla)}
\def\langnames@langs@glot@ssy{Saho}
\def\langnames@langs@glot@saj{Sahu}
\def\langnames@langs@glot@sku{Sakao}
\def\langnames@langs@glot@slr{Salar}
\def\langnames@langs@glot@sbe{Saliba (in Papua New Guinea)}
\def\langnames@langs@glot@sln{Salinan}
\def\langnames@langs@glot@slh{Salish (Southern Puget Sound)}
\def\langnames@langs@glot@sll{Salt-Yui}
\def\langnames@langs@glot@sse{Sama (Balangingi)}
\def\langnames@langs@glot@ssb{Sama (Southern)}
\def\langnames@langs@glot@ndi{Samba Leko}
\def\langnames@langs@glot@smq{Samo}
\def\langnames@langs@glot@smo{Samoan}
\def\langnames@langs@glot@sad{Sandawe}
\def\langnames@langs@glot@sxn{Sangir}
\def\langnames@langs@glot@sag{Sango}
\def\langnames@langs@glot@snq{Sangu}
\def\langnames@langs@glot@sce{Santa}
\def\langnames@langs@glot@sat{Santali}
\def\langnames@langs@glot@xsu{Sanuma}
\def\langnames@langs@glot@spu{Sapuan}
\def\langnames@langs@glot@srm{Saramaccan}
\def\langnames@langs@glot@srs{Sarcee}
\def\langnames@langs@glot@sro{Sardinian}
\def\langnames@langs@glot@dju{Sare}
\def\langnames@langs@glot@ybe{Saryg Yughur}
\def\langnames@langs@glot@sdg{Savi}
\def\langnames@langs@glot@svs{Savosavo}
\def\langnames@langs@glot@szw{Sawai}
\def\langnames@langs@glot@hvn{Sawu}
\def\langnames@langs@glot@pos{Sayula Popoluca}
\def\langnames@langs@glot@kpz{Sebei}
\def\langnames@langs@glot@sey{Secoya}
\def\langnames@langs@glot@sed{Sedang}
\def\langnames@langs@glot@trv{Seediq}
\def\langnames@langs@glot@slu{Selaru}
\def\langnames@langs@glot@sly{Selayar}
\def\langnames@langs@glot@spl{Selepet}
\def\langnames@langs@glot@ona{Selknam}
\def\langnames@langs@glot@sel{Selkup}
\def\langnames@langs@glot@nsm{Sema}
\def\langnames@langs@glot@sea{Semai}
\def\langnames@langs@glot@sif{Seme}
\def\langnames@langs@glot@sza{Semelai}
\def\langnames@langs@glot@seh{Sena}
\def\langnames@langs@glot@sef{Senadi}
\def\langnames@langs@glot@see{Seneca}
\def\langnames@langs@glot@szg{Sengele}
\def\langnames@langs@glot@set{Sentani}
\def\langnames@langs@glot@hbs{Serbian-Croatian}
\def\langnames@langs@glot@sei{Seri}
\def\langnames@langs@glot@ser{Serrano}
\def\langnames@langs@glot@sot{Sesotho}
\def\langnames@langs@glot@crs{Seychelles Creole}
\def\langnames@langs@glot@sbf{Shabo}
\def\langnames@langs@glot@ksb{Shambala}
\def\langnames@langs@glot@shn{Shan}
\def\langnames@langs@glot@mcd{Sharanahua}
\def\langnames@langs@glot@sht{Shasta}
\def\langnames@langs@glot@shj{Shatt}
\def\langnames@langs@glot@sjw{Shawnee}
\def\langnames@langs@glot@swv{Shekhawati}
\def\langnames@langs@glot@sdp{Sherdukpen}
\def\langnames@langs@glot@xsr{Sherpa}
\def\langnames@langs@glot@shk{Shilluk}
\def\langnames@langs@glot@scl{Shina}
\def\langnames@langs@glot@bwo{Shinassha}
\def\langnames@langs@glot@shp{Shipibo-Konibo}
\def\langnames@langs@glot@yuy{Shira Yughur}
\def\langnames@langs@glot@shb{Shiriana}
\def\langnames@langs@glot@sii{Shompen}
\def\langnames@langs@glot@sna{Shona}
\def\langnames@langs@glot@cjs{Shor}
\def\langnames@langs@glot@shh{Shoshone}
\def\langnames@langs@glot@sgh{Shughni}
\def\langnames@langs@glot@ryu{Shuri}
\def\langnames@langs@glot@shs{Shuswap}
\def\langnames@langs@glot@snp{Siane}
\def\langnames@langs@glot@sjr{Siar}
\def\langnames@langs@glot@sid{Sidaama}
\def\langnames@langs@glot@ski{Sika}
\def\langnames@langs@glot@tty{Sikaritai}
\def\langnames@langs@glot@sip{Sikkimese}
\def\langnames@langs@glot@skh{Sikule}
\def\langnames@langs@glot@dau{Sila}
\def\langnames@langs@glot@smr{Simeulue}
\def\langnames@langs@glot@snc{Sinaugoro}
\def\langnames@langs@glot@snd{Sindhi}
\def\langnames@langs@glot@sin{Sinhala}
\def\langnames@langs@glot@xsi{Sio}
\def\langnames@langs@glot@snn{Siona}
\def\langnames@langs@glot@qum{Sipakapense}
\def\langnames@langs@glot@fos{Siraya}
\def\langnames@langs@glot@sri{Siriano}
\def\langnames@langs@glot@srq{Sirionó}
\def\langnames@langs@glot@ssd{Siroi}
\def\langnames@langs@glot@sil{Sisaala}
\def\langnames@langs@glot@baa{Sisiqa}
\def\langnames@langs@glot@sis{Siuslaw}
\def\langnames@langs@glot@skv{Skou}
\def\langnames@langs@glot@den{Slave}
\def\langnames@langs@glot@xsl{Slavey}
\def\langnames@langs@glot@slk{Slovak}
\def\langnames@langs@glot@slv{Slovene}
\def\langnames@langs@glot@teu{So}
\def\langnames@langs@glot@sob{Sobei}
\def\langnames@langs@glot@gru{Soddo}
\def\langnames@langs@glot@evn{Solon}
\def\langnames@langs@glot@som{Somali}
\def\langnames@langs@glot@sop{Songe}
\def\langnames@langs@glot@snk{Soninke}
\def\langnames@langs@glot@sov{Sonsorol-Tobi}
\def\langnames@langs@glot@sqt{Soqotri}
\def\langnames@langs@glot@srb{Sora}
\def\langnames@langs@glot@dsb{Sorbian (Lower)}
\def\langnames@langs@glot@hsb{Sorbian (Upper)}
\def\langnames@langs@glot@nso{Sotho (Northern)}
\def\langnames@langs@glot@mnx{Sougb}
\def\langnames@langs@glot@kvk{South Korean Sign Language}
\def\langnames@langs@glot@tvk{Southeast Ambrym}
\def\langnames@langs@glot@wib{Southern Toussian}
\def\langnames@langs@glot@spa{Spanish}
\def\langnames@langs@glot@spt{Spitian}
\def\langnames@langs@glot@spo{Spokane}
\def\langnames@langs@glot@squ{Squamish}
\def\langnames@langs@glot@srn{Sranan}
\def\langnames@langs@glot@kpm{Sre}
\def\langnames@langs@glot@sto{Stoney}
\def\langnames@langs@glot@sbs{Subiya}
\def\langnames@langs@glot@tgo{Sudest}
\def\langnames@langs@glot@sue{Suena}
\def\langnames@langs@glot@swi{Sui}
\def\langnames@langs@glot@sui{Suki}
\def\langnames@langs@glot@sub{Suku}
\def\langnames@langs@glot@suk{Sukuma}
\def\langnames@langs@glot@sua{Sulka}
\def\langnames@langs@glot@suv{Sulung}
\def\langnames@langs@glot@sun{Sundanese}
\def\langnames@langs@glot@sjg{Sungor}
\def\langnames@langs@glot@spp{Supyire}
\def\langnames@langs@glot@sgz{Sursurunga}
\def\langnames@langs@glot@sus{Susu}
\def\langnames@langs@glot@sva{Svan}
\def\langnames@langs@glot@swl{Svenska Teckenspråket}
\def\langnames@langs@glot@swh{Swahili}
\def\langnames@langs@glot@ssw{Swati}
\def\langnames@langs@glot@swe{Swedish}
\def\langnames@langs@glot@slc{Sáliba (in Colombia)}
\def\langnames@langs@glot@mky{Taba}
\def\langnames@langs@glot@sst{Tabare}
\def\langnames@langs@glot@tby{Tabaru}
\def\langnames@langs@glot@tab{Tabassaran}
\def\langnames@langs@glot@tnm{Tabla}
\def\langnames@langs@glot@tap{Tabwa}
\def\langnames@langs@glot@tna{Tacana}
\def\langnames@langs@glot@tgl{Tagalog}
\def\langnames@langs@glot@tbw{Tagbanwa (Aborlan)}
\def\langnames@langs@glot@tah{Tahitian}
\def\langnames@langs@glot@gpn{Taiap}
\def\langnames@langs@glot@sps{Taiof}
\def\langnames@langs@glot@tbg{Tairora}
\def\langnames@langs@glot@tss{Taiwanese Sign Language (Ziran Shouyu)}
\def\langnames@langs@glot@tgk{Tajik}
\def\langnames@langs@glot@tkm{Takelma}
\def\langnames@langs@glot@tbc{Takia}
\def\langnames@langs@glot@tld{Talaud}
\def\langnames@langs@glot@tlj{Talinga}
\def\langnames@langs@glot@tly{Talysh (Azerbaijan)}
\def\langnames@langs@glot@tma{Tama}
\def\langnames@langs@glot@mla{Tamabo}
\def\langnames@langs@glot@tcg{Tamagario}
\def\langnames@langs@glot@taj{Tamang (Eastern)}
\def\langnames@langs@glot@taq{Tamashek}
\def\langnames@langs@glot@tam{Tamil}
\def\langnames@langs@glot@tpm{Tampulma}
\def\langnames@langs@glot@tcb{Tanacross}
\def\langnames@langs@glot@tfn{Tanaina}
\def\langnames@langs@glot@taa{Tanana (Lower)}
\def\langnames@langs@glot@tan{Tangale}
\def\langnames@langs@glot@skj{Tangbe}
\def\langnames@langs@glot@tgg{Tangga}
\def\langnames@langs@glot@tpg{Tanglapui}
\def\langnames@langs@glot@nwi{Tanna (Southwest)}
\def\langnames@langs@glot@tza{Tanzania Sign Language}
\def\langnames@langs@glot@tpj{Tapieté}
\def\langnames@langs@glot@tar{Tarahumara (Central)}
\def\langnames@langs@glot@tac{Tarahumara (Western)}
\def\langnames@langs@glot@txn{Tarangan (West)}
\def\langnames@langs@glot@tro{Tarao}
\def\langnames@langs@glot@tae{Tariana}
\def\langnames@langs@glot@yer{Tarok}
\def\langnames@langs@glot@shi{Tashlhiyt}
\def\langnames@langs@glot@ttt{Tat (Muslim)}
\def\langnames@langs@glot@txx{Tatana'}
\def\langnames@langs@glot@tat{Tatar}
\def\langnames@langs@glot@tks{Tati (Southern)}
\def\langnames@langs@glot@tav{Tatuyo}
\def\langnames@langs@glot@tuh{Taulil}
\def\langnames@langs@glot@trr{Taushiro}
\def\langnames@langs@glot@tsg{Tausug}
\def\langnames@langs@glot@tya{Tauya}
\def\langnames@langs@glot@tbo{Tawala}
\def\langnames@langs@glot@cks{Tayo}
\def\langnames@langs@glot@tbl{Tboli}
\def\langnames@langs@glot@ttc{Tectiteco}
\def\langnames@langs@glot@kps{Tehit}
\def\langnames@langs@glot@teh{Tehuelche}
\def\langnames@langs@glot@kkw{Teke (Southern)}
\def\langnames@langs@glot@tlf{Telefol}
\def\langnames@langs@glot@tel{Telugu}
\def\langnames@langs@glot@kdh{Tem}
\def\langnames@langs@glot@teq{Temein}
\def\langnames@langs@glot@tea{Temiar}
\def\langnames@langs@glot@tem{Temne}
\def\langnames@langs@glot@tex{Tennet}
\def\langnames@langs@glot@kza{Tenyer}
\def\langnames@langs@glot@tio{Teop}
\def\langnames@langs@glot@tep{Tepecano}
\def\langnames@langs@glot@tee{Tepehua (Huehuetla)}
\def\langnames@langs@glot@tpt{Tepehua (Tlachichilco)}
\def\langnames@langs@glot@ntp{Tepehuan (Northern)}
\def\langnames@langs@glot@stp{Tepehuan (Southeastern)}
\def\langnames@langs@glot@ttr{Tera}
\def\langnames@langs@glot@tfr{Teribe}
\def\langnames@langs@glot@tft{Ternate}
\def\langnames@langs@glot@ter{Terêna}
\def\langnames@langs@glot@teo{Teso}
\def\langnames@langs@glot@tll{Tetela}
\def\langnames@langs@glot@tet{Tetun}
\def\langnames@langs@glot@tew{Tewa (Arizona)}
\def\langnames@langs@glot@tcz{Thadou}
\def\langnames@langs@glot@tha{Thai}
\def\langnames@langs@glot@tsq{Thai Sign Language}
\def\langnames@langs@glot@ths{Thakali}
\def\langnames@langs@glot@thf{Thangmi}
\def\langnames@langs@glot@ssf{Thao}
\def\langnames@langs@glot@typ{Thaypan}
\def\langnames@langs@glot@thp{Thompson}
\def\langnames@langs@glot@tdh{Thulung}
\def\langnames@langs@glot@tca{Ticuna}
\def\langnames@langs@glot@tvo{Tidore}
\def\langnames@langs@glot@tif{Tifal}
\def\langnames@langs@glot@tgc{Tigak}
\def\langnames@langs@glot@tir{Tigrinya}
\def\langnames@langs@glot@tig{Tigré}
\def\langnames@langs@glot@dih{Tiipay (Jamul)}
\def\langnames@langs@glot@tik{Tikar}
\def\langnames@langs@glot@til{Tillamook}
\def\langnames@langs@glot@tms{Tima}
\def\langnames@langs@glot@aoz{Timorese}
\def\langnames@langs@glot@tjm{Timucua}
\def\langnames@langs@glot@tih{Timugon}
\def\langnames@langs@glot@lbf{Tinani}
\def\langnames@langs@glot@tin{Tindi}
\def\langnames@langs@glot@cir{Tinrin}
\def\langnames@langs@glot@tri{Tiriyo}
\def\langnames@langs@glot@tiy{Tiruray}
\def\langnames@langs@glot@tiv{Tiv}
\def\langnames@langs@glot@twf{Tiwa (Northern)}
\def\langnames@langs@glot@tix{Tiwa (Southern)}
\def\langnames@langs@glot@tiw{Tiwi}
\def\langnames@langs@glot@tcf{Tlapanec}
\def\langnames@langs@glot@tli{Tlingit}
\def\langnames@langs@glot@tqo{Toaripi}
\def\langnames@langs@glot@tob{Toba}
\def\langnames@langs@glot@tti{Tobati}
\def\langnames@langs@glot@tlb{Tobelo}
\def\langnames@langs@glot@sbu{Tod}
\def\langnames@langs@glot@tcx{Toda}
\def\langnames@langs@glot@kim{Tofa}
\def\langnames@langs@glot@toj{Tojolabal}
\def\langnames@langs@glot@tpi{Tok Pisin}
\def\langnames@langs@glot@tkl{Tokelauan}
\def\langnames@langs@glot@jic{Tol}
\def\langnames@langs@glot@ksd{Tolai}
\def\langnames@langs@glot@dto{Tommo So}
\def\langnames@langs@glot@tdn{Tondano}
\def\langnames@langs@glot@toi{Tonga (in Zambia)}
\def\langnames@langs@glot@ton{Tongan}
\def\langnames@langs@glot@tqw{Tonkawa}
\def\langnames@langs@glot@tnt{Tontemboan}
\def\langnames@langs@glot@mlu{Toqabaqita}
\def\langnames@langs@glot@sda{Toraja}
\def\langnames@langs@glot@rth{Toratán}
\def\langnames@langs@glot@dts{Toro So}
\def\langnames@langs@glot@trw{Torwali}
\def\langnames@langs@glot@tlc{Totonac (Misantla)}
\def\langnames@langs@glot@top{Totonac (Papantla)}
\def\langnames@langs@glot@tos{Totonac (Sierra)}
\def\langnames@langs@glot@too{Totonac (Xicotepec de Juárez)}
\def\langnames@langs@glot@trs{Trique (Chicahuaxtla)}
\def\langnames@langs@glot@trc{Trique (Copala)}
\def\langnames@langs@glot@tpy{Trumai}
\def\langnames@langs@glot@cof{Tsafiki}
\def\langnames@langs@glot@tkr{Tsakhur}
\def\langnames@langs@glot@huq{Tsat}
\def\langnames@langs@glot@ddo{Tsez}
\def\langnames@langs@glot@tsj{Tshangla}
\def\langnames@langs@glot@tsi{Tsimshian (Coast)}
\def\langnames@langs@glot@tsv{Tsogo}
\def\langnames@langs@glot@tso{Tsonga}
\def\langnames@langs@glot@tsu{Tsou}
\def\langnames@langs@glot@bbl{Tsova-Tush}
\def\langnames@langs@glot@tsn{Tswana}
\def\langnames@langs@glot@pmt{Tuamotuan}
\def\langnames@langs@glot@thz{Tuareg (Air)}
\def\langnames@langs@glot@thv{Tuareg (Ghat)}
\def\langnames@langs@glot@tbu{Tubar}
\def\langnames@langs@glot@tuo{Tucano}
\def\langnames@langs@glot@tzn{Tugun}
\def\langnames@langs@glot@bag{Tuki}
\def\langnames@langs@glot@tcy{Tulu}
\def\langnames@langs@glot@tmc{Tumak}
\def\langnames@langs@glot@tmq{Tumleo}
\def\langnames@langs@glot@tuf{Tunebo}
\def\langnames@langs@glot@tvu{Tunen}
\def\langnames@langs@glot@lcm{Tungak}
\def\langnames@langs@glot@tun{Tunica}
\def\langnames@langs@glot@tpn{Tupi}
\def\langnames@langs@glot@tui{Tupuri}
\def\langnames@langs@glot@tuv{Turkana}
\def\langnames@langs@glot@kmz{Turkic (West Xorasan)}
\def\langnames@langs@glot@tur{Turkish}
\def\langnames@langs@glot@tuk{Turkmen}
\def\langnames@langs@glot@tus{Tuscarora}
\def\langnames@langs@glot@ttm{Tutchone (Northern)}
\def\langnames@langs@glot@tta{Tutelo}
\def\langnames@langs@glot@tvt{Tutsa}
\def\langnames@langs@glot@tyv{Tuvan}
\def\langnames@langs@glot@tue{Tuyuca}
\def\langnames@langs@glot@twa{Twana}
\def\langnames@langs@glot@woa{Tyeraity}
\def\langnames@langs@glot@tzh{Tzeltal}
\def\langnames@langs@glot@tzo{Tzotzil}
\def\langnames@langs@glot@tzj{Tzutujil}
\def\langnames@langs@glot@tub{Tübatulabal}
\def\langnames@langs@glot@par{Tümpisa Shoshone}
\def\langnames@langs@glot@tsm{Türk Isaret Dili}
\def\langnames@langs@glot@umb{UMbundu}
\def\langnames@langs@glot@uby{Ubykh}
\def\langnames@langs@glot@udi{Udi}
\def\langnames@langs@glot@ude{Udihe}
\def\langnames@langs@glot@udm{Udmurt}
\def\langnames@langs@glot@ugn{Ugandan Sign Language}
\def\langnames@langs@glot@ukr{Ukrainian}
\def\langnames@langs@glot@ulc{Ulcha}
\def\langnames@langs@glot@udl{Uldeme}
\def\langnames@langs@glot@uli{Ulithian}
\def\langnames@langs@glot@ppk{Uma}
\def\langnames@langs@glot@cbd{Umaua}
\def\langnames@langs@glot@ubu{Umbu Ungu}
\def\langnames@langs@glot@ump{Umpila}
\def\langnames@langs@glot@mtg{Una}
\def\langnames@langs@glot@unm{Unami}
\def\langnames@langs@glot@ung{Ungarinjin}
\def\langnames@langs@glot@kuu{Upper Kuskokwim}
\def\langnames@langs@glot@uur{Ura}
\def\langnames@langs@glot@urf{Uradhi}
\def\langnames@langs@glot@urk{Urak Lawoi'}
\def\langnames@langs@glot@ura{Urarina}
\def\langnames@langs@glot@urt{Urat}
\def\langnames@langs@glot@urd{Urdu}
\def\langnames@langs@glot@urh{Urhobo}
\def\langnames@langs@glot@uri{Urim}
\def\langnames@langs@glot@ure{Uru}
\def\langnames@langs@glot@uks{Urubú Sign Language}
\def\langnames@langs@glot@urb{Urubú-Kaapor}
\def\langnames@langs@glot@uum{Urum}
\def\langnames@langs@glot@wnu{Usan}
\def\langnames@langs@glot@usa{Usarufa}
\def\langnames@langs@glot@ute{Ute}
\def\langnames@langs@glot@uig{Uyghur}
\def\langnames@langs@glot@uzn{Uzbek (Northern)}
\def\langnames@langs@glot@vaf{Vafsi}
\def\langnames@langs@glot@vag{Vagla}
\def\langnames@langs@glot@vai{Vai}
\def\langnames@langs@glot@vas{Vasavi}
\def\langnames@langs@glot@dic{Vata}
\def\langnames@langs@glot@ved{Vedda}
\def\langnames@langs@glot@ven{Venda}
\def\langnames@langs@glot@vep{Veps}
\def\langnames@langs@glot@vie{Vietnamese}
\def\langnames@langs@glot@vif{Vili}
\def\langnames@langs@glot@vnm{Vinmavis}
\def\langnames@langs@glot@vgt{Vlaamse Gebarentaal}
\def\langnames@langs@glot@vot{Votic}
\def\langnames@langs@glot@wwa{Waama}
\def\langnames@langs@glot@wkw{Wagawaga}
\def\langnames@langs@glot@waq{Wagiman}
\def\langnames@langs@glot@waw{Wai Wai}
\def\langnames@langs@glot@wbk{Waigali}
\def\langnames@langs@glot@bao{Waimaha}
\def\langnames@langs@glot@wbl{Wakhi}
\def\langnames@langs@glot@wls{Wallisian}
\def\langnames@langs@glot@van{Walman}
\def\langnames@langs@glot@wmt{Walmatjari}
\def\langnames@langs@glot@wmb{Wambaya}
\def\langnames@langs@glot@wms{Wambon}
\def\langnames@langs@glot@wme{Wambule}
\def\langnames@langs@glot@wan{Wan}
\def\langnames@langs@glot@wgg{Wangkangurru}
\def\langnames@langs@glot@xwk{Wangkumara}
\def\langnames@langs@glot@wbt{Wanman}
\def\langnames@langs@glot@wnc{Wantoat}
\def\langnames@langs@glot@auc{Waorani}
\def\langnames@langs@glot@wap{Wapishana}
\def\langnames@langs@glot@wao{Wappo}
\def\langnames@langs@glot@wba{Warao}
\def\langnames@langs@glot@wrz{Waray (in Australia)}
\def\langnames@langs@glot@war{Waray-Waray}
\def\langnames@langs@glot@wrr{Wardaman}
\def\langnames@langs@glot@gae{Warekena}
\def\langnames@langs@glot@wsa{Warembori}
\def\langnames@langs@glot@pav{Wari'}
\def\langnames@langs@glot@wrs{Waris}
\def\langnames@langs@glot@wbp{Warlpiri}
\def\langnames@langs@glot@wrb{Warluwara}
\def\langnames@langs@glot@wnd{Warndarang}
\def\langnames@langs@glot@wrp{Waropen}
\def\langnames@langs@glot@wgy{Warrgamay}
\def\langnames@langs@glot@gjm{Warrnambool}
\def\langnames@langs@glot@wrg{Warrongo}
\def\langnames@langs@glot@wwr{Warrwa}
\def\langnames@langs@glot@wrm{Warumungu}
\def\langnames@langs@glot@was{Washo}
\def\langnames@langs@glot@wsk{Waskia}
\def\langnames@langs@glot@wax{Watam}
\def\langnames@langs@glot@wth{Wathawurrung}
\def\langnames@langs@glot@wbv{Watjarri}
\def\langnames@langs@glot@noa{Waunana}
\def\langnames@langs@glot@wau{Waurá}
\def\langnames@langs@glot@oym{Wayampi}
\def\langnames@langs@glot@way{Wayana}
\def\langnames@langs@glot@wed{Wedau}
\def\langnames@langs@glot@cym{Welsh}
\def\langnames@langs@glot@xww{Wembawemba}
\def\langnames@langs@glot@wer{Weri}
\def\langnames@langs@glot@mqs{West Makian}
\def\langnames@langs@glot@lex{Wetan}
\def\langnames@langs@glot@wic{Wichita}
\def\langnames@langs@glot@mzh{Wichí}
\def\langnames@langs@glot@wim{Wik Munkan}
\def\langnames@langs@glot@wig{Wik Ngathana}
\def\langnames@langs@glot@yok{Wikchamni}
\def\langnames@langs@glot@win{Winnebago}
\def\langnames@langs@glot@wnw{Wintu}
\def\langnames@langs@glot@wgu{Wirangu}
\def\langnames@langs@glot@wiy{Wiyot}
\def\langnames@langs@glot@wob{Wobe}
\def\langnames@langs@glot@wog{Wogamusin}
\def\langnames@langs@glot@woi{Woisika}
\def\langnames@langs@glot@wyu{Woiwurrung}
\def\langnames@langs@glot@wal{Wolaytta}
\def\langnames@langs@glot@woe{Woleaian}
\def\langnames@langs@glot@wlo{Wolio}
\def\langnames@langs@glot@wol{Wolof}
\def\langnames@langs@glot@wmx{Womo}
\def\langnames@langs@glot@wro{Worora}
\def\langnames@langs@glot@wuu{Wu}
\def\langnames@langs@glot@wya{Wyandot}
\def\langnames@langs@glot@wem{Wéménugbé}
\def\langnames@langs@glot@kao{Xasonga}
\def\langnames@langs@glot@xav{Xavánte}
\def\langnames@langs@glot@xer{Xerénte}
\def\langnames@langs@glot@xho{Xhosa}
\def\langnames@langs@glot@xir{Xiriana}
\def\langnames@langs@glot@xok{Xokleng}
\def\langnames@langs@glot@ane{Xârâcùù}
\def\langnames@langs@glot@yai{Yaghnobi}
\def\langnames@langs@glot@yad{Yagua}
\def\langnames@langs@glot@yag{Yahgan}
\def\langnames@langs@glot@yaf{Yaka}
\def\langnames@langs@glot@yka{Yakan}
\def\langnames@langs@glot@yky{Yakoma}
\def\langnames@langs@glot@sah{Yakut}
\def\langnames@langs@glot@ylr{Yalarnnga}
\def\langnames@langs@glot@kkl{Yale (Kosarek)}
\def\langnames@langs@glot@yli{Yali}
\def\langnames@langs@glot@yam{Yamba}
\def\langnames@langs@glot@jmd{Yamdena}
\def\langnames@langs@glot@tao{Yami}
\def\langnames@langs@glot@yaa{Yaminahua}
\def\langnames@langs@glot@ybi{Yamphu}
\def\langnames@langs@glot@ynn{Yana}
\def\langnames@langs@glot@kdd{Yankuntjatjara}
\def\langnames@langs@glot@wca{Yanomámi}
\def\langnames@langs@glot@yns{Yansi}
\def\langnames@langs@glot@jao{Yanyuwa}
\def\langnames@langs@glot@yao{Yao (in Malawi)}
\def\langnames@langs@glot@yap{Yapese}
\def\langnames@langs@glot@jaq{Yaqay}
\def\langnames@langs@glot@yaq{Yaqui}
\def\langnames@langs@glot@yrb{Yareba}
\def\langnames@langs@glot@yae{Yaruro}
\def\langnames@langs@glot@yuf{Yavapai}
\def\langnames@langs@glot@yva{Yawa}
\def\langnames@langs@glot@ywr{Yawuru}
\def\langnames@langs@glot@pcc{Yay}
\def\langnames@langs@glot@xya{Yaygir}
\def\langnames@langs@glot@yah{Yazgulyam}
\def\langnames@langs@glot@kpv{Yazva}
\def\langnames@langs@glot@jei{Yei}
\def\langnames@langs@glot@jel{Yelmek}
\def\langnames@langs@glot@yle{Yelî Dnye}
\def\langnames@langs@glot@ybb{Yemba}
\def\langnames@langs@glot@jnj{Yemsa}
\def\langnames@langs@glot@yss{Yessan-Mayo}
\def\langnames@langs@glot@yey{Yeyi}
\def\langnames@langs@glot@ywq{Yi (Wuding-Luquan)}
\def\langnames@langs@glot@ydd{Yiddish}
\def\langnames@langs@glot@yii{Yidiny}
\def\langnames@langs@glot@yll{Yil}
\def\langnames@langs@glot@yee{Yimas}
\def\langnames@langs@glot@yij{Yindjibarndi}
\def\langnames@langs@glot@yia{Yingkarta}
\def\langnames@langs@glot@yyr{Yir Yoront}
\def\langnames@langs@glot@xyy{Yorta Yorta}
\def\langnames@langs@glot@yor{Yoruba}
\def\langnames@langs@glot@yua{Yucatec}
\def\langnames@langs@glot@yuc{Yuchi}
\def\langnames@langs@glot@ycn{Yucuna}
\def\langnames@langs@glot@yug{Yugh}
\def\langnames@langs@glot@yux{Yukaghir (Kolyma)}
\def\langnames@langs@glot@ykg{Yukaghir (Tundra)}
\def\langnames@langs@glot@yuk{Yuki}
\def\langnames@langs@glot@yup{Yukpa}
\def\langnames@langs@glot@gcd{Yukulta}
\def\langnames@langs@glot@mpj{Yulparija}
\def\langnames@langs@glot@yul{Yulu}
\def\langnames@langs@glot@esu{Yup'ik (Chevak)}
\def\langnames@langs@glot@ynk{Yupik (Naukan)}
\def\langnames@langs@glot@ess{Yupik (Siberian)}
\def\langnames@langs@glot@ysr{Yupik (Sirenik)}
\def\langnames@langs@glot@yuz{Yuracare}
\def\langnames@langs@glot@yur{Yurok}
\def\langnames@langs@glot@yui{Yuruti}
\def\langnames@langs@glot@zne{Zande}
\def\langnames@langs@glot@zro{Zaparo}
\def\langnames@langs@glot@zai{Zapotec (Isthmus)}
\def\langnames@langs@glot@zpd{Zapotec (Ixtlan)}
\def\langnames@langs@glot@zaa{Zapotec (Juárez)}
\def\langnames@langs@glot@zaw{Zapotec (Mitla)}
\def\langnames@langs@glot@zpm{Zapotec (Mixtepec)}
\def\langnames@langs@glot@zpi{Zapotec (Quiegolani)}
\def\langnames@langs@glot@zab{Zapotec (San Lucas Quiaviní)}
\def\langnames@langs@glot@zpz{Zapotec (Texmelucan)}
\def\langnames@langs@glot@zav{Zapotec (Yatzachi)}
\def\langnames@langs@glot@zpq{Zapotec (Zoogocho)}
\def\langnames@langs@glot@dje{Zarma}
\def\langnames@langs@glot@zay{Zayse}
\def\langnames@langs@glot@diq{Zazaki}
\def\langnames@langs@glot@zen{Zenaga}
\def\langnames@langs@glot@zgb{Zhuang (Northern)}
\def\langnames@langs@glot@zik{Zimakani}
\def\langnames@langs@glot@zoh{Zoque (Chimalapa)}
\def\langnames@langs@glot@zos{Zoque (Francisco León)}
\def\langnames@langs@glot@zoc{Zoque (Ostuacan)}
\def\langnames@langs@glot@zor{Zoque (Rayon)}
\def\langnames@langs@glot@zul{Zulu}
\def\langnames@langs@glot@zun{Zuni}
\def\langnames@langs@glot@eme{Émérillon}
\def\langnames@langs@glot@aom{Ömie}
\def\langnames@langs@glot@aas{Aasax}
\def\langnames@langs@glot@kbt{Abadi}
\def\langnames@langs@glot@abg{Abaga}
\def\langnames@langs@glot@abf{Abai Sungai}
\def\langnames@langs@glot@abm{Abanyom}
\def\langnames@langs@glot@mij{Mungbam}
\def\langnames@langs@glot@aba{Abé}
\def\langnames@langs@glot@abp{Abenlen Ayta}
\def\langnames@langs@glot@bsa{Abinomn}
\def\langnames@langs@glot@ash{Aewa}
\def\langnames@langs@glot@aob{Abom}
\def\langnames@langs@glot@abo{Abon}
\def\langnames@langs@glot@abr{Abron}
\def\langnames@langs@glot@abn{Abua}
\def\langnames@langs@glot@abu{Abure}
\def\langnames@langs@glot@mgj{Abureni}
\def\langnames@langs@glot@ado{Abu}
\def\langnames@langs@glot@tpx{Acatepec Me'phaa}
\def\langnames@langs@glot@yif{Ache}
\def\langnames@langs@glot@acz{Acheron}
\def\langnames@langs@glot@acs{Acroá}
\def\langnames@langs@glot@xad{Adai}
\def\langnames@langs@glot@ada{Adangme}
\def\langnames@langs@glot@adq{Adangbe}
\def\langnames@langs@glot@tiu{Adasen}
\def\langnames@langs@glot@ade{Adele}
\def\langnames@langs@glot@adh{Adhola}
\def\langnames@langs@glot@gas{Adiwasi Garasia}
\def\langnames@langs@glot@adr{Adonara}
\def\langnames@langs@glot@aez{Aeka}
\def\langnames@langs@glot@aeq{Aer}
\def\langnames@langs@glot@afg{Afghan Sign Language}
\def\langnames@langs@glot@aft{Afitti}
\def\langnames@langs@glot@afh{Afrihili}
\def\langnames@langs@glot@afs{Afro-Seminole Creole}
\def\langnames@langs@glot@agi{Agariya}
\def\langnames@langs@glot@agc{Agatu}
\def\langnames@langs@glot@avo{Agavotaguerra}
\def\langnames@langs@glot@ggr{Aghu Tharnggalu}
\def\langnames@langs@glot@xag{Aghwan}
\def\langnames@langs@glot@aif{Agi}
\def\langnames@langs@glot@kit{Agob-Ende-Kawam}
\def\langnames@langs@glot@ibm{Agoi}
\def\langnames@langs@glot@apf{Agta-Pahanan}
\def\langnames@langs@glot@aga{Aguano}
\def\langnames@langs@glot@aug{Aguna}
\def\langnames@langs@glot@msm{Agusan Manobo}
\def\langnames@langs@glot@agn{Agutaynen}
\def\langnames@langs@glot@yay{Agwagwune}
\def\langnames@langs@glot@aha{Ahanta}
\def\langnames@langs@glot@ahn{Àhàn}
\def\langnames@langs@glot@esg{Aheri Gondi}
\def\langnames@langs@glot@thm{Thavung}
\def\langnames@langs@glot@kak{Ahin-Kayapa Kalanguya}
\def\langnames@langs@glot@aho{Ahom}
\def\langnames@langs@glot@nfd{Ndunic}
\def\langnames@langs@glot@aih{Ai-Cham}
\def\langnames@langs@glot@aix{Aighon}
\def\langnames@langs@glot@mwg{Aiklep}
\def\langnames@langs@glot@aiq{Aimaq}
\def\langnames@langs@glot@ail{Aimele}
\def\langnames@langs@glot@aim{Aimol}
\def\langnames@langs@glot@aic{Ainbai}
\def\langnames@langs@glot@aki{Aiome}
\def\langnames@langs@glot@air{Airoran}
\def\langnames@langs@glot@aio{Aiton}
\def\langnames@langs@glot@ajw{Ajawa}
\def\langnames@langs@glot@cpc{Ajyíninka Apurucayali}
\def\langnames@langs@glot@soh{Aka}
\def\langnames@langs@glot@akm{Akabo}
\def\langnames@langs@glot@akj{Akajeru}
\def\langnames@langs@glot@ack{Akakora}
\def\langnames@langs@glot@aky{Akakol}
\def\langnames@langs@glot@acl{Akarbale}
\def\langnames@langs@glot@aks{Akaselem}
\def\langnames@langs@glot@aik{Akye}
\def\langnames@langs@glot@tsr{Akei}
\def\langnames@langs@glot@aeu{Akeu}
\def\langnames@langs@glot@sia{Akkala Saami}
\def\langnames@langs@glot@akk{Akkadian}
\def\langnames@langs@glot@akq{Ak}
\def\langnames@langs@glot@akt{Akolet}
\def\langnames@langs@glot@bss{Akoose}
\def\langnames@langs@glot@miw{Akoye}
\def\langnames@langs@glot@akf{Akpa}
\def\langnames@langs@glot@ibe{Akpes}
\def\langnames@langs@glot@afi{Chini}
\def\langnames@langs@glot@ayk{Akuku}
\def\langnames@langs@glot@aku{Akum}
\def\langnames@langs@glot@aqz{Akuntsu}
\def\langnames@langs@glot@ako{Akurio}
\def\langnames@langs@glot@dul{Alabat Island Agta}
\def\langnames@langs@glot@alw{Alaba-K'abeena}
\def\langnames@langs@glot@ala{Alago}
\def\langnames@langs@glot@alk{Alak}
\def\langnames@langs@glot@alj{Alangan}
\def\langnames@langs@glot@apv{Alapmunte}
\def\langnames@langs@glot@bhk{Inland-Buhi-Daraga Bikol}
\def\langnames@langs@glot@sqk{Albanian Sign Language}
\def\langnames@langs@glot@lsc{Albarradas Sign Language}
\def\langnames@langs@glot@xta{Alcozauca Mixtec}
\def\langnames@langs@glot@alf{Alege}
\def\langnames@langs@glot@asp{Algerian Sign Language}
\def\langnames@langs@glot@arq{Algerian Arabic}
\def\langnames@langs@glot@aao{Algerian Saharan Arabic}
\def\langnames@langs@glot@aiy{Ali}
\def\langnames@langs@glot@all{Allar}
\def\langnames@langs@glot@aid{Alngith}
\def\langnames@langs@glot@zaq{Aloápam Zapotec}
\def\langnames@langs@glot@ypo{Alo Phola}
\def\langnames@langs@glot@aol{Alorese}
\def\langnames@langs@glot@syy{Al-Sayyid Bedouin Sign Language}
\def\langnames@langs@glot@aub{Alugu}
\def\langnames@langs@glot@xua{Alu Kurumba}
\def\langnames@langs@glot@aab{Arum}
\def\langnames@langs@glot@yna{Aluo}
\def\langnames@langs@glot@alz{Alur}
\def\langnames@langs@glot@avd{Alviri-Vidari}
\def\langnames@langs@glot@amq{Amahai}
\def\langnames@langs@glot@ali{Amaimon}
\def\langnames@langs@glot@aad{Amal}
\def\langnames@langs@glot@jks{Amami O Shima Sign Language}
\def\langnames@langs@glot@ama{Amanayé}
\def\langnames@langs@glot@amg{Amurdak}
\def\langnames@langs@glot@aaz{Amarasi}
\def\langnames@langs@glot@zpo{Amatlán Zapotec}
\def\langnames@langs@glot@rwm{Amba (Uganda)}
\def\langnames@langs@glot@utp{Amba (Solomon Islands)}
\def\langnames@langs@glot@abc{Ambala Ayta}
\def\langnames@langs@glot@aew{Ambakich}
\def\langnames@langs@glot@ael{Ambele}
\def\langnames@langs@glot@amv{Ambelau}
\def\langnames@langs@glot@alm{Amblong}
\def\langnames@langs@glot@amb{Ambo}
\def\langnames@langs@glot@abs{Ambonese Malay}
\def\langnames@langs@glot@qva{Ambo-Pasco Quechua}
\def\langnames@langs@glot@aag{Ambrak}
\def\langnames@langs@glot@amj{Amdang}
\def\langnames@langs@glot@ifa{Amganad Ifugao}
\def\langnames@langs@glot@alx{Mol}
\def\langnames@langs@glot@mbz{Amoltepec Mixtec}
\def\langnames@langs@glot@aqd{Ampari Dogon}
\def\langnames@langs@glot@apg{Ampanang}
\def\langnames@langs@glot@ajz{Amri Karbi}
\def\langnames@langs@glot@amt{Amto}
\def\langnames@langs@glot@adw{Amundava}
\def\langnames@langs@glot@anw{Anaang}
\def\langnames@langs@glot@akg{Anakalangu}
\def\langnames@langs@glot@anm{Anal}
\def\langnames@langs@glot@pda{Anam}
\def\langnames@langs@glot@aan{Anambé}
\def\langnames@langs@glot@dti{Ana Tinga Dogon}
\def\langnames@langs@glot@grc{Ancient Greek}
\def\langnames@langs@glot@hbo{Ancient Hebrew}
\def\langnames@langs@glot@xna{Ancient North Arabian}
\def\langnames@langs@glot@xlg{Ancient Ligurian}
\def\langnames@langs@glot@hca{Andaman Creole Hindi}
\def\langnames@langs@glot@afd{Andai}
\def\langnames@langs@glot@aod{Andarum}
\def\langnames@langs@glot@ana{Andaqui}
\def\langnames@langs@glot@xaa{Andalusian Arabic}
\def\langnames@langs@glot@adg{Andegerebinha}
\def\langnames@langs@glot@bzb{Andio}
\def\langnames@langs@glot@anb{Andoa}
\def\langnames@langs@glot@anx{Andra-Hus}
\def\langnames@langs@glot@aby{Aneme Wake}
\def\langnames@langs@glot@myo{Anfillo}
\def\langnames@langs@glot@akh{Angal Heneng}
\def\langnames@langs@glot@age{Angal}
\def\langnames@langs@glot@aoe{Angal Enen}
\def\langnames@langs@glot@aqt{Angaité}
\def\langnames@langs@glot@avm{Angkamuthi}
\def\langnames@langs@glot@anp{Angika}
\def\langnames@langs@glot@rme{Archaic Angloromani}
\def\langnames@langs@glot@aog{Angoram}
\def\langnames@langs@glot@tnd{Angosturas Tunebo}
\def\langnames@langs@glot@blo{Anii}
\def\langnames@langs@glot@anf{Animere}
\def\langnames@langs@glot@aqk{Aninka}
\def\langnames@langs@glot@ypn{Ani Phowa}
\def\langnames@langs@glot@boj{Anjam}
\def\langnames@langs@glot@aak{Ankave}
\def\langnames@langs@glot@amx{Anmatyerre}
\def\langnames@langs@glot@anj{Anor}
\def\langnames@langs@glot@ans{Anserma}
\def\langnames@langs@glot@and{Ansus}
\def\langnames@langs@glot@ant{Antakarinya}
\def\langnames@langs@glot@xmv{Antankarana Malagasy}
\def\langnames@langs@glot@aig{Antigua and Barbuda Creole English}
\def\langnames@langs@glot@aui{Anuki}
\def\langnames@langs@glot@auq{Anus}
\def\langnames@langs@glot@aud{Anuta}
\def\langnames@langs@glot@anl{Anu-Hkongso}
\def\langnames@langs@glot@mtb{Anyin Morofo}
\def\langnames@langs@glot@pni{Aoheng-Seputan}
\def\langnames@langs@glot@aor{Aore}
\def\langnames@langs@glot@aou{A'ou}
\def\langnames@langs@glot@xap{Apalachee}
\def\langnames@langs@glot@apo{Apalik}
\def\langnames@langs@glot@ena{Apali}
\def\langnames@langs@glot@mip{Apasco-Apoala Mixtec}
\def\langnames@langs@glot@api{Apiaká}
\def\langnames@langs@glot@app{Apma}
\def\langnames@langs@glot@apx{Aputai}
\def\langnames@langs@glot@arg{Aragonese}
\def\langnames@langs@glot@stk{Arammba}
\def\langnames@langs@glot@aaf{Aranadan}
\def\langnames@langs@glot@xrt{Aranama}
\def\langnames@langs@glot@arj{Arapaso}
\def\langnames@langs@glot@awm{Arawum}
\def\langnames@langs@glot@awt{Araweté}
\def\langnames@langs@glot@aae{Arbëreshë Albanian}
\def\langnames@langs@glot@aea{Areba}
\def\langnames@langs@glot@mwc{Are}
\def\langnames@langs@glot@aem{Arem}
\def\langnames@langs@glot@qxu{Arequipa-La Unión Quechua}
\def\langnames@langs@glot@agj{Argobba}
\def\langnames@langs@glot@agf{Arguni}
\def\langnames@langs@glot@aqr{Arhâ}
\def\langnames@langs@glot@aok{Arhö}
\def\langnames@langs@glot@ylu{Aribwaung}
\def\langnames@langs@glot@aai{Arifama-Miniafia}
\def\langnames@langs@glot@aqg{Arigidi}
\def\langnames@langs@glot@aac{Ari}
\def\langnames@langs@glot@ait{Arikem}
\def\langnames@langs@glot@ark{Arikapú}
\def\langnames@langs@glot@xrn{Arin}
\def\langnames@langs@glot@luc{Aringa}
\def\langnames@langs@glot@dth{Aritinngitigh}
\def\langnames@langs@glot@aoh{Arma}
\def\langnames@langs@glot@aen{Armenian Sign Language}
\def\langnames@langs@glot@rup{Aromanian}
\def\langnames@langs@glot@aps{Arop-Sissano}
\def\langnames@langs@glot@atz{Arta}
\def\langnames@langs@glot@arx{Aruá (Rondonia State)}
\def\langnames@langs@glot@aru{Aruá (Amazonas State)}
\def\langnames@langs@glot@aur{Aruek}
\def\langnames@langs@glot@lsr{Srenge}
\def\langnames@langs@glot@atx{Arutani}
\def\langnames@langs@glot@aat{Arvanitika Albanian}
\def\langnames@langs@glot@mtv{Asaro'o}
\def\langnames@langs@glot@cni{Asháninka}
\def\langnames@langs@glot@ahs{Ashe}
\def\langnames@langs@glot@prq{Ashéninka Perené}
\def\langnames@langs@glot@ask{Ashkun}
\def\langnames@langs@glot@atn{Ashtiani}
\def\langnames@langs@glot@asl{Asilulu}
\def\langnames@langs@glot@eiv{Askopan}
\def\langnames@langs@glot@asv{Asoa}
\def\langnames@langs@glot@asb{Assiniboine}
\def\langnames@langs@glot@asz{As}
\def\langnames@langs@glot@aua{Asumboa}
\def\langnames@langs@glot@aum{Asu (Nigeria)}
\def\langnames@langs@glot@zoo{Asunción Mixtepec Zapotec}
\def\langnames@langs@glot@asr{Asuri}
\def\langnames@langs@glot@atm{Ata}
\def\langnames@langs@glot@amz{Atampaya}
\def\langnames@langs@glot@atd{Ata Manobo}
\def\langnames@langs@glot@ate{Mand}
\def\langnames@langs@glot@atk{Ati}
\def\langnames@langs@glot@aqm{Atohwaim}
\def\langnames@langs@glot@aot{Atong (India)}
\def\langnames@langs@glot@ato{Atong}
\def\langnames@langs@glot@aox{Atorada}
\def\langnames@langs@glot@cch{Atsam}
\def\langnames@langs@glot@atc{Atsahuaca}
\def\langnames@langs@glot@pkr{Attapady Kurumba}
\def\langnames@langs@glot@ati{Attié}
\def\langnames@langs@glot@kud{'Auhelawa}
\def\langnames@langs@glot@aux{Aurê y Aurá}
\def\langnames@langs@glot@auh{Aushi}
\def\langnames@langs@glot@avs{Aushiri}
\def\langnames@langs@glot@asq{Austrian Sign Language}
\def\langnames@langs@glot@asw{Australian Aborigines Sign Language}
\def\langnames@langs@glot@aut{Austral}
\def\langnames@langs@glot@smf{Auwe}
\def\langnames@langs@glot@auu{Auye}
\def\langnames@langs@glot@auo{Auyokawa}
\def\langnames@langs@glot@avv{Avá-Canoeiro}
\def\langnames@langs@glot@avb{Avau}
\def\langnames@langs@glot@ave{Avestan}
\def\langnames@langs@glot@awk{Awabakal}
\def\langnames@langs@glot@vwa{Lavia-Awalai-Damangnuo Awa}
\def\langnames@langs@glot@bcu{Awad Bing}
\def\langnames@langs@glot@awo{Awak}
\def\langnames@langs@glot@awx{Awara}
\def\langnames@langs@glot@aya{Awar}
\def\langnames@langs@glot@awh{Awbono}
\def\langnames@langs@glot@bob{Aweer}
\def\langnames@langs@glot@awr{Awera}
\def\langnames@langs@glot@awe{Awetí}
\def\langnames@langs@glot@azo{Awing}
\def\langnames@langs@glot@auj{Awjilah}
\def\langnames@langs@glot@aww{Auwon}
\def\langnames@langs@glot@afu{Awutu}
\def\langnames@langs@glot@yiu{Southern Awu (Lope)}
\def\langnames@langs@glot@ahb{Axamb}
\def\langnames@langs@glot@yix{Axi Yi}
\def\langnames@langs@glot@ayd{Yintyinka-Ayabadhu}
\def\langnames@langs@glot@vmy{Ayautla Mazatec}
\def\langnames@langs@glot@aye{Ayere}
\def\langnames@langs@glot@ayq{Ayi (Papua New Guinea)}
\def\langnames@langs@glot@yyz{Ayizi}
\def\langnames@langs@glot@ayb{Ayizo Gbe}
\def\langnames@langs@glot@zaf{Ayoquesco Zapotec}
\def\langnames@langs@glot@ayu{Ayu}
\def\langnames@langs@glot@aza{Azha}
\def\langnames@langs@glot@yiz{Azhe}
\def\langnames@langs@glot@tpc{Azoyú Me'phaa}
\def\langnames@langs@glot@bvj{Baan}
\def\langnames@langs@glot@bqx{Baangi}
\def\langnames@langs@glot@bbm{Babango}
\def\langnames@langs@glot@bbw{Baba}
\def\langnames@langs@glot@bbk{Babanki}
\def\langnames@langs@glot@mbf{Baba Malay}
\def\langnames@langs@glot@bcr{Witsuwit'en-Babine}
\def\langnames@langs@glot@bzg{Babuza}
\def\langnames@langs@glot@btj{Bacanese Malay}
\def\langnames@langs@glot@bcy{Bacama}
\def\langnames@langs@glot@xbc{Bactrian}
\def\langnames@langs@glot@bau{Bada (Nigeria)}
\def\langnames@langs@glot@bhz{Bada (Indonesia)}
\def\langnames@langs@glot@bdz{Badeshi}
\def\langnames@langs@glot@jbi{Badjirri}
\def\langnames@langs@glot@bac{Badui}
\def\langnames@langs@glot@pbp{Jaad-Badyara}
\def\langnames@langs@glot@bvd{Baeggu}
\def\langnames@langs@glot@bvc{Baelelea}
\def\langnames@langs@glot@btr{Baetora}
\def\langnames@langs@glot@bwt{Bafaw-Balong}
\def\langnames@langs@glot@bfj{Bafanji}
\def\langnames@langs@glot@bmd{Baga Manduri}
\def\langnames@langs@glot@bgo{Baga Koga}
\def\langnames@langs@glot@bcg{Pukur}
\def\langnames@langs@glot@bfy{Bagheli}
\def\langnames@langs@glot@fui{Bagirmi Fulfulde}
\def\langnames@langs@glot@bqg{Bago-Kusuntu}
\def\langnames@langs@glot@bqb{Bagusa}
\def\langnames@langs@glot@bpi{Bagupi}
\def\langnames@langs@glot@yha{Baha Buyang}
\def\langnames@langs@glot@bhv{Bahau}
\def\langnames@langs@glot@bah{Bahamas Creole English}
\def\langnames@langs@glot@bhj{Bahing}
\def\langnames@langs@glot@bsu{Bahonsuai}
\def\langnames@langs@glot@bbf{Baibai}
\def\langnames@langs@glot@bdj{Bai}
\def\langnames@langs@glot@bkx{Baikeno}
\def\langnames@langs@glot@bqh{Baima}
\def\langnames@langs@glot@bmx{Baimak}
\def\langnames@langs@glot@bab{Bainounk-Gujaher}
\def\langnames@langs@glot@bcz{Bainouk-Gunyaamolo-Gutobor}
\def\langnames@langs@glot@fah{Baissa Fali}
\def\langnames@langs@glot@bjs{Bajan}
\def\langnames@langs@glot@bjm{Bajelani}
\def\langnames@langs@glot@bqz{Bakaka}
\def\langnames@langs@glot@bqi{Bakhtiari}
\def\langnames@langs@glot@bki{Baki}
\def\langnames@langs@glot@bkh{Bakoko}
\def\langnames@langs@glot@kme{Bakole}
\def\langnames@langs@glot@bbs{Bakpinka}
\def\langnames@langs@glot@bkr{Bakumpai}
\def\langnames@langs@glot@bjw{Bakwé}
\def\langnames@langs@glot@ble{Balanta-Kentohe}
\def\langnames@langs@glot@bjt{Balanta-Ganja}
\def\langnames@langs@glot@bls{Balaesang}
\def\langnames@langs@glot@bdn{Baldemu}
\def\langnames@langs@glot@bcn{Bali (Nigeria)}
\def\langnames@langs@glot@bcp{Bali (Democratic Republic of Congo)}
\def\langnames@langs@glot@mhp{Balinese Malay}
\def\langnames@langs@glot@bgx{Rumelian Turkish}
\def\langnames@langs@glot@biz{Loi-Likila}
\def\langnames@langs@glot@bqo{Balo}
\def\langnames@langs@glot@blq{Paluai}
\def\langnames@langs@glot@bog{Langue de Signes Malienne}
\def\langnames@langs@glot@bbq{Bamali}
\def\langnames@langs@glot@myf{Bambassi}
\def\langnames@langs@glot@bmo{Bambalang}
\def\langnames@langs@glot@bce{Bamenyam}
\def\langnames@langs@glot@bqt{Bamukumbit}
\def\langnames@langs@glot@bvm{Bamunka}
\def\langnames@langs@glot@bcf{Bamu}
\def\langnames@langs@glot@bmg{Bamwe}
\def\langnames@langs@glot@bjx{Banao Itneg}
\def\langnames@langs@glot@byz{Banaro}
\def\langnames@langs@glot@bqj{Bandial}
\def\langnames@langs@glot@bqk{Banda-Mbrès}
\def\langnames@langs@glot@bpd{Banda-Banda}
\def\langnames@langs@glot@bfl{Banda-Ndélé}
\def\langnames@langs@glot@yaj{Banda-Yangere}
\def\langnames@langs@glot@bpq{Banda Malay}
\def\langnames@langs@glot@bnd{Banda (Indonesia)}
\def\langnames@langs@glot@bbe{Bangba}
\def\langnames@langs@glot@bgf{Ngombe-Bangandu}
\def\langnames@langs@glot@bsj{Bangwinji}
\def\langnames@langs@glot@bnx{Bangubangu}
\def\langnames@langs@glot@bxg{Bangala}
\def\langnames@langs@glot@bgj{Bangolan}
\def\langnames@langs@glot@mfb{Bangka}
\def\langnames@langs@glot@bjn{Banjar}
\def\langnames@langs@glot@bfk{Ban Khor Sign Language}
\def\langnames@langs@glot@bxw{Bankagooma}
\def\langnames@langs@glot@dbw{Bankan Tey Dogon}
\def\langnames@langs@glot@bap{Bantawa}
\def\langnames@langs@glot@bno{Bantoanon}
\def\langnames@langs@glot@bfx{Bantayanon}
\def\langnames@langs@glot@brd{Baraamu}
\def\langnames@langs@glot@bbg{Barama}
\def\langnames@langs@glot@baj{Barakai}
\def\langnames@langs@glot@bhr{Bara Malagasy}
\def\langnames@langs@glot@brs{Baras}
\def\langnames@langs@glot@brp{Barapasi}
\def\langnames@langs@glot@bmz{Baramu}
\def\langnames@langs@glot@bpb{Barbacoas}
\def\langnames@langs@glot@gry{Barclayville Grebo}
\def\langnames@langs@glot@bva{Barain}
\def\langnames@langs@glot@bxo{Barikanchi}
\def\langnames@langs@glot@bch{Bariai}
\def\langnames@langs@glot@bjc{Bariji}
\def\langnames@langs@glot@jbk{Barikewa}
\def\langnames@langs@glot@bbi{Barombi}
\def\langnames@langs@glot@bjk{Barok}
\def\langnames@langs@glot@bpt{Barrow Point}
\def\langnames@langs@glot@tbn{Barro Negro Tunebo}
\def\langnames@langs@glot@bjz{Baruga}
\def\langnames@langs@glot@bwg{Barwe}
\def\langnames@langs@glot@bjf{Barzani Jewish Neo-Aramaic}
\def\langnames@langs@glot@bsl{Basa-Gumna}
\def\langnames@langs@glot@buj{Basa-Gurmana}
\def\langnames@langs@glot@bzw{Basa (Nigeria)}
\def\langnames@langs@glot@bdb{Basap}
\def\langnames@langs@glot@byq{Basay}
\def\langnames@langs@glot@bsg{Bashkardi}
\def\langnames@langs@glot@bst{Basketo}
\def\langnames@langs@glot@bsr{Bassa-Kontagora}
\def\langnames@langs@glot@bsi{Bassossi}
\def\langnames@langs@glot@bnm{Batanga}
\def\langnames@langs@glot@bts{Batak Simalungun}
\def\langnames@langs@glot@akb{Batak Angkola}
\def\langnames@langs@glot@btm{Batak Mandailing}
\def\langnames@langs@glot@btd{Batak Dairi}
\def\langnames@langs@glot@ayt{Bataan Ayta}
\def\langnames@langs@glot@bta{Bata}
\def\langnames@langs@glot@btv{Bateri}
\def\langnames@langs@glot@btq{Batek}
\def\langnames@langs@glot@btc{Bati (Cameroon)}
\def\langnames@langs@glot@bvt{Bati (Indonesia)}
\def\langnames@langs@glot@btu{Batu}
\def\langnames@langs@glot@bay{Batuley}
\def\langnames@langs@glot@zbt{Batui}
\def\langnames@langs@glot@sne{Bau-Jagoi Bidayuh}
\def\langnames@langs@glot@bsf{Bauchi}
\def\langnames@langs@glot@bge{Bauria}
\def\langnames@langs@glot@bxa{Bauro}
\def\langnames@langs@glot@bwk{Bauwaki}
\def\langnames@langs@glot@bjy{Bayali}
\def\langnames@langs@glot@bvy{Baybayanon}
\def\langnames@langs@glot@byg{Baygo}
\def\langnames@langs@glot@mkq{Bay Miwok}
\def\langnames@langs@glot@bda{Kugere-Kuxinge}
\def\langnames@langs@glot@byl{Bayono}
\def\langnames@langs@glot@bfr{Bazigar}
\def\langnames@langs@glot@beo{Beami}
\def\langnames@langs@glot@bea{Beaver}
\def\langnames@langs@glot@bfp{Beba}
\def\langnames@langs@glot@beb{Bebele}
\def\langnames@langs@glot@bzv{Bebe}
\def\langnames@langs@glot@bek{Bebeli}
\def\langnames@langs@glot@bxp{Bebil}
\def\langnames@langs@glot@tnr{Bedik}
\def\langnames@langs@glot@bjv{Nangnda}
\def\langnames@langs@glot@bed{Bedoanas}
\def\langnames@langs@glot@bkf{Beeke}
\def\langnames@langs@glot@bxq{Beele}
\def\langnames@langs@glot@bnz{Beezen}
\def\langnames@langs@glot@bby{Menchum}
\def\langnames@langs@glot@bqv{Begbere-Ejar}
\def\langnames@langs@glot@bei{Riuk Bekati'}
\def\langnames@langs@glot@bkv{Bekwarra}
\def\langnames@langs@glot@bkw{Bekwil}
\def\langnames@langs@glot@bvi{Belanda Viri}
\def\langnames@langs@glot@bxb{Belanda Bor}
\def\langnames@langs@glot@beg{Lemeting}
\def\langnames@langs@glot@blm{Beli (South Sudan)}
\def\langnames@langs@glot@bey{Beli (Papua New Guinea)}
\def\langnames@langs@glot@bzj{Belize Kriol English}
\def\langnames@langs@glot@brw{Bellari}
\def\langnames@langs@glot@glb{Belneng}
\def\langnames@langs@glot@bmb{Bembe}
\def\langnames@langs@glot@yun{Bena (Nigeria)}
\def\langnames@langs@glot@bez{Bena (Tanzania)}
\def\langnames@langs@glot@bdp{Bende}
\def\langnames@langs@glot@bct{Bendi}
\def\langnames@langs@glot@bgy{Benggoi}
\def\langnames@langs@glot@bnu{Bentong}
\def\langnames@langs@glot@dbt{Ben Tey Dogon}
\def\langnames@langs@glot@byd{Benyadu'}
\def\langnames@langs@glot@bie{Bepour}
\def\langnames@langs@glot@bxv{Berakou}
\def\langnames@langs@glot@bve{Berau Malay}
\def\langnames@langs@glot@bit{Berinomo}
\def\langnames@langs@glot@byt{Berti}
\def\langnames@langs@glot@bes{Besme}
\def\langnames@langs@glot@bep{Besoa}
\def\langnames@langs@glot@bfe{Betaf}
\def\langnames@langs@glot@byf{Bete (Yukubenic)}
\def\langnames@langs@glot@btt{Bete-Bendi}
\def\langnames@langs@glot@eot{Beti (Côte d'Ivoire)}
\def\langnames@langs@glot@bhd{Bhadrawahi}
\def\langnames@langs@glot@bha{Bharia}
\def\langnames@langs@glot@bht{Bhattiyali}
\def\langnames@langs@glot@bgw{Bhatri}
\def\langnames@langs@glot@bhe{Bhaya}
\def\langnames@langs@glot@bhy{Bhele}
\def\langnames@langs@glot@bhi{Bhilali}
\def\langnames@langs@glot@nes{Bhoti Kinnauri}
\def\langnames@langs@glot@bhu{Bhunjia}
\def\langnames@langs@glot@bdf{Biage}
\def\langnames@langs@glot@beh{Biali}
\def\langnames@langs@glot@bpv{Bian Marind}
\def\langnames@langs@glot@big{Biangai}
\def\langnames@langs@glot@byk{Shidong Biao}
\def\langnames@langs@glot@bje{Biao-Jiao Mien}
\def\langnames@langs@glot@bmt{Biao Mon}
\def\langnames@langs@glot@bym{Bidyara}
\def\langnames@langs@glot@bjg{Kanyaki-Kagbaaga-Kajoko Bidyogo}
\def\langnames@langs@glot@bmc{Biem}
\def\langnames@langs@glot@bnk{Bierebo}
\def\langnames@langs@glot@brj{Bieria}
\def\langnames@langs@glot@biu{Biete}
\def\langnames@langs@glot@xbe{Bigambal}
\def\langnames@langs@glot@bhc{Biga}
\def\langnames@langs@glot@ibh{Bih}
\def\langnames@langs@glot@jbm{Bijim}
\def\langnames@langs@glot@bix{Bijori}
\def\langnames@langs@glot@byb{Bikya}
\def\langnames@langs@glot@kfs{Bilaspuri}
\def\langnames@langs@glot@bql{Karen}
\def\langnames@langs@glot@brz{Bilibil}
\def\langnames@langs@glot@bpz{Bilba}
\def\langnames@langs@glot@bil{Bile}
\def\langnames@langs@glot@bms{Bilma Kanuri}
\def\langnames@langs@glot@bxf{Bilur}
\def\langnames@langs@glot@bhl{Bimin}
\def\langnames@langs@glot@byj{Bina (Nigeria)}
\def\langnames@langs@glot@bmn{Bina (Papua New Guinea)}
\def\langnames@langs@glot@bxz{Binahari-Neme}
\def\langnames@langs@glot@bon{Bine}
\def\langnames@langs@glot@bpj{Binji}
\def\langnames@langs@glot@itb{Binongan Itneg}
\def\langnames@langs@glot@bne{Bintauna}
\def\langnames@langs@glot@bny{Bintulu}
\def\langnames@langs@glot@biq{Bipi}
\def\langnames@langs@glot@bxe{Ongota}
\def\langnames@langs@glot@brr{Birao}
\def\langnames@langs@glot@btf{Birgit}
\def\langnames@langs@glot@biy{Birhor}
\def\langnames@langs@glot@bqq{Biritai}
\def\langnames@langs@glot@brk{Birked}
\def\langnames@langs@glot@brl{Birwa}
\def\langnames@langs@glot@ije{Biseni}
\def\langnames@langs@glot@bpy{Bishnupriya Manipuri}
\def\langnames@langs@glot@bwh{Bishuo}
\def\langnames@langs@glot@bnw{Bisis}
\def\langnames@langs@glot@bir{Bisorio}
\def\langnames@langs@glot@bzi{Bisu}
\def\langnames@langs@glot@brt{Bitare}
\def\langnames@langs@glot@bgk{Bit}
\def\langnames@langs@glot@mcc{Bitur}
\def\langnames@langs@glot@bwm{Biwat}
\def\langnames@langs@glot@byo{Biyo}
\def\langnames@langs@glot@bpm{Biyom}
\def\langnames@langs@glot@blp{Blablanga}
\def\langnames@langs@glot@bfh{Mblafe-Ránmo}
\def\langnames@langs@glot@beu{Blagar}
\def\langnames@langs@glot@blr{Blang}
\def\langnames@langs@glot@zbl{Blissymbols}
\def\langnames@langs@glot@bzn{Boano (Maluku)}
\def\langnames@langs@glot@bzl{Boano (Sulawesi)}
\def\langnames@langs@glot@bty{Bobot}
\def\langnames@langs@glot@bgb{Bobongko}
\def\langnames@langs@glot@bdv{Bodo Parja}
\def\langnames@langs@glot@boy{Bodo (Central African Republic)}
\def\langnames@langs@glot@bff{Bofi}
\def\langnames@langs@glot@boq{Bogaya}
\def\langnames@langs@glot@bvw{Boga}
\def\langnames@langs@glot@bux{Boghom}
\def\langnames@langs@glot@bqu{Boguru}
\def\langnames@langs@glot@bhn{Gardabani Bohtan Neo-Aramaic}
\def\langnames@langs@glot@ybk{Bokha}
\def\langnames@langs@glot@bdt{Bokoto}
\def\langnames@langs@glot@bkp{Boko (Democratic Republic of Congo)}
\def\langnames@langs@glot@bus{Bokobaru}
\def\langnames@langs@glot@bky{Bokyi}
\def\langnames@langs@glot@bnp{Bola}
\def\langnames@langs@glot@bld{Bolango}
\def\langnames@langs@glot@xbo{Bolgarian}
\def\langnames@langs@glot@bvo{Bolgo}
\def\langnames@langs@glot@bvl{Bolivian Sign Language}
\def\langnames@langs@glot@smk{Bolinao}
\def\langnames@langs@glot@blv{Kibala}
\def\langnames@langs@glot@bkt{Boloki}
\def\langnames@langs@glot@bzm{Bolondo}
\def\langnames@langs@glot@bof{Bolon}
\def\langnames@langs@glot@blj{Bolongan}
\def\langnames@langs@glot@ply{Bolyu}
\def\langnames@langs@glot@boh{Boma Yumu}
\def\langnames@langs@glot@bml{Bomboli-Bozaba}
\def\langnames@langs@glot@bws{Bomboma}
\def\langnames@langs@glot@zmx{Bomitaba}
\def\langnames@langs@glot@bmf{Bom-Kim}
\def\langnames@langs@glot@bmq{Bomu}
\def\langnames@langs@glot@bmw{Bomwali}
\def\langnames@langs@glot@kzc{Bondoukou Kulango}
\def\langnames@langs@glot@bou{Bondei}
\def\langnames@langs@glot@dbu{Najamba-Kindige}
\def\langnames@langs@glot@bna{Bonerate}
\def\langnames@langs@glot@bnv{Bonerif}
\def\langnames@langs@glot@glc{Bon Gula}
\def\langnames@langs@glot@bui{Bongili}
\def\langnames@langs@glot@bpg{Bonggo}
\def\langnames@langs@glot@bok{Impfondo}
\def\langnames@langs@glot@bvg{Bonkeng}
\def\langnames@langs@glot@bop{Bonkiman}
\def\langnames@langs@glot@bnb{Bookan}
\def\langnames@langs@glot@bnl{Boon}
\def\langnames@langs@glot@bvf{Boor}
\def\langnames@langs@glot@bpw{Bo (Papua New Guinea)}
\def\langnames@langs@glot@gai{Borei}
\def\langnames@langs@glot@fue{Borgu Fulfulde}
\def\langnames@langs@glot@ksr{Borong}
\def\langnames@langs@glot@xxb{Boro}
\def\langnames@langs@glot@mae{Bo-Rukul}
\def\langnames@langs@glot@bwf{Boselewa}
\def\langnames@langs@glot@bqs{Bosngun}
\def\langnames@langs@glot@bmj{Bote}
\def\langnames@langs@glot@bph{Botlikh}
\def\langnames@langs@glot@sbl{Botolan Sambal}
\def\langnames@langs@glot@nku{Bouna Kulango}
\def\langnames@langs@glot@mux{Bo-Ung}
\def\langnames@langs@glot@suo{Bouni-Bobe}
\def\langnames@langs@glot@kxr{Manus Koro}
\def\langnames@langs@glot@aof{Bragat}
\def\langnames@langs@glot@bra{Braj}
\def\langnames@langs@glot@kvl{Brek Karen}
\def\langnames@langs@glot@buq{Barem}
\def\langnames@langs@glot@brq{Breri}
\def\langnames@langs@glot@rib{Bribri Sign Language}
\def\langnames@langs@glot@bzt{Brithenig}
\def\langnames@langs@glot@sgt{Brokpake}
\def\langnames@langs@glot@bro{Dur Brokkat}
\def\langnames@langs@glot@bpl{Broome Pearling Lugger Pidgin}
\def\langnames@langs@glot@plw{Brooke's Point Palawano}
\def\langnames@langs@glot@kxd{Brunei}
\def\langnames@langs@glot@bsb{Brunei Bisaya-Dusun}
\def\langnames@langs@glot@rnb{Brunca Sign Language}
\def\langnames@langs@glot@bub{Bua}
\def\langnames@langs@glot@cbl{Bualkhaw Chin}
\def\langnames@langs@glot@box{Buamu}
\def\langnames@langs@glot@buw{Bubi}
\def\langnames@langs@glot@stt{Budeh Stieng}
\def\langnames@langs@glot@btp{Budibud}
\def\langnames@langs@glot@bdx{Budong-Budong}
\def\langnames@langs@glot@bja{Budza}
\def\langnames@langs@glot@bbh{Bugan}
\def\langnames@langs@glot@buk{Bugawac}
\def\langnames@langs@glot@bgt{Bughotu}
\def\langnames@langs@glot@bku{Buhid}
\def\langnames@langs@glot@bxh{Buhutu}
\def\langnames@langs@glot@byh{Bujhyal}
\def\langnames@langs@glot@bvk{Bukat}
\def\langnames@langs@glot@bhh{Bukharic}
\def\langnames@langs@glot@bvu{Bukit Malay}
\def\langnames@langs@glot@bkn{Bukitan}
\def\langnames@langs@glot@tkb{Buksa}
\def\langnames@langs@glot@buz{Bukwen}
\def\langnames@langs@glot@bqn{Bulgarian Sign Language}
\def\langnames@langs@glot@bmp{Bulgebi}
\def\langnames@langs@glot@buy{Bullom So}
\def\langnames@langs@glot@sti{Bulo Stieng}
\def\langnames@langs@glot@bjl{Bulu (Papua New Guinea)}
\def\langnames@langs@glot@byp{Bumaji}
\def\langnames@langs@glot@aon{Bumbita Arapesh}
\def\langnames@langs@glot@bmv{Bum}
\def\langnames@langs@glot@kjz{Bumthangkha}
\def\langnames@langs@glot@bwx{Bu-Nao Bunu}
\def\langnames@langs@glot@bdd{Bunama}
\def\langnames@langs@glot@bvn{Buna}
\def\langnames@langs@glot@bfn{Bunak}
\def\langnames@langs@glot@bns{Bundeli}
\def\langnames@langs@glot@bqd{Bung}
\def\langnames@langs@glot@xbg{Bunganditj}
\def\langnames@langs@glot@wun{Bungu}
\def\langnames@langs@glot@bkz{Bungku}
\def\langnames@langs@glot@but{Bungain}
\def\langnames@langs@glot@buv{Bun}
\def\langnames@langs@glot@dgb{Bunoge Dogon}
\def\langnames@langs@glot@bnn{Bunun}
\def\langnames@langs@glot@blf{Buol}
\def\langnames@langs@glot@bys{Burak}
\def\langnames@langs@glot@bti{Burate}
\def\langnames@langs@glot@bxn{Burduna}
\def\langnames@langs@glot@bvh{Bure}
\def\langnames@langs@glot@pyx{Burma Pyu}
\def\langnames@langs@glot@vrt{Burmbar}
\def\langnames@langs@glot@bzu{Burmeso}
\def\langnames@langs@glot@bqw{Buru-Angwe}
\def\langnames@langs@glot@bdi{Northern Burun}
\def\langnames@langs@glot@bqr{Burusu}
\def\langnames@langs@glot@aip{Burumakok}
\def\langnames@langs@glot@asi{Buruwai}
\def\langnames@langs@glot@bry{Burui}
\def\langnames@langs@glot@bxs{Busam}
\def\langnames@langs@glot@bsm{Busami}
\def\langnames@langs@glot@bfg{Busang Kayan}
\def\langnames@langs@glot@buc{Kibosy Kiantalaotsy-Majunga}
\def\langnames@langs@glot@bup{Busoa}
\def\langnames@langs@glot@dox{Bussa}
\def\langnames@langs@glot@bju{Busuu}
\def\langnames@langs@glot@kyb{Butbut Kalinga}
\def\langnames@langs@glot@bnr{Farafi}
\def\langnames@langs@glot@btw{Butuanon}
\def\langnames@langs@glot@jid{Bu}
\def\langnames@langs@glot@bhs{Buwal}
\def\langnames@langs@glot@jiy{Buyuan Jinuo}
\def\langnames@langs@glot@byi{Buyu}
\def\langnames@langs@glot@bww{Bwa}
\def\langnames@langs@glot@bwd{Bwaidoka}
\def\langnames@langs@glot@tte{Bwanabwana}
\def\langnames@langs@glot@bwa{Bwatoo}
\def\langnames@langs@glot@bwl{Bwela}
\def\langnames@langs@glot@bwc{Bwile}
\def\langnames@langs@glot@bwz{Bwisi}
\def\langnames@langs@glot@mkk{Byep-Besep}
\def\langnames@langs@glot@msq{Caac}
\def\langnames@langs@glot@cbb{Cabiyarí}
\def\langnames@langs@glot@ccr{Cacaopera}
\def\langnames@langs@glot@miu{Cacaloxtepec Mixtec}
\def\langnames@langs@glot@roc{Cacgia Roglai}
\def\langnames@langs@glot@ccd{Cafundo}
\def\langnames@langs@glot@cah{Cahuarano}
\def\langnames@langs@glot@qvl{Cajatambo North Lima Quechua}
\def\langnames@langs@glot@zad{Cajonos Zapotec}
\def\langnames@langs@glot@frc{Cajun French}
\def\langnames@langs@glot@ckx{Caka}
\def\langnames@langs@glot@ckz{Cakchiquel-Quiché Mixed Language}
\def\langnames@langs@glot@cky{Cakfem-Mushere-Jibyal}
\def\langnames@langs@glot@tbk{Calamian Tagbanwa}
\def\langnames@langs@glot@qud{Calderón Highland Quichua}
\def\langnames@langs@glot@caw{Callawalla}
\def\langnames@langs@glot@rmq{Caló}
\def\langnames@langs@glot@clu{Caluyanun}
\def\langnames@langs@glot@abd{Camarines Norte Agta}
\def\langnames@langs@glot@csx{Cambodian Sign Language}
\def\langnames@langs@glot@mcu{Donga Mambila}
\def\langnames@langs@glot@wes{Cameroon Pidgin}
\def\langnames@langs@glot@cml{Campalagian}
\def\langnames@langs@glot@cmt{Camtho}
\def\langnames@langs@glot@xcc{Camunic}
\def\langnames@langs@glot@qxr{Cañar-Azuay-South Chimborazo Highland Quichua}
\def\langnames@langs@glot@caz{Canichana}
\def\langnames@langs@glot@mlc{Cao Lan}
\def\langnames@langs@glot@cov{Cao Miao}
\def\langnames@langs@glot@cps{Capiznon}
\def\langnames@langs@glot@cpg{Cappadocian Greek}
\def\langnames@langs@glot@cot{Caquinte}
\def\langnames@langs@glot@cby{Carabayo}
\def\langnames@langs@glot@cfd{Cara}
\def\langnames@langs@glot@crf{Caramanta}
\def\langnames@langs@glot@xcr{Carian}
\def\langnames@langs@glot@hns{Caribbean Hindustani}
\def\langnames@langs@glot@jvn{Caribbean Javanese}
\def\langnames@langs@glot@crr{Carolina Algonquian}
\def\langnames@langs@glot@rmc{Central Romani}
\def\langnames@langs@glot@asc{Casuarina Coast Asmat}
\def\langnames@langs@glot@csc{Catalan Sign Language}
\def\langnames@langs@glot@xcy{Cayuse}
\def\langnames@langs@glot@xce{Celtiberian}
\def\langnames@langs@glot@cen{Cen}
\def\langnames@langs@glot@hmm{Central Mashan Hmong}
\def\langnames@langs@glot@cmo{Central Mnong}
\def\langnames@langs@glot@zch{Central Hongshuihe Zhuang}
\def\langnames@langs@glot@hmc{Central Huishui Hmong}
\def\langnames@langs@glot@fuq{Central-Eastern Niger Fulfulde}
\def\langnames@langs@glot@grv{Central Grebo}
\def\langnames@langs@glot@cet{Jalaa}
\def\langnames@langs@glot@pse{South Barisan Malay}
\def\langnames@langs@glot@mwo{Central Maewo}
\def\langnames@langs@glot@mxz{Central Masela}
\def\langnames@langs@glot@syb{Central Subanen}
\def\langnames@langs@glot@tgt{Central Tagbanwa}
\def\langnames@langs@glot@plc{Central Palawano}
\def\langnames@langs@glot@sml{Central Sama}
\def\langnames@langs@glot@zbc{Central Berawan}
\def\langnames@langs@glot@dtp{Kadazan Dusun}
\def\langnames@langs@glot@awu{Central Awyu}
\def\langnames@langs@glot@ncx{Central Puebla Nahuatl}
\def\langnames@langs@glot@nch{Central Huasteca Nahuatl}
\def\langnames@langs@glot@ojc{Central Ojibwa}
\def\langnames@langs@glot@pbs{Central Pame}
\def\langnames@langs@glot@quk{Chachapoyas Quechua}
\def\langnames@langs@glot@cds{Chadian Sign Language}
\def\langnames@langs@glot@cdy{Chadong}
\def\langnames@langs@glot@chg{Chagatai}
\def\langnames@langs@glot@ciy{Chaima}
\def\langnames@langs@glot@ccp{Chakma}
\def\langnames@langs@glot@ckh{Chak}
\def\langnames@langs@glot@cli{Chakali}
\def\langnames@langs@glot@tgf{Chalikha}
\def\langnames@langs@glot@cll{Chala}
\def\langnames@langs@glot@cdh{Chambeali}
\def\langnames@langs@glot@ceg{Chamacoco}
\def\langnames@langs@glot@ccc{Chamicuro}
\def\langnames@langs@glot@cna{Changthang}
\def\langnames@langs@glot@cga{Changriwa}
\def\langnames@langs@glot@cra{Chara}
\def\langnames@langs@glot@crv{Chaura}
\def\langnames@langs@glot@xtb{Chazumba Mixtec}
\def\langnames@langs@glot@ruk{Che}
\def\langnames@langs@glot@cde{Chenchu}
\def\langnames@langs@glot@cjn{Chenapian}
\def\langnames@langs@glot@cnu{Western Algerian Berber}
\def\langnames@langs@glot@ycp{Chepya}
\def\langnames@langs@glot@cpn{Cherepon}
\def\langnames@langs@glot@ych{Chesu}
\def\langnames@langs@glot@cwg{Chewong}
\def\langnames@langs@glot@hne{Chhattisgarhi}
\def\langnames@langs@glot@ctn{Chintang}
\def\langnames@langs@glot@cur{Chhulung}
\def\langnames@langs@glot@csd{Chiangmai Sign Language}
\def\langnames@langs@glot@cip{Chiapanec}
\def\langnames@langs@glot@zpv{Chichicapan Zapotec}
\def\langnames@langs@glot@mii{Chigmecatitlán Mixtec}
\def\langnames@langs@glot@csg{Chilean Sign Language}
\def\langnames@langs@glot@clh{Chilisso}
\def\langnames@langs@glot@clc{Chilcotin-Nicola}
\def\langnames@langs@glot@csa{Chiltepec Chinantec}
\def\langnames@langs@glot@cpi{Chinese Pidgin English}
\def\langnames@langs@glot@chn{Creolized Grand Ronde Chinook Jargon}
\def\langnames@langs@glot@cih{Chinali}
\def\langnames@langs@glot@bxu{China Buriat}
\def\langnames@langs@glot@cnb{Chinbon Chin}
\def\langnames@langs@glot@qxc{Chincha Quechua}
\def\langnames@langs@glot@cdf{Chiru}
\def\langnames@langs@glot@nhd{Chiripá}
\def\langnames@langs@glot@the{Chitwania Tharu}
\def\langnames@langs@glot@cik{Chhitkul-Rakchham}
\def\langnames@langs@glot@zpc{Choapan Zapotec}
\def\langnames@langs@glot@cgk{Chocangacakha}
\def\langnames@langs@glot@cdi{Chodri}
\def\langnames@langs@glot@nri{Chokri Naga}
\def\langnames@langs@glot@cjk{Chokwe}
\def\langnames@langs@glot@cda{Choni}
\def\langnames@langs@glot@coh{Chonyi-Dzihana-Kauma}
\def\langnames@langs@glot@cce{Chopi}
\def\langnames@langs@glot@nct{Chothe}
\def\langnames@langs@glot@cvg{Duhumbi}
\def\langnames@langs@glot@cuw{Chukwa}
\def\langnames@langs@glot@cuh{Chuka}
\def\langnames@langs@glot@chu{Church Slavic}
\def\langnames@langs@glot@cdj{Churahi}
\def\langnames@langs@glot@scb{Chut}
\def\langnames@langs@glot@xcv{Chuvantsy}
\def\langnames@langs@glot@chw{Chuwabu}
\def\langnames@langs@glot@cia{Cia-Cia}
\def\langnames@langs@glot@ckl{Cibak}
\def\langnames@langs@glot@awc{Cicipu}
\def\langnames@langs@glot@cib{Ci Gbe}
\def\langnames@langs@glot@cim{Cimbrian}
\def\langnames@langs@glot@mkx{Cinamiguin Manobo}
\def\langnames@langs@glot@cdr{Yara}
\def\langnames@langs@glot@cie{Cineni}
\def\langnames@langs@glot@cin{Cinta Larga}
\def\langnames@langs@glot@xcg{Cisalpine Gaulish}
\def\langnames@langs@glot@asg{Western-Kambari-Cishingini}
\def\langnames@langs@glot@txt{Citak}
\def\langnames@langs@glot@tgd{Ciwogai}
\def\langnames@langs@glot@xcl{Classical-Middle Armenian}
\def\langnames@langs@glot@nci{Classical Nahuatl}
\def\langnames@langs@glot@qwc{Classical Quechua}
\def\langnames@langs@glot@syc{Classical Syriac}
\def\langnames@langs@glot@myz{Classical Mandaic}
\def\langnames@langs@glot@xct{Classical Tibetan}
\def\langnames@langs@glot@dri{C'lela}
\def\langnames@langs@glot@naz{Coatepec Nahuatl}
\def\langnames@langs@glot@zps{Coatlán Zapotec}
\def\langnames@langs@glot@zca{Coatecas Altas Zapotec}
\def\langnames@langs@glot@coj{Cochimi}
\def\langnames@langs@glot@coa{Cocos Islands Malay}
\def\langnames@langs@glot@liw{Col}
\def\langnames@langs@glot@csn{Colombian Sign Language}
\def\langnames@langs@glot@gct{Colonia Tovar German}
\def\langnames@langs@glot@cfg{Como Karim}
\def\langnames@langs@glot@swc{Congo Swahili}
\def\langnames@langs@glot@cnc{Côông}
\def\langnames@langs@glot@coq{Coquille}
\def\langnames@langs@glot@cry{Kyoli}
\def\langnames@langs@glot@qwa{Corongo Ancash Quechua}
\def\langnames@langs@glot@xxr{Koropó}
\def\langnames@langs@glot@cos{Corsican}
\def\langnames@langs@glot@csr{Costa Rican Sign Language}
\def\langnames@langs@glot@mta{Cotabato Manobo}
\def\langnames@langs@glot@xcn{Cotoname}
\def\langnames@langs@glot@cow{Cowlitz}
\def\langnames@langs@glot@toc{Coyutla Totonac}
\def\langnames@langs@glot@gyn{Guyanese Creole English}
\def\langnames@langs@glot@csq{Croatian Sign Language}
\def\langnames@langs@glot@mfn{Cross River Mbembe}
\def\langnames@langs@glot@crz{Cruzeño}
\def\langnames@langs@glot@csf{Cuba Sign Language}
\def\langnames@langs@glot@cbq{Cuba}
\def\langnames@langs@glot@cuo{Cumanagoto}
\def\langnames@langs@glot@xlu{Cuneiform Luwian}
\def\langnames@langs@glot@cnq{Chung}
\def\langnames@langs@glot@cuq{Cun}
\def\langnames@langs@glot@ccl{Cutchi-Swahili}
\def\langnames@langs@glot@cuv{Cuvok}
\def\langnames@langs@glot@xtu{Cuyamecalco Mixtec}
\def\langnames@langs@glot@cyo{Cuyonon}
\def\langnames@langs@glot@bwy{Cwi Bwamu}
\def\langnames@langs@glot@cse{Czech Sign Language}
\def\langnames@langs@glot@dao{Daai Chin}
\def\langnames@langs@glot@lni{Daantanai'}
\def\langnames@langs@glot@dtn{Daats'iin}
\def\langnames@langs@glot@dbr{Dabarre}
\def\langnames@langs@glot@dbe{Dabe}
\def\langnames@langs@glot@xdc{Dacian}
\def\langnames@langs@glot@dbd{Dadiya}
\def\langnames@langs@glot@dgd{Dagaari Dioula}
\def\langnames@langs@glot@dgk{Dagba}
\def\langnames@langs@glot@dec{Dagik}
\def\langnames@langs@glot@dgn{Dagoman}
\def\langnames@langs@glot@dlk{Dahalik}
\def\langnames@langs@glot@das{Daho-Doo}
\def\langnames@langs@glot@dij{Dai}
\def\langnames@langs@glot@drb{Dair}
\def\langnames@langs@glot@zhd{Dai Zhuang}
\def\langnames@langs@glot@bpa{Dakaka}
\def\langnames@langs@glot@dkk{Dakka}
\def\langnames@langs@glot@dka{Dakpakha}
\def\langnames@langs@glot@qer{Dalecarlian}
\def\langnames@langs@glot@dlm{Dalmatian}
\def\langnames@langs@glot@dmm{Dama (Cameroon)}
\def\langnames@langs@glot@dam{Damakawa}
\def\langnames@langs@glot@uhn{Damal}
\def\langnames@langs@glot@idb{Daman-Diu Portuguese}
\def\langnames@langs@glot@dac{Dambi}
\def\langnames@langs@glot@dml{Dameli}
\def\langnames@langs@glot@dms{Dampelas}
\def\langnames@langs@glot@dnu{Danau}
\def\langnames@langs@glot@dnr{Danaru}
\def\langnames@langs@glot@daq{Dandami Maria}
\def\langnames@langs@glot@thl{Dangaura Tharu}
\def\langnames@langs@glot@dsl{Danish Sign Language}
\def\langnames@langs@glot@daf{Dan}
\def\langnames@langs@glot@aso{Dano}
\def\langnames@langs@glot@gku{Danster !Ui}
\def\langnames@langs@glot@dnd{Daonda}
\def\langnames@langs@glot@daz{Dao}
\def\langnames@langs@glot@djc{Dar Daju Daju}
\def\langnames@langs@glot@dln{Darlong}
\def\langnames@langs@glot@dro{Daro-Matu Melanau}
\def\langnames@langs@glot@dot{Dass}
\def\langnames@langs@glot@daw{Davawenyo}
\def\langnames@langs@glot@dww{Dawawa}
\def\langnames@langs@glot@ddw{Dawera-Daweloor}
\def\langnames@langs@glot@dax{Dayi}
\def\langnames@langs@glot@dzg{Dazaga}
\def\langnames@langs@glot@dzd{Daza}
\def\langnames@langs@glot@ded{Dedua}
\def\langnames@langs@glot@gbh{Defi Gbe}
\def\langnames@langs@glot@dge{Degenan}
\def\langnames@langs@glot@mzw{Deg}
\def\langnames@langs@glot@deh{Dehwari}
\def\langnames@langs@glot@dek{Dek}
\def\langnames@langs@glot@row{Dela-Oenale}
\def\langnames@langs@glot@ntr{Delo}
\def\langnames@langs@glot@dmx{Dema}
\def\langnames@langs@glot@dei{Demisa}
\def\langnames@langs@glot@dem{Dem}
\def\langnames@langs@glot@dmy{Demta}
\def\langnames@langs@glot@deq{Dendi (Central African Republic)}
\def\langnames@langs@glot@ddn{Dendi (Benin)}
\def\langnames@langs@glot@dez{Dengese}
\def\langnames@langs@glot@dnk{Dengka}
\def\langnames@langs@glot@dbb{Deno}
\def\langnames@langs@glot@anv{Denya}
\def\langnames@langs@glot@dee{Dewoin}
\def\langnames@langs@glot@def{Dezfuli-Shushtari}
\def\langnames@langs@glot@dgh{Dghwede}
\def\langnames@langs@glot@dhs{Dhaiso}
\def\langnames@langs@glot@dhn{Dhanki}
\def\langnames@langs@glot@dwz{Dewas-Done Danuwar}
\def\langnames@langs@glot@nfa{Dhao}
\def\langnames@langs@glot@mki{Dhatki}
\def\langnames@langs@glot@dho{Dhodia-Kukna}
\def\langnames@langs@glot@adf{Dhofari Arabic}
\def\langnames@langs@glot@ddr{Dhudhuroa}
\def\langnames@langs@glot@dhd{Dhundari}
\def\langnames@langs@glot@dia{Alu-Sinagen}
\def\langnames@langs@glot@mbd{Dibabawon Manobo}
\def\langnames@langs@glot@dby{Dibiyaso}
\def\langnames@langs@glot@dio{Dibo}
\def\langnames@langs@glot@duy{Dicamay Agta}
\def\langnames@langs@glot@dig{Digo}
\def\langnames@langs@glot@cfa{Dijim-Bwilim}
\def\langnames@langs@glot@dil{Dilling}
\def\langnames@langs@glot@jma{Dima}
\def\langnames@langs@glot@dii{Dimbong}
\def\langnames@langs@glot@dmc{Gavak}
\def\langnames@langs@glot@ddi{Diodio}
\def\langnames@langs@glot@gdl{Dirasha}
\def\langnames@langs@glot@diu{Diriku-Shambyu}
\def\langnames@langs@glot@dir{Dirim}
\def\langnames@langs@glot@dwa{Diri}
\def\langnames@langs@glot@dsi{Dissa-Canton Mufa}
\def\langnames@langs@glot@tbz{Ditammari}
\def\langnames@langs@glot@diy{Diuwe}
\def\langnames@langs@glot@xtd{Diuxi-Tilantongo Mixtec}
\def\langnames@langs@glot@dix{Dixon Reef}
\def\langnames@langs@glot@djf{Djangun}
\def\langnames@langs@glot@djn{Jawoyn}
\def\langnames@langs@glot@djw{Djawi}
\def\langnames@langs@glot@djb{Djinba}
\def\langnames@langs@glot@dze{Djiwarli}
\def\langnames@langs@glot@dob{Dobu}
\def\langnames@langs@glot@doe{Doe}
\def\langnames@langs@glot@dgg{Doga}
\def\langnames@langs@glot@dgx{Doghoro}
\def\langnames@langs@glot@dgs{Dogoso}
\def\langnames@langs@glot@dos{Dogosé}
\def\langnames@langs@glot@dgr{Dogrib}
\def\langnames@langs@glot@dbg{Dogul Dom Dogon}
\def\langnames@langs@glot@dbi{Doka}
\def\langnames@langs@glot@uya{Doko-Uyanga}
\def\langnames@langs@glot@dre{Dolpo}
\def\langnames@langs@glot@dov{Toka-Leya-Dombe}
\def\langnames@langs@glot@doq{Dominican Sign Language}
\def\langnames@langs@glot@doa{Dom}
\def\langnames@langs@glot@doy{Dompo}
\def\langnames@langs@glot@dof{Domu}
\def\langnames@langs@glot@dev{Domung}
\def\langnames@langs@glot@dok{Dondo}
\def\langnames@langs@glot@yik{Dongshanba Lalo}
\def\langnames@langs@glot@doh{Dong}
\def\langnames@langs@glot@ddd{Dongotono}
\def\langnames@langs@glot@dde{Doondo}
\def\langnames@langs@glot@dor{Dori'o}
\def\langnames@langs@glot@kqc{Doromu-Koki}
\def\langnames@langs@glot@doz{Dorze}
\def\langnames@langs@glot@dol{Doso}
\def\langnames@langs@glot@dty{Dotyali}
\def\langnames@langs@glot@dup{Duano}
\def\langnames@langs@glot@dva{Duau}
\def\langnames@langs@glot@dub{Dubli}
\def\langnames@langs@glot@dmu{Dubu}
\def\langnames@langs@glot@duk{Duduela}
\def\langnames@langs@glot@ndu{Dugun}
\def\langnames@langs@glot@dbm{Duguri}
\def\langnames@langs@glot@dme{Dugwor}
\def\langnames@langs@glot@kbz{Duhwa}
\def\langnames@langs@glot@nke{Duke}
\def\langnames@langs@glot@dbo{Dulbu}
\def\langnames@langs@glot@duz{Duli-Gewe}
\def\langnames@langs@glot@dmv{Dumpas}
\def\langnames@langs@glot@wtf{Dumpu}
\def\langnames@langs@glot@dui{Dumun}
\def\langnames@langs@glot@duh{Dungra Bhil}
\def\langnames@langs@glot@raa{Dungmali}
\def\langnames@langs@glot@dng{Dungan}
\def\langnames@langs@glot@dbv{Dungu}
\def\langnames@langs@glot@drq{Dura}
\def\langnames@langs@glot@mvp{Duri}
\def\langnames@langs@glot@dbn{Duriankere}
\def\langnames@langs@glot@dug{Duruma}
\def\langnames@langs@glot@dsn{Dusner}
\def\langnames@langs@glot@duw{Dusun Witu}
\def\langnames@langs@glot@duq{Dusun Malang}
\def\langnames@langs@glot@dun{Dusun Deyah}
\def\langnames@langs@glot@dws{Dutton World Speedwords}
\def\langnames@langs@glot@dux{Duungooma}
\def\langnames@langs@glot@dae{Duupa}
\def\langnames@langs@glot@duv{Duvle}
\def\langnames@langs@glot@dbp{Duwai}
\def\langnames@langs@glot@gve{Duwet}
\def\langnames@langs@glot@nnu{Dwang}
\def\langnames@langs@glot@dyb{Dyaberdyaber}
\def\langnames@langs@glot@dyn{Dyangadi}
\def\langnames@langs@glot@dya{Dyan}
\def\langnames@langs@glot@dyd{Dyugun}
\def\langnames@langs@glot@jen{Dza}
\def\langnames@langs@glot@dzl{Dzalakha}
\def\langnames@langs@glot@dzn{Dzando}
\def\langnames@langs@glot@bpn{Dzao Min}
\def\langnames@langs@glot@add{Dzodinka}
\def\langnames@langs@glot@dzo{Dzongkha}
\def\langnames@langs@glot@dnn{Dzùùngoo}
\def\langnames@langs@glot@ktv{Eastern Katu}
\def\langnames@langs@glot@bgp{Eastern Balochi}
\def\langnames@langs@glot@lwl{Eastern Lawa}
\def\langnames@langs@glot@mng{Eastern Mnong}
\def\langnames@langs@glot@emu{Eastern Muria}
\def\langnames@langs@glot@tge{Eastern Gorkha Tamang}
\def\langnames@langs@glot@nos{Eastern Nisu}
\def\langnames@langs@glot@emq{Eastern Muya}
\def\langnames@langs@glot@kif{Eastern Parbate Kham}
\def\langnames@langs@glot@emg{Eastern Meohang}
\def\langnames@langs@glot@zeh{Eastern Hongshuihe Zhuang}
\def\langnames@langs@glot@hmq{Eastern Qiandong Miao}
\def\langnames@langs@glot@muq{Eastern Xiangxi Miao}
\def\langnames@langs@glot@hme{Eastern Huishui Hmong}
\def\langnames@langs@glot@lma{East Limba}
\def\langnames@langs@glot@gbx{Eastern Xwla Gbe}
\def\langnames@langs@glot@xrb{Eastern Karaboro}
\def\langnames@langs@glot@acp{Eastern Acipa}
\def\langnames@langs@glot@nle{East Nyala}
\def\langnames@langs@glot@kqo{Konobo-Eastern Krahn}
\def\langnames@langs@glot@vme{East Masela}
\def\langnames@langs@glot@tre{East Tarangan}
\def\langnames@langs@glot@dmr{East Damar}
\def\langnames@langs@glot@bnj{Eastern Tawbuid}
\def\langnames@langs@glot@pez{Eastern Penan}
\def\langnames@langs@glot@zbe{East Berawan}
\def\langnames@langs@glot@kjs{East Kewa}
\def\langnames@langs@glot@nhe{Eastern Huasteca Nahuatl}
\def\langnames@langs@glot@ojg{Eastern Ojibwa}
\def\langnames@langs@glot@aaq{Eastern Abenaki}
\def\langnames@langs@glot@qve{Eastern Apurímac Quechua}
\def\langnames@langs@glot@cly{Eastern Highland Chatino}
\def\langnames@langs@glot@avl{Eastern Egyptian Bedawi Arabic}
\def\langnames@langs@glot@sfe{Eastern Subanen}
\def\langnames@langs@glot@azd{Eastern Durango Nahuatl}
\def\langnames@langs@glot@yit{Eastern Lalu}
\def\langnames@langs@glot@cek{Eastern Khumi Chin}
\def\langnames@langs@glot@yol{Irish Anglo-Norman}
\def\langnames@langs@glot@xeb{Eblaite}
\def\langnames@langs@glot@ebr{Ebrié}
\def\langnames@langs@glot@ebg{Ebughu}
\def\langnames@langs@glot@ecs{Ecuadorian Sign Language}
\def\langnames@langs@glot@cbj{Ede Cabe}
\def\langnames@langs@glot@idd{Ede Idaca}
\def\langnames@langs@glot@ijj{Ede Ije}
\def\langnames@langs@glot@ica{Ede Ica}
\def\langnames@langs@glot@nqg{Ede Nago}
\def\langnames@langs@glot@awy{Edera Awyu}
\def\langnames@langs@glot@dbf{Edopi}
\def\langnames@langs@glot@eee{E}
\def\langnames@langs@glot@efa{Efai}
\def\langnames@langs@glot@efe{Efe}
\def\langnames@langs@glot@ofu{Efutop}
\def\langnames@langs@glot@ego{Eggon}
\def\langnames@langs@glot@esl{Egypt Sign Language}
\def\langnames@langs@glot@egy{Egyptian (Ancient)}
\def\langnames@langs@glot@ehu{Ehueun}
\def\langnames@langs@glot@eit{Eitiep}
\def\langnames@langs@glot@eja{Ejamat}
\def\langnames@langs@glot@eka{Ekajuk}
\def\langnames@langs@glot@eki{Eki}
\def\langnames@langs@glot@eke{Ekit}
\def\langnames@langs@glot@ekp{Ekpeye}
\def\langnames@langs@glot@zpp{El Alto Zapotec}
\def\langnames@langs@glot@elx{Elamite}
\def\langnames@langs@glot@elm{Eleme}
\def\langnames@langs@glot@ele{Elepi}
\def\langnames@langs@glot@elh{El Hugeirat}
\def\langnames@langs@glot@ekm{Elip}
\def\langnames@langs@glot@elk{Elkei}
\def\langnames@langs@glot@elo{El Molo}
\def\langnames@langs@glot@zte{Elotepec Zapotec}
\def\langnames@langs@glot@afo{Ajiri}
\def\langnames@langs@glot@elu{Elu}
\def\langnames@langs@glot@xly{Elymian}
\def\langnames@langs@glot@yzg{E'ma Buyang}
\def\langnames@langs@glot@emn{Eman}
\def\langnames@langs@glot@bdc{Emberá-Baudó}
\def\langnames@langs@glot@tdc{Emberá-Tadó}
\def\langnames@langs@glot@ebu{Embu}
\def\langnames@langs@glot@emw{Emplawas}
\def\langnames@langs@glot@enr{Emumu}
\def\langnames@langs@glot@unk{Enawené-Nawé}
\def\langnames@langs@glot@end{Ende}
\def\langnames@langs@glot@enc{En}
\def\langnames@langs@glot@ptt{Enrekang}
\def\langnames@langs@glot@enu{Enu}
\def\langnames@langs@glot@enw{Enwan (Akwa Ibom State)}
\def\langnames@langs@glot@env{Enwan (Edo State)}
\def\langnames@langs@glot@epi{Epie}
\def\langnames@langs@glot@emy{Epigraphic Mayan}
\def\langnames@langs@glot@era{Eravallan}
\def\langnames@langs@glot@kjy{Erave}
\def\langnames@langs@glot@twp{Ere}
\def\langnames@langs@glot@ert{Eritai}
\def\langnames@langs@glot@erw{Erokwanas}
\def\langnames@langs@glot@err{Erre}
\def\langnames@langs@glot@emx{Erromintxela}
\def\langnames@langs@glot@ers{Ersu}
\def\langnames@langs@glot@erh{Eruwa}
\def\langnames@langs@glot@ish{Esan}
\def\langnames@langs@glot@mcq{Ese}
\def\langnames@langs@glot@esh{Eshtehardi}
\def\langnames@langs@glot@ags{Esimbi}
\def\langnames@langs@glot@esy{Eskayan}
\def\langnames@langs@glot@epo{Esperanto}
\def\langnames@langs@glot@ots{Estado de México Otomi}
\def\langnames@langs@glot@eso{Estonian Sign Language}
\def\langnames@langs@glot@esm{Esuma}
\def\langnames@langs@glot@etb{Etebi}
\def\langnames@langs@glot@etx{Eten}
\def\langnames@langs@glot@ecr{Eteocretan}
\def\langnames@langs@glot@ecy{Eteocypriot}
\def\langnames@langs@glot@eth{Ethiopian Sign Language}
\def\langnames@langs@glot@ich{Etkywan}
\def\langnames@langs@glot@eto{Eton-Mengisa}
\def\langnames@langs@glot@etn{Eton (Vanuatu)}
\def\langnames@langs@glot@ett{Etruscan}
\def\langnames@langs@glot@utr{Etulo}
\def\langnames@langs@glot@bzz{Evant}
\def\langnames@langs@glot@gev{Viya}
\def\langnames@langs@glot@nou{Ewage-Notu}
\def\langnames@langs@glot@ext{Extremaduran}
\def\langnames@langs@glot@fab{Annobonese}
\def\langnames@langs@glot@faf{Fagani}
\def\langnames@langs@glot@fif{Faifi}
\def\langnames@langs@glot@azt{Faire Atta}
\def\langnames@langs@glot@faj{Kulsab}
\def\langnames@langs@glot@fai{Faiwol}
\def\langnames@langs@glot@fax{Fala}
\def\langnames@langs@glot@cfm{Falam Chin}
\def\langnames@langs@glot@fli{Fali}
\def\langnames@langs@glot@xfa{Faliscan}
\def\langnames@langs@glot@fam{Fam}
\def\langnames@langs@glot@fng{Fanagalo}
\def\langnames@langs@glot@fan{Fang (Equatorial Guinea)}
\def\langnames@langs@glot@fak{Fang (Cameroon)}
\def\langnames@langs@glot@fni{Fania}
\def\langnames@langs@glot@nsf{Far Northwestern Nisu}
\def\langnames@langs@glot@fmu{Far Western Muria}
\def\langnames@langs@glot@far{Fataleka}
\def\langnames@langs@glot@ddg{Fataluku}
\def\langnames@langs@glot@fau{Fayu}
\def\langnames@langs@glot@agl{Fembe}
\def\langnames@langs@glot@fpe{Pichi}
\def\langnames@langs@glot@fer{Feroge}
\def\langnames@langs@glot@hif{Fiji Hindi}
\def\langnames@langs@glot@fil{Filipino}
\def\langnames@langs@glot@tlp{Filomeno Mata Totonac}
\def\langnames@langs@glot@bkb{Eastern-Southern Bontok}
\def\langnames@langs@glot@fss{Finland-Swedish Sign Language}
\def\langnames@langs@glot@fag{Finongan}
\def\langnames@langs@glot@fip{Fipa}
\def\langnames@langs@glot@fir{Firan}
\def\langnames@langs@glot@fiw{Fiwaga}
\def\langnames@langs@glot@fln{Flinders Island}
\def\langnames@langs@glot@flh{Abawiri}
\def\langnames@langs@glot@fod{Foodo}
\def\langnames@langs@glot@frq{Forak}
\def\langnames@langs@glot@enf{Forest Enets}
\def\langnames@langs@glot@frt{Kiai}
\def\langnames@langs@glot@frp{Arpitan}
\def\langnames@langs@glot@fur{Friulian}
\def\langnames@langs@glot@flr{Fuliiru}
\def\langnames@langs@glot@ula{Fungwa}
\def\langnames@langs@glot@fuy{Fuyug}
\def\langnames@langs@glot@fwe{Fwe}
\def\langnames@langs@glot@fie{Fyer}
\def\langnames@langs@glot@ttb{Gaa}
\def\langnames@langs@glot@gie{Gabogbo}
\def\langnames@langs@glot@gab{Gabri}
\def\langnames@langs@glot@gdg{Ga'dang}
\def\langnames@langs@glot@gdk{Gadang}
\def\langnames@langs@glot@gbk{Gaddi}
\def\langnames@langs@glot@gad{Gaddang}
\def\langnames@langs@glot@gda{Gade Lohar}
\def\langnames@langs@glot@gdh{Gajirrabeng}
\def\langnames@langs@glot@gft{Gafat}
\def\langnames@langs@glot@btg{Gagnoa Bété}
\def\langnames@langs@glot@ggu{Gban}
\def\langnames@langs@glot@gbf{Gaikundi}
\def\langnames@langs@glot@gic{Gail}
\def\langnames@langs@glot@gcn{Gaina}
\def\langnames@langs@glot@xga{Galatian}
\def\langnames@langs@glot@glo{Galambu}
\def\langnames@langs@glot@gar{Galeya}
\def\langnames@langs@glot@gce{Galice}
\def\langnames@langs@glot@sdn{Gallurese Sardinian}
\def\langnames@langs@glot@gap{Gal}
\def\langnames@langs@glot@gal{Galoli-Talur}
\def\langnames@langs@glot@kgj{Gamale Kham}
\def\langnames@langs@glot@gma{Gambera}
\def\langnames@langs@glot@wof{Gambian Wolof}
\def\langnames@langs@glot@gbl{Gamit}
\def\langnames@langs@glot@gak{Gamkonora}
\def\langnames@langs@glot@bte{Gamo-Ningi}
\def\langnames@langs@glot@ihw{Birrdhawal}
\def\langnames@langs@glot@gne{Ganang}
\def\langnames@langs@glot@gnk{//Gana}
\def\langnames@langs@glot@gnq{Gana}
\def\langnames@langs@glot@unn{Ganai}
\def\langnames@langs@glot@gan{Gan Chinese}
\def\langnames@langs@glot@pgd{Gandhari}
\def\langnames@langs@glot@gzn{Gane}
\def\langnames@langs@glot@gnb{Gangte}
\def\langnames@langs@glot@gnl{Gangulu}
\def\langnames@langs@glot@ggl{Ganglau}
\def\langnames@langs@glot@gao{Gants}
\def\langnames@langs@glot@gza{Ganza}
\def\langnames@langs@glot@gnz{Ganzi}
\def\langnames@langs@glot@gga{Gao}
\def\langnames@langs@glot@gbm{Garhwali}
\def\langnames@langs@glot@ilg{Garig-Ilgar}
\def\langnames@langs@glot@gex{Garre}
\def\langnames@langs@glot@gaq{Gata'}
\def\langnames@langs@glot@gou{Gavar}
\def\langnames@langs@glot@gwt{Gawar-Bati}
\def\langnames@langs@glot@gyl{Gayil}
\def\langnames@langs@glot@gzi{Gazic}
\def\langnames@langs@glot@gbg{Gbanziri-Boraka}
\def\langnames@langs@glot@gbv{Gbanu}
\def\langnames@langs@glot@gby{Gbari}
\def\langnames@langs@glot@gyg{Gbayi}
\def\langnames@langs@glot@gbq{Gbaya-Bozoum}
\def\langnames@langs@glot@gbs{Gbesi Gbe}
\def\langnames@langs@glot@ggb{Gbii}
\def\langnames@langs@glot@xgb{Gbin}
\def\langnames@langs@glot@grh{Gbiri-Niragu}
\def\langnames@langs@glot@gec{Gboloo Grebo}
\def\langnames@langs@glot@kvq{Geba Karen}
\def\langnames@langs@glot@gei{Gebe}
\def\langnames@langs@glot@gdd{Gedaged}
\def\langnames@langs@glot@drs{Gedeo}
\def\langnames@langs@glot@hmj{Ge}
\def\langnames@langs@glot@gez{Geez}
\def\langnames@langs@glot@ghk{Geko Karen}
\def\langnames@langs@glot@giu{Gelao Mulao}
\def\langnames@langs@glot@geq{Geme}
\def\langnames@langs@glot@gaf{Gende}
\def\langnames@langs@glot@gej{Gen}
\def\langnames@langs@glot@ygp{Gepo}
\def\langnames@langs@glot@gew{Gera}
\def\langnames@langs@glot@gea{Geruma}
\def\langnames@langs@glot@ges{Geser-Gorom}
\def\langnames@langs@glot@gha{Ghadames}
\def\langnames@langs@glot@gse{Ghanaian Sign Language}
\def\langnames@langs@glot@ghn{Ghanongga}
\def\langnames@langs@glot@gpe{Ghanaian Pidgin English}
\def\langnames@langs@glot@gds{Ghandruk Sign Language}
\def\langnames@langs@glot@gri{Ghari}
\def\langnames@langs@glot@ajs{Ghardaia Sign Language}
\def\langnames@langs@glot@bmk{Ghayavi}
\def\langnames@langs@glot@aln{Gheg Albanian}
\def\langnames@langs@glot@ghr{Ghera}
\def\langnames@langs@glot@bbj{Ghomálá'}
\def\langnames@langs@glot@gho{Ghomara}
\def\langnames@langs@glot@bgi{Giangan}
\def\langnames@langs@glot@gib{Gibanawa}
\def\langnames@langs@glot@kks{Giiwo}
\def\langnames@langs@glot@acd{Gikyode}
\def\langnames@langs@glot@gix{Gilima}
\def\langnames@langs@glot@gip{Gimi (West New Britain)}
\def\langnames@langs@glot@gim{Gimi (Eastern Highlands)}
\def\langnames@langs@glot@kmp{Gimme}
\def\langnames@langs@glot@gmn{Gimnime}
\def\langnames@langs@glot@gnm{Ginuman}
\def\langnames@langs@glot@ayg{Ginyanga}
\def\langnames@langs@glot@bbr{Girawa}
\def\langnames@langs@glot@gii{Girirra}
\def\langnames@langs@glot@nyf{Giryama}
\def\langnames@langs@glot@toh{Gitonga}
\def\langnames@langs@glot@ggt{Gitua}
\def\langnames@langs@glot@giy{Giyug}
\def\langnames@langs@glot@tof{Gizrra}
\def\langnames@langs@glot@glr{Glaro-Twabo}
\def\langnames@langs@glot@glw{Glavda}
\def\langnames@langs@glot@oub{Glio-Oubi}
\def\langnames@langs@glot@gnu{Gnau}
\def\langnames@langs@glot@gom{Goan Konkani}
\def\langnames@langs@glot@gig{Goaria}
\def\langnames@langs@glot@goi{Gobasi}
\def\langnames@langs@glot@gox{Gobu}
\def\langnames@langs@glot@gdx{Godwari}
\def\langnames@langs@glot@gof{Gofa}
\def\langnames@langs@glot@gog{Gogo}
\def\langnames@langs@glot@goo{Gone Dau}
\def\langnames@langs@glot@goe{Gongduk}
\def\langnames@langs@glot@gjn{Gonja}
\def\langnames@langs@glot@gov{Goo}
\def\langnames@langs@glot@goq{Gorap}
\def\langnames@langs@glot@goc{Gorakor}
\def\langnames@langs@glot@grq{Gorovu}
\def\langnames@langs@glot@gqr{Gor}
\def\langnames@langs@glot@got{Gothic}
\def\langnames@langs@glot@goy{Goundo}
\def\langnames@langs@glot@gwf{Gowro}
\def\langnames@langs@glot@goz{Alamuti}
\def\langnames@langs@glot@nli{Grangali-Ningalami}
\def\langnames@langs@glot@giq{Hagei Gelao}
\def\langnames@langs@glot@gcl{Grenadian Creole English}
\def\langnames@langs@glot@grs{Gresi}
\def\langnames@langs@glot@gro{Groma}
\def\langnames@langs@glot@gos{Gronings}
\def\langnames@langs@glot@ats{Gros Ventre}
\def\langnames@langs@glot@gwx{Gua}
\def\langnames@langs@glot@gvj{Guajá}
\def\langnames@langs@glot@jiq{Khroskyabs}
\def\langnames@langs@glot@gnc{Guanche}
\def\langnames@langs@glot@gyr{Guarayu}
\def\langnames@langs@glot@gsm{Guatemalan Sign Language}
\def\langnames@langs@glot@xgd{Gudang}
\def\langnames@langs@glot@gdu{Gudu}
\def\langnames@langs@glot@zpg{Guevea De Humboldt Zapotec}
\def\langnames@langs@glot@gdc{Gugu Badhun}
\def\langnames@langs@glot@kkp{Gugubera}
\def\langnames@langs@glot@wrw{Roth's Gugu Warra}
\def\langnames@langs@glot@zgn{Guibian Zhuang}
\def\langnames@langs@glot@bet{Guiberoua Béte}
\def\langnames@langs@glot@ztu{Güilá Zapotec}
\def\langnames@langs@glot@gus{Guinean Sign Language}
\def\langnames@langs@glot@gkp{Guinea Kpelle}
\def\langnames@langs@glot@gqi{Guiqiong}
\def\langnames@langs@glot@gvl{Gulay}
\def\langnames@langs@glot@glu{Gula (Chad)}
\def\langnames@langs@glot@gmb{Gula'alaa}
\def\langnames@langs@glot@gly{Gule}
\def\langnames@langs@glot@gul{Sea Island Creole English}
\def\langnames@langs@glot@gmu{Gumalu}
\def\langnames@langs@glot@gdi{Gundi}
\def\langnames@langs@glot@gyf{Gungabula}
\def\langnames@langs@glot@rub{Gungu}
\def\langnames@langs@glot@gnt{Warta Thuntai}
\def\langnames@langs@glot@gpa{Gupa-Abawa}
\def\langnames@langs@glot@grz{Guramalum}
\def\langnames@langs@glot@gdj{Gurdjar}
\def\langnames@langs@glot@ggg{Gurgula}
\def\langnames@langs@glot@grx{Guriaso}
\def\langnames@langs@glot@gjr{Gurindji Kriol}
\def\langnames@langs@glot@gvm{Gurmana}
\def\langnames@langs@glot@gvr{Gurung}
\def\langnames@langs@glot@grd{Guruntum-Mbaaru}
\def\langnames@langs@glot@gsn{Gusan}
\def\langnames@langs@glot@gsl{Gusilay}
\def\langnames@langs@glot@xgw{Guwa}
\def\langnames@langs@glot@gwu{Guwamu}
\def\langnames@langs@glot@gvy{Guyani}
\def\langnames@langs@glot@gka{Guya}
\def\langnames@langs@glot@ngs{Gvoko}
\def\langnames@langs@glot@gwb{Gwa}
\def\langnames@langs@glot@dah{Gwahatike}
\def\langnames@langs@glot@bga{Gwamhi-Wuri}
\def\langnames@langs@glot@gwn{Gwandara}
\def\langnames@langs@glot@grw{Gweda}
\def\langnames@langs@glot@gwe{Gweno}
\def\langnames@langs@glot@gwr{Gwere}
\def\langnames@langs@glot@gwj{/Gwi}
\def\langnames@langs@glot@gyi{Gyele}
\def\langnames@langs@glot@gye{Gyem}
\def\langnames@langs@glot@haq{Ha}
\def\langnames@langs@glot@hbu{Habu}
\def\langnames@langs@glot@hdy{Hadiyya}
\def\langnames@langs@glot@hoj{Hadothi}
\def\langnames@langs@glot@xhd{Hadrami}
\def\langnames@langs@glot@ayh{Hadrami Arabic}
\def\langnames@langs@glot@aek{Haeke}
\def\langnames@langs@glot@hah{Hahon}
\def\langnames@langs@glot@hgw{Haigwai}
\def\langnames@langs@glot@bzx{Hainyaxo Bozo}
\def\langnames@langs@glot@hgm{Hai//om-Akhoe}
\def\langnames@langs@glot@haf{Haiphong Sign Language}
\def\langnames@langs@glot@hvc{Haitian Vodoun Culture Language}
\def\langnames@langs@glot@hji{Haji}
\def\langnames@langs@glot@haj{Hajong}
\def\langnames@langs@glot@hao{Hakö}
\def\langnames@langs@glot@hld{Halang Doan}
\def\langnames@langs@glot@hmu{Hamap}
\def\langnames@langs@glot@hba{Hamba de Lomela}
\def\langnames@langs@glot@hag{Hanga}
\def\langnames@langs@glot@han{Hangaza}
\def\langnames@langs@glot@haa{Han}
\def\langnames@langs@glot@hab{Hanoi Sign Language}
\def\langnames@langs@glot@xiv{Harappan}
\def\langnames@langs@glot@kjo{Indo-Aryan Kinnauri}
\def\langnames@langs@glot@hro{Haroi}
\def\langnames@langs@glot@hrk{Haruku}
\def\langnames@langs@glot@bgc{Haryanvi}
\def\langnames@langs@glot@hrz{Harzani-Kilit}
\def\langnames@langs@glot@ybj{Hasha}
\def\langnames@langs@glot@xht{Hattic}
\def\langnames@langs@glot@hsl{Hausa Sign Language}
\def\langnames@langs@glot@hvk{Haveke}
\def\langnames@langs@glot@hav{Havu}
\def\langnames@langs@glot@hps{Hawai'i Pidgin Sign Language}
\def\langnames@langs@glot@xda{Hawkesbury}
\def\langnames@langs@glot@haz{Hazaragi}
\def\langnames@langs@glot@hbn{Ebang}
\def\langnames@langs@glot@scp{Lamjung-Melamchi Yolmo}
\def\langnames@langs@glot@heg{Helong}
\def\langnames@langs@glot@nix{Hema}
\def\langnames@langs@glot@hed{Herde}
\def\langnames@langs@glot@llf{Hermit}
\def\langnames@langs@glot@hrt{Hertevin}
\def\langnames@langs@glot@ham{Hewa}
\def\langnames@langs@glot@auk{Heyo}
\def\langnames@langs@glot@hib{Hibito}
\def\langnames@langs@glot@hlu{Hieroglyphic Luwian}
\def\langnames@langs@glot@mba{Higaonon}
\def\langnames@langs@glot@kjk{Highland Konjo}
\def\langnames@langs@glot@hij{Hijuk}
\def\langnames@langs@glot@hir{Himarimã}
\def\langnames@langs@glot@hii{Hinduri}
\def\langnames@langs@glot@hmo{Hiri Motu}
\def\langnames@langs@glot@hit{Hittite}
\def\langnames@langs@glot@htu{Hitu}
\def\langnames@langs@glot@hiw{Hiw}
\def\langnames@langs@glot@yhl{Hlepho Phowa}
\def\langnames@langs@glot@hle{Hlersu}
\def\langnames@langs@glot@hmf{Hmong Don}
\def\langnames@langs@glot@hmz{Sinicized Miao}
\def\langnames@langs@glot@hmv{Hmong Dô}
\def\langnames@langs@glot@mrk{Hmwaveke}
\def\langnames@langs@glot@hoh{Hobyót}
\def\langnames@langs@glot@hos{Ho Chi Minh City Sign Language}
\def\langnames@langs@glot@hhi{Hoia Hoia-Ukusi-Koperami}
\def\langnames@langs@glot@hoy{Holiya}
\def\langnames@langs@glot@hoi{Holikachuk}
\def\langnames@langs@glot@hod{Holma}
\def\langnames@langs@glot@hol{Holu}
\def\langnames@langs@glot@hom{Homa}
\def\langnames@langs@glot@hds{Honduras Sign Language}
\def\langnames@langs@glot@juh{Hõne}
\def\langnames@langs@glot@how{Honi}
\def\langnames@langs@glot@hrm{Horned Miao}
\def\langnames@langs@glot@hoe{Horom}
\def\langnames@langs@glot@hor{Horo}
\def\langnames@langs@glot@ero{Stau-Dgebshes}
\def\langnames@langs@glot@hot{Hote}
\def\langnames@langs@glot@hti{Hoti of East Seram}
\def\langnames@langs@glot@hov{Hobongan}
\def\langnames@langs@glot@hhy{Hoyahoya-Matakaia}
\def\langnames@langs@glot@hoz{Hozo}
\def\langnames@langs@glot@hpo{Hpon}
\def\langnames@langs@glot@hra{Hrangkhol}
\def\langnames@langs@glot@hru{Hruso}
\def\langnames@langs@glot@hug{Huachipaeri}
\def\langnames@langs@glot@qvh{Huamalíes-Dos de Mayo Huánuco Quechua}
\def\langnames@langs@glot@hud{Huaulu}
\def\langnames@langs@glot@nhq{Huaxcaleca Nahuatl}
\def\langnames@langs@glot@qwh{Huaylas Ancash Quechua}
\def\langnames@langs@glot@qvw{Huaylla Wanca Quechua}
\def\langnames@langs@glot@huh{Huilliche}
\def\langnames@langs@glot@mxs{Huitepec Mixtec}
\def\langnames@langs@glot@czh{Hui Chinese}
\def\langnames@langs@glot@huw{Hukumina}
\def\langnames@langs@glot@hul{Hula}
\def\langnames@langs@glot@huy{Hulaulá}
\def\langnames@langs@glot@hui{Huli}
\def\langnames@langs@glot@huk{Hulung}
\def\langnames@langs@glot@hmb{Humburi Senni Songhay}
\def\langnames@langs@glot@huf{Humene}
\def\langnames@langs@glot@hut{Humla}
\def\langnames@langs@glot@hsh{Hungarian Sign Language}
\def\langnames@langs@glot@hnu{Hung}
\def\langnames@langs@glot@nat{Hungworo}
\def\langnames@langs@glot@hum{Hungan}
\def\langnames@langs@glot@hng{Hungu-Pombo}
\def\langnames@langs@glot@hkk{Hunjara-Kaina Ke}
\def\langnames@langs@glot@hap{Hupla}
\def\langnames@langs@glot@xhu{Hurrian}
\def\langnames@langs@glot@geh{Hutterite German}
\def\langnames@langs@glot@huo{Hu}
\def\langnames@langs@glot@hwo{Hwana}
\def\langnames@langs@glot@hya{Hya}
\def\langnames@langs@glot@jab{Hyam}
\def\langnames@langs@glot@yml{Iamalele}
\def\langnames@langs@glot@tek{Kwa South}
\def\langnames@langs@glot@ibl{Ibaloi}
\def\langnames@langs@glot@iby{Ibani}
\def\langnames@langs@glot@xib{Iberian}
\def\langnames@langs@glot@ibn{Ibino}
\def\langnames@langs@glot@ibr{Ibuoro}
\def\langnames@langs@glot@ibu{Ibu}
\def\langnames@langs@glot@bec{Iceve-Maci}
\def\langnames@langs@glot@ida{Idakho-Isukha-Tiriki}
\def\langnames@langs@glot@idt{Idaté}
\def\langnames@langs@glot@ide{Idere}
\def\langnames@langs@glot@idi{Idi-Taeme}
\def\langnames@langs@glot@idc{Idon}
\def\langnames@langs@glot@ido{Ido}
\def\langnames@langs@glot@ldb{Dũya}
\def\langnames@langs@glot@ife{Ifè}
\def\langnames@langs@glot@iff{Ifo}
\def\langnames@langs@glot@igl{Igala}
\def\langnames@langs@glot@igg{Igana}
\def\langnames@langs@glot@ahl{Igo}
\def\langnames@langs@glot@nar{Iguta}
\def\langnames@langs@glot@igw{Igwe}
\def\langnames@langs@glot@ihb{Iha-based Pidgin}
\def\langnames@langs@glot@ikk{Ika}
\def\langnames@langs@glot@ikr{Ikaranggal}
\def\langnames@langs@glot@ikz{Ikizu}
\def\langnames@langs@glot@meb{Ikobi}
\def\langnames@langs@glot@ntk{Ikoma-Nata}
\def\langnames@langs@glot@iki{Iko}
\def\langnames@langs@glot@ikp{Ikpeshi}
\def\langnames@langs@glot@txi{Ikpeng}
\def\langnames@langs@glot@ikv{Iku-Gora-Ankwa}
\def\langnames@langs@glot@ikl{Ikulu}
\def\langnames@langs@glot@ikw{Ikwere}
\def\langnames@langs@glot@ila{Ile Ape}
\def\langnames@langs@glot@mbi{Ilianen Manobo}
\def\langnames@langs@glot@ili{Ili Turki}
\def\langnames@langs@glot@ilu{Ili'uun}
\def\langnames@langs@glot@xil{Illyrian}
\def\langnames@langs@glot@ilk{Ilongot}
\def\langnames@langs@glot@ilv{Ilue}
\def\langnames@langs@glot@mlk{Ilwana}
\def\langnames@langs@glot@imo{Imbongu}
\def\langnames@langs@glot@arc{Imperial Aramaic (700-300 BCE)}
\def\langnames@langs@glot@imr{Imroing}
\def\langnames@langs@glot@abx{Inabaknon}
\def\langnames@langs@glot@mzu{Itutang-Inapang}
\def\langnames@langs@glot@inp{Iñapari}
\def\langnames@langs@glot@smn{Inari Saami}
\def\langnames@langs@glot@inl{Jakartan Sign Language}
\def\langnames@langs@glot@idr{Indri}
\def\langnames@langs@glot@mvy{Indus Kohistani}
\def\langnames@langs@glot@oin{Inebu One}
\def\langnames@langs@glot@iti{Inlaod Itneg}
\def\langnames@langs@glot@ino{Inoke-Yate}
\def\langnames@langs@glot@loc{Inonhan}
\def\langnames@langs@glot@ior{Inoric}
\def\langnames@langs@glot@ina{Interlingua (International Auxiliary Language Association)}
\def\langnames@langs@glot@ile{Interlingue (Occidental)}
\def\langnames@langs@glot@igs{Interglossa}
\def\langnames@langs@glot@int{Intha-Danu}
\def\langnames@langs@glot@iks{Inuit Sign Language}
\def\langnames@langs@glot@azm{Ipalapa Amuzgo}
\def\langnames@langs@glot@ipo{Ipiko}
\def\langnames@langs@glot@ipi{Ipili}
\def\langnames@langs@glot@ass{Ipulo-Olulu}
\def\langnames@langs@glot@ill{Iranun}
\def\langnames@langs@glot@iry{Iraya}
\def\langnames@langs@glot@ire{Yerisiam}
\def\langnames@langs@glot@iri{Irigwe}
\def\langnames@langs@glot@bto{Iriga Bicolano}
\def\langnames@langs@glot@iru{Irula of the Nilgiri}
\def\langnames@langs@glot@isa{Isabi}
\def\langnames@langs@glot@isn{Isanzu}
\def\langnames@langs@glot@agk{Isarog Agta}
\def\langnames@langs@glot@isc{Isconahua}
\def\langnames@langs@glot@igo{Isebe}
\def\langnames@langs@glot@inn{Isinai}
\def\langnames@langs@glot@crb{Island Carib}
\def\langnames@langs@glot@mir{Isthmus Mixe}
\def\langnames@langs@glot@nhk{Isthmus-Cosoleacaque Nahuatl}
\def\langnames@langs@glot@ist{Istriot}
\def\langnames@langs@glot@ruo{Istro Romanian}
\def\langnames@langs@glot@szv{Isu (Fako Division)}
\def\langnames@langs@glot@isu{Isu (Menchum Division)}
\def\langnames@langs@glot@ite{Itene}
\def\langnames@langs@glot@itr{Iteri}
\def\langnames@langs@glot@itx{Itik}
\def\langnames@langs@glot@itw{Ito}
\def\langnames@langs@glot@itm{Itu Mbon Uzo}
\def\langnames@langs@glot@mce{Itundujia Mixtec}
\def\langnames@langs@glot@ivv{Itbayat}
\def\langnames@langs@glot@atg{Ivbie North-Okpela-Arhe}
\def\langnames@langs@glot@iwk{I-Wak}
\def\langnames@langs@glot@kbm{Iwal}
\def\langnames@langs@glot@iwo{Morop-Dintere}
\def\langnames@langs@glot@mzi{Ixcatlán Mazatec}
\def\langnames@langs@glot@vmj{Ixtayutla Mixtec}
\def\langnames@langs@glot@iya{Iyayu}
\def\langnames@langs@glot@uiv{Iyive}
\def\langnames@langs@glot@crt{Iyojwa'ja Chorote}
\def\langnames@langs@glot@nca{Iyo}
\def\langnames@langs@glot@crq{Iyo'wujwa Chorote}
\def\langnames@langs@glot@izi{Izi-Ezaa-Ikwo-Mgbo}
\def\langnames@langs@glot@cbo{Izora}
\def\langnames@langs@glot@rzh{Jabal Razih}
\def\langnames@langs@glot@jdg{Jadgali}
\def\langnames@langs@glot@jad{Jahanka}
\def\langnames@langs@glot@jah{Jah Hut}
\def\langnames@langs@glot@awv{Kia River Awyu}
\def\langnames@langs@glot@jat{Inku}
\def\langnames@langs@glot@jak{Jakun}
\def\langnames@langs@glot@maj{Jalapa De Díaz Mazatec}
\def\langnames@langs@glot@bxl{Jalkunan}
\def\langnames@langs@glot@jcs{Jamaican Country Sign Language}
\def\langnames@langs@glot@jls{Jamaican Sign Language}
\def\langnames@langs@glot@jax{Jambi Malay}
\def\langnames@langs@glot@jnd{Jandavra}
\def\langnames@langs@glot@jna{Jangshung}
\def\langnames@langs@glot@djo{Jangkang}
\def\langnames@langs@glot@jni{Janji}
\def\langnames@langs@glot@jar{Jarawa (Nigeria)}
\def\langnames@langs@glot@jra{Jarai}
\def\langnames@langs@glot@jaf{Jara}
\def\langnames@langs@glot@qxw{Jauja Wanca Quechua}
\def\langnames@langs@glot@jns{Jaunsari}
\def\langnames@langs@glot@jvd{Javindo}
\def\langnames@langs@glot@jaz{Jawe}
\def\langnames@langs@glot@jyy{Jaya}
\def\langnames@langs@glot@jje{Jejueo}
\def\langnames@langs@glot@bze{Jenaama Bozo}
\def\langnames@langs@glot@xuj{Jennu Kurumba}
\def\langnames@langs@glot@jer{Jere}
\def\langnames@langs@glot@jee{Jerung}
\def\langnames@langs@glot@tmr{Jewish Babylonian Aramaic (ca. 200-1200 CE)}
\def\langnames@langs@glot@jhs{Jhankot Sign Language}
\def\langnames@langs@glot@jio{Jiamao}
\def\langnames@langs@glot@juo{Jiba}
\def\langnames@langs@glot@jib{Jibu}
\def\langnames@langs@glot@jii{Jiiddu}
\def\langnames@langs@glot@jie{Jilbe}
\def\langnames@langs@glot@jil{Jilim}
\def\langnames@langs@glot@jim{Jimi (Cameroon)}
\def\langnames@langs@glot@jmi{Jimi (Nigeria)}
\def\langnames@langs@glot@jia{Jina}
\def\langnames@langs@glot@cjy{Jinyu Chinese}
\def\langnames@langs@glot@pnu{Jiongnai Bunu}
\def\langnames@langs@glot@jul{Jirel}
\def\langnames@langs@glot@jrr{Jiru}
\def\langnames@langs@glot@jit{Jita}
\def\langnames@langs@glot@kaj{Jju}
\def\langnames@langs@glot@job{Joba}
\def\langnames@langs@glot@jbr{Jofotek-Bromnya}
\def\langnames@langs@glot@jeu{Jonkor Bourmataguil}
\def\langnames@langs@glot@jor{Jorá}
\def\langnames@langs@glot@jrt{Jakattoe}
\def\langnames@langs@glot@jow{Jowulu}
\def\langnames@langs@glot@itk{Judeo-Italian}
\def\langnames@langs@glot@jdt{Judeo-Tat}
\def\langnames@langs@glot@jpr{Judeo-Persian}
\def\langnames@langs@glot@yud{Judeo-Tripolitanian Arabic}
\def\langnames@langs@glot@aju{Judeo-Moroccan Arabic}
\def\langnames@langs@glot@yhd{Judeo-Iraqi Arabic}
\def\langnames@langs@glot@jye{Judeo-Yemeni Arabic}
\def\langnames@langs@glot@jum{Jumjum}
\def\langnames@langs@glot@jml{Jumli}
\def\langnames@langs@glot@jus{Jumla Sign Language}
\def\langnames@langs@glot@mxq{Juquila Mixe}
\def\langnames@langs@glot@juy{Juray}
\def\langnames@langs@glot@jut{Jutish}
\def\langnames@langs@glot@juu{Ju}
\def\langnames@langs@glot@mwb{Juwal}
\def\langnames@langs@glot@vmc{Juxtlahuaca Mixtec}
\def\langnames@langs@glot@jwi{Jwira-Pepesa}
\def\langnames@langs@glot@xku{Kaamba}
\def\langnames@langs@glot@gna{Kaansa}
\def\langnames@langs@glot@ldl{Kaan}
\def\langnames@langs@glot@ckn{Kaang Chin}
\def\langnames@langs@glot@ksp{Kaba}
\def\langnames@langs@glot@kvf{Kabalai}
\def\langnames@langs@glot@gbw{Kabikabi}
\def\langnames@langs@glot@klz{Kabola}
\def\langnames@langs@glot@onk{Kabore One}
\def\langnames@langs@glot@lkb{Kabras}
\def\langnames@langs@glot@uka{Kaburi}
\def\langnames@langs@glot@kbu{Kabutra}
\def\langnames@langs@glot@kea{Kabuverdianu}
\def\langnames@langs@glot@cwa{Kabwa}
\def\langnames@langs@glot@kcw{Kabwari}
\def\langnames@langs@glot@gjk{Kachi Koli}
\def\langnames@langs@glot@kfr{Kachchi}
\def\langnames@langs@glot@kcx{Kachama-Ganjule-Haro}
\def\langnames@langs@glot@xkk{Kaco'}
\def\langnames@langs@glot@kej{Kadar}
\def\langnames@langs@glot@kdu{Kadaru}
\def\langnames@langs@glot@kad{Kadara}
\def\langnames@langs@glot@kzd{Kadai}
\def\langnames@langs@glot@kdv{Kado}
\def\langnames@langs@glot@ktp{Kaduo}
\def\langnames@langs@glot@jka{Kaera}
\def\langnames@langs@glot@kpu{Kafoa}
\def\langnames@langs@glot@sqx{Kafr Qasem Sign Language}
\def\langnames@langs@glot@syw{Kagate}
\def\langnames@langs@glot@kll{Kagan Kalagan}
\def\langnames@langs@glot@cgc{Kagayanen}
\def\langnames@langs@glot@gel{Ut-Main}
\def\langnames@langs@glot@xkg{Kagoro}
\def\langnames@langs@glot@hka{Kahe}
\def\langnames@langs@glot@agw{Kahua}
\def\langnames@langs@glot@kzb{Kaibobo}
\def\langnames@langs@glot@kzp{Kaidipang}
\def\langnames@langs@glot@kbw{Kaiep}
\def\langnames@langs@glot@kep{Kaikadi}
\def\langnames@langs@glot@kzq{Kaike}
\def\langnames@langs@glot@kkq{Kaiku}
\def\langnames@langs@glot@xai{Kaimbé}
\def\langnames@langs@glot@zka{Kaimbulawa}
\def\langnames@langs@glot@krd{Kairui-Midiki}
\def\langnames@langs@glot@ckr{Kairak}
\def\langnames@langs@glot@kzm{Kais}
\def\langnames@langs@glot@kce{Kaivi}
\def\langnames@langs@glot@tcq{Kaiy}
\def\langnames@langs@glot@xkj{Kajali}
\def\langnames@langs@glot@kag{Kajaman}
\def\langnames@langs@glot@ckq{Kajakse}
\def\langnames@langs@glot@kjv{Kajkavian}
\def\langnames@langs@glot@xdq{Kajtak}
\def\langnames@langs@glot@kka{Kakanda}
\def\langnames@langs@glot@kke{Kakabe}
\def\langnames@langs@glot@kqf{Kakabai}
\def\langnames@langs@glot@kkj{Kako}
\def\langnames@langs@glot@keo{Kakwa}
\def\langnames@langs@glot@wkl{Kalanadi}
\def\langnames@langs@glot@kzz{Kalabra}
\def\langnames@langs@glot@kkf{Kalaktang Monpa}
\def\langnames@langs@glot@kba{Kalarko-Mirniny}
\def\langnames@langs@glot@gll{Bulloo River}
\def\langnames@langs@glot@ijn{Kalabari}
\def\langnames@langs@glot@knz{Kalamsé}
\def\langnames@langs@glot@kqe{Kalagan}
\def\langnames@langs@glot@kve{Kalabakan}
\def\langnames@langs@glot@kly{Kalao}
\def\langnames@langs@glot@lkm{Kalaamaya}
\def\langnames@langs@glot@xka{Kalkoti}
\def\langnames@langs@glot@rmf{Kalo Finnish Romani}
\def\langnames@langs@glot@ywa{Kalou}
\def\langnames@langs@glot@kli{Kalumpang}
\def\langnames@langs@glot@keq{Kamar}
\def\langnames@langs@glot@jmr{Kamara}
\def\langnames@langs@glot@kci{Kamantan}
\def\langnames@langs@glot@klp{Kamasa}
\def\langnames@langs@glot@kzx{Kamarian}
\def\langnames@langs@glot@kyk{Kamayo}
\def\langnames@langs@glot@kgx{Kamaru}
\def\langnames@langs@glot@vkm{Kamakan}
\def\langnames@langs@glot@xbw{Kambiwá}
\def\langnames@langs@glot@irx{Kamberau}
\def\langnames@langs@glot@kyy{Kambaira}
\def\langnames@langs@glot@ktb{Kambaata}
\def\langnames@langs@glot@kmi{Kami (Nigeria)}
\def\langnames@langs@glot@kdx{Kam}
\def\langnames@langs@glot@kcq{Kamo}
\def\langnames@langs@glot@xla{Kamula}
\def\langnames@langs@glot@hig{Kamwe}
\def\langnames@langs@glot@bjj{Kanauji}
\def\langnames@langs@glot@xnb{Kanakanavu}
\def\langnames@langs@glot@soq{Kanasi}
\def\langnames@langs@glot@kbs{Kande}
\def\langnames@langs@glot@kqw{Kandas}
\def\langnames@langs@glot@gam{Kandawo}
\def\langnames@langs@glot@xnr{Kangri}
\def\langnames@langs@glot@kxs{Kangjia}
\def\langnames@langs@glot@kzy{Kango (Tshopo District)}
\def\langnames@langs@glot@kty{Kango (Bas-Uélé District)}
\def\langnames@langs@glot@kcp{Kanga}
\def\langnames@langs@glot@kkv{Kangean}
\def\langnames@langs@glot@igm{Kanggape}
\def\langnames@langs@glot@kev{Kanikkaran}
\def\langnames@langs@glot@kdp{Kaningdon-Nindem}
\def\langnames@langs@glot@kzo{Kaningi}
\def\langnames@langs@glot@wat{Kaninuwa}
\def\langnames@langs@glot@ktk{Kaniet}
\def\langnames@langs@glot@knr{Kaningra}
\def\langnames@langs@glot@kmu{Kanite}
\def\langnames@langs@glot@kft{Kanjari}
\def\langnames@langs@glot@kbe{Kanju}
\def\langnames@langs@glot@kxn{Kanowit-Tanjong Melanau}
\def\langnames@langs@glot@ksk{Kansa}
\def\langnames@langs@glot@xkt{Kantosi}
\def\langnames@langs@glot@kni{Kanufi}
\def\langnames@langs@glot@khx{Kanu}
\def\langnames@langs@glot@kqn{Kaonde}
\def\langnames@langs@glot@kax{Kao}
\def\langnames@langs@glot@xpn{Kapinawá}
\def\langnames@langs@glot@tbx{Kapin}
\def\langnames@langs@glot@khp{Kapori}
\def\langnames@langs@glot@ykm{Kap}
\def\langnames@langs@glot@kbi{Kaptiau}
\def\langnames@langs@glot@klo{Kapya}
\def\langnames@langs@glot@xkh{Karahawyana}
\def\langnames@langs@glot@kzr{Karang}
\def\langnames@langs@glot@reg{Kara (Tanzania)}
\def\langnames@langs@glot@kth{Karanga}
\def\langnames@langs@glot@mry{Mandaya}
\def\langnames@langs@glot@xrw{Karawa}
\def\langnames@langs@glot@xar{Karami}
\def\langnames@langs@glot@kgv{Kalamang}
\def\langnames@langs@glot@kbn{Kare (Central African Republic)}
\def\langnames@langs@glot@kyd{Karey}
\def\langnames@langs@glot@kmf{Kare (Papua New Guinea)}
\def\langnames@langs@glot@kai{Karekare}
\def\langnames@langs@glot@kmv{Uaçá Creole French}
\def\langnames@langs@glot@kgn{Karingani-Kalasuri-Khoynarudi}
\def\langnames@langs@glot@kbj{Kari}
\def\langnames@langs@glot@kil{Kariya}
\def\langnames@langs@glot@kuq{Karipúna}
\def\langnames@langs@glot@kko{Karko}
\def\langnames@langs@glot@krb{Karkin}
\def\langnames@langs@glot@bbv{Karnai}
\def\langnames@langs@glot@krx{Karon}
\def\langnames@langs@glot@kxh{Karo (Ethiopia)}
\def\langnames@langs@glot@xkx{Karore}
\def\langnames@langs@glot@kyn{Northern Binukidnon}
\def\langnames@langs@glot@rxw{Karruwali}
\def\langnames@langs@glot@ccj{Kasanga}
\def\langnames@langs@glot@ksn{Kasiguranin}
\def\langnames@langs@glot@kkz{Kaska}
\def\langnames@langs@glot@khs{Kasua}
\def\langnames@langs@glot@ktq{Katabaga}
\def\langnames@langs@glot@xat{Katawixi}
\def\langnames@langs@glot@tmb{Avava}
\def\langnames@langs@glot@tkt{Kathoriya Tharu}
\def\langnames@langs@glot@ykt{Thou-Kathu}
\def\langnames@langs@glot@kfu{Katkari}
\def\langnames@langs@glot@kaf{Katso}
\def\langnames@langs@glot@kta{Katua}
\def\langnames@langs@glot@vkk{Kaur}
\def\langnames@langs@glot@xau{Kauwera}
\def\langnames@langs@glot@ckv{Kavalan}
\def\langnames@langs@glot@kcb{Kawacha}
\def\langnames@langs@glot@kgb{Kawe}
\def\langnames@langs@glot@kaw{Kawi}
\def\langnames@langs@glot@ktx{Kaxararí}
\def\langnames@langs@glot@kbb{Kaxuiâna}
\def\langnames@langs@glot@pdu{Kayan Lahwi}
\def\langnames@langs@glot@xay{Kayan Mahakam}
\def\langnames@langs@glot@xkn{Kayan River Kayan}
\def\langnames@langs@glot@kyt{Kayagar}
\def\langnames@langs@glot@kzl{Kayeli}
\def\langnames@langs@glot@kxy{Kayong}
\def\langnames@langs@glot@kzu{Kayupulau}
\def\langnames@langs@glot@kzk{Kazukuru}
\def\langnames@langs@glot@keh{Keak}
\def\langnames@langs@glot@khz{Keapara}
\def\langnames@langs@glot@meo{Kedah-Perak Malay}
\def\langnames@langs@glot@kdy{Keder}
\def\langnames@langs@glot@khh{Kehu}
\def\langnames@langs@glot@kec{Keiga}
\def\langnames@langs@glot@bmh{Kein}
\def\langnames@langs@glot@eyo{Keiyo}
\def\langnames@langs@glot@khy{Kele-Foma}
\def\langnames@langs@glot@keb{Kélé}
\def\langnames@langs@glot@ify{Keley-i Kallahan}
\def\langnames@langs@glot@kbo{Keliko}
\def\langnames@langs@glot@xel{Kelo}
\def\langnames@langs@glot@kyo{Klon}
\def\langnames@langs@glot@kem{Kemak}
\def\langnames@langs@glot@bzp{Kemberano}
\def\langnames@langs@glot@xem{Mateq}
\def\langnames@langs@glot@xkw{Kembra}
\def\langnames@langs@glot@dmo{Kemezung}
\def\langnames@langs@glot@sjk{Kemi Saami}
\def\langnames@langs@glot@xbn{Kenaboi}
\def\langnames@langs@glot@gat{Kenati}
\def\langnames@langs@glot@kvm{Kendem}
\def\langnames@langs@glot@klf{Kendeje}
\def\langnames@langs@glot@knx{Kendayan-Belangin}
\def\langnames@langs@glot@knl{Keninjal}
\def\langnames@langs@glot@kxi{Keningau Murut}
\def\langnames@langs@glot@kns{Kensiu}
\def\langnames@langs@glot@ndb{Kenswei Nsei}
\def\langnames@langs@glot@kzh{Kenuzi-Dongola}
\def\langnames@langs@glot@lke{Kenyi}
\def\langnames@langs@glot@xeu{Keoru-Ahia}
\def\langnames@langs@glot@kpn{Kepkiriwát}
\def\langnames@langs@glot@kuk{Kepo'}
\def\langnames@langs@glot@hhr{Keerak}
\def\langnames@langs@glot@ked{Kerewe}
\def\langnames@langs@glot@xke{Kereho}
\def\langnames@langs@glot@kxz{Kerewo}
\def\langnames@langs@glot@kvr{Kerinci}
\def\langnames@langs@glot@xes{Kesawai}
\def\langnames@langs@glot@kae{Ketangalan}
\def\langnames@langs@glot@ktt{Ketum}
\def\langnames@langs@glot@kyg{Keyagana}
\def\langnames@langs@glot@xkv{Kgalagadi}
\def\langnames@langs@glot@hkh{Khah}
\def\langnames@langs@glot@kbg{Khamba}
\def\langnames@langs@glot@kht{Khamti}
\def\langnames@langs@glot@ksu{Khamyang}
\def\langnames@langs@glot@khn{Khandesi}
\def\langnames@langs@glot@kjm{Kháng}
\def\langnames@langs@glot@ksy{Kharia Thar}
\def\langnames@langs@glot@kfw{Kharam Naga}
\def\langnames@langs@glot@lko{Khayo}
\def\langnames@langs@glot@kqg{Khe}
\def\langnames@langs@glot@tlx{Khehek}
\def\langnames@langs@glot@xkf{Khengkha}
\def\langnames@langs@glot@xhe{Khetrani}
\def\langnames@langs@glot@nkh{Khezha Naga}
\def\langnames@langs@glot@kix{Khiamniungan Naga}
\def\langnames@langs@glot@kwx{Khirwar}
\def\langnames@langs@glot@kqm{Khisa}
\def\langnames@langs@glot@ykl{Khlula}
\def\langnames@langs@glot@xkc{Kho'ini}
\def\langnames@langs@glot@nkb{Khoibu}
\def\langnames@langs@glot@ktc{Kholok}
\def\langnames@langs@glot@kho{Khotanese}
\def\langnames@langs@glot@khf{Khuen}
\def\langnames@langs@glot@kfm{Khunsaric}
\def\langnames@langs@glot@xco{Khwarezmian}
\def\langnames@langs@glot@kie{Kibet}
\def\langnames@langs@glot@prm{Kibiri}
\def\langnames@langs@glot@kzg{Kikai}
\def\langnames@langs@glot@kih{Kilmeri}
\def\langnames@langs@glot@kqr{Kimaragang}
\def\langnames@langs@glot@kmb{Kimbundu}
\def\langnames@langs@glot@kiv{Kimbu}
\def\langnames@langs@glot@sbt{Kimki}
\def\langnames@langs@glot@kqp{Kimre}
\def\langnames@langs@glot@krj{Kinaray-a}
\def\langnames@langs@glot@kco{Kinalakna}
\def\langnames@langs@glot@cbw{Kinabalian}
\def\langnames@langs@glot@knq{Kintaq}
\def\langnames@langs@glot@kkd{Kinuku}
\def\langnames@langs@glot@ues{Kioko}
\def\langnames@langs@glot@kkm{Kiong}
\def\langnames@langs@glot@apk{Kiowa Apache}
\def\langnames@langs@glot@sgc{Kipsigis}
\def\langnames@langs@glot@kyi{Kiput}
\def\langnames@langs@glot@kkr{Kir-Balar}
\def\langnames@langs@glot@okr{Kirike}
\def\langnames@langs@glot@kiu{Kirmanjki}
\def\langnames@langs@glot@fkk{Kirya-Konzel}
\def\langnames@langs@glot@lks{Kisa}
\def\langnames@langs@glot@kiz{Kisi}
\def\langnames@langs@glot@kis{Kis}
\def\langnames@langs@glot@zkt{Kitan}
\def\langnames@langs@glot@mwk{Kita Maninkakan}
\def\langnames@langs@glot@mkw{Kituba (Congo)}
\def\langnames@langs@glot@kqt{Klias River Kadazan}
\def\langnames@langs@glot@tlh{Klingon}
\def\langnames@langs@glot@kib{Koalib-Rere}
\def\langnames@langs@glot@kpd{Koba}
\def\langnames@langs@glot@kcj{Kobiana}
\def\langnames@langs@glot@kgu{Kobol}
\def\langnames@langs@glot@thq{Kochila Tharu}
\def\langnames@langs@glot@kdq{Koch}
\def\langnames@langs@glot@dhw{Kochariya-East Danuwar}
\def\langnames@langs@glot@cdz{Koda}
\def\langnames@langs@glot@ksz{Kodaku}
\def\langnames@langs@glot@vko{Kodeoha}
\def\langnames@langs@glot@kwp{Kodia}
\def\langnames@langs@glot@kod{Kodi-Gaura}
\def\langnames@langs@glot@kcs{Koenoem}
\def\langnames@langs@glot@kpi{Kofei}
\def\langnames@langs@glot@kwl{Pan}
\def\langnames@langs@glot@zkg{Koguryo}
\def\langnames@langs@glot@plk{Kohistani Shina}
\def\langnames@langs@glot@kkx{Kohin}
\def\langnames@langs@glot@kkt{Koi}
\def\langnames@langs@glot@nkd{Koireng}
\def\langnames@langs@glot@kxt{Koiwat}
\def\langnames@langs@glot@kou{Koke}
\def\langnames@langs@glot@gko{Kok-Nar}
\def\langnames@langs@glot@xod{Kokoda}
\def\langnames@langs@glot@kzn{Kokola}
\def\langnames@langs@glot@klc{Kolbila}
\def\langnames@langs@glot@ekl{Kol (Bangladesh)}
\def\langnames@langs@glot@biw{Kol (Cameroon)}
\def\langnames@langs@glot@skn{Kolibugan Subanon}
\def\langnames@langs@glot@klm{Kolom}
\def\langnames@langs@glot@kol{Kol (Papua New Guinea)}
\def\langnames@langs@glot@klx{Koluwawa}
\def\langnames@langs@glot@kmy{Koma Ndera}
\def\langnames@langs@glot@kpf{Komba}
\def\langnames@langs@glot@tyn{Kombai}
\def\langnames@langs@glot@kmm{Kom (India)}
\def\langnames@langs@glot@xoi{Kominimung}
\def\langnames@langs@glot@kmw{Komo (Democratic Republic of Congo)}
\def\langnames@langs@glot@kvh{Komodo}
\def\langnames@langs@glot@kvp{Kompane}
\def\langnames@langs@glot@kzv{Komyandaret}
\def\langnames@langs@glot@kxw{Konai}
\def\langnames@langs@glot@knd{Yaben (Konda)}
\def\langnames@langs@glot@kdw{Koneraw}
\def\langnames@langs@glot@klk{Kono (Nigeria)}
\def\langnames@langs@glot@kcz{Konongo-Ruwila}
\def\langnames@langs@glot@knu{Kono (Guinea)}
\def\langnames@langs@glot@kno{Kono (Sierra Leone)}
\def\langnames@langs@glot@koa{Konomala}
\def\langnames@langs@glot@kxc{Konso}
\def\langnames@langs@glot@nbe{Konyak Naga}
\def\langnames@langs@glot@mku{Konyanka Maninka}
\def\langnames@langs@glot@koo{Konzo}
\def\langnames@langs@glot@ozm{Koonzime}
\def\langnames@langs@glot@fuj{Ko}
\def\langnames@langs@glot@xop{Kopar}
\def\langnames@langs@glot@opk{Kopkaka}
\def\langnames@langs@glot@kcy{Korandje}
\def\langnames@langs@glot@koz{Korak}
\def\langnames@langs@glot@okh{Karanic}
\def\langnames@langs@glot@vkp{Korlai Portuguese}
\def\langnames@langs@glot@ktl{Koroshi}
\def\langnames@langs@glot@krp{Korop}
\def\langnames@langs@glot@kfo{Koro (Côte d'Ivoire)}
\def\langnames@langs@glot@krf{Koro-Olrat}
\def\langnames@langs@glot@xkq{Koroni}
\def\langnames@langs@glot@kqj{Koromira}
\def\langnames@langs@glot@jkr{Koro}
\def\langnames@langs@glot@vkn{Koro Nulu}
\def\langnames@langs@glot@vkz{Koro Zuba}
\def\langnames@langs@glot@kfd{Korra Koraga}
\def\langnames@langs@glot@kpq{Korupun-Sela}
\def\langnames@langs@glot@xor{Korubo}
\def\langnames@langs@glot@kfp{Korwa}
\def\langnames@langs@glot@kiq{Kosadle}
\def\langnames@langs@glot@kid{Koshin}
\def\langnames@langs@glot@kqk{Kotafon Gbe}
\def\langnames@langs@glot@koq{Kota (Gabon)}
\def\langnames@langs@glot@mqg{Kota Bangun Kutai Malay}
\def\langnames@langs@glot@grm{Kota Marudu Talantang}
\def\langnames@langs@glot@avk{Kotava}
\def\langnames@langs@glot@zko{Kott-Assan}
\def\langnames@langs@glot@kyf{Kouya}
\def\langnames@langs@glot@kqb{Kovai}
\def\langnames@langs@glot@kvc{Kove}
\def\langnames@langs@glot@xow{Kowaki}
\def\langnames@langs@glot@kwh{Kowiai}
\def\langnames@langs@glot@kga{Koyaga}
\def\langnames@langs@glot@koh{Koyo}
\def\langnames@langs@glot@kqd{Koy Sanjaq Jewish Neo-Aramaic}
\def\langnames@langs@glot@kuw{Kpagua}
\def\langnames@langs@glot@kpl{Kpala}
\def\langnames@langs@glot@pbn{Kpasam}
\def\langnames@langs@glot@koc{Kpati}
\def\langnames@langs@glot@cpo{Kpeego}
\def\langnames@langs@glot@kef{Kpessi}
\def\langnames@langs@glot@kph{Kplang}
\def\langnames@langs@glot@kye{Krache}
\def\langnames@langs@glot@rka{Kraol}
\def\langnames@langs@glot@xre{Northeastern Timbira}
\def\langnames@langs@glot@kri{Krio}
\def\langnames@langs@glot@kxb{Krobu}
\def\langnames@langs@glot@tyu{Southern Tshwa}
\def\langnames@langs@glot@yku{Kuamasi}
\def\langnames@langs@glot@uan{Kuan}
\def\langnames@langs@glot@kua{Kuanyama}
\def\langnames@langs@glot@ykn{Kua-nsi}
\def\langnames@langs@glot@ugh{Kubachi}
\def\langnames@langs@glot@kgf{Kulungtfu-Yuanggeng-Tobo}
\def\langnames@langs@glot@kof{Kubi}
\def\langnames@langs@glot@jko{Kubo}
\def\langnames@langs@glot@kvb{Kubu}
\def\langnames@langs@glot@lkc{Kucong}
\def\langnames@langs@glot@kfg{Kudiya}
\def\langnames@langs@glot@kyw{Kudmali}
\def\langnames@langs@glot@kov{Kudu-Camo}
\def\langnames@langs@glot@kow{Gengle-Kugama}
\def\langnames@langs@glot@kes{Kugbo}
\def\langnames@langs@glot@dkr{Kuijau}
\def\langnames@langs@glot@vkj{Kujarge}
\def\langnames@langs@glot@kux{Kukatja}
\def\langnames@langs@glot@kez{Kukele}
\def\langnames@langs@glot@kfn{Kuk}
\def\langnames@langs@glot@ugb{Kuku-Ugbanh}
\def\langnames@langs@glot@xmp{Kuku-Mu'inh}
\def\langnames@langs@glot@xmh{Kuku-Muminh}
\def\langnames@langs@glot@ukv{Kuku}
\def\langnames@langs@glot@kul{Kulere}
\def\langnames@langs@glot@kxj{Kulfa}
\def\langnames@langs@glot@vkl{Kulisusu}
\def\langnames@langs@glot@xpk{Kulina Pano}
\def\langnames@langs@glot@kfx{Kullu Pahari}
\def\langnames@langs@glot@pzh{Pazeh-Kahabu}
\def\langnames@langs@glot@uon{Kulon}
\def\langnames@langs@glot@bbu{Kulung (Nigeria)}
\def\langnames@langs@glot@kdi{Kumam}
\def\langnames@langs@glot@ksl{Kumalu}
\def\langnames@langs@glot@ksm{Kumba}
\def\langnames@langs@glot@xks{Kumbewaha}
\def\langnames@langs@glot@kra{Kumhali}
\def\langnames@langs@glot@kuo{Kumukio}
\def\langnames@langs@glot@zum{Kumzari}
\def\langnames@langs@glot@wku{Kunduvadi}
\def\langnames@langs@glot@kdn{Chikunda}
\def\langnames@langs@glot@shd{Kundal Shahi}
\def\langnames@langs@glot@kgl{Kunggari}
\def\langnames@langs@glot@ggk{Kungarakany}
\def\langnames@langs@glot@kfl{Kung}
\def\langnames@langs@glot@kse{Kuni}
\def\langnames@langs@glot@xug{Kunigami}
\def\langnames@langs@glot@pep{Kánchá}
\def\langnames@langs@glot@njx{Kunyi}
\def\langnames@langs@glot@kug{Kupa}
\def\langnames@langs@glot@mkn{Kupang Malay}
\def\langnames@langs@glot@key{Kupia}
\def\langnames@langs@glot@nqk{Kura Ede Nago}
\def\langnames@langs@glot@krh{Kurama}
\def\langnames@langs@glot@kfh{Kurichiya}
\def\langnames@langs@glot@kuj{Kuria}
\def\langnames@langs@glot@nbn{Nabi}
\def\langnames@langs@glot@kfv{Kurmukar}
\def\langnames@langs@glot@vku{Kurrama}
\def\langnames@langs@glot@kuv{Kur}
\def\langnames@langs@glot@xkz{Kurtokha}
\def\langnames@langs@glot@ktm{Kurti}
\def\langnames@langs@glot@kjr{Kurudu}
\def\langnames@langs@glot@kyr{Kuruáya}
\def\langnames@langs@glot@kus{Kusaal}
\def\langnames@langs@glot@ksg{Kusaghe-Njela}
\def\langnames@langs@glot@kuh{Kushi}
\def\langnames@langs@glot@ksv{Kusu}
\def\langnames@langs@glot@ght{Kutang Ghale}
\def\langnames@langs@glot@kub{Kutep}
\def\langnames@langs@glot@xut{Kuthant}
\def\langnames@langs@glot@kpa{Kutto}
\def\langnames@langs@glot@khj{Kuturmi}
\def\langnames@langs@glot@kdc{Kutu}
\def\langnames@langs@glot@uky{Kuuk-Yak}
\def\langnames@langs@glot@lku{Kuungkari of Barcoo River}
\def\langnames@langs@glot@olu{Kuvale}
\def\langnames@langs@glot@cwt{Kuwaataay}
\def\langnames@langs@glot@blh{Kuwaa}
\def\langnames@langs@glot@kdt{Kuy}
\def\langnames@langs@glot@fkv{Kven Finnish}
\def\langnames@langs@glot@kwb{Baa}
\def\langnames@langs@glot@bko{Kwa'}
\def\langnames@langs@glot@kwz{Kwadi}
\def\langnames@langs@glot@wka{Kw'adza}
\def\langnames@langs@glot@kdz{Kwaja-Ndaktup}
\def\langnames@langs@glot@kwu{Kwakum}
\def\langnames@langs@glot@qwt{Kwalhioqua-Clatskanie}
\def\langnames@langs@glot@kmq{Gwama}
\def\langnames@langs@glot@ktf{Kwami}
\def\langnames@langs@glot@kwm{Kwambi}
\def\langnames@langs@glot@okk{Kwamtim One}
\def\langnames@langs@glot@knp{Kwanja}
\def\langnames@langs@glot@kwj{Kwanga}
\def\langnames@langs@glot@kvi{Kwang}
\def\langnames@langs@glot@xdo{Kwandu}
\def\langnames@langs@glot@kwf{Kwara'ae}
\def\langnames@langs@glot@kop{Kwato}
\def\langnames@langs@glot@kya{Kwaya}
\def\langnames@langs@glot@cwe{Kwere}
\def\langnames@langs@glot@xwr{Kwerba Mamberamo}
\def\langnames@langs@glot@kkb{Kwerisa}
\def\langnames@langs@glot@kwr{Kwer}
\def\langnames@langs@glot@kws{Kwese}
\def\langnames@langs@glot@kwt{Kwesten}
\def\langnames@langs@glot@kuc{Kwinsu}
\def\langnames@langs@glot@kww{Kwinti}
\def\langnames@langs@glot@bka{Kyak}
\def\langnames@langs@glot@tye{Kyenga}
\def\langnames@langs@glot@kql{Kyenele}
\def\langnames@langs@glot@ldn{Láadan}
\def\langnames@langs@glot@bwj{Láá Láá Bwamu}
\def\langnames@langs@glot@ldi{Laari}
\def\langnames@langs@glot@lbb{Label}
\def\langnames@langs@glot@lbi{La'bi}
\def\langnames@langs@glot@jku{Labir}
\def\langnames@langs@glot@ypb{Labo Phowa}
\def\langnames@langs@glot@mwi{Ninde}
\def\langnames@langs@glot@dtb{Labuk-Kinabatangan Kadazan}
\def\langnames@langs@glot@zpl{Lachixío Zapotec}
\def\langnames@langs@glot@zpa{Lachiguiri Zapotec}
\def\langnames@langs@glot@lkl{Laeko-Libuat}
\def\langnames@langs@glot@lgh{Laghuu}
\def\langnames@langs@glot@lgb{Laghu}
\def\langnames@langs@glot@lhh{Laha (Indonesia)}
\def\langnames@langs@glot@lhn{Lahanan}
\def\langnames@langs@glot@lhl{Lahul Lohar}
\def\langnames@langs@glot@lhi{Lahu Shi}
\def\langnames@langs@glot@lmx{Laimbue}
\def\langnames@langs@glot@lji{Laiyolo}
\def\langnames@langs@glot@lap{Laka (Chad)}
\def\langnames@langs@glot@lka{Lakalei}
\def\langnames@langs@glot@lkh{Lakha}
\def\langnames@langs@glot@lki{Laki}
\def\langnames@langs@glot@lkn{Lakon}
\def\langnames@langs@glot@lkd{Lakondê}
\def\langnames@langs@glot@lxm{Lakuramau}
\def\langnames@langs@glot@lla{Lala-Roba}
\def\langnames@langs@glot@leb{Lala-Bisa}
\def\langnames@langs@glot@cnl{Lalana Chinantec}
\def\langnames@langs@glot@las{Lama (Togo)}
\def\langnames@langs@glot@lmr{Peripheral Lembata}
\def\langnames@langs@glot@lmq{Lamatuka}
\def\langnames@langs@glot@lai{Lambya}
\def\langnames@langs@glot@lmy{Lamboya}
\def\langnames@langs@glot@quf{Lambayeque Quechua}
\def\langnames@langs@glot@lbn{Lamet}
\def\langnames@langs@glot@bma{Lame}
\def\langnames@langs@glot@ldh{Lamja-Dengsa-Tola}
\def\langnames@langs@glot@lmk{Lamkang}
\def\langnames@langs@glot@lev{Western Pantar}
\def\langnames@langs@glot@lmg{Lamogai}
\def\langnames@langs@glot@abl{Lampung Nyo}
\def\langnames@langs@glot@llh{Lamu}
\def\langnames@langs@glot@ruu{Lanas Lobu}
\def\langnames@langs@glot@ldm{Landoma}
\def\langnames@langs@glot@sfb{Langue des signes de Belgique Francophone}
\def\langnames@langs@glot@yln{Langnian Buyang}
\def\langnames@langs@glot@lna{Langbashe}
\def\langnames@langs@glot@lno{Lango-Logire-Logir}
\def\langnames@langs@glot@lnm{Pondi}
\def\langnames@langs@glot@lnh{Lanoh}
\def\langnames@langs@glot@lwm{Laomian}
\def\langnames@langs@glot@ztl{Lapaguía-Guivini Zapotec}
\def\langnames@langs@glot@laa{Lapuyan Subanun}
\def\langnames@langs@glot@lrt{Larantuka Malay}
\def\langnames@langs@glot@lrv{Larevat}
\def\langnames@langs@glot@hmd{Diandongbei-Large Flowery Miao}
\def\langnames@langs@glot@lrl{Larestani}
\def\langnames@langs@glot@lro{Laru (North Sudan)}
\def\langnames@langs@glot@lar{Larteh}
\def\langnames@langs@glot@lan{Laru (Nigeria)}
\def\langnames@langs@glot@llm{Lasalimu}
\def\langnames@langs@glot@lsa{Lasgerdi}
\def\langnames@langs@glot@lsi{Lashi}
\def\langnames@langs@glot@lss{Lasi}
\def\langnames@langs@glot@lat{Latin}
\def\langnames@langs@glot@ltu{Latu}
\def\langnames@langs@glot@ltn{Latundê}
\def\langnames@langs@glot@lsl{Latvian Sign Language}
\def\langnames@langs@glot@llx{Lauan}
\def\langnames@langs@glot@luf{Laua}
\def\langnames@langs@glot@lre{Laurentian}
\def\langnames@langs@glot@clt{Lautu}
\def\langnames@langs@glot@lbv{Lavatbura-Lamusong}
\def\langnames@langs@glot@lbx{Lawangan}
\def\langnames@langs@glot@lvi{Lawi}
\def\langnames@langs@glot@tgi{Lawunuia}
\def\langnames@langs@glot@lwu{Lawu}
\def\langnames@langs@glot@lya{Layakha}
\def\langnames@langs@glot@ldk{Leelau}
\def\langnames@langs@glot@lfa{Lefa}
\def\langnames@langs@glot@lgm{Lega-Mwenga}
\def\langnames@langs@glot@lcc{Legenyem}
\def\langnames@langs@glot@cae{Lehar}
\def\langnames@langs@glot@tql{Lehali}
\def\langnames@langs@glot@urr{Lehalurup}
\def\langnames@langs@glot@lzn{Leinong Naga}
\def\langnames@langs@glot@lek{Leipon}
\def\langnames@langs@glot@llk{Lelak}
\def\langnames@langs@glot@lel{Lele (Democratic Republic of Congo)}
\def\langnames@langs@glot@llc{Lele (Guinea)}
\def\langnames@langs@glot@lpa{Lelepa}
\def\langnames@langs@glot@lle{Lele (Papua New Guinea)}
\def\langnames@langs@glot@leq{Lembena}
\def\langnames@langs@glot@lrz{Lemerig}
\def\langnames@langs@glot@lei{Lemio}
\def\langnames@langs@glot@xle{Lemnian}
\def\langnames@langs@glot@ldj{Lemoro}
\def\langnames@langs@glot@ley{Lemolang}
\def\langnames@langs@glot@lej{Lengola}
\def\langnames@langs@glot@lgr{Lengo}
\def\langnames@langs@glot@lgi{Lengilu}
\def\langnames@langs@glot@leh{Lenje}
\def\langnames@langs@glot@ler{Lenkau}
\def\langnames@langs@glot@ldg{Lenyima}
\def\langnames@langs@glot@lpe{Lepki}
\def\langnames@langs@glot@xlp{Lepontic}
\def\langnames@langs@glot@gnh{Lere}
\def\langnames@langs@glot@let{Lesing-Gelimi}
\def\langnames@langs@glot@nms{Letemboi-Repanbitip}
\def\langnames@langs@glot@leo{Leti (Cameroon)}
\def\langnames@langs@glot@lvu{Central Lembata-Lewokukun}
\def\langnames@langs@glot@lwe{Lewo Eleng}
\def\langnames@langs@glot@lwt{Lewotobi}
\def\langnames@langs@glot@ayi{Leyigha}
\def\langnames@langs@glot@lhp{Lhokpu}
\def\langnames@langs@glot@lix{Liabuku}
\def\langnames@langs@glot@njn{Liangmai Naga}
\def\langnames@langs@glot@zln{Lianshan Zhuang}
\def\langnames@langs@glot@ste{Liana-Seti}
\def\langnames@langs@glot@lir{Kru Pidgin English}
\def\langnames@langs@glot@liz{Libinza}
\def\langnames@langs@glot@liq{Libido}
\def\langnames@langs@glot@lbs{Libyan Sign Language}
\def\langnames@langs@glot@lig{Ligbi}
\def\langnames@langs@glot@lgz{Ligenza}
\def\langnames@langs@glot@lih{Lihir}
\def\langnames@langs@glot@mgi{Lijili}
\def\langnames@langs@glot@lik{Liko}
\def\langnames@langs@glot@lie{Balobo}
\def\langnames@langs@glot@lio{Liki}
\def\langnames@langs@glot@kxx{Likuba}
\def\langnames@langs@glot@lib{Likum}
\def\langnames@langs@glot@kwc{Likwala}
\def\langnames@langs@glot@lll{Lilau}
\def\langnames@langs@glot@bme{Limassa}
\def\langnames@langs@glot@lim{Limburgan}
\def\langnames@langs@glot@lmp{Limbum}
\def\langnames@langs@glot@ylm{Limi}
\def\langnames@langs@glot@kmk{Limos Kalinga}
\def\langnames@langs@glot@qlm{Limonese Creole}
\def\langnames@langs@glot@klw{Tado-Lindu}
\def\langnames@langs@glot@pml{Mediterranean Lingua Franca}
\def\langnames@langs@glot@onb{Western Ong-Be}
\def\langnames@langs@glot@lgk{Neverver}
\def\langnames@langs@glot@lfn{Lingua Franca Nova}
\def\langnames@langs@glot@ljl{Li'o}
\def\langnames@langs@glot@apl{Lipan Apache}
\def\langnames@langs@glot@lpo{Lipo}
\def\langnames@langs@glot@lcs{Lisabata-Nuniali}
\def\langnames@langs@glot@lcl{Lisela}
\def\langnames@langs@glot@lsh{Khispi}
\def\langnames@langs@glot@lsd{Lishana Deni}
\def\langnames@langs@glot@lzh{Literary Chinese}
\def\langnames@langs@glot@lls{Lithuanian Sign Language}
\def\langnames@langs@glot@lzl{Naman}
\def\langnames@langs@glot@zlj{Liujiang Zhuang}
\def\langnames@langs@glot@zlq{Liuqian Zhuang}
\def\langnames@langs@glot@olo{Livvi}
\def\langnames@langs@glot@loq{Lobala}
\def\langnames@langs@glot@lbm{Lodhi}
\def\langnames@langs@glot@lgq{Ikpana}
\def\langnames@langs@glot@rag{Logooli}
\def\langnames@langs@glot@liu{Logorik}
\def\langnames@langs@glot@lof{Logol}
\def\langnames@langs@glot@src{Logudorese Sardinian}
\def\langnames@langs@glot@qvj{Loja Highland Quichua}
\def\langnames@langs@glot@jbo{Lojban}
\def\langnames@langs@glot@yaz{Lokaa}
\def\langnames@langs@glot@lky{Lokoya}
\def\langnames@langs@glot@lcd{Lola}
\def\langnames@langs@glot@llq{Lolak}
\def\langnames@langs@glot@llg{Lole}
\def\langnames@langs@glot@ycl{Lolopo}
\def\langnames@langs@glot@llb{Lolo}
\def\langnames@langs@glot@loa{Loloda-Laba}
\def\langnames@langs@glot@rmi{Lomavren}
\def\langnames@langs@glot@loi{Loma (Côte d'Ivoire)}
\def\langnames@langs@glot@lmv{Lomaiviti}
\def\langnames@langs@glot@lmi{Lombi}
\def\langnames@langs@glot@lmo{Lombard}
\def\langnames@langs@glot@loo{Lombo}
\def\langnames@langs@glot@ngl{Mozambique Lomwe}
\def\langnames@langs@glot@lce{Loncong}
\def\langnames@langs@glot@lpn{Long Phuri Naga}
\def\langnames@langs@glot@wok{Longto}
\def\langnames@langs@glot@lnu{Longuda}
\def\langnames@langs@glot@ttw{Western Lowland Kenyah}
\def\langnames@langs@glot@ldo{Loo}
\def\langnames@langs@glot@lop{Lopa}
\def\langnames@langs@glot@lpx{Lopit}
\def\langnames@langs@glot@lrn{Lorang}
\def\langnames@langs@glot@spq{Peruvian Amazonian Spanish}
\def\langnames@langs@glot@lnn{Nethalp}
\def\langnames@langs@glot@uvl{Lote}
\def\langnames@langs@glot@lht{Lo-Toga}
\def\langnames@langs@glot@dtr{Lotud}
\def\langnames@langs@glot@lou{Louisiana Creole French}
\def\langnames@langs@glot@lox{Loun}
\def\langnames@langs@glot@xlo{Loup A}
\def\langnames@langs@glot@sli{Lower Silesian}
\def\langnames@langs@glot@tto{Lower Ta'oih}
\def\langnames@langs@glot@nsb{Lower-Nosop}
\def\langnames@langs@glot@kml{Tanudan Kalinga}
\def\langnames@langs@glot@cea{Lower Chehalis}
\def\langnames@langs@glot@axl{Lower Southern Aranda}
\def\langnames@langs@glot@ztp{Loxicha Zapotec}
\def\langnames@langs@glot@kcc{Lubila}
\def\langnames@langs@glot@lcf{Lubu}
\def\langnames@langs@glot@knb{Lubuagan Kalinga}
\def\langnames@langs@glot@luq{Lucumi}
\def\langnames@langs@glot@lud{Ludian}
\def\langnames@langs@glot@ldq{Lufu}
\def\langnames@langs@glot@ruf{Luguru}
\def\langnames@langs@glot@lcq{Luhu-Piru}
\def\langnames@langs@glot@lum{Luimbi}
\def\langnames@langs@glot@dop{Lukpa}
\def\langnames@langs@glot@smj{Lule Saami}
\def\langnames@langs@glot@lmz{Lumbee}
\def\langnames@langs@glot@lup{Lumbu}
\def\langnames@langs@glot@lmd{Lumun}
\def\langnames@langs@glot@luk{Lunanakha}
\def\langnames@langs@glot@luj{Luna}
\def\langnames@langs@glot@lga{Lungga}
\def\langnames@langs@glot@luw{Luo (Cameroon)}
\def\langnames@langs@glot@hml{Luopohe Hmong}
\def\langnames@langs@glot@ldd{Luri}
\def\langnames@langs@glot@lse{Lusengo}
\def\langnames@langs@glot@xls{Lusitanian}
\def\langnames@langs@glot@ndy{Lutos}
\def\langnames@langs@glot@luv{Luwati}
\def\langnames@langs@glot@lyn{Luyi}
\def\langnames@langs@glot@lwa{Lwalu}
\def\langnames@langs@glot@xlc{Lycian A}
\def\langnames@langs@glot@xld{Lydian}
\def\langnames@langs@glot@lyg{India Lyngam}
\def\langnames@langs@glot@cma{Maa}
\def\langnames@langs@glot@mew{Maaka}
\def\langnames@langs@glot@ymm{Maay}
\def\langnames@langs@glot@mmz{Mabaale}
\def\langnames@langs@glot@mfz{Mabaan}
\def\langnames@langs@glot@mqa{Maba (Indonesia)}
\def\langnames@langs@glot@kkg{Mabaka Valley Kalinga}
\def\langnames@langs@glot@muj{Mabire}
\def\langnames@langs@glot@mcl{Macaguaje}
\def\langnames@langs@glot@mzs{Macanese}
\def\langnames@langs@glot@mvw{Machinga}
\def\langnames@langs@glot@jmc{Machame}
\def\langnames@langs@glot@mpd{Machinere}
\def\langnames@langs@glot@wpc{Maco}
\def\langnames@langs@glot@mzc{Madagascar Sign Language}
\def\langnames@langs@glot@mmx{Madak}
\def\langnames@langs@glot@xmx{Salawati}
\def\langnames@langs@glot@grg{Madi (Papua New Guinea)}
\def\langnames@langs@glot@kmd{Madukayang Kalinga}
\def\langnames@langs@glot@mme{Tirax}
\def\langnames@langs@glot@itt{Maeng Itneg}
\def\langnames@langs@glot@maf{Mafa}
\def\langnames@langs@glot@mkv{Mafea}
\def\langnames@langs@glot@sgb{Mag-Anchi Ayta}
\def\langnames@langs@glot@mtw{Southern Binukidnon}
\def\langnames@langs@glot@xtm{Magdalena Peñasco Mixtec}
\def\langnames@langs@glot@gmd{Mághdì}
\def\langnames@langs@glot@blx{Mag-Indi Ayta}
\def\langnames@langs@glot@gkd{Magi}
\def\langnames@langs@glot@gmg{Magiyi}
\def\langnames@langs@glot@gmx{Magoma}
\def\langnames@langs@glot@zgr{Magori}
\def\langnames@langs@glot@bfz{Mahasu Pahari}
\def\langnames@langs@glot@mjx{Mahali}
\def\langnames@langs@glot@pmh{Maharastri Prakrit}
\def\langnames@langs@glot@mjy{Mohican}
\def\langnames@langs@glot@mhb{Mahongwe}
\def\langnames@langs@glot@mzz{Maiadomu}
\def\langnames@langs@glot@tnh{Maiani}
\def\langnames@langs@glot@sks{Maia}
\def\langnames@langs@glot@mmm{Maii}
\def\langnames@langs@glot@vmf{Ostfränkisch}
\def\langnames@langs@glot@cwb{Maindo}
\def\langnames@langs@glot@xkl{Usun Apau Kenyah}
\def\langnames@langs@glot@mum{Maiwala}
\def\langnames@langs@glot@wmm{Maiwa (Indonesia)}
\def\langnames@langs@glot@mti{Maiwa (Papua New Guinea)}
\def\langnames@langs@glot@xmj{Majera}
\def\langnames@langs@glot@mmj{Majhwar}
\def\langnames@langs@glot@mjz{Majhi}
\def\langnames@langs@glot@mfp{Makassar Malay}
\def\langnames@langs@glot@aup{Makayam}
\def\langnames@langs@glot@mkg{Mak (China)}
\def\langnames@langs@glot@vmk{Makhuwa-Shirima}
\def\langnames@langs@glot@xmc{Makhuwa-Marrevone}
\def\langnames@langs@glot@vmw{Makhuwa}
\def\langnames@langs@glot@mhm{Makhuwa-Moniga}
\def\langnames@langs@glot@xsq{Makhuwa-Saka}
\def\langnames@langs@glot@pbl{Mak (Nigeria)}
\def\langnames@langs@glot@zmh{Makolkol}
\def\langnames@langs@glot@jmn{Makuri Naga}
\def\langnames@langs@glot@lva{Maku'a}
\def\langnames@langs@glot@mpu{Makuráp}
\def\langnames@langs@glot@ymk{Makwe}
\def\langnames@langs@glot@umn{Makyan Naga}
\def\langnames@langs@glot@lon{Malawi Lomwe}
\def\langnames@langs@glot@xml{Malaysian Sign Language}
\def\langnames@langs@glot@ima{Mala Malasar}
\def\langnames@langs@glot@ymr{Malasar}
\def\langnames@langs@glot@mjo{Malankuravan}
\def\langnames@langs@glot@mjr{Malavedan}
\def\langnames@langs@glot@mjq{Malaryan}
\def\langnames@langs@glot@mjp{Malapandaram}
\def\langnames@langs@glot@ruy{Mala (Nigeria)}
\def\langnames@langs@glot@swk{Malawi Sena}
\def\langnames@langs@glot@ccm{Malaccan Creole Malay}
\def\langnames@langs@glot@mln{Malango}
\def\langnames@langs@glot@mqz{Malasanga}
\def\langnames@langs@glot@mmt{Malalamai}
\def\langnames@langs@glot@ped{Mala (Papua New Guinea)}
\def\langnames@langs@glot@mkr{Manep}
\def\langnames@langs@glot@lws{Malawian Sign Language}
\def\langnames@langs@glot@bfo{Malba Birifor}
\def\langnames@langs@glot@pkt{Maleng}
\def\langnames@langs@glot@mdc{Male (Papua New Guinea)}
\def\langnames@langs@glot@gut{Maléku Jaíka}
\def\langnames@langs@glot@mlx{Na'ahai}
\def\langnames@langs@glot@vml{Malgana}
\def\langnames@langs@glot@mxf{Malgbe}
\def\langnames@langs@glot@mgq{Malila}
\def\langnames@langs@glot@mzd{Malimba}
\def\langnames@langs@glot@mli{Malimpung}
\def\langnames@langs@glot@mlf{Mal}
\def\langnames@langs@glot@mbk{Malol}
\def\langnames@langs@glot@mkb{Mar Paharia of Dumka}
\def\langnames@langs@glot@mdl{Maltese Sign Language}
\def\langnames@langs@glot@mll{Malua Bay}
\def\langnames@langs@glot@mup{Malvi}
\def\langnames@langs@glot@myk{Mamara Senoufo}
\def\langnames@langs@glot@mma{Mama}
\def\langnames@langs@glot@mhf{Mamaa}
\def\langnames@langs@glot@wmd{Mamaindé}
\def\langnames@langs@glot@mvd{Mamboru}
\def\langnames@langs@glot@mgm{Mambae}
\def\langnames@langs@glot@kdf{Mamusi}
\def\langnames@langs@glot@mqx{Mamuju}
\def\langnames@langs@glot@znk{Manangkari}
\def\langnames@langs@glot@mjl{Mandeali}
\def\langnames@langs@glot@mha{Manda (India)}
\def\langnames@langs@glot@zma{Manda (Australia)}
\def\langnames@langs@glot@zmk{Mandandanyi}
\def\langnames@langs@glot@mgs{Manda-Matumba}
\def\langnames@langs@glot@mqu{Mandari}
\def\langnames@langs@glot@tbf{Mandara}
\def\langnames@langs@glot@mqr{Mander}
\def\langnames@langs@glot@aax{Mandobo Atas}
\def\langnames@langs@glot@bwp{Mandobo Bawah}
\def\langnames@langs@glot@mht{Mandahuaca}
\def\langnames@langs@glot@zng{Mang}
\def\langnames@langs@glot@zme{Mangerr}
\def\langnames@langs@glot@mem{Mangala}
\def\langnames@langs@glot@myj{Mangayat}
\def\langnames@langs@glot@mdk{Mangbutu}
\def\langnames@langs@glot@kby{Manga Kanuri}
\def\langnames@langs@glot@mrv{Mangareva}
\def\langnames@langs@glot@mbh{Mangseng}
\def\langnames@langs@glot@mmo{Mangga Buang}
\def\langnames@langs@glot@zns{Mangas}
\def\langnames@langs@glot@xkb{Manigri-Kambolé Ede Nago}
\def\langnames@langs@glot@mqp{Manipa}
\def\langnames@langs@glot@nlm{Mankiyali}
\def\langnames@langs@glot@mml{Man Met}
\def\langnames@langs@glot@mjv{Mannan}
\def\langnames@langs@glot@woo{Manombai}
\def\langnames@langs@glot@msw{Mansoanka}
\def\langnames@langs@glot@msk{Mansaka}
\def\langnames@langs@glot@nty{Mantsi}
\def\langnames@langs@glot@myg{Manta}
\def\langnames@langs@glot@kxf{Manumanaw Karen}
\def\langnames@langs@glot@wha{Manusela}
\def\langnames@langs@glot@mxc{Manyika}
\def\langnames@langs@glot@mny{Manyawa}
\def\langnames@langs@glot@mzj{Manya}
\def\langnames@langs@glot@mzv{Manza}
\def\langnames@langs@glot@mmd{Maonan}
\def\langnames@langs@glot@mjn{Ma (Papua New Guinea)}
\def\langnames@langs@glot@mlh{Mape}
\def\langnames@langs@glot@mnm{Mapena}
\def\langnames@langs@glot@mpy{Mapia}
\def\langnames@langs@glot@mpw{Mapidian-Mawayana}
\def\langnames@langs@glot@bzh{Mapos Buang}
\def\langnames@langs@glot@sjm{Mapun}
\def\langnames@langs@glot@vmh{Maraghei}
\def\langnames@langs@glot@nma{Maram Naga}
\def\langnames@langs@glot@lrm{Marama}
\def\langnames@langs@glot@lri{Marachi}
\def\langnames@langs@glot@mgb{Mararit}
\def\langnames@langs@glot@mvr{Marau}
\def\langnames@langs@glot@mrs{Maragus}
\def\langnames@langs@glot@mpg{Marba}
\def\langnames@langs@glot@dsz{Mardin Sign Language}
\def\langnames@langs@glot@vmr{Marenje}
\def\langnames@langs@glot@mrx{Maremgi}
\def\langnames@langs@glot@mvu{Marfa}
\def\langnames@langs@glot@mhg{Margu}
\def\langnames@langs@glot@qvm{Margos-Yarowilca-Lauricocha Quechua}
\def\langnames@langs@glot@mfm{Marghi South}
\def\langnames@langs@glot@nsr{Maritime Sign Language}
\def\langnames@langs@glot@mrr{Maria (India)}
\def\langnames@langs@glot@nng{Maring Naga}
\def\langnames@langs@glot@zmm{Marimanindji}
\def\langnames@langs@glot@zmj{Maridjabin}
\def\langnames@langs@glot@zmd{Maridan}
\def\langnames@langs@glot@zmy{Mariyedi}
\def\langnames@langs@glot@mrb{Sunwadia}
\def\langnames@langs@glot@dad{Marik}
\def\langnames@langs@glot@hob{Mari (Madang Province)}
\def\langnames@langs@glot@mqi{Mariri}
\def\langnames@langs@glot@mbx{Mari (East Sepik Province)}
\def\langnames@langs@glot@mds{Maria (Papua New Guinea)}
\def\langnames@langs@glot@msp{Maritsauá}
\def\langnames@langs@glot@enb{Markweeta}
\def\langnames@langs@glot@rkm{Marka}
\def\langnames@langs@glot@mvo{Marovo}
\def\langnames@langs@glot@xru{Marriammu}
\def\langnames@langs@glot@mre{Martha's Vineyard Sign Language}
\def\langnames@langs@glot@zmg{Marti Ke}
\def\langnames@langs@glot@mzr{Marúbo}
\def\langnames@langs@glot@mve{Marwari (Pakistan)}
\def\langnames@langs@glot@rwr{Marwari (India)}
\def\langnames@langs@glot@myx{Masaaba}
\def\langnames@langs@glot@tis{Masadiit Itneg}
\def\langnames@langs@glot@bks{Masbate Sorsogon}
\def\langnames@langs@glot@msb{Masbatenyo}
\def\langnames@langs@glot@mho{Mashi (Zambia)}
\def\langnames@langs@glot@jms{Mashi (Nigeria)}
\def\langnames@langs@glot@cuj{Mashco Piro}
\def\langnames@langs@glot@ism{Masimasi}
\def\langnames@langs@glot@bnf{Masiwang}
\def\langnames@langs@glot@msh{Masikoro Malagasy}
\def\langnames@langs@glot@klv{Maskelynes}
\def\langnames@langs@glot@msv{Maslam}
\def\langnames@langs@glot@mes{Masmaje}
\def\langnames@langs@glot@mdg{Massalat}
\def\langnames@langs@glot@mvs{Massep}
\def\langnames@langs@glot@mtn{Matagalpa}
\def\langnames@langs@glot@mfh{Matal}
\def\langnames@langs@glot@xmt{Matbat}
\def\langnames@langs@glot@mgv{Matengo}
\def\langnames@langs@glot@mqe{Matepi}
\def\langnames@langs@glot@mzo{Matipuhy}
\def\langnames@langs@glot@mtm{Mator-Taigi-Karagas}
\def\langnames@langs@glot@met{Mato}
\def\langnames@langs@glot@axg{Mato Grosso Arára}
\def\langnames@langs@glot@stj{Matya Samo}
\def\langnames@langs@glot@cty{Maundadan Chetti}
\def\langnames@langs@glot@lsy{Mauritian Sign Language}
\def\langnames@langs@glot@mhl{Mauwake}
\def\langnames@langs@glot@wma{Mawa (Nigeria)}
\def\langnames@langs@glot@mjj{Mawak}
\def\langnames@langs@glot@mcz{Mawan}
\def\langnames@langs@glot@mcw{Mawa (Chad)}
\def\langnames@langs@glot@mgk{Mawes}
\def\langnames@langs@glot@mxl{Maxi Gbe}
\def\langnames@langs@glot@xmy{Mayaguduna}
\def\langnames@langs@glot@sym{Maya Samo}
\def\langnames@langs@glot@mnt{Maykulan}
\def\langnames@langs@glot@ifu{Mayoyao Ifugao}
\def\langnames@langs@glot@mzl{Mazatlán Mixe}
\def\langnames@langs@glot@zpy{Mazaltepec Zapotec}
\def\langnames@langs@glot@vmz{Mazatlán Mazatec}
\def\langnames@langs@glot@dkx{Mazagway}
\def\langnames@langs@glot@mdp{Mbala}
\def\langnames@langs@glot@mgn{Mbangi}
\def\langnames@langs@glot@zmz{Mbandja}
\def\langnames@langs@glot@mxg{Mbangala}
\def\langnames@langs@glot@zmn{Mbangwe}
\def\langnames@langs@glot@zmv{Rimanggudhinma}
\def\langnames@langs@glot@mvl{Mbara-Yanga}
\def\langnames@langs@glot@gwa{Mbato}
\def\langnames@langs@glot@mdn{Mbati}
\def\langnames@langs@glot@xmd{Mbedam}
\def\langnames@langs@glot@mfo{Mbe}
\def\langnames@langs@glot@mql{Mbelime}
\def\langnames@langs@glot@zms{Mbesa}
\def\langnames@langs@glot@emz{Mbessa}
\def\langnames@langs@glot@mbo{Mbo (Cameroon)}
\def\langnames@langs@glot@zmw{Mbo (Democratic Republic of Congo)}
\def\langnames@langs@glot@moi{Mboi}
\def\langnames@langs@glot@mdu{Mboko}
\def\langnames@langs@glot@xmb{Mbonga}
\def\langnames@langs@glot@bgu{Mbongno}
\def\langnames@langs@glot@mxo{Mbowe}
\def\langnames@langs@glot@mka{Mbre}
\def\langnames@langs@glot@mgz{Mbugwe}
\def\langnames@langs@glot@mhw{Mbukushu}
\def\langnames@langs@glot@mqb{Mbuko}
\def\langnames@langs@glot@bpc{Mbuk}
\def\langnames@langs@glot@mbv{Mbulungish}
\def\langnames@langs@glot@mbu{Mbula-Bwazza}
\def\langnames@langs@glot@mlb{Mbule}
\def\langnames@langs@glot@mgy{Mbunga}
\def\langnames@langs@glot@mck{Mbunda}
\def\langnames@langs@glot@bbt{Mburku}
\def\langnames@langs@glot@muc{Ajumbu}
\def\langnames@langs@glot@mfu{Mbwela}
\def\langnames@langs@glot@gun{Mbyá Guaraní}
\def\langnames@langs@glot@mjm{Medebur}
\def\langnames@langs@glot@dmf{Medefidrin}
\def\langnames@langs@glot@mue{Media Lengua}
\def\langnames@langs@glot@mud{Mednyj Aleut}
\def\langnames@langs@glot@byv{Medumba}
\def\langnames@langs@glot@mfj{Mefele}
\def\langnames@langs@glot@mef{Bangladesh Lyngam}
\def\langnames@langs@glot@ruq{Megleno Romanian}
\def\langnames@langs@glot@mmh{Mehináku}
\def\langnames@langs@glot@mvk{Mekmek}
\def\langnames@langs@glot@msf{Mekwei}
\def\langnames@langs@glot@hkn{Mel-Khaonh}
\def\langnames@langs@glot@mfx{Melo}
\def\langnames@langs@glot@med{Melpa}
\def\langnames@langs@glot@mby{Memoni}
\def\langnames@langs@glot@mfd{Mendankwe-Nkwen}
\def\langnames@langs@glot@xkd{Mendalam Kayan}
\def\langnames@langs@glot@sim{Mende (Papua New Guinea)}
\def\langnames@langs@glot@xmg{Mengaka}
\def\langnames@langs@glot@mee{Mengen}
\def\langnames@langs@glot@mea{Menka}
\def\langnames@langs@glot@mvx{Meoswar}
\def\langnames@langs@glot@mxm{Meramera}
\def\langnames@langs@glot@lmb{Merei}
\def\langnames@langs@glot@meq{Merey}
\def\langnames@langs@glot@mrm{Merlav}
\def\langnames@langs@glot@xmr{Meroitic}
\def\langnames@langs@glot@mnu{Mer}
\def\langnames@langs@glot@mer{Meru}
\def\langnames@langs@glot@wry{Merwari}
\def\langnames@langs@glot@iyo{Mesaka}
\def\langnames@langs@glot@mci{Mese}
\def\langnames@langs@glot@zim{Mesme}
\def\langnames@langs@glot@mys{Mesmes}
\def\langnames@langs@glot@mvz{Mesqan}
\def\langnames@langs@glot@cms{Messapic}
\def\langnames@langs@glot@mgo{Meta'}
\def\langnames@langs@glot@mxv{Metlatónoc Mixtec}
\def\langnames@langs@glot@mtr{Mewari}
\def\langnames@langs@glot@wtm{Mewati}
\def\langnames@langs@glot@mfs{Mexican Sign Language}
\def\langnames@langs@glot@zmf{Mfinu}
\def\langnames@langs@glot@nfu{Southern Mfumte}
\def\langnames@langs@glot@zam{Cuixtla-Xitla Zapotec}
\def\langnames@langs@glot@pla{Miani}
\def\langnames@langs@glot@xmi{Miarrã}
\def\langnames@langs@glot@mmc{Michoacán Mazahua}
\def\langnames@langs@glot@enm{Middle English}
\def\langnames@langs@glot@gml{Middle Low German}
\def\langnames@langs@glot@dum{Middle Dutch}
\def\langnames@langs@glot@mpl{Middle Watut}
\def\langnames@langs@glot@gmh{Middle High German}
\def\langnames@langs@glot@ltc{Middle Chinese}
\def\langnames@langs@glot@xng{Middle Mongol}
\def\langnames@langs@glot@dnt{Mid Grand Valley Dani}
\def\langnames@langs@glot@bjo{Mid-Southern Banda}
\def\langnames@langs@glot@mpp{Migabac}
\def\langnames@langs@glot@ymh{Mili}
\def\langnames@langs@glot@mlj{Miltu}
\def\langnames@langs@glot@iml{Miluk}
\def\langnames@langs@glot@imy{Milyan}
\def\langnames@langs@glot@mcv{Minanibai-Foia Foia}
\def\langnames@langs@glot@inm{Minaean}
\def\langnames@langs@glot@mnp{Min Bei Chinese}
\def\langnames@langs@glot@mpn{Mindiri}
\def\langnames@langs@glot@drc{Minderico}
\def\langnames@langs@glot@mko{Mingang Doso}
\def\langnames@langs@glot@vmg{Minigir}
\def\langnames@langs@glot@wii{Minidien}
\def\langnames@langs@glot@xxm{Minkin}
\def\langnames@langs@glot@omn{Minoan}
\def\langnames@langs@glot@mqq{Minokok}
\def\langnames@langs@glot@mnq{Minriq}
\def\langnames@langs@glot@mzt{Mintil}
\def\langnames@langs@glot@czo{Min Zhong Chinese}
\def\langnames@langs@glot@zgm{Minz Zhuang}
\def\langnames@langs@glot@yiq{Miqie}
\def\langnames@langs@glot@mwl{Mirandese}
\def\langnames@langs@glot@mvh{Mire}
\def\langnames@langs@glot@mmv{Miriti}
\def\langnames@langs@glot@rsm{Miriwoong Sign Language}
\def\langnames@langs@glot@mjs{Miship}
\def\langnames@langs@glot@mpx{Misima-Paneati}
\def\langnames@langs@glot@vmm{Mitlatongo Mixtec}
\def\langnames@langs@glot@mwu{Mittu}
\def\langnames@langs@glot@mpo{Miu}
\def\langnames@langs@glot@vmi{Miwa}
\def\langnames@langs@glot@mfg{Mixifore}
\def\langnames@langs@glot@mix{Mixtepec Mixtec}
\def\langnames@langs@glot@mvi{Miyako}
\def\langnames@langs@glot@ehs{Miyakubo Sign Language}
\def\langnames@langs@glot@soy{Miyobe}
\def\langnames@langs@glot@lhs{Mlahsô}
\def\langnames@langs@glot@kja{Mlap}
\def\langnames@langs@glot@mlo{Mlomp}
\def\langnames@langs@glot@mmu{Mmaala}
\def\langnames@langs@glot@bfm{Mmen}
\def\langnames@langs@glot@mfq{Moba}
\def\langnames@langs@glot@mod{Mobilian}
\def\langnames@langs@glot@ahm{Mobumrin Aizi}
\def\langnames@langs@glot@jkm{Mobwa Karen}
\def\langnames@langs@glot@mhn{Mòcheno}
\def\langnames@langs@glot@mhc{Mocho}
\def\langnames@langs@glot@gbn{Mo'da}
\def\langnames@langs@glot@mxd{Modang}
\def\langnames@langs@glot@mqo{Modole}
\def\langnames@langs@glot@mvq{Moere}
\def\langnames@langs@glot@mou{Mogum}
\def\langnames@langs@glot@mof{Mohegan-Montauk-Narragansett}
\def\langnames@langs@glot@mow{Moi (Congo)}
\def\langnames@langs@glot@mxn{Moi (Indonesia)}
\def\langnames@langs@glot@mkp{Moikodi}
\def\langnames@langs@glot@mwz{Moingi}
\def\langnames@langs@glot@ymi{Moji}
\def\langnames@langs@glot@mft{Mokerang}
\def\langnames@langs@glot@mwt{Moken}
\def\langnames@langs@glot@mqt{Mok}
\def\langnames@langs@glot@mkm{Moklen}
\def\langnames@langs@glot@mkl{Mokole}
\def\langnames@langs@glot@vms{Moksela}
\def\langnames@langs@glot@pwm{Molbog}
\def\langnames@langs@glot@vsi{Moldova Sign Language}
\def\langnames@langs@glot@bxc{Molengue}
\def\langnames@langs@glot@mox{Molima}
\def\langnames@langs@glot@zmo{Molo}
\def\langnames@langs@glot@msl{Molof}
\def\langnames@langs@glot@mlw{Moloko}
\def\langnames@langs@glot@myl{Moma}
\def\langnames@langs@glot@msz{Momare}
\def\langnames@langs@glot@dmb{Mombo Dogon}
\def\langnames@langs@glot@mmb{Momina}
\def\langnames@langs@glot@ver{Mom Jango}
\def\langnames@langs@glot@mzg{Monastic Sign Language}
\def\langnames@langs@glot@npn{Mondropolon}
\def\langnames@langs@glot@msr{Mongolian Sign Language}
\def\langnames@langs@glot@mgt{Mwakai}
\def\langnames@langs@glot@mom{Mangue}
\def\langnames@langs@glot@moo{Monom}
\def\langnames@langs@glot@mru{Mono (Cameroon)}
\def\langnames@langs@glot@mnh{Mono (Democratic Republic of Congo)}
\def\langnames@langs@glot@nmh{Monsang Naga}
\def\langnames@langs@glot@mtl{Montol}
\def\langnames@langs@glot@gwg{Moo}
\def\langnames@langs@glot@crm{Moose Cree}
\def\langnames@langs@glot@msg{Moraid}
\def\langnames@langs@glot@mze{Morawa}
\def\langnames@langs@glot@moq{Mor (Bomberai Peninsula)}
\def\langnames@langs@glot@msx{Moresada}
\def\langnames@langs@glot@xmo{Morerebi}
\def\langnames@langs@glot@xmz{Mori Bawah}
\def\langnames@langs@glot@mzq{Mori Atas}
\def\langnames@langs@glot@mdb{Morigi}
\def\langnames@langs@glot@xms{Moroccan Sign Language}
\def\langnames@langs@glot@bdo{Morom}
\def\langnames@langs@glot@mgc{Morokodo}
\def\langnames@langs@glot@mrp{Morouas}
\def\langnames@langs@glot@mqn{Moronene}
\def\langnames@langs@glot@mrl{Mortlockese}
\def\langnames@langs@glot@mwy{Akie}
\def\langnames@langs@glot@mqv{Mosimo}
\def\langnames@langs@glot@mtj{Moskona}
\def\langnames@langs@glot@mtt{Mota}
\def\langnames@langs@glot@mwh{Mouk-Aria}
\def\langnames@langs@glot@jmw{Mouwase}
\def\langnames@langs@glot@ity{Moyadan Itneg}
\def\langnames@langs@glot@nmo{Moyon}
\def\langnames@langs@glot@mzy{Mozambican Sign Language}
\def\langnames@langs@glot@mxi{Mozarabic}
\def\langnames@langs@glot@xnq{Mozambican Ngoni}
\def\langnames@langs@glot@mpi{Mpade}
\def\langnames@langs@glot@mcx{Mpiemo}
\def\langnames@langs@glot@mpz{Mpi}
\def\langnames@langs@glot@pnd{Mpinda}
\def\langnames@langs@glot@mgg{Mpongmpong}
\def\langnames@langs@glot@mpa{Mpoto}
\def\langnames@langs@glot@mvt{Mpotovoro}
\def\langnames@langs@glot@zmp{Mbuun}
\def\langnames@langs@glot@cmr{Mro Chin}
\def\langnames@langs@glot@mro{Mru}
\def\langnames@langs@glot@kqx{Mser}
\def\langnames@langs@glot@agz{Mt. Iriga Agta}
\def\langnames@langs@glot@atl{Mt. Iraya Agta}
\def\langnames@langs@glot@mtd{Mualang}
\def\langnames@langs@glot@tsx{Mubami}
\def\langnames@langs@glot@mub{Mubi}
\def\langnames@langs@glot@ymd{Muda}
\def\langnames@langs@glot@gau{Mudhili Gadaba}
\def\langnames@langs@glot@udg{Muduga}
\def\langnames@langs@glot@vmd{Mudu Koraga}
\def\langnames@langs@glot@wiv{Muduapa}
\def\langnames@langs@glot@muk{Mugom}
\def\langnames@langs@glot@mmk{Mukha-Dora}
\def\langnames@langs@glot@mfw{Mulaha}
\def\langnames@langs@glot@kpb{Mullu Kurumba}
\def\langnames@langs@glot@vmu{Muluridyi}
\def\langnames@langs@glot@kqa{Mum}
\def\langnames@langs@glot@mwq{Mün Chin}
\def\langnames@langs@glot@boe{Mundabli-Mufu}
\def\langnames@langs@glot@mmf{Mindat}
\def\langnames@langs@glot@mth{Munggui}
\def\langnames@langs@glot@mpv{Mungkip}
\def\langnames@langs@glot@mtc{Munit}
\def\langnames@langs@glot@myr{Muniche}
\def\langnames@langs@glot@mnj{Munji}
\def\langnames@langs@glot@asx{Muratayak}
\def\langnames@langs@glot@mxr{Murik (Malaysia)}
\def\langnames@langs@glot@rmh{Murkim}
\def\langnames@langs@glot@tkv{Mur Pano}
\def\langnames@langs@glot@mqw{Murupi}
\def\langnames@langs@glot@smm{Musasa}
\def\langnames@langs@glot@mmi{Hember Avu}
\def\langnames@langs@glot@mmq{Aisi}
\def\langnames@langs@glot@mse{Musey}
\def\langnames@langs@glot@mui{Musi}
\def\langnames@langs@glot@mje{Muskum}
\def\langnames@langs@glot@muv{Muthuvan}
\def\langnames@langs@glot@tuc{Mutu}
\def\langnames@langs@glot@muy{Muyang}
\def\langnames@langs@glot@ymz{Muzi}
\def\langnames@langs@glot@mcj{Mvano}
\def\langnames@langs@glot@mxh{Mvuba}
\def\langnames@langs@glot@wlc{Mwali Comorian}
\def\langnames@langs@glot@wmw{Mwani}
\def\langnames@langs@glot@moa{Mwan}
\def\langnames@langs@glot@mwa{Mwatebu}
\def\langnames@langs@glot@mjh{Mwera (Nyasa)}
\def\langnames@langs@glot@mws{Mwimbi-Muthambi}
\def\langnames@langs@glot@gmy{Mycenaean Greek}
\def\langnames@langs@glot@nme{Mzieme Naga}
\def\langnames@langs@glot@nbt{Na}
\def\langnames@langs@glot@nao{Naaba}
\def\langnames@langs@glot@mne{Naba}
\def\langnames@langs@glot@mty{Nabi-Metan}
\def\langnames@langs@glot@ncd{Nachering}
\def\langnames@langs@glot@srf{Nafi}
\def\langnames@langs@glot@nxx{Nafri}
\def\langnames@langs@glot@jbn{Nafusi}
\def\langnames@langs@glot@nbg{Nagarchal}
\def\langnames@langs@glot@nxe{Nage}
\def\langnames@langs@glot@ngv{Nagumi}
\def\langnames@langs@glot@nlx{Nahali-Baglani}
\def\langnames@langs@glot@nhh{Nahari}
\def\langnames@langs@glot@ars{Najdi Arabic}
\def\langnames@langs@glot@nae{Naka'ela}
\def\langnames@langs@glot@nib{Nakama}
\def\langnames@langs@glot@nkj{Nakai}
\def\langnames@langs@glot@nbk{Nake}
\def\langnames@langs@glot@mff{Naki}
\def\langnames@langs@glot@nax{Nakwi}
\def\langnames@langs@glot@nlc{Nalca}
\def\langnames@langs@glot@nss{Nali}
\def\langnames@langs@glot@nlz{Nalögo}
\def\langnames@langs@glot@ylo{Naluo Yi}
\def\langnames@langs@glot@naj{Nalu}
\def\langnames@langs@glot@nmx{Nama (Papua New Guinea)}
\def\langnames@langs@glot@nkm{Namat}
\def\langnames@langs@glot@nmk{Namakura}
\def\langnames@langs@glot@nmq{Nambya}
\def\langnames@langs@glot@ncm{Nambo}
\def\langnames@langs@glot@neo{Ná-Meo}
\def\langnames@langs@glot@nbs{Namibian Sign Language}
\def\langnames@langs@glot@nvm{Namiae}
\def\langnames@langs@glot@naa{Namla}
\def\langnames@langs@glot@mxw{Namo}
\def\langnames@langs@glot@nmt{Namonuito}
\def\langnames@langs@glot@bwb{Namosi-Naitasiri-Serua}
\def\langnames@langs@glot@nmy{Namuyi}
\def\langnames@langs@glot@nnc{Nancere}
\def\langnames@langs@glot@nzz{Nanga}
\def\langnames@langs@glot@ngr{Nanggu}
\def\langnames@langs@glot@cox{Nanti}
\def\langnames@langs@glot@afk{Nanubae-Imangae}
\def\langnames@langs@glot@qvo{Napo Lowland Quechua}
\def\langnames@langs@glot@nrg{Narango}
\def\langnames@langs@glot@nac{Narak}
\def\langnames@langs@glot@loh{Narim}
\def\langnames@langs@glot@nnr{Narungga}
\def\langnames@langs@glot@nsy{Nasal}
\def\langnames@langs@glot@nvh{Nasarian}
\def\langnames@langs@glot@ntz{Natanzic}
\def\langnames@langs@glot@nte{Nathembo}
\def\langnames@langs@glot@nti{Natioro}
\def\langnames@langs@glot@nxa{Nauete}
\def\langnames@langs@glot@ncn{Nauna}
\def\langnames@langs@glot@nwo{Nauo}
\def\langnames@langs@glot@nsw{Navut}
\def\langnames@langs@glot@nwr{Nawaru}
\def\langnames@langs@glot@nwa{Nawathinehena}
\def\langnames@langs@glot@nmz{Nawdm}
\def\langnames@langs@glot@naw{Nawuri}
\def\langnames@langs@glot@nyq{Nayinic}
\def\langnames@langs@glot@noz{Nayi}
\def\langnames@langs@glot@ncr{Ncane-Mungong}
\def\langnames@langs@glot@nlu{Nchumbulu}
\def\langnames@langs@glot@gke{Ndai}
\def\langnames@langs@glot@ndk{Ndaka}
\def\langnames@langs@glot@ndh{Ndali}
\def\langnames@langs@glot@ndj{Ndamba}
\def\langnames@langs@glot@ndm{Ndam}
\def\langnames@langs@glot@nxo{Ndambomo}
\def\langnames@langs@glot@nnz{Nda'nda'}
\def\langnames@langs@glot@nda{Ndasa}
\def\langnames@langs@glot@ndc{Ndau}
\def\langnames@langs@glot@nml{Ndemli}
\def\langnames@langs@glot@ndg{Ndengereko}
\def\langnames@langs@glot@dne{Ndendeule}
\def\langnames@langs@glot@ndd{Nde-Nsele-Nta}
\def\langnames@langs@glot@eli{Nding}
\def\langnames@langs@glot@ndw{Ndobo}
\def\langnames@langs@glot@nbb{Ndoe}
\def\langnames@langs@glot@ndl{Ndolo}
\def\langnames@langs@glot@ndq{Ndombe}
\def\langnames@langs@glot@nqm{Ndom}
\def\langnames@langs@glot@ndr{Ndoola}
\def\langnames@langs@glot@ndp{Ndo}
\def\langnames@langs@glot@dno{Ndrulo}
\def\langnames@langs@glot@ndx{Nduga}
\def\langnames@langs@glot@nuh{Ndunda}
\def\langnames@langs@glot@nww{Ndwewe}
\def\langnames@langs@glot@njt{Ndyuka-Trio Pidgin}
\def\langnames@langs@glot@wni{Ndzwani Comorian}
\def\langnames@langs@glot@nec{Klamu}
\def\langnames@langs@glot@nef{Nefamese}
\def\langnames@langs@glot@dcr{Negerhollands}
\def\langnames@langs@glot@nkg{Nekgini}
\def\langnames@langs@glot@nif{Nek}
\def\langnames@langs@glot@nej{Neko}
\def\langnames@langs@glot@nek{Neku}
\def\langnames@langs@glot@nex{Neme}
\def\langnames@langs@glot@nem{Nemi}
\def\langnames@langs@glot@nqn{Nen}
\def\langnames@langs@glot@neu{Neo (Artificial Language)}
\def\langnames@langs@glot@nsp{Nepalese Sign Language}
\def\langnames@langs@glot@net{Nete}
\def\langnames@langs@glot@jas{New Caledonian Javanese}
\def\langnames@langs@glot@jui{Ngadjuri}
\def\langnames@langs@glot@nnf{Ngaing}
\def\langnames@langs@glot@hlt{Nga La}
\def\langnames@langs@glot@szb{Ngalum}
\def\langnames@langs@glot@nud{Ngala}
\def\langnames@langs@glot@nmv{Ngamini-Yarluyandi-Karangura}
\def\langnames@langs@glot@nbv{Ngamambo}
\def\langnames@langs@glot@nmc{Ngam}
\def\langnames@langs@glot@nbh{Ngamo}
\def\langnames@langs@glot@nyx{Nganyaywana}
\def\langnames@langs@glot@gng{Ngangam}
\def\langnames@langs@glot@nne{Ngandyera}
\def\langnames@langs@glot@nxd{Ngando-Lalia}
\def\langnames@langs@glot@ngd{Ngando (Central African Republic)}
\def\langnames@langs@glot@nji{Ngarnka}
\def\langnames@langs@glot@rxd{Ngardi}
\def\langnames@langs@glot@nsg{Ngasa}
\def\langnames@langs@glot@ngm{Ngatik Men's Creole}
\def\langnames@langs@glot@cnw{Ngawn Chin}
\def\langnames@langs@glot@zdj{Ngazidja Comorian}
\def\langnames@langs@glot@ngg{Ngbaka Manza}
\def\langnames@langs@glot@jgb{Ngbee}
\def\langnames@langs@glot@nbd{Ngbinda-Mayeka}
\def\langnames@langs@glot@nuu{Ngbundu}
\def\langnames@langs@glot@gnj{Ngen of Djonkro}
\def\langnames@langs@glot@nql{Ngendelengo}
\def\langnames@langs@glot@ngt{Kriang-Khlor}
\def\langnames@langs@glot@nnn{Ngete}
\def\langnames@langs@glot@nbq{Nggem}
\def\langnames@langs@glot@ngx{Nggwahyi}
\def\langnames@langs@glot@nnh{Ngiemboon}
\def\langnames@langs@glot@ngj{Ngie}
\def\langnames@langs@glot@nnq{Ngindo}
\def\langnames@langs@glot@nra{Ngom}
\def\langnames@langs@glot@nla{Ngombale}
\def\langnames@langs@glot@jgo{Ngomba}
\def\langnames@langs@glot@noq{Ngongo}
\def\langnames@langs@glot@nsh{Ngoshie}
\def\langnames@langs@glot@nuw{Nguluwan}
\def\langnames@langs@glot@ngp{Ngulu}
\def\langnames@langs@glot@nlo{Ngwii}
\def\langnames@langs@glot@xnm{Ngumbarl}
\def\langnames@langs@glot@nui{Ngumbi}
\def\langnames@langs@glot@nue{Ngundu}
\def\langnames@langs@glot@ndn{Ngundi}
\def\langnames@langs@glot@ngz{Ngungwel}
\def\langnames@langs@glot@nuo{Nguôn}
\def\langnames@langs@glot@nrx{Ngurmbur}
\def\langnames@langs@glot@nbx{Wilson River (Grey Range)}
\def\langnames@langs@glot@ngq{Ngoreme}
\def\langnames@langs@glot@ngw{Ngwaba}
\def\langnames@langs@glot@nwe{Ngwe}
\def\langnames@langs@glot@ngn{Ngwo}
\def\langnames@langs@glot@yrl{Nhengatu}
\def\langnames@langs@glot@nhf{Nhuwala}
\def\langnames@langs@glot@ncs{Nicaraguan Sign Language}
\def\langnames@langs@glot@nsi{Nigerian Sign Language}
\def\langnames@langs@glot@mzk{Western Mambila}
\def\langnames@langs@glot@nii{Nii}
\def\langnames@langs@glot@xny{Nyiyaparli-Palyku}
\def\langnames@langs@glot@gbe{Niksek}
\def\langnames@langs@glot@nim{Nilamba}
\def\langnames@langs@glot@nil{Nila}
\def\langnames@langs@glot@noe{Nimadi}
\def\langnames@langs@glot@nmp{Nimanbur}
\def\langnames@langs@glot@nmr{Nimbari}
\def\langnames@langs@glot@nis{Nimi}
\def\langnames@langs@glot@nmw{Nimoa}
\def\langnames@langs@glot@niw{Nimo}
\def\langnames@langs@glot@nxi{Nindi}
\def\langnames@langs@glot@nxr{Ninggerum}
\def\langnames@langs@glot@nby{Ningera}
\def\langnames@langs@glot@nlk{Ninia Yali}
\def\langnames@langs@glot@nin{Ninzo}
\def\langnames@langs@glot@nps{Nipsan}
\def\langnames@langs@glot@njs{Nisa-Anasi}
\def\langnames@langs@glot@yso{Nisi (China)}
\def\langnames@langs@glot@nkp{Niuatoputapu}
\def\langnames@langs@glot@njl{Njalgulgule}
\def\langnames@langs@glot@nzb{Njebi}
\def\langnames@langs@glot@njj{Njen}
\def\langnames@langs@glot@njr{Njerep}
\def\langnames@langs@glot@njy{Njyem}
\def\langnames@langs@glot@nkq{Nkami}
\def\langnames@langs@glot@nkn{Nkangala}
\def\langnames@langs@glot@nkz{Nkari}
\def\langnames@langs@glot@khu{Nkhumbi}
\def\langnames@langs@glot@nqo{N'Ko}
\def\langnames@langs@glot@nkc{Nkongho}
\def\langnames@langs@glot@nkx{Nkoroo}
\def\langnames@langs@glot@nka{Nkoya}
\def\langnames@langs@glot@nbo{Nkukoli}
\def\langnames@langs@glot@nkw{Nkutu}
\def\langnames@langs@glot@nbp{Nnam}
\def\langnames@langs@glot@ngh{N||ng}
\def\langnames@langs@glot@gaw{Nobonob}
\def\langnames@langs@glot@noi{Noiri}
\def\langnames@langs@glot@nkk{Nokuku}
\def\langnames@langs@glot@lem{Nomaande}
\def\langnames@langs@glot@nof{Nomane}
\def\langnames@langs@glot@noh{Nomu}
\def\langnames@langs@glot@zhn{Nong Zhuang}
\def\langnames@langs@glot@noj{Nonuya}
\def\langnames@langs@glot@nok{Nooksack}
\def\langnames@langs@glot@nrc{Noric}
\def\langnames@langs@glot@nrp{North Picene}
\def\langnames@langs@glot@huj{Northern Guiyang Hmong}
\def\langnames@langs@glot@hmp{Northern Mashan Hmong}
\def\langnames@langs@glot@crl{Northern East Cree}
\def\langnames@langs@glot@pbu{Northern Pashto}
\def\langnames@langs@glot@hno{Northern Hindko}
\def\langnames@langs@glot@glh{Northwest Pashayi}
\def\langnames@langs@glot@aee{Northeast Pashayi}
\def\langnames@langs@glot@kxm{Northern Khmer}
\def\langnames@langs@glot@atv{Northern Altai}
\def\langnames@langs@glot@azj{North Azerbaijani}
\def\langnames@langs@glot@ghh{Northern Ghale}
\def\langnames@langs@glot@ymx{Northern Muji}
\def\langnames@langs@glot@yiv{Northern Nisu}
\def\langnames@langs@glot@cng{Northern Qiang}
\def\langnames@langs@glot@bfc{Northern Bai}
\def\langnames@langs@glot@nnl{Northern Rengma Naga}
\def\langnames@langs@glot@lbr{Lohorung}
\def\langnames@langs@glot@tji{Northern Tujia}
\def\langnames@langs@glot@doc{Northern Dong}
\def\langnames@langs@glot@nod{Northern Thai}
\def\langnames@langs@glot@tts{Northeastern Thai}
\def\langnames@langs@glot@hea{Northern Qiandong Miao}
\def\langnames@langs@glot@hmi{Northern Huishui Hmong}
\def\langnames@langs@glot@kqs{Northern Kissi}
\def\langnames@langs@glot@fll{North Fali}
\def\langnames@langs@glot@dgi{Northern Dagara}
\def\langnames@langs@glot@tsp{Northern Toussian}
\def\langnames@langs@glot@gbo{Northern Grebo}
\def\langnames@langs@glot@dip{Northeastern Dinka}
\def\langnames@langs@glot@diw{Northwestern Dinka}
\def\langnames@langs@glot@max{North Moluccan Malay}
\def\langnames@langs@glot@mmg{North Ambrym}
\def\langnames@langs@glot@mrq{North Marquesan}
\def\langnames@langs@glot@tnn{North Tanna}
\def\langnames@langs@glot@una{North Watut}
\def\langnames@langs@glot@bcd{North Babar}
\def\langnames@langs@glot@weo{Wemale}
\def\langnames@langs@glot@nni{North Nuaulu}
\def\langnames@langs@glot@aqn{Northern Alta}
\def\langnames@langs@glot@xnn{Northern Kankanay}
\def\langnames@langs@glot@cts{Northern Catanduanes Bicolano}
\def\langnames@langs@glot@stb{Northern Subanen}
\def\langnames@langs@glot@bmm{Northern Betsimisaraka Malagasy}
\def\langnames@langs@glot@onr{Northern One}
\def\langnames@langs@glot@kti{North Muyu}
\def\langnames@langs@glot@nks{Momogo-Pupis-Irogo}
\def\langnames@langs@glot@yir{North Awyu}
\def\langnames@langs@glot@whg{North Wahgi}
\def\langnames@langs@glot@kiw{Northeast Kiwai}
\def\langnames@langs@glot@ryn{Northern Amami-Oshima}
\def\langnames@langs@glot@neq{North Central Mixe}
\def\langnames@langs@glot@scs{North Slavey}
\def\langnames@langs@glot@esk{Seward Alaska Inupiatun}
\def\langnames@langs@glot@thh{Northern Tarahumara}
\def\langnames@langs@glot@nhy{Northern Oaxaca Nahuatl}
\def\langnames@langs@glot@ojb{Northwestern Ojibwa}
\def\langnames@langs@glot@pef{Northeastern Russian River Pomo}
\def\langnames@langs@glot@cst{San Francisco Bay Ohlone}
\def\langnames@langs@glot@enl{Enlhet Norte}
\def\langnames@langs@glot@qvz{Northern Pastaza Quichua}
\def\langnames@langs@glot@qul{North Bolivian Quechua}
\def\langnames@langs@glot@qxn{Northern Conchucos Ancash Quechua}
\def\langnames@langs@glot@pmq{Northern Pame}
\def\langnames@langs@glot@xtn{Northern Tlaxiaco Mixtec}
\def\langnames@langs@glot@mxa{Northwest Oaxaca Mixtec}
\def\langnames@langs@glot@mfk{North Mofu}
\def\langnames@langs@glot@ayp{North Mesopotamian Arabic}
\def\langnames@langs@glot@ntd{Northern Tidung}
\def\langnames@langs@glot@cnp{Northern Pinghua}
\def\langnames@langs@glot@ncq{Northern Katang}
\def\langnames@langs@glot@bly{Notre}
\def\langnames@langs@glot@ncf{Notsi}
\def\langnames@langs@glot@ntw{Nottoway}
\def\langnames@langs@glot@nov{Novial}
\def\langnames@langs@glot@noy{Noy}
\def\langnames@langs@glot@asj{Nsari}
\def\langnames@langs@glot@nsc{Nshi}
\def\langnames@langs@glot@nsx{Nsongo}
\def\langnames@langs@glot@baf{Nubaca}
\def\langnames@langs@glot@kte{Gyalsumdo-Nubri}
\def\langnames@langs@glot@wbm{Zhenkang Wa}
\def\langnames@langs@glot@bsq{Bassa}
\def\langnames@langs@glot@wla{Walio}
\def\langnames@langs@glot@wgi{Wahgi}
\def\langnames@langs@glot@gyz{Gyaazi}
\def\langnames@langs@glot@nqt{Nteng}
\def\langnames@langs@glot@nnv{Nugunu (Australia)}
\def\langnames@langs@glot@noc{Nuk}
\def\langnames@langs@glot@klt{Nukna}
\def\langnames@langs@glot@nuq{Nukumanu}
\def\langnames@langs@glot@nur{Nukuria}
\def\langnames@langs@glot@nuc{Nukuini}
\def\langnames@langs@glot@nbr{Numana}
\def\langnames@langs@glot@nop{Numanggang}
\def\langnames@langs@glot@sij{Numbami}
\def\langnames@langs@glot@tgs{Nume}
\def\langnames@langs@glot@kdk{Numee}
\def\langnames@langs@glot@nxm{Numidian}
\def\langnames@langs@glot@nug{Nungali}
\def\langnames@langs@glot@rin{Nungu}
\def\langnames@langs@glot@nul{Nusa Laut}
\def\langnames@langs@glot@nwb{Nyabwa}
\def\langnames@langs@glot@nev{Nyaheun}
\def\langnames@langs@glot@nyy{Nyakyusa-Ngonde}
\def\langnames@langs@glot@nlj{Nyali}
\def\langnames@langs@glot@mwn{Nyamwanga}
\def\langnames@langs@glot@nwm{Nyamusa-Molo}
\def\langnames@langs@glot@nmi{Nyam}
\def\langnames@langs@glot@nny{Yangkaal}
\def\langnames@langs@glot@nyb{Nyangbo}
\def\langnames@langs@glot@nyc{Nyanga-li}
\def\langnames@langs@glot@nyk{Nyaneka}
\def\langnames@langs@glot@nnj{Nyangatom}
\def\langnames@langs@glot@sev{Nyarafolo Senoufo}
\def\langnames@langs@glot@nba{Nyemba}
\def\langnames@langs@glot@neh{Upper Mangdep}
\def\langnames@langs@glot@nye{Nyengo}
\def\langnames@langs@glot@nyl{Nyeu}
\def\langnames@langs@glot@nyr{Nyiha (Malawi)}
\def\langnames@langs@glot@nkv{Nyika (Malawi and Zambia)}
\def\langnames@langs@glot@nkt{Nyika (Tanzania)}
\def\langnames@langs@glot@nyg{Nyindu}
\def\langnames@langs@glot@lid{Nyindrou}
\def\langnames@langs@glot@nvo{Nyokon}
\def\langnames@langs@glot@nuj{Nyole}
\def\langnames@langs@glot@muo{Nyong}
\def\langnames@langs@glot@nyd{Nyore}
\def\langnames@langs@glot@nyu{Nyungwe}
\def\langnames@langs@glot@nzd{Nzadi}
\def\langnames@langs@glot@nzy{Nzakambay}
\def\langnames@langs@glot@nja{Nzanyi}
\def\langnames@langs@glot@nzi{Nzima}
\def\langnames@langs@glot@bzy{Obanliku}
\def\langnames@langs@glot@obi{Obispeño}
\def\langnames@langs@glot@obl{Oblo}
\def\langnames@langs@glot@obo{Obo Manobo}
\def\langnames@langs@glot@obu{Obulom-Ochichi}
\def\langnames@langs@glot@zac{Ocotlán Zapotec}
\def\langnames@langs@glot@odk{Od}
\def\langnames@langs@glot@bhf{Odiai}
\def\langnames@langs@glot@kkc{Odoodee}
\def\langnames@langs@glot@odu{Odual}
\def\langnames@langs@glot@tyh{O'du}
\def\langnames@langs@glot@opy{Ofayé}
\def\langnames@langs@glot@ofo{Ofo}
\def\langnames@langs@glot@ogc{Ogbah}
\def\langnames@langs@glot@ogg{Ogbogolo}
\def\langnames@langs@glot@eri{Ogea}
\def\langnames@langs@glot@oia{Oirata}
\def\langnames@langs@glot@chj{Ojitlán Chinantec}
\def\langnames@langs@glot@oki{Okiek}
\def\langnames@langs@glot@okn{Oki-No-Erabu}
\def\langnames@langs@glot@okb{Okobo}
\def\langnames@langs@glot@okd{Okodia}
\def\langnames@langs@glot@oks{Oko-Eni-Osayen}
\def\langnames@langs@glot@okj{Okojuwoi}
\def\langnames@langs@glot@kqv{Okolod}
\def\langnames@langs@glot@oie{Okolie}
\def\langnames@langs@glot@opa{Okpamheri}
\def\langnames@langs@glot@okx{Okpe (Northwestern Edo)}
\def\langnames@langs@glot@oke{Okpe (Southwestern Edo)}
\def\langnames@langs@glot@oar{Old Aramaic-Sam'alian}
\def\langnames@langs@glot@obr{Old Burmese}
\def\langnames@langs@glot@och{Old Chinese}
\def\langnames@langs@glot@odt{Old Dutch-Old Frankish}
\def\langnames@langs@glot@ang{Old English (ca. 450-1100)}
\def\langnames@langs@glot@fro{Old French (842-ca. 1400)}
\def\langnames@langs@glot@ofs{Old Frisian}
\def\langnames@langs@glot@oge{Old Georgian}
\def\langnames@langs@glot@goh{Old High German (ca. 750-1050)}
\def\langnames@langs@glot@sga{Early Irish}
\def\langnames@langs@glot@ojp{Old Japanese}
\def\langnames@langs@glot@okl{Old Kentish Sign Language}
\def\langnames@langs@glot@qok{Old Khmer}
\def\langnames@langs@glot@qkn{Old Kannada}
\def\langnames@langs@glot@qbb{Old Latin}
\def\langnames@langs@glot@omx{Old Mon}
\def\langnames@langs@glot@omr{Old Marathi}
\def\langnames@langs@glot@non{Old Norse}
\def\langnames@langs@glot@onw{Old Nubian}
\def\langnames@langs@glot@oos{Old Ossetic}
\def\langnames@langs@glot@pro{Old Provençal}
\def\langnames@langs@glot@peo{Old Persian (ca. 600-400 B.C.)}
\def\langnames@langs@glot@orv{Old Russian}
\def\langnames@langs@glot@osp{Old Spanish}
\def\langnames@langs@glot@osx{Old Saxon}
\def\langnames@langs@glot@oty{Old Tamil}
\def\langnames@langs@glot@oui{Old Turkic}
\def\langnames@langs@glot@owl{Old-Middle Welsh}
\def\langnames@langs@glot@ole{Olekha}
\def\langnames@langs@glot@olm{Oloma}
\def\langnames@langs@glot@lul{Olu'bo}
\def\langnames@langs@glot@iko{Olulumo-Ikom}
\def\langnames@langs@glot@acx{Omani Arabic}
\def\langnames@langs@glot@oml{Ombo}
\def\langnames@langs@glot@nht{Ometepec Nahuatl}
\def\langnames@langs@glot@omi{Omi}
\def\langnames@langs@glot@omt{Omotik}
\def\langnames@langs@glot@omu{Omurano}
\def\langnames@langs@glot@oog{Ong-Ir}
\def\langnames@langs@glot@onx{Onin Pidgin}
\def\langnames@langs@glot@oni{Onin}
\def\langnames@langs@glot@onj{Onjob}
\def\langnames@langs@glot@onn{Onobasulu}
\def\langnames@langs@glot@oor{Oorlams}
\def\langnames@langs@glot@opo{Opao}
\def\langnames@langs@glot@opt{Teguima}
\def\langnames@langs@glot@lgn{Opo}
\def\langnames@langs@glot@orn{Orang Kanaq}
\def\langnames@langs@glot@ors{Orang Seletar}
\def\langnames@langs@glot@sdr{Oraon Sadri}
\def\langnames@langs@glot@org{Oring}
\def\langnames@langs@glot@nlv{Orizaba Nahuatl}
\def\langnames@langs@glot@fnb{Orkon-Fanbak}
\def\langnames@langs@glot@orc{Orma}
\def\langnames@langs@glot@orz{Ormu}
\def\langnames@langs@glot@ora{Oroha}
\def\langnames@langs@glot@orx{Oro}
\def\langnames@langs@glot@orh{Oroqen}
\def\langnames@langs@glot@bpk{Orowe}
\def\langnames@langs@glot@orw{Oro Win}
\def\langnames@langs@glot@orr{Oruma}
\def\langnames@langs@glot@syx{Osamayi}
\def\langnames@langs@glot@ost{Osatu}
\def\langnames@langs@glot@osc{Oscan}
\def\langnames@langs@glot@osi{Osing}
\def\langnames@langs@glot@oso{Ososo}
\def\langnames@langs@glot@uta{Otank}
\def\langnames@langs@glot@otd{Ot Danum}
\def\langnames@langs@glot@oti{Oti}
\def\langnames@langs@glot@otw{Ottawa}
\def\langnames@langs@glot@lot{Otuho}
\def\langnames@langs@glot@otu{Otuke}
\def\langnames@langs@glot@oum{Ouma}
\def\langnames@langs@glot@oue{Ounge}
\def\langnames@langs@glot@stn{Owa}
\def\langnames@langs@glot@wsr{Oweina}
\def\langnames@langs@glot@oyy{Oya'oya}
\def\langnames@langs@glot@oyd{Oyda}
\def\langnames@langs@glot@zao{Ozolotepec Zapotec}
\def\langnames@langs@glot@chz{Ozumacín Chinantec}
\def\langnames@langs@glot@pfa{Pááfang}
\def\langnames@langs@glot@sig{Paasaal}
\def\langnames@langs@glot@qvp{Pacaraos Quechua}
\def\langnames@langs@glot@pcp{Pacahuara}
\def\langnames@langs@glot@pdi{Pa Di}
\def\langnames@langs@glot@pkc{Paekche}
\def\langnames@langs@glot@pae{Pagibete}
\def\langnames@langs@glot@pgi{Pagi}
\def\langnames@langs@glot@phr{Pahari Potwari}
\def\langnames@langs@glot@phj{Pahari Newari}
\def\langnames@langs@glot@lgt{Pahi}
\def\langnames@langs@glot@phv{Pahlavani}
\def\langnames@langs@glot@pal{Pahlavi}
\def\langnames@langs@glot@pha{Pa-Hng}
\def\langnames@langs@glot@pri{Paicî}
\def\langnames@langs@glot@ppi{Paipai}
\def\langnames@langs@glot@qpp{Paisaci Prakrit}
\def\langnames@langs@glot@pta{Pai Tavytera}
\def\langnames@langs@glot@pkg{Pak-Tong}
\def\langnames@langs@glot@jkp{Paku Karen}
\def\langnames@langs@glot@pku{Paku}
\def\langnames@langs@glot@pfl{Pfaelzisch-Lothringisch}
\def\langnames@langs@glot@plq{Palaic}
\def\langnames@langs@glot@plr{Palaka Senoufo}
\def\langnames@langs@glot@pln{Palenquero}
\def\langnames@langs@glot@pnl{Palen}
\def\langnames@langs@glot@pli{Pali}
\def\langnames@langs@glot@pcf{Paliyan}
\def\langnames@langs@glot@pmd{Pallanganmiddang}
\def\langnames@langs@glot@abw{Pal}
\def\langnames@langs@glot@pmc{Palumata}
\def\langnames@langs@glot@ple{Palu'e}
\def\langnames@langs@glot@plz{Paluan}
\def\langnames@langs@glot@bpx{Palya Bareli}
\def\langnames@langs@glot@pmb{Pambia}
\def\langnames@langs@glot@pmn{Pam (Cameroon)}
\def\langnames@langs@glot@hih{Pamosu}
\def\langnames@langs@glot@att{Pamplona Atta}
\def\langnames@langs@glot@pnz{Pana (Central African Republic)}
\def\langnames@langs@glot@pnq{Pana (Burkina Faso)}
\def\langnames@langs@glot@pwb{Panawa}
\def\langnames@langs@glot@psn{Panasuan}
\def\langnames@langs@glot@qxh{Panao Huánuco Quechua}
\def\langnames@langs@glot@lsp{Panamanian Sign Language}
\def\langnames@langs@glot@tdb{Panchpargania}
\def\langnames@langs@glot@pnp{Pancana}
\def\langnames@langs@glot@bkj{Pande}
\def\langnames@langs@glot@pgg{Pangwali}
\def\langnames@langs@glot@pgs{Pangseng}
\def\langnames@langs@glot@slm{Pangutaran Sama}
\def\langnames@langs@glot@pcg{Paniya}
\def\langnames@langs@glot@pnr{Panim}
\def\langnames@langs@glot@pax{Pankararé}
\def\langnames@langs@glot@pkh{Pangkhua}
\def\langnames@langs@glot@paz{Pankararú}
\def\langnames@langs@glot@pnc{Pannei}
\def\langnames@langs@glot@knt{Panoan Katukína}
\def\langnames@langs@glot@pno{Panobo}
\def\langnames@langs@glot@blk{Pa'o Karen}
\def\langnames@langs@glot@ppv{Papavô}
\def\langnames@langs@glot@ppn{Papapana}
\def\langnames@langs@glot@dpp{Papar}
\def\langnames@langs@glot@pas{Papasena}
\def\langnames@langs@glot@pbo{Papel}
\def\langnames@langs@glot@ppe{Papi}
\def\langnames@langs@glot@ppu{Papora-Hoanya}
\def\langnames@langs@glot@ppm{Papuma}
\def\langnames@langs@glot@pgz{Papua New Guinean Sign Language}
\def\langnames@langs@glot@prc{Parachi}
\def\langnames@langs@glot@pzn{Jejara Naga}
\def\langnames@langs@glot@prf{Paranan}
\def\langnames@langs@glot@prw{Parawen}
\def\langnames@langs@glot@aap{Pará Arára}
\def\langnames@langs@glot@pak{Parakanã}
\def\langnames@langs@glot@paf{Paranawát}
\def\langnames@langs@glot@gvp{Pará-Maranhão Gavião}
\def\langnames@langs@glot@pbg{Paraujano}
\def\langnames@langs@glot@pys{Paraguayan Sign Language}
\def\langnames@langs@glot@pcl{Pardhi}
\def\langnames@langs@glot@pch{Pardhan}
\def\langnames@langs@glot@pcj{Gorum-Parenga}
\def\langnames@langs@glot@ppt{Pare}
\def\langnames@langs@glot@kvx{Parkari Koli}
\def\langnames@langs@glot@xpr{Parthian}
\def\langnames@langs@glot@paq{Parya}
\def\langnames@langs@glot@psq{Pasi}
\def\langnames@langs@glot@yac{Pass Valley Yali}
\def\langnames@langs@glot@ptn{Patani}
\def\langnames@langs@glot@pth{Pataxó Hã-Ha-Hãe}
\def\langnames@langs@glot@pbc{Patamona}
\def\langnames@langs@glot@pty{Pathiya}
\def\langnames@langs@glot@ptq{Pattapu}
\def\langnames@langs@glot@mfa{Kelantan-Pattani Malay}
\def\langnames@langs@glot@pnk{Paunaka}
\def\langnames@langs@glot@bfb{Pauri Bareli}
\def\langnames@langs@glot@psm{Warázu}
\def\langnames@langs@glot@pmr{Manat}
\def\langnames@langs@glot@pcb{Pear}
\def\langnames@langs@glot@xpc{Pecheneg}
\def\langnames@langs@glot@pai{Pye}
\def\langnames@langs@glot@pfe{Peere}
\def\langnames@langs@glot@ppq{Pei}
\def\langnames@langs@glot@pel{Pekal}
\def\langnames@langs@glot@bxd{Pela}
\def\langnames@langs@glot@ata{Pele-Ata}
\def\langnames@langs@glot@pev{Pémono}
\def\langnames@langs@glot@psg{Penang Sign Language}
\def\langnames@langs@glot@pek{Penchal}
\def\langnames@langs@glot@ums{Pendau}
\def\langnames@langs@glot@pdc{Pennsylvania German}
\def\langnames@langs@glot@pnh{Māngarongaro}
\def\langnames@langs@glot@ptw{Pentlatch}
\def\langnames@langs@glot@pea{Peranakan Indonesian}
\def\langnames@langs@glot@wet{Perai}
\def\langnames@langs@glot@psc{Zaban Eshareh Irani}
\def\langnames@langs@glot@prl{Peruvian Sign Language}
\def\langnames@langs@glot@pex{Petats}
\def\langnames@langs@glot@zpe{Petapa Zapotec}
\def\langnames@langs@glot@pey{Petjo}
\def\langnames@langs@glot@prt{Prai}
\def\langnames@langs@glot@phk{Phake}
\def\langnames@langs@glot@phl{Palula}
\def\langnames@langs@glot@ypa{Phala}
\def\langnames@langs@glot@phq{Phana'}
\def\langnames@langs@glot@pem{Phende}
\def\langnames@langs@glot@psp{Philippine Sign Language}
\def\langnames@langs@glot@phm{Phimbi}
\def\langnames@langs@glot@phn{Phoenician}
\def\langnames@langs@glot@yip{Pholo}
\def\langnames@langs@glot@ypg{Phola}
\def\langnames@langs@glot@nph{Phom Naga}
\def\langnames@langs@glot@pnx{Phong-Kniang}
\def\langnames@langs@glot@kjt{Phrae Pwo Karen}
\def\langnames@langs@glot@xpg{Phrygian}
\def\langnames@langs@glot@phu{Phuan}
\def\langnames@langs@glot@phd{Phudagi}
\def\langnames@langs@glot@pug{Phuie}
\def\langnames@langs@glot@phh{Phukha}
\def\langnames@langs@glot@ypm{Phuma}
\def\langnames@langs@glot@pho{Phunoi}
\def\langnames@langs@glot@phg{Phuong}
\def\langnames@langs@glot@yph{Phupha}
\def\langnames@langs@glot@ypp{Phupa}
\def\langnames@langs@glot@pht{Phu Thai}
\def\langnames@langs@glot@ypz{Phuza}
\def\langnames@langs@glot@ptr{Piamatsina}
\def\langnames@langs@glot@pin{Piame}
\def\langnames@langs@glot@pcd{Picard}
\def\langnames@langs@glot@cpu{Pichis Ashéninka}
\def\langnames@langs@glot@xpi{Pictish}
\def\langnames@langs@glot@dep{Pidgin Delaware}
\def\langnames@langs@glot@pij{Pijao}
\def\langnames@langs@glot@piz{Pije}
\def\langnames@langs@glot@pis{Pijin}
\def\langnames@langs@glot@piw{Pimbwe}
\def\langnames@langs@glot@pnn{Pinai-Hagahai}
\def\langnames@langs@glot@pnv{Pinigura}
\def\langnames@langs@glot@tjp{Lake Carnegie Western Desert}
\def\langnames@langs@glot@pic{Pinji}
\def\langnames@langs@glot@pti{Pintiini}
\def\langnames@langs@glot@pny{Pinyin}
\def\langnames@langs@glot@bxi{Pirlatapa}
\def\langnames@langs@glot@pie{Piro}
\def\langnames@langs@glot@xpa{Pirriya}
\def\langnames@langs@glot@tpp{Pisaflores Tepehua}
\def\langnames@langs@glot@pig{Pisabo}
\def\langnames@langs@glot@psy{Piscataway}
\def\langnames@langs@glot@xps{Pisidian}
\def\langnames@langs@glot@pih{Pitcairn-Norfolk}
\def\langnames@langs@glot@sje{Pite Saami}
\def\langnames@langs@glot@pcn{Piti}
\def\langnames@langs@glot@pix{Piu}
\def\langnames@langs@glot@piy{Piya-Kwonci}
\def\langnames@langs@glot@ktj{Plapo Krumen}
\def\langnames@langs@glot@pdt{Plautdietsch}
\def\langnames@langs@glot@pbv{Pnar}
\def\langnames@langs@glot@npo{Pochuri Naga}
\def\langnames@langs@glot@pdn{Podena}
\def\langnames@langs@glot@pof{Poke}
\def\langnames@langs@glot@pkb{Pokomo}
\def\langnames@langs@glot@pld{Polari}
\def\langnames@langs@glot@plj{Pesse}
\def\langnames@langs@glot@pso{Polish Sign Language}
\def\langnames@langs@glot@plb{Polonombauk}
\def\langnames@langs@glot@pmo{Pom}
\def\langnames@langs@glot@pmm{Pol}
\def\langnames@langs@glot@ncc{Ponam}
\def\langnames@langs@glot@png{Pongu}
\def\langnames@langs@glot@pns{Ponosakan}
\def\langnames@langs@glot@pnt{Pontic}
\def\langnames@langs@glot@prh{Porohanon}
\def\langnames@langs@glot@ptv{Daakie}
\def\langnames@langs@glot@pmx{Poumei Naga}
\def\langnames@langs@glot@bye{Pouye}
\def\langnames@langs@glot@pwr{Powari}
\def\langnames@langs@glot@pyn{Poyanáwa}
\def\langnames@langs@glot@prz{Providencia Sign Language}
\def\langnames@langs@glot@prg{Old Prussian}
\def\langnames@langs@glot@kvj{Psikye}
\def\langnames@langs@glot@pux{Puare}
\def\langnames@langs@glot@atp{Pudtol Atta}
\def\langnames@langs@glot@pbm{Puebla and Northeastern Mazatec}
\def\langnames@langs@glot@psl{Puerto Rican Sign Language}
\def\langnames@langs@glot@pkp{Pukapuka}
\def\langnames@langs@glot@pup{Pulabu}
\def\langnames@langs@glot@pum{Puma}
\def\langnames@langs@glot@xpm{Pumpokol}
\def\langnames@langs@glot@puj{Punan Tubu}
\def\langnames@langs@glot@pud{Punan Aput}
\def\langnames@langs@glot@puf{Punan Merah}
\def\langnames@langs@glot@pna{Punan Bah-Biau}
\def\langnames@langs@glot@pnm{Punan Batu 1}
\def\langnames@langs@glot@xpu{Punic}
\def\langnames@langs@glot@qxp{Puno Quechua}
\def\langnames@langs@glot@puu{Punu}
\def\langnames@langs@glot@pru{Puragi}
\def\langnames@langs@glot@iar{Purari}
\def\langnames@langs@glot@puy{Purisimeño}
\def\langnames@langs@glot@prr{Puri}
\def\langnames@langs@glot@pur{Puruborá}
\def\langnames@langs@glot@pub{Purum}
\def\langnames@langs@glot@mfl{Putai}
\def\langnames@langs@glot@afe{Utugwang-Irungene-Afrike}
\def\langnames@langs@glot@cpx{Pu-Xian Chinese}
\def\langnames@langs@glot@pyu{Puyuma}
\def\langnames@langs@glot@pme{Pwaamei}
\def\langnames@langs@glot@pop{Pwapwa}
\def\langnames@langs@glot@pwo{Pwo Western Karen}
\def\langnames@langs@glot@pcw{Pyapun}
\def\langnames@langs@glot@pye{Pye Krumen}
\def\langnames@langs@glot@pyy{Pyen}
\def\langnames@langs@glot@pby{Pyu}
\def\langnames@langs@glot@laq{Pubiao-Qabiao}
\def\langnames@langs@glot@qxq{Qashqa'i}
\def\langnames@langs@glot@xqt{Qatabanian}
\def\langnames@langs@glot@ymq{Qila Muji}
\def\langnames@langs@glot@zqe{Qiubei Zhuang}
\def\langnames@langs@glot@qua{Quapaw}
\def\langnames@langs@glot@qya{Quenya}
\def\langnames@langs@glot@qvy{Queyu}
\def\langnames@langs@glot@zpj{Quiavicuzas Zapotec}
\def\langnames@langs@glot@quq{Quinqui}
\def\langnames@langs@glot@qun{Quinault}
\def\langnames@langs@glot@ztq{Quioquitani-Quieri Zapotec}
\def\langnames@langs@glot@rah{Rabha}
\def\langnames@langs@glot@xrr{Raetic}
\def\langnames@langs@glot@raz{Rahambuu}
\def\langnames@langs@glot@mqk{Rajah Kabunsuwan Manobo}
\def\langnames@langs@glot@rjs{Rajbanshi}
\def\langnames@langs@glot@rjg{Rajong}
\def\langnames@langs@glot@gra{Rajput Garasia}
\def\langnames@langs@glot@rkh{Rakahanga-Manihiki}
\def\langnames@langs@glot@rki{Rakhine}
\def\langnames@langs@glot@rai{Ramoaaina}
\def\langnames@langs@glot@kjx{Ramopa}
\def\langnames@langs@glot@lje{Rampi}
\def\langnames@langs@glot@thr{Rana Tharu}
\def\langnames@langs@glot@rkt{Central-Eastern Kamta}
\def\langnames@langs@glot@rnl{Halam}
\def\langnames@langs@glot@rax{Rang}
\def\langnames@langs@glot@ray{Mangaia-Old Rapa}
\def\langnames@langs@glot@rpt{Rapting}
\def\langnames@langs@glot@lra{Rara Bakati'}
\def\langnames@langs@glot@rar{Southern Cook Island Maori}
\def\langnames@langs@glot@rac{Rasawa}
\def\langnames@langs@glot@btn{Ratagnon}
\def\langnames@langs@glot@bgd{Rathwi Bareli}
\def\langnames@langs@glot@rtw{Rathawi}
\def\langnames@langs@glot@rau{Raute}
\def\langnames@langs@glot@yea{Ravula}
\def\langnames@langs@glot@jnl{Rawat}
\def\langnames@langs@glot@rat{Razajerdi}
\def\langnames@langs@glot@gir{Red Gelao}
\def\langnames@langs@glot@atu{Reel}
\def\langnames@langs@glot@ree{Rejang Kayan}
\def\langnames@langs@glot@rei{Reli}
\def\langnames@langs@glot@bow{Rema}
\def\langnames@langs@glot@reb{Rembong-Wangka}
\def\langnames@langs@glot@agv{Hatang Kayi}
\def\langnames@langs@glot@rem{Remo of the Moa river}
\def\langnames@langs@glot@rmp{Rempi}
\def\langnames@langs@glot@lkj{Remun}
\def\langnames@langs@glot@rsi{Rennellese Sign Language}
\def\langnames@langs@glot@rea{Rerau}
\def\langnames@langs@glot@rer{Rer Bare}
\def\langnames@langs@glot@pgk{Rerep}
\def\langnames@langs@glot@res{Reshe}
\def\langnames@langs@glot@ret{Reta}
\def\langnames@langs@glot@rcf{Réunion Creole French}
\def\langnames@langs@glot@rey{Reyesano}
\def\langnames@langs@glot@ril{Riang (Myanmar)}
\def\langnames@langs@glot@ria{Riang (India)}
\def\langnames@langs@glot@rir{Ribun}
\def\langnames@langs@glot@zar{Rincón Zapotec}
\def\langnames@langs@glot@rgu{Ringgou}
\def\langnames@langs@glot@hrx{Hunsrik}
\def\langnames@langs@glot@rri{Ririo}
\def\langnames@langs@glot@riu{Riung}
\def\langnames@langs@glot@snj{Riverain Sango}
\def\langnames@langs@glot@rod{Rogo}
\def\langnames@langs@glot@rhg{Rohingya}
\def\langnames@langs@glot@rge{Romano-Greek}
\def\langnames@langs@glot@rms{Romanian Sign Language}
\def\langnames@langs@glot@rgn{Romagnol}
\def\langnames@langs@glot@rmx{Romam}
\def\langnames@langs@glot@rmm{Roma}
\def\langnames@langs@glot@rmv{Romanova}
\def\langnames@langs@glot@rof{Rombo}
\def\langnames@langs@glot@rol{Romblomanon}
\def\langnames@langs@glot@rmk{Romkun}
\def\langnames@langs@glot@ror{Rongga}
\def\langnames@langs@glot@roe{Ronji}
\def\langnames@langs@glot@rnn{Roon}
\def\langnames@langs@glot@rga{Mores}
\def\langnames@langs@glot@pce{Ruching Palaung}
\def\langnames@langs@glot@rdb{Rudbari}
\def\langnames@langs@glot@ruh{Ruga}
\def\langnames@langs@glot@rbb{Rumai Palaung}
\def\langnames@langs@glot@ruz{Ruma}
\def\langnames@langs@glot@rna{Runa}
\def\langnames@langs@glot@rnw{Rungwa}
\def\langnames@langs@glot@drg{Rungus}
\def\langnames@langs@glot@bxr{Russia Buriat}
\def\langnames@langs@glot@rue{Rusyn}
\def\langnames@langs@glot@ruc{Ruuli}
\def\langnames@langs@glot@rnd{Ruund}
\def\langnames@langs@glot@rwk{Rwa}
\def\langnames@langs@glot@rsn{Rwandan Sign Language}
\def\langnames@langs@glot@sax{Sa}
\def\langnames@langs@glot@sav{Saafi-Saafi}
\def\langnames@langs@glot@raq{Saam}
\def\langnames@langs@glot@lsm{Saamia}
\def\langnames@langs@glot@sxr{Saaroa}
\def\langnames@langs@glot@spy{Sabaot}
\def\langnames@langs@glot@msi{Sabah Malay}
\def\langnames@langs@glot@bsy{Sabah Bisaya}
\def\langnames@langs@glot@sae{Sabanê}
\def\langnames@langs@glot@saa{Saba}
\def\langnames@langs@glot@xsa{Sabaic}
\def\langnames@langs@glot@qhr{Old Sabellic}
\def\langnames@langs@glot@sbo{Sabüm}
\def\langnames@langs@glot@quv{Sacapulteco}
\def\langnames@langs@glot@sck{Sadri}
\def\langnames@langs@glot@spd{Saep}
\def\langnames@langs@glot@saf{Safaliba}
\def\langnames@langs@glot@sbk{Safwa}
\def\langnames@langs@glot@sbm{Sagala}
\def\langnames@langs@glot@tga{Sagalla}
\def\langnames@langs@glot@aec{Saidi Arabic}
\def\langnames@langs@glot@acf{Saint Lucian Creole French}
\def\langnames@langs@glot@xsy{Saisiyat}
\def\langnames@langs@glot@sjl{Sajolang}
\def\langnames@langs@glot@sjb{Sajau-Latti}
\def\langnames@langs@glot@sch{Sakachep-Chorei}
\def\langnames@langs@glot@skt{Sakata}
\def\langnames@langs@glot@skg{West Malagasy Sakalava}
\def\langnames@langs@glot@skm{Sakam}
\def\langnames@langs@glot@sak{Sake}
\def\langnames@langs@glot@szy{Sakizaya}
\def\langnames@langs@glot@shq{Sala}
\def\langnames@langs@glot@slx{Salampasu}
\def\langnames@langs@glot@sgu{Salas}
\def\langnames@langs@glot@qxl{Tungurahua Highland Quichua}
\def\langnames@langs@glot@mnd{Salamãi}
\def\langnames@langs@glot@slq{Salchuq}
\def\langnames@langs@glot@sau{Saleman}
\def\langnames@langs@glot@loe{Saluan}
\def\langnames@langs@glot@esn{Salvadoran Sign Language}
\def\langnames@langs@glot@tmj{Samarokena}
\def\langnames@langs@glot@ysd{Samatao}
\def\langnames@langs@glot@smp{Samaritan}
\def\langnames@langs@glot@xab{Sambe}
\def\langnames@langs@glot@smx{Samba}
\def\langnames@langs@glot@ccg{Samba Daka}
\def\langnames@langs@glot@saq{Samburu}
\def\langnames@langs@glot@ssx{Samberigi}
\def\langnames@langs@glot@spv{Sambalpuri}
\def\langnames@langs@glot@smh{Samei}
\def\langnames@langs@glot@snx{Sam}
\def\langnames@langs@glot@swm{Samosa}
\def\langnames@langs@glot@rav{Sampang}
\def\langnames@langs@glot@stu{Samtao}
\def\langnames@langs@glot@smv{Samvedi}
\def\langnames@langs@glot@ztm{San Agustín Mixtepec Zapotec}
\def\langnames@langs@glot@icr{San Andres Creole English}
\def\langnames@langs@glot@spn{Sanapaná}
\def\langnames@langs@glot@zpx{San Baltazar Loxicha Zapotec}
\def\langnames@langs@glot@cuk{San Blas Kuna}
\def\langnames@langs@glot@hve{San Dionisio del Mar Huave}
\def\langnames@langs@glot@hue{San Francisco del Mar Huave}
\def\langnames@langs@glot@mat{San Francisco Matlatzinca}
\def\langnames@langs@glot@pow{San Felipe Otlaltepec Popoloca}
\def\langnames@langs@glot@xso{San Francisco Solano}
\def\langnames@langs@glot@sgr{Sangisari}
\def\langnames@langs@glot@sgk{Sangkong}
\def\langnames@langs@glot@nsa{Sangtam Naga}
\def\langnames@langs@glot@xsn{Sanga (Nigeria)}
\def\langnames@langs@glot@sbp{Sangu (Tanzania)}
\def\langnames@langs@glot@sng{Sanga (Democratic Republic of Congo)}
\def\langnames@langs@glot@snl{Sangil}
\def\langnames@langs@glot@scg{Sanggau}
\def\langnames@langs@glot@sgy{Sanglechi}
\def\langnames@langs@glot@ysy{Sanie}
\def\langnames@langs@glot@ysn{Sani}
\def\langnames@langs@glot@sny{Saniyo-Hiyewe}
\def\langnames@langs@glot@xtj{San Juan Teita Mixtec}
\def\langnames@langs@glot@maa{San Jerónimo Tecóatl Mazatec}
\def\langnames@langs@glot@msc{Sankaran Maninka}
\def\langnames@langs@glot@pps{San Luís Temalacayuca Popoloca}
\def\langnames@langs@glot@qvs{San Martín Quechua}
\def\langnames@langs@glot@xtp{San Miguel Piedras Mixtec}
\def\langnames@langs@glot@trq{San Martín Itunyoso Triqui}
\def\langnames@langs@glot@pls{San Marcos Tlalcoyalco Popoloca}
\def\langnames@langs@glot@azg{San Pedro Amuzgos Amuzgo}
\def\langnames@langs@glot@zpf{San Pedro Quiatoni Zapotec}
\def\langnames@langs@glot@san{Sanskrit}
\def\langnames@langs@glot@ssi{Sansi}
\def\langnames@langs@glot@kwy{San Salvador Kongo}
\def\langnames@langs@glot@hvv{Santa María del Mar Huave}
\def\langnames@langs@glot@nhz{Santa María La Alta Nahuatl}
\def\langnames@langs@glot@cok{Santa Teresa Cora}
\def\langnames@langs@glot@qus{Santiago del Estero Quichua}
\def\langnames@langs@glot@mza{Santa María Zacatepec Mixtec}
\def\langnames@langs@glot@mdv{Santa Lucía Monteverde Mixtec}
\def\langnames@langs@glot@zpn{Santa Inés Yatzechi Zapotec}
\def\langnames@langs@glot@ztn{Santa Catarina Albarradas Zapotec}
\def\langnames@langs@glot@zas{Santo Domingo Albarradas Zapotec}
\def\langnames@langs@glot@zpr{Santiago Xanica Zapotec}
\def\langnames@langs@glot@pca{Santa Inés Ahuatempan Popoloca}
\def\langnames@langs@glot@zpt{San Vicente Coatlán Zapotec}
\def\langnames@langs@glot@scq{Sa'och}
\def\langnames@langs@glot@zkp{São Paulo Kaingáng}
\def\langnames@langs@glot@cri{Sãotomense}
\def\langnames@langs@glot@spr{Saparua}
\def\langnames@langs@glot@spc{Sapé}
\def\langnames@langs@glot@krn{Sapo}
\def\langnames@langs@glot@spi{Saponi}
\def\langnames@langs@glot@sbz{Sara Kaba}
\def\langnames@langs@glot@kwv{Sara Kaba Náà}
\def\langnames@langs@glot@kwg{Sara Kaba Deme}
\def\langnames@langs@glot@zsa{Sarasira}
\def\langnames@langs@glot@bps{Sarangani Blaan}
\def\langnames@langs@glot@mbs{Sarangani Manobo}
\def\langnames@langs@glot@sre{Sara Bakati'}
\def\langnames@langs@glot@sar{Saraveca}
\def\langnames@langs@glot@srh{Sarikoli}
\def\langnames@langs@glot@mwm{Sar}
\def\langnames@langs@glot@onp{Sartang}
\def\langnames@langs@glot@sdu{Sarudu}
\def\langnames@langs@glot@sra{Saruga}
\def\langnames@langs@glot@swy{Sarua}
\def\langnames@langs@glot@sxs{Sasaru}
\def\langnames@langs@glot@sas{Sasak}
\def\langnames@langs@glot@sdc{Sassarese Sardinian}
\def\langnames@langs@glot@stw{Satawalese}
\def\langnames@langs@glot@stq{Ems-Weser Frisian}
\def\langnames@langs@glot@mav{Sateré-Mawé}
\def\langnames@langs@glot@sdl{Saudi Arabian Sign Language}
\def\langnames@langs@glot@skc{Ma Manda}
\def\langnames@langs@glot@saz{Saurashtra}
\def\langnames@langs@glot@mjt{Sauria Paharia}
\def\langnames@langs@glot@srt{Sauri}
\def\langnames@langs@glot@psu{Sauraseni Prakrit}
\def\langnames@langs@glot@ssj{Sausi}
\def\langnames@langs@glot@sao{Sause}
\def\langnames@langs@glot@swr{Saweru}
\def\langnames@langs@glot@swt{Sawila}
\def\langnames@langs@glot@saw{Sawi}
\def\langnames@langs@glot@swn{Sawknah-Fogaha}
\def\langnames@langs@glot@sxw{Saxwe Gbe}
\def\langnames@langs@glot@say{Saya}
\def\langnames@langs@glot@sco{Scots}
\def\langnames@langs@glot@kdg{Seba}
\def\langnames@langs@glot@sbx{Seberuang}
\def\langnames@langs@glot@sib{Sebop}
\def\langnames@langs@glot@sec{Sechelt}
\def\langnames@langs@glot@tvw{Sedoa}
\def\langnames@langs@glot@sos{Seeku}
\def\langnames@langs@glot@sge{Segai}
\def\langnames@langs@glot@sbg{Seget}
\def\langnames@langs@glot@seg{Segeju}
\def\langnames@langs@glot@sfw{Sehwi}
\def\langnames@langs@glot@ssg{Seimat}
\def\langnames@langs@glot@hik{Seit-Kaitetu}
\def\langnames@langs@glot@skz{Sekar}
\def\langnames@langs@glot@skp{Sekapan}
\def\langnames@langs@glot@sek{Sekani}
\def\langnames@langs@glot@ske{Seke (Vanuatu)}
\def\langnames@langs@glot@syi{Seki}
\def\langnames@langs@glot@sko{Seko Tengah}
\def\langnames@langs@glot@skx{Seko Padang}
\def\langnames@langs@glot@lip{Sekpele}
\def\langnames@langs@glot@kgi{Selangor Sign Language}
\def\langnames@langs@glot@snw{Selee}
\def\langnames@langs@glot@sws{Seluwasan}
\def\langnames@langs@glot@slg{Selungai Murut}
\def\langnames@langs@glot@szc{Semaq Beri}
\def\langnames@langs@glot@sbr{Sembakung Murut}
\def\langnames@langs@glot@etz{Semimi}
\def\langnames@langs@glot@smy{Semnani-Biyabuneki}
\def\langnames@langs@glot@ssm{Semnam}
\def\langnames@langs@glot@xse{Sempan}
\def\langnames@langs@glot@seq{Senar de Kankalaba}
\def\langnames@langs@glot@sej{Sene}
\def\langnames@langs@glot@sds{Sened}
\def\langnames@langs@glot@ssz{Sengseng}
\def\langnames@langs@glot@spk{Sengo}
\def\langnames@langs@glot@snu{Senggi}
\def\langnames@langs@glot@sjs{Senhaja De Srair}
\def\langnames@langs@glot@sni{Sensi}
\def\langnames@langs@glot@std{Sentinel}
\def\langnames@langs@glot@sez{Senthang Chin}
\def\langnames@langs@glot@spe{Sepa (Papua New Guinea)}
\def\langnames@langs@glot@spb{Sepa (Indonesia)}
\def\langnames@langs@glot@spm{Sepen}
\def\langnames@langs@glot@iws{Sepik Iwam}
\def\langnames@langs@glot@skr{Saraiki}
\def\langnames@langs@glot@sry{Sera}
\def\langnames@langs@glot@srr{Sereer}
\def\langnames@langs@glot@swf{Sere}
\def\langnames@langs@glot@sve{Serili}
\def\langnames@langs@glot@seu{Serui-Laut}
\def\langnames@langs@glot@srw{Serua}
\def\langnames@langs@glot@srk{Serudung Murut}
\def\langnames@langs@glot@stf{Seta}
\def\langnames@langs@glot@stm{Setaman}
\def\langnames@langs@glot@sbi{Seti}
\def\langnames@langs@glot@sta{KiSetla}
\def\langnames@langs@glot@sew{Sewa Bay}
\def\langnames@langs@glot@lsw{Seychelles Sign Language}
\def\langnames@langs@glot@sze{Seze}
\def\langnames@langs@glot@scw{Sya}
\def\langnames@langs@glot@sdb{Shabaki}
\def\langnames@langs@glot@srz{Shahmirzadi}
\def\langnames@langs@glot@sha{Shall-Zwall}
\def\langnames@langs@glot@xsh{Shamang}
\def\langnames@langs@glot@sqa{Shama-Sambuga}
\def\langnames@langs@glot@jih{Stodsde}
\def\langnames@langs@glot@sho{Shanga}
\def\langnames@langs@glot@swo{Shanenawa}
\def\langnames@langs@glot@ssv{Ngen}
\def\langnames@langs@glot@swq{Sharwa}
\def\langnames@langs@glot@sqh{Shau}
\def\langnames@langs@glot@shx{She}
\def\langnames@langs@glot@she{Sheko}
\def\langnames@langs@glot@sth{Shelta}
\def\langnames@langs@glot@shl{Shendu}
\def\langnames@langs@glot@scv{Sheni-Ziriya}
\def\langnames@langs@glot@bun{Sherbro}
\def\langnames@langs@glot@kip{Sheshi Kham}
\def\langnames@langs@glot@ssh{Shihhi Arabic}
\def\langnames@langs@glot@shr{Shi}
\def\langnames@langs@glot@gua{Shiki}
\def\langnames@langs@glot@snh{Shinabo}
\def\langnames@langs@glot@sxg{Shixing}
\def\langnames@langs@glot@sle{Sholaga}
\def\langnames@langs@glot@bcv{Shoo-Minda-Nye}
\def\langnames@langs@glot@suj{Shubi}
\def\langnames@langs@glot@sts{Shumashti}
\def\langnames@langs@glot@scu{Shumcho}
\def\langnames@langs@glot@ksa{Shuwa-Zamani}
\def\langnames@langs@glot@shw{Shwai}
\def\langnames@langs@glot@slw{Sialum}
\def\langnames@langs@glot@sya{Siang}
\def\langnames@langs@glot@spg{Sihan}
\def\langnames@langs@glot@mmp{Siawi}
\def\langnames@langs@glot@nco{Sibe (Nasioi)}
\def\langnames@langs@glot@sty{Siberian Tatar}
\def\langnames@langs@glot@sdx{Sibu Melanau}
\def\langnames@langs@glot@sxc{Sicana}
\def\langnames@langs@glot@scn{Sicilian}
\def\langnames@langs@glot@sep{Sìcìté Sénoufo}
\def\langnames@langs@glot@scx{Sicula}
\def\langnames@langs@glot@xsd{Sidetic}
\def\langnames@langs@glot@sgx{Sierra Leone Sign Language}
\def\langnames@langs@glot@nsu{Sierra Negra Nahuatl}
\def\langnames@langs@glot@sxe{Sighu}
\def\langnames@langs@glot@snr{Sihan (Gum)}
\def\langnames@langs@glot@qws{Sihuas Ancash Quechua}
\def\langnames@langs@glot@sky{Sikaiana}
\def\langnames@langs@glot@slt{Sila}
\def\langnames@langs@glot@szl{Silesian}
\def\langnames@langs@glot@sbq{Sirva}
\def\langnames@langs@glot@mkc{Siliput}
\def\langnames@langs@glot@wul{Silimo}
\def\langnames@langs@glot@xsp{Silopi}
\def\langnames@langs@glot@stv{Silt'e}
\def\langnames@langs@glot@sie{Simaa}
\def\langnames@langs@glot@sbw{Simba}
\def\langnames@langs@glot@smb{Simbari}
\def\langnames@langs@glot@sbb{Simbo}
\def\langnames@langs@glot@smg{Simbali}
\def\langnames@langs@glot@smz{Simeku}
\def\langnames@langs@glot@smt{Simte}
\def\langnames@langs@glot@siu{Galu}
\def\langnames@langs@glot@sbn{Sindhi Bhil}
\def\langnames@langs@glot@xts{Sindihui Mixtec}
\def\langnames@langs@glot@sjn{Sindarin}
\def\langnames@langs@glot@sgp{Northern Jinghpaw}
\def\langnames@langs@glot@sgm{Singa}
\def\langnames@langs@glot@skq{Sininkere}
\def\langnames@langs@glot@xti{Sinicahua Mixtec}
\def\langnames@langs@glot@snz{Kou}
\def\langnames@langs@glot@sys{Sinyar}
\def\langnames@langs@glot@swj{Sira}
\def\langnames@langs@glot@sir{Siri}
\def\langnames@langs@glot@srx{Sirmauri}
\def\langnames@langs@glot@sld{Sissala of Burkina Faso}
\def\langnames@langs@glot@sso{Sissano}
\def\langnames@langs@glot@siy{Sivandi}
\def\langnames@langs@glot@lsv{Sivia Sign Language}
\def\langnames@langs@glot@akp{Siwu}
\def\langnames@langs@glot@skw{Skepi Creole Dutch}
\def\langnames@langs@glot@sms{Skolt Saami}
\def\langnames@langs@glot@svm{Slavomolisano}
\def\langnames@langs@glot@svk{Slovakian Sign Language}
\def\langnames@langs@glot@sfm{Gha-mu}
\def\langnames@langs@glot@kxq{Smärky Kanum}
\def\langnames@langs@glot@sox{So (Cameroon)}
\def\langnames@langs@glot@soc{So (Democratic Republic of Congo)}
\def\langnames@langs@glot@xog{Soga}
\def\langnames@langs@glot@sog{Sogdian}
\def\langnames@langs@glot@soj{Soic}
\def\langnames@langs@glot@sok{Sokoro}
\def\langnames@langs@glot@sby{Soli}
\def\langnames@langs@glot@sol{Solos}
\def\langnames@langs@glot@aaw{Solong}
\def\langnames@langs@glot@szs{Solomon Islands Sign Language}
\def\langnames@langs@glot@smc{Som}
\def\langnames@langs@glot@smu{Somray of Battambang-Somre of Siem Reap}
\def\langnames@langs@glot@sor{Somrai}
\def\langnames@langs@glot@kgt{Somyev}
\def\langnames@langs@glot@ysg{Sonaga}
\def\langnames@langs@glot@shc{Sonde}
\def\langnames@langs@glot@soo{Nsong-Mpiin}
\def\langnames@langs@glot@sod{Songoora}
\def\langnames@langs@glot@soe{Ohendo}
\def\langnames@langs@glot@soi{Sonha}
\def\langnames@langs@glot@siq{Sonia}
\def\langnames@langs@glot@sss{Sô}
\def\langnames@langs@glot@urw{Sop}
\def\langnames@langs@glot@sbh{Sori-Harengan}
\def\langnames@langs@glot@sqo{Sorkhei-Aftari}
\def\langnames@langs@glot@ays{Sorsogon Ayta}
\def\langnames@langs@glot@sdk{Sos Kundi}
\def\langnames@langs@glot@krz{Sota Kanum}
\def\langnames@langs@glot@sfs{South African Sign Language}
\def\langnames@langs@glot@nit{Southeastern Kolami}
\def\langnames@langs@glot@hmy{Southern Guiyang Hmong}
\def\langnames@langs@glot@hma{Southern Mashan Hmong}
\def\langnames@langs@glot@sdh{Southern Kurdish}
\def\langnames@langs@glot@bcc{Southern Balochi}
\def\langnames@langs@glot@fay{Fars Dialects}
\def\langnames@langs@glot@luz{Southern Luri}
\def\langnames@langs@glot@pbt{Southern Pashto}
\def\langnames@langs@glot@hnd{Southern Hindko}
\def\langnames@langs@glot@psh{Southwest Pashayi}
\def\langnames@langs@glot@psi{Southeast Pashayi}
\def\langnames@langs@glot@vro{South Estonian}
\def\langnames@langs@glot@nik{Southern Nicobarese}
\def\langnames@langs@glot@mnn{Southern Mnong}
\def\langnames@langs@glot@uzs{Southern Uzbek}
\def\langnames@langs@glot@ghe{Southern Ghale}
\def\langnames@langs@glot@ymc{Southern Muji}
\def\langnames@langs@glot@nsd{Southern Nisu}
\def\langnames@langs@glot@qxs{Southern Qiang}
\def\langnames@langs@glot@pmj{Southern Pumi}
\def\langnames@langs@glot@bfs{Southern Bai}
\def\langnames@langs@glot@nre{Southern Rengma Naga}
\def\langnames@langs@glot@lrr{Southern Yamphu}
\def\langnames@langs@glot@tjs{Southern Tujia}
\def\langnames@langs@glot@sou{Southern Thai}
\def\langnames@langs@glot@hms{Southern Qiandong Miao}
\def\langnames@langs@glot@hmh{Southwestern Huishui Hmong}
\def\langnames@langs@glot@hmg{Southwestern Guiyang Hmong}
\def\langnames@langs@glot@xtv{Southern Coastal Yuin}
\def\langnames@langs@glot@ijs{Southeast Ijo}
\def\langnames@langs@glot@fal{South Fali}
\def\langnames@langs@glot@nbw{Southern Ngbandi}
\def\langnames@langs@glot@lnl{South Central Banda}
\def\langnames@langs@glot@biv{Southern Birifor}
\def\langnames@langs@glot@nnw{Southern Nuni}
\def\langnames@langs@glot@snm{Southern Ma'di}
\def\langnames@langs@glot@dik{Southwestern Dinka}
\def\langnames@langs@glot@dib{South Central Dinka}
\def\langnames@langs@glot@dks{Southeastern Dinka}
\def\langnames@langs@glot@bwq{Southern Bobo Madaré}
\def\langnames@langs@glot@sbd{Southern Samo}
\def\langnames@langs@glot@sns{Nahavaq}
\def\langnames@langs@glot@mqm{South Marquesan}
\def\langnames@langs@glot@mcy{South Watut}
\def\langnames@langs@glot@vbb{Southeast Babar}
\def\langnames@langs@glot@lmf{Eastern Atadei}
\def\langnames@langs@glot@agy{Southern Alta}
\def\langnames@langs@glot@ksc{Bangad}
\def\langnames@langs@glot@bln{Coastal-Virac Bikol}
\def\langnames@langs@glot@plv{Southwest Palawano}
\def\langnames@langs@glot@bzc{Southern Betsimisaraka Malagasy}
\def\langnames@langs@glot@osu{Southern One}
\def\langnames@langs@glot@aws{South Awyu}
\def\langnames@langs@glot@omw{South Tairora}
\def\langnames@langs@glot@ams{Southern Amami-Oshima}
\def\langnames@langs@glot@hax{Southern Haida}
\def\langnames@langs@glot@tce{Southern Tutchone}
\def\langnames@langs@glot@caf{Southern Carrier}
\def\langnames@langs@glot@twr{Southwestern Tarahumara}
\def\langnames@langs@glot@tcu{Southeastern Tarahumara}
\def\langnames@langs@glot@npl{Nahuatl, Southeastern Puebla}
\def\langnames@langs@glot@tla{Southwestern Tepehuan}
\def\langnames@langs@glot@crj{Southern East Cree}
\def\langnames@langs@glot@peq{Southern Pomo}
\def\langnames@langs@glot@qup{Southern Pastaza Quechua}
\def\langnames@langs@glot@qxo{Southern Conchucos Ancash Quechua}
\def\langnames@langs@glot@ayc{Southern Aymara}
\def\langnames@langs@glot@meh{Southwestern Tlaxiaco Mixtec}
\def\langnames@langs@glot@mit{Southern Puebla Mixtec}
\def\langnames@langs@glot@mxy{Southeastern Nochixtlán Mixtec}
\def\langnames@langs@glot@rgs{Southern Roglai}
\def\langnames@langs@glot@giz{South Giziga}
\def\langnames@langs@glot@cpy{South Ucayali Ashéninka}
\def\langnames@langs@glot@itd{Southern Tidung}
\def\langnames@langs@glot@csp{Southern Pinghua}
\def\langnames@langs@glot@sct{Southern Katang}
\def\langnames@langs@glot@sqq{Sou}
\def\langnames@langs@glot@sww{Sowa}
\def\langnames@langs@glot@sow{Sowanda}
\def\langnames@langs@glot@vmq{Soyaltepec Mixtec}
\def\langnames@langs@glot@vmp{Soyaltepec Mazatec}
\def\langnames@langs@glot@sqs{Sri Lankan Sign Language}
\def\langnames@langs@glot@sci{Sri Lanka Malay}
\def\langnames@langs@glot@seo{Asabano}
\def\langnames@langs@glot@swp{Suau}
\def\langnames@langs@glot@sxb{Suba}
\def\langnames@langs@glot@ssc{Suba-Simbiti}
\def\langnames@langs@glot@sut{Subtiaba}
\def\langnames@langs@glot@apd{Sudanese Arabic}
\def\langnames@langs@glot@pga{South Sudanese Creole Arabic}
\def\langnames@langs@glot@sgi{Nizaa}
\def\langnames@langs@glot@sug{Suganga}
\def\langnames@langs@glot@kzs{Sugut Dusun}
\def\langnames@langs@glot@zsu{Sukurum}
\def\langnames@langs@glot@syk{Sukur}
\def\langnames@langs@glot@szn{Sula}
\def\langnames@langs@glot@srg{Sulod}
\def\langnames@langs@glot@sqm{Suma}
\def\langnames@langs@glot@siv{Sumariup}
\def\langnames@langs@glot@six{Sumau}
\def\langnames@langs@glot@suw{Sumbwa}
\def\langnames@langs@glot@smw{Sumbawa}
\def\langnames@langs@glot@sux{Sumerian}
\def\langnames@langs@glot@csv{Sumtu Chin}
\def\langnames@langs@glot@ssk{Sunam}
\def\langnames@langs@glot@suz{Sunwar}
\def\langnames@langs@glot@syo{Suoy}
\def\langnames@langs@glot@sbj{Surbakhal}
\def\langnames@langs@glot@sgd{Surigaonon}
\def\langnames@langs@glot@sjp{Surjapuri}
\def\langnames@langs@glot@tdl{Sur}
\def\langnames@langs@glot@sde{Vori}
\def\langnames@langs@glot@mdz{Suruí Do Pará}
\def\langnames@langs@glot@sru{Suruí}
\def\langnames@langs@glot@swx{Suruahá}
\def\langnames@langs@glot@sqn{Susquehannock}
\def\langnames@langs@glot@ssu{Susuami}
\def\langnames@langs@glot@sdj{Suundi}
\def\langnames@langs@glot@swu{Suwawa}
\def\langnames@langs@glot@suy{Suyá}
\def\langnames@langs@glot@swg{Swabian}
\def\langnames@langs@glot@slf{Swiss-Italian Sign Language}
\def\langnames@langs@glot@sgg{Swiss-German Sign Language}
\def\langnames@langs@glot@ssr{Swiss-French Sign Language}
\def\langnames@langs@glot@xdk{Sydney}
\def\langnames@langs@glot@syl{Sylheti}
\def\langnames@langs@glot@zoq{Tabasco Zoque}
\def\langnames@langs@glot@nhc{Tabasco Nahuatl}
\def\langnames@langs@glot@zat{Tabaa Zapotec}
\def\langnames@langs@glot@knv{Tabo}
\def\langnames@langs@glot@tzx{Tabriak}
\def\langnames@langs@glot@xtt{Tacahua-Yolotepec Mixtec}
\def\langnames@langs@glot@lts{Tachoni}
\def\langnames@langs@glot@dsq{Tadaksahak}
\def\langnames@langs@glot@tdy{Tadyawan}
\def\langnames@langs@glot@rob{Tae'}
\def\langnames@langs@glot@tcd{Tafi}
\def\langnames@langs@glot@klg{Tagakaulu Kalagan}
\def\langnames@langs@glot@bgs{Tagabawa}
\def\langnames@langs@glot@mvv{Tagal Murut}
\def\langnames@langs@glot@tgz{Tagalaka}
\def\langnames@langs@glot@tbm{Tagbu}
\def\langnames@langs@glot@tda{Tagdal}
\def\langnames@langs@glot@tgx{Tagish}
\def\langnames@langs@glot@tgj{Tagin}
\def\langnames@langs@glot@tgw{Tagwana Senoufo}
\def\langnames@langs@glot@tht{Tahltan}
\def\langnames@langs@glot@blt{Tai Dam}
\def\langnames@langs@glot@tyj{Tai Do-Mene-Yo}
\def\langnames@langs@glot@tyr{Tai Daeng-Meuay}
\def\langnames@langs@glot@twh{Tai Dón}
\def\langnames@langs@glot@tiz{Tai Hongjin}
\def\langnames@langs@glot@taw{Tai}
\def\langnames@langs@glot@aos{Taikat}
\def\langnames@langs@glot@tlq{Muak}
\def\langnames@langs@glot@thi{Tai Long}
\def\langnames@langs@glot@tjl{Tai Laing}
\def\langnames@langs@glot@tdd{Tai Nüa}
\def\langnames@langs@glot@ago{Tainae}
\def\langnames@langs@glot@tnq{Taino}
\def\langnames@langs@glot@tpo{Tai Pao}
\def\langnames@langs@glot@uar{Tairuma}
\def\langnames@langs@glot@tmm{Tai Thanh}
\def\langnames@langs@glot@cuu{Tai Ya}
\def\langnames@langs@glot@acq{Ta'izzi-Adeni Arabic}
\def\langnames@langs@glot@pee{Taje}
\def\langnames@langs@glot@tdj{Tajio}
\def\langnames@langs@glot@abh{Tajiki Arabic}
\def\langnames@langs@glot@tja{Tajuasohn}
\def\langnames@langs@glot@tkz{Takua}
\def\langnames@langs@glot@nho{Takuu}
\def\langnames@langs@glot@tke{Takwane}
\def\langnames@langs@glot@tak{Tala}
\def\langnames@langs@glot@tdf{Talieng}
\def\langnames@langs@glot@tlr{Talise}
\def\langnames@langs@glot@tlv{Taliabu}
\def\langnames@langs@glot@tal{Tal}
\def\langnames@langs@glot@tln{Talondo'}
\def\langnames@langs@glot@tlk{Taloki}
\def\langnames@langs@glot@tzl{Talossan}
\def\langnames@langs@glot@yta{Lavu-Yongsheng-Talu}
\def\langnames@langs@glot@tcl{Taman (Myanmar)}
\def\langnames@langs@glot@tmn{Taman (Indonesia)}
\def\langnames@langs@glot@tmz{Tamanaku}
\def\langnames@langs@glot@vmx{Tamazola Mixtec}
\def\langnames@langs@glot@ten{Tama (Colombia)}
\def\langnames@langs@glot@tls{Tambotalo}
\def\langnames@langs@glot@xxt{Tambora}
\def\langnames@langs@glot@tdk{Tambas}
\def\langnames@langs@glot@tmy{Tami}
\def\langnames@langs@glot@tax{Tamki}
\def\langnames@langs@glot@tml{Tamnim Citak}
\def\langnames@langs@glot@tpu{Tampuan}
\def\langnames@langs@glot@low{Tampias Lobu}
\def\langnames@langs@glot@tpv{Tanapag}
\def\langnames@langs@glot@tcm{Tanahmerah}
\def\langnames@langs@glot@tni{Tandia}
\def\langnames@langs@glot@tdx{Tandroy Malagasy}
\def\langnames@langs@glot@tgn{Tandaganon}
\def\langnames@langs@glot@tnx{Tanema}
\def\langnames@langs@glot@tnv{Tangchangya}
\def\langnames@langs@glot@txg{Tangut}
\def\langnames@langs@glot@tgp{Movono}
\def\langnames@langs@glot@tkx{Tangko}
\def\langnames@langs@glot@tgu{Tanggu}
\def\langnames@langs@glot@tbs{Tanguat}
\def\langnames@langs@glot@ytl{Tanglang-Toloza}
\def\langnames@langs@glot@tbe{Tanimbili}
\def\langnames@langs@glot@uji{Rjili}
\def\langnames@langs@glot@txy{Tanosy Malagasy}
\def\langnames@langs@glot@xnj{Tanzanian Ngoni}
\def\langnames@langs@glot@qcs{Tapachultec}
\def\langnames@langs@glot@afp{Tapei}
\def\langnames@langs@glot@taf{Tapirapé}
\def\langnames@langs@glot@txj{Tarjumo}
\def\langnames@langs@glot@tpf{Tarpia}
\def\langnames@langs@glot@txr{Tartessian}
\def\langnames@langs@glot@tdm{Taruma}
\def\langnames@langs@glot@twq{Tasawaq}
\def\langnames@langs@glot@tmt{Tasmate}
\def\langnames@langs@glot@ttd{Tauade}
\def\langnames@langs@glot@tco{Taungyo}
\def\langnames@langs@glot@tpa{Taupota}
\def\langnames@langs@glot@tad{Tause}
\def\langnames@langs@glot@tvs{Taveta}
\def\langnames@langs@glot@tvn{Tavoyan}
\def\langnames@langs@glot@rmu{Tavringer Romani}
\def\langnames@langs@glot@twl{Tawara}
\def\langnames@langs@glot@xtw{Tawandê}
\def\langnames@langs@glot@ttq{Tawallammat Tamajaq}
\def\langnames@langs@glot@twy{Tawoyan}
\def\langnames@langs@glot@tbp{Taworta}
\def\langnames@langs@glot@tcp{Laamtuk Thet}
\def\langnames@langs@glot@ayy{Tayabas Ayta near Lucena City in Western Quezon}
\def\langnames@langs@glot@tas{Tay Boi}
\def\langnames@langs@glot@tnu{Tay Khang}
\def\langnames@langs@glot@tys{Tày Sa Pa}
\def\langnames@langs@glot@tyt{Tày Tac}
\def\langnames@langs@glot@tyz{Tày}
\def\langnames@langs@glot@tck{Tchitchege}
\def\langnames@langs@glot@bqa{Tchumbuli}
\def\langnames@langs@glot@dtu{Tebul Ure Dogon}
\def\langnames@langs@glot@tsy{Tebul Sign Language}
\def\langnames@langs@glot@tcw{Tecpatlán Totonac}
\def\langnames@langs@glot@tuq{Tedaga}
\def\langnames@langs@glot@tkq{Tee}
\def\langnames@langs@glot@lor{Téén}
\def\langnames@langs@glot@tfo{Tefaro}
\def\langnames@langs@glot@twe{Teiwa}
\def\langnames@langs@glot@ztt{Tejalapan Zapotec}
\def\langnames@langs@glot@teg{Latege}
\def\langnames@langs@glot@tyx{Teke-Tyee}
\def\langnames@langs@glot@lli{Teke-Laali}
\def\langnames@langs@glot@ebo{Teke-Eboo-Nzikou}
\def\langnames@langs@glot@tyi{Teke-Tsaayi}
\def\langnames@langs@glot@tvm{Tela-Masbuar}
\def\langnames@langs@glot@tlt{Teluti}
\def\langnames@langs@glot@nhv{Temascaltepec Nahuatl}
\def\langnames@langs@glot@tjo{Oued Righ}
\def\langnames@langs@glot@tbt{Tembo (Kitembo)}
\def\langnames@langs@glot@tmv{Motembo-Kunda}
\def\langnames@langs@glot@tqb{Tenetehara}
\def\langnames@langs@glot@tdo{Teme}
\def\langnames@langs@glot@soz{Temi}
\def\langnames@langs@glot@tmo{Temoq}
\def\langnames@langs@glot@ott{Temoaya Otomi}
\def\langnames@langs@glot@tmw{Temuan}
\def\langnames@langs@glot@quw{Tena Lowland Quichua}
\def\langnames@langs@glot@otn{Tenango Otomi}
\def\langnames@langs@glot@dtk{Tengou-Togo Dogon}
\def\langnames@langs@glot@tes{Tengger}
\def\langnames@langs@glot@pah{Tenharim-Parintintin-Diahoi}
\def\langnames@langs@glot@tqn{Tenino}
\def\langnames@langs@glot@tns{Tenis}
\def\langnames@langs@glot@tct{T'en}
\def\langnames@langs@glot@tev{Teor}
\def\langnames@langs@glot@cux{Tepeuxila Cuicatec}
\def\langnames@langs@glot@cte{Tepinapa Chinantec}
\def\langnames@langs@glot@ted{Tepo Krumen}
\def\langnames@langs@glot@tef{Teressa}
\def\langnames@langs@glot@trb{Terebu}
\def\langnames@langs@glot@twg{Tereweng}
\def\langnames@langs@glot@tec{Terik}
\def\langnames@langs@glot@tmg{Ternateño}
\def\langnames@langs@glot@sjt{Ter Saami}
\def\langnames@langs@glot@tkg{Tesaka Malagasy}
\def\langnames@langs@glot@keg{Tese}
\def\langnames@langs@glot@twc{Teshenawa}
\def\langnames@langs@glot@tez{Tetserret}
\def\langnames@langs@glot@tdt{Tetun Dili}
\def\langnames@langs@glot@tve{Te'un}
\def\langnames@langs@glot@cut{Teutila Cuicatec}
\def\langnames@langs@glot@twx{Tewe}
\def\langnames@langs@glot@otx{Texcatepec Otomi}
\def\langnames@langs@glot@poq{Texistepec Popoluca}
\def\langnames@langs@glot@mxb{Tezoatlán Mixtec}
\def\langnames@langs@glot@thy{Tha}
\def\langnames@langs@glot@thn{Thachanadan}
\def\langnames@langs@glot@soa{Thai Song}
\def\langnames@langs@glot@nki{Thangal Naga}
\def\langnames@langs@glot@thk{Tharaka}
\def\langnames@langs@glot@iin{Thiin}
\def\langnames@langs@glot@tou{Tho}
\def\langnames@langs@glot@ytp{Thopho}
\def\langnames@langs@glot@txh{Thracian}
\def\langnames@langs@glot@thu{Thuri}
\def\langnames@langs@glot@ahi{Tiagbamrin Aizi}
\def\langnames@langs@glot@mnl{Tiale}
\def\langnames@langs@glot@tbj{Tiang}
\def\langnames@langs@glot@ngy{Tibea}
\def\langnames@langs@glot@lsn{Tibetan Sign Language}
\def\langnames@langs@glot@tcn{Tichurong}
\def\langnames@langs@glot@mtx{Tidaá Mixtec}
\def\langnames@langs@glot@tia{Tidikelt-Tuat Tamazight}
\def\langnames@langs@glot@tiq{Tiefo-Daramandugu}
\def\langnames@langs@glot@boo{Tiemacèwè Bozo}
\def\langnames@langs@glot@tii{Tiene}
\def\langnames@langs@glot@nza{Tigon Mbembe}
\def\langnames@langs@glot@txq{Tii}
\def\langnames@langs@glot@xtl{Tijaltepec Mixtec}
\def\langnames@langs@glot@tkp{Tikopia}
\def\langnames@langs@glot@otl{Tilapa Otomi}
\def\langnames@langs@glot@zts{Tilquiapan Zapotec}
\def\langnames@langs@glot@tij{Tilung}
\def\langnames@langs@glot@tim{Timbe}
\def\langnames@langs@glot@tvy{Timor Pidgin}
\def\langnames@langs@glot@xsb{Tinà Sambal}
\def\langnames@langs@glot@tit{Tinigua}
\def\langnames@langs@glot@tpz{Tinputz}
\def\langnames@langs@glot@tpe{Tippera}
\def\langnames@langs@glot@tra{Tirahi}
\def\langnames@langs@glot@tic{Tira}
\def\langnames@langs@glot@tde{Tiranige Diga Dogon}
\def\langnames@langs@glot@tdq{Tita}
\def\langnames@langs@glot@ttv{Titan}
\def\langnames@langs@glot@lax{Tiwa (India)}
\def\langnames@langs@glot@tju{Tjurruru}
\def\langnames@langs@glot@tpl{Tlacoapa Me'phaa}
\def\langnames@langs@glot@ctl{Tlacoatzintepec Chinantec}
\def\langnames@langs@glot@zpk{Tlacolulita Zapotec}
\def\langnames@langs@glot@nuz{Tlamacazapa Nahuatl}
\def\langnames@langs@glot@mqh{Tlazoyaltepec Mixtec}
\def\langnames@langs@glot@tmf{Toba-Enenlhet}
\def\langnames@langs@glot@tng{Tobanga}
\def\langnames@langs@glot@tgh{Tobagonian Creole English}
\def\langnames@langs@glot@tox{Tobian}
\def\langnames@langs@glot@tgb{Tobilung}
\def\langnames@langs@glot@taz{Tocho}
\def\langnames@langs@glot@tdr{Todrah}
\def\langnames@langs@glot@tlg{Tofanma}
\def\langnames@langs@glot@tfi{Tofin Gbe}
\def\langnames@langs@glot@tor{Togbo-Vara Banda}
\def\langnames@langs@glot@tgy{Togoyo}
\def\langnames@langs@glot@zuh{Tokano}
\def\langnames@langs@glot@xto{Tokharian A}
\def\langnames@langs@glot@txb{Tokharian B}
\def\langnames@langs@glot@tok{Toki Pona}
\def\langnames@langs@glot@tkn{Toku-No-Shima}
\def\langnames@langs@glot@lbw{Tolaki}
\def\langnames@langs@glot@tlm{Tolomako}
\def\langnames@langs@glot@tol{Tolowa-Chetco}
\def\langnames@langs@glot@tod{Toma}
\def\langnames@langs@glot@tdi{Tomadino}
\def\langnames@langs@glot@tom{Tombulu}
\def\langnames@langs@glot@txa{Tombonuo}
\def\langnames@langs@glot@ttp{Tombelala}
\def\langnames@langs@glot@txm{Tomini}
\def\langnames@langs@glot@dtm{Tomo Kan Dogon}
\def\langnames@langs@glot@tqp{Tomoip}
\def\langnames@langs@glot@tst{Tondi Songway Kiini}
\def\langnames@langs@glot@tnz{Maniq}
\def\langnames@langs@glot@tny{Tongwe}
\def\langnames@langs@glot@tog{Tonga (Nyasa)}
\def\langnames@langs@glot@xgf{Tongva}
\def\langnames@langs@glot@tjn{Tonjon}
\def\langnames@langs@glot@tnw{Tonsawang}
\def\langnames@langs@glot@txs{Tonsea}
\def\langnames@langs@glot@toz{To}
\def\langnames@langs@glot@ttj{Tooro}
\def\langnames@langs@glot@toq{Toposa}
\def\langnames@langs@glot@toy{Topoiyo}
\def\langnames@langs@glot@ttu{Torau}
\def\langnames@langs@glot@trz{Torá}
\def\langnames@langs@glot@trj{Toram}
\def\langnames@langs@glot@fit{Meänkieli}
\def\langnames@langs@glot@tdv{Toro}
\def\langnames@langs@glot@tqr{Torona}
\def\langnames@langs@glot@dtt{Toro Tegu Dogon}
\def\langnames@langs@glot@tno{Toromono}
\def\langnames@langs@glot@tei{Aro}
\def\langnames@langs@glot@als{Northern Tosk Albanian}
\def\langnames@langs@glot@ttl{Totela}
\def\langnames@langs@glot@txo{Toto}
\def\langnames@langs@glot@txe{Totoli}
\def\langnames@langs@glot@ttk{Totoro}
\def\langnames@langs@glot@zph{Totomachapan Zapotec}
\def\langnames@langs@glot@tqu{Touo}
\def\langnames@langs@glot@neb{Toura (Côte d'Ivoire)}
\def\langnames@langs@glot@don{Toura (Papua New Guinea)}
\def\langnames@langs@glot@ttn{Towei}
\def\langnames@langs@glot@xtg{Transalpine Gaulish}
\def\langnames@langs@glot@trl{Traveller Scottish}
\def\langnames@langs@glot@rmg{Traveller Norwegian}
\def\langnames@langs@glot@rmd{Traveller Danish}
\def\langnames@langs@glot@trm{Tregami}
\def\langnames@langs@glot@tme{Tremembé}
\def\langnames@langs@glot@stg{Trieng}
\def\langnames@langs@glot@tip{Trimuris}
\def\langnames@langs@glot@trx{Tringgus-Sembaan Bidayuh}
\def\langnames@langs@glot@tgq{Tring}
\def\langnames@langs@glot@trn{Trinitario-Javeriano-Loretano}
\def\langnames@langs@glot@trf{Trinidadian Creole English}
\def\langnames@langs@glot@lst{Trinidad and Tobago Sign Language}
\def\langnames@langs@glot@tka{Truká}
\def\langnames@langs@glot@tsa{Tsaangi}
\def\langnames@langs@glot@tsd{Tsakonian}
\def\langnames@langs@glot@kvz{Tsaukambo}
\def\langnames@langs@glot@tsb{Tsamai}
\def\langnames@langs@glot@tsk{Tseku}
\def\langnames@langs@glot@txc{Tsetsaut}
\def\langnames@langs@glot@kdl{Tsikimba}
\def\langnames@langs@glot@xmw{Tsimihety Malagasy}
\def\langnames@langs@glot@tsw{Salka-Tsishingini}
\def\langnames@langs@glot@hio{Northern Tshwa}
\def\langnames@langs@glot@ldp{Tso}
\def\langnames@langs@glot@lto{Tsotso}
\def\langnames@langs@glot@fly{Tsotsitaal}
\def\langnames@langs@glot@ttz{Tsum}
\def\langnames@langs@glot@tsl{Ts'ün-Lao}
\def\langnames@langs@glot@tvd{Tsuvadi}
\def\langnames@langs@glot@tsh{Tsuvan}
\def\langnames@langs@glot@two{Tswapong}
\def\langnames@langs@glot@tsc{Tswa}
\def\langnames@langs@glot@nrt{Tualatin-Yamhill}
\def\langnames@langs@glot@tuy{Tugen}
\def\langnames@langs@glot@tuj{Tugutil}
\def\langnames@langs@glot@khc{Tukang Besi North}
\def\langnames@langs@glot@bhq{Tukang Besi South}
\def\langnames@langs@glot@tkf{Tukumanféd}
\def\langnames@langs@glot@tkd{Tukudede}
\def\langnames@langs@glot@tul{Tula}
\def\langnames@langs@glot@tlu{Tulehu}
\def\langnames@langs@glot@tey{Tulishi}
\def\langnames@langs@glot@rak{Tulu-Bohuai}
\def\langnames@langs@glot@krt{Tumari Kanuri}
\def\langnames@langs@glot@iou{Tuma-Irumu}
\def\langnames@langs@glot@tum{Tumbuka}
\def\langnames@langs@glot@kku{Tumi}
\def\langnames@langs@glot@xtq{Tumshuqese}
\def\langnames@langs@glot@tbr{Tumtum}
\def\langnames@langs@glot@enh{Tundra Enets}
\def\langnames@langs@glot@trt{Tunggare}
\def\langnames@langs@glot@tse{Tunisian Sign Language}
\def\langnames@langs@glot@tug{Tunia}
\def\langnames@langs@glot@tjg{Tunjung}
\def\langnames@langs@glot@tqq{Tunni}
\def\langnames@langs@glot@dza{Tunzu}
\def\langnames@langs@glot@ttf{Tuotomb}
\def\langnames@langs@glot@tpr{Tuparí}
\def\langnames@langs@glot@tpw{Lingua Geral Paulista}
\def\langnames@langs@glot@trh{Turaka}
\def\langnames@langs@glot@trd{Turi}
\def\langnames@langs@glot@twt{Turiwára}
\def\langnames@langs@glot@tuz{Turka}
\def\langnames@langs@glot@tch{Turks And Caicos Creole English}
\def\langnames@langs@glot@tru{Turoyo}
\def\langnames@langs@glot@try{Turung}
\def\langnames@langs@glot@tqm{Turumsa}
\def\langnames@langs@glot@ttg{Tutong}
\def\langnames@langs@glot@tmi{Tutuba}
\def\langnames@langs@glot@mtu{Tututepec Mixtec}
\def\langnames@langs@glot@tww{Tuwari}
\def\langnames@langs@glot@ifk{Tuwali Ifugao}
\def\langnames@langs@glot@bov{Tuwuli}
\def\langnames@langs@glot@tud{Tuxá}
\def\langnames@langs@glot@tux{Tuxináwa}
\def\langnames@langs@glot@xjb{Tweed-Albert}
\def\langnames@langs@glot@twn{Cambap-Langa}
\def\langnames@langs@glot@uam{Uamué}
\def\langnames@langs@glot@ksj{Uare}
\def\langnames@langs@glot@byc{Ubaghara}
\def\langnames@langs@glot@uba{Ubang}
\def\langnames@langs@glot@ubi{Ubi}
\def\langnames@langs@glot@ubr{Ubir}
\def\langnames@langs@glot@cpb{Ucayali-Yurúa Ashéninka}
\def\langnames@langs@glot@uda{Uda}
\def\langnames@langs@glot@udu{Uduk}
\def\langnames@langs@glot@ufi{Ufim}
\def\langnames@langs@glot@uga{Ugaritic}
\def\langnames@langs@glot@uge{Ughele}
\def\langnames@langs@glot@ugo{Ugong}
\def\langnames@langs@glot@uha{Uhami}
\def\langnames@langs@glot@uis{Uisai}
\def\langnames@langs@glot@udj{Ujir}
\def\langnames@langs@glot@kcf{Ukaan}
\def\langnames@langs@glot@ukh{Ukhwejo}
\def\langnames@langs@glot@umi{Ukit}
\def\langnames@langs@glot@ukp{Ukpe-Bayobiri}
\def\langnames@langs@glot@akd{Ukpet-Ehom}
\def\langnames@langs@glot@ukl{Ukrainian Sign Language}
\def\langnames@langs@glot@uku{Ukue}
\def\langnames@langs@glot@ukg{Ukuriguma}
\def\langnames@langs@glot@ukq{Ukwa}
\def\langnames@langs@glot@ukw{Ukwuani-Aboh-Ndoni}
\def\langnames@langs@glot@svb{Ulau-Suain}
\def\langnames@langs@glot@ull{Ullatan}
\def\langnames@langs@glot@ulb{Ulukwumi}
\def\langnames@langs@glot@ulm{Ulumanda'}
\def\langnames@langs@glot@ulw{Ulwa}
\def\langnames@langs@glot@ulu{Uma' Lung}
\def\langnames@langs@glot@xky{Uma' Lasan}
\def\langnames@langs@glot@gdn{Umanakaina}
\def\langnames@langs@glot@umd{Umbindhamu}
\def\langnames@langs@glot@xum{Umbrian}
\def\langnames@langs@glot@umr{Umbugarla}
\def\langnames@langs@glot@umg{Umbuygamu}
\def\langnames@langs@glot@upi{Umeda-Punda}
\def\langnames@langs@glot@sju{Ume Saami}
\def\langnames@langs@glot@due{Umiray Dumaget Agta}
\def\langnames@langs@glot@umm{Umon}
\def\langnames@langs@glot@umo{Umotína}
\def\langnames@langs@glot@unz{Unde Kaili}
\def\langnames@langs@glot@bbn{Uneapa}
\def\langnames@langs@glot@une{Uneme}
\def\langnames@langs@glot@xgu{Unggumi}
\def\langnames@langs@glot@uni{Uni}
\def\langnames@langs@glot@uln{Unserdeutsch}
\def\langnames@langs@glot@onu{Unua}
\def\langnames@langs@glot@unu{Unubahe}
\def\langnames@langs@glot@tov{Upper Taromi}
\def\langnames@langs@glot@tku{Upper Necaxa Totonac}
\def\langnames@langs@glot@sxu{Central East Middle German}
\def\langnames@langs@glot@tth{Upper Ta'oih}
\def\langnames@langs@glot@dmg{Upper Kinabatangan}
\def\langnames@langs@glot@dna{Upper Grand Valley Dani}
\def\langnames@langs@glot@xup{Upper Umpqua}
\def\langnames@langs@glot@tau{Upper Tanana}
\def\langnames@langs@glot@url{Urali of Idukki}
\def\langnames@langs@glot@urm{Urapmin}
\def\langnames@langs@glot@uro{Ura (Papua New Guinea)}
\def\langnames@langs@glot@xur{Urartian}
\def\langnames@langs@glot@urg{Urigina}
\def\langnames@langs@glot@uvh{Uri}
\def\langnames@langs@glot@urx{Urimo}
\def\langnames@langs@glot@urc{Urningangg}
\def\langnames@langs@glot@urv{Uruava}
\def\langnames@langs@glot@urn{Uruangnirin}
\def\langnames@langs@glot@urz{Uru-Eu-Wau-Wau}
\def\langnames@langs@glot@ugy{Uruguayan Sign Language}
\def\langnames@langs@glot@uru{Urumi}
\def\langnames@langs@glot@urp{Uru-Pa-In}
\def\langnames@langs@glot@usk{Usaghade}
\def\langnames@langs@glot@ush{Ushojo}
\def\langnames@langs@glot@ulf{Usku}
\def\langnames@langs@glot@usp{Uspanteco}
\def\langnames@langs@glot@usi{Usui}
\def\langnames@langs@glot@omo{Utarmbung}
\def\langnames@langs@glot@wsg{Adilabad Gondi}
\def\langnames@langs@glot@utu{Utu}
\def\langnames@langs@glot@uuu{U}
\def\langnames@langs@glot@evh{Uvbie}
\def\langnames@langs@glot@usu{Uya}
\def\langnames@langs@glot@auz{Uzbeki Arabic}
\def\langnames@langs@glot@eze{Uzekwe}
\def\langnames@langs@glot@vaa{Vaagri Booli}
\def\langnames@langs@glot@kqu{Vaal-Orange}
\def\langnames@langs@glot@vgr{Vaghri}
\def\langnames@langs@glot@dkg{Kadung}
\def\langnames@langs@glot@tva{Vaghua}
\def\langnames@langs@glot@vap{Vaiphei}
\def\langnames@langs@glot@vae{Vale}
\def\langnames@langs@glot@vsv{Valencian Sign Language}
\def\langnames@langs@glot@vmv{Valley Maidu}
\def\langnames@langs@glot@cvn{Valle Nacional Chinantec}
\def\langnames@langs@glot@vlp{Valpei}
\def\langnames@langs@glot@mkt{Vamale}
\def\langnames@langs@glot@mlr{Vame}
\def\langnames@langs@glot@mpr{Vangunu}
\def\langnames@langs@glot@vnk{Lovono}
\def\langnames@langs@glot@vau{Vanuma}
\def\langnames@langs@glot@vao{Vao}
\def\langnames@langs@glot@vah{Varhadi-Nagpuri}
\def\langnames@langs@glot@vrs{Varisi}
\def\langnames@langs@glot@vav{Dungar Varli}
\def\langnames@langs@glot@vaj{Northern Ju}
\def\langnames@langs@glot@val{Vehes}
\def\langnames@langs@glot@vem{Vemgo-Mabas}
\def\langnames@langs@glot@vsl{Venezuelan Sign Language}
\def\langnames@langs@glot@xve{Venetic}
\def\langnames@langs@glot@vec{Venetian}
\def\langnames@langs@glot@veo{Ventureño}
\def\langnames@langs@glot@vra{Vera'a}
\def\langnames@langs@glot@vid{Vidunda}
\def\langnames@langs@glot@vig{Viemo}
\def\langnames@langs@glot@vil{Vilela}
\def\langnames@langs@glot@dyg{Villa Viciosa Agta}
\def\langnames@langs@glot@svc{Vincentian Creole English}
\def\langnames@langs@glot@vin{Vinza}
\def\langnames@langs@glot@vic{Virgin Islands Creole English}
\def\langnames@langs@glot@vis{Vishavan}
\def\langnames@langs@glot@vit{Viti}
\def\langnames@langs@glot@vto{Vitou}
\def\langnames@langs@glot@vls{Western Flemish}
\def\langnames@langs@glot@vol{Volapük}
\def\langnames@langs@glot@kch{Vono}
\def\langnames@langs@glot@vor{Voro}
\def\langnames@langs@glot@vum{Vumbu}
\def\langnames@langs@glot@vnp{Vunapu}
\def\langnames@langs@glot@vun{Vunjo}
\def\langnames@langs@glot@msn{Vurës}
\def\langnames@langs@glot@vut{Vute}
\def\langnames@langs@glot@wbi{Vwanji}
\def\langnames@langs@glot@wmn{Waamwang}
\def\langnames@langs@glot@wab{Wab}
\def\langnames@langs@glot@wbb{Wabo}
\def\langnames@langs@glot@kmx{Waboda}
\def\langnames@langs@glot@wci{Waci Gbe}
\def\langnames@langs@glot@wdg{Wadaginam}
\def\langnames@langs@glot@wbq{Waddar}
\def\langnames@langs@glot@kxp{Wadiyara Koli}
\def\langnames@langs@glot@wdu{Wadjigu}
\def\langnames@langs@glot@wag{Wa'ema}
\def\langnames@langs@glot@wrx{Kolor}
\def\langnames@langs@glot@waj{Waffa}
\def\langnames@langs@glot@wga{Wagaya}
\def\langnames@langs@glot@wgb{Wagawaga}
\def\langnames@langs@glot@wbr{Wagdi}
\def\langnames@langs@glot@fad{Wagi (Papua New Guinea)}
\def\langnames@langs@glot@whk{Eastern Lowland Kenyah}
\def\langnames@langs@glot@wgo{Waigeo}
\def\langnames@langs@glot@wlr{Ale}
\def\langnames@langs@glot@wlk{Eel River Athabaskan}
\def\langnames@langs@glot@wmh{Waima'a}
\def\langnames@langs@glot@atr{Waimiri-Atroari}
\def\langnames@langs@glot@wli{Waioli}
\def\langnames@langs@glot@wja{Waja}
\def\langnames@langs@glot@wav{Waka}
\def\langnames@langs@glot@wwb{Wakabunga}
\def\langnames@langs@glot@wkd{Wakde}
\def\langnames@langs@glot@waf{Wakoná}
\def\langnames@langs@glot@lgl{Wala}
\def\langnames@langs@glot@wlw{Walak}
\def\langnames@langs@glot@wly{Waling}
\def\langnames@langs@glot@wll{Wali (Sudan)}
\def\langnames@langs@glot@wlx{Wali (Ghana)}
\def\langnames@langs@glot@waa{Northeast Sahaptin}
\def\langnames@langs@glot@wln{Walloon}
\def\langnames@langs@glot@wae{Walser}
\def\langnames@langs@glot@ola{Walungge}
\def\langnames@langs@glot@wmc{Wamas}
\def\langnames@langs@glot@wmi{Wamin}
\def\langnames@langs@glot@lbq{Wampar}
\def\langnames@langs@glot@waz{Wampur}
\def\langnames@langs@glot@qyp{Wampano}
\def\langnames@langs@glot@wnp{Wanap}
\def\langnames@langs@glot@wnb{Mokati}
\def\langnames@langs@glot@nnp{Wancho Naga}
\def\langnames@langs@glot@wbh{Wanda}
\def\langnames@langs@glot@wdd{Wandji}
\def\langnames@langs@glot@wad{Wandamen}
\def\langnames@langs@glot@mfi{Wandala}
\def\langnames@langs@glot@wne{Waneci}
\def\langnames@langs@glot@hwa{Wané}
\def\langnames@langs@glot@wnm{Wanggamala}
\def\langnames@langs@glot@lwg{Wanga}
\def\langnames@langs@glot@wng{Wanggom}
\def\langnames@langs@glot@jub{Wannu}
\def\langnames@langs@glot@wno{Wano}
\def\langnames@langs@glot@wnk{Wanukaka}
\def\langnames@langs@glot@wny{Wanyi}
\def\langnames@langs@glot@juk{Wapan}
\def\langnames@langs@glot@juw{Wãpha}
\def\langnames@langs@glot@wbf{Samue}
\def\langnames@langs@glot@tci{Anta-Komnzo-Wára-Wérè-Kémä}
\def\langnames@langs@glot@srv{Waray Sorsogon}
\def\langnames@langs@glot@bpe{Bauni}
\def\langnames@langs@glot@wre{Ware}
\def\langnames@langs@glot@wai{Wares}
\def\langnames@langs@glot@wri{Wariyangga}
\def\langnames@langs@glot@wbe{Waritai}
\def\langnames@langs@glot@aml{War-Jaintia}
\def\langnames@langs@glot@wji{Warji}
\def\langnames@langs@glot@bgv{Warkay-Bipim}
\def\langnames@langs@glot@wrl{Warlmanpa}
\def\langnames@langs@glot@wrn{Warnang}
\def\langnames@langs@glot@wru{Waru}
\def\langnames@langs@glot@wrv{Waruna}
\def\langnames@langs@glot@wss{Wasa}
\def\langnames@langs@glot@gsp{Wasembo}
\def\langnames@langs@glot@wsu{Wasu}
\def\langnames@langs@glot@wtk{Watakataui}
\def\langnames@langs@glot@wah{Watubela}
\def\langnames@langs@glot@wuy{Wauyai}
\def\langnames@langs@glot@www{Wawa}
\def\langnames@langs@glot@wow{Wawonii}
\def\langnames@langs@glot@wxa{Waxianghua}
\def\langnames@langs@glot@ctt{Wayanad Chetti}
\def\langnames@langs@glot@wyr{Wayoró}
\def\langnames@langs@glot@weh{Weh}
\def\langnames@langs@glot@wew{Wewewa}
\def\langnames@langs@glot@wlh{Welaun}
\def\langnames@langs@glot@klh{Weliki}
\def\langnames@langs@glot@wei{Were}
\def\langnames@langs@glot@gxx{Wè Southern}
\def\langnames@langs@glot@ywl{Western Lalu}
\def\langnames@langs@glot@hmw{Western Mashan Hmong}
\def\langnames@langs@glot@ojw{Western Ojibwa}
\def\langnames@langs@glot@tqt{Ozumatlán Totonac}
\def\langnames@langs@glot@yih{Western Yiddish}
\def\langnames@langs@glot@pnb{Western Panjabi}
\def\langnames@langs@glot@lcp{Western Lawa}
\def\langnames@langs@glot@kuf{Western Katu}
\def\langnames@langs@glot@mut{Western Muria}
\def\langnames@langs@glot@kyu{Western Kayah}
\def\langnames@langs@glot@tdg{Western Tamang}
\def\langnames@langs@glot@wmg{Western Muya}
\def\langnames@langs@glot@raf{Western Meohang}
\def\langnames@langs@glot@mmr{Western Xiangxi Miao}
\def\langnames@langs@glot@lia{West-Central Limba}
\def\langnames@langs@glot@xwl{Western Xwla Gbe}
\def\langnames@langs@glot@bbp{West Central Banda}
\def\langnames@langs@glot@ssl{Western Sisaala}
\def\langnames@langs@glot@krw{Western Krahn}
\def\langnames@langs@glot@nnd{West Ambae}
\def\langnames@langs@glot@uve{West Uvean}
\def\langnames@langs@glot@mss{West Masela}
\def\langnames@langs@glot@lmj{West Lembata}
\def\langnames@langs@glot@drn{West Damar}
\def\langnames@langs@glot@suc{Western Subanon}
\def\langnames@langs@glot@twb{Western Tawbuid}
\def\langnames@langs@glot@pne{Western Penan}
\def\langnames@langs@glot@zbw{West Berawan}
\def\langnames@langs@glot@dnw{Western Dani}
\def\langnames@langs@glot@nhw{Western Huasteca Nahuatl}
\def\langnames@langs@glot@pua{Western Highland Purepecha}
\def\langnames@langs@glot@gnw{Western Bolivian Guaraní}
\def\langnames@langs@glot@jmx{Western Juxtlahuaca Mixtec}
\def\langnames@langs@glot@tnb{Western Tunebo}
\def\langnames@langs@glot@amw{Western Neo-Aramaic}
\def\langnames@langs@glot@azn{Western Durango Nahuatl}
\def\langnames@langs@glot@wwo{Dorig}
\def\langnames@langs@glot@wea{Wewaw}
\def\langnames@langs@glot@wec{Wè Western}
\def\langnames@langs@glot@woy{Weyto}
\def\langnames@langs@glot@lwh{White Lachi}
\def\langnames@langs@glot@giw{Duoluo Gelao}
\def\langnames@langs@glot@tnp{Whitesands}
\def\langnames@langs@glot@tua{Wiarumus}
\def\langnames@langs@glot@mtp{Wichí Lhamtés Nocten}
\def\langnames@langs@glot@wlv{Wichí Lhamtés Vejoz}
\def\langnames@langs@glot@wik{Wikalkan}
\def\langnames@langs@glot@wie{Wik-Epa}
\def\langnames@langs@glot@wij{Wik-Iiyanh}
\def\langnames@langs@glot@wif{Wik-Keyangan}
\def\langnames@langs@glot@wih{Wik-Me'anha}
\def\langnames@langs@glot@wua{Wikngenchera}
\def\langnames@langs@glot@wil{Wilawila}
\def\langnames@langs@glot@wit{Wintu}
\def\langnames@langs@glot@gdr{Wipi}
\def\langnames@langs@glot@wrh{Wiradhuri}
\def\langnames@langs@glot@wir{Wiraféd}
\def\langnames@langs@glot@wiu{Wiru}
\def\langnames@langs@glot@xwc{Woccon}
\def\langnames@langs@glot@woc{Wogeo}
\def\langnames@langs@glot@wbw{Woi}
\def\langnames@langs@glot@wyi{Woiwurrung-Thagungwurrung}
\def\langnames@langs@glot@jod{Wojenaka}
\def\langnames@langs@glot@wod{Wolani}
\def\langnames@langs@glot@wle{Wolane}
\def\langnames@langs@glot@wom{Wom (Nigeria)}
\def\langnames@langs@glot@wmo{Wom (Papua New Guinea)}
\def\langnames@langs@glot@won{Wongo}
\def\langnames@langs@glot@cwd{Woods Cree}
\def\langnames@langs@glot@kda{Worimi}
\def\langnames@langs@glot@wor{Woria}
\def\langnames@langs@glot@jud{Worodougou}
\def\langnames@langs@glot@wsv{Wotapuri-Katarqalai}
\def\langnames@langs@glot@wtw{Wotu}
\def\langnames@langs@glot@wud{Wudu}
\def\langnames@langs@glot@qgu{Wulguru}
\def\langnames@langs@glot@wlu{Wuliwuli}
\def\langnames@langs@glot@wux{Wulna}
\def\langnames@langs@glot@bqm{Wumboko-Bubia}
\def\langnames@langs@glot@wum{Wumbvu}
\def\langnames@langs@glot@ywu{Wumeng Nasu}
\def\langnames@langs@glot@bwn{Wunai Bunu}
\def\langnames@langs@glot@wub{Wunambal}
\def\langnames@langs@glot@wur{Wurrugu}
\def\langnames@langs@glot@yig{Wusa Nasu}
\def\langnames@langs@glot@bse{Wushi}
\def\langnames@langs@glot@wsi{Kula (Vanuatu)}
\def\langnames@langs@glot@wuh{Wutunhua}
\def\langnames@langs@glot@wut{Wutung}
\def\langnames@langs@glot@wuv{Wuvulu-Aua}
\def\langnames@langs@glot@wym{Wymysorys}
\def\langnames@langs@glot@zax{Xadani Zapotec}
\def\langnames@langs@glot@xkr{Xakriabá}
\def\langnames@langs@glot@xan{Xamtanga}
\def\langnames@langs@glot@ztg{Xanaguía Zapotec}
\def\langnames@langs@glot@axx{Xaragure}
\def\langnames@langs@glot@xeg{//Xegwi}
\def\langnames@langs@glot@xet{Xetá}
\def\langnames@langs@glot@hsn{Xiang Chinese}
\def\langnames@langs@glot@sjo{Xibe}
\def\langnames@langs@glot@asn{Xingú Asuriní}
\def\langnames@langs@glot@xiy{Xipaya}
\def\langnames@langs@glot@xip{Xipináwa}
\def\langnames@langs@glot@xii{Xiri}
\def\langnames@langs@glot@xoo{Xukurú}
\def\langnames@langs@glot@xwe{Xwela Gbe}
\def\langnames@langs@glot@tyy{Tiyaa}
\def\langnames@langs@glot@muu{Yaaku}
\def\langnames@langs@glot@yar{Yabarana}
\def\langnames@langs@glot@ybn{Yabaâna-Mainatari}
\def\langnames@langs@glot@ybm{Yaben}
\def\langnames@langs@glot@ybo{Yabong}
\def\langnames@langs@glot@ekr{Yace}
\def\langnames@langs@glot@rys{Yaeyama}
\def\langnames@langs@glot@wfg{Yafi}
\def\langnames@langs@glot@ygm{Yagomi}
\def\langnames@langs@glot@ygw{Yagwoia}
\def\langnames@langs@glot@rhp{Yahang}
\def\langnames@langs@glot@ner{Yahadian}
\def\langnames@langs@glot@ynu{Yahuna}
\def\langnames@langs@glot@iyx{Yaka (Congo)}
\def\langnames@langs@glot@ykk{Yakaikeke}
\def\langnames@langs@glot@ybh{Yakkha}
\def\langnames@langs@glot@xyl{Yalakalore}
\def\langnames@langs@glot@yba{Yala}
\def\langnames@langs@glot@jal{Yalahatan-Haruru-Awaiya}
\def\langnames@langs@glot@zpu{Yalálag Zapotec}
\def\langnames@langs@glot@yal{Yalunka}
\def\langnames@langs@glot@ymp{Yamap}
\def\langnames@langs@glot@yat{Yambeta}
\def\langnames@langs@glot@ymb{Yambes}
\def\langnames@langs@glot@yme{Yameo}
\def\langnames@langs@glot@ymn{Yamna}
\def\langnames@langs@glot@qur{Chaupihuaranga Quechua}
\def\langnames@langs@glot@yda{Yanda}
\def\langnames@langs@glot@dym{Yanda Dom Dogon}
\def\langnames@langs@glot@xyb{Yandjibara}
\def\langnames@langs@glot@zyg{Yang Zhuang}
\def\langnames@langs@glot@jng{Yangman}
\def\langnames@langs@glot@yng{Yango}
\def\langnames@langs@glot@bsx{Yangkam}
\def\langnames@langs@glot@yav{Yangben}
\def\langnames@langs@glot@ygl{Yangum Gel}
\def\langnames@langs@glot@ymo{Yangum Mon}
\def\langnames@langs@glot@yde{Yangum Dey}
\def\langnames@langs@glot@ynl{Yangulam}
\def\langnames@langs@glot@tjj{Yangathimri}
\def\langnames@langs@glot@ysm{Yangon Myanmar Sign Language}
\def\langnames@langs@glot@jay{Nhangu}
\def\langnames@langs@glot@guu{Yanomamö}
\def\langnames@langs@glot@asy{Yaosakor Asmat}
\def\langnames@langs@glot@yre{Yaouré}
\def\langnames@langs@glot@yev{Yeri}
\def\langnames@langs@glot@yrw{Yarawata}
\def\langnames@langs@glot@zae{Yareni Zapotec}
\def\langnames@langs@glot@yro{Yaroame}
\def\langnames@langs@glot@yko{Yasa}
\def\langnames@langs@glot@zty{Yatee Zapotec}
\def\langnames@langs@glot@yla{Ulwa (Papua New Guinea)}
\def\langnames@langs@glot@yuw{Yau-Nungon}
\def\langnames@langs@glot@jau{Yaur}
\def\langnames@langs@glot@yyu{Yau (Sandaun Province)}
\def\langnames@langs@glot@zpb{Yautepec Zapotec}
\def\langnames@langs@glot@qux{Yauyos Quechua}
\def\langnames@langs@glot@yvt{Yavitero-Pareni}
\def\langnames@langs@glot@yww{Yawarawarga}
\def\langnames@langs@glot@ywn{Yawanawa}
\def\langnames@langs@glot@yaw{Yawalapití}
\def\langnames@langs@glot@yby{Yaweyuha}
\def\langnames@langs@glot@ybx{Yawiyo}
\def\langnames@langs@glot@ykr{Yekora}
\def\langnames@langs@glot@yel{Yela-Kela}
\def\langnames@langs@glot@ylg{Yalaku}
\def\langnames@langs@glot@ynq{Yendang}
\def\langnames@langs@glot@yec{Yeniche}
\def\langnames@langs@glot@yei{Yeni}
\def\langnames@langs@glot@yra{Yerakai}
\def\langnames@langs@glot@gop{Yeretuar}
\def\langnames@langs@glot@yrn{Yerong-Southern Buyang}
\def\langnames@langs@glot@yeu{Yerukula}
\def\langnames@langs@glot@yes{Yeskwa}
\def\langnames@langs@glot@yet{Yetfa}
\def\langnames@langs@glot@yej{Yevanic}
\def\langnames@langs@glot@ydg{Yidgha}
\def\langnames@langs@glot@yim{Yimchungru Naga}
\def\langnames@langs@glot@kvu{Yinbaw Karen}
\def\langnames@langs@glot@yin{Yinchia}
\def\langnames@langs@glot@yil{Yindjilandji}
\def\langnames@langs@glot@ywg{Yinhawangka}
\def\langnames@langs@glot@kvy{Yintale Karen}
\def\langnames@langs@glot@yxm{Yinwum}
\def\langnames@langs@glot@ljw{Yirandhali}
\def\langnames@langs@glot@yiy{Yir-Yoront}
\def\langnames@langs@glot@yis{Yis}
\def\langnames@langs@glot@gek{Yiwom}
\def\langnames@langs@glot@yob{Yoba}
\def\langnames@langs@glot@gud{Yocoboué Dida}
\def\langnames@langs@glot@yog{Yogad}
\def\langnames@langs@glot@ydk{Yoidik}
\def\langnames@langs@glot@yki{Yoke}
\def\langnames@langs@glot@ygs{Yolngu Sign Language}
\def\langnames@langs@glot@xty{Yoloxochitl Mixtec}
\def\langnames@langs@glot@pil{Yom}
\def\langnames@langs@glot@yoi{Yonaguni}
\def\langnames@langs@glot@sxk{Yoncalla}
\def\langnames@langs@glot@nru{Narua}
\def\langnames@langs@glot@zyn{Yongnan Zhuang}
\def\langnames@langs@glot@zyb{Yongbei Zhuang}
\def\langnames@langs@glot@yno{Yong}
\def\langnames@langs@glot@yon{Yonggom}
\def\langnames@langs@glot@yut{Yopno}
\def\langnames@langs@glot@mts{Yora}
\def\langnames@langs@glot@yox{Yoron}
\def\langnames@langs@glot@yot{Yotti}
\def\langnames@langs@glot@zyj{Youjiang Zhuang}
\def\langnames@langs@glot@ytw{Yout Wam}
\def\langnames@langs@glot@yoy{Yoy}
\def\langnames@langs@glot@nua{Yuaga}
\def\langnames@langs@glot@msd{Yucatec Maya Sign Language}
\def\langnames@langs@glot@mvg{Yucuañe Mixtec}
\def\langnames@langs@glot@yub{Yugambal}
\def\langnames@langs@glot@ysl{Yugoslavian Sign Language}
\def\langnames@langs@glot@ygu{Yugul}
\def\langnames@langs@glot@yab{Yuhup}
\def\langnames@langs@glot@omk{Malyj Anjuj Omok}
\def\langnames@langs@glot@ybl{Yukuben}
\def\langnames@langs@glot@yuq{Yuqui}
\def\langnames@langs@glot@ljx{Yuru}
\def\langnames@langs@glot@mab{Yutanduchi Mixtec}
\def\langnames@langs@glot@yau{Hoti}
\def\langnames@langs@glot@ztx{Zaachila Zapotec}
\def\langnames@langs@glot@kji{Zabana}
\def\langnames@langs@glot@nhi{Zacatlán-Ahuacatlán-Tepetzintla Nahuatl}
\def\langnames@langs@glot@ctz{Zacatepec Chatino}
\def\langnames@langs@glot@atb{Zaiwa}
\def\langnames@langs@glot@zkr{Zakhring}
\def\langnames@langs@glot@zsl{Zambian Sign Language}
\def\langnames@langs@glot@zak{Zanaki}
\def\langnames@langs@glot@zau{Zangskari}
\def\langnames@langs@glot@zna{Zan Gula}
\def\langnames@langs@glot@zah{Zangwal}
\def\langnames@langs@glot@zpw{Zaniza Zapotec}
\def\langnames@langs@glot@zaj{Zaramo}
\def\langnames@langs@glot@zbu{Bu (Zaranda)}
\def\langnames@langs@glot@zaz{Zari}
\def\langnames@langs@glot@zal{Zauzou}
\def\langnames@langs@glot@kxk{Lahta-Zayein Karen}
\def\langnames@langs@glot@zwa{Zay}
\def\langnames@langs@glot@jaj{Zazao}
\def\langnames@langs@glot@zua{Zeem}
\def\langnames@langs@glot@dhm{Zemba}
\def\langnames@langs@glot@zeg{Zenag}
\def\langnames@langs@glot@czn{Zenzontepec Chatino}
\def\langnames@langs@glot@zhb{Zhaba}
\def\langnames@langs@glot@xzh{Zhangzhung}
\def\langnames@langs@glot@zhi{Zhire}
\def\langnames@langs@glot@zhw{Zhoa}
\def\langnames@langs@glot@zia{Zia}
\def\langnames@langs@glot@zil{Zialo}
\def\langnames@langs@glot@ziw{Zigula-Mushungulu}
\def\langnames@langs@glot@zib{Zimbabwe Sign Language}
\def\langnames@langs@glot@zmb{Zimba}
\def\langnames@langs@glot@zin{Zinza}
\def\langnames@langs@glot@sih{Zire}
\def\langnames@langs@glot@zrn{Zirenkel}
\def\langnames@langs@glot@ziz{Zizilivakan}
\def\langnames@langs@glot@pto{Zo'é}
\def\langnames@langs@glot@yzk{Zokhuo}
\def\langnames@langs@glot@gbz{Zoroastrian Yazdi}
\def\langnames@langs@glot@czt{Zotung Chin}
\def\langnames@langs@glot@zom{Zou}
\def\langnames@langs@glot@zla{Zula}
\def\langnames@langs@glot@gnd{Zulgo-Gemzek}
\def\langnames@langs@glot@zuy{Zumaya}
\def\langnames@langs@glot@jmb{Zumbun}
\def\langnames@langs@glot@zzj{Zuojiang Zhuang}
\def\langnames@langs@glot@zyp{Zyphe}
%
  \define@key{fams}{knw}{Kxa}
\define@key{fams}{nmn}{Tu}
\define@key{fams}{alu}{Austronesian}
\define@key{fams}{hnh}{Khoe-Kwadi}
\define@key{fams}{xam}{Tu}
\define@key{fams}{huc}{Kxa}
\define@key{fams}{apq}{Great Andamanese}
\define@key{fams}{aiw}{Afro-Asiatic}
\define@key{fams}{aau}{Sepik}
\define@key{fams}{abq}{Northwest Caucasian}
\define@key{fams}{abe}{Algic}
\define@key{fams}{abi}{Niger-Congo}
\define@key{fams}{axb}{Guaicuruan}
\define@key{fams}{abk}{Northwest Caucasian}
\define@key{fams}{abz}{Greater West Bomberai}
\define@key{fams}{kgr}{Isolate}
\define@key{fams}{ace}{Austronesian}
\define@key{fams}{aca}{Arawakan}
\define@key{fams}{acn}{Sino-Tibetan}
\define@key{fams}{ach}{Eastern Sudanic}
\define@key{fams}{acu}{Jivaroan}
\define@key{fams}{acv}{Hokan}
\define@key{fams}{guq}{Tupian}
\define@key{fams}{acr}{Mayan}
\define@key{fams}{kjq}{Keresan}
\define@key{fams}{ads}{other}
\define@key{fams}{adn}{Greater West Bomberai}
\define@key{fams}{adj}{Niger-Congo}
\define@key{fams}{ady}{Northwest Caucasian}
\define@key{fams}{adt}{Pama-Nyungan}
\define@key{fams}{adz}{Austronesian}
\define@key{fams}{awi}{Kamula-Elevala}
\define@key{fams}{afr}{Indo-European}
\define@key{fams}{agd}{Trans-New Guinea}
\define@key{fams}{agq}{Niger-Congo}
\define@key{fams}{ahh}{Trans-New Guinea}
\define@key{fams}{agx}{Nakh-Daghestanian}
\define@key{fams}{agt}{Austronesian}
\define@key{fams}{duo}{Austronesian}
\define@key{fams}{agu}{Mayan}
\define@key{fams}{agr}{Jivaroan}
\define@key{fams}{aht}{Na-Dene}
\define@key{fams}{tba}{Isolate}
\define@key{fams}{ain}{Isolate}
\define@key{fams}{ahp}{Niger-Congo}
\define@key{fams}{aja}{Central Sudanic}
\define@key{fams}{ajg}{Niger-Congo}
\define@key{fams}{aji}{Austronesian}
\define@key{fams}{axk}{Niger-Congo}
\define@key{fams}{abj}{Great Andamanese}
\define@key{fams}{aci}{Great Andamanese}
\define@key{fams}{akx}{Great Andamanese}
\define@key{fams}{aka}{Niger-Congo}
\define@key{fams}{ake}{Cariban}
\define@key{fams}{ahk}{Sino-Tibetan}
\define@key{fams}{akv}{Nakh-Daghestanian}
\define@key{fams}{akl}{Austronesian}
\define@key{fams}{akw}{Niger-Congo}
\define@key{fams}{nrz}{Austronesian}
\define@key{fams}{akz}{Muskogean}
\define@key{fams}{wbj}{Afro-Asiatic}
\define@key{fams}{amp}{Sepik}
\define@key{fams}{btz}{Austronesian}
\define@key{fams}{alh}{Mangarrayi-Maran}
\define@key{fams}{sqi}{Indo-European}
\define@key{fams}{ale}{Eskimo-Aleut}
\define@key{fams}{alq}{Algic}
\define@key{fams}{ald}{Niger-Congo}
\define@key{fams}{gsw}{Indo-European}
\define@key{fams}{aes}{Oregon Coast}
\define@key{fams}{alt}{Altaic}
\define@key{fams}{alp}{Austronesian}
\define@key{fams}{ems}{Eskimo-Aleut}
\define@key{fams}{alr}{Chukotko-Kamchatkan}
\define@key{fams}{aly}{Pama-Nyungan}
\define@key{fams}{amm}{Left May}
\define@key{fams}{amc}{Pano-Tacanan}
\define@key{fams}{amn}{Border}
\define@key{fams}{aie}{Austronesian}
\define@key{fams}{amr}{Harakmbet}
\define@key{fams}{omb}{Austronesian}
\define@key{fams}{amk}{Austronesian}
\define@key{fams}{abt}{Sepik}
\define@key{fams}{adx}{Sino-Tibetan}
\define@key{fams}{aey}{Trans-New Guinea}
\define@key{fams}{ase}{other}
\define@key{fams}{amh}{Afro-Asiatic}
\define@key{fams}{ami}{Austronesian}
\define@key{fams}{amo}{Niger-Congo}
\define@key{fams}{apz}{Trans-New Guinea}
\define@key{fams}{ame}{Arawakan}
\define@key{fams}{amu}{Oto-Manguean}
\define@key{fams}{imi}{Trans-New Guinea}
\define@key{fams}{ani}{Nakh-Daghestanian}
\define@key{fams}{ano}{Isolate}
\define@key{fams}{aty}{Austronesian}
\define@key{fams}{agm}{Trans-New Guinea}
\define@key{fams}{njm}{Sino-Tibetan}
\define@key{fams}{anc}{Afro-Asiatic}
\define@key{fams}{agg}{Senagi}
\define@key{fams}{aoa}{other}
\define@key{fams}{awg}{Pama-Nyungan}
\define@key{fams}{aoi}{Gunwinyguan}
\define@key{fams}{nun}{Sino-Tibetan}
\define@key{fams}{cko}{Niger-Congo}
\define@key{fams}{any}{Niger-Congo}
\define@key{fams}{anu}{Eastern Sudanic}
\define@key{fams}{anz}{Isolate}
\define@key{fams}{njo}{Sino-Tibetan}
\define@key{fams}{apm}{Na-Dene}
\define@key{fams}{apj}{Na-Dene}
\define@key{fams}{apw}{Na-Dene}
\define@key{fams}{apy}{Cariban}
\define@key{fams}{apt}{Sino-Tibetan}
\define@key{fams}{apn}{Macro-Ge}
\define@key{fams}{apu}{Arawakan}
\define@key{fams}{ard}{Pama-Nyungan}
\define@key{fams}{arl}{Zaparoan}
\define@key{fams}{abv}{Afro-Asiatic}
\define@key{fams}{mey}{Afro-Asiatic}
\define@key{fams}{shu}{Afro-Asiatic}
\define@key{fams}{ayl}{Afro-Asiatic}
\define@key{fams}{arz}{Afro-Asiatic}
\define@key{fams}{afb}{Afro-Asiatic}
\define@key{fams}{acw}{Afro-Asiatic}
\define@key{fams}{acm}{Afro-Asiatic}
\define@key{fams}{acy}{Afro-Asiatic}
\define@key{fams}{arb}{Afro-Asiatic}
\define@key{fams}{ary}{Afro-Asiatic}
\define@key{fams}{ajp}{Afro-Asiatic}
\define@key{fams}{ayn}{Afro-Asiatic}
\define@key{fams}{apc}{Afro-Asiatic}
\define@key{fams}{aeb}{Afro-Asiatic}
\define@key{fams}{rmz}{Sino-Tibetan}
\define@key{fams}{akr}{Austronesian}
\define@key{fams}{atq}{Austronesian}
\define@key{fams}{jbj}{South Bird's Head}
\define@key{fams}{aro}{Pano-Tacanan}
\define@key{fams}{arp}{Algic}
\define@key{fams}{aah}{Torricelli}
\define@key{fams}{ape}{Torricelli}
\define@key{fams}{arv}{Afro-Asiatic}
\define@key{fams}{aqc}{Nakh-Daghestanian}
\define@key{fams}{laz}{Austronesian}
\define@key{fams}{ari}{Caddoan}
\define@key{fams}{hye}{Indo-European}
\define@key{fams}{hyw}{Indo-European}
\define@key{fams}{apr}{Austronesian}
\define@key{fams}{aia}{Austronesian}
\define@key{fams}{aer}{Pama-Nyungan}
\define@key{fams}{are}{Pama-Nyungan}
\define@key{fams}{cns}{Asmat-Kamrau Bay}
\define@key{fams}{asm}{Indo-European}
\define@key{fams}{ast}{Indo-European}
\define@key{fams}{asu}{Tupian}
\define@key{fams}{kuz}{Kunza}
\define@key{fams}{aqp}{Isolate}
\define@key{fams}{tay}{Austronesian}
\define@key{fams}{upv}{Austronesian}
\define@key{fams}{aph}{Sino-Tibetan}
\define@key{fams}{atj}{Algic}
\define@key{fams}{atw}{Hokan}
\define@key{fams}{avt}{Torricelli}
\define@key{fams}{aul}{Austronesian}
\define@key{fams}{asf}{other}
\define@key{fams}{auy}{Trans-New Guinea}
\define@key{fams}{ava}{Nakh-Daghestanian}
\define@key{fams}{avn}{Niger-Congo}
\define@key{fams}{avi}{Niger-Congo}
\define@key{fams}{avu}{Central Sudanic}
\define@key{fams}{awb}{Trans-New Guinea}
\define@key{fams}{kwi}{Barbacoan}
\define@key{fams}{awa}{Indo-European}
\define@key{fams}{awn}{Afro-Asiatic}
\define@key{fams}{kmn}{Sepik}
\define@key{fams}{auw}{Border}
\define@key{fams}{nfl}{Austronesian}
\define@key{fams}{ayr}{Aymaran}
\define@key{fams}{aib}{Altaic}
\define@key{fams}{ayo}{Zamucoan}
\define@key{fams}{azb}{Altaic}
\define@key{fams}{koe}{Eastern Sudanic}
\define@key{fams}{bvx}{Niger-Congo}
\define@key{fams}{bav}{Niger-Congo}
\define@key{fams}{wdj}{Wandjiginy}
\define@key{fams}{bfq}{Dravidian}
\define@key{fams}{bde}{Afro-Asiatic}
\define@key{fams}{bia}{Pama-Nyungan}
\define@key{fams}{ksf}{Niger-Congo}
\define@key{fams}{bfd}{Niger-Congo}
\define@key{fams}{bsp}{Niger-Congo}
\define@key{fams}{bmi}{Central Sudanic}
\define@key{fams}{fuu}{Central Sudanic}
\define@key{fams}{bgq}{Indo-European}
\define@key{fams}{kva}{Nakh-Daghestanian}
\define@key{fams}{bdw}{Greater West Bomberai}
\define@key{fams}{bjh}{Sepik}
\define@key{fams}{bdq}{Austro-Asiatic}
\define@key{fams}{bca}{Sino-Tibetan}
\define@key{fams}{bdl}{Austronesian}
\define@key{fams}{bdr}{Austronesian}
\define@key{fams}{bkc}{Niger-Congo}
\define@key{fams}{bdh}{Central Sudanic}
\define@key{fams}{bkq}{Cariban}
\define@key{fams}{bri}{Niger-Congo}
\define@key{fams}{blw}{Austronesian}
\define@key{fams}{blz}{Austronesian}
\define@key{fams}{ban}{Austronesian}
\define@key{fams}{bft}{Sino-Tibetan}
\define@key{fams}{bgn}{Indo-European}
\define@key{fams}{ptu}{Austronesian}
\define@key{fams}{bam}{Mande}
\define@key{fams}{bax}{Niger-Congo}
\define@key{fams}{bcw}{Afro-Asiatic}
\define@key{fams}{jaa}{Arauan}
\define@key{fams}{bza}{Mande}
\define@key{fams}{bdy}{Pama-Nyungan}
\define@key{fams}{bgz}{Austronesian}
\define@key{fams}{bjb}{Pama-Nyungan}
\define@key{fams}{bdg}{Austronesian}
\define@key{fams}{dba}{Isolate}
\define@key{fams}{bvv}{Arawakan}
\define@key{fams}{bwi}{Arawakan}
\define@key{fams}{abb}{Niger-Congo}
\define@key{fams}{bcm}{Austronesian}
\define@key{fams}{bnq}{Austronesian}
\define@key{fams}{peh}{Altaic}
\define@key{fams}{bci}{Niger-Congo}
\define@key{fams}{loy}{Sino-Tibetan}
\define@key{fams}{bbb}{Trans-New Guinea}
\define@key{fams}{brm}{Niger-Congo}
\define@key{fams}{bsn}{Tucanoan}
\define@key{fams}{bcj}{Nyulnyulan}
\define@key{fams}{mlp}{Trans-New Guinea}
\define@key{fams}{bfa}{Eastern Sudanic}
\define@key{fams}{bba}{Niger-Congo}
\define@key{fams}{wra}{Skou}
\define@key{fams}{byr}{Trans-New Guinea}
\define@key{fams}{bae}{Arawakan}
\define@key{fams}{mot}{Chibchan}
\define@key{fams}{bsc}{Niger-Congo}
\define@key{fams}{bas}{Niger-Congo}
\define@key{fams}{bak}{Altaic}
\define@key{fams}{eus}{Isolate}
\define@key{fams}{bya}{Austronesian}
\define@key{fams}{btx}{Austronesian}
\define@key{fams}{bbc}{Austronesian}
\define@key{fams}{bhm}{Afro-Asiatic}
\define@key{fams}{bbd}{Trans-New Guinea}
\define@key{fams}{brg}{Arawakan}
\define@key{fams}{bvz}{Geelvink Bay}
\define@key{fams}{bgr}{Sino-Tibetan}
\define@key{fams}{bsw}{Afro-Asiatic}
\define@key{fams}{bxj}{Pama-Nyungan}
\define@key{fams}{beq}{Niger-Congo}
\define@key{fams}{dbj}{Austronesian}
\define@key{fams}{bej}{Afro-Asiatic}
\define@key{fams}{byw}{Sino-Tibetan}
\define@key{fams}{blc}{Salishan}
\define@key{fams}{bel}{Indo-European}
\define@key{fams}{bem}{Niger-Congo}
\define@key{fams}{bef}{Trans-New Guinea}
\define@key{fams}{nhb}{Mande}
\define@key{fams}{bng}{Niger-Congo}
\define@key{fams}{ben}{Indo-European}
\define@key{fams}{ctg}{Indo-European}
\define@key{fams}{bue}{Isolate}
\define@key{fams}{brf}{Niger-Congo}
\define@key{fams}{shy}{Afro-Asiatic}
\define@key{fams}{grr}{Afro-Asiatic}
\define@key{fams}{tzm}{Afro-Asiatic}
\define@key{fams}{mzb}{Afro-Asiatic}
\define@key{fams}{rif}{Afro-Asiatic}
\define@key{fams}{siz}{Afro-Asiatic}
\define@key{fams}{oua}{Afro-Asiatic}
\define@key{fams}{brc}{other}
\define@key{fams}{zag}{Saharan}
\define@key{fams}{bkl}{Tor-Kwerba}
\define@key{fams}{wti}{Isolate}
\define@key{fams}{xub}{Dravidian}
\define@key{fams}{kap}{Nakh-Daghestanian}
\define@key{fams}{bhb}{Indo-European}
\define@key{fams}{bho}{Indo-European}
\define@key{fams}{unr}{Austro-Asiatic}
\define@key{fams}{bif}{Niger-Congo}
\define@key{fams}{bhw}{Austronesian}
\define@key{fams}{bth}{Austronesian}
\define@key{fams}{bid}{Afro-Asiatic}
\define@key{fams}{bcl}{Austronesian}
\define@key{fams}{bip}{Niger-Congo}
\define@key{fams}{bpr}{Austronesian}
\define@key{fams}{byn}{Afro-Asiatic}
\define@key{fams}{nbj}{Pama-Nyungan}
\define@key{fams}{bll}{Siouan}
\define@key{fams}{blb}{Solomons East Papuan}
\define@key{fams}{bhp}{Austronesian}
\define@key{fams}{bim}{Niger-Congo}
\define@key{fams}{bhg}{Trans-New Guinea}
\define@key{fams}{bin}{Niger-Congo}
\define@key{fams}{gup}{Gunwinyguan}
\define@key{fams}{bkd}{Austronesian}
\define@key{fams}{bjr}{Trans-New Guinea}
\define@key{fams}{bzr}{Pama-Nyungan}
\define@key{fams}{bom}{Niger-Congo}
\define@key{fams}{bvq}{Central Sudanic}
\define@key{fams}{bib}{Mande}
\define@key{fams}{bis}{other}
\define@key{fams}{bla}{Algic}
\define@key{fams}{kvg}{Trans-New Guinea}
\define@key{fams}{bni}{Niger-Congo}
\define@key{fams}{bbo}{Mande}
\define@key{fams}{brx}{Sino-Tibetan}
\define@key{fams}{bzf}{Sepik}
\define@key{fams}{bqc}{Mande}
\define@key{fams}{bol}{Afro-Asiatic}
\define@key{fams}{bli}{Niger-Congo}
\define@key{fams}{bot}{Central Sudanic}
\define@key{fams}{bpu}{Trans-New Guinea}
\define@key{fams}{lbk}{Austronesian}
\define@key{fams}{boa}{Boran}
\define@key{fams}{adi}{Sino-Tibetan}
\define@key{fams}{bor}{Bororoan}
\define@key{fams}{brn}{Chibchan}
\define@key{fams}{bos}{Indo-European}
\define@key{fams}{boz}{Mande}
\define@key{fams}{brh}{Dravidian}
\define@key{fams}{brb}{Austro-Asiatic}
\define@key{fams}{bre}{Indo-European}
\define@key{fams}{bzd}{Chibchan}
\define@key{fams}{bfi}{other}
\define@key{fams}{tcs}{other}
\define@key{fams}{bkk}{Indo-European}
\define@key{fams}{bru}{Austro-Asiatic}
\define@key{fams}{brv}{Austro-Asiatic}
\define@key{fams}{bvb}{Niger-Congo}
\define@key{fams}{buu}{Niger-Congo}
\define@key{fams}{bdk}{Nakh-Daghestanian}
\define@key{fams}{bdm}{Afro-Asiatic}
\define@key{fams}{bug}{Austronesian}
\define@key{fams}{sab}{Chibchan}
\define@key{fams}{bgg}{Sino-Tibetan}
\define@key{fams}{buo}{South Bougainville}
\define@key{fams}{nmg}{Niger-Congo}
\define@key{fams}{bxk}{Niger-Congo}
\define@key{fams}{bul}{Indo-European}
\define@key{fams}{bwu}{Niger-Congo}
\define@key{fams}{bzq}{Austronesian}
\define@key{fams}{bum}{Niger-Congo}
\define@key{fams}{tkw}{Austronesian}
\define@key{fams}{bfu}{Sino-Tibetan}
\define@key{fams}{buh}{Hmong-Mien}
\define@key{fams}{bck}{Bunuban}
\define@key{fams}{bwr}{Afro-Asiatic}
\define@key{fams}{bvr}{Mangrida}
\define@key{fams}{bxm}{Altaic}
\define@key{fams}{bji}{Afro-Asiatic}
\define@key{fams}{mya}{Sino-Tibetan}
\define@key{fams}{mhs}{Austronesian}
\define@key{fams}{bmu}{Trans-New Guinea}
\define@key{fams}{bds}{Afro-Asiatic}
\define@key{fams}{bsk}{Isolate}
\define@key{fams}{bqp}{Mande}
\define@key{fams}{buf}{Niger-Congo}
\define@key{fams}{ngc}{Niger-Congo}
\define@key{fams}{bee}{Sino-Tibetan}
\define@key{fams}{bev}{Niger-Congo}
\define@key{fams}{cjp}{Chibchan}
\define@key{fams}{cbv}{Cacua-Nukak}
\define@key{fams}{cad}{Caddoan}
\define@key{fams}{chl}{Uto-Aztecan}
\define@key{fams}{cak}{Mayan}
\define@key{fams}{rab}{Sino-Tibetan}
\define@key{fams}{cjo}{Arawakan}
\define@key{fams}{kbh}{Isolate}
\define@key{fams}{knm}{Katukinan}
\define@key{fams}{cbu}{Isolate}
\define@key{fams}{ram}{Macro-Ge}
\define@key{fams}{yue}{Sino-Tibetan}
\define@key{fams}{kaq}{Pano-Tacanan}
\define@key{fams}{cbc}{Tucanoan}
\define@key{fams}{car}{Cariban}
\define@key{fams}{mch}{Cariban}
\define@key{fams}{cal}{Austronesian}
\define@key{fams}{crx}{Na-Dene}
\define@key{fams}{cbr}{Pano-Tacanan}
\define@key{fams}{cbs}{Pano-Tacanan}
\define@key{fams}{cat}{Indo-European}
\define@key{fams}{chc}{Siouan}
\define@key{fams}{cto}{Choco}
\define@key{fams}{cav}{Pano-Tacanan}
\define@key{fams}{cbi}{Barbacoan}
\define@key{fams}{cay}{Iroquoian}
\define@key{fams}{cyb}{Isolate}
\define@key{fams}{ceb}{Austronesian}
\define@key{fams}{old}{Niger-Congo}
\define@key{fams}{suq}{Eastern Sudanic}
\define@key{fams}{cld}{Afro-Asiatic}
\define@key{fams}{cjm}{Austronesian}
\define@key{fams}{cja}{Austronesian}
\define@key{fams}{cji}{Nakh-Daghestanian}
\define@key{fams}{can}{Lower Sepik-Ramu}
\define@key{fams}{cha}{Austronesian}
\define@key{fams}{nbc}{Sino-Tibetan}
\define@key{fams}{chx}{Sino-Tibetan}
\define@key{fams}{tuu}{Na-Dene}
\define@key{fams}{cya}{Oto-Manguean}
\define@key{fams}{cta}{Oto-Manguean}
\define@key{fams}{ctp}{Oto-Manguean}
\define@key{fams}{cdn}{Sino-Tibetan}
\define@key{fams}{cbk}{other}
\define@key{fams}{cbt}{Cahuapanan}
\define@key{fams}{che}{Nakh-Daghestanian}
\define@key{fams}{cjh}{Salishan}
\define@key{fams}{mrn}{Austronesian}
\define@key{fams}{xch}{Chimakuan}
\define@key{fams}{cdm}{Sino-Tibetan}
\define@key{fams}{chr}{Iroquoian}
\define@key{fams}{chy}{Algic}
\define@key{fams}{nya}{Niger-Congo}
\define@key{fams}{pei}{Oto-Manguean}
\define@key{fams}{cic}{Muskogean}
\define@key{fams}{cob}{Mayan}
\define@key{fams}{cid}{Hokan}
\define@key{fams}{cbg}{Chibchan}
\define@key{fams}{mrh}{Sino-Tibetan}
\define@key{fams}{csy}{Sino-Tibetan}
\define@key{fams}{ctd}{Sino-Tibetan}
\define@key{fams}{cco}{Oto-Manguean}
\define@key{fams}{cle}{Oto-Manguean}
\define@key{fams}{cpa}{Oto-Manguean}
\define@key{fams}{chq}{Oto-Manguean}
\define@key{fams}{cuc}{Oto-Manguean}
\define@key{fams}{cso}{Oto-Manguean}
\define@key{fams}{cnt}{Oto-Manguean}
\define@key{fams}{csl}{other}
\define@key{fams}{chh}{Penutian}
\define@key{fams}{wac}{Penutian}
\define@key{fams}{cap}{Uru-Chipaya}
\define@key{fams}{chp}{Na-Dene}
\define@key{fams}{cax}{Isolate}
\define@key{fams}{gui}{Tupian}
\define@key{fams}{ctm}{Isolate}
\define@key{fams}{coz}{Oto-Manguean}
\define@key{fams}{cho}{Muskogean}
\define@key{fams}{ctu}{Mayan}
\define@key{fams}{cht}{Hobitu-Cholon}
\define@key{fams}{chd}{Hokan}
\define@key{fams}{clo}{Hokan}
\define@key{fams}{chf}{Mayan}
\define@key{fams}{caa}{Mayan}
\define@key{fams}{crw}{Austro-Asiatic}
\define@key{fams}{cje}{Austronesian}
\define@key{fams}{cjv}{Trans-New Guinea}
\define@key{fams}{cac}{Mayan}
\define@key{fams}{ckt}{Chukotko-Kamchatkan}
\define@key{fams}{clw}{Altaic}
\define@key{fams}{boi}{Chumash}
\define@key{fams}{inz}{Chumash}
\define@key{fams}{ncu}{Niger-Congo}
\define@key{fams}{chk}{Austronesian}
\define@key{fams}{chv}{Altaic}
\define@key{fams}{cao}{Pano-Tacanan}
\define@key{fams}{lua}{Niger-Congo}
\define@key{fams}{clm}{Salishan}
\define@key{fams}{xcw}{Coahuiltecan}
\define@key{fams}{cod}{Tupian}
\define@key{fams}{coc}{Hokan}
\define@key{fams}{crd}{Salishan}
\define@key{fams}{con}{Isolate}
\define@key{fams}{kog}{Chibchan}
\define@key{fams}{col}{Salishan}
\define@key{fams}{com}{Uto-Aztecan}
\define@key{fams}{xcm}{Hokan}
\define@key{fams}{swb}{Niger-Congo}
\define@key{fams}{coo}{Salishan}
\define@key{fams}{csz}{Oregon Coast}
\define@key{fams}{cop}{Afro-Asiatic}
\define@key{fams}{crn}{Uto-Aztecan}
\define@key{fams}{cor}{Indo-European}
\define@key{fams}{crk}{Algic}
\define@key{fams}{csw}{Algic}
\define@key{fams}{mus}{Muskogean}
\define@key{fams}{crh}{Altaic}
\define@key{fams}{cro}{Siouan}
\define@key{fams}{cua}{Austro-Asiatic}
\define@key{fams}{cub}{Tucanoan}
\define@key{fams}{cui}{Guahiban}
\define@key{fams}{cuy}{Isolate}
\define@key{fams}{cul}{Arauan}
\define@key{fams}{cup}{Uto-Aztecan}
\define@key{fams}{kpc}{Arawakan}
\define@key{fams}{ces}{Indo-European}
\define@key{fams}{cam}{Austronesian}
\define@key{fams}{kzf}{Austronesian}
\define@key{fams}{dbq}{Afro-Asiatic}
\define@key{fams}{dav}{Niger-Congo}
\define@key{fams}{mps}{Teberan-Pawaian}
\define@key{fams}{dgz}{Trans-New Guinea}
\define@key{fams}{dga}{Niger-Congo}
\define@key{fams}{dag}{Niger-Congo}
\define@key{fams}{dta}{Altaic}
\define@key{fams}{dal}{Afro-Asiatic}
\define@key{fams}{daj}{Eastern Sudanic}
\define@key{fams}{dak}{Siouan}
\define@key{fams}{mbp}{Chibchan}
\define@key{fams}{dnj}{Mande}
\define@key{fams}{daa}{Afro-Asiatic}
\define@key{fams}{dni}{Trans-New Guinea}
\define@key{fams}{dan}{Indo-European}
\define@key{fams}{dry}{Indo-European}
\define@key{fams}{dar}{Nakh-Daghestanian}
\define@key{fams}{prs}{Indo-European}
\define@key{fams}{drd}{Sino-Tibetan}
\define@key{fams}{tcc}{Eastern Sudanic}
\define@key{fams}{dai}{Niger-Congo}
\define@key{fams}{afn}{Ijoid}
\define@key{fams}{deg}{Niger-Congo}
\define@key{fams}{ing}{Na-Dene}
\define@key{fams}{dny}{Arauan}
\define@key{fams}{des}{Tucanoan}
\define@key{fams}{shg}{Khoe-Kwadi}
\define@key{fams}{der}{Sino-Tibetan}
\define@key{fams}{gsg}{other}
\define@key{fams}{dsh}{Afro-Asiatic}
\define@key{fams}{dhl}{Pama-Nyungan}
\define@key{fams}{tbh}{Pama-Nyungan}
\define@key{fams}{dhr}{Pama-Nyungan}
\define@key{fams}{xgm}{Pama-Nyungan}
\define@key{fams}{dhi}{Sino-Tibetan}
\define@key{fams}{div}{Indo-European}
\define@key{fams}{dhu}{Pama-Nyungan}
\define@key{fams}{did}{Eastern Sudanic}
\define@key{fams}{mhu}{Sino-Tibetan}
\define@key{fams}{dur}{Niger-Congo}
\define@key{fams}{dis}{Sino-Tibetan}
\define@key{fams}{dim}{Afro-Asiatic}
\define@key{fams}{diz}{Niger-Congo}
\define@key{fams}{din}{Eastern Sudanic}
\define@key{fams}{dyo}{Niger-Congo}
\define@key{fams}{csk}{Niger-Congo}
\define@key{fams}{dif}{Pama-Nyungan}
\define@key{fams}{mdx}{Afro-Asiatic}
\define@key{fams}{dyy}{Pama-Nyungan}
\define@key{fams}{djr}{Pama-Nyungan}
\define@key{fams}{duj}{Pama-Nyungan}
\define@key{fams}{ddj}{Pama-Nyungan}
\define@key{fams}{dji}{Pama-Nyungan}
\define@key{fams}{jig}{Mirndi}
\define@key{fams}{kbv}{Senagi}
\define@key{fams}{kvo}{Austronesian}
\define@key{fams}{dgo}{Indo-European}
\define@key{fams}{dlg}{Altaic}
\define@key{fams}{dmk}{Indo-European}
\define@key{fams}{rmt}{Indo-European}
\define@key{fams}{kmc}{Tai-Kadai}
\define@key{fams}{doo}{Niger-Congo}
\define@key{fams}{dds}{Dogon}
\define@key{fams}{tds}{Lakes Plain}
\define@key{fams}{dow}{Niger-Congo}
\define@key{fams}{dhv}{Austronesian}
\define@key{fams}{dua}{Niger-Congo}
\define@key{fams}{dud}{Niger-Congo}
\define@key{fams}{gwd}{Afro-Asiatic}
\define@key{fams}{duu}{Sino-Tibetan}
\define@key{fams}{dma}{Niger-Congo}
\define@key{fams}{dgc}{Austronesian}
\define@key{fams}{dus}{Sino-Tibetan}
\define@key{fams}{vam}{Skou}
\define@key{fams}{duc}{Duna-Bogaya}
\define@key{fams}{nld}{Indo-European}
\define@key{fams}{zea}{Indo-European}
\define@key{fams}{dyi}{Niger-Congo}
\define@key{fams}{dbl}{Pama-Nyungan}
\define@key{fams}{dyu}{Mande}
\define@key{fams}{kwa}{Nadahup}
\define@key{fams}{igb}{Niger-Congo}
\define@key{fams}{etr}{Trans-New Guinea}
\define@key{fams}{erk}{Austronesian}
\define@key{fams}{efi}{Niger-Congo}
\define@key{fams}{ega}{Niger-Congo}
\define@key{fams}{eip}{Trans-New Guinea}
\define@key{fams}{etu}{Niger-Congo}
\define@key{fams}{ekg}{Trans-New Guinea}
\define@key{fams}{eko}{Niger-Congo}
\define@key{fams}{mrf}{Morwap}
\define@key{fams}{ema}{Niger-Congo}
\define@key{fams}{emb}{Austronesian}
\define@key{fams}{cmi}{Choco}
\define@key{fams}{emp}{Choco}
\define@key{fams}{amy}{Western Daly}
\define@key{fams}{enq}{Trans-New Guinea}
\define@key{fams}{enn}{Niger-Congo}
\define@key{fams}{eno}{Austronesian}
\define@key{fams}{eng}{Indo-European}
\define@key{fams}{gey}{Niger-Congo}
\define@key{fams}{sja}{Choco}
\define@key{fams}{erg}{Austronesian}
\define@key{fams}{ese}{Pano-Tacanan}
\define@key{fams}{esq}{Isolate}
\define@key{fams}{ekk}{Uralic}
\define@key{fams}{ets}{Niger-Congo}
\define@key{fams}{eve}{Altaic}
\define@key{fams}{ewe}{Niger-Congo}
\define@key{fams}{ewo}{Niger-Congo}
\define@key{fams}{eya}{Na-Dene}
\define@key{fams}{fao}{Indo-European}
\define@key{fams}{faa}{Trans-New Guinea}
\define@key{fams}{fmp}{Niger-Congo}
\define@key{fams}{fij}{Austronesian}
\define@key{fams}{fin}{Uralic}
\define@key{fams}{fse}{other}
\define@key{fams}{foi}{Trans-New Guinea}
\define@key{fams}{ppo}{Teberan-Pawaian}
\define@key{fams}{fon}{Niger-Congo}
\define@key{fams}{frd}{Austronesian}
\define@key{fams}{for}{Trans-New Guinea}
\define@key{fams}{sac}{Algic}
\define@key{fams}{fra}{Indo-European}
\define@key{fams}{fry}{Indo-European}
\define@key{fams}{frs}{Indo-European}
\define@key{fams}{frr}{Indo-European}
\define@key{fams}{fuh}{Niger-Congo}
\define@key{fams}{fuf}{Niger-Congo}
\define@key{fams}{fub}{Niger-Congo}
\define@key{fams}{ffm}{Niger-Congo}
\define@key{fams}{fuv}{Niger-Congo}
\define@key{fams}{fun}{Isolate}
\define@key{fams}{fvr}{Isolate}
\define@key{fams}{fud}{Austronesian}
\define@key{fams}{fut}{Austronesian}
\define@key{fams}{cdo}{Sino-Tibetan}
\define@key{fams}{pym}{Niger-Congo}
\define@key{fams}{gqa}{Afro-Asiatic}
\define@key{fams}{gbu}{Isolate}
\define@key{fams}{dhg}{Pama-Nyungan}
\define@key{fams}{gdb}{Dravidian}
\define@key{fams}{ged}{Niger-Congo}
\define@key{fams}{gaj}{Trans-New Guinea}
\define@key{fams}{gla}{Indo-European}
\define@key{fams}{gag}{Altaic}
\define@key{fams}{gah}{Trans-New Guinea}
\define@key{fams}{gbi}{North Halmaheran}
\define@key{fams}{glg}{Indo-European}
\define@key{fams}{adl}{Sino-Tibetan}
\define@key{fams}{kld}{Pama-Nyungan}
\define@key{fams}{gmv}{Afro-Asiatic}
\define@key{fams}{pwg}{Austronesian}
\define@key{fams}{grt}{Sino-Tibetan}
\define@key{fams}{wrk}{Garrwan}
\define@key{fams}{gyb}{Trans-New Guinea}
\define@key{fams}{cab}{Arawakan}
\define@key{fams}{gvo}{Tupian}
\define@key{fams}{gay}{Austronesian}
\define@key{fams}{gya}{Niger-Congo}
\define@key{fams}{gso}{Niger-Congo}
\define@key{fams}{gbp}{Niger-Congo}
\define@key{fams}{nlg}{Austronesian}
\define@key{fams}{gqu}{Tai-Kadai}
\define@key{fams}{kat}{Kartvelian}
\define@key{fams}{deu}{Indo-European}
\define@key{fams}{bar}{Indo-European}
\define@key{fams}{ksh}{Indo-European}
\define@key{fams}{wep}{Indo-European}
\define@key{fams}{aaa}{Niger-Congo}
\define@key{fams}{ghl}{Eastern Sudanic}
\define@key{fams}{gih}{Pama-Nyungan}
\define@key{fams}{gid}{Afro-Asiatic}
\define@key{fams}{glk}{Indo-European}
\define@key{fams}{bcq}{Afro-Asiatic}
\define@key{fams}{git}{Tsimshianic}
\define@key{fams}{gis}{Afro-Asiatic}
\define@key{fams}{guc}{Arawakan}
\define@key{fams}{god}{Niger-Congo}
\define@key{fams}{gdo}{Nakh-Daghestanian}
\define@key{fams}{ank}{Afro-Asiatic}
\define@key{fams}{ggw}{Trans-New Guinea}
\define@key{fams}{gju}{Indo-European}
\define@key{fams}{gkn}{Niger-Congo}
\define@key{fams}{gol}{Niger-Congo}
\define@key{fams}{gvf}{Trans-New Guinea}
\define@key{fams}{gno}{Dravidian}
\define@key{fams}{gni}{Bunuban}
\define@key{fams}{gor}{Austronesian}
\define@key{fams}{gow}{Afro-Asiatic}
\define@key{fams}{grj}{Niger-Congo}
\define@key{fams}{ell}{Indo-European}
\define@key{fams}{gss}{other}
\define@key{fams}{kal}{Eskimo-Aleut}
\define@key{fams}{guh}{Guahiban}
\define@key{fams}{gub}{Tupian}
\define@key{fams}{gum}{Barbacoan}
\define@key{fams}{gva}{Mascoian}
\define@key{fams}{gvc}{Tucanoan}
\define@key{fams}{gug}{Tupian}
\define@key{fams}{var}{Uto-Aztecan}
\define@key{fams}{gta}{Isolate}
\define@key{fams}{guo}{Guahiban}
\define@key{fams}{gde}{Afro-Asiatic}
\define@key{fams}{gdf}{Afro-Asiatic}
\define@key{fams}{ktd}{Pama-Nyungan}
\define@key{fams}{ggd}{Pama-Nyungan}
\define@key{fams}{ghs}{Trans-New Guinea}
\define@key{fams}{gcr}{other}
\define@key{fams}{pov}{other}
\define@key{fams}{guj}{Indo-European}
\define@key{fams}{kcm}{Central Sudanic}
\define@key{fams}{glj}{Niger-Congo}
\define@key{fams}{gnn}{Pama-Nyungan}
\define@key{fams}{gvs}{Austronesian}
\define@key{fams}{kgs}{Pama-Nyungan}
\define@key{fams}{guk}{Isolate}
\define@key{fams}{wlg}{Gunwinyguan}
\define@key{fams}{guw}{Niger-Congo}
\define@key{fams}{gww}{Worrorran}
\define@key{fams}{yas}{Niger-Congo}
\define@key{fams}{gyy}{Pama-Nyungan}
\define@key{fams}{guf}{Pama-Nyungan}
\define@key{fams}{gnr}{Pama-Nyungan}
\define@key{fams}{gur}{Niger-Congo}
\define@key{fams}{gue}{Pama-Nyungan}
\define@key{fams}{gux}{Niger-Congo}
\define@key{fams}{goa}{Mande}
\define@key{fams}{gge}{Mangrida}
\define@key{fams}{guz}{Niger-Congo}
\define@key{fams}{gbj}{Austro-Asiatic}
\define@key{fams}{kky}{Pama-Nyungan}
\define@key{fams}{gbr}{Niger-Congo}
\define@key{fams}{kcg}{Niger-Congo}
\define@key{fams}{gaa}{Niger-Congo}
\define@key{fams}{pue}{Chonan}
\define@key{fams}{hts}{Isolate}
\define@key{fams}{hai}{Isolate}
\define@key{fams}{hdn}{Haida}
\define@key{fams}{has}{Wakashan}
\define@key{fams}{hat}{other}
\define@key{fams}{hak}{Sino-Tibetan}
\define@key{fams}{hal}{Austro-Asiatic}
\define@key{fams}{hlb}{Indo-European}
\define@key{fams}{hla}{Austronesian}
\define@key{fams}{amf}{Afro-Asiatic}
\define@key{fams}{hmt}{Trans-New Guinea}
\define@key{fams}{wos}{Sepik}
\define@key{fams}{hni}{Sino-Tibetan}
\define@key{fams}{hnn}{Austronesian}
\define@key{fams}{har}{Afro-Asiatic}
\define@key{fams}{hss}{Afro-Asiatic}
\define@key{fams}{tmd}{Piawi}
\define@key{fams}{had}{Hatim-Mansim}
\define@key{fams}{hau}{Afro-Asiatic}
\define@key{fams}{haw}{Austronesian}
\define@key{fams}{hwc}{other}
\define@key{fams}{hac}{Indo-European}
\define@key{fams}{hay}{Niger-Congo}
\define@key{fams}{vay}{Sino-Tibetan}
\define@key{fams}{xed}{Afro-Asiatic}
\define@key{fams}{heb}{Afro-Asiatic}
\define@key{fams}{heh}{Niger-Congo}
\define@key{fams}{hei}{Wakashan}
\define@key{fams}{hem}{Niger-Congo}
\define@key{fams}{her}{Niger-Congo}
\define@key{fams}{hid}{Siouan}
\define@key{fams}{hil}{Austronesian}
\define@key{fams}{hin}{Indo-European}
\define@key{fams}{gin}{Nakh-Daghestanian}
\define@key{fams}{hix}{Cariban}
\define@key{fams}{lic}{Tai-Kadai}
\define@key{fams}{hmr}{Sino-Tibetan}
\define@key{fams}{mww}{Hmong-Mien}
\define@key{fams}{hnj}{Hmong-Mien}
\define@key{fams}{hoc}{Austro-Asiatic}
\define@key{fams}{hoa}{Austronesian}
\define@key{fams}{hoo}{Niger-Congo}
\define@key{fams}{hks}{other}
\define@key{fams}{hop}{Uto-Aztecan}
\define@key{fams}{hre}{Austro-Asiatic}
\define@key{fams}{ygr}{Trans-New Guinea}
\define@key{fams}{hub}{Jivaroan}
\define@key{fams}{hus}{Mayan}
\define@key{fams}{huv}{Huavean}
\define@key{fams}{hch}{Uto-Aztecan}
\define@key{fams}{hto}{Witotoan}
\define@key{fams}{hux}{Witotoan}
\define@key{fams}{huu}{Witotoan}
\define@key{fams}{hke}{Niger-Congo}
\define@key{fams}{hun}{Uralic}
\define@key{fams}{huz}{Nakh-Daghestanian}
\define@key{fams}{jup}{Nadahup}
\define@key{fams}{hup}{Na-Dene}
\define@key{fams}{csh}{Sino-Tibetan}
\define@key{fams}{ksi}{Skou}
\define@key{fams}{iai}{Austronesian}
\define@key{fams}{ian}{Sepik}
\define@key{fams}{tmu}{Lakes Plain}
\define@key{fams}{iba}{Austronesian}
\define@key{fams}{ibg}{Austronesian}
\define@key{fams}{ibb}{Niger-Congo}
\define@key{fams}{isl}{Indo-European}
\define@key{fams}{icl}{other}
\define@key{fams}{idu}{Niger-Congo}
\define@key{fams}{clk}{Sino-Tibetan}
\define@key{fams}{viv}{Austronesian}
\define@key{fams}{mxe}{Austronesian}
\define@key{fams}{ifb}{Austronesian}
\define@key{fams}{ifm}{Niger-Congo}
\define@key{fams}{ibo}{Niger-Congo}
\define@key{fams}{ige}{Niger-Congo}
\define@key{fams}{ign}{Arawakan}
\define@key{fams}{ihp}{Greater West Bomberai}
\define@key{fams}{ijc}{Ijoid}
\define@key{fams}{ikx}{Eastern Sudanic}
\define@key{fams}{arh}{Chibchan}
\define@key{fams}{ilb}{Niger-Congo}
\define@key{fams}{mia}{Algic}
\define@key{fams}{ilo}{Austronesian}
\define@key{fams}{imn}{Border}
\define@key{fams}{szp}{South Bird's Head}
\define@key{fams}{ins}{other}
\define@key{fams}{pks}{other}
\define@key{fams}{ind}{Austronesian}
\define@key{fams}{pmy}{Austronesian}
\define@key{fams}{inb}{Quechuan}
\define@key{fams}{tbi}{Eastern Sudanic}
\define@key{fams}{inh}{Nakh-Daghestanian}
\define@key{fams}{ynd}{Pama-Nyungan}
\define@key{fams}{ils}{other}
\define@key{fams}{ike}{Eskimo-Aleut}
\define@key{fams}{iqu}{Zaparoan}
\define@key{fams}{irn}{Isolate}
\define@key{fams}{irk}{Afro-Asiatic}
\define@key{fams}{irh}{Austronesian}
\define@key{fams}{gle}{Indo-European}
\define@key{fams}{isg}{other}
\define@key{fams}{its}{Niger-Congo}
\define@key{fams}{isk}{Indo-European}
\define@key{fams}{srl}{Tor-Kwerba}
\define@key{fams}{isd}{Austronesian}
\define@key{fams}{iso}{Niger-Congo}
\define@key{fams}{isr}{other}
\define@key{fams}{ita}{Indo-European}
\define@key{fams}{egl}{Indo-European}
\define@key{fams}{lij}{Indo-European}
\define@key{fams}{nap}{Indo-European}
\define@key{fams}{pms}{Indo-European}
\define@key{fams}{itv}{Austronesian}
\define@key{fams}{itl}{Chukotko-Kamchatkan}
\define@key{fams}{ito}{Isolate}
\define@key{fams}{itz}{Mayan}
\define@key{fams}{ivb}{Austronesian}
\define@key{fams}{ibd}{Iwaidjan}
\define@key{fams}{iwm}{Sepik}
\define@key{fams}{yom}{Niger-Congo}
\define@key{fams}{ixc}{Oto-Manguean}
\define@key{fams}{ixl}{Mayan}
\define@key{fams}{izr}{Niger-Congo}
\define@key{fams}{izh}{Uralic}
\define@key{fams}{izz}{Niger-Congo}
\define@key{fams}{esi}{Eskimo-Aleut}
\define@key{fams}{jbt}{Macro-Ge}
\define@key{fams}{jae}{Austronesian}
\define@key{fams}{jda}{Sino-Tibetan}
\define@key{fams}{jhi}{Austro-Asiatic}
\define@key{fams}{jac}{Mayan}
\define@key{fams}{jam}{other}
\define@key{fams}{djd}{Mirndi}
\define@key{fams}{djm}{Dogon}
\define@key{fams}{jpn}{Isolate}
\define@key{fams}{jru}{Cariban}
\define@key{fams}{jqr}{Aymaran}
\define@key{fams}{anq}{South Andamanese}
\define@key{fams}{jav}{Austronesian}
\define@key{fams}{jeb}{Cahuapanan}
\define@key{fams}{jeh}{Austro-Asiatic}
\define@key{fams}{jek}{Mande}
\define@key{fams}{tow}{Kiowa-Tanoan}
\define@key{fams}{jya}{Sino-Tibetan}
\define@key{fams}{shv}{Afro-Asiatic}
\define@key{fams}{kac}{Sino-Tibetan}
\define@key{fams}{jiu}{Sino-Tibetan}
\define@key{fams}{jiv}{Jivaroan}
\define@key{fams}{rgk}{Sino-Tibetan}
\define@key{fams}{tlo}{Kordofanian}
\define@key{fams}{jun}{Austro-Asiatic}
\define@key{fams}{nst}{Sino-Tibetan}
\define@key{fams}{jbu}{Niger-Congo}
\define@key{fams}{bex}{Central Sudanic}
\define@key{fams}{juc}{Altaic}
\define@key{fams}{jur}{Tupian}
\define@key{fams}{ktz}{Kxa}
\define@key{fams}{jua}{Tupian}
\define@key{fams}{kek}{Mayan}
\define@key{fams}{kbd}{Northwest Caucasian}
\define@key{fams}{xkp}{Indo-European}
\define@key{fams}{kbp}{Niger-Congo}
\define@key{fams}{nbu}{Sino-Tibetan}
\define@key{fams}{kab}{Afro-Asiatic}
\define@key{fams}{xac}{Sino-Tibetan}
\define@key{fams}{kzj}{Austronesian}
\define@key{fams}{kbc}{Guaicuruan}
\define@key{fams}{kdm}{Niger-Congo}
\define@key{fams}{kki}{Niger-Congo}
\define@key{fams}{kct}{Lower Sepik-Ramu}
\define@key{fams}{lew}{Austronesian}
\define@key{fams}{kgp}{Macro-Ge}
\define@key{fams}{kxa}{Austronesian}
\define@key{fams}{kgk}{Tupian}
\define@key{fams}{tbd}{Tate}
\define@key{fams}{mwp}{Pama-Nyungan}
\define@key{fams}{kmh}{Trans-New Guinea}
\define@key{fams}{gwc}{Indo-European}
\define@key{fams}{kck}{Niger-Congo}
\define@key{fams}{kyl}{Kalapuyan}
\define@key{fams}{kls}{Indo-European}
\define@key{fams}{fla}{Salishan}
\define@key{fams}{ktg}{Pama-Nyungan}
\define@key{fams}{bco}{Trans-New Guinea}
\define@key{fams}{kay}{Tupian}
\define@key{fams}{kbq}{Trans-New Guinea}
\define@key{fams}{kms}{Torricelli}
\define@key{fams}{xas}{Uralic}
\define@key{fams}{kam}{Niger-Congo}
\define@key{fams}{xbr}{Austronesian}
\define@key{fams}{kbx}{Lower Sepik-Ramu}
\define@key{fams}{kcu}{Niger-Congo}
\define@key{fams}{kgq}{Asmat-Kamrau Bay}
\define@key{fams}{xmu}{Eastern Daly}
\define@key{fams}{ogo}{Niger-Congo}
\define@key{fams}{kna}{Afro-Asiatic}
\define@key{fams}{xns}{Sino-Tibetan}
\define@key{fams}{kbl}{Saharan}
\define@key{fams}{ikt}{Eskimo-Aleut}
\define@key{fams}{kjb}{Mayan}
\define@key{fams}{knj}{Mayan}
\define@key{fams}{kne}{Austronesian}
\define@key{fams}{kan}{Dravidian}
\define@key{fams}{kxo}{Kapixana}
\define@key{fams}{khd}{Yam}
\define@key{fams}{kcd}{Yam}
\define@key{fams}{knc}{Saharan}
\define@key{fams}{kny}{Niger-Congo}
\define@key{fams}{pam}{Austronesian}
\define@key{fams}{kpg}{Austronesian}
\define@key{fams}{kah}{Central Sudanic}
\define@key{fams}{leu}{Austronesian}
\define@key{fams}{krc}{Altaic}
\define@key{fams}{gbd}{Pama-Nyungan}
\define@key{fams}{kdr}{Altaic}
\define@key{fams}{kpj}{Macro-Ge}
\define@key{fams}{kaa}{Altaic}
\define@key{fams}{zkk}{Isolate}
\define@key{fams}{kyj}{Austronesian}
\define@key{fams}{kpt}{Nakh-Daghestanian}
\define@key{fams}{krl}{Uralic}
\define@key{fams}{bwe}{Sino-Tibetan}
\define@key{fams}{kjp}{Sino-Tibetan}
\define@key{fams}{ksw}{Sino-Tibetan}
\define@key{fams}{vka}{Pama-Nyungan}
\define@key{fams}{kdj}{Eastern Sudanic}
\define@key{fams}{ktn}{Tupian}
\define@key{fams}{yuj}{Pauwasi}
\define@key{fams}{kyh}{Hokan}
\define@key{fams}{arr}{Tupian}
\define@key{fams}{xsm}{Niger-Congo}
\define@key{fams}{kju}{Hokan}
\define@key{fams}{kas}{Indo-European}
\define@key{fams}{csb}{Indo-European}
\define@key{fams}{cog}{Austro-Asiatic}
\define@key{fams}{bqy}{other}
\define@key{fams}{xtc}{Kadu}
\define@key{fams}{bsh}{Indo-European}
\define@key{fams}{kts}{Trans-New Guinea}
\define@key{fams}{kcr}{Kordofanian}
\define@key{fams}{ktw}{Na-Dene}
\define@key{fams}{pss}{Austronesian}
\define@key{fams}{bpp}{Isolate}
\define@key{fams}{zku}{Pama-Nyungan}
\define@key{fams}{xaw}{Uto-Aztecan}
\define@key{fams}{kyz}{Tupian}
\define@key{fams}{eky}{Sino-Tibetan}
\define@key{fams}{kys}{Austronesian}
\define@key{fams}{txu}{Macro-Ge}
\define@key{fams}{gyd}{Tangkic}
\define@key{fams}{gbb}{Pama-Nyungan}
\define@key{fams}{kaz}{Altaic}
\define@key{fams}{ksx}{Austronesian}
\define@key{fams}{kbr}{Afro-Asiatic}
\define@key{fams}{kei}{Austronesian}
\define@key{fams}{kcl}{Austronesian}
\define@key{fams}{kzi}{Austronesian}
\define@key{fams}{sbc}{Austronesian}
\define@key{fams}{ahg}{Afro-Asiatic}
\define@key{fams}{kmt}{Nimboran}
\define@key{fams}{kyq}{Central Sudanic}
\define@key{fams}{keu}{Austronesian}
\define@key{fams}{xki}{other}
\define@key{fams}{ken}{Niger-Congo}
\define@key{fams}{xxk}{Austronesian}
\define@key{fams}{ker}{Afro-Asiatic}
\define@key{fams}{krk}{Chukotko-Kamchatkan}
\define@key{fams}{kee}{Keresan}
\define@key{fams}{ket}{Yeniseian}
\define@key{fams}{xdy}{Austronesian}
\define@key{fams}{kcv}{Niger-Congo}
\define@key{fams}{xte}{Trans-New Guinea}
\define@key{fams}{kew}{Trans-New Guinea}
\define@key{fams}{kjh}{Altaic}
\define@key{fams}{klj}{Altaic}
\define@key{fams}{klr}{Sino-Tibetan}
\define@key{fams}{khk}{Altaic}
\define@key{fams}{kjl}{Sino-Tibetan}
\define@key{fams}{khg}{Sino-Tibetan}
\define@key{fams}{kca}{Uralic}
\define@key{fams}{khr}{Austro-Asiatic}
\define@key{fams}{kha}{Austro-Asiatic}
\define@key{fams}{kjj}{Nakh-Daghestanian}
\define@key{fams}{khm}{Austro-Asiatic}
\define@key{fams}{kjg}{Austro-Asiatic}
\define@key{fams}{khw}{Indo-European}
\define@key{fams}{cnk}{Sino-Tibetan}
\define@key{fams}{khv}{Nakh-Daghestanian}
\define@key{fams}{kkh}{Tai-Kadai}
\define@key{fams}{kic}{Algic}
\define@key{fams}{kik}{Niger-Congo}
\define@key{fams}{hbb}{Afro-Asiatic}
\define@key{fams}{kij}{Austronesian}
\define@key{fams}{klb}{Hokan}
\define@key{fams}{lub}{Niger-Congo}
\define@key{fams}{kig}{Kolopom}
\define@key{fams}{zga}{Niger-Congo}
\define@key{fams}{kfk}{Sino-Tibetan}
\define@key{fams}{kin}{Niger-Congo}
\define@key{fams}{kio}{Kiowa-Tanoan}
\define@key{fams}{kzw}{Kariri}
\define@key{fams}{geb}{Lower Sepik-Ramu}
\define@key{fams}{kir}{Altaic}
\define@key{fams}{gil}{Austronesian}
\define@key{fams}{kiy}{Lakes Plain}
\define@key{fams}{cme}{Niger-Congo}
\define@key{fams}{kje}{Austronesian}
\define@key{fams}{kss}{Niger-Congo}
\define@key{fams}{gia}{Jarrakan}
\define@key{fams}{kii}{Caddoan}
\define@key{fams}{ktu}{other}
\define@key{fams}{kjd}{Trans-New Guinea}
\define@key{fams}{kla}{Penutian}
\define@key{fams}{klu}{Niger-Congo}
\define@key{fams}{yak}{Penutian}
\define@key{fams}{kst}{Niger-Congo}
\define@key{fams}{cku}{Muskogean}
\define@key{fams}{kpw}{Trans-New Guinea}
\define@key{fams}{kfa}{Dravidian}
\define@key{fams}{xwg}{Eastern Sudanic}
\define@key{fams}{xuo}{Niger-Congo}
\define@key{fams}{bcs}{Niger-Congo}
\define@key{fams}{kpx}{Trans-New Guinea}
\define@key{fams}{kbk}{Trans-New Guinea}
\define@key{fams}{kqi}{Trans-New Guinea}
\define@key{fams}{trp}{Sino-Tibetan}
\define@key{fams}{kex}{Indo-European}
\define@key{fams}{kkk}{Austronesian}
\define@key{fams}{kvv}{Austronesian}
\define@key{fams}{kfb}{Dravidian}
\define@key{fams}{kvw}{Greater West Bomberai}
\define@key{fams}{shm}{Indo-European}
\define@key{fams}{bkm}{Niger-Congo}
\define@key{fams}{xbi}{Torricelli}
\define@key{fams}{kge}{Austronesian}
\define@key{fams}{koi}{Uralic}
\define@key{fams}{xom}{Koman}
\define@key{fams}{kfc}{Dravidian}
\define@key{fams}{kng}{Niger-Congo}
\define@key{fams}{kjc}{Austronesian}
\define@key{fams}{knn}{Indo-European}
\define@key{fams}{xon}{Niger-Congo}
\define@key{fams}{mjd}{Penutian}
\define@key{fams}{kma}{Niger-Congo}
\define@key{fams}{kyx}{West Bougainville}
\define@key{fams}{cou}{Niger-Congo}
\define@key{fams}{kqy}{Afro-Asiatic}
\define@key{fams}{kpr}{Trans-New Guinea}
\define@key{fams}{kqz}{Khoe-Kwadi}
\define@key{fams}{knk}{Mande}
\define@key{fams}{kor}{Isolate}
\define@key{fams}{coe}{Tucanoan}
\define@key{fams}{kfq}{Austro-Asiatic}
\define@key{fams}{kfz}{Niger-Congo}
\define@key{fams}{khe}{Trans-New Guinea}
\define@key{fams}{kpy}{Chukotko-Kamchatkan}
\define@key{fams}{kia}{Niger-Congo}
\define@key{fams}{kos}{Austronesian}
\define@key{fams}{kfe}{Dravidian}
\define@key{fams}{aal}{Afro-Asiatic}
\define@key{fams}{kff}{Dravidian}
\define@key{fams}{khq}{Songhay}
\define@key{fams}{ses}{Songhay}
\define@key{fams}{koy}{Na-Dene}
\define@key{fams}{kpk}{Niger-Congo}
\define@key{fams}{xpe}{Mande}
\define@key{fams}{kpo}{Niger-Congo}
\define@key{fams}{xra}{Macro-Ge}
\define@key{fams}{kqq}{Macro-Ge}
\define@key{fams}{krs}{Central Sudanic}
\define@key{fams}{rop}{other}
\define@key{fams}{kgo}{Kadu}
\define@key{fams}{jct}{Altaic}
\define@key{fams}{kry}{Nakh-Daghestanian}
\define@key{fams}{puo}{Austro-Asiatic}
\define@key{fams}{sdm}{Austronesian}
\define@key{fams}{uwa}{Pama-Nyungan}
\define@key{fams}{kxu}{Dravidian}
\define@key{fams}{kvd}{Greater West Bomberai}
\define@key{fams}{kui}{Cariban}
\define@key{fams}{gvn}{Pama-Nyungan}
\define@key{fams}{mbt}{Austronesian}
\define@key{fams}{dwr}{Afro-Asiatic}
\define@key{fams}{kle}{Sino-Tibetan}
\define@key{fams}{kue}{Trans-New Guinea}
\define@key{fams}{kfy}{Indo-European}
\define@key{fams}{kum}{Altaic}
\define@key{fams}{kvn}{Chibchan}
\define@key{fams}{kun}{Isolate}
\define@key{fams}{kup}{Trans-New Guinea}
\define@key{fams}{kjn}{Pama-Nyungan}
\define@key{fams}{cmn}{Sino-Tibetan}
\define@key{fams}{kto}{Isolate}
\define@key{fams}{ckb}{Indo-European}
\define@key{fams}{kmr}{Indo-European}
\define@key{fams}{kru}{Dravidian}
\define@key{fams}{kgg}{Isolate}
\define@key{fams}{vkt}{Austronesian}
\define@key{fams}{gwi}{Na-Dene}
\define@key{fams}{kut}{Isolate}
\define@key{fams}{thd}{Pama-Nyungan}
\define@key{fams}{kuy}{Pama-Nyungan}
\define@key{fams}{kxv}{Dravidian}
\define@key{fams}{kwd}{Austronesian}
\define@key{fams}{kwk}{Wakashan}
\define@key{fams}{tnk}{Austronesian}
\define@key{fams}{ksq}{Afro-Asiatic}
\define@key{fams}{kwn}{Niger-Congo}
\define@key{fams}{xwa}{Isolate}
\define@key{fams}{kwe}{Tor-Kwerba}
\define@key{fams}{kmo}{Sepik}
\define@key{fams}{kwo}{Isolate}
\define@key{fams}{xuu}{Khoe-Kwadi}
\define@key{fams}{kyc}{Trans-New Guinea}
\define@key{fams}{kgy}{Sino-Tibetan}
\define@key{fams}{nuk}{Wakashan}
\define@key{fams}{kmg}{Trans-New Guinea}
\define@key{fams}{gdm}{Isolate}
\define@key{fams}{lbu}{Austronesian}
\define@key{fams}{lac}{Mayan}
\define@key{fams}{lbt}{Tai-Kadai}
\define@key{fams}{lbj}{Sino-Tibetan}
\define@key{fams}{lld}{Indo-European}
\define@key{fams}{lad}{Indo-European}
\define@key{fams}{laf}{Kordofanian}
\define@key{fams}{kot}{Afro-Asiatic}
\define@key{fams}{lha}{Tai-Kadai}
\define@key{fams}{lhu}{Sino-Tibetan}
\define@key{fams}{cnh}{Sino-Tibetan}
\define@key{fams}{lbe}{Nakh-Daghestanian}
\define@key{fams}{lkt}{Siouan}
\define@key{fams}{lbc}{Tai-Kadai}
\define@key{fams}{ywt}{Sino-Tibetan}
\define@key{fams}{slp}{Austronesian}
\define@key{fams}{hia}{Afro-Asiatic}
\define@key{fams}{lmn}{Indo-European}
\define@key{fams}{lam}{Niger-Congo}
\define@key{fams}{lmu}{Austronesian}
\define@key{fams}{lns}{Niger-Congo}
\define@key{fams}{ljp}{Austronesian}
\define@key{fams}{lby}{Pama-Nyungan}
\define@key{fams}{lme}{Afro-Asiatic}
\define@key{fams}{lag}{Niger-Congo}
\define@key{fams}{laj}{Eastern Sudanic}
\define@key{fams}{fsl}{other}
\define@key{fams}{fcs}{other}
\define@key{fams}{lao}{Tai-Kadai}
\define@key{fams}{lrg}{Darwin Region}
\define@key{fams}{lbz}{Tangkic}
\define@key{fams}{alo}{Austronesian}
\define@key{fams}{lav}{Indo-European}
\define@key{fams}{llu}{Austronesian}
\define@key{fams}{law}{Austronesian}
\define@key{fams}{lvk}{Solomons East Papuan}
\define@key{fams}{lzz}{Kartvelian}
\define@key{fams}{agh}{Niger-Congo}
\define@key{fams}{lea}{Niger-Congo}
\define@key{fams}{agb}{Niger-Congo}
\define@key{fams}{lec}{Isolate}
\define@key{fams}{lln}{Afro-Asiatic}
\define@key{fams}{lef}{Niger-Congo}
\define@key{fams}{tnl}{Austronesian}
\define@key{fams}{led}{Central Sudanic}
\define@key{fams}{enx}{Mascoian}
\define@key{fams}{aed}{other}
\define@key{fams}{ssp}{other}
\define@key{fams}{lep}{Sino-Tibetan}
\define@key{fams}{les}{Central Sudanic}
\define@key{fams}{lti}{Austronesian}
\define@key{fams}{lww}{Austronesian}
\define@key{fams}{lez}{Nakh-Daghestanian}
\define@key{fams}{lhm}{Sino-Tibetan}
\define@key{fams}{lil}{Salishan}
\define@key{fams}{lif}{Sino-Tibetan}
\define@key{fams}{lmc}{Darwin Region}
\define@key{fams}{liy}{Niger-Congo}
\define@key{fams}{lin}{Niger-Congo}
\define@key{fams}{ise}{other}
\define@key{fams}{lnj}{Pama-Nyungan}
\define@key{fams}{lis}{Sino-Tibetan}
\define@key{fams}{lit}{Indo-European}
\define@key{fams}{liv}{Uralic}
\define@key{fams}{lob}{Niger-Congo}
\define@key{fams}{log}{Central Sudanic}
\define@key{fams}{lok}{Mande}
\define@key{fams}{arw}{Arawakan}
\define@key{fams}{lom}{Mande}
\define@key{fams}{bdu}{Niger-Congo}
\define@key{fams}{lgu}{Austronesian}
\define@key{fams}{los}{Austronesian}
\define@key{fams}{crc}{Austronesian}
\define@key{fams}{njh}{Sino-Tibetan}
\define@key{fams}{loj}{Austronesian}
\define@key{fams}{lbo}{Austro-Asiatic}
\define@key{fams}{nds}{Indo-European}
\define@key{fams}{loz}{Niger-Congo}
\define@key{fams}{nie}{Niger-Congo}
\define@key{fams}{ojv}{Austronesian}
\define@key{fams}{lch}{Niger-Congo}
\define@key{fams}{lug}{Niger-Congo}
\define@key{fams}{lgg}{Central Sudanic}
\define@key{fams}{jos}{other}
\define@key{fams}{lui}{Uto-Aztecan}
\define@key{fams}{ule}{Isolate}
\define@key{fams}{str}{Salishan}
\define@key{fams}{lnd}{Austronesian}
\define@key{fams}{lun}{Niger-Congo}
\define@key{fams}{luo}{Eastern Sudanic}
\define@key{fams}{lrc}{Indo-European}
\define@key{fams}{lut}{Salishan}
\define@key{fams}{khl}{Austronesian}
\define@key{fams}{lue}{Niger-Congo}
\define@key{fams}{lwo}{Eastern Sudanic}
\define@key{fams}{ltz}{Indo-European}
\define@key{fams}{luy}{Niger-Congo}
\define@key{fams}{lee}{Niger-Congo}
\define@key{fams}{psr}{other}
\define@key{fams}{bzs}{other}
\define@key{fams}{khb}{Tai-Kadai}
\define@key{fams}{msj}{Niger-Congo}
\define@key{fams}{mhy}{Austronesian}
\define@key{fams}{mhi}{Central Sudanic}
\define@key{fams}{slz}{Austronesian}
\define@key{fams}{mdy}{Afro-Asiatic}
\define@key{fams}{mas}{Eastern Sudanic}
\define@key{fams}{mde}{Maban}
\define@key{fams}{mca}{Matacoan}
\define@key{fams}{mbn}{Guahiban}
\define@key{fams}{mkd}{Indo-European}
\define@key{fams}{mcb}{Arawakan}
\define@key{fams}{myy}{Tucanoan}
\define@key{fams}{mbc}{Cariban}
\define@key{fams}{mxu}{Afro-Asiatic}
\define@key{fams}{mda}{Niger-Congo}
\define@key{fams}{dmd}{Pama-Nyungan}
\define@key{fams}{mad}{Austronesian}
\define@key{fams}{mmw}{Austronesian}
\define@key{fams}{mag}{Indo-European}
\define@key{fams}{mgp}{Sino-Tibetan}
\define@key{fams}{mrd}{Sino-Tibetan}
\define@key{fams}{mgu}{Trans-New Guinea}
\define@key{fams}{mdh}{Austronesian}
\define@key{fams}{mhe}{Austro-Asiatic}
\define@key{fams}{xpq}{Algic}
\define@key{fams}{nmu}{Penutian}
\define@key{fams}{zrs}{Mairasic}
\define@key{fams}{mbq}{Austronesian}
\define@key{fams}{mai}{Indo-European}
\define@key{fams}{mpe}{Eastern Sudanic}
\define@key{fams}{mcp}{Niger-Congo}
\define@key{fams}{myh}{Wakashan}
\define@key{fams}{mkz}{Greater West Bomberai}
\define@key{fams}{mak}{Austronesian}
\define@key{fams}{mgf}{Bulaka River}
\define@key{fams}{kde}{Niger-Congo}
\define@key{fams}{mgh}{Niger-Congo}
\define@key{fams}{mcm}{other}
\define@key{fams}{plt}{Austronesian}
\define@key{fams}{mpb}{Northern Daly}
\define@key{fams}{zsm}{Austronesian}
\define@key{fams}{zlm}{Austronesian}
\define@key{fams}{zmi}{Austronesian}
\define@key{fams}{mal}{Dravidian}
\define@key{fams}{mgl}{Austronesian}
\define@key{fams}{gcc}{Baining}
\define@key{fams}{mlt}{Afro-Asiatic}
\define@key{fams}{kmj}{Dravidian}
\define@key{fams}{mam}{Mayan}
\define@key{fams}{mmn}{Austronesian}
\define@key{fams}{mqj}{Austronesian}
\define@key{fams}{mcs}{Niger-Congo}
\define@key{fams}{mgr}{Niger-Congo}
\define@key{fams}{maw}{Niger-Congo}
\define@key{fams}{mdi}{Central Sudanic}
\define@key{fams}{xmm}{Austronesian}
\define@key{fams}{mva}{Austronesian}
\define@key{fams}{mle}{Sepik}
\define@key{fams}{nmm}{Sino-Tibetan}
\define@key{fams}{mnc}{Altaic}
\define@key{fams}{mid}{Afro-Asiatic}
\define@key{fams}{mhq}{Siouan}
\define@key{fams}{mdr}{Austronesian}
\define@key{fams}{mnk}{Mande}
\define@key{fams}{jet}{Border}
\define@key{fams}{mna}{Austronesian}
\define@key{fams}{mpc}{Mangarrayi-Maran}
\define@key{fams}{mdj}{Central Sudanic}
\define@key{fams}{mqy}{Austronesian}
\define@key{fams}{mjg}{Altaic}
\define@key{fams}{mge}{Central Sudanic}
\define@key{fams}{emk}{Mande}
\define@key{fams}{mlq}{Mande}
\define@key{fams}{mfv}{Niger-Congo}
\define@key{fams}{knf}{Niger-Congo}
\define@key{fams}{nge}{Niger-Congo}
\define@key{fams}{mev}{Mande}
\define@key{fams}{mbb}{Austronesian}
\define@key{fams}{mns}{Uralic}
\define@key{fams}{glv}{Indo-European}
\define@key{fams}{mri}{Austronesian}
\define@key{fams}{mcg}{Cariban}
\define@key{fams}{arn}{Araucanian}
\define@key{fams}{mec}{Mangarrayi-Maran}
\define@key{fams}{mrw}{Austronesian}
\define@key{fams}{zmr}{Western Daly}
\define@key{fams}{mar}{Indo-European}
\define@key{fams}{rnp}{Sino-Tibetan}
\define@key{fams}{zmc}{Pama-Nyungan}
\define@key{fams}{mrt}{Afro-Asiatic}
\define@key{fams}{mrj}{Uralic}
\define@key{fams}{mhr}{Uralic}
\define@key{fams}{mrc}{Hokan}
\define@key{fams}{mrz}{Trans-New Guinea}
\define@key{fams}{mbw}{Trans-New Guinea}
\define@key{fams}{zmt}{Western Daly}
\define@key{fams}{mfr}{Western Daly}
\define@key{fams}{mah}{Austronesian}
\define@key{fams}{gcf}{other}
\define@key{fams}{vma}{Pama-Nyungan}
\define@key{fams}{mhx}{Sino-Tibetan}
\define@key{fams}{mcn}{Afro-Asiatic}
\define@key{fams}{jle}{Kordofanian}
\define@key{fams}{mls}{Maban}
\define@key{fams}{wam}{Algic}
\define@key{fams}{mpq}{Pano-Tacanan}
\define@key{fams}{zml}{Eastern Daly}
\define@key{fams}{mcf}{Pano-Tacanan}
\define@key{fams}{mvb}{Na-Dene}
\define@key{fams}{mjk}{Austronesian}
\define@key{fams}{mgw}{Niger-Congo}
\define@key{fams}{mxx}{Mande}
\define@key{fams}{mph}{Iwaidjan}
\define@key{fams}{mfe}{other}
\define@key{fams}{mke}{Indo-European}
\define@key{fams}{mbl}{Macro-Ge}
\define@key{fams}{yan}{Misumalpan}
\define@key{fams}{ayz}{Isolate}
\define@key{fams}{xyj}{Pama-Nyungan}
\define@key{fams}{mfy}{Uto-Aztecan}
\define@key{fams}{mdm}{Niger-Congo}
\define@key{fams}{maz}{Oto-Manguean}
\define@key{fams}{mzn}{Indo-European}
\define@key{fams}{maq}{Oto-Manguean}
\define@key{fams}{mau}{Oto-Manguean}
\define@key{fams}{mfc}{Niger-Congo}
\define@key{fams}{vmb}{Pama-Nyungan}
\define@key{fams}{lnb}{Niger-Congo}
\define@key{fams}{mpk}{Afro-Asiatic}
\define@key{fams}{myb}{Central Sudanic}
\define@key{fams}{mtk}{Niger-Congo}
\define@key{fams}{mdt}{Niger-Congo}
\define@key{fams}{baw}{Niger-Congo}
\define@key{fams}{gmm}{Niger-Congo}
\define@key{fams}{mdq}{Niger-Congo}
\define@key{fams}{mdw}{Niger-Congo}
\define@key{fams}{mhd}{Afro-Asiatic}
\define@key{fams}{mdd}{Niger-Congo}
\define@key{fams}{mym}{Eastern Sudanic}
\define@key{fams}{nux}{Sepik}
\define@key{fams}{gdq}{Afro-Asiatic}
\define@key{fams}{mni}{Sino-Tibetan}
\define@key{fams}{skf}{Tupian}
\define@key{fams}{mek}{Austronesian}
\define@key{fams}{mel}{Austronesian}
\define@key{fams}{bew}{other}
\define@key{fams}{men}{Mande}
\define@key{fams}{mez}{Algic}
\define@key{fams}{mwv}{Austronesian}
\define@key{fams}{sdo}{Austronesian}
\define@key{fams}{mcr}{Trans-New Guinea}
\define@key{fams}{ulk}{Eastern Trans-Fly}
\define@key{fams}{mej}{East Bird's Head}
\define@key{fams}{mpt}{Trans-New Guinea}
\define@key{fams}{crg}{Algic}
\define@key{fams}{mic}{Algic}
\define@key{fams}{mei}{Eastern Sudanic}
\define@key{fams}{ium}{Hmong-Mien}
\define@key{fams}{mmy}{Afro-Asiatic}
\define@key{fams}{mxj}{Sino-Tibetan}
\define@key{fams}{msy}{Lower Sepik-Ramu}
\define@key{fams}{mik}{Muskogean}
\define@key{fams}{mjw}{Sino-Tibetan}
\define@key{fams}{hna}{Afro-Asiatic}
\define@key{fams}{min}{Austronesian}
\define@key{fams}{mvn}{Austronesian}
\define@key{fams}{xmf}{Kartvelian}
\define@key{fams}{mep}{Jarrakan}
\define@key{fams}{nju}{Pama-Nyungan}
\define@key{fams}{mrg}{Sino-Tibetan}
\define@key{fams}{miq}{Misumalpan}
\define@key{fams}{zmq}{Niger-Congo}
\define@key{fams}{csi}{Penutian}
\define@key{fams}{csm}{Penutian}
\define@key{fams}{lmw}{Penutian}
\define@key{fams}{nsq}{Penutian}
\define@key{fams}{pmw}{Penutian}
\define@key{fams}{skd}{Penutian}
\define@key{fams}{mxp}{Mixe-Zoque}
\define@key{fams}{mco}{Mixe-Zoque}
\define@key{fams}{mto}{Mixe-Zoque}
\define@key{fams}{mim}{Oto-Manguean}
\define@key{fams}{mib}{Oto-Manguean}
\define@key{fams}{miy}{Oto-Manguean}
\define@key{fams}{mih}{Oto-Manguean}
\define@key{fams}{miz}{Oto-Manguean}
\define@key{fams}{mxt}{Oto-Manguean}
\define@key{fams}{mio}{Oto-Manguean}
\define@key{fams}{mig}{Oto-Manguean}
\define@key{fams}{mie}{Oto-Manguean}
\define@key{fams}{mil}{Oto-Manguean}
\define@key{fams}{mjc}{Oto-Manguean}
\define@key{fams}{mks}{Oto-Manguean}
\define@key{fams}{mpm}{Oto-Manguean}
\define@key{fams}{mkf}{Afro-Asiatic}
\define@key{fams}{lus}{Sino-Tibetan}
\define@key{fams}{mra}{Austro-Asiatic}
\define@key{fams}{moy}{Afro-Asiatic}
\define@key{fams}{omc}{Isolate}
\define@key{fams}{moc}{Guaicuruan}
\define@key{fams}{mif}{Afro-Asiatic}
\define@key{fams}{mhj}{Altaic}
\define@key{fams}{moh}{Iroquoian}
\define@key{fams}{mov}{Hokan}
\define@key{fams}{mkj}{Austronesian}
\define@key{fams}{moz}{Afro-Asiatic}
\define@key{fams}{mbe}{Penutian}
\define@key{fams}{mso}{Isolate}
\define@key{fams}{fqs}{Baibai-Fas}
\define@key{fams}{mqf}{Trans-New Guinea}
\define@key{fams}{mnw}{Austro-Asiatic}
\define@key{fams}{ndt}{Niger-Congo}
\define@key{fams}{lol}{Niger-Congo}
\define@key{fams}{mog}{Austronesian}
\define@key{fams}{mnz}{Trans-New Guinea}
\define@key{fams}{mnr}{Uto-Aztecan}
\define@key{fams}{mte}{Austronesian}
\define@key{fams}{moe}{Algic}
\define@key{fams}{mxk}{Bogia}
\define@key{fams}{mos}{Niger-Congo}
\define@key{fams}{mop}{Mayan}
\define@key{fams}{mhz}{Austronesian}
\define@key{fams}{mok}{Isolate}
\define@key{fams}{myv}{Uralic}
\define@key{fams}{mdf}{Uralic}
\define@key{fams}{mor}{Kordofanian}
\define@key{fams}{mgd}{Central Sudanic}
\define@key{fams}{cas}{Mosetenan}
\define@key{fams}{meu}{Austronesian}
\define@key{fams}{siw}{South Bougainville}
\define@key{fams}{mzp}{Isolate}
\define@key{fams}{mye}{Niger-Congo}
\define@key{fams}{akc}{Isolate}
\define@key{fams}{dmw}{Pama-Nyungan}
\define@key{fams}{aoj}{Torricelli}
\define@key{fams}{sgw}{Afro-Asiatic}
\define@key{fams}{bmr}{Boran}
\define@key{fams}{chb}{Chibchan}
\define@key{fams}{mlm}{Tai-Kadai}
\define@key{fams}{mzm}{Niger-Congo}
\define@key{fams}{mji}{Hmong-Mien}
\define@key{fams}{mnb}{Austronesian}
\define@key{fams}{mua}{Niger-Congo}
\define@key{fams}{mnf}{Niger-Congo}
\define@key{fams}{myu}{Tupian}
\define@key{fams}{mhk}{Niger-Congo}
\define@key{fams}{umu}{Algic}
\define@key{fams}{moj}{Niger-Congo}
\define@key{fams}{mtq}{Austro-Asiatic}
\define@key{fams}{sur}{Afro-Asiatic}
\define@key{fams}{mtf}{Lower Sepik-Ramu}
\define@key{fams}{mur}{Eastern Sudanic}
\define@key{fams}{mwf}{Southern Daly}
\define@key{fams}{muz}{Eastern Sudanic}
\define@key{fams}{zmu}{Pama-Nyungan}
\define@key{fams}{mug}{Afro-Asiatic}
\define@key{fams}{msu}{Austronesian}
\define@key{fams}{hur}{Salishan}
\define@key{fams}{emi}{Austronesian}
\define@key{fams}{css}{Penutian}
\define@key{fams}{myw}{Austronesian}
\define@key{fams}{mwe}{Niger-Congo}
\define@key{fams}{mlv}{Austronesian}
\define@key{fams}{xak}{Isolate}
\define@key{fams}{bzk}{other}
\define@key{fams}{muh}{Niger-Congo}
\define@key{fams}{naf}{Trans-New Guinea}
\define@key{fams}{wyy}{Austronesian}
\define@key{fams}{mbj}{Nadahup}
\define@key{fams}{nfr}{Niger-Congo}
\define@key{fams}{nbi}{Sino-Tibetan}
\define@key{fams}{nmf}{Sino-Tibetan}
\define@key{fams}{nzm}{Sino-Tibetan}
\define@key{fams}{nag}{other}
\define@key{fams}{nce}{Yale}
\define@key{fams}{nll}{Isolate}
\define@key{fams}{nhn}{Uto-Aztecan}
\define@key{fams}{ncj}{Uto-Aztecan}
\define@key{fams}{nhx}{Uto-Aztecan}
\define@key{fams}{ncl}{Uto-Aztecan}
\define@key{fams}{nhm}{Uto-Aztecan}
\define@key{fams}{nhp}{Uto-Aztecan}
\define@key{fams}{xpo}{Uto-Aztecan}
\define@key{fams}{azz}{Uto-Aztecan}
\define@key{fams}{nhg}{Uto-Aztecan}
\define@key{fams}{ngu}{Uto-Aztecan}
\define@key{fams}{bio}{Kwomtari}
\define@key{fams}{nak}{Austronesian}
\define@key{fams}{nck}{Mangrida}
\define@key{fams}{nal}{Austronesian}
\define@key{fams}{naq}{Khoe-Kwadi}
\define@key{fams}{nmb}{Austronesian}
\define@key{fams}{nab}{Nambikuaran}
\define@key{fams}{nnm}{Sepik}
\define@key{fams}{gld}{Altaic}
\define@key{fams}{ncb}{Austro-Asiatic}
\define@key{fams}{nnb}{Niger-Congo}
\define@key{fams}{niq}{Eastern Sudanic}
\define@key{fams}{sen}{Niger-Congo}
\define@key{fams}{nnk}{Trans-New Guinea}
\define@key{fams}{nnt}{Algic}
\define@key{fams}{tvl}{Austronesian}
\define@key{fams}{npy}{Austronesian}
\define@key{fams}{npa}{Sino-Tibetan}
\define@key{fams}{nrb}{Eastern Sudanic}
\define@key{fams}{nrm}{Austronesian}
\define@key{fams}{nas}{South Bougainville}
\define@key{fams}{nsk}{Algic}
\define@key{fams}{ncz}{Isolate}
\define@key{fams}{ntm}{Niger-Congo}
\define@key{fams}{ntu}{Austronesian}
\define@key{fams}{nau}{Austronesian}
\define@key{fams}{nav}{Na-Dene}
\define@key{fams}{nxq}{Sino-Tibetan}
\define@key{fams}{bud}{Niger-Congo}
\define@key{fams}{nde}{Niger-Congo}
\define@key{fams}{djj}{Mangrida}
\define@key{fams}{ndz}{Niger-Congo}
\define@key{fams}{ndo}{Niger-Congo}
\define@key{fams}{nmd}{Niger-Congo}
\define@key{fams}{ndv}{Niger-Congo}
\define@key{fams}{djk}{other}
\define@key{fams}{dse}{other}
\define@key{fams}{neg}{Altaic}
\define@key{fams}{nsn}{Austronesian}
\define@key{fams}{nee}{Austronesian}
\define@key{fams}{anh}{Trans-New Guinea}
\define@key{fams}{yrk}{Uralic}
\define@key{fams}{nen}{Austronesian}
\define@key{fams}{aij}{Afro-Asiatic}
\define@key{fams}{aii}{Afro-Asiatic}
\define@key{fams}{trg}{Afro-Asiatic}
\define@key{fams}{npi}{Indo-European}
\define@key{fams}{pia}{Uto-Aztecan}
\define@key{fams}{nzs}{other}
\define@key{fams}{new}{Sino-Tibetan}
\define@key{fams}{ney}{Niger-Congo}
\define@key{fams}{nez}{Penutian}
\define@key{fams}{ntj}{Pama-Nyungan}
\define@key{fams}{nxg}{Austronesian}
\define@key{fams}{nig}{Gunwinyguan}
\define@key{fams}{ngk}{Gunwinyguan}
\define@key{fams}{sba}{Central Sudanic}
\define@key{fams}{nam}{Southern Daly}
\define@key{fams}{nio}{Uralic}
\define@key{fams}{nid}{Gunwinyguan}
\define@key{fams}{nay}{Pama-Nyungan}
\define@key{fams}{nrk}{Pama-Nyungan}
\define@key{fams}{nrl}{Pama-Nyungan}
\define@key{fams}{nxn}{Pama-Nyungan}
\define@key{fams}{nbm}{Niger-Congo}
\define@key{fams}{nga}{Niger-Congo}
\define@key{fams}{ngb}{Niger-Congo}
\define@key{fams}{niy}{Central Sudanic}
\define@key{fams}{wyb}{Pama-Nyungan}
\define@key{fams}{ngi}{Afro-Asiatic}
\define@key{fams}{ngo}{Niger-Congo}
\define@key{fams}{llp}{Austronesian}
\define@key{fams}{gym}{Chibchan}
\define@key{fams}{nha}{Pama-Nyungan}
\define@key{fams}{nhr}{Khoe-Kwadi}
\define@key{fams}{nia}{Austronesian}
\define@key{fams}{caq}{Austro-Asiatic}
\define@key{fams}{pcm}{other}
\define@key{fams}{jsl}{other}
\define@key{fams}{nir}{Isolate}
\define@key{fams}{niz}{Torricelli}
\define@key{fams}{nsz}{Penutian}
\define@key{fams}{ncg}{Tsimshianic}
\define@key{fams}{dtd}{Wakashan}
\define@key{fams}{num}{Austronesian}
\define@key{fams}{niu}{Austronesian}
\define@key{fams}{cag}{Matacoan}
\define@key{fams}{niv}{Isolate}
\define@key{fams}{isi}{Niger-Congo}
\define@key{fams}{nko}{Niger-Congo}
\define@key{fams}{cgg}{Niger-Congo}
\define@key{fams}{fia}{Eastern Sudanic}
\define@key{fams}{njb}{Sino-Tibetan}
\define@key{fams}{nog}{Altaic}
\define@key{fams}{not}{Arawakan}
\define@key{fams}{nhu}{Niger-Congo}
\define@key{fams}{snf}{Niger-Congo}
\define@key{fams}{nsl}{other}
\define@key{fams}{nor}{Indo-European}
\define@key{fams}{nse}{Niger-Congo}
\define@key{fams}{nto}{Niger-Congo}
\define@key{fams}{nxl}{Austronesian}
\define@key{fams}{kcn}{other}
\define@key{fams}{dgl}{Eastern Sudanic}
\define@key{fams}{xnz}{Eastern Sudanic}
\define@key{fams}{nus}{Eastern Sudanic}
\define@key{fams}{mbr}{Cacua-Nukak}
\define@key{fams}{nkr}{Austronesian}
\define@key{fams}{nut}{Tai-Kadai}
\define@key{fams}{nuy}{Gunwinyguan}
\define@key{fams}{nuv}{Niger-Congo}
\define@key{fams}{iii}{Sino-Tibetan}
\define@key{fams}{nup}{Niger-Congo}
\define@key{fams}{nuf}{Sino-Tibetan}
\define@key{fams}{cbn}{Austro-Asiatic}
\define@key{fams}{nly}{Pama-Nyungan}
\define@key{fams}{now}{Niger-Congo}
\define@key{fams}{tpq}{Sino-Tibetan}
\define@key{fams}{nym}{Niger-Congo}
\define@key{fams}{nyj}{Niger-Congo}
\define@key{fams}{nyp}{Eastern Sudanic}
\define@key{fams}{nna}{Pama-Nyungan}
\define@key{fams}{nyt}{Pama-Nyungan}
\define@key{fams}{yly}{Austronesian}
\define@key{fams}{nyh}{Nyulnyulan}
\define@key{fams}{nih}{Niger-Congo}
\define@key{fams}{nyi}{Eastern Sudanic}
\define@key{fams}{njz}{Sino-Tibetan}
\define@key{fams}{nyv}{Nyulnyulan}
\define@key{fams}{nys}{Pama-Nyungan}
\define@key{fams}{nzk}{Niger-Congo}
\define@key{fams}{ood}{Uto-Aztecan}
\define@key{fams}{afz}{Lakes Plain}
\define@key{fams}{ann}{Niger-Congo}
\define@key{fams}{oca}{Witotoan}
\define@key{fams}{oci}{Indo-European}
\define@key{fams}{ocu}{Oto-Manguean}
\define@key{fams}{ogb}{Niger-Congo}
\define@key{fams}{ogu}{Niger-Congo}
\define@key{fams}{oyb}{Austro-Asiatic}
\define@key{fams}{xal}{Altaic}
\define@key{fams}{ojs}{Algic}
\define@key{fams}{ciw}{Algic}
\define@key{fams}{oka}{Salishan}
\define@key{fams}{opm}{Trans-New Guinea}
\define@key{fams}{oku}{Niger-Congo}
\define@key{fams}{ong}{Torricelli}
\define@key{fams}{plo}{Mixe-Zoque}
\define@key{fams}{omg}{Tupian}
\define@key{fams}{oma}{Siouan}
\define@key{fams}{aun}{Torricelli}
\define@key{fams}{one}{Iroquoian}
\define@key{fams}{oon}{South Andamanese}
\define@key{fams}{ons}{Trans-New Guinea}
\define@key{fams}{ono}{Iroquoian}
\define@key{fams}{mvf}{Altaic}
\define@key{fams}{ore}{Tucanoan}
\define@key{fams}{tag}{Kordofanian}
\define@key{fams}{ory}{Indo-European}
\define@key{fams}{ort}{Indo-European}
\define@key{fams}{oru}{Indo-European}
\define@key{fams}{oac}{Altaic}
\define@key{fams}{oaa}{Altaic}
\define@key{fams}{okv}{Trans-New Guinea}
\define@key{fams}{oro}{Eleman}
\define@key{fams}{gax}{Afro-Asiatic}
\define@key{fams}{hae}{Afro-Asiatic}
\define@key{fams}{ssn}{Afro-Asiatic}
\define@key{fams}{gaz}{Afro-Asiatic}
\define@key{fams}{ury}{Tor-Kwerba}
\define@key{fams}{osa}{Siouan}
\define@key{fams}{oss}{Indo-European}
\define@key{fams}{iow}{Siouan}
\define@key{fams}{otz}{Oto-Manguean}
\define@key{fams}{ote}{Oto-Manguean}
\define@key{fams}{otq}{Oto-Manguean}
\define@key{fams}{otm}{Oto-Manguean}
\define@key{fams}{otr}{Kordofanian}
\define@key{fams}{owi}{Left May}
\define@key{fams}{pqa}{Afro-Asiatic}
\define@key{fams}{drl}{Pama-Nyungan}
\define@key{fams}{pma}{Austronesian}
\define@key{fams}{pac}{Austro-Asiatic}
\define@key{fams}{pdo}{Austronesian}
\define@key{fams}{pgu}{North Halmaheran}
\define@key{fams}{duf}{Austronesian}
\define@key{fams}{pck}{Sino-Tibetan}
\define@key{fams}{pao}{Uto-Aztecan}
\define@key{fams}{pwn}{Austronesian}
\define@key{fams}{pkn}{Pama-Nyungan}
\define@key{fams}{pau}{Austronesian}
\define@key{fams}{pll}{Austro-Asiatic}
\define@key{fams}{plu}{Arawakan}
\define@key{fams}{fap}{Niger-Congo}
\define@key{fams}{nad}{Pama-Nyungan}
\define@key{fams}{pmz}{Oto-Manguean}
\define@key{fams}{pmf}{Austronesian}
\define@key{fams}{pbh}{Cariban}
\define@key{fams}{kre}{Macro-Ge}
\define@key{fams}{pag}{Austronesian}
\define@key{fams}{pbr}{Niger-Congo}
\define@key{fams}{pan}{Indo-European}
\define@key{fams}{pnw}{Pama-Nyungan}
\define@key{fams}{pap}{other}
\define@key{fams}{prk}{Austro-Asiatic}
\define@key{fams}{asa}{Niger-Congo}
\define@key{fams}{pab}{Arawakan}
\define@key{fams}{pci}{Dravidian}
\define@key{fams}{pst}{Indo-European}
\define@key{fams}{pqm}{Algic}
\define@key{fams}{ptp}{Austronesian}
\define@key{fams}{gfk}{Austronesian}
\define@key{fams}{lae}{Sino-Tibetan}
\define@key{fams}{pwi}{Penutian}
\define@key{fams}{plh}{Austronesian}
\define@key{fams}{pad}{Arauan}
\define@key{fams}{pwa}{Teberan-Pawaian}
\define@key{fams}{paw}{Caddoan}
\define@key{fams}{pay}{Chibchan}
\define@key{fams}{aoc}{Cariban}
\define@key{fams}{peg}{Dravidian}
\define@key{fams}{pip}{Afro-Asiatic}
\define@key{fams}{pes}{Indo-European}
\define@key{fams}{pww}{Sino-Tibetan}
\define@key{fams}{pio}{Arawakan}
\define@key{fams}{pid}{Sáliban}
\define@key{fams}{plg}{Guaicuruan}
\define@key{fams}{piv}{Austronesian}
\define@key{fams}{pif}{Austronesian}
\define@key{fams}{piu}{Pama-Nyungan}
\define@key{fams}{ppl}{Uto-Aztecan}
\define@key{fams}{myp}{Mura}
\define@key{fams}{pir}{Tucanoan}
\define@key{fams}{pib}{Arawakan}
\define@key{fams}{psa}{Trans-New Guinea}
\define@key{fams}{pjt}{Pama-Nyungan}
\define@key{fams}{pit}{Pama-Nyungan}
\define@key{fams}{psd}{other}
\define@key{fams}{gob}{Guahiban}
\define@key{fams}{fwa}{Austronesian}
\define@key{fams}{pbi}{Afro-Asiatic}
\define@key{fams}{poy}{Niger-Congo}
\define@key{fams}{pon}{Austronesian}
\define@key{fams}{rwa}{Skou}
\define@key{fams}{poh}{Mayan}
\define@key{fams}{pko}{Eastern Sudanic}
\define@key{fams}{pox}{Indo-European}
\define@key{fams}{pol}{Indo-European}
\define@key{fams}{poo}{Hokan}
\define@key{fams}{peb}{Hokan}
\define@key{fams}{pej}{Hokan}
\define@key{fams}{pom}{Hokan}
\define@key{fams}{pbe}{Oto-Manguean}
\define@key{fams}{poe}{Oto-Manguean}
\define@key{fams}{pbf}{Oto-Manguean}
\define@key{fams}{poi}{Mixe-Zoque}
\define@key{fams}{poc}{Mayan}
\define@key{fams}{psw}{Austronesian}
\define@key{fams}{por}{Indo-European}
\define@key{fams}{pot}{Algic}
\define@key{fams}{pim}{Algic}
\define@key{fams}{prn}{Indo-European}
\define@key{fams}{pre}{other}
\define@key{fams}{pui}{Isolate}
\define@key{fams}{fuc}{Niger-Congo}
\define@key{fams}{nij}{Austronesian}
\define@key{fams}{puw}{Austronesian}
\define@key{fams}{pmi}{Sino-Tibetan}
\define@key{fams}{puq}{Isolate}
\define@key{fams}{prx}{Sino-Tibetan}
\define@key{fams}{tsz}{Tarascan}
\define@key{fams}{pbb}{Páezan}
\define@key{fams}{lkr}{Eastern Sudanic}
\define@key{fams}{aar}{Afro-Asiatic}
\define@key{fams}{byx}{Baining}
\define@key{fams}{alc}{Alacalufan}
\define@key{fams}{yum}{Hokan}
\define@key{fams}{qxa}{Quechuan}
\define@key{fams}{quy}{Quechuan}
\define@key{fams}{qvc}{Quechuan}
\define@key{fams}{quh}{Quechuan}
\define@key{fams}{quz}{Quechuan}
\define@key{fams}{qug}{Quechuan}
\define@key{fams}{qub}{Quechuan}
\define@key{fams}{qvi}{Quechuan}
\define@key{fams}{qvn}{Quechuan}
\define@key{fams}{quc}{Mayan}
\define@key{fams}{qui}{Chimakuan}
\define@key{fams}{rad}{Austronesian}
\define@key{fams}{lml}{Austronesian}
\define@key{fams}{rji}{Sino-Tibetan}
\define@key{fams}{ral}{Sino-Tibetan}
\define@key{fams}{rma}{Chibchan}
\define@key{fams}{bod}{Sino-Tibetan}
\define@key{fams}{rao}{Lower Sepik-Ramu}
\define@key{fams}{rap}{Austronesian}
\define@key{fams}{ras}{Kordofanian}
\define@key{fams}{rwo}{Trans-New Guinea}
\define@key{fams}{raw}{Sino-Tibetan}
\define@key{fams}{rej}{Austronesian}
\define@key{fams}{rmb}{Gunwinyguan}
\define@key{fams}{bfw}{Austro-Asiatic}
\define@key{fams}{rel}{Afro-Asiatic}
\define@key{fams}{ren}{Austro-Asiatic}
\define@key{fams}{mnv}{Austronesian}
\define@key{fams}{rgr}{Arawakan}
\define@key{fams}{tnc}{Tucanoan}
\define@key{fams}{ran}{Kolopom}
\define@key{fams}{rkb}{Macro-Ge}
\define@key{fams}{rim}{Niger-Congo}
\define@key{fams}{rit}{Pama-Nyungan}
\define@key{fams}{rog}{Austronesian}
\define@key{fams}{rmn}{Indo-European}
\define@key{fams}{rmo}{Indo-European}
\define@key{fams}{rmy}{Indo-European}
\define@key{fams}{rml}{Indo-European}
\define@key{fams}{rmw}{Indo-European}
\define@key{fams}{ron}{Indo-European}
\define@key{fams}{roh}{Indo-European}
\define@key{fams}{cla}{Afro-Asiatic}
\define@key{fams}{rng}{Niger-Congo}
\define@key{fams}{rro}{Austronesian}
\define@key{fams}{twu}{Austronesian}
\define@key{fams}{roo}{West Bougainville}
\define@key{fams}{rtm}{Austronesian}
\define@key{fams}{rug}{Austronesian}
\define@key{fams}{dru}{Austronesian}
\define@key{fams}{klq}{Trans-New Guinea}
\define@key{fams}{run}{Niger-Congo}
\define@key{fams}{rou}{Maban}
\define@key{fams}{nyn}{Niger-Congo}
\define@key{fams}{nyo}{Niger-Congo}
\define@key{fams}{rus}{Indo-European}
\define@key{fams}{rsl}{other}
\define@key{fams}{rut}{Nakh-Daghestanian}
\define@key{fams}{apb}{Austronesian}
\define@key{fams}{snv}{Austronesian}
\define@key{fams}{sma}{Uralic}
\define@key{fams}{sjd}{Uralic}
\define@key{fams}{sme}{Uralic}
\define@key{fams}{skb}{Tai-Kadai}
\define@key{fams}{uma}{Penutian}
\define@key{fams}{ssy}{Afro-Asiatic}
\define@key{fams}{saj}{North Halmaheran}
\define@key{fams}{sku}{Austronesian}
\define@key{fams}{slr}{Altaic}
\define@key{fams}{sbe}{Austronesian}
\define@key{fams}{sln}{Hokan}
\define@key{fams}{slh}{Salishan}
\define@key{fams}{sll}{Trans-New Guinea}
\define@key{fams}{sse}{Austronesian}
\define@key{fams}{ssb}{Austronesian}
\define@key{fams}{ndi}{Niger-Congo}
\define@key{fams}{smq}{Trans-New Guinea}
\define@key{fams}{smo}{Austronesian}
\define@key{fams}{sad}{Isolate}
\define@key{fams}{sxn}{Austronesian}
\define@key{fams}{sag}{Niger-Congo}
\define@key{fams}{snq}{Niger-Congo}
\define@key{fams}{sce}{Altaic}
\define@key{fams}{sat}{Austro-Asiatic}
\define@key{fams}{xsu}{Yanomam}
\define@key{fams}{spu}{Austro-Asiatic}
\define@key{fams}{srm}{other}
\define@key{fams}{srs}{Na-Dene}
\define@key{fams}{sro}{Indo-European}
\define@key{fams}{dju}{Sepik}
\define@key{fams}{ybe}{Altaic}
\define@key{fams}{sdg}{Indo-European}
\define@key{fams}{svs}{Solomons East Papuan}
\define@key{fams}{szw}{Austronesian}
\define@key{fams}{hvn}{Austronesian}
\define@key{fams}{pos}{Mixe-Zoque}
\define@key{fams}{kpz}{Eastern Sudanic}
\define@key{fams}{sey}{Tucanoan}
\define@key{fams}{sed}{Austro-Asiatic}
\define@key{fams}{trv}{Austronesian}
\define@key{fams}{slu}{Austronesian}
\define@key{fams}{sly}{Austronesian}
\define@key{fams}{spl}{Trans-New Guinea}
\define@key{fams}{ona}{Chonan}
\define@key{fams}{sel}{Uralic}
\define@key{fams}{nsm}{Sino-Tibetan}
\define@key{fams}{sea}{Austro-Asiatic}
\define@key{fams}{sif}{Niger-Congo}
\define@key{fams}{sza}{Austro-Asiatic}
\define@key{fams}{seh}{Niger-Congo}
\define@key{fams}{sef}{Niger-Congo}
\define@key{fams}{see}{Iroquoian}
\define@key{fams}{szg}{Niger-Congo}
\define@key{fams}{set}{Isolate}
\define@key{fams}{hbs}{Indo-European}
\define@key{fams}{sei}{Hokan}
\define@key{fams}{ser}{Uto-Aztecan}
\define@key{fams}{sot}{Niger-Congo}
\define@key{fams}{crs}{other}
\define@key{fams}{sbf}{Isolate}
\define@key{fams}{ksb}{Niger-Congo}
\define@key{fams}{shn}{Tai-Kadai}
\define@key{fams}{mcd}{Pano-Tacanan}
\define@key{fams}{sht}{Hokan}
\define@key{fams}{shj}{Eastern Sudanic}
\define@key{fams}{sjw}{Algic}
\define@key{fams}{swv}{Indo-European}
\define@key{fams}{sdp}{Sino-Tibetan}
\define@key{fams}{xsr}{Sino-Tibetan}
\define@key{fams}{shk}{Eastern Sudanic}
\define@key{fams}{scl}{Indo-European}
\define@key{fams}{bwo}{Afro-Asiatic}
\define@key{fams}{shp}{Pano-Tacanan}
\define@key{fams}{yuy}{Altaic}
\define@key{fams}{shb}{Yanomam}
\define@key{fams}{sii}{Isolate}
\define@key{fams}{sna}{Niger-Congo}
\define@key{fams}{cjs}{Altaic}
\define@key{fams}{shh}{Uto-Aztecan}
\define@key{fams}{sgh}{Indo-European}
\define@key{fams}{ryu}{Japanese}
\define@key{fams}{shs}{Salishan}
\define@key{fams}{snp}{Trans-New Guinea}
\define@key{fams}{sjr}{Austronesian}
\define@key{fams}{sid}{Afro-Asiatic}
\define@key{fams}{ski}{Austronesian}
\define@key{fams}{tty}{Lakes Plain}
\define@key{fams}{sip}{Sino-Tibetan}
\define@key{fams}{skh}{Austronesian}
\define@key{fams}{dau}{Eastern Sudanic}
\define@key{fams}{smr}{Austronesian}
\define@key{fams}{snc}{Austronesian}
\define@key{fams}{snd}{Indo-European}
\define@key{fams}{sin}{Indo-European}
\define@key{fams}{xsi}{Austronesian}
\define@key{fams}{snn}{Tucanoan}
\define@key{fams}{qum}{Mayan}
\define@key{fams}{fos}{Austronesian}
\define@key{fams}{sri}{Tucanoan}
\define@key{fams}{srq}{Tupian}
\define@key{fams}{ssd}{Trans-New Guinea}
\define@key{fams}{sil}{Niger-Congo}
\define@key{fams}{baa}{Austronesian}
\define@key{fams}{sis}{Oregon Coast}
\define@key{fams}{skv}{Skou}
\define@key{fams}{den}{Na-Dene}
\define@key{fams}{xsl}{Na-Dene}
\define@key{fams}{slk}{Indo-European}
\define@key{fams}{slv}{Indo-European}
\define@key{fams}{teu}{Eastern Sudanic}
\define@key{fams}{sob}{Austronesian}
\define@key{fams}{gru}{Afro-Asiatic}
\define@key{fams}{evn}{Altaic}
\define@key{fams}{som}{Afro-Asiatic}
\define@key{fams}{sop}{Niger-Congo}
\define@key{fams}{snk}{Mande}
\define@key{fams}{sov}{Austronesian}
\define@key{fams}{sqt}{Afro-Asiatic}
\define@key{fams}{srb}{Austro-Asiatic}
\define@key{fams}{dsb}{Indo-European}
\define@key{fams}{hsb}{Indo-European}
\define@key{fams}{nso}{Niger-Congo}
\define@key{fams}{mnx}{East Bird's Head}
\define@key{fams}{kvk}{other}
\define@key{fams}{tvk}{Austronesian}
\define@key{fams}{wib}{Niger-Congo}
\define@key{fams}{spa}{Indo-European}
\define@key{fams}{spt}{Sino-Tibetan}
\define@key{fams}{spo}{Salishan}
\define@key{fams}{squ}{Salishan}
\define@key{fams}{srn}{other}
\define@key{fams}{kpm}{Austro-Asiatic}
\define@key{fams}{sto}{Siouan}
\define@key{fams}{sbs}{Niger-Congo}
\define@key{fams}{tgo}{Austronesian}
\define@key{fams}{sue}{Trans-New Guinea}
\define@key{fams}{swi}{Tai-Kadai}
\define@key{fams}{sui}{Trans-New Guinea}
\define@key{fams}{sub}{Niger-Congo}
\define@key{fams}{suk}{Niger-Congo}
\define@key{fams}{sua}{Isolate}
\define@key{fams}{suv}{Isolate}
\define@key{fams}{sun}{Austronesian}
\define@key{fams}{sjg}{Eastern Sudanic}
\define@key{fams}{spp}{Niger-Congo}
\define@key{fams}{sgz}{Austronesian}
\define@key{fams}{sus}{Mande}
\define@key{fams}{sva}{Kartvelian}
\define@key{fams}{swl}{other}
\define@key{fams}{swh}{Niger-Congo}
\define@key{fams}{ssw}{Niger-Congo}
\define@key{fams}{swe}{Indo-European}
\define@key{fams}{slc}{Sáliban}
\define@key{fams}{mky}{Austronesian}
\define@key{fams}{sst}{Trans-New Guinea}
\define@key{fams}{tby}{North Halmaheran}
\define@key{fams}{tab}{Nakh-Daghestanian}
\define@key{fams}{tnm}{Sentani}
\define@key{fams}{tap}{Niger-Congo}
\define@key{fams}{tna}{Pano-Tacanan}
\define@key{fams}{tgl}{Austronesian}
\define@key{fams}{tbw}{Austronesian}
\define@key{fams}{tah}{Austronesian}
\define@key{fams}{gpn}{Gapun}
\define@key{fams}{sps}{Austronesian}
\define@key{fams}{tbg}{Trans-New Guinea}
\define@key{fams}{tss}{other}
\define@key{fams}{tgk}{Indo-European}
\define@key{fams}{tkm}{Isolate}
\define@key{fams}{tbc}{Austronesian}
\define@key{fams}{tld}{Austronesian}
\define@key{fams}{tlj}{Niger-Congo}
\define@key{fams}{tly}{Indo-European}
\define@key{fams}{tma}{Eastern Sudanic}
\define@key{fams}{mla}{Austronesian}
\define@key{fams}{tcg}{Kayagar}
\define@key{fams}{taj}{Sino-Tibetan}
\define@key{fams}{taq}{Afro-Asiatic}
\define@key{fams}{tam}{Dravidian}
\define@key{fams}{tpm}{Niger-Congo}
\define@key{fams}{tcb}{Na-Dene}
\define@key{fams}{tfn}{Na-Dene}
\define@key{fams}{taa}{Na-Dene}
\define@key{fams}{tan}{Afro-Asiatic}
\define@key{fams}{skj}{Sino-Tibetan}
\define@key{fams}{tgg}{Austronesian}
\define@key{fams}{tpg}{Greater West Bomberai}
\define@key{fams}{nwi}{Austronesian}
\define@key{fams}{tza}{other}
\define@key{fams}{tpj}{Tupian}
\define@key{fams}{tar}{Uto-Aztecan}
\define@key{fams}{tac}{Uto-Aztecan}
\define@key{fams}{txn}{Austronesian}
\define@key{fams}{tro}{Sino-Tibetan}
\define@key{fams}{tae}{Arawakan}
\define@key{fams}{yer}{Niger-Congo}
\define@key{fams}{shi}{Afro-Asiatic}
\define@key{fams}{ttt}{Indo-European}
\define@key{fams}{txx}{Austronesian}
\define@key{fams}{tat}{Altaic}
\define@key{fams}{tks}{Indo-European}
\define@key{fams}{tav}{Tucanoan}
\define@key{fams}{tuh}{Isolate}
\define@key{fams}{trr}{Isolate}
\define@key{fams}{tsg}{Austronesian}
\define@key{fams}{tya}{Trans-New Guinea}
\define@key{fams}{tbo}{Austronesian}
\define@key{fams}{cks}{other}
\define@key{fams}{tbl}{Austronesian}
\define@key{fams}{ttc}{Mayan}
\define@key{fams}{kps}{West Bird's Head}
\define@key{fams}{teh}{Chonan}
\define@key{fams}{kkw}{Niger-Congo}
\define@key{fams}{tlf}{Trans-New Guinea}
\define@key{fams}{tel}{Dravidian}
\define@key{fams}{kdh}{Niger-Congo}
\define@key{fams}{teq}{Eastern Sudanic}
\define@key{fams}{tea}{Austro-Asiatic}
\define@key{fams}{tem}{Niger-Congo}
\define@key{fams}{tex}{Eastern Sudanic}
\define@key{fams}{kza}{Niger-Congo}
\define@key{fams}{tio}{Austronesian}
\define@key{fams}{tep}{Uto-Aztecan}
\define@key{fams}{tee}{Totonacan}
\define@key{fams}{tpt}{Totonacan}
\define@key{fams}{ntp}{Uto-Aztecan}
\define@key{fams}{stp}{Uto-Aztecan}
\define@key{fams}{ttr}{Afro-Asiatic}
\define@key{fams}{tfr}{Chibchan}
\define@key{fams}{tft}{North Halmaheran}
\define@key{fams}{ter}{Arawakan}
\define@key{fams}{teo}{Eastern Sudanic}
\define@key{fams}{tll}{Niger-Congo}
\define@key{fams}{tet}{Austronesian}
\define@key{fams}{tew}{Kiowa-Tanoan}
\define@key{fams}{tcz}{Sino-Tibetan}
\define@key{fams}{tha}{Tai-Kadai}
\define@key{fams}{tsq}{other}
\define@key{fams}{ths}{Sino-Tibetan}
\define@key{fams}{thf}{Sino-Tibetan}
\define@key{fams}{ssf}{Austronesian}
\define@key{fams}{typ}{Pama-Nyungan}
\define@key{fams}{thp}{Salishan}
\define@key{fams}{tdh}{Sino-Tibetan}
\define@key{fams}{tca}{Isolate}
\define@key{fams}{tvo}{North Halmaheran}
\define@key{fams}{tif}{Trans-New Guinea}
\define@key{fams}{tgc}{Austronesian}
\define@key{fams}{tir}{Afro-Asiatic}
\define@key{fams}{tig}{Afro-Asiatic}
\define@key{fams}{dih}{Hokan}
\define@key{fams}{tik}{Niger-Congo}
\define@key{fams}{til}{Salishan}
\define@key{fams}{tms}{Kordofanian}
\define@key{fams}{aoz}{Austronesian}
\define@key{fams}{tjm}{Isolate}
\define@key{fams}{tih}{Austronesian}
\define@key{fams}{lbf}{Sino-Tibetan}
\define@key{fams}{tin}{Nakh-Daghestanian}
\define@key{fams}{cir}{Austronesian}
\define@key{fams}{tri}{Cariban}
\define@key{fams}{tiy}{Austronesian}
\define@key{fams}{tiv}{Niger-Congo}
\define@key{fams}{twf}{Kiowa-Tanoan}
\define@key{fams}{tix}{Kiowa-Tanoan}
\define@key{fams}{tiw}{Tiwian}
\define@key{fams}{tcf}{Oto-Manguean}
\define@key{fams}{tli}{Na-Dene}
\define@key{fams}{tqo}{Eleman}
\define@key{fams}{tob}{Guaicuruan}
\define@key{fams}{tti}{Austronesian}
\define@key{fams}{tlb}{North Halmaheran}
\define@key{fams}{sbu}{Sino-Tibetan}
\define@key{fams}{tcx}{Dravidian}
\define@key{fams}{kim}{Altaic}
\define@key{fams}{toj}{Mayan}
\define@key{fams}{tpi}{other}
\define@key{fams}{tkl}{Austronesian}
\define@key{fams}{jic}{Isolate}
\define@key{fams}{ksd}{Austronesian}
\define@key{fams}{dto}{Dogon}
\define@key{fams}{tdn}{Austronesian}
\define@key{fams}{toi}{Niger-Congo}
\define@key{fams}{ton}{Austronesian}
\define@key{fams}{tqw}{Isolate}
\define@key{fams}{tnt}{Austronesian}
\define@key{fams}{mlu}{Austronesian}
\define@key{fams}{sda}{Austronesian}
\define@key{fams}{rth}{Austronesian}
\define@key{fams}{dts}{Dogon}
\define@key{fams}{trw}{Indo-European}
\define@key{fams}{tlc}{Totonacan}
\define@key{fams}{top}{Totonacan}
\define@key{fams}{tos}{Totonacan}
\define@key{fams}{too}{Totonacan}
\define@key{fams}{trs}{Oto-Manguean}
\define@key{fams}{trc}{Oto-Manguean}
\define@key{fams}{tpy}{Isolate}
\define@key{fams}{cof}{Barbacoan}
\define@key{fams}{tkr}{Nakh-Daghestanian}
\define@key{fams}{huq}{Austronesian}
\define@key{fams}{ddo}{Nakh-Daghestanian}
\define@key{fams}{tsj}{Sino-Tibetan}
\define@key{fams}{tsi}{Tsimshianic}
\define@key{fams}{tsv}{Niger-Congo}
\define@key{fams}{tso}{Niger-Congo}
\define@key{fams}{tsu}{Austronesian}
\define@key{fams}{bbl}{Nakh-Daghestanian}
\define@key{fams}{tsn}{Niger-Congo}
\define@key{fams}{pmt}{Austronesian}
\define@key{fams}{thz}{Afro-Asiatic}
\define@key{fams}{thv}{Afro-Asiatic}
\define@key{fams}{tbu}{Uto-Aztecan}
\define@key{fams}{tuo}{Tucanoan}
\define@key{fams}{tzn}{Austronesian}
\define@key{fams}{bag}{Niger-Congo}
\define@key{fams}{tcy}{Dravidian}
\define@key{fams}{tmc}{Afro-Asiatic}
\define@key{fams}{tmq}{Austronesian}
\define@key{fams}{tuf}{Chibchan}
\define@key{fams}{tvu}{Niger-Congo}
\define@key{fams}{lcm}{Austronesian}
\define@key{fams}{tun}{Isolate}
\define@key{fams}{tpn}{Tupian}
\define@key{fams}{tui}{Niger-Congo}
\define@key{fams}{tuv}{Eastern Sudanic}
\define@key{fams}{kmz}{Altaic}
\define@key{fams}{tur}{Altaic}
\define@key{fams}{tuk}{Altaic}
\define@key{fams}{tus}{Iroquoian}
\define@key{fams}{ttm}{Na-Dene}
\define@key{fams}{tta}{Siouan}
\define@key{fams}{tvt}{Sino-Tibetan}
\define@key{fams}{tyv}{Altaic}
\define@key{fams}{tue}{Tucanoan}
\define@key{fams}{twa}{Salishan}
\define@key{fams}{woa}{Northern Daly}
\define@key{fams}{tzh}{Mayan}
\define@key{fams}{tzo}{Mayan}
\define@key{fams}{tzj}{Mayan}
\define@key{fams}{tub}{Uto-Aztecan}
\define@key{fams}{par}{Uto-Aztecan}
\define@key{fams}{tsm}{other}
\define@key{fams}{umb}{Niger-Congo}
\define@key{fams}{uby}{Northwest Caucasian}
\define@key{fams}{udi}{Nakh-Daghestanian}
\define@key{fams}{ude}{Altaic}
\define@key{fams}{udm}{Uralic}
\define@key{fams}{ugn}{other}
\define@key{fams}{ukr}{Indo-European}
\define@key{fams}{ulc}{Altaic}
\define@key{fams}{udl}{Afro-Asiatic}
\define@key{fams}{uli}{Austronesian}
\define@key{fams}{ppk}{Austronesian}
\define@key{fams}{cbd}{Cariban}
\define@key{fams}{ubu}{Trans-New Guinea}
\define@key{fams}{ump}{Pama-Nyungan}
\define@key{fams}{mtg}{Trans-New Guinea}
\define@key{fams}{unm}{Algic}
\define@key{fams}{ung}{Worrorran}
\define@key{fams}{kuu}{Na-Dene}
\define@key{fams}{uur}{Austronesian}
\define@key{fams}{urf}{Pama-Nyungan}
\define@key{fams}{urk}{Austronesian}
\define@key{fams}{ura}{Isolate}
\define@key{fams}{urt}{Torricelli}
\define@key{fams}{urd}{Indo-European}
\define@key{fams}{urh}{Niger-Congo}
\define@key{fams}{uri}{Torricelli}
\define@key{fams}{ure}{Uru-Chipaya}
\define@key{fams}{uks}{other}
\define@key{fams}{urb}{Tupian}
\define@key{fams}{uum}{Altaic}
\define@key{fams}{wnu}{Trans-New Guinea}
\define@key{fams}{usa}{Trans-New Guinea}
\define@key{fams}{ute}{Uto-Aztecan}
\define@key{fams}{uig}{Altaic}
\define@key{fams}{uzn}{Altaic}
\define@key{fams}{vaf}{Indo-European}
\define@key{fams}{vag}{Niger-Congo}
\define@key{fams}{vai}{Mande}
\define@key{fams}{vas}{Indo-European}
\define@key{fams}{dic}{Niger-Congo}
\define@key{fams}{ved}{Indo-European}
\define@key{fams}{ven}{Niger-Congo}
\define@key{fams}{vep}{Uralic}
\define@key{fams}{vie}{Austro-Asiatic}
\define@key{fams}{vif}{Niger-Congo}
\define@key{fams}{vnm}{Austronesian}
\define@key{fams}{vgt}{other}
\define@key{fams}{vot}{Uralic}
\define@key{fams}{wwa}{Niger-Congo}
\define@key{fams}{wkw}{Pama-Nyungan}
\define@key{fams}{waq}{Isolate}
\define@key{fams}{waw}{Cariban}
\define@key{fams}{wbk}{Indo-European}
\define@key{fams}{bao}{Tucanoan}
\define@key{fams}{wbl}{Indo-European}
\define@key{fams}{wls}{Austronesian}
\define@key{fams}{van}{Torricelli}
\define@key{fams}{wmt}{Pama-Nyungan}
\define@key{fams}{wmb}{Mirndi}
\define@key{fams}{wms}{Trans-New Guinea}
\define@key{fams}{wme}{Sino-Tibetan}
\define@key{fams}{wan}{Mande}
\define@key{fams}{wgg}{Pama-Nyungan}
\define@key{fams}{xwk}{Pama-Nyungan}
\define@key{fams}{wbt}{Pama-Nyungan}
\define@key{fams}{wnc}{Trans-New Guinea}
\define@key{fams}{auc}{Isolate}
\define@key{fams}{wap}{Arawakan}
\define@key{fams}{wao}{Wappo-Yukian}
\define@key{fams}{wba}{Isolate}
\define@key{fams}{wrz}{Gunwinyguan}
\define@key{fams}{war}{Austronesian}
\define@key{fams}{wrr}{Yangmanic}
\define@key{fams}{gae}{Arawakan}
\define@key{fams}{wsa}{Austronesian}
\define@key{fams}{pav}{Chapacura-Wanham}
\define@key{fams}{wrs}{Border}
\define@key{fams}{wbp}{Pama-Nyungan}
\define@key{fams}{wrb}{Pama-Nyungan}
\define@key{fams}{wnd}{Mangarrayi-Maran}
\define@key{fams}{wrp}{Austronesian}
\define@key{fams}{wgy}{Pama-Nyungan}
\define@key{fams}{gjm}{Pama-Nyungan}
\define@key{fams}{wrg}{Pama-Nyungan}
\define@key{fams}{wwr}{Nyulnyulan}
\define@key{fams}{wrm}{Pama-Nyungan}
\define@key{fams}{was}{Isolate}
\define@key{fams}{wsk}{Trans-New Guinea}
\define@key{fams}{wax}{Lower Sepik-Ramu}
\define@key{fams}{wth}{Pama-Nyungan}
\define@key{fams}{wbv}{Pama-Nyungan}
\define@key{fams}{noa}{Choco}
\define@key{fams}{wau}{Arawakan}
\define@key{fams}{oym}{Tupian}
\define@key{fams}{way}{Cariban}
\define@key{fams}{wed}{Austronesian}
\define@key{fams}{cym}{Indo-European}
\define@key{fams}{xww}{Pama-Nyungan}
\define@key{fams}{wer}{Trans-New Guinea}
\define@key{fams}{mqs}{North Halmaheran}
\define@key{fams}{lex}{Austronesian}
\define@key{fams}{wic}{Caddoan}
\define@key{fams}{mzh}{Matacoan}
\define@key{fams}{wim}{Pama-Nyungan}
\define@key{fams}{wig}{Pama-Nyungan}
\define@key{fams}{yok}{Penutian}
\define@key{fams}{win}{Siouan}
\define@key{fams}{wnw}{Penutian}
\define@key{fams}{wgu}{Pama-Nyungan}
\define@key{fams}{wiy}{Algic}
\define@key{fams}{wob}{Niger-Congo}
\define@key{fams}{wog}{Sepik}
\define@key{fams}{woi}{Greater West Bomberai}
\define@key{fams}{wyu}{Pama-Nyungan}
\define@key{fams}{wal}{Afro-Asiatic}
\define@key{fams}{woe}{Austronesian}
\define@key{fams}{wlo}{Austronesian}
\define@key{fams}{wol}{Niger-Congo}
\define@key{fams}{wmx}{Skou}
\define@key{fams}{wro}{Worrorran}
\define@key{fams}{wuu}{Sino-Tibetan}
\define@key{fams}{wya}{Iroquoian}
\define@key{fams}{wem}{Niger-Congo}
\define@key{fams}{kao}{Mande}
\define@key{fams}{xav}{Macro-Ge}
\define@key{fams}{xer}{Macro-Ge}
\define@key{fams}{xho}{Niger-Congo}
\define@key{fams}{xir}{Arawakan}
\define@key{fams}{xok}{Macro-Ge}
\define@key{fams}{ane}{Austronesian}
\define@key{fams}{yai}{Indo-European}
\define@key{fams}{yad}{Peba-Yaguan}
\define@key{fams}{yag}{Yámana}
\define@key{fams}{yaf}{Niger-Congo}
\define@key{fams}{yka}{Austronesian}
\define@key{fams}{yky}{Niger-Congo}
\define@key{fams}{sah}{Altaic}
\define@key{fams}{ylr}{Pama-Nyungan}
\define@key{fams}{kkl}{Trans-New Guinea}
\define@key{fams}{yli}{Trans-New Guinea}
\define@key{fams}{yam}{Niger-Congo}
\define@key{fams}{jmd}{Austronesian}
\define@key{fams}{tao}{Austronesian}
\define@key{fams}{yaa}{Pano-Tacanan}
\define@key{fams}{ybi}{Sino-Tibetan}
\define@key{fams}{ynn}{Hokan}
\define@key{fams}{kdd}{Pama-Nyungan}
\define@key{fams}{wca}{Yanomam}
\define@key{fams}{yns}{Niger-Congo}
\define@key{fams}{jao}{Pama-Nyungan}
\define@key{fams}{yao}{Niger-Congo}
\define@key{fams}{yap}{Austronesian}
\define@key{fams}{jaq}{Trans-New Guinea}
\define@key{fams}{yaq}{Uto-Aztecan}
\define@key{fams}{yrb}{Trans-New Guinea}
\define@key{fams}{yae}{Isolate}
\define@key{fams}{yuf}{Hokan}
\define@key{fams}{yva}{Isolate}
\define@key{fams}{ywr}{Nyulnyulan}
\define@key{fams}{pcc}{Tai-Kadai}
\define@key{fams}{xya}{Pama-Nyungan}
\define@key{fams}{yah}{Indo-European}
\define@key{fams}{kpv}{Uralic}
\define@key{fams}{jei}{Yam}
\define@key{fams}{jel}{Bulaka River}
\define@key{fams}{yle}{Yele}
\define@key{fams}{ybb}{Niger-Congo}
\define@key{fams}{jnj}{Afro-Asiatic}
\define@key{fams}{yss}{Sepik}
\define@key{fams}{yey}{Niger-Congo}
\define@key{fams}{ywq}{Sino-Tibetan}
\define@key{fams}{ydd}{Indo-European}
\define@key{fams}{yii}{Pama-Nyungan}
\define@key{fams}{yll}{Torricelli}
\define@key{fams}{yee}{Lower Sepik-Ramu}
\define@key{fams}{yij}{Pama-Nyungan}
\define@key{fams}{yia}{Pama-Nyungan}
\define@key{fams}{yyr}{Pama-Nyungan}
\define@key{fams}{xyy}{Pama-Nyungan}
\define@key{fams}{yor}{Niger-Congo}
\define@key{fams}{yua}{Mayan}
\define@key{fams}{yuc}{Isolate}
\define@key{fams}{ycn}{Arawakan}
\define@key{fams}{yug}{Yeniseian}
\define@key{fams}{yux}{Yukaghir}
\define@key{fams}{ykg}{Yukaghir}
\define@key{fams}{yuk}{Wappo-Yukian}
\define@key{fams}{yup}{Cariban}
\define@key{fams}{gcd}{Tangkic}
\define@key{fams}{mpj}{Pama-Nyungan}
\define@key{fams}{yul}{Central Sudanic}
\define@key{fams}{esu}{Eskimo-Aleut}
\define@key{fams}{ynk}{Eskimo-Aleut}
\define@key{fams}{ess}{Eskimo-Aleut}
\define@key{fams}{ysr}{Eskimo-Aleut}
\define@key{fams}{yuz}{Isolate}
\define@key{fams}{yur}{Algic}
\define@key{fams}{yui}{Tucanoan}
\define@key{fams}{zne}{Niger-Congo}
\define@key{fams}{zro}{Zaparoan}
\define@key{fams}{zai}{Oto-Manguean}
\define@key{fams}{zpd}{Oto-Manguean}
\define@key{fams}{zaa}{Oto-Manguean}
\define@key{fams}{zaw}{Oto-Manguean}
\define@key{fams}{zpm}{Oto-Manguean}
\define@key{fams}{zpi}{Oto-Manguean}
\define@key{fams}{zab}{Oto-Manguean}
\define@key{fams}{zpz}{Oto-Manguean}
\define@key{fams}{zav}{Oto-Manguean}
\define@key{fams}{zpq}{Oto-Manguean}
\define@key{fams}{dje}{Songhay}
\define@key{fams}{zay}{Afro-Asiatic}
\define@key{fams}{diq}{Indo-European}
\define@key{fams}{zen}{Afro-Asiatic}
\define@key{fams}{zgb}{Tai-Kadai}
\define@key{fams}{zik}{Trans-New Guinea}
\define@key{fams}{zoh}{Mixe-Zoque}
\define@key{fams}{zos}{Mixe-Zoque}
\define@key{fams}{zoc}{Mixe-Zoque}
\define@key{fams}{zor}{Mixe-Zoque}
\define@key{fams}{zul}{Niger-Congo}
\define@key{fams}{zun}{Isolate}
\define@key{fams}{eme}{Tupian}
\define@key{fams}{aom}{Trans-New Guinea}
\define@key{fams}{aas}{Afro-Asiatic}
\define@key{fams}{kbt}{Austronesian}
\define@key{fams}{abg}{Nuclear Trans New Guinea}
\define@key{fams}{abf}{Austronesian}
\define@key{fams}{abm}{Atlantic-Congo}
\define@key{fams}{mij}{Atlantic-Congo}
\define@key{fams}{aba}{Atlantic-Congo}
\define@key{fams}{abp}{Austronesian}
\define@key{fams}{bsa}{Isolate}
\define@key{fams}{ash}{Isolate}
\define@key{fams}{aob}{Anim}
\define@key{fams}{abo}{Atlantic-Congo}
\define@key{fams}{abr}{Atlantic-Congo}
\define@key{fams}{abn}{Atlantic-Congo}
\define@key{fams}{abu}{Atlantic-Congo}
\define@key{fams}{mgj}{Atlantic-Congo}
\define@key{fams}{ado}{Lower Sepik-Ramu}
\define@key{fams}{tpx}{Otomanguean}
\define@key{fams}{yif}{Sino-Tibetan}
\define@key{fams}{acz}{Narrow Talodi}
\define@key{fams}{acs}{Nuclear-Macro-Je}
\define@key{fams}{xad}{Isolate}
\define@key{fams}{ada}{Atlantic-Congo}
\define@key{fams}{adq}{Atlantic-Congo}
\define@key{fams}{tiu}{Austronesian}
\define@key{fams}{ade}{Atlantic-Congo}
\define@key{fams}{adh}{Nilotic}
\define@key{fams}{gas}{Indo-European}
\define@key{fams}{adr}{Austronesian}
\define@key{fams}{aez}{Nuclear Trans New Guinea}
\define@key{fams}{aeq}{Indo-European}
\define@key{fams}{afg}{Sign Language}
\define@key{fams}{aft}{Nyimang}
\define@key{fams}{afh}{Artificial Language}
\define@key{fams}{afs}{Indo-European}
\define@key{fams}{agi}{Unattested}
\define@key{fams}{agc}{Atlantic-Congo}
\define@key{fams}{avo}{Unattested}
\define@key{fams}{ggr}{Pama-Nyungan}
\define@key{fams}{xag}{Nakh-Daghestanian}
\define@key{fams}{aif}{Nuclear Torricelli}
\define@key{fams}{kit}{Pahoturi}
\define@key{fams}{ibm}{Atlantic-Congo}
\define@key{fams}{apf}{Austronesian}
\define@key{fams}{aga}{Unattested}
\define@key{fams}{aug}{Atlantic-Congo}
\define@key{fams}{msm}{Austronesian}
\define@key{fams}{agn}{Austronesian}
\define@key{fams}{yay}{Atlantic-Congo}
\define@key{fams}{aha}{Atlantic-Congo}
\define@key{fams}{ahn}{Atlantic-Congo}
\define@key{fams}{esg}{Dravidian}
\define@key{fams}{thm}{Austroasiatic}
\define@key{fams}{kak}{Austronesian}
\define@key{fams}{aho}{Tai-Kadai}
\define@key{fams}{nfd}{Atlantic-Congo}
\define@key{fams}{aih}{Tai-Kadai}
\define@key{fams}{aix}{Austronesian}
\define@key{fams}{mwg}{Austronesian}
\define@key{fams}{aiq}{Indo-European}
\define@key{fams}{ail}{Bosavi}
\define@key{fams}{aim}{Sino-Tibetan}
\define@key{fams}{aic}{Border}
\define@key{fams}{aki}{Lower Sepik-Ramu}
\define@key{fams}{air}{Greater Kwerba}
\define@key{fams}{aio}{Tai-Kadai}
\define@key{fams}{ajw}{Afro-Asiatic}
\define@key{fams}{cpc}{Arawakan}
\define@key{fams}{soh}{Eastern Jebel}
\define@key{fams}{akm}{Great Andamanese}
\define@key{fams}{akj}{Great Andamanese}
\define@key{fams}{ack}{Great Andamanese}
\define@key{fams}{aky}{Great Andamanese}
\define@key{fams}{acl}{Great Andamanese}
\define@key{fams}{aks}{Atlantic-Congo}
\define@key{fams}{aik}{Atlantic-Congo}
\define@key{fams}{tsr}{Austronesian}
\define@key{fams}{aeu}{Sino-Tibetan}
\define@key{fams}{sia}{Uralic}
\define@key{fams}{akk}{Afro-Asiatic}
\define@key{fams}{akq}{Sepik}
\define@key{fams}{akt}{Austronesian}
\define@key{fams}{bss}{Atlantic-Congo}
\define@key{fams}{miw}{Angan}
\define@key{fams}{akf}{Atlantic-Congo}
\define@key{fams}{ibe}{Atlantic-Congo}
\define@key{fams}{afi}{Lower Sepik-Ramu}
\define@key{fams}{ayk}{Atlantic-Congo}
\define@key{fams}{aku}{Atlantic-Congo}
\define@key{fams}{aqz}{Tupian}
\define@key{fams}{ako}{Cariban}
\define@key{fams}{dul}{Austronesian}
\define@key{fams}{alw}{Afro-Asiatic}
\define@key{fams}{ala}{Atlantic-Congo}
\define@key{fams}{alk}{Austroasiatic}
\define@key{fams}{alj}{Austronesian}
\define@key{fams}{apv}{Unattested}
\define@key{fams}{bhk}{Austronesian}
\define@key{fams}{sqk}{Sign Language}
\define@key{fams}{lsc}{Sign Language}
\define@key{fams}{xta}{Otomanguean}
\define@key{fams}{alf}{Atlantic-Congo}
\define@key{fams}{asp}{Sign Language}
\define@key{fams}{arq}{Afro-Asiatic}
\define@key{fams}{aao}{Afro-Asiatic}
\define@key{fams}{aiy}{Atlantic-Congo}
\define@key{fams}{all}{Dravidian}
\define@key{fams}{aid}{Pama-Nyungan}
\define@key{fams}{zaq}{Otomanguean}
\define@key{fams}{ypo}{Sino-Tibetan}
\define@key{fams}{aol}{Austronesian}
\define@key{fams}{syy}{Sign Language}
\define@key{fams}{aub}{Sino-Tibetan}
\define@key{fams}{xua}{Dravidian}
\define@key{fams}{aab}{Atlantic-Congo}
\define@key{fams}{yna}{Sino-Tibetan}
\define@key{fams}{alz}{Nilotic}
\define@key{fams}{avd}{Indo-European}
\define@key{fams}{amq}{Austronesian}
\define@key{fams}{ali}{Nuclear Trans New Guinea}
\define@key{fams}{aad}{Sepik}
\define@key{fams}{jks}{Sign Language}
\define@key{fams}{ama}{Tupian}
\define@key{fams}{amg}{Iwaidjan Proper}
\define@key{fams}{aaz}{Austronesian}
\define@key{fams}{zpo}{Otomanguean}
\define@key{fams}{rwm}{Atlantic-Congo}
\define@key{fams}{utp}{Austronesian}
\define@key{fams}{abc}{Austronesian}
\define@key{fams}{aew}{Keram}
\define@key{fams}{ael}{Atlantic-Congo}
\define@key{fams}{amv}{Austronesian}
\define@key{fams}{alm}{Austronesian}
\define@key{fams}{amb}{Atlantic-Congo}
\define@key{fams}{abs}{Austronesian}
\define@key{fams}{qva}{Quechuan}
\define@key{fams}{aag}{Nuclear Torricelli}
\define@key{fams}{amj}{Furan}
\define@key{fams}{ifa}{Austronesian}
\define@key{fams}{alx}{Nuclear Torricelli}
\define@key{fams}{mbz}{Otomanguean}
\define@key{fams}{aqd}{Dogon}
\define@key{fams}{apg}{Austronesian}
\define@key{fams}{ajz}{Sino-Tibetan}
\define@key{fams}{amt}{Amto-Musan}
\define@key{fams}{adw}{Tupian}
\define@key{fams}{anw}{Atlantic-Congo}
\define@key{fams}{akg}{Austronesian}
\define@key{fams}{anm}{Sino-Tibetan}
\define@key{fams}{pda}{Nuclear Trans New Guinea}
\define@key{fams}{aan}{Tupian}
\define@key{fams}{dti}{Dogon}
\define@key{fams}{grc}{Indo-European}
\define@key{fams}{hbo}{Afro-Asiatic}
\define@key{fams}{xna}{Afro-Asiatic}
\define@key{fams}{xlg}{Unclassifiable}
\define@key{fams}{hca}{Indo-European}
\define@key{fams}{afd}{Arafundi}
\define@key{fams}{aod}{Lower Sepik-Ramu}
\define@key{fams}{ana}{Isolate}
\define@key{fams}{xaa}{Afro-Asiatic}
\define@key{fams}{adg}{Pama-Nyungan}
\define@key{fams}{bzb}{Austronesian}
\define@key{fams}{anb}{Zaparoan}
\define@key{fams}{anx}{Austronesian}
\define@key{fams}{aby}{Yareban}
\define@key{fams}{myo}{Ta-Ne-Omotic}
\define@key{fams}{akh}{Nuclear Trans New Guinea}
\define@key{fams}{age}{Nuclear Trans New Guinea}
\define@key{fams}{aoe}{Nuclear Trans New Guinea}
\define@key{fams}{aqt}{Lengua-Mascoy}
\define@key{fams}{avm}{Pama-Nyungan}
\define@key{fams}{anp}{Indo-European}
\define@key{fams}{rme}{Indo-European}
\define@key{fams}{aog}{Lower Sepik-Ramu}
\define@key{fams}{tnd}{Chibchan}
\define@key{fams}{blo}{Atlantic-Congo}
\define@key{fams}{anf}{Atlantic-Congo}
\define@key{fams}{aqk}{Atlantic-Congo}
\define@key{fams}{ypn}{Sino-Tibetan}
\define@key{fams}{boj}{Nuclear Trans New Guinea}
\define@key{fams}{aak}{Angan}
\define@key{fams}{amx}{Pama-Nyungan}
\define@key{fams}{anj}{Lower Sepik-Ramu}
\define@key{fams}{ans}{Chocoan}
\define@key{fams}{and}{Austronesian}
\define@key{fams}{ant}{Pama-Nyungan}
\define@key{fams}{xmv}{Austronesian}
\define@key{fams}{aig}{Indo-European}
\define@key{fams}{aui}{Austronesian}
\define@key{fams}{auq}{Austronesian}
\define@key{fams}{aud}{Austronesian}
\define@key{fams}{anl}{Sino-Tibetan}
\define@key{fams}{mtb}{Atlantic-Congo}
\define@key{fams}{pni}{Austronesian}
\define@key{fams}{aor}{Austronesian}
\define@key{fams}{aou}{Tai-Kadai}
\define@key{fams}{xap}{Muskogean}
\define@key{fams}{apo}{Austronesian}
\define@key{fams}{ena}{Nuclear Trans New Guinea}
\define@key{fams}{mip}{Otomanguean}
\define@key{fams}{api}{Tupian}
\define@key{fams}{app}{Austronesian}
\define@key{fams}{apx}{Austronesian}
\define@key{fams}{arg}{Indo-European}
\define@key{fams}{stk}{Yam}
\define@key{fams}{aaf}{Dravidian}
\define@key{fams}{xrt}{Unclassifiable}
\define@key{fams}{arj}{Tucanoan}
\define@key{fams}{awm}{Nuclear Trans New Guinea}
\define@key{fams}{awt}{Tupian}
\define@key{fams}{aae}{Indo-European}
\define@key{fams}{aea}{Pama-Nyungan}
\define@key{fams}{mwc}{Austronesian}
\define@key{fams}{aem}{Austroasiatic}
\define@key{fams}{qxu}{Quechuan}
\define@key{fams}{agj}{Afro-Asiatic}
\define@key{fams}{agf}{Austronesian}
\define@key{fams}{aqr}{Austronesian}
\define@key{fams}{aok}{Austronesian}
\define@key{fams}{ylu}{Austronesian}
\define@key{fams}{aai}{Austronesian}
\define@key{fams}{aqg}{Atlantic-Congo}
\define@key{fams}{aac}{Suki-Gogodala}
\define@key{fams}{ait}{Tupian}
\define@key{fams}{ark}{Nuclear-Macro-Je}
\define@key{fams}{xrn}{Yeniseian}
\define@key{fams}{luc}{Central Sudanic}
\define@key{fams}{dth}{Pama-Nyungan}
\define@key{fams}{aoh}{Unattested}
\define@key{fams}{aen}{Sign Language}
\define@key{fams}{rup}{Indo-European}
\define@key{fams}{aps}{Austronesian}
\define@key{fams}{atz}{Austronesian}
\define@key{fams}{arx}{Tupian}
\define@key{fams}{aru}{Arawan}
\define@key{fams}{aur}{Nuclear Torricelli}
\define@key{fams}{lsr}{Nuclear Torricelli}
\define@key{fams}{atx}{Isolate}
\define@key{fams}{aat}{Indo-European}
\define@key{fams}{mtv}{Nuclear Trans New Guinea}
\define@key{fams}{cni}{Arawakan}
\define@key{fams}{ahs}{Atlantic-Congo}
\define@key{fams}{prq}{Arawakan}
\define@key{fams}{ask}{Indo-European}
\define@key{fams}{atn}{Indo-European}
\define@key{fams}{asl}{Austronesian}
\define@key{fams}{eiv}{North Bougainville}
\define@key{fams}{asv}{Central Sudanic}
\define@key{fams}{asb}{Siouan}
\define@key{fams}{asz}{Austronesian}
\define@key{fams}{aua}{Austronesian}
\define@key{fams}{aum}{Atlantic-Congo}
\define@key{fams}{zoo}{Otomanguean}
\define@key{fams}{asr}{Austroasiatic}
\define@key{fams}{atm}{Austronesian}
\define@key{fams}{amz}{Pama-Nyungan}
\define@key{fams}{atd}{Austronesian}
\define@key{fams}{ate}{Nuclear Trans New Guinea}
\define@key{fams}{atk}{Austronesian}
\define@key{fams}{aqm}{Kayagaric}
\define@key{fams}{aot}{Sino-Tibetan}
\define@key{fams}{ato}{Atlantic-Congo}
\define@key{fams}{aox}{Arawakan}
\define@key{fams}{cch}{Atlantic-Congo}
\define@key{fams}{atc}{Pano-Tacanan}
\define@key{fams}{pkr}{Dravidian}
\define@key{fams}{ati}{Atlantic-Congo}
\define@key{fams}{kud}{Austronesian}
\define@key{fams}{aux}{Tupian}
\define@key{fams}{auh}{Atlantic-Congo}
\define@key{fams}{avs}{Zaparoan}
\define@key{fams}{asq}{Sign Language}
\define@key{fams}{asw}{Sign Language}
\define@key{fams}{aut}{Austronesian}
\define@key{fams}{smf}{Border}
\define@key{fams}{auu}{Nuclear Trans New Guinea}
\define@key{fams}{auo}{Afro-Asiatic}
\define@key{fams}{avv}{Tupian}
\define@key{fams}{avb}{Austronesian}
\define@key{fams}{ave}{Indo-European}
\define@key{fams}{awk}{Pama-Nyungan}
\define@key{fams}{vwa}{Austroasiatic}
\define@key{fams}{bcu}{Austronesian}
\define@key{fams}{awo}{Atlantic-Congo}
\define@key{fams}{awx}{Nuclear Trans New Guinea}
\define@key{fams}{aya}{Lower Sepik-Ramu}
\define@key{fams}{awh}{Bayono-Awbono}
\define@key{fams}{bob}{Afro-Asiatic}
\define@key{fams}{awr}{Lakes Plain}
\define@key{fams}{awe}{Tupian}
\define@key{fams}{azo}{Atlantic-Congo}
\define@key{fams}{auj}{Afro-Asiatic}
\define@key{fams}{aww}{Sepik}
\define@key{fams}{afu}{Atlantic-Congo}
\define@key{fams}{yiu}{Sino-Tibetan}
\define@key{fams}{ahb}{Austronesian}
\define@key{fams}{yix}{Sino-Tibetan}
\define@key{fams}{ayd}{Pama-Nyungan}
\define@key{fams}{vmy}{Otomanguean}
\define@key{fams}{aye}{Atlantic-Congo}
\define@key{fams}{ayq}{Sepik}
\define@key{fams}{yyz}{Sino-Tibetan}
\define@key{fams}{ayb}{Atlantic-Congo}
\define@key{fams}{zaf}{Otomanguean}
\define@key{fams}{ayu}{Atlantic-Congo}
\define@key{fams}{aza}{Sino-Tibetan}
\define@key{fams}{yiz}{Sino-Tibetan}
\define@key{fams}{tpc}{Otomanguean}
\define@key{fams}{bvj}{Atlantic-Congo}
\define@key{fams}{bqx}{Atlantic-Congo}
\define@key{fams}{bbm}{Atlantic-Congo}
\define@key{fams}{bbw}{Atlantic-Congo}
\define@key{fams}{bbk}{Atlantic-Congo}
\define@key{fams}{mbf}{Austronesian}
\define@key{fams}{bcr}{Athabaskan-Eyak-Tlingit}
\define@key{fams}{bzg}{Austronesian}
\define@key{fams}{btj}{Austronesian}
\define@key{fams}{bcy}{Afro-Asiatic}
\define@key{fams}{xbc}{Indo-European}
\define@key{fams}{bau}{Atlantic-Congo}
\define@key{fams}{bhz}{Austronesian}
\define@key{fams}{bdz}{Unattested}
\define@key{fams}{jbi}{Pama-Nyungan}
\define@key{fams}{bac}{Austronesian}
\define@key{fams}{pbp}{Atlantic-Congo}
\define@key{fams}{bvd}{Austronesian}
\define@key{fams}{bvc}{Austronesian}
\define@key{fams}{btr}{Austronesian}
\define@key{fams}{bwt}{Atlantic-Congo}
\define@key{fams}{bfj}{Atlantic-Congo}
\define@key{fams}{bmd}{Atlantic-Congo}
\define@key{fams}{bgo}{Atlantic-Congo}
\define@key{fams}{bcg}{Atlantic-Congo}
\define@key{fams}{bfy}{Indo-European}
\define@key{fams}{fui}{Atlantic-Congo}
\define@key{fams}{bqg}{Atlantic-Congo}
\define@key{fams}{bqb}{Greater Kwerba}
\define@key{fams}{bpi}{Nuclear Trans New Guinea}
\define@key{fams}{yha}{Tai-Kadai}
\define@key{fams}{bhv}{Austronesian}
\define@key{fams}{bah}{Indo-European}
\define@key{fams}{bhj}{Sino-Tibetan}
\define@key{fams}{bsu}{Austronesian}
\define@key{fams}{bbf}{Baibai-Fas}
\define@key{fams}{bdj}{Atlantic-Congo}
\define@key{fams}{bkx}{Austronesian}
\define@key{fams}{bqh}{Sino-Tibetan}
\define@key{fams}{bmx}{Nuclear Trans New Guinea}
\define@key{fams}{bab}{Atlantic-Congo}
\define@key{fams}{bcz}{Atlantic-Congo}
\define@key{fams}{fah}{Atlantic-Congo}
\define@key{fams}{bjs}{Indo-European}
\define@key{fams}{bjm}{Indo-European}
\define@key{fams}{bqz}{Atlantic-Congo}
\define@key{fams}{bqi}{Indo-European}
\define@key{fams}{bki}{Austronesian}
\define@key{fams}{bkh}{Atlantic-Congo}
\define@key{fams}{kme}{Atlantic-Congo}
\define@key{fams}{bbs}{Atlantic-Congo}
\define@key{fams}{bkr}{Austronesian}
\define@key{fams}{bjw}{Kru}
\define@key{fams}{ble}{Atlantic-Congo}
\define@key{fams}{bjt}{Atlantic-Congo}
\define@key{fams}{bls}{Austronesian}
\define@key{fams}{bdn}{Afro-Asiatic}
\define@key{fams}{bcn}{Atlantic-Congo}
\define@key{fams}{bcp}{Atlantic-Congo}
\define@key{fams}{mhp}{Austronesian}
\define@key{fams}{bgx}{Turkic}
\define@key{fams}{biz}{Atlantic-Congo}
\define@key{fams}{bqo}{Atlantic-Congo}
\define@key{fams}{blq}{Austronesian}
\define@key{fams}{bog}{Sign Language}
\define@key{fams}{bbq}{Atlantic-Congo}
\define@key{fams}{myf}{Blue Nile Mao}
\define@key{fams}{bmo}{Atlantic-Congo}
\define@key{fams}{bce}{Atlantic-Congo}
\define@key{fams}{bqt}{Atlantic-Congo}
\define@key{fams}{bvm}{Atlantic-Congo}
\define@key{fams}{bcf}{Kiwaian}
\define@key{fams}{bmg}{Atlantic-Congo}
\define@key{fams}{bjx}{Austronesian}
\define@key{fams}{byz}{Lower Sepik-Ramu}
\define@key{fams}{bqj}{Atlantic-Congo}
\define@key{fams}{bqk}{Atlantic-Congo}
\define@key{fams}{bpd}{Atlantic-Congo}
\define@key{fams}{bfl}{Atlantic-Congo}
\define@key{fams}{yaj}{Atlantic-Congo}
\define@key{fams}{bpq}{Austronesian}
\define@key{fams}{bnd}{Austronesian}
\define@key{fams}{bbe}{Atlantic-Congo}
\define@key{fams}{bgf}{Atlantic-Congo}
\define@key{fams}{bsj}{Atlantic-Congo}
\define@key{fams}{bnx}{Atlantic-Congo}
\define@key{fams}{bxg}{Atlantic-Congo}
\define@key{fams}{bgj}{Atlantic-Congo}
\define@key{fams}{mfb}{Austronesian}
\define@key{fams}{bjn}{Austronesian}
\define@key{fams}{bfk}{Sign Language}
\define@key{fams}{bxw}{Mande}
\define@key{fams}{dbw}{Dogon}
\define@key{fams}{bap}{Sino-Tibetan}
\define@key{fams}{bno}{Austronesian}
\define@key{fams}{bfx}{Austronesian}
\define@key{fams}{brd}{Sino-Tibetan}
\define@key{fams}{bbg}{Atlantic-Congo}
\define@key{fams}{baj}{Austronesian}
\define@key{fams}{bhr}{Austronesian}
\define@key{fams}{brs}{Austronesian}
\define@key{fams}{brp}{Geelvink Bay}
\define@key{fams}{bmz}{Anim}
\define@key{fams}{bpb}{Unattested}
\define@key{fams}{gry}{Kru}
\define@key{fams}{bva}{Afro-Asiatic}
\define@key{fams}{bxo}{Pidgin}
\define@key{fams}{bch}{Austronesian}
\define@key{fams}{bjc}{Yareban}
\define@key{fams}{jbk}{Turama-Kikori}
\define@key{fams}{bbi}{Atlantic-Congo}
\define@key{fams}{bjk}{Austronesian}
\define@key{fams}{bpt}{Pama-Nyungan}
\define@key{fams}{tbn}{Chibchan}
\define@key{fams}{bjz}{Nuclear Trans New Guinea}
\define@key{fams}{bwg}{Atlantic-Congo}
\define@key{fams}{bjf}{Afro-Asiatic}
\define@key{fams}{bsl}{Atlantic-Congo}
\define@key{fams}{buj}{Atlantic-Congo}
\define@key{fams}{bzw}{Atlantic-Congo}
\define@key{fams}{bdb}{Austronesian}
\define@key{fams}{byq}{Austronesian}
\define@key{fams}{bsg}{Indo-European}
\define@key{fams}{bst}{Ta-Ne-Omotic}
\define@key{fams}{bsr}{Atlantic-Congo}
\define@key{fams}{bsi}{Atlantic-Congo}
\define@key{fams}{bnm}{Atlantic-Congo}
\define@key{fams}{bts}{Austronesian}
\define@key{fams}{akb}{Austronesian}
\define@key{fams}{btm}{Austronesian}
\define@key{fams}{btd}{Austronesian}
\define@key{fams}{ayt}{Austronesian}
\define@key{fams}{bta}{Afro-Asiatic}
\define@key{fams}{btv}{Indo-European}
\define@key{fams}{btq}{Austroasiatic}
\define@key{fams}{btc}{Atlantic-Congo}
\define@key{fams}{bvt}{Austronesian}
\define@key{fams}{btu}{Atlantic-Congo}
\define@key{fams}{bay}{Austronesian}
\define@key{fams}{zbt}{Austronesian}
\define@key{fams}{sne}{Austronesian}
\define@key{fams}{bsf}{Atlantic-Congo}
\define@key{fams}{bge}{Indo-European}
\define@key{fams}{bxa}{Austronesian}
\define@key{fams}{bwk}{Mailuan}
\define@key{fams}{bjy}{Pama-Nyungan}
\define@key{fams}{bvy}{Austronesian}
\define@key{fams}{byg}{Dajuic}
\define@key{fams}{mkq}{Miwok-Costanoan}
\define@key{fams}{bda}{Atlantic-Congo}
\define@key{fams}{byl}{Bayono-Awbono}
\define@key{fams}{bfr}{Unclassifiable}
\define@key{fams}{beo}{Bosavi}
\define@key{fams}{bea}{Athabaskan-Eyak-Tlingit}
\define@key{fams}{bfp}{Atlantic-Congo}
\define@key{fams}{beb}{Atlantic-Congo}
\define@key{fams}{bzv}{Atlantic-Congo}
\define@key{fams}{bek}{Austronesian}
\define@key{fams}{bxp}{Atlantic-Congo}
\define@key{fams}{tnr}{Atlantic-Congo}
\define@key{fams}{bjv}{Central Sudanic}
\define@key{fams}{bed}{Austronesian}
\define@key{fams}{bkf}{Atlantic-Congo}
\define@key{fams}{bxq}{Afro-Asiatic}
\define@key{fams}{bnz}{Atlantic-Congo}
\define@key{fams}{bby}{Atlantic-Congo}
\define@key{fams}{bqv}{Atlantic-Congo}
\define@key{fams}{bei}{Austronesian}
\define@key{fams}{bkv}{Atlantic-Congo}
\define@key{fams}{bkw}{Atlantic-Congo}
\define@key{fams}{bvi}{Atlantic-Congo}
\define@key{fams}{bxb}{Nilotic}
\define@key{fams}{beg}{Austronesian}
\define@key{fams}{blm}{Central Sudanic}
\define@key{fams}{bey}{Nuclear Torricelli}
\define@key{fams}{bzj}{Indo-European}
\define@key{fams}{brw}{Dravidian}
\define@key{fams}{glb}{Afro-Asiatic}
\define@key{fams}{bmb}{Atlantic-Congo}
\define@key{fams}{yun}{Atlantic-Congo}
\define@key{fams}{bez}{Atlantic-Congo}
\define@key{fams}{bdp}{Atlantic-Congo}
\define@key{fams}{bct}{Central Sudanic}
\define@key{fams}{bgy}{Austronesian}
\define@key{fams}{bnu}{Austronesian}
\define@key{fams}{dbt}{Dogon}
\define@key{fams}{byd}{Austronesian}
\define@key{fams}{bie}{Nuclear Trans New Guinea}
\define@key{fams}{bxv}{Central Sudanic}
\define@key{fams}{bve}{Austronesian}
\define@key{fams}{bit}{Sepik}
\define@key{fams}{byt}{Saharan}
\define@key{fams}{bes}{Atlantic-Congo}
\define@key{fams}{bep}{Austronesian}
\define@key{fams}{bfe}{Tor-Orya}
\define@key{fams}{byf}{Atlantic-Congo}
\define@key{fams}{btt}{Atlantic-Congo}
\define@key{fams}{eot}{Atlantic-Congo}
\define@key{fams}{bhd}{Indo-European}
\define@key{fams}{bha}{Indo-European}
\define@key{fams}{bht}{Indo-European}
\define@key{fams}{bgw}{Indo-European}
\define@key{fams}{bhe}{Indo-European}
\define@key{fams}{bhy}{Atlantic-Congo}
\define@key{fams}{bhi}{Indo-European}
\define@key{fams}{nes}{Sino-Tibetan}
\define@key{fams}{bhu}{Indo-European}
\define@key{fams}{bdf}{Koiarian}
\define@key{fams}{beh}{Atlantic-Congo}
\define@key{fams}{bpv}{Anim}
\define@key{fams}{big}{Goilalan}
\define@key{fams}{byk}{Tai-Kadai}
\define@key{fams}{bje}{Hmong-Mien}
\define@key{fams}{bmt}{Hmong-Mien}
\define@key{fams}{bym}{Pama-Nyungan}
\define@key{fams}{bjg}{Atlantic-Congo}
\define@key{fams}{bmc}{Austronesian}
\define@key{fams}{bnk}{Austronesian}
\define@key{fams}{brj}{Austronesian}
\define@key{fams}{biu}{Sino-Tibetan}
\define@key{fams}{xbe}{Pama-Nyungan}
\define@key{fams}{bhc}{Austronesian}
\define@key{fams}{ibh}{Austronesian}
\define@key{fams}{jbm}{Atlantic-Congo}
\define@key{fams}{bix}{Austroasiatic}
\define@key{fams}{byb}{Atlantic-Congo}
\define@key{fams}{kfs}{Indo-European}
\define@key{fams}{bql}{Nuclear Trans New Guinea}
\define@key{fams}{brz}{Austronesian}
\define@key{fams}{bpz}{Austronesian}
\define@key{fams}{bil}{Atlantic-Congo}
\define@key{fams}{bms}{Saharan}
\define@key{fams}{bxf}{Austronesian}
\define@key{fams}{bhl}{Nuclear Trans New Guinea}
\define@key{fams}{byj}{Atlantic-Congo}
\define@key{fams}{bmn}{Austronesian}
\define@key{fams}{bxz}{Mailuan}
\define@key{fams}{bon}{Eastern Trans-Fly}
\define@key{fams}{bpj}{Atlantic-Congo}
\define@key{fams}{itb}{Austronesian}
\define@key{fams}{bne}{Austronesian}
\define@key{fams}{bny}{Austronesian}
\define@key{fams}{biq}{Austronesian}
\define@key{fams}{bxe}{Isolate}
\define@key{fams}{brr}{Austronesian}
\define@key{fams}{btf}{Afro-Asiatic}
\define@key{fams}{biy}{Austroasiatic}
\define@key{fams}{bqq}{Lakes Plain}
\define@key{fams}{brk}{Nubian}
\define@key{fams}{brl}{Atlantic-Congo}
\define@key{fams}{ije}{Ijoid}
\define@key{fams}{bpy}{Indo-European}
\define@key{fams}{bwh}{Atlantic-Congo}
\define@key{fams}{bnw}{Sepik}
\define@key{fams}{bir}{Nuclear Trans New Guinea}
\define@key{fams}{bzi}{Sino-Tibetan}
\define@key{fams}{brt}{Atlantic-Congo}
\define@key{fams}{bgk}{Austroasiatic}
\define@key{fams}{mcc}{Anim}
\define@key{fams}{bwm}{Yuat}
\define@key{fams}{byo}{Sino-Tibetan}
\define@key{fams}{bpm}{Nuclear Trans New Guinea}
\define@key{fams}{blp}{Austronesian}
\define@key{fams}{bfh}{Yam}
\define@key{fams}{beu}{Timor-Alor-Pantar}
\define@key{fams}{blr}{Austroasiatic}
\define@key{fams}{zbl}{Artificial Language}
\define@key{fams}{bzn}{Austronesian}
\define@key{fams}{bzl}{Austronesian}
\define@key{fams}{bty}{Austronesian}
\define@key{fams}{bgb}{Austronesian}
\define@key{fams}{bdv}{Indo-European}
\define@key{fams}{boy}{Atlantic-Congo}
\define@key{fams}{bff}{Atlantic-Congo}
\define@key{fams}{boq}{Isolate}
\define@key{fams}{bvw}{Afro-Asiatic}
\define@key{fams}{bux}{Afro-Asiatic}
\define@key{fams}{bqu}{Atlantic-Congo}
\define@key{fams}{bhn}{Afro-Asiatic}
\define@key{fams}{ybk}{Sino-Tibetan}
\define@key{fams}{bdt}{Atlantic-Congo}
\define@key{fams}{bkp}{Atlantic-Congo}
\define@key{fams}{bus}{Mande}
\define@key{fams}{bky}{Atlantic-Congo}
\define@key{fams}{bnp}{Austronesian}
\define@key{fams}{bld}{Austronesian}
\define@key{fams}{xbo}{Turkic}
\define@key{fams}{bvo}{Atlantic-Congo}
\define@key{fams}{bvl}{Sign Language}
\define@key{fams}{smk}{Austronesian}
\define@key{fams}{blv}{Atlantic-Congo}
\define@key{fams}{bkt}{Atlantic-Congo}
\define@key{fams}{bzm}{Atlantic-Congo}
\define@key{fams}{bof}{Mande}
\define@key{fams}{blj}{Austronesian}
\define@key{fams}{ply}{Austroasiatic}
\define@key{fams}{boh}{Atlantic-Congo}
\define@key{fams}{bml}{Atlantic-Congo}
\define@key{fams}{bws}{Atlantic-Congo}
\define@key{fams}{zmx}{Atlantic-Congo}
\define@key{fams}{bmf}{Atlantic-Congo}
\define@key{fams}{bmq}{Atlantic-Congo}
\define@key{fams}{bmw}{Atlantic-Congo}
\define@key{fams}{kzc}{Atlantic-Congo}
\define@key{fams}{bou}{Atlantic-Congo}
\define@key{fams}{dbu}{Dogon}
\define@key{fams}{bna}{Austronesian}
\define@key{fams}{bnv}{Tor-Orya}
\define@key{fams}{glc}{Atlantic-Congo}
\define@key{fams}{bui}{Atlantic-Congo}
\define@key{fams}{bpg}{Austronesian}
\define@key{fams}{bok}{Atlantic-Congo}
\define@key{fams}{bvg}{Atlantic-Congo}
\define@key{fams}{bop}{Nuclear Trans New Guinea}
\define@key{fams}{bnb}{Austronesian}
\define@key{fams}{bnl}{Afro-Asiatic}
\define@key{fams}{bvf}{Afro-Asiatic}
\define@key{fams}{bpw}{Left May}
\define@key{fams}{gai}{Lower Sepik-Ramu}
\define@key{fams}{fue}{Atlantic-Congo}
\define@key{fams}{ksr}{Nuclear Trans New Guinea}
\define@key{fams}{xxb}{Atlantic-Congo}
\define@key{fams}{mae}{Atlantic-Congo}
\define@key{fams}{bwf}{Austronesian}
\define@key{fams}{bqs}{Lower Sepik-Ramu}
\define@key{fams}{bmj}{Indo-European}
\define@key{fams}{bph}{Nakh-Daghestanian}
\define@key{fams}{sbl}{Austronesian}
\define@key{fams}{nku}{Atlantic-Congo}
\define@key{fams}{mux}{Nuclear Trans New Guinea}
\define@key{fams}{suo}{Sko}
\define@key{fams}{kxr}{Austronesian}
\define@key{fams}{aof}{Nuclear Torricelli}
\define@key{fams}{bra}{Indo-European}
\define@key{fams}{kvl}{Sino-Tibetan}
\define@key{fams}{buq}{Nuclear Trans New Guinea}
\define@key{fams}{brq}{Lower Sepik-Ramu}
\define@key{fams}{rib}{Sign Language}
\define@key{fams}{bzt}{Artificial Language}
\define@key{fams}{sgt}{Sino-Tibetan}
\define@key{fams}{bro}{Sino-Tibetan}
\define@key{fams}{bpl}{Pidgin}
\define@key{fams}{plw}{Austronesian}
\define@key{fams}{kxd}{Austronesian}
\define@key{fams}{bsb}{Austronesian}
\define@key{fams}{rnb}{Sign Language}
\define@key{fams}{bub}{Atlantic-Congo}
\define@key{fams}{cbl}{Sino-Tibetan}
\define@key{fams}{box}{Atlantic-Congo}
\define@key{fams}{buw}{Atlantic-Congo}
\define@key{fams}{stt}{Austroasiatic}
\define@key{fams}{btp}{Austronesian}
\define@key{fams}{bdx}{Austronesian}
\define@key{fams}{bja}{Atlantic-Congo}
\define@key{fams}{bbh}{Austroasiatic}
\define@key{fams}{buk}{Austronesian}
\define@key{fams}{bgt}{Austronesian}
\define@key{fams}{bku}{Austronesian}
\define@key{fams}{bxh}{Austronesian}
\define@key{fams}{byh}{Sino-Tibetan}
\define@key{fams}{bvk}{Austronesian}
\define@key{fams}{bhh}{Indo-European}
\define@key{fams}{bvu}{Austronesian}
\define@key{fams}{bkn}{Austronesian}
\define@key{fams}{tkb}{Indo-European}
\define@key{fams}{buz}{Atlantic-Congo}
\define@key{fams}{bqn}{Sign Language}
\define@key{fams}{bmp}{Nuclear Trans New Guinea}
\define@key{fams}{buy}{Atlantic-Congo}
\define@key{fams}{sti}{Austroasiatic}
\define@key{fams}{bjl}{Austronesian}
\define@key{fams}{byp}{Atlantic-Congo}
\define@key{fams}{aon}{Nuclear Torricelli}
\define@key{fams}{bmv}{Atlantic-Congo}
\define@key{fams}{kjz}{Sino-Tibetan}
\define@key{fams}{bwx}{Hmong-Mien}
\define@key{fams}{bdd}{Austronesian}
\define@key{fams}{bvn}{Nuclear Torricelli}
\define@key{fams}{bfn}{Timor-Alor-Pantar}
\define@key{fams}{bns}{Indo-European}
\define@key{fams}{bqd}{Atlantic-Congo}
\define@key{fams}{xbg}{Pama-Nyungan}
\define@key{fams}{wun}{Atlantic-Congo}
\define@key{fams}{bkz}{Austronesian}
\define@key{fams}{but}{Nuclear Torricelli}
\define@key{fams}{buv}{Yuat}
\define@key{fams}{dgb}{Dogon}
\define@key{fams}{bnn}{Austronesian}
\define@key{fams}{blf}{Austronesian}
\define@key{fams}{bys}{Atlantic-Congo}
\define@key{fams}{bti}{Geelvink Bay}
\define@key{fams}{bxn}{Pama-Nyungan}
\define@key{fams}{bvh}{Afro-Asiatic}
\define@key{fams}{pyx}{Sino-Tibetan}
\define@key{fams}{vrt}{Austronesian}
\define@key{fams}{bzu}{Isolate}
\define@key{fams}{bqw}{Atlantic-Congo}
\define@key{fams}{bdi}{Nilotic}
\define@key{fams}{bqr}{Austronesian}
\define@key{fams}{aip}{Nuclear Trans New Guinea}
\define@key{fams}{asi}{Nuclear Trans New Guinea}
\define@key{fams}{bry}{Ndu}
\define@key{fams}{bxs}{Atlantic-Congo}
\define@key{fams}{bsm}{Austronesian}
\define@key{fams}{bfg}{Austronesian}
\define@key{fams}{buc}{Austronesian}
\define@key{fams}{bup}{Austronesian}
\define@key{fams}{dox}{Afro-Asiatic}
\define@key{fams}{bju}{Atlantic-Congo}
\define@key{fams}{kyb}{Austronesian}
\define@key{fams}{bnr}{Austronesian}
\define@key{fams}{btw}{Austronesian}
\define@key{fams}{jid}{Atlantic-Congo}
\define@key{fams}{bhs}{Afro-Asiatic}
\define@key{fams}{jiy}{Sino-Tibetan}
\define@key{fams}{byi}{Atlantic-Congo}
\define@key{fams}{bww}{Atlantic-Congo}
\define@key{fams}{bwd}{Austronesian}
\define@key{fams}{tte}{Austronesian}
\define@key{fams}{bwa}{Austronesian}
\define@key{fams}{bwl}{Atlantic-Congo}
\define@key{fams}{bwc}{Atlantic-Congo}
\define@key{fams}{bwz}{Atlantic-Congo}
\define@key{fams}{mkk}{Atlantic-Congo}
\define@key{fams}{msq}{Austronesian}
\define@key{fams}{cbb}{Arawakan}
\define@key{fams}{ccr}{Misumalpan}
\define@key{fams}{miu}{Otomanguean}
\define@key{fams}{roc}{Austronesian}
\define@key{fams}{ccd}{Indo-European}
\define@key{fams}{cah}{Zaparoan}
\define@key{fams}{qvl}{Quechuan}
\define@key{fams}{zad}{Otomanguean}
\define@key{fams}{frc}{Indo-European}
\define@key{fams}{ckx}{Atlantic-Congo}
\define@key{fams}{ckz}{Mixed Language}
\define@key{fams}{cky}{Afro-Asiatic}
\define@key{fams}{tbk}{Austronesian}
\define@key{fams}{qud}{Quechuan}
\define@key{fams}{caw}{Speech Register}
\define@key{fams}{rmq}{Indo-European}
\define@key{fams}{clu}{Austronesian}
\define@key{fams}{abd}{Austronesian}
\define@key{fams}{csx}{Sign Language}
\define@key{fams}{mcu}{Atlantic-Congo}
\define@key{fams}{wes}{Indo-European}
\define@key{fams}{cml}{Austronesian}
\define@key{fams}{cmt}{Speech Register}
\define@key{fams}{xcc}{Unclassifiable}
\define@key{fams}{qxr}{Quechuan}
\define@key{fams}{caz}{Isolate}
\define@key{fams}{mlc}{Tai-Kadai}
\define@key{fams}{cov}{Tai-Kadai}
\define@key{fams}{cps}{Austronesian}
\define@key{fams}{cpg}{Indo-European}
\define@key{fams}{cot}{Arawakan}
\define@key{fams}{cby}{Unclassifiable}
\define@key{fams}{cfd}{Atlantic-Congo}
\define@key{fams}{crf}{Chocoan}
\define@key{fams}{xcr}{Indo-European}
\define@key{fams}{hns}{Indo-European}
\define@key{fams}{jvn}{Austronesian}
\define@key{fams}{crr}{Algic}
\define@key{fams}{rmc}{Indo-European}
\define@key{fams}{asc}{Nuclear Trans New Guinea}
\define@key{fams}{csc}{Sign Language}
\define@key{fams}{xcy}{Isolate}
\define@key{fams}{xce}{Indo-European}
\define@key{fams}{cen}{Atlantic-Congo}
\define@key{fams}{hmm}{Hmong-Mien}
\define@key{fams}{cmo}{Austroasiatic}
\define@key{fams}{zch}{Tai-Kadai}
\define@key{fams}{hmc}{Hmong-Mien}
\define@key{fams}{fuq}{Atlantic-Congo}
\define@key{fams}{grv}{Kru}
\define@key{fams}{cet}{Isolate}
\define@key{fams}{pse}{Austronesian}
\define@key{fams}{mwo}{Austronesian}
\define@key{fams}{mxz}{Austronesian}
\define@key{fams}{syb}{Austronesian}
\define@key{fams}{tgt}{Austronesian}
\define@key{fams}{plc}{Austronesian}
\define@key{fams}{sml}{Austronesian}
\define@key{fams}{zbc}{Austronesian}
\define@key{fams}{dtp}{Austronesian}
\define@key{fams}{awu}{Nuclear Trans New Guinea}
\define@key{fams}{ncx}{Uto-Aztecan}
\define@key{fams}{nch}{Uto-Aztecan}
\define@key{fams}{ojc}{Algic}
\define@key{fams}{pbs}{Otomanguean}
\define@key{fams}{quk}{Quechuan}
\define@key{fams}{cds}{Sign Language}
\define@key{fams}{cdy}{Tai-Kadai}
\define@key{fams}{chg}{Turkic}
\define@key{fams}{ciy}{Cariban}
\define@key{fams}{ccp}{Indo-European}
\define@key{fams}{ckh}{Sino-Tibetan}
\define@key{fams}{cli}{Atlantic-Congo}
\define@key{fams}{tgf}{Sino-Tibetan}
\define@key{fams}{cll}{Atlantic-Congo}
\define@key{fams}{cdh}{Indo-European}
\define@key{fams}{ceg}{Zamucoan}
\define@key{fams}{ccc}{Arawakan}
\define@key{fams}{cna}{Sino-Tibetan}
\define@key{fams}{cga}{Yuat}
\define@key{fams}{cra}{Ta-Ne-Omotic}
\define@key{fams}{crv}{Austroasiatic}
\define@key{fams}{xtb}{Otomanguean}
\define@key{fams}{ruk}{Atlantic-Congo}
\define@key{fams}{cde}{Dravidian}
\define@key{fams}{cjn}{Sepik}
\define@key{fams}{cnu}{Afro-Asiatic}
\define@key{fams}{ycp}{Sino-Tibetan}
\define@key{fams}{cpn}{Atlantic-Congo}
\define@key{fams}{ych}{Sino-Tibetan}
\define@key{fams}{cwg}{Austroasiatic}
\define@key{fams}{hne}{Indo-European}
\define@key{fams}{ctn}{Sino-Tibetan}
\define@key{fams}{cur}{Sino-Tibetan}
\define@key{fams}{csd}{Sign Language}
\define@key{fams}{cip}{Otomanguean}
\define@key{fams}{zpv}{Otomanguean}
\define@key{fams}{mii}{Otomanguean}
\define@key{fams}{csg}{Sign Language}
\define@key{fams}{clh}{Indo-European}
\define@key{fams}{clc}{Athabaskan-Eyak-Tlingit}
\define@key{fams}{csa}{Otomanguean}
\define@key{fams}{cpi}{Pidgin}
\define@key{fams}{chn}{Chinookan}
\define@key{fams}{cih}{Indo-European}
\define@key{fams}{bxu}{Mongolic-Khitan}
\define@key{fams}{cnb}{Sino-Tibetan}
\define@key{fams}{qxc}{Quechuan}
\define@key{fams}{cdf}{Sino-Tibetan}
\define@key{fams}{nhd}{Tupian}
\define@key{fams}{the}{Indo-European}
\define@key{fams}{cik}{Sino-Tibetan}
\define@key{fams}{zpc}{Otomanguean}
\define@key{fams}{cgk}{Sino-Tibetan}
\define@key{fams}{cdi}{Indo-European}
\define@key{fams}{nri}{Sino-Tibetan}
\define@key{fams}{cjk}{Atlantic-Congo}
\define@key{fams}{cda}{Sino-Tibetan}
\define@key{fams}{coh}{Atlantic-Congo}
\define@key{fams}{cce}{Atlantic-Congo}
\define@key{fams}{nct}{Sino-Tibetan}
\define@key{fams}{cvg}{Sino-Tibetan}
\define@key{fams}{cuw}{Sino-Tibetan}
\define@key{fams}{cuh}{Atlantic-Congo}
\define@key{fams}{chu}{Indo-European}
\define@key{fams}{cdj}{Indo-European}
\define@key{fams}{scb}{Austroasiatic}
\define@key{fams}{xcv}{Yukaghir}
\define@key{fams}{chw}{Atlantic-Congo}
\define@key{fams}{cia}{Austronesian}
\define@key{fams}{ckl}{Afro-Asiatic}
\define@key{fams}{awc}{Atlantic-Congo}
\define@key{fams}{cib}{Atlantic-Congo}
\define@key{fams}{cim}{Indo-European}
\define@key{fams}{mkx}{Austronesian}
\define@key{fams}{cdr}{Atlantic-Congo}
\define@key{fams}{cie}{Afro-Asiatic}
\define@key{fams}{cin}{Tupian}
\define@key{fams}{xcg}{Indo-European}
\define@key{fams}{asg}{Atlantic-Congo}
\define@key{fams}{txt}{Nuclear Trans New Guinea}
\define@key{fams}{tgd}{Afro-Asiatic}
\define@key{fams}{xcl}{Indo-European}
\define@key{fams}{nci}{Uto-Aztecan}
\define@key{fams}{qwc}{Quechuan}
\define@key{fams}{syc}{Afro-Asiatic}
\define@key{fams}{myz}{Afro-Asiatic}
\define@key{fams}{xct}{Sino-Tibetan}
\define@key{fams}{dri}{Atlantic-Congo}
\define@key{fams}{naz}{Uto-Aztecan}
\define@key{fams}{zps}{Otomanguean}
\define@key{fams}{zca}{Otomanguean}
\define@key{fams}{coj}{Cochimi-Yuman}
\define@key{fams}{coa}{Austronesian}
\define@key{fams}{liw}{Austronesian}
\define@key{fams}{csn}{Sign Language}
\define@key{fams}{gct}{Indo-European}
\define@key{fams}{cfg}{Atlantic-Congo}
\define@key{fams}{swc}{Atlantic-Congo}
\define@key{fams}{cnc}{Sino-Tibetan}
\define@key{fams}{coq}{Athabaskan-Eyak-Tlingit}
\define@key{fams}{cry}{Atlantic-Congo}
\define@key{fams}{qwa}{Quechuan}
\define@key{fams}{xxr}{Nuclear-Macro-Je}
\define@key{fams}{cos}{Indo-European}
\define@key{fams}{csr}{Sign Language}
\define@key{fams}{mta}{Austronesian}
\define@key{fams}{xcn}{Isolate}
\define@key{fams}{cow}{Salishan}
\define@key{fams}{toc}{Totonacan}
\define@key{fams}{gyn}{Indo-European}
\define@key{fams}{csq}{Sign Language}
\define@key{fams}{mfn}{Atlantic-Congo}
\define@key{fams}{crz}{Chumashan}
\define@key{fams}{csf}{Sign Language}
\define@key{fams}{cbq}{Atlantic-Congo}
\define@key{fams}{cuo}{Cariban}
\define@key{fams}{xlu}{Indo-European}
\define@key{fams}{cnq}{Atlantic-Congo}
\define@key{fams}{cuq}{Tai-Kadai}
\define@key{fams}{ccl}{Atlantic-Congo}
\define@key{fams}{cuv}{Afro-Asiatic}
\define@key{fams}{xtu}{Otomanguean}
\define@key{fams}{cyo}{Austronesian}
\define@key{fams}{bwy}{Atlantic-Congo}
\define@key{fams}{cse}{Sign Language}
\define@key{fams}{dao}{Sino-Tibetan}
\define@key{fams}{lni}{South Bougainville}
\define@key{fams}{dtn}{Gumuz}
\define@key{fams}{dbr}{Afro-Asiatic}
\define@key{fams}{dbe}{Tor-Orya}
\define@key{fams}{xdc}{Indo-European}
\define@key{fams}{dbd}{Atlantic-Congo}
\define@key{fams}{dgd}{Atlantic-Congo}
\define@key{fams}{dgk}{Central Sudanic}
\define@key{fams}{dec}{Narrow Talodi}
\define@key{fams}{dgn}{Yangmanic}
\define@key{fams}{dlk}{Afro-Asiatic}
\define@key{fams}{das}{Kru}
\define@key{fams}{dij}{Austronesian}
\define@key{fams}{drb}{Nubian}
\define@key{fams}{zhd}{Tai-Kadai}
\define@key{fams}{bpa}{Austronesian}
\define@key{fams}{dkk}{Austronesian}
\define@key{fams}{dka}{Sino-Tibetan}
\define@key{fams}{qer}{Indo-European}
\define@key{fams}{dlm}{Indo-European}
\define@key{fams}{dmm}{Atlantic-Congo}
\define@key{fams}{dam}{Atlantic-Congo}
\define@key{fams}{uhn}{Isolate}
\define@key{fams}{idb}{Indo-European}
\define@key{fams}{dac}{Austronesian}
\define@key{fams}{dml}{Indo-European}
\define@key{fams}{dms}{Austronesian}
\define@key{fams}{dnu}{Austroasiatic}
\define@key{fams}{dnr}{Nuclear Trans New Guinea}
\define@key{fams}{daq}{Dravidian}
\define@key{fams}{thl}{Indo-European}
\define@key{fams}{dsl}{Sign Language}
\define@key{fams}{daf}{Mande}
\define@key{fams}{aso}{Nuclear Trans New Guinea}
\define@key{fams}{gku}{Tuu}
\define@key{fams}{dnd}{Border}
\define@key{fams}{daz}{Nuclear Trans New Guinea}
\define@key{fams}{djc}{Dajuic}
\define@key{fams}{dln}{Sino-Tibetan}
\define@key{fams}{dro}{Austronesian}
\define@key{fams}{dot}{Afro-Asiatic}
\define@key{fams}{daw}{Austronesian}
\define@key{fams}{dww}{Austronesian}
\define@key{fams}{ddw}{Austronesian}
\define@key{fams}{dax}{Pama-Nyungan}
\define@key{fams}{dzg}{Saharan}
\define@key{fams}{dzd}{Unattested}
\define@key{fams}{ded}{Nuclear Trans New Guinea}
\define@key{fams}{gbh}{Atlantic-Congo}
\define@key{fams}{dge}{Nuclear Trans New Guinea}
\define@key{fams}{mzw}{Atlantic-Congo}
\define@key{fams}{deh}{Indo-European}
\define@key{fams}{dek}{Unattested}
\define@key{fams}{row}{Austronesian}
\define@key{fams}{ntr}{Atlantic-Congo}
\define@key{fams}{dmx}{Atlantic-Congo}
\define@key{fams}{dei}{Geelvink Bay}
\define@key{fams}{dem}{Isolate}
\define@key{fams}{dmy}{Sentanic}
\define@key{fams}{deq}{Atlantic-Congo}
\define@key{fams}{ddn}{Songhay}
\define@key{fams}{dez}{Atlantic-Congo}
\define@key{fams}{dnk}{Austronesian}
\define@key{fams}{dbb}{Afro-Asiatic}
\define@key{fams}{anv}{Atlantic-Congo}
\define@key{fams}{dee}{Kru}
\define@key{fams}{def}{Indo-European}
\define@key{fams}{dgh}{Afro-Asiatic}
\define@key{fams}{dhs}{Atlantic-Congo}
\define@key{fams}{dhn}{Indo-European}
\define@key{fams}{dwz}{Indo-European}
\define@key{fams}{nfa}{Austronesian}
\define@key{fams}{mki}{Indo-European}
\define@key{fams}{dho}{Indo-European}
\define@key{fams}{adf}{Afro-Asiatic}
\define@key{fams}{ddr}{Pama-Nyungan}
\define@key{fams}{dhd}{Indo-European}
\define@key{fams}{dia}{Nuclear Torricelli}
\define@key{fams}{mbd}{Austronesian}
\define@key{fams}{dby}{Isolate}
\define@key{fams}{dio}{Atlantic-Congo}
\define@key{fams}{duy}{Austronesian}
\define@key{fams}{dig}{Atlantic-Congo}
\define@key{fams}{cfa}{Atlantic-Congo}
\define@key{fams}{dil}{Nubian}
\define@key{fams}{jma}{Dagan}
\define@key{fams}{dii}{Atlantic-Congo}
\define@key{fams}{dmc}{Nuclear Trans New Guinea}
\define@key{fams}{ddi}{Austronesian}
\define@key{fams}{gdl}{Afro-Asiatic}
\define@key{fams}{diu}{Atlantic-Congo}
\define@key{fams}{dir}{Atlantic-Congo}
\define@key{fams}{dwa}{Afro-Asiatic}
\define@key{fams}{dsi}{Central Sudanic}
\define@key{fams}{tbz}{Atlantic-Congo}
\define@key{fams}{diy}{Nuclear Trans New Guinea}
\define@key{fams}{xtd}{Otomanguean}
\define@key{fams}{dix}{Austronesian}
\define@key{fams}{djf}{Pama-Nyungan}
\define@key{fams}{djn}{Gunwinyguan}
\define@key{fams}{djw}{Nyulnyulan}
\define@key{fams}{djb}{Pama-Nyungan}
\define@key{fams}{dze}{Pama-Nyungan}
\define@key{fams}{dob}{Austronesian}
\define@key{fams}{doe}{Atlantic-Congo}
\define@key{fams}{dgg}{Austronesian}
\define@key{fams}{dgx}{Nuclear Trans New Guinea}
\define@key{fams}{dgs}{Atlantic-Congo}
\define@key{fams}{dos}{Atlantic-Congo}
\define@key{fams}{dgr}{Athabaskan-Eyak-Tlingit}
\define@key{fams}{dbg}{Dogon}
\define@key{fams}{dbi}{Atlantic-Congo}
\define@key{fams}{uya}{Atlantic-Congo}
\define@key{fams}{dre}{Sino-Tibetan}
\define@key{fams}{dov}{Atlantic-Congo}
\define@key{fams}{doq}{Sign Language}
\define@key{fams}{doa}{Nuclear Trans New Guinea}
\define@key{fams}{doy}{Atlantic-Congo}
\define@key{fams}{dof}{Mailuan}
\define@key{fams}{dev}{Nuclear Trans New Guinea}
\define@key{fams}{dok}{Austronesian}
\define@key{fams}{yik}{Sino-Tibetan}
\define@key{fams}{doh}{Atlantic-Congo}
\define@key{fams}{ddd}{Nilotic}
\define@key{fams}{dde}{Atlantic-Congo}
\define@key{fams}{dor}{Austronesian}
\define@key{fams}{kqc}{Manubaran}
\define@key{fams}{doz}{Ta-Ne-Omotic}
\define@key{fams}{dol}{Doso-Turumsa}
\define@key{fams}{dty}{Indo-European}
\define@key{fams}{dup}{Austronesian}
\define@key{fams}{dva}{Austronesian}
\define@key{fams}{dub}{Indo-European}
\define@key{fams}{dmu}{Pauwasi}
\define@key{fams}{duk}{Nuclear Trans New Guinea}
\define@key{fams}{ndu}{Atlantic-Congo}
\define@key{fams}{dbm}{Atlantic-Congo}
\define@key{fams}{dme}{Afro-Asiatic}
\define@key{fams}{kbz}{Afro-Asiatic}
\define@key{fams}{nke}{Austronesian}
\define@key{fams}{dbo}{Atlantic-Congo}
\define@key{fams}{duz}{Atlantic-Congo}
\define@key{fams}{dmv}{Austronesian}
\define@key{fams}{wtf}{Nuclear Trans New Guinea}
\define@key{fams}{dui}{Nuclear Trans New Guinea}
\define@key{fams}{duh}{Indo-European}
\define@key{fams}{raa}{Sino-Tibetan}
\define@key{fams}{dng}{Sino-Tibetan}
\define@key{fams}{dbv}{Unattested}
\define@key{fams}{drq}{Sino-Tibetan}
\define@key{fams}{mvp}{Austronesian}
\define@key{fams}{dbn}{Inanwatan}
\define@key{fams}{dug}{Atlantic-Congo}
\define@key{fams}{dsn}{Austronesian}
\define@key{fams}{duw}{Austronesian}
\define@key{fams}{duq}{Austronesian}
\define@key{fams}{dun}{Austronesian}
\define@key{fams}{dws}{Artificial Language}
\define@key{fams}{dux}{Mande}
\define@key{fams}{dae}{Atlantic-Congo}
\define@key{fams}{duv}{Lakes Plain}
\define@key{fams}{dbp}{Afro-Asiatic}
\define@key{fams}{gve}{Austronesian}
\define@key{fams}{nnu}{Atlantic-Congo}
\define@key{fams}{dyb}{Nyulnyulan}
\define@key{fams}{dyn}{Pama-Nyungan}
\define@key{fams}{dya}{Atlantic-Congo}
\define@key{fams}{dyd}{Nyulnyulan}
\define@key{fams}{jen}{Atlantic-Congo}
\define@key{fams}{dzl}{Sino-Tibetan}
\define@key{fams}{dzn}{Atlantic-Congo}
\define@key{fams}{bpn}{Hmong-Mien}
\define@key{fams}{add}{Atlantic-Congo}
\define@key{fams}{dzo}{Sino-Tibetan}
\define@key{fams}{dnn}{Mande}
\define@key{fams}{ktv}{Austroasiatic}
\define@key{fams}{bgp}{Indo-European}
\define@key{fams}{lwl}{Austroasiatic}
\define@key{fams}{mng}{Austroasiatic}
\define@key{fams}{emu}{Dravidian}
\define@key{fams}{tge}{Sino-Tibetan}
\define@key{fams}{nos}{Sino-Tibetan}
\define@key{fams}{emq}{Sino-Tibetan}
\define@key{fams}{kif}{Sino-Tibetan}
\define@key{fams}{emg}{Sino-Tibetan}
\define@key{fams}{zeh}{Tai-Kadai}
\define@key{fams}{hmq}{Hmong-Mien}
\define@key{fams}{muq}{Hmong-Mien}
\define@key{fams}{hme}{Hmong-Mien}
\define@key{fams}{lma}{Atlantic-Congo}
\define@key{fams}{gbx}{Atlantic-Congo}
\define@key{fams}{xrb}{Atlantic-Congo}
\define@key{fams}{acp}{Atlantic-Congo}
\define@key{fams}{nle}{Atlantic-Congo}
\define@key{fams}{kqo}{Kru}
\define@key{fams}{vme}{Austronesian}
\define@key{fams}{tre}{Austronesian}
\define@key{fams}{dmr}{Austronesian}
\define@key{fams}{bnj}{Austronesian}
\define@key{fams}{pez}{Austronesian}
\define@key{fams}{zbe}{Austronesian}
\define@key{fams}{kjs}{Nuclear Trans New Guinea}
\define@key{fams}{nhe}{Uto-Aztecan}
\define@key{fams}{ojg}{Algic}
\define@key{fams}{aaq}{Algic}
\define@key{fams}{qve}{Quechuan}
\define@key{fams}{cly}{Otomanguean}
\define@key{fams}{avl}{Afro-Asiatic}
\define@key{fams}{sfe}{Austronesian}
\define@key{fams}{azd}{Uto-Aztecan}
\define@key{fams}{yit}{Sino-Tibetan}
\define@key{fams}{cek}{Sino-Tibetan}
\define@key{fams}{yol}{Indo-European}
\define@key{fams}{xeb}{Afro-Asiatic}
\define@key{fams}{ebr}{Atlantic-Congo}
\define@key{fams}{ebg}{Atlantic-Congo}
\define@key{fams}{ecs}{Sign Language}
\define@key{fams}{cbj}{Atlantic-Congo}
\define@key{fams}{idd}{Atlantic-Congo}
\define@key{fams}{ijj}{Atlantic-Congo}
\define@key{fams}{ica}{Atlantic-Congo}
\define@key{fams}{nqg}{Atlantic-Congo}
\define@key{fams}{awy}{Nuclear Trans New Guinea}
\define@key{fams}{dbf}{Lakes Plain}
\define@key{fams}{eee}{Tai-Kadai}
\define@key{fams}{efa}{Atlantic-Congo}
\define@key{fams}{efe}{Central Sudanic}
\define@key{fams}{ofu}{Atlantic-Congo}
\define@key{fams}{ego}{Atlantic-Congo}
\define@key{fams}{esl}{Sign Language}
\define@key{fams}{egy}{Afro-Asiatic}
\define@key{fams}{ehu}{Atlantic-Congo}
\define@key{fams}{eit}{Nuclear Torricelli}
\define@key{fams}{eja}{Atlantic-Congo}
\define@key{fams}{eka}{Atlantic-Congo}
\define@key{fams}{eki}{Atlantic-Congo}
\define@key{fams}{eke}{Atlantic-Congo}
\define@key{fams}{ekp}{Atlantic-Congo}
\define@key{fams}{zpp}{Otomanguean}
\define@key{fams}{elx}{Isolate}
\define@key{fams}{elm}{Atlantic-Congo}
\define@key{fams}{ele}{Nuclear Torricelli}
\define@key{fams}{elh}{Nubian}
\define@key{fams}{ekm}{Atlantic-Congo}
\define@key{fams}{elk}{Nuclear Torricelli}
\define@key{fams}{elo}{Afro-Asiatic}
\define@key{fams}{zte}{Otomanguean}
\define@key{fams}{afo}{Atlantic-Congo}
\define@key{fams}{elu}{Austronesian}
\define@key{fams}{xly}{Unclassifiable}
\define@key{fams}{yzg}{Tai-Kadai}
\define@key{fams}{emn}{Atlantic-Congo}
\define@key{fams}{bdc}{Chocoan}
\define@key{fams}{tdc}{Chocoan}
\define@key{fams}{ebu}{Atlantic-Congo}
\define@key{fams}{emw}{Austronesian}
\define@key{fams}{enr}{Pauwasi}
\define@key{fams}{unk}{Arawakan}
\define@key{fams}{end}{Austronesian}
\define@key{fams}{enc}{Tai-Kadai}
\define@key{fams}{ptt}{Austronesian}
\define@key{fams}{enu}{Sino-Tibetan}
\define@key{fams}{enw}{Atlantic-Congo}
\define@key{fams}{env}{Atlantic-Congo}
\define@key{fams}{epi}{Atlantic-Congo}
\define@key{fams}{emy}{Mayan}
\define@key{fams}{era}{Dravidian}
\define@key{fams}{kjy}{Nuclear Trans New Guinea}
\define@key{fams}{twp}{Austronesian}
\define@key{fams}{ert}{Lakes Plain}
\define@key{fams}{erw}{Austronesian}
\define@key{fams}{err}{Giimbiyu}
\define@key{fams}{emx}{Speech Register}
\define@key{fams}{ers}{Sino-Tibetan}
\define@key{fams}{erh}{Atlantic-Congo}
\define@key{fams}{ish}{Atlantic-Congo}
\define@key{fams}{mcq}{Koiarian}
\define@key{fams}{esh}{Indo-European}
\define@key{fams}{ags}{Atlantic-Congo}
\define@key{fams}{esy}{Artificial Language}
\define@key{fams}{epo}{Artificial Language}
\define@key{fams}{ots}{Otomanguean}
\define@key{fams}{eso}{Sign Language}
\define@key{fams}{esm}{Unattested}
\define@key{fams}{etb}{Atlantic-Congo}
\define@key{fams}{etx}{Atlantic-Congo}
\define@key{fams}{ecr}{Unclassifiable}
\define@key{fams}{ecy}{Unclassifiable}
\define@key{fams}{eth}{Sign Language}
\define@key{fams}{ich}{Atlantic-Congo}
\define@key{fams}{eto}{Atlantic-Congo}
\define@key{fams}{etn}{Austronesian}
\define@key{fams}{ett}{Isolate}
\define@key{fams}{utr}{Atlantic-Congo}
\define@key{fams}{bzz}{Atlantic-Congo}
\define@key{fams}{gev}{Atlantic-Congo}
\define@key{fams}{nou}{Nuclear Trans New Guinea}
\define@key{fams}{ext}{Indo-European}
\define@key{fams}{fab}{Indo-European}
\define@key{fams}{faf}{Austronesian}
\define@key{fams}{fif}{Afro-Asiatic}
\define@key{fams}{azt}{Austronesian}
\define@key{fams}{faj}{Nuclear Trans New Guinea}
\define@key{fams}{fai}{Nuclear Trans New Guinea}
\define@key{fams}{fax}{Indo-European}
\define@key{fams}{cfm}{Sino-Tibetan}
\define@key{fams}{fli}{Afro-Asiatic}
\define@key{fams}{xfa}{Indo-European}
\define@key{fams}{fam}{Atlantic-Congo}
\define@key{fams}{fng}{Pidgin}
\define@key{fams}{fan}{Atlantic-Congo}
\define@key{fams}{fak}{Atlantic-Congo}
\define@key{fams}{fni}{Atlantic-Congo}
\define@key{fams}{nsf}{Sino-Tibetan}
\define@key{fams}{fmu}{Dravidian}
\define@key{fams}{far}{Austronesian}
\define@key{fams}{ddg}{Timor-Alor-Pantar}
\define@key{fams}{fau}{Lakes Plain}
\define@key{fams}{agl}{East Strickland}
\define@key{fams}{fpe}{Indo-European}
\define@key{fams}{fer}{Atlantic-Congo}
\define@key{fams}{hif}{Indo-European}
\define@key{fams}{fil}{Austronesian}
\define@key{fams}{tlp}{Totonacan}
\define@key{fams}{bkb}{Austronesian}
\define@key{fams}{fss}{Sign Language}
\define@key{fams}{fag}{Nuclear Trans New Guinea}
\define@key{fams}{fip}{Atlantic-Congo}
\define@key{fams}{fir}{Atlantic-Congo}
\define@key{fams}{fiw}{East Kutubu}
\define@key{fams}{fln}{Pama-Nyungan}
\define@key{fams}{flh}{Lakes Plain}
\define@key{fams}{fod}{Atlantic-Congo}
\define@key{fams}{frq}{Nuclear Trans New Guinea}
\define@key{fams}{enf}{Uralic}
\define@key{fams}{frt}{Austronesian}
\define@key{fams}{frp}{Indo-European}
\define@key{fams}{fur}{Indo-European}
\define@key{fams}{flr}{Atlantic-Congo}
\define@key{fams}{ula}{Atlantic-Congo}
\define@key{fams}{fuy}{Goilalan}
\define@key{fams}{fwe}{Atlantic-Congo}
\define@key{fams}{fie}{Afro-Asiatic}
\define@key{fams}{ttb}{Atlantic-Congo}
\define@key{fams}{gie}{Kru}
\define@key{fams}{gab}{Afro-Asiatic}
\define@key{fams}{gdg}{Austronesian}
\define@key{fams}{gdk}{Afro-Asiatic}
\define@key{fams}{gbk}{Indo-European}
\define@key{fams}{gad}{Austronesian}
\define@key{fams}{gda}{Indo-European}
\define@key{fams}{gdh}{Jarrakan}
\define@key{fams}{gft}{Afro-Asiatic}
\define@key{fams}{btg}{Kru}
\define@key{fams}{ggu}{Mande}
\define@key{fams}{gbf}{Ndu}
\define@key{fams}{gic}{Unclassifiable}
\define@key{fams}{gcn}{Nuclear Trans New Guinea}
\define@key{fams}{xga}{Indo-European}
\define@key{fams}{glo}{Afro-Asiatic}
\define@key{fams}{gar}{Austronesian}
\define@key{fams}{gce}{Athabaskan-Eyak-Tlingit}
\define@key{fams}{sdn}{Indo-European}
\define@key{fams}{gap}{Nuclear Trans New Guinea}
\define@key{fams}{gal}{Austronesian}
\define@key{fams}{kgj}{Sino-Tibetan}
\define@key{fams}{gma}{Worrorran}
\define@key{fams}{wof}{Atlantic-Congo}
\define@key{fams}{gbl}{Indo-European}
\define@key{fams}{gak}{North Halmahera}
\define@key{fams}{bte}{Atlantic-Congo}
\define@key{fams}{ihw}{Pama-Nyungan}
\define@key{fams}{gne}{Atlantic-Congo}
\define@key{fams}{gnk}{Khoe-Kwadi}
\define@key{fams}{gnq}{Austronesian}
\define@key{fams}{unn}{Pama-Nyungan}
\define@key{fams}{gan}{Sino-Tibetan}
\define@key{fams}{pgd}{Indo-European}
\define@key{fams}{gzn}{Austronesian}
\define@key{fams}{gnb}{Sino-Tibetan}
\define@key{fams}{gnl}{Pama-Nyungan}
\define@key{fams}{ggl}{Nuclear Trans New Guinea}
\define@key{fams}{gao}{Nuclear Trans New Guinea}
\define@key{fams}{gza}{Blue Nile Mao}
\define@key{fams}{gnz}{Atlantic-Congo}
\define@key{fams}{gga}{Austronesian}
\define@key{fams}{gbm}{Indo-European}
\define@key{fams}{ilg}{Iwaidjan Proper}
\define@key{fams}{gex}{Afro-Asiatic}
\define@key{fams}{gaq}{Austroasiatic}
\define@key{fams}{gou}{Afro-Asiatic}
\define@key{fams}{gwt}{Indo-European}
\define@key{fams}{gyl}{South Omotic}
\define@key{fams}{gzi}{Indo-European}
\define@key{fams}{gbg}{Atlantic-Congo}
\define@key{fams}{gbv}{Atlantic-Congo}
\define@key{fams}{gby}{Atlantic-Congo}
\define@key{fams}{gyg}{Atlantic-Congo}
\define@key{fams}{gbq}{Atlantic-Congo}
\define@key{fams}{gbs}{Atlantic-Congo}
\define@key{fams}{ggb}{Kru}
\define@key{fams}{xgb}{Mande}
\define@key{fams}{grh}{Atlantic-Congo}
\define@key{fams}{gec}{Kru}
\define@key{fams}{kvq}{Sino-Tibetan}
\define@key{fams}{gei}{Austronesian}
\define@key{fams}{gdd}{Austronesian}
\define@key{fams}{drs}{Afro-Asiatic}
\define@key{fams}{hmj}{Hmong-Mien}
\define@key{fams}{gez}{Afro-Asiatic}
\define@key{fams}{ghk}{Sino-Tibetan}
\define@key{fams}{giu}{Tai-Kadai}
\define@key{fams}{geq}{Atlantic-Congo}
\define@key{fams}{gaf}{Nuclear Trans New Guinea}
\define@key{fams}{gej}{Atlantic-Congo}
\define@key{fams}{ygp}{Sino-Tibetan}
\define@key{fams}{gew}{Afro-Asiatic}
\define@key{fams}{gea}{Afro-Asiatic}
\define@key{fams}{ges}{Austronesian}
\define@key{fams}{gha}{Afro-Asiatic}
\define@key{fams}{gse}{Sign Language}
\define@key{fams}{ghn}{Austronesian}
\define@key{fams}{gpe}{Indo-European}
\define@key{fams}{gds}{Sign Language}
\define@key{fams}{gri}{Austronesian}
\define@key{fams}{ajs}{Sign Language}
\define@key{fams}{bmk}{Austronesian}
\define@key{fams}{aln}{Indo-European}
\define@key{fams}{ghr}{Indo-European}
\define@key{fams}{bbj}{Atlantic-Congo}
\define@key{fams}{gho}{Afro-Asiatic}
\define@key{fams}{bgi}{Austronesian}
\define@key{fams}{gib}{Pidgin}
\define@key{fams}{kks}{Afro-Asiatic}
\define@key{fams}{acd}{Atlantic-Congo}
\define@key{fams}{gix}{Atlantic-Congo}
\define@key{fams}{gip}{Austronesian}
\define@key{fams}{gim}{Nuclear Trans New Guinea}
\define@key{fams}{kmp}{Atlantic-Congo}
\define@key{fams}{gmn}{Atlantic-Congo}
\define@key{fams}{gnm}{Dagan}
\define@key{fams}{ayg}{Atlantic-Congo}
\define@key{fams}{bbr}{Nuclear Trans New Guinea}
\define@key{fams}{gii}{Afro-Asiatic}
\define@key{fams}{nyf}{Atlantic-Congo}
\define@key{fams}{toh}{Atlantic-Congo}
\define@key{fams}{ggt}{Austronesian}
\define@key{fams}{giy}{Unattested}
\define@key{fams}{tof}{Eastern Trans-Fly}
\define@key{fams}{glr}{Kru}
\define@key{fams}{glw}{Afro-Asiatic}
\define@key{fams}{oub}{Kru}
\define@key{fams}{gnu}{Nuclear Torricelli}
\define@key{fams}{gom}{Indo-European}
\define@key{fams}{gig}{Indo-European}
\define@key{fams}{goi}{East Strickland}
\define@key{fams}{gox}{Atlantic-Congo}
\define@key{fams}{gdx}{Indo-European}
\define@key{fams}{gof}{Ta-Ne-Omotic}
\define@key{fams}{gog}{Atlantic-Congo}
\define@key{fams}{goo}{Austronesian}
\define@key{fams}{goe}{Sino-Tibetan}
\define@key{fams}{gjn}{Atlantic-Congo}
\define@key{fams}{gov}{Mande}
\define@key{fams}{goq}{Austronesian}
\define@key{fams}{goc}{Austronesian}
\define@key{fams}{grq}{Lower Sepik-Ramu}
\define@key{fams}{gqr}{Central Sudanic}
\define@key{fams}{got}{Indo-European}
\define@key{fams}{goy}{Atlantic-Congo}
\define@key{fams}{gwf}{Indo-European}
\define@key{fams}{goz}{Indo-European}
\define@key{fams}{nli}{Indo-European}
\define@key{fams}{giq}{Tai-Kadai}
\define@key{fams}{gcl}{Indo-European}
\define@key{fams}{grs}{Nimboranic}
\define@key{fams}{gro}{Sino-Tibetan}
\define@key{fams}{gos}{Indo-European}
\define@key{fams}{ats}{Algic}
\define@key{fams}{gwx}{Atlantic-Congo}
\define@key{fams}{gvj}{Tupian}
\define@key{fams}{jiq}{Sino-Tibetan}
\define@key{fams}{gnc}{Afro-Asiatic}
\define@key{fams}{gyr}{Tupian}
\define@key{fams}{gsm}{Sign Language}
\define@key{fams}{xgd}{Pama-Nyungan}
\define@key{fams}{gdu}{Afro-Asiatic}
\define@key{fams}{zpg}{Otomanguean}
\define@key{fams}{gdc}{Pama-Nyungan}
\define@key{fams}{kkp}{Pama-Nyungan}
\define@key{fams}{wrw}{Pama-Nyungan}
\define@key{fams}{zgn}{Tai-Kadai}
\define@key{fams}{bet}{Kru}
\define@key{fams}{ztu}{Otomanguean}
\define@key{fams}{gus}{Sign Language}
\define@key{fams}{gkp}{Mande}
\define@key{fams}{gqi}{Sino-Tibetan}
\define@key{fams}{gvl}{Central Sudanic}
\define@key{fams}{glu}{Central Sudanic}
\define@key{fams}{gmb}{Austronesian}
\define@key{fams}{gly}{Isolate}
\define@key{fams}{gul}{Indo-European}
\define@key{fams}{gmu}{Nuclear Trans New Guinea}
\define@key{fams}{gdi}{Atlantic-Congo}
\define@key{fams}{gyf}{Pama-Nyungan}
\define@key{fams}{rub}{Atlantic-Congo}
\define@key{fams}{gnt}{Yam}
\define@key{fams}{gpa}{Atlantic-Congo}
\define@key{fams}{grz}{Austronesian}
\define@key{fams}{gdj}{Pama-Nyungan}
\define@key{fams}{ggg}{Indo-European}
\define@key{fams}{grx}{Isolate}
\define@key{fams}{gjr}{Mixed Language}
\define@key{fams}{gvm}{Atlantic-Congo}
\define@key{fams}{gvr}{Sino-Tibetan}
\define@key{fams}{grd}{Afro-Asiatic}
\define@key{fams}{gsn}{Nuclear Trans New Guinea}
\define@key{fams}{gsl}{Atlantic-Congo}
\define@key{fams}{xgw}{Pama-Nyungan}
\define@key{fams}{gwu}{Pama-Nyungan}
\define@key{fams}{gvy}{Pama-Nyungan}
\define@key{fams}{gka}{Nuclear Trans New Guinea}
\define@key{fams}{ngs}{Afro-Asiatic}
\define@key{fams}{gwb}{Atlantic-Congo}
\define@key{fams}{dah}{Nuclear Trans New Guinea}
\define@key{fams}{bga}{Atlantic-Congo}
\define@key{fams}{gwn}{Afro-Asiatic}
\define@key{fams}{grw}{Austronesian}
\define@key{fams}{gwe}{Atlantic-Congo}
\define@key{fams}{gwr}{Atlantic-Congo}
\define@key{fams}{gwj}{Khoe-Kwadi}
\define@key{fams}{gyi}{Atlantic-Congo}
\define@key{fams}{gye}{Atlantic-Congo}
\define@key{fams}{haq}{Atlantic-Congo}
\define@key{fams}{hbu}{Austronesian}
\define@key{fams}{hdy}{Afro-Asiatic}
\define@key{fams}{hoj}{Indo-European}
\define@key{fams}{xhd}{Afro-Asiatic}
\define@key{fams}{ayh}{Afro-Asiatic}
\define@key{fams}{aek}{Austronesian}
\define@key{fams}{hah}{Austronesian}
\define@key{fams}{hgw}{Austronesian}
\define@key{fams}{bzx}{Mande}
\define@key{fams}{hgm}{Khoe-Kwadi}
\define@key{fams}{haf}{Sign Language}
\define@key{fams}{hvc}{Unclassifiable}
\define@key{fams}{hji}{Austronesian}
\define@key{fams}{haj}{Indo-European}
\define@key{fams}{hao}{Austronesian}
\define@key{fams}{hld}{Austroasiatic}
\define@key{fams}{hmu}{Timor-Alor-Pantar}
\define@key{fams}{hba}{Atlantic-Congo}
\define@key{fams}{hag}{Atlantic-Congo}
\define@key{fams}{han}{Atlantic-Congo}
\define@key{fams}{haa}{Athabaskan-Eyak-Tlingit}
\define@key{fams}{hab}{Sign Language}
\define@key{fams}{xiv}{Unattested}
\define@key{fams}{kjo}{Indo-European}
\define@key{fams}{hro}{Austronesian}
\define@key{fams}{hrk}{Austronesian}
\define@key{fams}{bgc}{Indo-European}
\define@key{fams}{hrz}{Indo-European}
\define@key{fams}{ybj}{Atlantic-Congo}
\define@key{fams}{xht}{Isolate}
\define@key{fams}{hsl}{Sign Language}
\define@key{fams}{hvk}{Austronesian}
\define@key{fams}{hav}{Atlantic-Congo}
\define@key{fams}{hps}{Sign Language}
\define@key{fams}{xda}{Pama-Nyungan}
\define@key{fams}{haz}{Indo-European}
\define@key{fams}{hbn}{Heibanic}
\define@key{fams}{scp}{Sino-Tibetan}
\define@key{fams}{heg}{Austronesian}
\define@key{fams}{nix}{Atlantic-Congo}
\define@key{fams}{hed}{Afro-Asiatic}
\define@key{fams}{llf}{Austronesian}
\define@key{fams}{hrt}{Afro-Asiatic}
\define@key{fams}{ham}{Sepik}
\define@key{fams}{auk}{Nuclear Torricelli}
\define@key{fams}{hib}{Hibito-Cholon}
\define@key{fams}{hlu}{Indo-European}
\define@key{fams}{mba}{Austronesian}
\define@key{fams}{kjk}{Austronesian}
\define@key{fams}{hij}{Atlantic-Congo}
\define@key{fams}{hir}{Unattested}
\define@key{fams}{hii}{Indo-European}
\define@key{fams}{hmo}{Pidgin}
\define@key{fams}{hit}{Indo-European}
\define@key{fams}{htu}{Austronesian}
\define@key{fams}{hiw}{Austronesian}
\define@key{fams}{yhl}{Sino-Tibetan}
\define@key{fams}{hle}{Sino-Tibetan}
\define@key{fams}{hmf}{Hmong-Mien}
\define@key{fams}{hmz}{Hmong-Mien}
\define@key{fams}{hmv}{Hmong-Mien}
\define@key{fams}{mrk}{Austronesian}
\define@key{fams}{hoh}{Afro-Asiatic}
\define@key{fams}{hos}{Sign Language}
\define@key{fams}{hhi}{Anim}
\define@key{fams}{hoy}{Dravidian}
\define@key{fams}{hoi}{Athabaskan-Eyak-Tlingit}
\define@key{fams}{hod}{Afro-Asiatic}
\define@key{fams}{hol}{Atlantic-Congo}
\define@key{fams}{hom}{Atlantic-Congo}
\define@key{fams}{hds}{Sign Language}
\define@key{fams}{juh}{Atlantic-Congo}
\define@key{fams}{how}{Sino-Tibetan}
\define@key{fams}{hrm}{Hmong-Mien}
\define@key{fams}{hoe}{Atlantic-Congo}
\define@key{fams}{hor}{Central Sudanic}
\define@key{fams}{ero}{Sino-Tibetan}
\define@key{fams}{hot}{Austronesian}
\define@key{fams}{hti}{Austronesian}
\define@key{fams}{hov}{Austronesian}
\define@key{fams}{hhy}{Anim}
\define@key{fams}{hoz}{Blue Nile Mao}
\define@key{fams}{hpo}{Sino-Tibetan}
\define@key{fams}{hra}{Sino-Tibetan}
\define@key{fams}{hru}{Isolate}
\define@key{fams}{hug}{Harakmbut}
\define@key{fams}{qvh}{Quechuan}
\define@key{fams}{hud}{Austronesian}
\define@key{fams}{nhq}{Uto-Aztecan}
\define@key{fams}{qwh}{Quechuan}
\define@key{fams}{qvw}{Quechuan}
\define@key{fams}{huh}{Araucanian}
\define@key{fams}{mxs}{Otomanguean}
\define@key{fams}{czh}{Sino-Tibetan}
\define@key{fams}{huw}{Austronesian}
\define@key{fams}{hul}{Austronesian}
\define@key{fams}{huy}{Afro-Asiatic}
\define@key{fams}{hui}{Nuclear Trans New Guinea}
\define@key{fams}{huk}{Austronesian}
\define@key{fams}{hmb}{Songhay}
\define@key{fams}{huf}{Kwalean}
\define@key{fams}{hut}{Sino-Tibetan}
\define@key{fams}{hsh}{Sign Language}
\define@key{fams}{hnu}{Austroasiatic}
\define@key{fams}{nat}{Atlantic-Congo}
\define@key{fams}{hum}{Atlantic-Congo}
\define@key{fams}{hng}{Atlantic-Congo}
\define@key{fams}{hkk}{Nuclear Trans New Guinea}
\define@key{fams}{hap}{Nuclear Trans New Guinea}
\define@key{fams}{xhu}{Hurro-Urartian}
\define@key{fams}{geh}{Indo-European}
\define@key{fams}{huo}{Austroasiatic}
\define@key{fams}{hwo}{Afro-Asiatic}
\define@key{fams}{hya}{Afro-Asiatic}
\define@key{fams}{jab}{Atlantic-Congo}
\define@key{fams}{yml}{Austronesian}
\define@key{fams}{tek}{Atlantic-Congo}
\define@key{fams}{ibl}{Austronesian}
\define@key{fams}{iby}{Ijoid}
\define@key{fams}{xib}{Isolate}
\define@key{fams}{ibn}{Atlantic-Congo}
\define@key{fams}{ibr}{Atlantic-Congo}
\define@key{fams}{ibu}{North Halmahera}
\define@key{fams}{bec}{Atlantic-Congo}
\define@key{fams}{ida}{Atlantic-Congo}
\define@key{fams}{idt}{Austronesian}
\define@key{fams}{ide}{Atlantic-Congo}
\define@key{fams}{idi}{Pahoturi}
\define@key{fams}{idc}{Atlantic-Congo}
\define@key{fams}{ido}{Artificial Language}
\define@key{fams}{ldb}{Atlantic-Congo}
\define@key{fams}{ife}{Atlantic-Congo}
\define@key{fams}{iff}{Austronesian}
\define@key{fams}{igl}{Atlantic-Congo}
\define@key{fams}{igg}{Lower Sepik-Ramu}
\define@key{fams}{ahl}{Atlantic-Congo}
\define@key{fams}{nar}{Atlantic-Congo}
\define@key{fams}{igw}{Atlantic-Congo}
\define@key{fams}{ihb}{Pidgin}
\define@key{fams}{ikk}{Atlantic-Congo}
\define@key{fams}{ikr}{Pama-Nyungan}
\define@key{fams}{ikz}{Atlantic-Congo}
\define@key{fams}{meb}{Turama-Kikori}
\define@key{fams}{ntk}{Atlantic-Congo}
\define@key{fams}{iki}{Atlantic-Congo}
\define@key{fams}{ikp}{Atlantic-Congo}
\define@key{fams}{txi}{Cariban}
\define@key{fams}{ikv}{Atlantic-Congo}
\define@key{fams}{ikl}{Atlantic-Congo}
\define@key{fams}{ikw}{Atlantic-Congo}
\define@key{fams}{ila}{Austronesian}
\define@key{fams}{mbi}{Austronesian}
\define@key{fams}{ili}{Turkic}
\define@key{fams}{ilu}{Austronesian}
\define@key{fams}{xil}{Unclassifiable}
\define@key{fams}{ilk}{Austronesian}
\define@key{fams}{ilv}{Atlantic-Congo}
\define@key{fams}{mlk}{Atlantic-Congo}
\define@key{fams}{imo}{Nuclear Trans New Guinea}
\define@key{fams}{arc}{Afro-Asiatic}
\define@key{fams}{imr}{Austronesian}
\define@key{fams}{abx}{Austronesian}
\define@key{fams}{mzu}{Lower Sepik-Ramu}
\define@key{fams}{inp}{Arawakan}
\define@key{fams}{smn}{Uralic}
\define@key{fams}{inl}{Sign Language}
\define@key{fams}{idr}{Atlantic-Congo}
\define@key{fams}{mvy}{Indo-European}
\define@key{fams}{oin}{Nuclear Torricelli}
\define@key{fams}{iti}{Austronesian}
\define@key{fams}{ino}{Nuclear Trans New Guinea}
\define@key{fams}{loc}{Austronesian}
\define@key{fams}{ior}{Afro-Asiatic}
\define@key{fams}{ina}{Artificial Language}
\define@key{fams}{ile}{Artificial Language}
\define@key{fams}{igs}{Artificial Language}
\define@key{fams}{int}{Sino-Tibetan}
\define@key{fams}{iks}{Sign Language}
\define@key{fams}{azm}{Otomanguean}
\define@key{fams}{ipo}{Anim}
\define@key{fams}{ipi}{Nuclear Trans New Guinea}
\define@key{fams}{ass}{Atlantic-Congo}
\define@key{fams}{ill}{Austronesian}
\define@key{fams}{iry}{Austronesian}
\define@key{fams}{ire}{Austronesian}
\define@key{fams}{iri}{Atlantic-Congo}
\define@key{fams}{bto}{Austronesian}
\define@key{fams}{iru}{Dravidian}
\define@key{fams}{isa}{Nuclear Trans New Guinea}
\define@key{fams}{isn}{Atlantic-Congo}
\define@key{fams}{agk}{Austronesian}
\define@key{fams}{isc}{Pano-Tacanan}
\define@key{fams}{igo}{Nuclear Trans New Guinea}
\define@key{fams}{inn}{Austronesian}
\define@key{fams}{crb}{Arawakan}
\define@key{fams}{mir}{Mixe-Zoque}
\define@key{fams}{nhk}{Uto-Aztecan}
\define@key{fams}{ist}{Indo-European}
\define@key{fams}{ruo}{Indo-European}
\define@key{fams}{szv}{Atlantic-Congo}
\define@key{fams}{isu}{Atlantic-Congo}
\define@key{fams}{ite}{Chapacuran}
\define@key{fams}{itr}{Left May}
\define@key{fams}{itx}{Tor-Orya}
\define@key{fams}{itw}{Atlantic-Congo}
\define@key{fams}{itm}{Atlantic-Congo}
\define@key{fams}{mce}{Otomanguean}
\define@key{fams}{ivv}{Austronesian}
\define@key{fams}{atg}{Atlantic-Congo}
\define@key{fams}{iwk}{Austronesian}
\define@key{fams}{kbm}{Austronesian}
\define@key{fams}{iwo}{Nuclear Trans New Guinea}
\define@key{fams}{mzi}{Otomanguean}
\define@key{fams}{vmj}{Otomanguean}
\define@key{fams}{iya}{Atlantic-Congo}
\define@key{fams}{uiv}{Atlantic-Congo}
\define@key{fams}{crt}{Matacoan}
\define@key{fams}{nca}{Nuclear Trans New Guinea}
\define@key{fams}{crq}{Matacoan}
\define@key{fams}{izi}{Atlantic-Congo}
\define@key{fams}{cbo}{Atlantic-Congo}
\define@key{fams}{rzh}{Afro-Asiatic}
\define@key{fams}{jdg}{Indo-European}
\define@key{fams}{jad}{Mande}
\define@key{fams}{jah}{Austroasiatic}
\define@key{fams}{awv}{Nuclear Trans New Guinea}
\define@key{fams}{jat}{Indo-European}
\define@key{fams}{jak}{Austronesian}
\define@key{fams}{maj}{Otomanguean}
\define@key{fams}{bxl}{Mande}
\define@key{fams}{jcs}{Sign Language}
\define@key{fams}{jls}{Sign Language}
\define@key{fams}{jax}{Austronesian}
\define@key{fams}{jnd}{Indo-European}
\define@key{fams}{jna}{Sino-Tibetan}
\define@key{fams}{djo}{Austronesian}
\define@key{fams}{jni}{Atlantic-Congo}
\define@key{fams}{jar}{Atlantic-Congo}
\define@key{fams}{jra}{Austronesian}
\define@key{fams}{jaf}{Afro-Asiatic}
\define@key{fams}{qxw}{Quechuan}
\define@key{fams}{jns}{Indo-European}
\define@key{fams}{jvd}{Indo-European}
\define@key{fams}{jaz}{Austronesian}
\define@key{fams}{jyy}{Central Sudanic}
\define@key{fams}{jje}{Koreanic}
\define@key{fams}{bze}{Mande}
\define@key{fams}{xuj}{Dravidian}
\define@key{fams}{jer}{Atlantic-Congo}
\define@key{fams}{jee}{Sino-Tibetan}
\define@key{fams}{tmr}{Afro-Asiatic}
\define@key{fams}{jhs}{Sign Language}
\define@key{fams}{jio}{Tai-Kadai}
\define@key{fams}{juo}{Atlantic-Congo}
\define@key{fams}{jib}{Atlantic-Congo}
\define@key{fams}{jii}{Afro-Asiatic}
\define@key{fams}{jie}{Afro-Asiatic}
\define@key{fams}{jil}{Nuclear Trans New Guinea}
\define@key{fams}{jim}{Afro-Asiatic}
\define@key{fams}{jmi}{Afro-Asiatic}
\define@key{fams}{jia}{Afro-Asiatic}
\define@key{fams}{cjy}{Sino-Tibetan}
\define@key{fams}{pnu}{Hmong-Mien}
\define@key{fams}{jul}{Sino-Tibetan}
\define@key{fams}{jrr}{Atlantic-Congo}
\define@key{fams}{jit}{Atlantic-Congo}
\define@key{fams}{kaj}{Atlantic-Congo}
\define@key{fams}{job}{Atlantic-Congo}
\define@key{fams}{jbr}{Tor-Orya}
\define@key{fams}{jeu}{Afro-Asiatic}
\define@key{fams}{jor}{Tupian}
\define@key{fams}{jrt}{Afro-Asiatic}
\define@key{fams}{jow}{Mande}
\define@key{fams}{itk}{Indo-European}
\define@key{fams}{jdt}{Indo-European}
\define@key{fams}{jpr}{Indo-European}
\define@key{fams}{yud}{Afro-Asiatic}
\define@key{fams}{aju}{Afro-Asiatic}
\define@key{fams}{yhd}{Afro-Asiatic}
\define@key{fams}{jye}{Afro-Asiatic}
\define@key{fams}{jum}{Nilotic}
\define@key{fams}{jml}{Indo-European}
\define@key{fams}{jus}{Sign Language}
\define@key{fams}{mxq}{Mixe-Zoque}
\define@key{fams}{juy}{Austroasiatic}
\define@key{fams}{jut}{Indo-European}
\define@key{fams}{juu}{Afro-Asiatic}
\define@key{fams}{mwb}{Nuclear Torricelli}
\define@key{fams}{vmc}{Otomanguean}
\define@key{fams}{jwi}{Atlantic-Congo}
\define@key{fams}{xku}{Atlantic-Congo}
\define@key{fams}{gna}{Atlantic-Congo}
\define@key{fams}{ldl}{Atlantic-Congo}
\define@key{fams}{ckn}{Sino-Tibetan}
\define@key{fams}{ksp}{Central Sudanic}
\define@key{fams}{kvf}{Afro-Asiatic}
\define@key{fams}{gbw}{Pama-Nyungan}
\define@key{fams}{klz}{Timor-Alor-Pantar}
\define@key{fams}{onk}{Nuclear Torricelli}
\define@key{fams}{lkb}{Atlantic-Congo}
\define@key{fams}{uka}{South Bird's Head Family}
\define@key{fams}{kbu}{Indo-European}
\define@key{fams}{kea}{Indo-European}
\define@key{fams}{cwa}{Atlantic-Congo}
\define@key{fams}{kcw}{Atlantic-Congo}
\define@key{fams}{gjk}{Indo-European}
\define@key{fams}{kfr}{Indo-European}
\define@key{fams}{kcx}{Ta-Ne-Omotic}
\define@key{fams}{xkk}{Austroasiatic}
\define@key{fams}{kej}{Dravidian}
\define@key{fams}{kdu}{Nubian}
\define@key{fams}{kad}{Atlantic-Congo}
\define@key{fams}{kzd}{Austronesian}
\define@key{fams}{kdv}{Sino-Tibetan}
\define@key{fams}{ktp}{Sino-Tibetan}
\define@key{fams}{jka}{Timor-Alor-Pantar}
\define@key{fams}{kpu}{Timor-Alor-Pantar}
\define@key{fams}{sqx}{Sign Language}
\define@key{fams}{syw}{Sino-Tibetan}
\define@key{fams}{kll}{Austronesian}
\define@key{fams}{cgc}{Austronesian}
\define@key{fams}{gel}{Atlantic-Congo}
\define@key{fams}{xkg}{Mande}
\define@key{fams}{hka}{Atlantic-Congo}
\define@key{fams}{agw}{Austronesian}
\define@key{fams}{kzb}{Austronesian}
\define@key{fams}{kzp}{Austronesian}
\define@key{fams}{kbw}{Austronesian}
\define@key{fams}{kep}{Dravidian}
\define@key{fams}{kzq}{Sino-Tibetan}
\define@key{fams}{kkq}{Atlantic-Congo}
\define@key{fams}{xai}{Unclassifiable}
\define@key{fams}{zka}{Austronesian}
\define@key{fams}{krd}{Austronesian}
\define@key{fams}{ckr}{Baining}
\define@key{fams}{kzm}{South Bird's Head Family}
\define@key{fams}{kce}{Unattested}
\define@key{fams}{tcq}{Lakes Plain}
\define@key{fams}{xkj}{Indo-European}
\define@key{fams}{kag}{Austronesian}
\define@key{fams}{ckq}{Afro-Asiatic}
\define@key{fams}{kjv}{Indo-European}
\define@key{fams}{xdq}{Nakh-Daghestanian}
\define@key{fams}{kka}{Atlantic-Congo}
\define@key{fams}{kke}{Mande}
\define@key{fams}{kqf}{Austronesian}
\define@key{fams}{kkj}{Atlantic-Congo}
\define@key{fams}{keo}{Nilotic}
\define@key{fams}{wkl}{Dravidian}
\define@key{fams}{kzz}{West Bird's Head}
\define@key{fams}{kkf}{Sino-Tibetan}
\define@key{fams}{kba}{Pama-Nyungan}
\define@key{fams}{gll}{Pama-Nyungan}
\define@key{fams}{ijn}{Ijoid}
\define@key{fams}{knz}{Atlantic-Congo}
\define@key{fams}{kqe}{Austronesian}
\define@key{fams}{kve}{Austronesian}
\define@key{fams}{kly}{Austronesian}
\define@key{fams}{lkm}{Pama-Nyungan}
\define@key{fams}{xka}{Indo-European}
\define@key{fams}{rmf}{Indo-European}
\define@key{fams}{ywa}{Sepik}
\define@key{fams}{kli}{Austronesian}
\define@key{fams}{keq}{Indo-European}
\define@key{fams}{jmr}{Atlantic-Congo}
\define@key{fams}{kci}{Atlantic-Congo}
\define@key{fams}{klp}{Angan}
\define@key{fams}{kzx}{Austronesian}
\define@key{fams}{kyk}{Austronesian}
\define@key{fams}{kgx}{Austronesian}
\define@key{fams}{vkm}{Kamakanan}
\define@key{fams}{xbw}{Unclassifiable}
\define@key{fams}{irx}{Nuclear Trans New Guinea}
\define@key{fams}{kyy}{Nuclear Trans New Guinea}
\define@key{fams}{ktb}{Afro-Asiatic}
\define@key{fams}{kmi}{Atlantic-Congo}
\define@key{fams}{kdx}{Atlantic-Congo}
\define@key{fams}{kcq}{Atlantic-Congo}
\define@key{fams}{xla}{Kamula-Elevala}
\define@key{fams}{hig}{Afro-Asiatic}
\define@key{fams}{bjj}{Indo-European}
\define@key{fams}{xnb}{Austronesian}
\define@key{fams}{soq}{Dagan}
\define@key{fams}{kbs}{Atlantic-Congo}
\define@key{fams}{kqw}{Austronesian}
\define@key{fams}{gam}{Nuclear Trans New Guinea}
\define@key{fams}{xnr}{Indo-European}
\define@key{fams}{kxs}{Mongolic-Khitan}
\define@key{fams}{kzy}{Atlantic-Congo}
\define@key{fams}{kty}{Atlantic-Congo}
\define@key{fams}{kcp}{Kadugli-Krongo}
\define@key{fams}{kkv}{Austronesian}
\define@key{fams}{igm}{Lower Sepik-Ramu}
\define@key{fams}{kev}{Dravidian}
\define@key{fams}{kdp}{Atlantic-Congo}
\define@key{fams}{kzo}{Atlantic-Congo}
\define@key{fams}{wat}{Austronesian}
\define@key{fams}{ktk}{Austronesian}
\define@key{fams}{knr}{Sepik}
\define@key{fams}{kmu}{Nuclear Trans New Guinea}
\define@key{fams}{kft}{Indo-European}
\define@key{fams}{kbe}{Pama-Nyungan}
\define@key{fams}{kxn}{Austronesian}
\define@key{fams}{ksk}{Siouan}
\define@key{fams}{xkt}{Atlantic-Congo}
\define@key{fams}{kni}{Atlantic-Congo}
\define@key{fams}{khx}{Atlantic-Congo}
\define@key{fams}{kqn}{Atlantic-Congo}
\define@key{fams}{kax}{North Halmahera}
\define@key{fams}{xpn}{Unclassifiable}
\define@key{fams}{tbx}{Austronesian}
\define@key{fams}{khp}{Isolate}
\define@key{fams}{ykm}{Austronesian}
\define@key{fams}{kbi}{Austronesian}
\define@key{fams}{klo}{Atlantic-Congo}
\define@key{fams}{xkh}{Unattested}
\define@key{fams}{kzr}{Atlantic-Congo}
\define@key{fams}{reg}{Atlantic-Congo}
\define@key{fams}{kth}{Maban}
\define@key{fams}{mry}{Austronesian}
\define@key{fams}{xrw}{Sepik}
\define@key{fams}{xar}{Isolate}
\define@key{fams}{kgv}{West Bomberai}
\define@key{fams}{kbn}{Atlantic-Congo}
\define@key{fams}{kyd}{Austronesian}
\define@key{fams}{kmf}{Nuclear Trans New Guinea}
\define@key{fams}{kai}{Afro-Asiatic}
\define@key{fams}{kmv}{Indo-European}
\define@key{fams}{kgn}{Indo-European}
\define@key{fams}{kbj}{Atlantic-Congo}
\define@key{fams}{kil}{Afro-Asiatic}
\define@key{fams}{kuq}{Tupian}
\define@key{fams}{kko}{Nubian}
\define@key{fams}{krb}{Miwok-Costanoan}
\define@key{fams}{bbv}{Austronesian}
\define@key{fams}{krx}{Atlantic-Congo}
\define@key{fams}{kxh}{South Omotic}
\define@key{fams}{xkx}{Austronesian}
\define@key{fams}{kyn}{Austronesian}
\define@key{fams}{rxw}{Pama-Nyungan}
\define@key{fams}{ccj}{Atlantic-Congo}
\define@key{fams}{ksn}{Austronesian}
\define@key{fams}{kkz}{Athabaskan-Eyak-Tlingit}
\define@key{fams}{khs}{Bosavi}
\define@key{fams}{ktq}{Unclassifiable}
\define@key{fams}{xat}{Katukinan}
\define@key{fams}{tmb}{Austronesian}
\define@key{fams}{tkt}{Indo-European}
\define@key{fams}{ykt}{Sino-Tibetan}
\define@key{fams}{kfu}{Indo-European}
\define@key{fams}{kaf}{Sino-Tibetan}
\define@key{fams}{kta}{Austroasiatic}
\define@key{fams}{vkk}{Austronesian}
\define@key{fams}{xau}{Greater Kwerba}
\define@key{fams}{ckv}{Austronesian}
\define@key{fams}{kcb}{Angan}
\define@key{fams}{kgb}{Austronesian}
\define@key{fams}{kaw}{Austronesian}
\define@key{fams}{ktx}{Pano-Tacanan}
\define@key{fams}{kbb}{Cariban}
\define@key{fams}{pdu}{Sino-Tibetan}
\define@key{fams}{xay}{Austronesian}
\define@key{fams}{xkn}{Austronesian}
\define@key{fams}{kyt}{Kayagaric}
\define@key{fams}{kzl}{Austronesian}
\define@key{fams}{kxy}{Austroasiatic}
\define@key{fams}{kzu}{Austronesian}
\define@key{fams}{kzk}{Austronesian}
\define@key{fams}{keh}{Ndu}
\define@key{fams}{khz}{Austronesian}
\define@key{fams}{meo}{Austronesian}
\define@key{fams}{kdy}{Tor-Orya}
\define@key{fams}{khh}{Isolate}
\define@key{fams}{kec}{Kadugli-Krongo}
\define@key{fams}{bmh}{Nuclear Trans New Guinea}
\define@key{fams}{eyo}{Nilotic}
\define@key{fams}{khy}{Atlantic-Congo}
\define@key{fams}{keb}{Atlantic-Congo}
\define@key{fams}{ify}{Austronesian}
\define@key{fams}{kbo}{Central Sudanic}
\define@key{fams}{xel}{Eastern Jebel}
\define@key{fams}{kyo}{Timor-Alor-Pantar}
\define@key{fams}{kem}{Austronesian}
\define@key{fams}{bzp}{South Bird's Head Family}
\define@key{fams}{xem}{Austronesian}
\define@key{fams}{xkw}{Lepki-Murkim-Kembra}
\define@key{fams}{dmo}{Atlantic-Congo}
\define@key{fams}{sjk}{Uralic}
\define@key{fams}{xbn}{Isolate}
\define@key{fams}{gat}{Nuclear Trans New Guinea}
\define@key{fams}{kvm}{Atlantic-Congo}
\define@key{fams}{klf}{Maban}
\define@key{fams}{knx}{Austronesian}
\define@key{fams}{knl}{Austronesian}
\define@key{fams}{kxi}{Austronesian}
\define@key{fams}{kns}{Austroasiatic}
\define@key{fams}{ndb}{Atlantic-Congo}
\define@key{fams}{kzh}{Nubian}
\define@key{fams}{lke}{Atlantic-Congo}
\define@key{fams}{xeu}{Eleman}
\define@key{fams}{kpn}{Tupian}
\define@key{fams}{kuk}{Austronesian}
\define@key{fams}{hhr}{Atlantic-Congo}
\define@key{fams}{ked}{Atlantic-Congo}
\define@key{fams}{xke}{Austronesian}
\define@key{fams}{kxz}{Kiwaian}
\define@key{fams}{kvr}{Austronesian}
\define@key{fams}{xes}{Nuclear Trans New Guinea}
\define@key{fams}{kae}{Austronesian}
\define@key{fams}{ktt}{Nuclear Trans New Guinea}
\define@key{fams}{kyg}{Nuclear Trans New Guinea}
\define@key{fams}{xkv}{Atlantic-Congo}
\define@key{fams}{hkh}{Indo-European}
\define@key{fams}{kbg}{Sino-Tibetan}
\define@key{fams}{kht}{Tai-Kadai}
\define@key{fams}{ksu}{Tai-Kadai}
\define@key{fams}{khn}{Indo-European}
\define@key{fams}{kjm}{Austroasiatic}
\define@key{fams}{ksy}{Indo-European}
\define@key{fams}{kfw}{Sino-Tibetan}
\define@key{fams}{lko}{Atlantic-Congo}
\define@key{fams}{kqg}{Atlantic-Congo}
\define@key{fams}{tlx}{Austronesian}
\define@key{fams}{xkf}{Sino-Tibetan}
\define@key{fams}{xhe}{Indo-European}
\define@key{fams}{nkh}{Sino-Tibetan}
\define@key{fams}{kix}{Sino-Tibetan}
\define@key{fams}{kwx}{Dravidian}
\define@key{fams}{kqm}{Atlantic-Congo}
\define@key{fams}{ykl}{Sino-Tibetan}
\define@key{fams}{xkc}{Indo-European}
\define@key{fams}{nkb}{Sino-Tibetan}
\define@key{fams}{ktc}{Afro-Asiatic}
\define@key{fams}{kho}{Indo-European}
\define@key{fams}{khf}{Austroasiatic}
\define@key{fams}{kfm}{Indo-European}
\define@key{fams}{xco}{Indo-European}
\define@key{fams}{kie}{Maban}
\define@key{fams}{prm}{Isolate}
\define@key{fams}{kzg}{Japonic}
\define@key{fams}{kih}{Border}
\define@key{fams}{kqr}{Austronesian}
\define@key{fams}{kmb}{Atlantic-Congo}
\define@key{fams}{kiv}{Atlantic-Congo}
\define@key{fams}{sbt}{Isolate}
\define@key{fams}{kqp}{Afro-Asiatic}
\define@key{fams}{krj}{Austronesian}
\define@key{fams}{kco}{Nuclear Trans New Guinea}
\define@key{fams}{cbw}{Austronesian}
\define@key{fams}{knq}{Austroasiatic}
\define@key{fams}{kkd}{Atlantic-Congo}
\define@key{fams}{ues}{Austronesian}
\define@key{fams}{kkm}{Atlantic-Congo}
\define@key{fams}{apk}{Athabaskan-Eyak-Tlingit}
\define@key{fams}{sgc}{Nilotic}
\define@key{fams}{kyi}{Austronesian}
\define@key{fams}{kkr}{Afro-Asiatic}
\define@key{fams}{okr}{Ijoid}
\define@key{fams}{kiu}{Indo-European}
\define@key{fams}{fkk}{Afro-Asiatic}
\define@key{fams}{lks}{Atlantic-Congo}
\define@key{fams}{kiz}{Atlantic-Congo}
\define@key{fams}{kis}{Austronesian}
\define@key{fams}{zkt}{Mongolic-Khitan}
\define@key{fams}{mwk}{Mande}
\define@key{fams}{mkw}{Atlantic-Congo}
\define@key{fams}{kqt}{Austronesian}
\define@key{fams}{tlh}{Artificial Language}
\define@key{fams}{kib}{Heibanic}
\define@key{fams}{kpd}{Austronesian}
\define@key{fams}{kcj}{Atlantic-Congo}
\define@key{fams}{kgu}{Nuclear Trans New Guinea}
\define@key{fams}{thq}{Indo-European}
\define@key{fams}{kdq}{Sino-Tibetan}
\define@key{fams}{dhw}{Indo-European}
\define@key{fams}{cdz}{Austroasiatic}
\define@key{fams}{ksz}{Austroasiatic}
\define@key{fams}{vko}{Austronesian}
\define@key{fams}{kwp}{Kru}
\define@key{fams}{kod}{Austronesian}
\define@key{fams}{kcs}{Afro-Asiatic}
\define@key{fams}{kpi}{Geelvink Bay}
\define@key{fams}{kwl}{Afro-Asiatic}
\define@key{fams}{zkg}{Unclassifiable}
\define@key{fams}{plk}{Indo-European}
\define@key{fams}{kkx}{Austronesian}
\define@key{fams}{kkt}{Sino-Tibetan}
\define@key{fams}{nkd}{Sino-Tibetan}
\define@key{fams}{kxt}{Ndu}
\define@key{fams}{kou}{Atlantic-Congo}
\define@key{fams}{gko}{Pama-Nyungan}
\define@key{fams}{xod}{South Bird's Head Family}
\define@key{fams}{kzn}{Atlantic-Congo}
\define@key{fams}{klc}{Atlantic-Congo}
\define@key{fams}{ekl}{Austroasiatic}
\define@key{fams}{biw}{Atlantic-Congo}
\define@key{fams}{skn}{Austronesian}
\define@key{fams}{klm}{Nuclear Trans New Guinea}
\define@key{fams}{kol}{Isolate}
\define@key{fams}{klx}{Austronesian}
\define@key{fams}{kmy}{Atlantic-Congo}
\define@key{fams}{kpf}{Nuclear Trans New Guinea}
\define@key{fams}{tyn}{Nuclear Trans New Guinea}
\define@key{fams}{kmm}{Sino-Tibetan}
\define@key{fams}{xoi}{Lower Sepik-Ramu}
\define@key{fams}{kmw}{Atlantic-Congo}
\define@key{fams}{kvh}{Austronesian}
\define@key{fams}{kvp}{Austronesian}
\define@key{fams}{kzv}{Nuclear Trans New Guinea}
\define@key{fams}{kxw}{East Strickland}
\define@key{fams}{knd}{Konda-Yahadian}
\define@key{fams}{kdw}{Mombum-Koneraw}
\define@key{fams}{klk}{Atlantic-Congo}
\define@key{fams}{kcz}{Atlantic-Congo}
\define@key{fams}{knu}{Mande}
\define@key{fams}{kno}{Mande}
\define@key{fams}{koa}{Austronesian}
\define@key{fams}{kxc}{Afro-Asiatic}
\define@key{fams}{nbe}{Sino-Tibetan}
\define@key{fams}{mku}{Mande}
\define@key{fams}{koo}{Atlantic-Congo}
\define@key{fams}{ozm}{Atlantic-Congo}
\define@key{fams}{fuj}{Heibanic}
\define@key{fams}{xop}{Lower Sepik-Ramu}
\define@key{fams}{opk}{Nuclear Trans New Guinea}
\define@key{fams}{kcy}{Songhay}
\define@key{fams}{koz}{Nuclear Trans New Guinea}
\define@key{fams}{okh}{Indo-European}
\define@key{fams}{vkp}{Indo-European}
\define@key{fams}{ktl}{Indo-European}
\define@key{fams}{krp}{Atlantic-Congo}
\define@key{fams}{kfo}{Mande}
\define@key{fams}{krf}{Austronesian}
\define@key{fams}{xkq}{Austronesian}
\define@key{fams}{kqj}{South Bougainville}
\define@key{fams}{jkr}{Sino-Tibetan}
\define@key{fams}{vkn}{Atlantic-Congo}
\define@key{fams}{vkz}{Atlantic-Congo}
\define@key{fams}{kfd}{Dravidian}
\define@key{fams}{kpq}{Nuclear Trans New Guinea}
\define@key{fams}{xor}{Pano-Tacanan}
\define@key{fams}{kfp}{Austroasiatic}
\define@key{fams}{kiq}{Kaure-Kosare}
\define@key{fams}{kid}{Atlantic-Congo}
\define@key{fams}{kqk}{Atlantic-Congo}
\define@key{fams}{koq}{Atlantic-Congo}
\define@key{fams}{mqg}{Austronesian}
\define@key{fams}{grm}{Austronesian}
\define@key{fams}{avk}{Artificial Language}
\define@key{fams}{zko}{Yeniseian}
\define@key{fams}{kyf}{Kru}
\define@key{fams}{kqb}{Nuclear Trans New Guinea}
\define@key{fams}{kvc}{Austronesian}
\define@key{fams}{xow}{Nuclear Trans New Guinea}
\define@key{fams}{kwh}{Austronesian}
\define@key{fams}{kga}{Mande}
\define@key{fams}{koh}{Atlantic-Congo}
\define@key{fams}{kqd}{Afro-Asiatic}
\define@key{fams}{kuw}{Atlantic-Congo}
\define@key{fams}{kpl}{Atlantic-Congo}
\define@key{fams}{pbn}{Atlantic-Congo}
\define@key{fams}{koc}{Atlantic-Congo}
\define@key{fams}{cpo}{Mande}
\define@key{fams}{kef}{Atlantic-Congo}
\define@key{fams}{kph}{Atlantic-Congo}
\define@key{fams}{kye}{Atlantic-Congo}
\define@key{fams}{rka}{Austroasiatic}
\define@key{fams}{xre}{Nuclear-Macro-Je}
\define@key{fams}{kri}{Indo-European}
\define@key{fams}{kxb}{Atlantic-Congo}
\define@key{fams}{tyu}{Khoe-Kwadi}
\define@key{fams}{yku}{Sino-Tibetan}
\define@key{fams}{uan}{Tai-Kadai}
\define@key{fams}{kua}{Atlantic-Congo}
\define@key{fams}{ykn}{Sino-Tibetan}
\define@key{fams}{ugh}{Nakh-Daghestanian}
\define@key{fams}{kgf}{Nuclear Trans New Guinea}
\define@key{fams}{kof}{Afro-Asiatic}
\define@key{fams}{jko}{East Strickland}
\define@key{fams}{kvb}{Austronesian}
\define@key{fams}{lkc}{Sino-Tibetan}
\define@key{fams}{kfg}{Dravidian}
\define@key{fams}{kyw}{Indo-European}
\define@key{fams}{kov}{Atlantic-Congo}
\define@key{fams}{kow}{Atlantic-Congo}
\define@key{fams}{kes}{Atlantic-Congo}
\define@key{fams}{dkr}{Austronesian}
\define@key{fams}{vkj}{Isolate}
\define@key{fams}{kux}{Pama-Nyungan}
\define@key{fams}{kez}{Atlantic-Congo}
\define@key{fams}{kfn}{Atlantic-Congo}
\define@key{fams}{ugb}{Pama-Nyungan}
\define@key{fams}{xmp}{Pama-Nyungan}
\define@key{fams}{xmh}{Pama-Nyungan}
\define@key{fams}{ukv}{Nilotic}
\define@key{fams}{kul}{Afro-Asiatic}
\define@key{fams}{kxj}{Central Sudanic}
\define@key{fams}{vkl}{Austronesian}
\define@key{fams}{xpk}{Pano-Tacanan}
\define@key{fams}{kfx}{Indo-European}
\define@key{fams}{pzh}{Austronesian}
\define@key{fams}{uon}{Austronesian}
\define@key{fams}{bbu}{Atlantic-Congo}
\define@key{fams}{kdi}{Nilotic}
\define@key{fams}{ksl}{Austronesian}
\define@key{fams}{ksm}{Atlantic-Congo}
\define@key{fams}{xks}{Austronesian}
\define@key{fams}{kra}{Indo-European}
\define@key{fams}{kuo}{Nuclear Trans New Guinea}
\define@key{fams}{zum}{Indo-European}
\define@key{fams}{wku}{Dravidian}
\define@key{fams}{kdn}{Atlantic-Congo}
\define@key{fams}{shd}{Indo-European}
\define@key{fams}{kgl}{Pama-Nyungan}
\define@key{fams}{ggk}{Isolate}
\define@key{fams}{kfl}{Atlantic-Congo}
\define@key{fams}{kse}{Austronesian}
\define@key{fams}{xug}{Japonic}
\define@key{fams}{pep}{Yam}
\define@key{fams}{njx}{Atlantic-Congo}
\define@key{fams}{kug}{Atlantic-Congo}
\define@key{fams}{mkn}{Austronesian}
\define@key{fams}{key}{Indo-European}
\define@key{fams}{nqk}{Atlantic-Congo}
\define@key{fams}{krh}{Atlantic-Congo}
\define@key{fams}{kfh}{Dravidian}
\define@key{fams}{kuj}{Atlantic-Congo}
\define@key{fams}{nbn}{Austronesian}
\define@key{fams}{kfv}{Indo-European}
\define@key{fams}{vku}{Pama-Nyungan}
\define@key{fams}{kuv}{Austronesian}
\define@key{fams}{xkz}{Sino-Tibetan}
\define@key{fams}{ktm}{Austronesian}
\define@key{fams}{kjr}{Austronesian}
\define@key{fams}{kyr}{Tupian}
\define@key{fams}{kus}{Atlantic-Congo}
\define@key{fams}{ksg}{Austronesian}
\define@key{fams}{kuh}{Afro-Asiatic}
\define@key{fams}{ksv}{Atlantic-Congo}
\define@key{fams}{ght}{Sino-Tibetan}
\define@key{fams}{kub}{Atlantic-Congo}
\define@key{fams}{xut}{Pama-Nyungan}
\define@key{fams}{kpa}{Afro-Asiatic}
\define@key{fams}{khj}{Atlantic-Congo}
\define@key{fams}{kdc}{Atlantic-Congo}
\define@key{fams}{uky}{Pama-Nyungan}
\define@key{fams}{lku}{Pama-Nyungan}
\define@key{fams}{olu}{Atlantic-Congo}
\define@key{fams}{cwt}{Atlantic-Congo}
\define@key{fams}{blh}{Kru}
\define@key{fams}{kdt}{Austroasiatic}
\define@key{fams}{fkv}{Uralic}
\define@key{fams}{kwb}{Atlantic-Congo}
\define@key{fams}{bko}{Atlantic-Congo}
\define@key{fams}{kwz}{Khoe-Kwadi}
\define@key{fams}{wka}{Afro-Asiatic}
\define@key{fams}{kdz}{Atlantic-Congo}
\define@key{fams}{kwu}{Atlantic-Congo}
\define@key{fams}{qwt}{Athabaskan-Eyak-Tlingit}
\define@key{fams}{kmq}{Koman}
\define@key{fams}{ktf}{Atlantic-Congo}
\define@key{fams}{kwm}{Atlantic-Congo}
\define@key{fams}{okk}{Nuclear Torricelli}
\define@key{fams}{knp}{Atlantic-Congo}
\define@key{fams}{kwj}{Sepik}
\define@key{fams}{kvi}{Afro-Asiatic}
\define@key{fams}{xdo}{Atlantic-Congo}
\define@key{fams}{kwf}{Austronesian}
\define@key{fams}{kop}{Nuclear Trans New Guinea}
\define@key{fams}{kya}{Atlantic-Congo}
\define@key{fams}{cwe}{Atlantic-Congo}
\define@key{fams}{xwr}{Greater Kwerba}
\define@key{fams}{kkb}{Lakes Plain}
\define@key{fams}{kwr}{Nuclear Trans New Guinea}
\define@key{fams}{kws}{Atlantic-Congo}
\define@key{fams}{kwt}{Tor-Orya}
\define@key{fams}{kuc}{Tor-Orya}
\define@key{fams}{kww}{Indo-European}
\define@key{fams}{bka}{Atlantic-Congo}
\define@key{fams}{tye}{Mande}
\define@key{fams}{kql}{Yuat}
\define@key{fams}{ldn}{Artificial Language}
\define@key{fams}{bwj}{Atlantic-Congo}
\define@key{fams}{ldi}{Atlantic-Congo}
\define@key{fams}{lbb}{Austronesian}
\define@key{fams}{lbi}{Speech Register}
\define@key{fams}{jku}{Atlantic-Congo}
\define@key{fams}{ypb}{Sino-Tibetan}
\define@key{fams}{mwi}{Austronesian}
\define@key{fams}{dtb}{Austronesian}
\define@key{fams}{zpl}{Otomanguean}
\define@key{fams}{zpa}{Otomanguean}
\define@key{fams}{lkl}{Nuclear Torricelli}
\define@key{fams}{lgh}{Sino-Tibetan}
\define@key{fams}{lgb}{Austronesian}
\define@key{fams}{lhh}{Austronesian}
\define@key{fams}{lhn}{Austronesian}
\define@key{fams}{lhl}{Indo-European}
\define@key{fams}{lhi}{Sino-Tibetan}
\define@key{fams}{lmx}{Atlantic-Congo}
\define@key{fams}{lji}{Austronesian}
\define@key{fams}{lap}{Central Sudanic}
\define@key{fams}{lka}{Austronesian}
\define@key{fams}{lkh}{Sino-Tibetan}
\define@key{fams}{lki}{Indo-European}
\define@key{fams}{lkn}{Austronesian}
\define@key{fams}{lkd}{Nambiquaran}
\define@key{fams}{lxm}{Austronesian}
\define@key{fams}{lla}{Atlantic-Congo}
\define@key{fams}{leb}{Atlantic-Congo}
\define@key{fams}{cnl}{Otomanguean}
\define@key{fams}{las}{Atlantic-Congo}
\define@key{fams}{lmr}{Austronesian}
\define@key{fams}{lmq}{Austronesian}
\define@key{fams}{lai}{Atlantic-Congo}
\define@key{fams}{lmy}{Austronesian}
\define@key{fams}{quf}{Quechuan}
\define@key{fams}{lbn}{Austroasiatic}
\define@key{fams}{bma}{Atlantic-Congo}
\define@key{fams}{ldh}{Atlantic-Congo}
\define@key{fams}{lmk}{Sino-Tibetan}
\define@key{fams}{lev}{Timor-Alor-Pantar}
\define@key{fams}{lmg}{Austronesian}
\define@key{fams}{abl}{Austronesian}
\define@key{fams}{llh}{Sino-Tibetan}
\define@key{fams}{ruu}{Austronesian}
\define@key{fams}{ldm}{Atlantic-Congo}
\define@key{fams}{sfb}{Sign Language}
\define@key{fams}{yln}{Tai-Kadai}
\define@key{fams}{lna}{Atlantic-Congo}
\define@key{fams}{lno}{Nilotic}
\define@key{fams}{lnm}{Keram}
\define@key{fams}{lnh}{Austroasiatic}
\define@key{fams}{lwm}{Sino-Tibetan}
\define@key{fams}{ztl}{Otomanguean}
\define@key{fams}{laa}{Austronesian}
\define@key{fams}{lrt}{Austronesian}
\define@key{fams}{lrv}{Austronesian}
\define@key{fams}{hmd}{Hmong-Mien}
\define@key{fams}{lrl}{Indo-European}
\define@key{fams}{lro}{Heibanic}
\define@key{fams}{lar}{Atlantic-Congo}
\define@key{fams}{lan}{Atlantic-Congo}
\define@key{fams}{llm}{Austronesian}
\define@key{fams}{lsa}{Indo-European}
\define@key{fams}{lsi}{Sino-Tibetan}
\define@key{fams}{lss}{Indo-European}
\define@key{fams}{lat}{Indo-European}
\define@key{fams}{ltu}{Austronesian}
\define@key{fams}{ltn}{Nambiquaran}
\define@key{fams}{lsl}{Sign Language}
\define@key{fams}{llx}{Austronesian}
\define@key{fams}{luf}{Mailuan}
\define@key{fams}{lre}{Iroquoian}
\define@key{fams}{clt}{Sino-Tibetan}
\define@key{fams}{lbv}{Austronesian}
\define@key{fams}{lbx}{Austronesian}
\define@key{fams}{lvi}{Austroasiatic}
\define@key{fams}{tgi}{Austronesian}
\define@key{fams}{lwu}{Sino-Tibetan}
\define@key{fams}{lya}{Sino-Tibetan}
\define@key{fams}{ldk}{Atlantic-Congo}
\define@key{fams}{lfa}{Atlantic-Congo}
\define@key{fams}{lgm}{Atlantic-Congo}
\define@key{fams}{lcc}{Austronesian}
\define@key{fams}{cae}{Atlantic-Congo}
\define@key{fams}{tql}{Austronesian}
\define@key{fams}{urr}{Austronesian}
\define@key{fams}{lzn}{Sino-Tibetan}
\define@key{fams}{lek}{Austronesian}
\define@key{fams}{llk}{Austronesian}
\define@key{fams}{lel}{Atlantic-Congo}
\define@key{fams}{llc}{Mande}
\define@key{fams}{lpa}{Austronesian}
\define@key{fams}{lle}{Austronesian}
\define@key{fams}{leq}{Nuclear Trans New Guinea}
\define@key{fams}{lrz}{Austronesian}
\define@key{fams}{lei}{Nuclear Trans New Guinea}
\define@key{fams}{xle}{Unclassifiable}
\define@key{fams}{ldj}{Atlantic-Congo}
\define@key{fams}{ley}{Austronesian}
\define@key{fams}{lej}{Atlantic-Congo}
\define@key{fams}{lgr}{Austronesian}
\define@key{fams}{lgi}{Austronesian}
\define@key{fams}{leh}{Atlantic-Congo}
\define@key{fams}{ler}{Austronesian}
\define@key{fams}{ldg}{Atlantic-Congo}
\define@key{fams}{lpe}{Lepki-Murkim-Kembra}
\define@key{fams}{xlp}{Indo-European}
\define@key{fams}{gnh}{Atlantic-Congo}
\define@key{fams}{let}{Austronesian}
\define@key{fams}{nms}{Austronesian}
\define@key{fams}{leo}{Atlantic-Congo}
\define@key{fams}{lvu}{Austronesian}
\define@key{fams}{lwe}{Austronesian}
\define@key{fams}{lwt}{Austronesian}
\define@key{fams}{ayi}{Atlantic-Congo}
\define@key{fams}{lhp}{Sino-Tibetan}
\define@key{fams}{lix}{Austronesian}
\define@key{fams}{njn}{Sino-Tibetan}
\define@key{fams}{zln}{Tai-Kadai}
\define@key{fams}{ste}{Austronesian}
\define@key{fams}{lir}{Pidgin}
\define@key{fams}{liz}{Atlantic-Congo}
\define@key{fams}{liq}{Afro-Asiatic}
\define@key{fams}{lbs}{Sign Language}
\define@key{fams}{lig}{Mande}
\define@key{fams}{lgz}{Atlantic-Congo}
\define@key{fams}{lih}{Austronesian}
\define@key{fams}{mgi}{Atlantic-Congo}
\define@key{fams}{lik}{Atlantic-Congo}
\define@key{fams}{lie}{Atlantic-Congo}
\define@key{fams}{lio}{Austronesian}
\define@key{fams}{kxx}{Atlantic-Congo}
\define@key{fams}{lib}{Austronesian}
\define@key{fams}{kwc}{Atlantic-Congo}
\define@key{fams}{lll}{Bogia}
\define@key{fams}{bme}{Atlantic-Congo}
\define@key{fams}{lim}{Indo-European}
\define@key{fams}{lmp}{Atlantic-Congo}
\define@key{fams}{ylm}{Sino-Tibetan}
\define@key{fams}{kmk}{Austronesian}
\define@key{fams}{qlm}{Indo-European}
\define@key{fams}{klw}{Austronesian}
\define@key{fams}{pml}{Pidgin}
\define@key{fams}{onb}{Tai-Kadai}
\define@key{fams}{lgk}{Austronesian}
\define@key{fams}{lfn}{Artificial Language}
\define@key{fams}{ljl}{Austronesian}
\define@key{fams}{apl}{Athabaskan-Eyak-Tlingit}
\define@key{fams}{lpo}{Sino-Tibetan}
\define@key{fams}{lcs}{Austronesian}
\define@key{fams}{lcl}{Austronesian}
\define@key{fams}{lsh}{Sino-Tibetan}
\define@key{fams}{lsd}{Afro-Asiatic}
\define@key{fams}{lzh}{Sino-Tibetan}
\define@key{fams}{lls}{Sign Language}
\define@key{fams}{lzl}{Austronesian}
\define@key{fams}{zlj}{Tai-Kadai}
\define@key{fams}{zlq}{Tai-Kadai}
\define@key{fams}{olo}{Uralic}
\define@key{fams}{loq}{Atlantic-Congo}
\define@key{fams}{lbm}{Indo-European}
\define@key{fams}{lgq}{Atlantic-Congo}
\define@key{fams}{rag}{Atlantic-Congo}
\define@key{fams}{liu}{Dajuic}
\define@key{fams}{lof}{Heibanic}
\define@key{fams}{src}{Indo-European}
\define@key{fams}{qvj}{Quechuan}
\define@key{fams}{jbo}{Artificial Language}
\define@key{fams}{yaz}{Atlantic-Congo}
\define@key{fams}{lky}{Nilotic}
\define@key{fams}{lcd}{Austronesian}
\define@key{fams}{llq}{Austronesian}
\define@key{fams}{llg}{Austronesian}
\define@key{fams}{ycl}{Sino-Tibetan}
\define@key{fams}{llb}{Atlantic-Congo}
\define@key{fams}{loa}{North Halmahera}
\define@key{fams}{rmi}{Speech Register}
\define@key{fams}{loi}{Atlantic-Congo}
\define@key{fams}{lmv}{Austronesian}
\define@key{fams}{lmi}{Central Sudanic}
\define@key{fams}{lmo}{Indo-European}
\define@key{fams}{loo}{Atlantic-Congo}
\define@key{fams}{ngl}{Atlantic-Congo}
\define@key{fams}{lce}{Austronesian}
\define@key{fams}{lpn}{Sino-Tibetan}
\define@key{fams}{wok}{Atlantic-Congo}
\define@key{fams}{lnu}{Atlantic-Congo}
\define@key{fams}{ttw}{Austronesian}
\define@key{fams}{ldo}{Atlantic-Congo}
\define@key{fams}{lop}{Atlantic-Congo}
\define@key{fams}{lpx}{Nilotic}
\define@key{fams}{lrn}{Austronesian}
\define@key{fams}{spq}{Indo-European}
\define@key{fams}{lnn}{Austronesian}
\define@key{fams}{uvl}{Austronesian}
\define@key{fams}{lht}{Austronesian}
\define@key{fams}{dtr}{Austronesian}
\define@key{fams}{lou}{Indo-European}
\define@key{fams}{lox}{Austronesian}
\define@key{fams}{xlo}{Algic}
\define@key{fams}{sli}{Indo-European}
\define@key{fams}{tto}{Austroasiatic}
\define@key{fams}{nsb}{Tuu}
\define@key{fams}{kml}{Austronesian}
\define@key{fams}{cea}{Salishan}
\define@key{fams}{axl}{Pama-Nyungan}
\define@key{fams}{ztp}{Otomanguean}
\define@key{fams}{kcc}{Atlantic-Congo}
\define@key{fams}{lcf}{Austronesian}
\define@key{fams}{knb}{Austronesian}
\define@key{fams}{luq}{Atlantic-Congo}
\define@key{fams}{lud}{Uralic}
\define@key{fams}{ldq}{Atlantic-Congo}
\define@key{fams}{ruf}{Atlantic-Congo}
\define@key{fams}{lcq}{Austronesian}
\define@key{fams}{lum}{Atlantic-Congo}
\define@key{fams}{dop}{Atlantic-Congo}
\define@key{fams}{smj}{Uralic}
\define@key{fams}{lmz}{Unattested}
\define@key{fams}{lup}{Atlantic-Congo}
\define@key{fams}{lmd}{Narrow Talodi}
\define@key{fams}{luk}{Sino-Tibetan}
\define@key{fams}{luj}{Atlantic-Congo}
\define@key{fams}{lga}{Austronesian}
\define@key{fams}{luw}{Atlantic-Congo}
\define@key{fams}{hml}{Hmong-Mien}
\define@key{fams}{ldd}{Afro-Asiatic}
\define@key{fams}{lse}{Atlantic-Congo}
\define@key{fams}{xls}{Indo-European}
\define@key{fams}{ndy}{Central Sudanic}
\define@key{fams}{luv}{Indo-European}
\define@key{fams}{lyn}{Atlantic-Congo}
\define@key{fams}{lwa}{Atlantic-Congo}
\define@key{fams}{xlc}{Indo-European}
\define@key{fams}{xld}{Indo-European}
\define@key{fams}{lyg}{Austroasiatic}
\define@key{fams}{cma}{Austroasiatic}
\define@key{fams}{mew}{Afro-Asiatic}
\define@key{fams}{ymm}{Afro-Asiatic}
\define@key{fams}{mmz}{Atlantic-Congo}
\define@key{fams}{mfz}{Nilotic}
\define@key{fams}{mqa}{Austronesian}
\define@key{fams}{kkg}{Austronesian}
\define@key{fams}{muj}{Afro-Asiatic}
\define@key{fams}{mcl}{Tucanoan}
\define@key{fams}{mzs}{Indo-European}
\define@key{fams}{mvw}{Atlantic-Congo}
\define@key{fams}{jmc}{Atlantic-Congo}
\define@key{fams}{mpd}{Arawakan}
\define@key{fams}{wpc}{Saliban}
\define@key{fams}{mzc}{Sign Language}
\define@key{fams}{mmx}{Austronesian}
\define@key{fams}{xmx}{Austronesian}
\define@key{fams}{grg}{Nuclear Trans New Guinea}
\define@key{fams}{kmd}{Austronesian}
\define@key{fams}{mme}{Austronesian}
\define@key{fams}{itt}{Austronesian}
\define@key{fams}{maf}{Afro-Asiatic}
\define@key{fams}{mkv}{Austronesian}
\define@key{fams}{sgb}{Austronesian}
\define@key{fams}{mtw}{Austronesian}
\define@key{fams}{xtm}{Otomanguean}
\define@key{fams}{gmd}{Atlantic-Congo}
\define@key{fams}{blx}{Austronesian}
\define@key{fams}{gkd}{Nuclear Trans New Guinea}
\define@key{fams}{gmg}{Nuclear Trans New Guinea}
\define@key{fams}{gmx}{Atlantic-Congo}
\define@key{fams}{zgr}{Austronesian}
\define@key{fams}{bfz}{Indo-European}
\define@key{fams}{mjx}{Austroasiatic}
\define@key{fams}{pmh}{Indo-European}
\define@key{fams}{mjy}{Algic}
\define@key{fams}{mhb}{Atlantic-Congo}
\define@key{fams}{mzz}{Austronesian}
\define@key{fams}{tnh}{Nuclear Trans New Guinea}
\define@key{fams}{sks}{Nuclear Trans New Guinea}
\define@key{fams}{mmm}{Austronesian}
\define@key{fams}{vmf}{Indo-European}
\define@key{fams}{cwb}{Atlantic-Congo}
\define@key{fams}{xkl}{Austronesian}
\define@key{fams}{mum}{Austronesian}
\define@key{fams}{wmm}{Austronesian}
\define@key{fams}{mti}{Dagan}
\define@key{fams}{xmj}{Afro-Asiatic}
\define@key{fams}{mmj}{Austroasiatic}
\define@key{fams}{mjz}{Indo-European}
\define@key{fams}{mfp}{Austronesian}
\define@key{fams}{aup}{Anim}
\define@key{fams}{mkg}{Tai-Kadai}
\define@key{fams}{vmk}{Atlantic-Congo}
\define@key{fams}{xmc}{Atlantic-Congo}
\define@key{fams}{vmw}{Atlantic-Congo}
\define@key{fams}{mhm}{Atlantic-Congo}
\define@key{fams}{xsq}{Atlantic-Congo}
\define@key{fams}{pbl}{Atlantic-Congo}
\define@key{fams}{zmh}{Baining}
\define@key{fams}{jmn}{Sino-Tibetan}
\define@key{fams}{lva}{Austronesian}
\define@key{fams}{mpu}{Tupian}
\define@key{fams}{ymk}{Atlantic-Congo}
\define@key{fams}{umn}{Sino-Tibetan}
\define@key{fams}{lon}{Atlantic-Congo}
\define@key{fams}{xml}{Sign Language}
\define@key{fams}{ima}{Dravidian}
\define@key{fams}{ymr}{Dravidian}
\define@key{fams}{mjo}{Dravidian}
\define@key{fams}{mjr}{Dravidian}
\define@key{fams}{mjq}{Dravidian}
\define@key{fams}{mjp}{Dravidian}
\define@key{fams}{ruy}{Unattested}
\define@key{fams}{swk}{Atlantic-Congo}
\define@key{fams}{ccm}{Austronesian}
\define@key{fams}{mln}{Austronesian}
\define@key{fams}{mqz}{Austronesian}
\define@key{fams}{mmt}{Austronesian}
\define@key{fams}{ped}{Nuclear Trans New Guinea}
\define@key{fams}{mkr}{Nuclear Trans New Guinea}
\define@key{fams}{lws}{Artificial Language}
\define@key{fams}{bfo}{Atlantic-Congo}
\define@key{fams}{pkt}{Austroasiatic}
\define@key{fams}{mdc}{Nuclear Trans New Guinea}
\define@key{fams}{gut}{Chibchan}
\define@key{fams}{mlx}{Austronesian}
\define@key{fams}{vml}{Pama-Nyungan}
\define@key{fams}{mxf}{Afro-Asiatic}
\define@key{fams}{mgq}{Atlantic-Congo}
\define@key{fams}{mzd}{Atlantic-Congo}
\define@key{fams}{mli}{Austronesian}
\define@key{fams}{mlf}{Austroasiatic}
\define@key{fams}{mbk}{Austronesian}
\define@key{fams}{mkb}{Indo-European}
\define@key{fams}{mdl}{Sign Language}
\define@key{fams}{mll}{Austronesian}
\define@key{fams}{mup}{Indo-European}
\define@key{fams}{myk}{Atlantic-Congo}
\define@key{fams}{mma}{Atlantic-Congo}
\define@key{fams}{mhf}{Nuclear Trans New Guinea}
\define@key{fams}{wmd}{Nambiquaran}
\define@key{fams}{mvd}{Austronesian}
\define@key{fams}{mgm}{Austronesian}
\define@key{fams}{kdf}{Austronesian}
\define@key{fams}{mqx}{Austronesian}
\define@key{fams}{znk}{Unattested}
\define@key{fams}{mjl}{Indo-European}
\define@key{fams}{mha}{Dravidian}
\define@key{fams}{zma}{Western Daly}
\define@key{fams}{zmk}{Pama-Nyungan}
\define@key{fams}{mgs}{Atlantic-Congo}
\define@key{fams}{mqu}{Nilotic}
\define@key{fams}{tbf}{Austronesian}
\define@key{fams}{mqr}{Tor-Orya}
\define@key{fams}{aax}{Nuclear Trans New Guinea}
\define@key{fams}{bwp}{Nuclear Trans New Guinea}
\define@key{fams}{mht}{Arawakan}
\define@key{fams}{zng}{Austroasiatic}
\define@key{fams}{zme}{Giimbiyu}
\define@key{fams}{mem}{Pama-Nyungan}
\define@key{fams}{myj}{Atlantic-Congo}
\define@key{fams}{mdk}{Central Sudanic}
\define@key{fams}{kby}{Saharan}
\define@key{fams}{mrv}{Austronesian}
\define@key{fams}{mbh}{Austronesian}
\define@key{fams}{mmo}{Austronesian}
\define@key{fams}{zns}{Afro-Asiatic}
\define@key{fams}{xkb}{Atlantic-Congo}
\define@key{fams}{mqp}{Austronesian}
\define@key{fams}{nlm}{Indo-European}
\define@key{fams}{mml}{Austroasiatic}
\define@key{fams}{mjv}{Dravidian}
\define@key{fams}{woo}{Austronesian}
\define@key{fams}{msw}{Atlantic-Congo}
\define@key{fams}{msk}{Austronesian}
\define@key{fams}{nty}{Sino-Tibetan}
\define@key{fams}{myg}{Atlantic-Congo}
\define@key{fams}{kxf}{Sino-Tibetan}
\define@key{fams}{wha}{Austronesian}
\define@key{fams}{mxc}{Atlantic-Congo}
\define@key{fams}{mny}{Atlantic-Congo}
\define@key{fams}{mzj}{Mande}
\define@key{fams}{mzv}{Atlantic-Congo}
\define@key{fams}{mmd}{Tai-Kadai}
\define@key{fams}{mjn}{Nuclear Trans New Guinea}
\define@key{fams}{mlh}{Nuclear Trans New Guinea}
\define@key{fams}{mnm}{Dagan}
\define@key{fams}{mpy}{Austronesian}
\define@key{fams}{mpw}{Arawakan}
\define@key{fams}{bzh}{Austronesian}
\define@key{fams}{sjm}{Austronesian}
\define@key{fams}{vmh}{Indo-European}
\define@key{fams}{nma}{Sino-Tibetan}
\define@key{fams}{lrm}{Atlantic-Congo}
\define@key{fams}{lri}{Atlantic-Congo}
\define@key{fams}{mgb}{Tamaic}
\define@key{fams}{mvr}{Austronesian}
\define@key{fams}{mrs}{Austronesian}
\define@key{fams}{mpg}{Afro-Asiatic}
\define@key{fams}{dsz}{Sign Language}
\define@key{fams}{vmr}{Atlantic-Congo}
\define@key{fams}{mrx}{Tor-Orya}
\define@key{fams}{mvu}{Maban}
\define@key{fams}{mhg}{Marrku-Wurrugu}
\define@key{fams}{qvm}{Quechuan}
\define@key{fams}{mfm}{Afro-Asiatic}
\define@key{fams}{nsr}{Sign Language}
\define@key{fams}{mrr}{Dravidian}
\define@key{fams}{nng}{Sino-Tibetan}
\define@key{fams}{zmm}{Western Daly}
\define@key{fams}{zmj}{Western Daly}
\define@key{fams}{zmd}{Western Daly}
\define@key{fams}{zmy}{Western Daly}
\define@key{fams}{mrb}{Austronesian}
\define@key{fams}{dad}{Austronesian}
\define@key{fams}{hob}{Austronesian}
\define@key{fams}{mqi}{Austronesian}
\define@key{fams}{mbx}{Sepik}
\define@key{fams}{mds}{Manubaran}
\define@key{fams}{msp}{Tupian}
\define@key{fams}{enb}{Nilotic}
\define@key{fams}{rkm}{Mande}
\define@key{fams}{mvo}{Austronesian}
\define@key{fams}{xru}{Western Daly}
\define@key{fams}{mre}{Sign Language}
\define@key{fams}{zmg}{Western Daly}
\define@key{fams}{mzr}{Pano-Tacanan}
\define@key{fams}{mve}{Indo-European}
\define@key{fams}{rwr}{Indo-European}
\define@key{fams}{myx}{Atlantic-Congo}
\define@key{fams}{tis}{Austronesian}
\define@key{fams}{bks}{Austronesian}
\define@key{fams}{msb}{Austronesian}
\define@key{fams}{mho}{Atlantic-Congo}
\define@key{fams}{jms}{Atlantic-Congo}
\define@key{fams}{cuj}{Arawakan}
\define@key{fams}{ism}{Austronesian}
\define@key{fams}{bnf}{Austronesian}
\define@key{fams}{msh}{Austronesian}
\define@key{fams}{klv}{Austronesian}
\define@key{fams}{msv}{Afro-Asiatic}
\define@key{fams}{mes}{Afro-Asiatic}
\define@key{fams}{mdg}{Maban}
\define@key{fams}{mvs}{Isolate}
\define@key{fams}{mtn}{Misumalpan}
\define@key{fams}{mfh}{Afro-Asiatic}
\define@key{fams}{xmt}{Austronesian}
\define@key{fams}{mgv}{Atlantic-Congo}
\define@key{fams}{mqe}{Nuclear Trans New Guinea}
\define@key{fams}{mzo}{Cariban}
\define@key{fams}{mtm}{Uralic}
\define@key{fams}{met}{Austronesian}
\define@key{fams}{axg}{Isolate}
\define@key{fams}{stj}{Mande}
\define@key{fams}{cty}{Dravidian}
\define@key{fams}{lsy}{Sign Language}
\define@key{fams}{mhl}{Nuclear Trans New Guinea}
\define@key{fams}{wma}{Unattested}
\define@key{fams}{mjj}{Nuclear Trans New Guinea}
\define@key{fams}{mcz}{Nuclear Trans New Guinea}
\define@key{fams}{mcw}{Afro-Asiatic}
\define@key{fams}{mgk}{Isolate}
\define@key{fams}{mxl}{Atlantic-Congo}
\define@key{fams}{xmy}{Pama-Nyungan}
\define@key{fams}{sym}{Mande}
\define@key{fams}{mnt}{Pama-Nyungan}
\define@key{fams}{ifu}{Austronesian}
\define@key{fams}{mzl}{Mixe-Zoque}
\define@key{fams}{zpy}{Otomanguean}
\define@key{fams}{vmz}{Otomanguean}
\define@key{fams}{dkx}{Afro-Asiatic}
\define@key{fams}{mdp}{Atlantic-Congo}
\define@key{fams}{mgn}{Atlantic-Congo}
\define@key{fams}{zmz}{Atlantic-Congo}
\define@key{fams}{mxg}{Atlantic-Congo}
\define@key{fams}{zmn}{Atlantic-Congo}
\define@key{fams}{zmv}{Pama-Nyungan}
\define@key{fams}{mvl}{Pama-Nyungan}
\define@key{fams}{gwa}{Atlantic-Congo}
\define@key{fams}{mdn}{Atlantic-Congo}
\define@key{fams}{xmd}{Afro-Asiatic}
\define@key{fams}{mfo}{Atlantic-Congo}
\define@key{fams}{mql}{Atlantic-Congo}
\define@key{fams}{zms}{Atlantic-Congo}
\define@key{fams}{emz}{Atlantic-Congo}
\define@key{fams}{mbo}{Atlantic-Congo}
\define@key{fams}{zmw}{Atlantic-Congo}
\define@key{fams}{moi}{Atlantic-Congo}
\define@key{fams}{mdu}{Atlantic-Congo}
\define@key{fams}{xmb}{Atlantic-Congo}
\define@key{fams}{bgu}{Atlantic-Congo}
\define@key{fams}{mxo}{Atlantic-Congo}
\define@key{fams}{mka}{Atlantic-Congo}
\define@key{fams}{mgz}{Atlantic-Congo}
\define@key{fams}{mhw}{Atlantic-Congo}
\define@key{fams}{mqb}{Afro-Asiatic}
\define@key{fams}{bpc}{Atlantic-Congo}
\define@key{fams}{mbv}{Atlantic-Congo}
\define@key{fams}{mbu}{Atlantic-Congo}
\define@key{fams}{mlb}{Atlantic-Congo}
\define@key{fams}{mgy}{Atlantic-Congo}
\define@key{fams}{mck}{Atlantic-Congo}
\define@key{fams}{bbt}{Afro-Asiatic}
\define@key{fams}{muc}{Atlantic-Congo}
\define@key{fams}{mfu}{Atlantic-Congo}
\define@key{fams}{gun}{Tupian}
\define@key{fams}{mjm}{Austronesian}
\define@key{fams}{dmf}{Speech Register}
\define@key{fams}{mue}{Mixed Language}
\define@key{fams}{mud}{Eskimo-Aleut}
\define@key{fams}{byv}{Atlantic-Congo}
\define@key{fams}{mfj}{Afro-Asiatic}
\define@key{fams}{mef}{Austroasiatic}
\define@key{fams}{ruq}{Indo-European}
\define@key{fams}{mmh}{Arawakan}
\define@key{fams}{mvk}{Yuat}
\define@key{fams}{msf}{Nimboranic}
\define@key{fams}{hkn}{Austroasiatic}
\define@key{fams}{mfx}{Ta-Ne-Omotic}
\define@key{fams}{med}{Nuclear Trans New Guinea}
\define@key{fams}{mby}{Indo-European}
\define@key{fams}{mfd}{Atlantic-Congo}
\define@key{fams}{xkd}{Austronesian}
\define@key{fams}{sim}{Sepik}
\define@key{fams}{xmg}{Atlantic-Congo}
\define@key{fams}{mee}{Austronesian}
\define@key{fams}{mea}{Atlantic-Congo}
\define@key{fams}{mvx}{Austronesian}
\define@key{fams}{mxm}{Austronesian}
\define@key{fams}{lmb}{Austronesian}
\define@key{fams}{meq}{Afro-Asiatic}
\define@key{fams}{mrm}{Austronesian}
\define@key{fams}{xmr}{Isolate}
\define@key{fams}{mnu}{Mairasic}
\define@key{fams}{mer}{Atlantic-Congo}
\define@key{fams}{wry}{Indo-European}
\define@key{fams}{iyo}{Atlantic-Congo}
\define@key{fams}{mci}{Nuclear Trans New Guinea}
\define@key{fams}{zim}{Afro-Asiatic}
\define@key{fams}{mys}{Afro-Asiatic}
\define@key{fams}{mvz}{Afro-Asiatic}
\define@key{fams}{cms}{Indo-European}
\define@key{fams}{mgo}{Atlantic-Congo}
\define@key{fams}{mxv}{Otomanguean}
\define@key{fams}{mtr}{Indo-European}
\define@key{fams}{wtm}{Indo-European}
\define@key{fams}{mfs}{Sign Language}
\define@key{fams}{zmf}{Atlantic-Congo}
\define@key{fams}{nfu}{Atlantic-Congo}
\define@key{fams}{zam}{Otomanguean}
\define@key{fams}{pla}{Nuclear Trans New Guinea}
\define@key{fams}{xmi}{Unattested}
\define@key{fams}{mmc}{Otomanguean}
\define@key{fams}{enm}{Indo-European}
\define@key{fams}{gml}{Indo-European}
\define@key{fams}{dum}{Indo-European}
\define@key{fams}{mpl}{Austronesian}
\define@key{fams}{gmh}{Indo-European}
\define@key{fams}{ltc}{Sino-Tibetan}
\define@key{fams}{xng}{Mongolic-Khitan}
\define@key{fams}{dnt}{Nuclear Trans New Guinea}
\define@key{fams}{bjo}{Atlantic-Congo}
\define@key{fams}{mpp}{Nuclear Trans New Guinea}
\define@key{fams}{ymh}{Sino-Tibetan}
\define@key{fams}{mlj}{Afro-Asiatic}
\define@key{fams}{iml}{Coosan}
\define@key{fams}{imy}{Indo-European}
\define@key{fams}{mcv}{Anim}
\define@key{fams}{inm}{Afro-Asiatic}
\define@key{fams}{mnp}{Sino-Tibetan}
\define@key{fams}{mpn}{Austronesian}
\define@key{fams}{drc}{Indo-European}
\define@key{fams}{mko}{Atlantic-Congo}
\define@key{fams}{vmg}{Austronesian}
\define@key{fams}{wii}{Nuclear Torricelli}
\define@key{fams}{xxm}{Isolate}
\define@key{fams}{omn}{Unclassifiable}
\define@key{fams}{mqq}{Austronesian}
\define@key{fams}{mnq}{Austroasiatic}
\define@key{fams}{mzt}{Austroasiatic}
\define@key{fams}{czo}{Sino-Tibetan}
\define@key{fams}{zgm}{Tai-Kadai}
\define@key{fams}{yiq}{Sino-Tibetan}
\define@key{fams}{mwl}{Indo-European}
\define@key{fams}{mvh}{Afro-Asiatic}
\define@key{fams}{mmv}{Tucanoan}
\define@key{fams}{rsm}{Sign Language}
\define@key{fams}{mjs}{Afro-Asiatic}
\define@key{fams}{mpx}{Austronesian}
\define@key{fams}{vmm}{Otomanguean}
\define@key{fams}{mwu}{Central Sudanic}
\define@key{fams}{mpo}{Austronesian}
\define@key{fams}{vmi}{Worrorran}
\define@key{fams}{mfg}{Mande}
\define@key{fams}{mix}{Otomanguean}
\define@key{fams}{mvi}{Japonic}
\define@key{fams}{ehs}{Sign Language}
\define@key{fams}{soy}{Atlantic-Congo}
\define@key{fams}{lhs}{Afro-Asiatic}
\define@key{fams}{kja}{Nimboranic}
\define@key{fams}{mlo}{Atlantic-Congo}
\define@key{fams}{mmu}{Atlantic-Congo}
\define@key{fams}{bfm}{Atlantic-Congo}
\define@key{fams}{mfq}{Atlantic-Congo}
\define@key{fams}{mod}{Pidgin}
\define@key{fams}{ahm}{Kru}
\define@key{fams}{jkm}{Sino-Tibetan}
\define@key{fams}{mhn}{Indo-European}
\define@key{fams}{mhc}{Mayan}
\define@key{fams}{gbn}{Central Sudanic}
\define@key{fams}{mxd}{Austronesian}
\define@key{fams}{mqo}{North Halmahera}
\define@key{fams}{mvq}{Nuclear Trans New Guinea}
\define@key{fams}{mou}{Afro-Asiatic}
\define@key{fams}{mof}{Algic}
\define@key{fams}{mow}{Atlantic-Congo}
\define@key{fams}{mxn}{West Bird's Head}
\define@key{fams}{mkp}{Yareban}
\define@key{fams}{mwz}{Atlantic-Congo}
\define@key{fams}{ymi}{Sino-Tibetan}
\define@key{fams}{mft}{Austronesian}
\define@key{fams}{mwt}{Austronesian}
\define@key{fams}{mqt}{Austroasiatic}
\define@key{fams}{mkm}{Austronesian}
\define@key{fams}{mkl}{Atlantic-Congo}
\define@key{fams}{vms}{Unattested}
\define@key{fams}{pwm}{Austronesian}
\define@key{fams}{vsi}{Sign Language}
\define@key{fams}{bxc}{Atlantic-Congo}
\define@key{fams}{mox}{Austronesian}
\define@key{fams}{zmo}{Eastern Jebel}
\define@key{fams}{msl}{Isolate}
\define@key{fams}{mlw}{Afro-Asiatic}
\define@key{fams}{myl}{Austronesian}
\define@key{fams}{msz}{Nuclear Trans New Guinea}
\define@key{fams}{dmb}{Dogon}
\define@key{fams}{mmb}{Somahai}
\define@key{fams}{ver}{Atlantic-Congo}
\define@key{fams}{mzg}{Sign Language}
\define@key{fams}{npn}{Austronesian}
\define@key{fams}{msr}{Sign Language}
\define@key{fams}{mgt}{Keram}
\define@key{fams}{mom}{Otomanguean}
\define@key{fams}{moo}{Austroasiatic}
\define@key{fams}{mru}{Atlantic-Congo}
\define@key{fams}{mnh}{Atlantic-Congo}
\define@key{fams}{nmh}{Sino-Tibetan}
\define@key{fams}{mtl}{Afro-Asiatic}
\define@key{fams}{gwg}{Atlantic-Congo}
\define@key{fams}{crm}{Algic}
\define@key{fams}{msg}{West Bird's Head}
\define@key{fams}{mze}{Mailuan}
\define@key{fams}{moq}{Isolate}
\define@key{fams}{msx}{Nuclear Trans New Guinea}
\define@key{fams}{xmo}{Unattested}
\define@key{fams}{xmz}{Austronesian}
\define@key{fams}{mzq}{Austronesian}
\define@key{fams}{mdb}{Kiwaian}
\define@key{fams}{xms}{Sign Language}
\define@key{fams}{bdo}{Central Sudanic}
\define@key{fams}{mgc}{Central Sudanic}
\define@key{fams}{mrp}{Austronesian}
\define@key{fams}{mqn}{Austronesian}
\define@key{fams}{mrl}{Austronesian}
\define@key{fams}{mwy}{Nilotic}
\define@key{fams}{mqv}{Nuclear Trans New Guinea}
\define@key{fams}{mtj}{East Bird's Head}
\define@key{fams}{mtt}{Austronesian}
\define@key{fams}{mwh}{Austronesian}
\define@key{fams}{jmw}{Turama-Kikori}
\define@key{fams}{ity}{Austronesian}
\define@key{fams}{nmo}{Sino-Tibetan}
\define@key{fams}{mzy}{Sign Language}
\define@key{fams}{mxi}{Indo-European}
\define@key{fams}{xnq}{Atlantic-Congo}
\define@key{fams}{mpi}{Afro-Asiatic}
\define@key{fams}{mcx}{Atlantic-Congo}
\define@key{fams}{mpz}{Sino-Tibetan}
\define@key{fams}{pnd}{Atlantic-Congo}
\define@key{fams}{mgg}{Atlantic-Congo}
\define@key{fams}{mpa}{Atlantic-Congo}
\define@key{fams}{mvt}{Austronesian}
\define@key{fams}{zmp}{Atlantic-Congo}
\define@key{fams}{cmr}{Sino-Tibetan}
\define@key{fams}{mro}{Sino-Tibetan}
\define@key{fams}{kqx}{Afro-Asiatic}
\define@key{fams}{agz}{Austronesian}
\define@key{fams}{atl}{Austronesian}
\define@key{fams}{mtd}{Austronesian}
\define@key{fams}{tsx}{Anim}
\define@key{fams}{mub}{Afro-Asiatic}
\define@key{fams}{ymd}{Sino-Tibetan}
\define@key{fams}{gau}{Dravidian}
\define@key{fams}{udg}{Dravidian}
\define@key{fams}{vmd}{Dravidian}
\define@key{fams}{wiv}{Austronesian}
\define@key{fams}{muk}{Sino-Tibetan}
\define@key{fams}{mmk}{Dravidian}
\define@key{fams}{mfw}{Kwalean}
\define@key{fams}{kpb}{Dravidian}
\define@key{fams}{vmu}{Pama-Nyungan}
\define@key{fams}{kqa}{Nuclear Trans New Guinea}
\define@key{fams}{mwq}{Sino-Tibetan}
\define@key{fams}{boe}{Atlantic-Congo}
\define@key{fams}{mmf}{Afro-Asiatic}
\define@key{fams}{mth}{Austronesian}
\define@key{fams}{mpv}{Nuclear Trans New Guinea}
\define@key{fams}{mtc}{Nuclear Trans New Guinea}
\define@key{fams}{myr}{Isolate}
\define@key{fams}{mnj}{Indo-European}
\define@key{fams}{asx}{Nuclear Trans New Guinea}
\define@key{fams}{mxr}{Austronesian}
\define@key{fams}{rmh}{Lepki-Murkim-Kembra}
\define@key{fams}{tkv}{Austronesian}
\define@key{fams}{mqw}{Nuclear Trans New Guinea}
\define@key{fams}{smm}{Indo-European}
\define@key{fams}{mmi}{Nuclear Trans New Guinea}
\define@key{fams}{mmq}{Nuclear Trans New Guinea}
\define@key{fams}{mse}{Afro-Asiatic}
\define@key{fams}{mui}{Austronesian}
\define@key{fams}{mje}{Afro-Asiatic}
\define@key{fams}{muv}{Dravidian}
\define@key{fams}{tuc}{Austronesian}
\define@key{fams}{muy}{Afro-Asiatic}
\define@key{fams}{ymz}{Sino-Tibetan}
\define@key{fams}{mcj}{Atlantic-Congo}
\define@key{fams}{mxh}{Central Sudanic}
\define@key{fams}{wlc}{Atlantic-Congo}
\define@key{fams}{wmw}{Atlantic-Congo}
\define@key{fams}{moa}{Mande}
\define@key{fams}{mwa}{Austronesian}
\define@key{fams}{mjh}{Atlantic-Congo}
\define@key{fams}{mws}{Atlantic-Congo}
\define@key{fams}{gmy}{Indo-European}
\define@key{fams}{nme}{Sino-Tibetan}
\define@key{fams}{nbt}{Sino-Tibetan}
\define@key{fams}{nao}{Sino-Tibetan}
\define@key{fams}{mne}{Central Sudanic}
\define@key{fams}{mty}{Nuclear Torricelli}
\define@key{fams}{ncd}{Sino-Tibetan}
\define@key{fams}{srf}{Austronesian}
\define@key{fams}{nxx}{Sentanic}
\define@key{fams}{jbn}{Afro-Asiatic}
\define@key{fams}{nbg}{Unattested}
\define@key{fams}{nxe}{Austronesian}
\define@key{fams}{ngv}{Atlantic-Congo}
\define@key{fams}{nlx}{Indo-European}
\define@key{fams}{nhh}{Indo-European}
\define@key{fams}{ars}{Afro-Asiatic}
\define@key{fams}{nae}{Austronesian}
\define@key{fams}{nib}{Nuclear Trans New Guinea}
\define@key{fams}{nkj}{Nuclear Trans New Guinea}
\define@key{fams}{nbk}{Nuclear Trans New Guinea}
\define@key{fams}{mff}{Atlantic-Congo}
\define@key{fams}{nax}{Left May}
\define@key{fams}{nlc}{Nuclear Trans New Guinea}
\define@key{fams}{nss}{Austronesian}
\define@key{fams}{nlz}{Austronesian}
\define@key{fams}{ylo}{Sino-Tibetan}
\define@key{fams}{naj}{Atlantic-Congo}
\define@key{fams}{nmx}{Yam}
\define@key{fams}{nkm}{Yam}
\define@key{fams}{nmk}{Austronesian}
\define@key{fams}{nmq}{Atlantic-Congo}
\define@key{fams}{ncm}{Yam}
\define@key{fams}{neo}{Unclassifiable}
\define@key{fams}{nbs}{Sign Language}
\define@key{fams}{nvm}{Koiarian}
\define@key{fams}{naa}{Namla-Tofanma}
\define@key{fams}{mxw}{Yam}
\define@key{fams}{nmt}{Austronesian}
\define@key{fams}{bwb}{Austronesian}
\define@key{fams}{nmy}{Sino-Tibetan}
\define@key{fams}{nnc}{Afro-Asiatic}
\define@key{fams}{nzz}{Dogon}
\define@key{fams}{ngr}{Austronesian}
\define@key{fams}{cox}{Arawakan}
\define@key{fams}{afk}{Arafundi}
\define@key{fams}{qvo}{Quechuan}
\define@key{fams}{nrg}{Austronesian}
\define@key{fams}{nac}{Nuclear Trans New Guinea}
\define@key{fams}{loh}{Surmic}
\define@key{fams}{nnr}{Pama-Nyungan}
\define@key{fams}{nsy}{Austronesian}
\define@key{fams}{nvh}{Austronesian}
\define@key{fams}{ntz}{Indo-European}
\define@key{fams}{nte}{Atlantic-Congo}
\define@key{fams}{nti}{Atlantic-Congo}
\define@key{fams}{nxa}{Austronesian}
\define@key{fams}{ncn}{Austronesian}
\define@key{fams}{nwo}{Pama-Nyungan}
\define@key{fams}{nsw}{Austronesian}
\define@key{fams}{nwr}{Yareban}
\define@key{fams}{nwa}{Algic}
\define@key{fams}{nmz}{Atlantic-Congo}
\define@key{fams}{naw}{Atlantic-Congo}
\define@key{fams}{nyq}{Indo-European}
\define@key{fams}{noz}{Dizoid}
\define@key{fams}{ncr}{Atlantic-Congo}
\define@key{fams}{nlu}{Atlantic-Congo}
\define@key{fams}{gke}{Atlantic-Congo}
\define@key{fams}{ndk}{Atlantic-Congo}
\define@key{fams}{ndh}{Atlantic-Congo}
\define@key{fams}{ndj}{Atlantic-Congo}
\define@key{fams}{ndm}{Afro-Asiatic}
\define@key{fams}{nxo}{Atlantic-Congo}
\define@key{fams}{nnz}{Atlantic-Congo}
\define@key{fams}{nda}{Atlantic-Congo}
\define@key{fams}{ndc}{Atlantic-Congo}
\define@key{fams}{nml}{Atlantic-Congo}
\define@key{fams}{ndg}{Atlantic-Congo}
\define@key{fams}{dne}{Atlantic-Congo}
\define@key{fams}{ndd}{Atlantic-Congo}
\define@key{fams}{eli}{Narrow Talodi}
\define@key{fams}{ndw}{Atlantic-Congo}
\define@key{fams}{nbb}{Atlantic-Congo}
\define@key{fams}{ndl}{Atlantic-Congo}
\define@key{fams}{ndq}{Atlantic-Congo}
\define@key{fams}{nqm}{Kolopom}
\define@key{fams}{ndr}{Atlantic-Congo}
\define@key{fams}{ndp}{Central Sudanic}
\define@key{fams}{dno}{Central Sudanic}
\define@key{fams}{ndx}{Nuclear Trans New Guinea}
\define@key{fams}{nuh}{Atlantic-Congo}
\define@key{fams}{nww}{Atlantic-Congo}
\define@key{fams}{njt}{Pidgin}
\define@key{fams}{wni}{Atlantic-Congo}
\define@key{fams}{nec}{Timor-Alor-Pantar}
\define@key{fams}{nef}{Pidgin}
\define@key{fams}{dcr}{Indo-European}
\define@key{fams}{nkg}{Nuclear Trans New Guinea}
\define@key{fams}{nif}{Nuclear Trans New Guinea}
\define@key{fams}{nej}{Nuclear Trans New Guinea}
\define@key{fams}{nek}{Austronesian}
\define@key{fams}{nex}{Yam}
\define@key{fams}{nem}{Austronesian}
\define@key{fams}{nqn}{Yam}
\define@key{fams}{neu}{Artificial Language}
\define@key{fams}{nsp}{Sign Language}
\define@key{fams}{net}{Nuclear Trans New Guinea}
\define@key{fams}{jas}{Austronesian}
\define@key{fams}{jui}{Pama-Nyungan}
\define@key{fams}{nnf}{Nuclear Trans New Guinea}
\define@key{fams}{hlt}{Sino-Tibetan}
\define@key{fams}{szb}{Nuclear Trans New Guinea}
\define@key{fams}{nud}{Ndu}
\define@key{fams}{nmv}{Pama-Nyungan}
\define@key{fams}{nbv}{Atlantic-Congo}
\define@key{fams}{nmc}{Central Sudanic}
\define@key{fams}{nbh}{Afro-Asiatic}
\define@key{fams}{nyx}{Pama-Nyungan}
\define@key{fams}{gng}{Atlantic-Congo}
\define@key{fams}{nne}{Atlantic-Congo}
\define@key{fams}{nxd}{Atlantic-Congo}
\define@key{fams}{ngd}{Atlantic-Congo}
\define@key{fams}{nji}{Mirndi}
\define@key{fams}{rxd}{Pama-Nyungan}
\define@key{fams}{nsg}{Nilotic}
\define@key{fams}{ngm}{Speech Register}
\define@key{fams}{cnw}{Sino-Tibetan}
\define@key{fams}{zdj}{Atlantic-Congo}
\define@key{fams}{ngg}{Atlantic-Congo}
\define@key{fams}{jgb}{Atlantic-Congo}
\define@key{fams}{nbd}{Atlantic-Congo}
\define@key{fams}{nuu}{Atlantic-Congo}
\define@key{fams}{gnj}{Mande}
\define@key{fams}{nql}{Atlantic-Congo}
\define@key{fams}{ngt}{Austroasiatic}
\define@key{fams}{nnn}{Afro-Asiatic}
\define@key{fams}{nbq}{Nuclear Trans New Guinea}
\define@key{fams}{ngx}{Afro-Asiatic}
\define@key{fams}{nnh}{Atlantic-Congo}
\define@key{fams}{ngj}{Atlantic-Congo}
\define@key{fams}{nnq}{Atlantic-Congo}
\define@key{fams}{nra}{Atlantic-Congo}
\define@key{fams}{nla}{Atlantic-Congo}
\define@key{fams}{jgo}{Atlantic-Congo}
\define@key{fams}{noq}{Atlantic-Congo}
\define@key{fams}{nsh}{Atlantic-Congo}
\define@key{fams}{nuw}{Austronesian}
\define@key{fams}{ngp}{Atlantic-Congo}
\define@key{fams}{nlo}{Atlantic-Congo}
\define@key{fams}{xnm}{Nyulnyulan}
\define@key{fams}{nui}{Atlantic-Congo}
\define@key{fams}{nue}{Atlantic-Congo}
\define@key{fams}{ndn}{Atlantic-Congo}
\define@key{fams}{ngz}{Atlantic-Congo}
\define@key{fams}{nuo}{Austroasiatic}
\define@key{fams}{nrx}{Unattested}
\define@key{fams}{nbx}{Pama-Nyungan}
\define@key{fams}{ngq}{Atlantic-Congo}
\define@key{fams}{ngw}{Afro-Asiatic}
\define@key{fams}{nwe}{Atlantic-Congo}
\define@key{fams}{ngn}{Atlantic-Congo}
\define@key{fams}{yrl}{Tupian}
\define@key{fams}{nhf}{Pama-Nyungan}
\define@key{fams}{ncs}{Sign Language}
\define@key{fams}{nsi}{Sign Language}
\define@key{fams}{mzk}{Atlantic-Congo}
\define@key{fams}{nii}{Nuclear Trans New Guinea}
\define@key{fams}{xny}{Pama-Nyungan}
\define@key{fams}{gbe}{Sepik}
\define@key{fams}{nim}{Atlantic-Congo}
\define@key{fams}{nil}{Austronesian}
\define@key{fams}{noe}{Indo-European}
\define@key{fams}{nmp}{Nyulnyulan}
\define@key{fams}{nmr}{Atlantic-Congo}
\define@key{fams}{nis}{Nuclear Trans New Guinea}
\define@key{fams}{nmw}{Austronesian}
\define@key{fams}{niw}{Left May}
\define@key{fams}{nxi}{Atlantic-Congo}
\define@key{fams}{nxr}{Nuclear Trans New Guinea}
\define@key{fams}{nby}{Border}
\define@key{fams}{nlk}{Nuclear Trans New Guinea}
\define@key{fams}{nin}{Atlantic-Congo}
\define@key{fams}{nps}{Nuclear Trans New Guinea}
\define@key{fams}{njs}{Geelvink Bay}
\define@key{fams}{yso}{Sino-Tibetan}
\define@key{fams}{nkp}{Austronesian}
\define@key{fams}{njl}{Dajuic}
\define@key{fams}{nzb}{Atlantic-Congo}
\define@key{fams}{njj}{Atlantic-Congo}
\define@key{fams}{njr}{Atlantic-Congo}
\define@key{fams}{njy}{Atlantic-Congo}
\define@key{fams}{nkq}{Atlantic-Congo}
\define@key{fams}{nkn}{Atlantic-Congo}
\define@key{fams}{nkz}{Atlantic-Congo}
\define@key{fams}{khu}{Atlantic-Congo}
\define@key{fams}{nqo}{Artificial Language}
\define@key{fams}{nkc}{Atlantic-Congo}
\define@key{fams}{nkx}{Ijoid}
\define@key{fams}{nka}{Atlantic-Congo}
\define@key{fams}{nbo}{Atlantic-Congo}
\define@key{fams}{nkw}{Atlantic-Congo}
\define@key{fams}{nbp}{Atlantic-Congo}
\define@key{fams}{ngh}{Tuu}
\define@key{fams}{gaw}{Nuclear Trans New Guinea}
\define@key{fams}{noi}{Indo-European}
\define@key{fams}{nkk}{Austronesian}
\define@key{fams}{lem}{Atlantic-Congo}
\define@key{fams}{nof}{Nuclear Trans New Guinea}
\define@key{fams}{noh}{Nuclear Trans New Guinea}
\define@key{fams}{zhn}{Tai-Kadai}
\define@key{fams}{noj}{Huitotoan}
\define@key{fams}{nok}{Salishan}
\define@key{fams}{nrc}{Indo-European}
\define@key{fams}{nrp}{Unclassifiable}
\define@key{fams}{huj}{Hmong-Mien}
\define@key{fams}{hmp}{Hmong-Mien}
\define@key{fams}{crl}{Algic}
\define@key{fams}{pbu}{Indo-European}
\define@key{fams}{hno}{Indo-European}
\define@key{fams}{glh}{Indo-European}
\define@key{fams}{aee}{Indo-European}
\define@key{fams}{kxm}{Austroasiatic}
\define@key{fams}{atv}{Turkic}
\define@key{fams}{azj}{Turkic}
\define@key{fams}{ghh}{Sino-Tibetan}
\define@key{fams}{ymx}{Sino-Tibetan}
\define@key{fams}{yiv}{Sino-Tibetan}
\define@key{fams}{cng}{Sino-Tibetan}
\define@key{fams}{bfc}{Sino-Tibetan}
\define@key{fams}{nnl}{Sino-Tibetan}
\define@key{fams}{lbr}{Sino-Tibetan}
\define@key{fams}{tji}{Sino-Tibetan}
\define@key{fams}{doc}{Tai-Kadai}
\define@key{fams}{nod}{Tai-Kadai}
\define@key{fams}{tts}{Tai-Kadai}
\define@key{fams}{hea}{Hmong-Mien}
\define@key{fams}{hmi}{Hmong-Mien}
\define@key{fams}{kqs}{Atlantic-Congo}
\define@key{fams}{fll}{Atlantic-Congo}
\define@key{fams}{dgi}{Atlantic-Congo}
\define@key{fams}{tsp}{Atlantic-Congo}
\define@key{fams}{gbo}{Kru}
\define@key{fams}{dip}{Nilotic}
\define@key{fams}{diw}{Nilotic}
\define@key{fams}{max}{Austronesian}
\define@key{fams}{mmg}{Austronesian}
\define@key{fams}{mrq}{Austronesian}
\define@key{fams}{tnn}{Austronesian}
\define@key{fams}{una}{Austronesian}
\define@key{fams}{bcd}{Austronesian}
\define@key{fams}{weo}{Austronesian}
\define@key{fams}{nni}{Austronesian}
\define@key{fams}{aqn}{Austronesian}
\define@key{fams}{xnn}{Austronesian}
\define@key{fams}{cts}{Austronesian}
\define@key{fams}{stb}{Austronesian}
\define@key{fams}{bmm}{Austronesian}
\define@key{fams}{onr}{Nuclear Torricelli}
\define@key{fams}{kti}{Nuclear Trans New Guinea}
\define@key{fams}{nks}{Nuclear Trans New Guinea}
\define@key{fams}{yir}{Nuclear Trans New Guinea}
\define@key{fams}{whg}{Nuclear Trans New Guinea}
\define@key{fams}{kiw}{Kiwaian}
\define@key{fams}{ryn}{Japonic}
\define@key{fams}{neq}{Mixe-Zoque}
\define@key{fams}{scs}{Athabaskan-Eyak-Tlingit}
\define@key{fams}{esk}{Eskimo-Aleut}
\define@key{fams}{thh}{Uto-Aztecan}
\define@key{fams}{nhy}{Uto-Aztecan}
\define@key{fams}{ojb}{Algic}
\define@key{fams}{pef}{Pomoan}
\define@key{fams}{cst}{Miwok-Costanoan}
\define@key{fams}{enl}{Lengua-Mascoy}
\define@key{fams}{qvz}{Quechuan}
\define@key{fams}{qul}{Quechuan}
\define@key{fams}{qxn}{Quechuan}
\define@key{fams}{pmq}{Otomanguean}
\define@key{fams}{xtn}{Otomanguean}
\define@key{fams}{mxa}{Otomanguean}
\define@key{fams}{mfk}{Afro-Asiatic}
\define@key{fams}{ayp}{Afro-Asiatic}
\define@key{fams}{ntd}{Austronesian}
\define@key{fams}{cnp}{Sino-Tibetan}
\define@key{fams}{ncq}{Austroasiatic}
\define@key{fams}{bly}{Atlantic-Congo}
\define@key{fams}{ncf}{Austronesian}
\define@key{fams}{ntw}{Iroquoian}
\define@key{fams}{nov}{Artificial Language}
\define@key{fams}{noy}{Atlantic-Congo}
\define@key{fams}{asj}{Atlantic-Congo}
\define@key{fams}{nsc}{Unattested}
\define@key{fams}{nsx}{Atlantic-Congo}
\define@key{fams}{baf}{Atlantic-Congo}
\define@key{fams}{kte}{Sino-Tibetan}
\define@key{fams}{wbm}{Austroasiatic}
\define@key{fams}{bsq}{Kru}
\define@key{fams}{wla}{Walioic}
\define@key{fams}{wgi}{Nuclear Trans New Guinea}
\define@key{fams}{gyz}{Afro-Asiatic}
\define@key{fams}{nqt}{Afro-Asiatic}
\define@key{fams}{nnv}{Pama-Nyungan}
\define@key{fams}{noc}{Nuclear Trans New Guinea}
\define@key{fams}{klt}{Nuclear Trans New Guinea}
\define@key{fams}{nuq}{Austronesian}
\define@key{fams}{nur}{Austronesian}
\define@key{fams}{nuc}{Pano-Tacanan}
\define@key{fams}{nbr}{Atlantic-Congo}
\define@key{fams}{nop}{Nuclear Trans New Guinea}
\define@key{fams}{sij}{Austronesian}
\define@key{fams}{tgs}{Austronesian}
\define@key{fams}{kdk}{Austronesian}
\define@key{fams}{nxm}{Unclassifiable}
\define@key{fams}{nug}{Mirndi}
\define@key{fams}{rin}{Atlantic-Congo}
\define@key{fams}{nul}{Austronesian}
\define@key{fams}{nwb}{Kru}
\define@key{fams}{nev}{Austroasiatic}
\define@key{fams}{nyy}{Atlantic-Congo}
\define@key{fams}{nlj}{Atlantic-Congo}
\define@key{fams}{mwn}{Atlantic-Congo}
\define@key{fams}{nwm}{Central Sudanic}
\define@key{fams}{nmi}{Afro-Asiatic}
\define@key{fams}{nny}{Tangkic}
\define@key{fams}{nyb}{Atlantic-Congo}
\define@key{fams}{nyc}{Atlantic-Congo}
\define@key{fams}{nyk}{Atlantic-Congo}
\define@key{fams}{nnj}{Nilotic}
\define@key{fams}{sev}{Atlantic-Congo}
\define@key{fams}{nba}{Atlantic-Congo}
\define@key{fams}{neh}{Sino-Tibetan}
\define@key{fams}{nye}{Atlantic-Congo}
\define@key{fams}{nyl}{Austroasiatic}
\define@key{fams}{nyr}{Atlantic-Congo}
\define@key{fams}{nkv}{Atlantic-Congo}
\define@key{fams}{nkt}{Atlantic-Congo}
\define@key{fams}{nyg}{Atlantic-Congo}
\define@key{fams}{lid}{Austronesian}
\define@key{fams}{nvo}{Atlantic-Congo}
\define@key{fams}{nuj}{Atlantic-Congo}
\define@key{fams}{muo}{Atlantic-Congo}
\define@key{fams}{nyd}{Atlantic-Congo}
\define@key{fams}{nyu}{Atlantic-Congo}
\define@key{fams}{nzd}{Atlantic-Congo}
\define@key{fams}{nzy}{Atlantic-Congo}
\define@key{fams}{nja}{Afro-Asiatic}
\define@key{fams}{nzi}{Atlantic-Congo}
\define@key{fams}{bzy}{Atlantic-Congo}
\define@key{fams}{obi}{Chumashan}
\define@key{fams}{obl}{Atlantic-Congo}
\define@key{fams}{obo}{Austronesian}
\define@key{fams}{obu}{Atlantic-Congo}
\define@key{fams}{zac}{Otomanguean}
\define@key{fams}{odk}{Indo-European}
\define@key{fams}{bhf}{Isolate}
\define@key{fams}{kkc}{East Strickland}
\define@key{fams}{odu}{Atlantic-Congo}
\define@key{fams}{tyh}{Austroasiatic}
\define@key{fams}{opy}{Nuclear-Macro-Je}
\define@key{fams}{ofo}{Siouan}
\define@key{fams}{ogc}{Atlantic-Congo}
\define@key{fams}{ogg}{Atlantic-Congo}
\define@key{fams}{eri}{Nuclear Trans New Guinea}
\define@key{fams}{oia}{Timor-Alor-Pantar}
\define@key{fams}{chj}{Otomanguean}
\define@key{fams}{oki}{Nilotic}
\define@key{fams}{okn}{Japonic}
\define@key{fams}{okb}{Atlantic-Congo}
\define@key{fams}{okd}{Ijoid}
\define@key{fams}{oks}{Atlantic-Congo}
\define@key{fams}{okj}{Great Andamanese}
\define@key{fams}{kqv}{Austronesian}
\define@key{fams}{oie}{Nilotic}
\define@key{fams}{opa}{Atlantic-Congo}
\define@key{fams}{okx}{Atlantic-Congo}
\define@key{fams}{oke}{Atlantic-Congo}
\define@key{fams}{oar}{Afro-Asiatic}
\define@key{fams}{obr}{Sino-Tibetan}
\define@key{fams}{och}{Sino-Tibetan}
\define@key{fams}{odt}{Indo-European}
\define@key{fams}{ang}{Indo-European}
\define@key{fams}{fro}{Indo-European}
\define@key{fams}{ofs}{Indo-European}
\define@key{fams}{oge}{Kartvelian}
\define@key{fams}{goh}{Indo-European}
\define@key{fams}{sga}{Indo-European}
\define@key{fams}{ojp}{Japonic}
\define@key{fams}{okl}{Sign Language}
\define@key{fams}{qok}{Austroasiatic}
\define@key{fams}{qkn}{Dravidian}
\define@key{fams}{qbb}{Indo-European}
\define@key{fams}{omx}{Austroasiatic}
\define@key{fams}{omr}{Indo-European}
\define@key{fams}{non}{Indo-European}
\define@key{fams}{onw}{Nubian}
\define@key{fams}{oos}{Indo-European}
\define@key{fams}{pro}{Indo-European}
\define@key{fams}{peo}{Indo-European}
\define@key{fams}{orv}{Indo-European}
\define@key{fams}{osp}{Indo-European}
\define@key{fams}{osx}{Indo-European}
\define@key{fams}{oty}{Dravidian}
\define@key{fams}{oui}{Turkic}
\define@key{fams}{owl}{Indo-European}
\define@key{fams}{ole}{Sino-Tibetan}
\define@key{fams}{olm}{Atlantic-Congo}
\define@key{fams}{lul}{Central Sudanic}
\define@key{fams}{iko}{Atlantic-Congo}
\define@key{fams}{acx}{Afro-Asiatic}
\define@key{fams}{oml}{Atlantic-Congo}
\define@key{fams}{nht}{Uto-Aztecan}
\define@key{fams}{omi}{Central Sudanic}
\define@key{fams}{omt}{Nilotic}
\define@key{fams}{omu}{Isolate}
\define@key{fams}{oog}{Austroasiatic}
\define@key{fams}{onx}{Pidgin}
\define@key{fams}{oni}{Austronesian}
\define@key{fams}{onj}{Dagan}
\define@key{fams}{onn}{Bosavi}
\define@key{fams}{oor}{Indo-European}
\define@key{fams}{opo}{Eleman}
\define@key{fams}{opt}{Uto-Aztecan}
\define@key{fams}{lgn}{Koman}
\define@key{fams}{orn}{Austronesian}
\define@key{fams}{ors}{Austronesian}
\define@key{fams}{sdr}{Indo-European}
\define@key{fams}{org}{Atlantic-Congo}
\define@key{fams}{nlv}{Uto-Aztecan}
\define@key{fams}{fnb}{Austronesian}
\define@key{fams}{orc}{Afro-Asiatic}
\define@key{fams}{orz}{Austronesian}
\define@key{fams}{ora}{Austronesian}
\define@key{fams}{orx}{Atlantic-Congo}
\define@key{fams}{orh}{Tungusic}
\define@key{fams}{bpk}{Austronesian}
\define@key{fams}{orw}{Chapacuran}
\define@key{fams}{orr}{Ijoid}
\define@key{fams}{syx}{Atlantic-Congo}
\define@key{fams}{ost}{Atlantic-Congo}
\define@key{fams}{osc}{Indo-European}
\define@key{fams}{osi}{Austronesian}
\define@key{fams}{oso}{Atlantic-Congo}
\define@key{fams}{uta}{Atlantic-Congo}
\define@key{fams}{otd}{Austronesian}
\define@key{fams}{oti}{Isolate}
\define@key{fams}{otw}{Algic}
\define@key{fams}{lot}{Nilotic}
\define@key{fams}{otu}{Bororoan}
\define@key{fams}{oum}{Austronesian}
\define@key{fams}{oue}{South Bougainville}
\define@key{fams}{stn}{Austronesian}
\define@key{fams}{wsr}{Nuclear Trans New Guinea}
\define@key{fams}{oyy}{Austronesian}
\define@key{fams}{oyd}{Ta-Ne-Omotic}
\define@key{fams}{zao}{Otomanguean}
\define@key{fams}{chz}{Otomanguean}
\define@key{fams}{pfa}{Austronesian}
\define@key{fams}{sig}{Atlantic-Congo}
\define@key{fams}{qvp}{Quechuan}
\define@key{fams}{pcp}{Pano-Tacanan}
\define@key{fams}{pdi}{Tai-Kadai}
\define@key{fams}{pkc}{Unclassifiable}
\define@key{fams}{pae}{Atlantic-Congo}
\define@key{fams}{pgi}{Border}
\define@key{fams}{phr}{Indo-European}
\define@key{fams}{phj}{Sino-Tibetan}
\define@key{fams}{lgt}{Sepik}
\define@key{fams}{phv}{Indo-European}
\define@key{fams}{pal}{Indo-European}
\define@key{fams}{pha}{Hmong-Mien}
\define@key{fams}{pri}{Austronesian}
\define@key{fams}{ppi}{Cochimi-Yuman}
\define@key{fams}{qpp}{Indo-European}
\define@key{fams}{pta}{Tupian}
\define@key{fams}{pkg}{Austronesian}
\define@key{fams}{jkp}{Sino-Tibetan}
\define@key{fams}{pku}{Austronesian}
\define@key{fams}{pfl}{Indo-European}
\define@key{fams}{plq}{Indo-European}
\define@key{fams}{plr}{Atlantic-Congo}
\define@key{fams}{pln}{Indo-European}
\define@key{fams}{pnl}{Atlantic-Congo}
\define@key{fams}{pli}{Indo-European}
\define@key{fams}{pcf}{Dravidian}
\define@key{fams}{pmd}{Pama-Nyungan}
\define@key{fams}{abw}{Nuclear Trans New Guinea}
\define@key{fams}{pmc}{Unattested}
\define@key{fams}{ple}{Austronesian}
\define@key{fams}{plz}{Austronesian}
\define@key{fams}{bpx}{Indo-European}
\define@key{fams}{pmb}{Atlantic-Congo}
\define@key{fams}{pmn}{Atlantic-Congo}
\define@key{fams}{hih}{Nuclear Trans New Guinea}
\define@key{fams}{att}{Austronesian}
\define@key{fams}{pnz}{Atlantic-Congo}
\define@key{fams}{pnq}{Atlantic-Congo}
\define@key{fams}{pwb}{Atlantic-Congo}
\define@key{fams}{psn}{Austronesian}
\define@key{fams}{qxh}{Quechuan}
\define@key{fams}{lsp}{Sign Language}
\define@key{fams}{tdb}{Indo-European}
\define@key{fams}{pnp}{Austronesian}
\define@key{fams}{bkj}{Atlantic-Congo}
\define@key{fams}{pgg}{Indo-European}
\define@key{fams}{pgs}{Atlantic-Congo}
\define@key{fams}{slm}{Austronesian}
\define@key{fams}{pcg}{Dravidian}
\define@key{fams}{pnr}{Nuclear Trans New Guinea}
\define@key{fams}{pax}{Unattested}
\define@key{fams}{pkh}{Sino-Tibetan}
\define@key{fams}{paz}{Isolate}
\define@key{fams}{pnc}{Austronesian}
\define@key{fams}{knt}{Pano-Tacanan}
\define@key{fams}{pno}{Pano-Tacanan}
\define@key{fams}{blk}{Sino-Tibetan}
\define@key{fams}{ppv}{Unattested}
\define@key{fams}{ppn}{Austronesian}
\define@key{fams}{dpp}{Austronesian}
\define@key{fams}{pas}{Lakes Plain}
\define@key{fams}{pbo}{Atlantic-Congo}
\define@key{fams}{ppe}{Isolate}
\define@key{fams}{ppu}{Austronesian}
\define@key{fams}{ppm}{Austronesian}
\define@key{fams}{pgz}{Sign Language}
\define@key{fams}{prc}{Indo-European}
\define@key{fams}{pzn}{Sino-Tibetan}
\define@key{fams}{prf}{Austronesian}
\define@key{fams}{prw}{Nuclear Trans New Guinea}
\define@key{fams}{aap}{Cariban}
\define@key{fams}{pak}{Tupian}
\define@key{fams}{paf}{Tupian}
\define@key{fams}{gvp}{Nuclear-Macro-Je}
\define@key{fams}{pbg}{Arawakan}
\define@key{fams}{pys}{Sign Language}
\define@key{fams}{pcl}{Indo-European}
\define@key{fams}{pch}{Unattested}
\define@key{fams}{pcj}{Austroasiatic}
\define@key{fams}{ppt}{Kamula-Elevala}
\define@key{fams}{kvx}{Indo-European}
\define@key{fams}{xpr}{Indo-European}
\define@key{fams}{paq}{Indo-European}
\define@key{fams}{psq}{Sepik}
\define@key{fams}{yac}{Nuclear Trans New Guinea}
\define@key{fams}{ptn}{Austronesian}
\define@key{fams}{pth}{Nuclear-Macro-Je}
\define@key{fams}{pbc}{Cariban}
\define@key{fams}{pty}{Dravidian}
\define@key{fams}{ptq}{Dravidian}
\define@key{fams}{mfa}{Austronesian}
\define@key{fams}{pnk}{Arawakan}
\define@key{fams}{bfb}{Indo-European}
\define@key{fams}{psm}{Tupian}
\define@key{fams}{pmr}{Nuclear Trans New Guinea}
\define@key{fams}{pcb}{Austroasiatic}
\define@key{fams}{xpc}{Turkic}
\define@key{fams}{pai}{Atlantic-Congo}
\define@key{fams}{pfe}{Atlantic-Congo}
\define@key{fams}{ppq}{Walioic}
\define@key{fams}{pel}{Austronesian}
\define@key{fams}{bxd}{Sino-Tibetan}
\define@key{fams}{ata}{Isolate}
\define@key{fams}{pev}{Cariban}
\define@key{fams}{psg}{Sign Language}
\define@key{fams}{pek}{Austronesian}
\define@key{fams}{ums}{Austronesian}
\define@key{fams}{pdc}{Indo-European}
\define@key{fams}{pnh}{Austronesian}
\define@key{fams}{ptw}{Salishan}
\define@key{fams}{pea}{Austronesian}
\define@key{fams}{wet}{Austronesian}
\define@key{fams}{psc}{Sign Language}
\define@key{fams}{prl}{Sign Language}
\define@key{fams}{pex}{Austronesian}
\define@key{fams}{zpe}{Otomanguean}
\define@key{fams}{pey}{Indo-European}
\define@key{fams}{prt}{Austroasiatic}
\define@key{fams}{phk}{Tai-Kadai}
\define@key{fams}{phl}{Indo-European}
\define@key{fams}{ypa}{Sino-Tibetan}
\define@key{fams}{phq}{Sino-Tibetan}
\define@key{fams}{pem}{Atlantic-Congo}
\define@key{fams}{psp}{Sign Language}
\define@key{fams}{phm}{Atlantic-Congo}
\define@key{fams}{phn}{Afro-Asiatic}
\define@key{fams}{yip}{Sino-Tibetan}
\define@key{fams}{ypg}{Sino-Tibetan}
\define@key{fams}{nph}{Sino-Tibetan}
\define@key{fams}{pnx}{Austroasiatic}
\define@key{fams}{kjt}{Sino-Tibetan}
\define@key{fams}{xpg}{Indo-European}
\define@key{fams}{phu}{Tai-Kadai}
\define@key{fams}{phd}{Indo-European}
\define@key{fams}{pug}{Atlantic-Congo}
\define@key{fams}{phh}{Sino-Tibetan}
\define@key{fams}{ypm}{Sino-Tibetan}
\define@key{fams}{pho}{Sino-Tibetan}
\define@key{fams}{phg}{Austroasiatic}
\define@key{fams}{yph}{Sino-Tibetan}
\define@key{fams}{ypp}{Sino-Tibetan}
\define@key{fams}{pht}{Tai-Kadai}
\define@key{fams}{ypz}{Sino-Tibetan}
\define@key{fams}{ptr}{Austronesian}
\define@key{fams}{pin}{Sepik}
\define@key{fams}{pcd}{Indo-European}
\define@key{fams}{cpu}{Arawakan}
\define@key{fams}{xpi}{Unclassifiable}
\define@key{fams}{dep}{Pidgin}
\define@key{fams}{pij}{Unclassifiable}
\define@key{fams}{piz}{Austronesian}
\define@key{fams}{pis}{Indo-European}
\define@key{fams}{piw}{Atlantic-Congo}
\define@key{fams}{pnn}{Piawi}
\define@key{fams}{pnv}{Pama-Nyungan}
\define@key{fams}{tjp}{Pama-Nyungan}
\define@key{fams}{pic}{Atlantic-Congo}
\define@key{fams}{pti}{Pama-Nyungan}
\define@key{fams}{pny}{Atlantic-Congo}
\define@key{fams}{bxi}{Pama-Nyungan}
\define@key{fams}{pie}{Kiowa-Tanoan}
\define@key{fams}{xpa}{Pama-Nyungan}
\define@key{fams}{tpp}{Totonacan}
\define@key{fams}{pig}{Unattested}
\define@key{fams}{psy}{Algic}
\define@key{fams}{xps}{Indo-European}
\define@key{fams}{pih}{Indo-European}
\define@key{fams}{sje}{Uralic}
\define@key{fams}{pcn}{Atlantic-Congo}
\define@key{fams}{pix}{Austronesian}
\define@key{fams}{piy}{Afro-Asiatic}
\define@key{fams}{ktj}{Kru}
\define@key{fams}{pdt}{Indo-European}
\define@key{fams}{pbv}{Austroasiatic}
\define@key{fams}{npo}{Sino-Tibetan}
\define@key{fams}{pdn}{Austronesian}
\define@key{fams}{pof}{Atlantic-Congo}
\define@key{fams}{pkb}{Atlantic-Congo}
\define@key{fams}{pld}{Unclassifiable}
\define@key{fams}{plj}{Afro-Asiatic}
\define@key{fams}{pso}{Sign Language}
\define@key{fams}{plb}{Austronesian}
\define@key{fams}{pmo}{Austronesian}
\define@key{fams}{pmm}{Atlantic-Congo}
\define@key{fams}{ncc}{Austronesian}
\define@key{fams}{png}{Atlantic-Congo}
\define@key{fams}{pns}{Austronesian}
\define@key{fams}{pnt}{Indo-European}
\define@key{fams}{prh}{Austronesian}
\define@key{fams}{ptv}{Austronesian}
\define@key{fams}{pmx}{Sino-Tibetan}
\define@key{fams}{bye}{Sepik}
\define@key{fams}{pwr}{Indo-European}
\define@key{fams}{pyn}{Pano-Tacanan}
\define@key{fams}{prz}{Sign Language}
\define@key{fams}{prg}{Indo-European}
\define@key{fams}{kvj}{Afro-Asiatic}
\define@key{fams}{pux}{Sko}
\define@key{fams}{atp}{Austronesian}
\define@key{fams}{pbm}{Otomanguean}
\define@key{fams}{psl}{Sign Language}
\define@key{fams}{pkp}{Austronesian}
\define@key{fams}{pup}{Nuclear Trans New Guinea}
\define@key{fams}{pum}{Sino-Tibetan}
\define@key{fams}{xpm}{Yeniseian}
\define@key{fams}{puj}{Austronesian}
\define@key{fams}{pud}{Austronesian}
\define@key{fams}{puf}{Austronesian}
\define@key{fams}{pna}{Austronesian}
\define@key{fams}{pnm}{Austronesian}
\define@key{fams}{xpu}{Afro-Asiatic}
\define@key{fams}{qxp}{Quechuan}
\define@key{fams}{puu}{Atlantic-Congo}
\define@key{fams}{pru}{South Bird's Head Family}
\define@key{fams}{iar}{Isolate}
\define@key{fams}{puy}{Chumashan}
\define@key{fams}{prr}{Puri-Coroado}
\define@key{fams}{pur}{Tupian}
\define@key{fams}{pub}{Sino-Tibetan}
\define@key{fams}{mfl}{Afro-Asiatic}
\define@key{fams}{afe}{Atlantic-Congo}
\define@key{fams}{cpx}{Sino-Tibetan}
\define@key{fams}{pyu}{Austronesian}
\define@key{fams}{pme}{Austronesian}
\define@key{fams}{pop}{Austronesian}
\define@key{fams}{pwo}{Sino-Tibetan}
\define@key{fams}{pcw}{Afro-Asiatic}
\define@key{fams}{pye}{Kru}
\define@key{fams}{pyy}{Sino-Tibetan}
\define@key{fams}{pby}{Isolate}
\define@key{fams}{laq}{Tai-Kadai}
\define@key{fams}{qxq}{Turkic}
\define@key{fams}{xqt}{Afro-Asiatic}
\define@key{fams}{ymq}{Sino-Tibetan}
\define@key{fams}{zqe}{Tai-Kadai}
\define@key{fams}{qua}{Siouan}
\define@key{fams}{qya}{Artificial Language}
\define@key{fams}{qvy}{Sino-Tibetan}
\define@key{fams}{zpj}{Otomanguean}
\define@key{fams}{quq}{Unclassifiable}
\define@key{fams}{qun}{Salishan}
\define@key{fams}{ztq}{Otomanguean}
\define@key{fams}{rah}{Sino-Tibetan}
\define@key{fams}{xrr}{Unclassifiable}
\define@key{fams}{raz}{Austronesian}
\define@key{fams}{mqk}{Austronesian}
\define@key{fams}{rjs}{Indo-European}
\define@key{fams}{rjg}{Austronesian}
\define@key{fams}{gra}{Indo-European}
\define@key{fams}{rkh}{Austronesian}
\define@key{fams}{rki}{Sino-Tibetan}
\define@key{fams}{rai}{Austronesian}
\define@key{fams}{kjx}{North Bougainville}
\define@key{fams}{lje}{Austronesian}
\define@key{fams}{thr}{Indo-European}
\define@key{fams}{rkt}{Indo-European}
\define@key{fams}{rnl}{Sino-Tibetan}
\define@key{fams}{rax}{Atlantic-Congo}
\define@key{fams}{ray}{Austronesian}
\define@key{fams}{rpt}{Nuclear Trans New Guinea}
\define@key{fams}{lra}{Austronesian}
\define@key{fams}{rar}{Austronesian}
\define@key{fams}{rac}{Lakes Plain}
\define@key{fams}{btn}{Austronesian}
\define@key{fams}{bgd}{Indo-European}
\define@key{fams}{rtw}{Indo-European}
\define@key{fams}{rau}{Sino-Tibetan}
\define@key{fams}{yea}{Dravidian}
\define@key{fams}{jnl}{Sino-Tibetan}
\define@key{fams}{rat}{Indo-European}
\define@key{fams}{gir}{Tai-Kadai}
\define@key{fams}{atu}{Nilotic}
\define@key{fams}{ree}{Austronesian}
\define@key{fams}{rei}{Indo-European}
\define@key{fams}{bow}{Yam}
\define@key{fams}{reb}{Austronesian}
\define@key{fams}{agv}{Austronesian}
\define@key{fams}{rem}{Pano-Tacanan}
\define@key{fams}{rmp}{Nuclear Trans New Guinea}
\define@key{fams}{lkj}{Austronesian}
\define@key{fams}{rsi}{Artificial Language}
\define@key{fams}{rea}{Nuclear Trans New Guinea}
\define@key{fams}{rer}{Unattested}
\define@key{fams}{pgk}{Austronesian}
\define@key{fams}{res}{Atlantic-Congo}
\define@key{fams}{ret}{Timor-Alor-Pantar}
\define@key{fams}{rcf}{Indo-European}
\define@key{fams}{rey}{Pano-Tacanan}
\define@key{fams}{ril}{Austroasiatic}
\define@key{fams}{ria}{Sino-Tibetan}
\define@key{fams}{rir}{Austronesian}
\define@key{fams}{zar}{Otomanguean}
\define@key{fams}{rgu}{Austronesian}
\define@key{fams}{hrx}{Indo-European}
\define@key{fams}{rri}{Austronesian}
\define@key{fams}{riu}{Austronesian}
\define@key{fams}{snj}{Atlantic-Congo}
\define@key{fams}{rod}{Atlantic-Congo}
\define@key{fams}{rhg}{Indo-European}
\define@key{fams}{rge}{Indo-European}
\define@key{fams}{rms}{Sign Language}
\define@key{fams}{rgn}{Indo-European}
\define@key{fams}{rmx}{Austroasiatic}
\define@key{fams}{rmm}{Austronesian}
\define@key{fams}{rmv}{Artificial Language}
\define@key{fams}{rof}{Atlantic-Congo}
\define@key{fams}{rol}{Austronesian}
\define@key{fams}{rmk}{Lower Sepik-Ramu}
\define@key{fams}{ror}{Austronesian}
\define@key{fams}{roe}{Austronesian}
\define@key{fams}{rnn}{Austronesian}
\define@key{fams}{rga}{Austronesian}
\define@key{fams}{pce}{Austroasiatic}
\define@key{fams}{rdb}{Indo-European}
\define@key{fams}{ruh}{Sino-Tibetan}
\define@key{fams}{rbb}{Austroasiatic}
\define@key{fams}{ruz}{Unattested}
\define@key{fams}{rna}{Unattested}
\define@key{fams}{rnw}{Atlantic-Congo}
\define@key{fams}{drg}{Austronesian}
\define@key{fams}{bxr}{Mongolic-Khitan}
\define@key{fams}{rue}{Indo-European}
\define@key{fams}{ruc}{Atlantic-Congo}
\define@key{fams}{rnd}{Atlantic-Congo}
\define@key{fams}{rwk}{Atlantic-Congo}
\define@key{fams}{rsn}{Sign Language}
\define@key{fams}{sax}{Austronesian}
\define@key{fams}{sav}{Atlantic-Congo}
\define@key{fams}{raq}{Sino-Tibetan}
\define@key{fams}{lsm}{Atlantic-Congo}
\define@key{fams}{sxr}{Austronesian}
\define@key{fams}{spy}{Nilotic}
\define@key{fams}{msi}{Austronesian}
\define@key{fams}{bsy}{Austronesian}
\define@key{fams}{sae}{Nambiquaran}
\define@key{fams}{saa}{Afro-Asiatic}
\define@key{fams}{xsa}{Afro-Asiatic}
\define@key{fams}{qhr}{Indo-European}
\define@key{fams}{sbo}{Austroasiatic}
\define@key{fams}{quv}{Mayan}
\define@key{fams}{sck}{Indo-European}
\define@key{fams}{spd}{Nuclear Trans New Guinea}
\define@key{fams}{saf}{Atlantic-Congo}
\define@key{fams}{sbk}{Atlantic-Congo}
\define@key{fams}{sbm}{Atlantic-Congo}
\define@key{fams}{tga}{Atlantic-Congo}
\define@key{fams}{aec}{Afro-Asiatic}
\define@key{fams}{acf}{Indo-European}
\define@key{fams}{xsy}{Austronesian}
\define@key{fams}{sjl}{Sino-Tibetan}
\define@key{fams}{sjb}{Austronesian}
\define@key{fams}{sch}{Sino-Tibetan}
\define@key{fams}{skt}{Atlantic-Congo}
\define@key{fams}{skg}{Austronesian}
\define@key{fams}{skm}{Nuclear Trans New Guinea}
\define@key{fams}{sak}{Atlantic-Congo}
\define@key{fams}{szy}{Austronesian}
\define@key{fams}{shq}{Atlantic-Congo}
\define@key{fams}{slx}{Atlantic-Congo}
\define@key{fams}{sgu}{Austronesian}
\define@key{fams}{qxl}{Quechuan}
\define@key{fams}{mnd}{Tupian}
\define@key{fams}{slq}{Turkic}
\define@key{fams}{sau}{Austronesian}
\define@key{fams}{loe}{Austronesian}
\define@key{fams}{esn}{Sign Language}
\define@key{fams}{tmj}{Greater Kwerba}
\define@key{fams}{ysd}{Sino-Tibetan}
\define@key{fams}{smp}{Afro-Asiatic}
\define@key{fams}{xab}{Atlantic-Congo}
\define@key{fams}{smx}{Atlantic-Congo}
\define@key{fams}{ccg}{Atlantic-Congo}
\define@key{fams}{saq}{Nilotic}
\define@key{fams}{ssx}{Nuclear Trans New Guinea}
\define@key{fams}{spv}{Indo-European}
\define@key{fams}{smh}{Sino-Tibetan}
\define@key{fams}{snx}{Nuclear Trans New Guinea}
\define@key{fams}{swm}{Nuclear Trans New Guinea}
\define@key{fams}{rav}{Sino-Tibetan}
\define@key{fams}{stu}{Austroasiatic}
\define@key{fams}{smv}{Indo-European}
\define@key{fams}{ztm}{Otomanguean}
\define@key{fams}{icr}{Indo-European}
\define@key{fams}{spn}{Lengua-Mascoy}
\define@key{fams}{zpx}{Otomanguean}
\define@key{fams}{cuk}{Chibchan}
\define@key{fams}{hve}{Huavean}
\define@key{fams}{hue}{Huavean}
\define@key{fams}{mat}{Otomanguean}
\define@key{fams}{pow}{Otomanguean}
\define@key{fams}{xso}{Unclassifiable}
\define@key{fams}{sgr}{Indo-European}
\define@key{fams}{sgk}{Sino-Tibetan}
\define@key{fams}{nsa}{Sino-Tibetan}
\define@key{fams}{xsn}{Atlantic-Congo}
\define@key{fams}{sbp}{Atlantic-Congo}
\define@key{fams}{sng}{Atlantic-Congo}
\define@key{fams}{snl}{Austronesian}
\define@key{fams}{scg}{Austronesian}
\define@key{fams}{sgy}{Indo-European}
\define@key{fams}{ysy}{Sino-Tibetan}
\define@key{fams}{ysn}{Sino-Tibetan}
\define@key{fams}{sny}{Sepik}
\define@key{fams}{xtj}{Otomanguean}
\define@key{fams}{maa}{Otomanguean}
\define@key{fams}{msc}{Mande}
\define@key{fams}{pps}{Otomanguean}
\define@key{fams}{qvs}{Quechuan}
\define@key{fams}{xtp}{Otomanguean}
\define@key{fams}{trq}{Otomanguean}
\define@key{fams}{pls}{Otomanguean}
\define@key{fams}{azg}{Otomanguean}
\define@key{fams}{zpf}{Otomanguean}
\define@key{fams}{san}{Indo-European}
\define@key{fams}{ssi}{Indo-European}
\define@key{fams}{kwy}{Atlantic-Congo}
\define@key{fams}{hvv}{Huavean}
\define@key{fams}{nhz}{Uto-Aztecan}
\define@key{fams}{cok}{Uto-Aztecan}
\define@key{fams}{qus}{Quechuan}
\define@key{fams}{mza}{Otomanguean}
\define@key{fams}{mdv}{Otomanguean}
\define@key{fams}{zpn}{Otomanguean}
\define@key{fams}{ztn}{Otomanguean}
\define@key{fams}{zas}{Otomanguean}
\define@key{fams}{zpr}{Otomanguean}
\define@key{fams}{pca}{Otomanguean}
\define@key{fams}{zpt}{Otomanguean}
\define@key{fams}{scq}{Austroasiatic}
\define@key{fams}{zkp}{Nuclear-Macro-Je}
\define@key{fams}{cri}{Indo-European}
\define@key{fams}{spr}{Austronesian}
\define@key{fams}{spc}{Isolate}
\define@key{fams}{krn}{Kru}
\define@key{fams}{spi}{Lakes Plain}
\define@key{fams}{sbz}{Central Sudanic}
\define@key{fams}{kwv}{Central Sudanic}
\define@key{fams}{kwg}{Central Sudanic}
\define@key{fams}{zsa}{Austronesian}
\define@key{fams}{bps}{Austronesian}
\define@key{fams}{mbs}{Austronesian}
\define@key{fams}{sre}{Austronesian}
\define@key{fams}{sar}{Arawakan}
\define@key{fams}{srh}{Indo-European}
\define@key{fams}{mwm}{Central Sudanic}
\define@key{fams}{onp}{Sino-Tibetan}
\define@key{fams}{sdu}{Austronesian}
\define@key{fams}{sra}{Nuclear Trans New Guinea}
\define@key{fams}{swy}{Afro-Asiatic}
\define@key{fams}{sxs}{Atlantic-Congo}
\define@key{fams}{sas}{Austronesian}
\define@key{fams}{sdc}{Indo-European}
\define@key{fams}{stw}{Austronesian}
\define@key{fams}{stq}{Indo-European}
\define@key{fams}{mav}{Tupian}
\define@key{fams}{sdl}{Sign Language}
\define@key{fams}{skc}{Nuclear Trans New Guinea}
\define@key{fams}{saz}{Indo-European}
\define@key{fams}{mjt}{Dravidian}
\define@key{fams}{srt}{Geelvink Bay}
\define@key{fams}{psu}{Indo-European}
\define@key{fams}{ssj}{Nuclear Trans New Guinea}
\define@key{fams}{sao}{Isolate}
\define@key{fams}{swr}{Yawa-Saweru}
\define@key{fams}{swt}{Timor-Alor-Pantar}
\define@key{fams}{saw}{Nuclear Trans New Guinea}
\define@key{fams}{swn}{Afro-Asiatic}
\define@key{fams}{sxw}{Atlantic-Congo}
\define@key{fams}{say}{Afro-Asiatic}
\define@key{fams}{sco}{Indo-European}
\define@key{fams}{kdg}{Atlantic-Congo}
\define@key{fams}{sbx}{Austronesian}
\define@key{fams}{sib}{Austronesian}
\define@key{fams}{sec}{Salishan}
\define@key{fams}{tvw}{Austronesian}
\define@key{fams}{sos}{Mande}
\define@key{fams}{sge}{Austronesian}
\define@key{fams}{sbg}{West Bird's Head}
\define@key{fams}{seg}{Atlantic-Congo}
\define@key{fams}{sfw}{Atlantic-Congo}
\define@key{fams}{ssg}{Austronesian}
\define@key{fams}{hik}{Austronesian}
\define@key{fams}{skz}{Austronesian}
\define@key{fams}{skp}{Austronesian}
\define@key{fams}{sek}{Athabaskan-Eyak-Tlingit}
\define@key{fams}{ske}{Austronesian}
\define@key{fams}{syi}{Atlantic-Congo}
\define@key{fams}{sko}{Austronesian}
\define@key{fams}{skx}{Austronesian}
\define@key{fams}{lip}{Atlantic-Congo}
\define@key{fams}{kgi}{Sign Language}
\define@key{fams}{snw}{Atlantic-Congo}
\define@key{fams}{sws}{Austronesian}
\define@key{fams}{slg}{Austronesian}
\define@key{fams}{szc}{Austroasiatic}
\define@key{fams}{sbr}{Austronesian}
\define@key{fams}{etz}{Mairasic}
\define@key{fams}{smy}{Indo-European}
\define@key{fams}{ssm}{Austroasiatic}
\define@key{fams}{xse}{Nuclear Trans New Guinea}
\define@key{fams}{seq}{Atlantic-Congo}
\define@key{fams}{sej}{Nuclear Trans New Guinea}
\define@key{fams}{sds}{Afro-Asiatic}
\define@key{fams}{ssz}{Austronesian}
\define@key{fams}{spk}{Ndu}
\define@key{fams}{snu}{Border}
\define@key{fams}{sjs}{Afro-Asiatic}
\define@key{fams}{sni}{Pano-Tacanan}
\define@key{fams}{std}{Unattested}
\define@key{fams}{sez}{Sino-Tibetan}
\define@key{fams}{spe}{Austronesian}
\define@key{fams}{spb}{Austronesian}
\define@key{fams}{spm}{Lower Sepik-Ramu}
\define@key{fams}{iws}{Sepik}
\define@key{fams}{skr}{Indo-European}
\define@key{fams}{sry}{Austronesian}
\define@key{fams}{srr}{Atlantic-Congo}
\define@key{fams}{swf}{Atlantic-Congo}
\define@key{fams}{sve}{Austronesian}
\define@key{fams}{seu}{Austronesian}
\define@key{fams}{srw}{Austronesian}
\define@key{fams}{srk}{Austronesian}
\define@key{fams}{stf}{Nuclear Torricelli}
\define@key{fams}{stm}{Nuclear Trans New Guinea}
\define@key{fams}{sbi}{Nuclear Torricelli}
\define@key{fams}{sta}{Pidgin}
\define@key{fams}{sew}{Austronesian}
\define@key{fams}{lsw}{Sign Language}
\define@key{fams}{sze}{Blue Nile Mao}
\define@key{fams}{scw}{Afro-Asiatic}
\define@key{fams}{sdb}{Indo-European}
\define@key{fams}{srz}{Indo-European}
\define@key{fams}{sha}{Atlantic-Congo}
\define@key{fams}{xsh}{Atlantic-Congo}
\define@key{fams}{sqa}{Atlantic-Congo}
\define@key{fams}{jih}{Sino-Tibetan}
\define@key{fams}{sho}{Mande}
\define@key{fams}{swo}{Pano-Tacanan}
\define@key{fams}{ssv}{Austronesian}
\define@key{fams}{swq}{Afro-Asiatic}
\define@key{fams}{sqh}{Atlantic-Congo}
\define@key{fams}{shx}{Hmong-Mien}
\define@key{fams}{she}{Dizoid}
\define@key{fams}{sth}{Speech Register}
\define@key{fams}{shl}{Sino-Tibetan}
\define@key{fams}{scv}{Atlantic-Congo}
\define@key{fams}{bun}{Atlantic-Congo}
\define@key{fams}{kip}{Sino-Tibetan}
\define@key{fams}{ssh}{Afro-Asiatic}
\define@key{fams}{shr}{Atlantic-Congo}
\define@key{fams}{gua}{Atlantic-Congo}
\define@key{fams}{snh}{Unattested}
\define@key{fams}{sxg}{Sino-Tibetan}
\define@key{fams}{sle}{Dravidian}
\define@key{fams}{bcv}{Atlantic-Congo}
\define@key{fams}{suj}{Atlantic-Congo}
\define@key{fams}{sts}{Indo-European}
\define@key{fams}{scu}{Sino-Tibetan}
\define@key{fams}{ksa}{Unattested}
\define@key{fams}{shw}{Heibanic}
\define@key{fams}{slw}{Nuclear Trans New Guinea}
\define@key{fams}{sya}{Austronesian}
\define@key{fams}{spg}{Austronesian}
\define@key{fams}{mmp}{Amto-Musan}
\define@key{fams}{nco}{South Bougainville}
\define@key{fams}{sty}{Turkic}
\define@key{fams}{sdx}{Austronesian}
\define@key{fams}{sxc}{Unclassifiable}
\define@key{fams}{scn}{Indo-European}
\define@key{fams}{sep}{Atlantic-Congo}
\define@key{fams}{scx}{Unclassifiable}
\define@key{fams}{xsd}{Indo-European}
\define@key{fams}{sgx}{Sign Language}
\define@key{fams}{nsu}{Uto-Aztecan}
\define@key{fams}{sxe}{Atlantic-Congo}
\define@key{fams}{snr}{Nuclear Trans New Guinea}
\define@key{fams}{qws}{Quechuan}
\define@key{fams}{sky}{Austronesian}
\define@key{fams}{slt}{Sino-Tibetan}
\define@key{fams}{szl}{Indo-European}
\define@key{fams}{sbq}{Nuclear Trans New Guinea}
\define@key{fams}{mkc}{Nuclear Torricelli}
\define@key{fams}{wul}{Nuclear Trans New Guinea}
\define@key{fams}{xsp}{Nuclear Trans New Guinea}
\define@key{fams}{stv}{Afro-Asiatic}
\define@key{fams}{sie}{Atlantic-Congo}
\define@key{fams}{sbw}{Atlantic-Congo}
\define@key{fams}{smb}{Angan}
\define@key{fams}{sbb}{Austronesian}
\define@key{fams}{smg}{Baining}
\define@key{fams}{smz}{South Bougainville}
\define@key{fams}{smt}{Sino-Tibetan}
\define@key{fams}{siu}{Nuclear Torricelli}
\define@key{fams}{sbn}{Indo-European}
\define@key{fams}{xts}{Otomanguean}
\define@key{fams}{sjn}{Artificial Language}
\define@key{fams}{sgp}{Sino-Tibetan}
\define@key{fams}{sgm}{Atlantic-Congo}
\define@key{fams}{skq}{Mande}
\define@key{fams}{xti}{Otomanguean}
\define@key{fams}{snz}{Nuclear Trans New Guinea}
\define@key{fams}{sys}{Central Sudanic}
\define@key{fams}{swj}{Atlantic-Congo}
\define@key{fams}{sir}{Afro-Asiatic}
\define@key{fams}{srx}{Indo-European}
\define@key{fams}{sld}{Atlantic-Congo}
\define@key{fams}{sso}{Austronesian}
\define@key{fams}{siy}{Indo-European}
\define@key{fams}{lsv}{Sign Language}
\define@key{fams}{akp}{Atlantic-Congo}
\define@key{fams}{skw}{Indo-European}
\define@key{fams}{sms}{Uralic}
\define@key{fams}{svm}{Indo-European}
\define@key{fams}{svk}{Sign Language}
\define@key{fams}{sfm}{Hmong-Mien}
\define@key{fams}{kxq}{Yam}
\define@key{fams}{sox}{Atlantic-Congo}
\define@key{fams}{soc}{Atlantic-Congo}
\define@key{fams}{xog}{Atlantic-Congo}
\define@key{fams}{sog}{Indo-European}
\define@key{fams}{soj}{Indo-European}
\define@key{fams}{sok}{Afro-Asiatic}
\define@key{fams}{sby}{Atlantic-Congo}
\define@key{fams}{sol}{Austronesian}
\define@key{fams}{aaw}{Austronesian}
\define@key{fams}{szs}{Sign Language}
\define@key{fams}{smc}{Nuclear Trans New Guinea}
\define@key{fams}{smu}{Austroasiatic}
\define@key{fams}{sor}{Afro-Asiatic}
\define@key{fams}{kgt}{Atlantic-Congo}
\define@key{fams}{ysg}{Sino-Tibetan}
\define@key{fams}{shc}{Atlantic-Congo}
\define@key{fams}{soo}{Atlantic-Congo}
\define@key{fams}{sod}{Atlantic-Congo}
\define@key{fams}{soe}{Atlantic-Congo}
\define@key{fams}{soi}{Indo-European}
\define@key{fams}{siq}{Bosavi}
\define@key{fams}{sss}{Austroasiatic}
\define@key{fams}{urw}{Nuclear Trans New Guinea}
\define@key{fams}{sbh}{Austronesian}
\define@key{fams}{sqo}{Indo-European}
\define@key{fams}{ays}{Unattested}
\define@key{fams}{sdk}{Ndu}
\define@key{fams}{krz}{Yam}
\define@key{fams}{sfs}{Sign Language}
\define@key{fams}{nit}{Dravidian}
\define@key{fams}{hmy}{Hmong-Mien}
\define@key{fams}{hma}{Hmong-Mien}
\define@key{fams}{sdh}{Indo-European}
\define@key{fams}{bcc}{Indo-European}
\define@key{fams}{fay}{Indo-European}
\define@key{fams}{luz}{Indo-European}
\define@key{fams}{pbt}{Indo-European}
\define@key{fams}{hnd}{Indo-European}
\define@key{fams}{psh}{Indo-European}
\define@key{fams}{psi}{Indo-European}
\define@key{fams}{vro}{Uralic}
\define@key{fams}{nik}{Austroasiatic}
\define@key{fams}{mnn}{Austroasiatic}
\define@key{fams}{uzs}{Turkic}
\define@key{fams}{ghe}{Sino-Tibetan}
\define@key{fams}{ymc}{Sino-Tibetan}
\define@key{fams}{nsd}{Sino-Tibetan}
\define@key{fams}{qxs}{Sino-Tibetan}
\define@key{fams}{pmj}{Sino-Tibetan}
\define@key{fams}{bfs}{Sino-Tibetan}
\define@key{fams}{nre}{Sino-Tibetan}
\define@key{fams}{lrr}{Sino-Tibetan}
\define@key{fams}{tjs}{Sino-Tibetan}
\define@key{fams}{sou}{Tai-Kadai}
\define@key{fams}{hms}{Hmong-Mien}
\define@key{fams}{hmh}{Hmong-Mien}
\define@key{fams}{hmg}{Hmong-Mien}
\define@key{fams}{xtv}{Pama-Nyungan}
\define@key{fams}{ijs}{Ijoid}
\define@key{fams}{fal}{Atlantic-Congo}
\define@key{fams}{nbw}{Atlantic-Congo}
\define@key{fams}{lnl}{Atlantic-Congo}
\define@key{fams}{biv}{Atlantic-Congo}
\define@key{fams}{nnw}{Atlantic-Congo}
\define@key{fams}{snm}{Central Sudanic}
\define@key{fams}{dik}{Nilotic}
\define@key{fams}{dib}{Nilotic}
\define@key{fams}{dks}{Nilotic}
\define@key{fams}{bwq}{Mande}
\define@key{fams}{sbd}{Mande}
\define@key{fams}{sns}{Austronesian}
\define@key{fams}{mqm}{Austronesian}
\define@key{fams}{mcy}{Austronesian}
\define@key{fams}{vbb}{Austronesian}
\define@key{fams}{lmf}{Austronesian}
\define@key{fams}{agy}{Austronesian}
\define@key{fams}{ksc}{Austronesian}
\define@key{fams}{bln}{Austronesian}
\define@key{fams}{plv}{Austronesian}
\define@key{fams}{bzc}{Austronesian}
\define@key{fams}{osu}{Nuclear Torricelli}
\define@key{fams}{aws}{Nuclear Trans New Guinea}
\define@key{fams}{omw}{Nuclear Trans New Guinea}
\define@key{fams}{ams}{Japonic}
\define@key{fams}{hax}{Haida}
\define@key{fams}{tce}{Athabaskan-Eyak-Tlingit}
\define@key{fams}{caf}{Athabaskan-Eyak-Tlingit}
\define@key{fams}{twr}{Uto-Aztecan}
\define@key{fams}{tcu}{Uto-Aztecan}
\define@key{fams}{npl}{Uto-Aztecan}
\define@key{fams}{tla}{Uto-Aztecan}
\define@key{fams}{crj}{Algic}
\define@key{fams}{peq}{Pomoan}
\define@key{fams}{qup}{Quechuan}
\define@key{fams}{qxo}{Quechuan}
\define@key{fams}{ayc}{Aymaran}
\define@key{fams}{meh}{Otomanguean}
\define@key{fams}{mit}{Otomanguean}
\define@key{fams}{mxy}{Otomanguean}
\define@key{fams}{rgs}{Austronesian}
\define@key{fams}{giz}{Afro-Asiatic}
\define@key{fams}{cpy}{Arawakan}
\define@key{fams}{itd}{Austronesian}
\define@key{fams}{csp}{Sino-Tibetan}
\define@key{fams}{sct}{Austroasiatic}
\define@key{fams}{sqq}{Austroasiatic}
\define@key{fams}{sww}{Austronesian}
\define@key{fams}{sow}{Border}
\define@key{fams}{vmq}{Otomanguean}
\define@key{fams}{vmp}{Otomanguean}
\define@key{fams}{sqs}{Sign Language}
\define@key{fams}{sci}{Austronesian}
\define@key{fams}{seo}{Isolate}
\define@key{fams}{swp}{Austronesian}
\define@key{fams}{sxb}{Atlantic-Congo}
\define@key{fams}{ssc}{Atlantic-Congo}
\define@key{fams}{sut}{Otomanguean}
\define@key{fams}{apd}{Afro-Asiatic}
\define@key{fams}{pga}{Afro-Asiatic}
\define@key{fams}{sgi}{Atlantic-Congo}
\define@key{fams}{sug}{Nuclear Trans New Guinea}
\define@key{fams}{kzs}{Austronesian}
\define@key{fams}{zsu}{Austronesian}
\define@key{fams}{syk}{Afro-Asiatic}
\define@key{fams}{szn}{Austronesian}
\define@key{fams}{srg}{Austronesian}
\define@key{fams}{sqm}{Atlantic-Congo}
\define@key{fams}{siv}{Sepik}
\define@key{fams}{six}{Nuclear Trans New Guinea}
\define@key{fams}{suw}{Atlantic-Congo}
\define@key{fams}{smw}{Austronesian}
\define@key{fams}{sux}{Isolate}
\define@key{fams}{csv}{Sino-Tibetan}
\define@key{fams}{ssk}{Sino-Tibetan}
\define@key{fams}{suz}{Sino-Tibetan}
\define@key{fams}{syo}{Austroasiatic}
\define@key{fams}{sbj}{Maban}
\define@key{fams}{sgd}{Austronesian}
\define@key{fams}{sjp}{Indo-European}
\define@key{fams}{tdl}{Atlantic-Congo}
\define@key{fams}{sde}{Atlantic-Congo}
\define@key{fams}{mdz}{Tupian}
\define@key{fams}{sru}{Tupian}
\define@key{fams}{swx}{Arawan}
\define@key{fams}{sqn}{Iroquoian}
\define@key{fams}{ssu}{Angan}
\define@key{fams}{sdj}{Atlantic-Congo}
\define@key{fams}{swu}{Austronesian}
\define@key{fams}{suy}{Nuclear-Macro-Je}
\define@key{fams}{swg}{Indo-European}
\define@key{fams}{slf}{Sign Language}
\define@key{fams}{sgg}{Sign Language}
\define@key{fams}{ssr}{Sign Language}
\define@key{fams}{xdk}{Pama-Nyungan}
\define@key{fams}{syl}{Indo-European}
\define@key{fams}{zoq}{Mixe-Zoque}
\define@key{fams}{nhc}{Uto-Aztecan}
\define@key{fams}{zat}{Otomanguean}
\define@key{fams}{knv}{Isolate}
\define@key{fams}{tzx}{Lower Sepik-Ramu}
\define@key{fams}{xtt}{Otomanguean}
\define@key{fams}{lts}{Atlantic-Congo}
\define@key{fams}{dsq}{Songhay}
\define@key{fams}{tdy}{Austronesian}
\define@key{fams}{rob}{Austronesian}
\define@key{fams}{tcd}{Atlantic-Congo}
\define@key{fams}{klg}{Austronesian}
\define@key{fams}{bgs}{Austronesian}
\define@key{fams}{mvv}{Austronesian}
\define@key{fams}{tgz}{Pama-Nyungan}
\define@key{fams}{tbm}{Atlantic-Congo}
\define@key{fams}{tda}{Songhay}
\define@key{fams}{tgx}{Athabaskan-Eyak-Tlingit}
\define@key{fams}{tgj}{Sino-Tibetan}
\define@key{fams}{tgw}{Atlantic-Congo}
\define@key{fams}{tht}{Athabaskan-Eyak-Tlingit}
\define@key{fams}{blt}{Tai-Kadai}
\define@key{fams}{tyj}{Tai-Kadai}
\define@key{fams}{tyr}{Tai-Kadai}
\define@key{fams}{twh}{Tai-Kadai}
\define@key{fams}{tiz}{Tai-Kadai}
\define@key{fams}{taw}{Nuclear Trans New Guinea}
\define@key{fams}{aos}{Border}
\define@key{fams}{tlq}{Austroasiatic}
\define@key{fams}{thi}{Tai-Kadai}
\define@key{fams}{tjl}{Tai-Kadai}
\define@key{fams}{tdd}{Tai-Kadai}
\define@key{fams}{ago}{Angan}
\define@key{fams}{tnq}{Arawakan}
\define@key{fams}{tpo}{Tai-Kadai}
\define@key{fams}{uar}{Eleman}
\define@key{fams}{tmm}{Tai-Kadai}
\define@key{fams}{cuu}{Tai-Kadai}
\define@key{fams}{acq}{Afro-Asiatic}
\define@key{fams}{pee}{Austronesian}
\define@key{fams}{tdj}{Austronesian}
\define@key{fams}{abh}{Afro-Asiatic}
\define@key{fams}{tja}{Kru}
\define@key{fams}{tkz}{Austroasiatic}
\define@key{fams}{nho}{Austronesian}
\define@key{fams}{tke}{Atlantic-Congo}
\define@key{fams}{tak}{Afro-Asiatic}
\define@key{fams}{tdf}{Austroasiatic}
\define@key{fams}{tlr}{Austronesian}
\define@key{fams}{tlv}{Austronesian}
\define@key{fams}{tal}{Afro-Asiatic}
\define@key{fams}{tln}{Austronesian}
\define@key{fams}{tlk}{Austronesian}
\define@key{fams}{tzl}{Artificial Language}
\define@key{fams}{yta}{Sino-Tibetan}
\define@key{fams}{tcl}{Sino-Tibetan}
\define@key{fams}{tmn}{Austronesian}
\define@key{fams}{tmz}{Cariban}
\define@key{fams}{vmx}{Otomanguean}
\define@key{fams}{ten}{Tucanoan}
\define@key{fams}{tls}{Austronesian}
\define@key{fams}{xxt}{Isolate}
\define@key{fams}{tdk}{Afro-Asiatic}
\define@key{fams}{tmy}{Austronesian}
\define@key{fams}{tax}{Afro-Asiatic}
\define@key{fams}{tml}{Nuclear Trans New Guinea}
\define@key{fams}{tpu}{Austroasiatic}
\define@key{fams}{low}{Austronesian}
\define@key{fams}{tpv}{Austronesian}
\define@key{fams}{tcm}{Isolate}
\define@key{fams}{tni}{Austronesian}
\define@key{fams}{tdx}{Austronesian}
\define@key{fams}{tgn}{Austronesian}
\define@key{fams}{tnx}{Austronesian}
\define@key{fams}{tnv}{Indo-European}
\define@key{fams}{txg}{Sino-Tibetan}
\define@key{fams}{tgp}{Austronesian}
\define@key{fams}{tkx}{Nuclear Trans New Guinea}
\define@key{fams}{tgu}{Lower Sepik-Ramu}
\define@key{fams}{tbs}{Lower Sepik-Ramu}
\define@key{fams}{ytl}{Sino-Tibetan}
\define@key{fams}{tbe}{Austronesian}
\define@key{fams}{uji}{Atlantic-Congo}
\define@key{fams}{txy}{Austronesian}
\define@key{fams}{xnj}{Atlantic-Congo}
\define@key{fams}{qcs}{Mixe-Zoque}
\define@key{fams}{afp}{Arafundi}
\define@key{fams}{taf}{Tupian}
\define@key{fams}{txj}{Saharan}
\define@key{fams}{tpf}{Austronesian}
\define@key{fams}{txr}{Unclassifiable}
\define@key{fams}{tdm}{Isolate}
\define@key{fams}{twq}{Songhay}
\define@key{fams}{tmt}{Austronesian}
\define@key{fams}{ttd}{Goilalan}
\define@key{fams}{tco}{Sino-Tibetan}
\define@key{fams}{tpa}{Austronesian}
\define@key{fams}{tad}{Lakes Plain}
\define@key{fams}{tvs}{Atlantic-Congo}
\define@key{fams}{tvn}{Sino-Tibetan}
\define@key{fams}{rmu}{Speech Register}
\define@key{fams}{twl}{Atlantic-Congo}
\define@key{fams}{xtw}{Nambiquaran}
\define@key{fams}{ttq}{Afro-Asiatic}
\define@key{fams}{twy}{Austronesian}
\define@key{fams}{tbp}{Lakes Plain}
\define@key{fams}{tcp}{Sino-Tibetan}
\define@key{fams}{ayy}{Unattested}
\define@key{fams}{tas}{Pidgin}
\define@key{fams}{tnu}{Tai-Kadai}
\define@key{fams}{tys}{Tai-Kadai}
\define@key{fams}{tyt}{Tai-Kadai}
\define@key{fams}{tyz}{Tai-Kadai}
\define@key{fams}{tck}{Atlantic-Congo}
\define@key{fams}{bqa}{Atlantic-Congo}
\define@key{fams}{dtu}{Dogon}
\define@key{fams}{tsy}{Sign Language}
\define@key{fams}{tcw}{Totonacan}
\define@key{fams}{tuq}{Saharan}
\define@key{fams}{tkq}{Atlantic-Congo}
\define@key{fams}{lor}{Atlantic-Congo}
\define@key{fams}{tfo}{Geelvink Bay}
\define@key{fams}{twe}{Timor-Alor-Pantar}
\define@key{fams}{ztt}{Otomanguean}
\define@key{fams}{teg}{Atlantic-Congo}
\define@key{fams}{tyx}{Atlantic-Congo}
\define@key{fams}{lli}{Atlantic-Congo}
\define@key{fams}{ebo}{Atlantic-Congo}
\define@key{fams}{tyi}{Atlantic-Congo}
\define@key{fams}{tvm}{Austronesian}
\define@key{fams}{tlt}{Austronesian}
\define@key{fams}{nhv}{Uto-Aztecan}
\define@key{fams}{tjo}{Afro-Asiatic}
\define@key{fams}{tbt}{Atlantic-Congo}
\define@key{fams}{tmv}{Atlantic-Congo}
\define@key{fams}{tqb}{Tupian}
\define@key{fams}{tdo}{Atlantic-Congo}
\define@key{fams}{soz}{Atlantic-Congo}
\define@key{fams}{tmo}{Austroasiatic}
\define@key{fams}{ott}{Otomanguean}
\define@key{fams}{tmw}{Austronesian}
\define@key{fams}{quw}{Quechuan}
\define@key{fams}{otn}{Otomanguean}
\define@key{fams}{dtk}{Dogon}
\define@key{fams}{tes}{Austronesian}
\define@key{fams}{pah}{Tupian}
\define@key{fams}{tqn}{Sahaptian}
\define@key{fams}{tns}{Austronesian}
\define@key{fams}{tct}{Tai-Kadai}
\define@key{fams}{tev}{Austronesian}
\define@key{fams}{cux}{Otomanguean}
\define@key{fams}{cte}{Otomanguean}
\define@key{fams}{ted}{Kru}
\define@key{fams}{tef}{Austroasiatic}
\define@key{fams}{trb}{Austronesian}
\define@key{fams}{twg}{Timor-Alor-Pantar}
\define@key{fams}{tec}{Nilotic}
\define@key{fams}{tmg}{Indo-European}
\define@key{fams}{sjt}{Uralic}
\define@key{fams}{tkg}{Austronesian}
\define@key{fams}{keg}{Temeinic}
\define@key{fams}{twc}{Afro-Asiatic}
\define@key{fams}{tez}{Afro-Asiatic}
\define@key{fams}{tdt}{Austronesian}
\define@key{fams}{tve}{Austronesian}
\define@key{fams}{cut}{Otomanguean}
\define@key{fams}{twx}{Atlantic-Congo}
\define@key{fams}{otx}{Otomanguean}
\define@key{fams}{poq}{Mixe-Zoque}
\define@key{fams}{mxb}{Otomanguean}
\define@key{fams}{thy}{Atlantic-Congo}
\define@key{fams}{thn}{Dravidian}
\define@key{fams}{soa}{Tai-Kadai}
\define@key{fams}{nki}{Sino-Tibetan}
\define@key{fams}{thk}{Atlantic-Congo}
\define@key{fams}{iin}{Pama-Nyungan}
\define@key{fams}{tou}{Austroasiatic}
\define@key{fams}{ytp}{Sino-Tibetan}
\define@key{fams}{txh}{Indo-European}
\define@key{fams}{thu}{Nilotic}
\define@key{fams}{ahi}{Kru}
\define@key{fams}{mnl}{Austronesian}
\define@key{fams}{tbj}{Austronesian}
\define@key{fams}{ngy}{Atlantic-Congo}
\define@key{fams}{lsn}{Sign Language}
\define@key{fams}{tcn}{Sino-Tibetan}
\define@key{fams}{mtx}{Otomanguean}
\define@key{fams}{tia}{Afro-Asiatic}
\define@key{fams}{tiq}{Atlantic-Congo}
\define@key{fams}{boo}{Mande}
\define@key{fams}{tii}{Atlantic-Congo}
\define@key{fams}{nza}{Atlantic-Congo}
\define@key{fams}{txq}{Austronesian}
\define@key{fams}{xtl}{Otomanguean}
\define@key{fams}{tkp}{Austronesian}
\define@key{fams}{otl}{Otomanguean}
\define@key{fams}{zts}{Otomanguean}
\define@key{fams}{tij}{Sino-Tibetan}
\define@key{fams}{tim}{Nuclear Trans New Guinea}
\define@key{fams}{tvy}{Indo-European}
\define@key{fams}{xsb}{Austronesian}
\define@key{fams}{tit}{Isolate}
\define@key{fams}{tpz}{Austronesian}
\define@key{fams}{tpe}{Sino-Tibetan}
\define@key{fams}{tra}{Indo-European}
\define@key{fams}{tic}{Heibanic}
\define@key{fams}{tde}{Dogon}
\define@key{fams}{tdq}{Atlantic-Congo}
\define@key{fams}{ttv}{Austronesian}
\define@key{fams}{lax}{Sino-Tibetan}
\define@key{fams}{tju}{Pama-Nyungan}
\define@key{fams}{tpl}{Otomanguean}
\define@key{fams}{ctl}{Otomanguean}
\define@key{fams}{zpk}{Otomanguean}
\define@key{fams}{nuz}{Uto-Aztecan}
\define@key{fams}{mqh}{Otomanguean}
\define@key{fams}{tmf}{Lengua-Mascoy}
\define@key{fams}{tng}{Afro-Asiatic}
\define@key{fams}{tgh}{Indo-European}
\define@key{fams}{tox}{Austronesian}
\define@key{fams}{tgb}{Austronesian}
\define@key{fams}{taz}{Narrow Talodi}
\define@key{fams}{tdr}{Austroasiatic}
\define@key{fams}{tlg}{Namla-Tofanma}
\define@key{fams}{tfi}{Atlantic-Congo}
\define@key{fams}{tor}{Atlantic-Congo}
\define@key{fams}{tgy}{Atlantic-Congo}
\define@key{fams}{zuh}{Nuclear Trans New Guinea}
\define@key{fams}{xto}{Indo-European}
\define@key{fams}{txb}{Indo-European}
\define@key{fams}{tok}{Artificial Language}
\define@key{fams}{tkn}{Japonic}
\define@key{fams}{lbw}{Austronesian}
\define@key{fams}{tlm}{Austronesian}
\define@key{fams}{tol}{Athabaskan-Eyak-Tlingit}
\define@key{fams}{tod}{Mande}
\define@key{fams}{tdi}{Austronesian}
\define@key{fams}{tom}{Austronesian}
\define@key{fams}{txa}{Austronesian}
\define@key{fams}{ttp}{Austronesian}
\define@key{fams}{txm}{Austronesian}
\define@key{fams}{dtm}{Dogon}
\define@key{fams}{tqp}{Austronesian}
\define@key{fams}{tst}{Songhay}
\define@key{fams}{tnz}{Austroasiatic}
\define@key{fams}{tny}{Atlantic-Congo}
\define@key{fams}{tog}{Atlantic-Congo}
\define@key{fams}{xgf}{Uto-Aztecan}
\define@key{fams}{tjn}{Mande}
\define@key{fams}{tnw}{Austronesian}
\define@key{fams}{txs}{Austronesian}
\define@key{fams}{toz}{Atlantic-Congo}
\define@key{fams}{ttj}{Atlantic-Congo}
\define@key{fams}{toq}{Nilotic}
\define@key{fams}{toy}{Austronesian}
\define@key{fams}{ttu}{Austronesian}
\define@key{fams}{trz}{Chapacuran}
\define@key{fams}{trj}{Afro-Asiatic}
\define@key{fams}{fit}{Uralic}
\define@key{fams}{tdv}{Atlantic-Congo}
\define@key{fams}{tqr}{Narrow Talodi}
\define@key{fams}{dtt}{Dogon}
\define@key{fams}{tno}{Pano-Tacanan}
\define@key{fams}{tei}{Nuclear Torricelli}
\define@key{fams}{als}{Indo-European}
\define@key{fams}{ttl}{Atlantic-Congo}
\define@key{fams}{txo}{Sino-Tibetan}
\define@key{fams}{txe}{Austronesian}
\define@key{fams}{ttk}{Barbacoan}
\define@key{fams}{zph}{Otomanguean}
\define@key{fams}{tqu}{Isolate}
\define@key{fams}{neb}{Mande}
\define@key{fams}{don}{Austronesian}
\define@key{fams}{ttn}{Pauwasi}
\define@key{fams}{xtg}{Indo-European}
\define@key{fams}{trl}{Unclassifiable}
\define@key{fams}{rmg}{Speech Register}
\define@key{fams}{rmd}{Speech Register}
\define@key{fams}{trm}{Indo-European}
\define@key{fams}{tme}{Unattested}
\define@key{fams}{stg}{Austroasiatic}
\define@key{fams}{tip}{Greater Kwerba}
\define@key{fams}{trx}{Austronesian}
\define@key{fams}{tgq}{Austronesian}
\define@key{fams}{trn}{Arawakan}
\define@key{fams}{trf}{Indo-European}
\define@key{fams}{lst}{Sign Language}
\define@key{fams}{tka}{Unattested}
\define@key{fams}{tsa}{Atlantic-Congo}
\define@key{fams}{tsd}{Indo-European}
\define@key{fams}{kvz}{Nuclear Trans New Guinea}
\define@key{fams}{tsb}{Afro-Asiatic}
\define@key{fams}{tsk}{Sino-Tibetan}
\define@key{fams}{txc}{Athabaskan-Eyak-Tlingit}
\define@key{fams}{kdl}{Atlantic-Congo}
\define@key{fams}{xmw}{Austronesian}
\define@key{fams}{tsw}{Atlantic-Congo}
\define@key{fams}{hio}{Khoe-Kwadi}
\define@key{fams}{ldp}{Atlantic-Congo}
\define@key{fams}{lto}{Atlantic-Congo}
\define@key{fams}{fly}{Speech Register}
\define@key{fams}{ttz}{Sino-Tibetan}
\define@key{fams}{tsl}{Tai-Kadai}
\define@key{fams}{tvd}{Atlantic-Congo}
\define@key{fams}{tsh}{Afro-Asiatic}
\define@key{fams}{two}{Atlantic-Congo}
\define@key{fams}{tsc}{Atlantic-Congo}
\define@key{fams}{nrt}{Kalapuyan}
\define@key{fams}{tuy}{Nilotic}
\define@key{fams}{tuj}{North Halmahera}
\define@key{fams}{khc}{Austronesian}
\define@key{fams}{bhq}{Austronesian}
\define@key{fams}{tkf}{Unattested}
\define@key{fams}{tkd}{Austronesian}
\define@key{fams}{tul}{Atlantic-Congo}
\define@key{fams}{tlu}{Austronesian}
\define@key{fams}{tey}{Kadugli-Krongo}
\define@key{fams}{rak}{Austronesian}
\define@key{fams}{krt}{Saharan}
\define@key{fams}{iou}{Nuclear Trans New Guinea}
\define@key{fams}{tum}{Atlantic-Congo}
\define@key{fams}{kku}{Unattested}
\define@key{fams}{xtq}{Indo-European}
\define@key{fams}{tbr}{Kadugli-Krongo}
\define@key{fams}{enh}{Uralic}
\define@key{fams}{trt}{Geelvink Bay}
\define@key{fams}{tse}{Sign Language}
\define@key{fams}{tug}{Atlantic-Congo}
\define@key{fams}{tjg}{Austronesian}
\define@key{fams}{tqq}{Afro-Asiatic}
\define@key{fams}{dza}{Atlantic-Congo}
\define@key{fams}{ttf}{Atlantic-Congo}
\define@key{fams}{tpr}{Tupian}
\define@key{fams}{tpw}{Tupian}
\define@key{fams}{trh}{Dagan}
\define@key{fams}{trd}{Austroasiatic}
\define@key{fams}{twt}{Tupian}
\define@key{fams}{tuz}{Atlantic-Congo}
\define@key{fams}{tch}{Indo-European}
\define@key{fams}{tru}{Afro-Asiatic}
\define@key{fams}{try}{Tai-Kadai}
\define@key{fams}{tqm}{Doso-Turumsa}
\define@key{fams}{ttg}{Austronesian}
\define@key{fams}{tmi}{Austronesian}
\define@key{fams}{mtu}{Otomanguean}
\define@key{fams}{tww}{Walioic}
\define@key{fams}{ifk}{Austronesian}
\define@key{fams}{bov}{Atlantic-Congo}
\define@key{fams}{tud}{Isolate}
\define@key{fams}{tux}{Pano-Tacanan}
\define@key{fams}{xjb}{Pama-Nyungan}
\define@key{fams}{twn}{Atlantic-Congo}
\define@key{fams}{uam}{Unclassifiable}
\define@key{fams}{ksj}{Kwalean}
\define@key{fams}{byc}{Atlantic-Congo}
\define@key{fams}{uba}{Atlantic-Congo}
\define@key{fams}{ubi}{Afro-Asiatic}
\define@key{fams}{ubr}{Austronesian}
\define@key{fams}{cpb}{Arawakan}
\define@key{fams}{uda}{Atlantic-Congo}
\define@key{fams}{udu}{Koman}
\define@key{fams}{ufi}{Nuclear Trans New Guinea}
\define@key{fams}{uga}{Afro-Asiatic}
\define@key{fams}{uge}{Austronesian}
\define@key{fams}{ugo}{Sino-Tibetan}
\define@key{fams}{uha}{Atlantic-Congo}
\define@key{fams}{uis}{South Bougainville}
\define@key{fams}{udj}{Austronesian}
\define@key{fams}{kcf}{Atlantic-Congo}
\define@key{fams}{ukh}{Atlantic-Congo}
\define@key{fams}{umi}{Austronesian}
\define@key{fams}{ukp}{Atlantic-Congo}
\define@key{fams}{akd}{Atlantic-Congo}
\define@key{fams}{ukl}{Sign Language}
\define@key{fams}{uku}{Atlantic-Congo}
\define@key{fams}{ukg}{Nuclear Trans New Guinea}
\define@key{fams}{ukq}{Atlantic-Congo}
\define@key{fams}{ukw}{Atlantic-Congo}
\define@key{fams}{svb}{Austronesian}
\define@key{fams}{ull}{Dravidian}
\define@key{fams}{ulb}{Atlantic-Congo}
\define@key{fams}{ulm}{Austronesian}
\define@key{fams}{ulw}{Misumalpan}
\define@key{fams}{ulu}{Austronesian}
\define@key{fams}{xky}{Austronesian}
\define@key{fams}{gdn}{Dagan}
\define@key{fams}{umd}{Pama-Nyungan}
\define@key{fams}{xum}{Indo-European}
\define@key{fams}{umr}{Isolate}
\define@key{fams}{umg}{Pama-Nyungan}
\define@key{fams}{upi}{Border}
\define@key{fams}{sju}{Uralic}
\define@key{fams}{due}{Austronesian}
\define@key{fams}{umm}{Atlantic-Congo}
\define@key{fams}{umo}{Bororoan}
\define@key{fams}{unz}{Austronesian}
\define@key{fams}{bbn}{Austronesian}
\define@key{fams}{une}{Atlantic-Congo}
\define@key{fams}{xgu}{Worrorran}
\define@key{fams}{uni}{Sko}
\define@key{fams}{uln}{Indo-European}
\define@key{fams}{onu}{Austronesian}
\define@key{fams}{unu}{Austronesian}
\define@key{fams}{tov}{Indo-European}
\define@key{fams}{tku}{Totonacan}
\define@key{fams}{sxu}{Indo-European}
\define@key{fams}{tth}{Austroasiatic}
\define@key{fams}{dmg}{Austronesian}
\define@key{fams}{dna}{Nuclear Trans New Guinea}
\define@key{fams}{xup}{Athabaskan-Eyak-Tlingit}
\define@key{fams}{tau}{Athabaskan-Eyak-Tlingit}
\define@key{fams}{url}{Dravidian}
\define@key{fams}{urm}{Nuclear Trans New Guinea}
\define@key{fams}{uro}{Baining}
\define@key{fams}{xur}{Hurro-Urartian}
\define@key{fams}{urg}{Nuclear Trans New Guinea}
\define@key{fams}{uvh}{Nuclear Trans New Guinea}
\define@key{fams}{urx}{Nuclear Torricelli}
\define@key{fams}{urc}{Giimbiyu}
\define@key{fams}{urv}{Austronesian}
\define@key{fams}{urn}{Austronesian}
\define@key{fams}{urz}{Tupian}
\define@key{fams}{ugy}{Sign Language}
\define@key{fams}{uru}{Tupian}
\define@key{fams}{urp}{Unclassifiable}
\define@key{fams}{usk}{Atlantic-Congo}
\define@key{fams}{ush}{Indo-European}
\define@key{fams}{ulf}{Isolate}
\define@key{fams}{usp}{Mayan}
\define@key{fams}{usi}{Sino-Tibetan}
\define@key{fams}{omo}{Nuclear Trans New Guinea}
\define@key{fams}{wsg}{Dravidian}
\define@key{fams}{utu}{Nuclear Trans New Guinea}
\define@key{fams}{uuu}{Austroasiatic}
\define@key{fams}{evh}{Atlantic-Congo}
\define@key{fams}{usu}{Nuclear Trans New Guinea}
\define@key{fams}{auz}{Afro-Asiatic}
\define@key{fams}{eze}{Atlantic-Congo}
\define@key{fams}{vaa}{Indo-European}
\define@key{fams}{kqu}{Tuu}
\define@key{fams}{vgr}{Indo-European}
\define@key{fams}{dkg}{Atlantic-Congo}
\define@key{fams}{tva}{Austronesian}
\define@key{fams}{vap}{Sino-Tibetan}
\define@key{fams}{vae}{Central Sudanic}
\define@key{fams}{vsv}{Sign Language}
\define@key{fams}{vmv}{Maiduan}
\define@key{fams}{cvn}{Otomanguean}
\define@key{fams}{vlp}{Austronesian}
\define@key{fams}{mkt}{Austronesian}
\define@key{fams}{mlr}{Afro-Asiatic}
\define@key{fams}{mpr}{Austronesian}
\define@key{fams}{vnk}{Austronesian}
\define@key{fams}{vau}{Atlantic-Congo}
\define@key{fams}{vao}{Austronesian}
\define@key{fams}{vah}{Indo-European}
\define@key{fams}{vrs}{Austronesian}
\define@key{fams}{vav}{Indo-European}
\define@key{fams}{vaj}{Kxa}
\define@key{fams}{val}{Austronesian}
\define@key{fams}{vem}{Afro-Asiatic}
\define@key{fams}{vsl}{Sign Language}
\define@key{fams}{xve}{Indo-European}
\define@key{fams}{vec}{Indo-European}
\define@key{fams}{veo}{Chumashan}
\define@key{fams}{vra}{Austronesian}
\define@key{fams}{vid}{Atlantic-Congo}
\define@key{fams}{vig}{Atlantic-Congo}
\define@key{fams}{vil}{Isolate}
\define@key{fams}{dyg}{Unattested}
\define@key{fams}{svc}{Indo-European}
\define@key{fams}{vin}{Atlantic-Congo}
\define@key{fams}{vic}{Indo-European}
\define@key{fams}{vis}{Dravidian}
\define@key{fams}{vit}{Atlantic-Congo}
\define@key{fams}{vto}{Tor-Orya}
\define@key{fams}{vls}{Indo-European}
\define@key{fams}{vol}{Artificial Language}
\define@key{fams}{kch}{Unattested}
\define@key{fams}{vor}{Atlantic-Congo}
\define@key{fams}{vum}{Atlantic-Congo}
\define@key{fams}{vnp}{Austronesian}
\define@key{fams}{vun}{Atlantic-Congo}
\define@key{fams}{msn}{Austronesian}
\define@key{fams}{vut}{Atlantic-Congo}
\define@key{fams}{wbi}{Atlantic-Congo}
\define@key{fams}{wmn}{Austronesian}
\define@key{fams}{wab}{Austronesian}
\define@key{fams}{wbb}{Austronesian}
\define@key{fams}{kmx}{Kiwaian}
\define@key{fams}{wci}{Atlantic-Congo}
\define@key{fams}{wdg}{Nuclear Trans New Guinea}
\define@key{fams}{wbq}{Dravidian}
\define@key{fams}{kxp}{Indo-European}
\define@key{fams}{wdu}{Pama-Nyungan}
\define@key{fams}{wag}{Austronesian}
\define@key{fams}{wrx}{Austronesian}
\define@key{fams}{waj}{Nuclear Trans New Guinea}
\define@key{fams}{wga}{Pama-Nyungan}
\define@key{fams}{wgb}{Austronesian}
\define@key{fams}{wbr}{Indo-European}
\define@key{fams}{fad}{Nuclear Trans New Guinea}
\define@key{fams}{whk}{Austronesian}
\define@key{fams}{wgo}{Austronesian}
\define@key{fams}{wlr}{Austronesian}
\define@key{fams}{wlk}{Athabaskan-Eyak-Tlingit}
\define@key{fams}{wmh}{Austronesian}
\define@key{fams}{atr}{Cariban}
\define@key{fams}{wli}{North Halmahera}
\define@key{fams}{wja}{Atlantic-Congo}
\define@key{fams}{wav}{Atlantic-Congo}
\define@key{fams}{wwb}{Unclassifiable}
\define@key{fams}{wkd}{Austronesian}
\define@key{fams}{waf}{Unattested}
\define@key{fams}{lgl}{Austronesian}
\define@key{fams}{wlw}{Nuclear Trans New Guinea}
\define@key{fams}{wly}{Sino-Tibetan}
\define@key{fams}{wll}{Nubian}
\define@key{fams}{wlx}{Atlantic-Congo}
\define@key{fams}{waa}{Sahaptian}
\define@key{fams}{wln}{Indo-European}
\define@key{fams}{wae}{Indo-European}
\define@key{fams}{ola}{Sino-Tibetan}
\define@key{fams}{wmc}{Nuclear Trans New Guinea}
\define@key{fams}{wmi}{Pama-Nyungan}
\define@key{fams}{lbq}{Austronesian}
\define@key{fams}{waz}{Austronesian}
\define@key{fams}{qyp}{Algic}
\define@key{fams}{wnp}{Nuclear Torricelli}
\define@key{fams}{wnb}{Nuclear Trans New Guinea}
\define@key{fams}{nnp}{Sino-Tibetan}
\define@key{fams}{wbh}{Atlantic-Congo}
\define@key{fams}{wdd}{Atlantic-Congo}
\define@key{fams}{wad}{Austronesian}
\define@key{fams}{mfi}{Afro-Asiatic}
\define@key{fams}{wne}{Indo-European}
\define@key{fams}{hwa}{Kru}
\define@key{fams}{wnm}{Pama-Nyungan}
\define@key{fams}{lwg}{Atlantic-Congo}
\define@key{fams}{wng}{Nuclear Trans New Guinea}
\define@key{fams}{jub}{Atlantic-Congo}
\define@key{fams}{wno}{Nuclear Trans New Guinea}
\define@key{fams}{wnk}{Austronesian}
\define@key{fams}{wny}{Garrwan}
\define@key{fams}{juk}{Atlantic-Congo}
\define@key{fams}{juw}{Atlantic-Congo}
\define@key{fams}{wbf}{Atlantic-Congo}
\define@key{fams}{tci}{Yam}
\define@key{fams}{srv}{Austronesian}
\define@key{fams}{bpe}{Sko}
\define@key{fams}{wre}{Unattested}
\define@key{fams}{wai}{Unattested}
\define@key{fams}{wri}{Pama-Nyungan}
\define@key{fams}{wbe}{Lakes Plain}
\define@key{fams}{aml}{Austroasiatic}
\define@key{fams}{wji}{Afro-Asiatic}
\define@key{fams}{bgv}{Anim}
\define@key{fams}{wrl}{Pama-Nyungan}
\define@key{fams}{wrn}{Heibanic}
\define@key{fams}{wru}{Austronesian}
\define@key{fams}{wrv}{Suki-Gogodala}
\define@key{fams}{wss}{Atlantic-Congo}
\define@key{fams}{gsp}{Nuclear Trans New Guinea}
\define@key{fams}{wsu}{Unattested}
\define@key{fams}{wtk}{Sepik}
\define@key{fams}{wah}{Austronesian}
\define@key{fams}{wuy}{Austronesian}
\define@key{fams}{www}{Atlantic-Congo}
\define@key{fams}{wow}{Austronesian}
\define@key{fams}{wxa}{Sino-Tibetan}
\define@key{fams}{ctt}{Dravidian}
\define@key{fams}{wyr}{Tupian}
\define@key{fams}{weh}{Atlantic-Congo}
\define@key{fams}{wew}{Austronesian}
\define@key{fams}{wlh}{Austronesian}
\define@key{fams}{klh}{Nuclear Trans New Guinea}
\define@key{fams}{wei}{Anim}
\define@key{fams}{gxx}{Kru}
\define@key{fams}{ywl}{Sino-Tibetan}
\define@key{fams}{hmw}{Hmong-Mien}
\define@key{fams}{ojw}{Algic}
\define@key{fams}{tqt}{Totonacan}
\define@key{fams}{yih}{Indo-European}
\define@key{fams}{pnb}{Indo-European}
\define@key{fams}{lcp}{Austroasiatic}
\define@key{fams}{kuf}{Austroasiatic}
\define@key{fams}{mut}{Dravidian}
\define@key{fams}{kyu}{Sino-Tibetan}
\define@key{fams}{tdg}{Sino-Tibetan}
\define@key{fams}{wmg}{Sino-Tibetan}
\define@key{fams}{raf}{Sino-Tibetan}
\define@key{fams}{mmr}{Hmong-Mien}
\define@key{fams}{lia}{Atlantic-Congo}
\define@key{fams}{xwl}{Atlantic-Congo}
\define@key{fams}{bbp}{Atlantic-Congo}
\define@key{fams}{ssl}{Atlantic-Congo}
\define@key{fams}{krw}{Kru}
\define@key{fams}{nnd}{Austronesian}
\define@key{fams}{uve}{Austronesian}
\define@key{fams}{mss}{Austronesian}
\define@key{fams}{lmj}{Austronesian}
\define@key{fams}{drn}{Austronesian}
\define@key{fams}{suc}{Austronesian}
\define@key{fams}{twb}{Austronesian}
\define@key{fams}{pne}{Austronesian}
\define@key{fams}{zbw}{Austronesian}
\define@key{fams}{dnw}{Nuclear Trans New Guinea}
\define@key{fams}{nhw}{Uto-Aztecan}
\define@key{fams}{pua}{Tarascan}
\define@key{fams}{gnw}{Tupian}
\define@key{fams}{jmx}{Otomanguean}
\define@key{fams}{tnb}{Chibchan}
\define@key{fams}{amw}{Afro-Asiatic}
\define@key{fams}{azn}{Uto-Aztecan}
\define@key{fams}{wwo}{Austronesian}
\define@key{fams}{wea}{Sino-Tibetan}
\define@key{fams}{wec}{Kru}
\define@key{fams}{woy}{Unattested}
\define@key{fams}{lwh}{Tai-Kadai}
\define@key{fams}{giw}{Tai-Kadai}
\define@key{fams}{tnp}{Austronesian}
\define@key{fams}{tua}{Nuclear Torricelli}
\define@key{fams}{mtp}{Matacoan}
\define@key{fams}{wlv}{Matacoan}
\define@key{fams}{wik}{Pama-Nyungan}
\define@key{fams}{wie}{Pama-Nyungan}
\define@key{fams}{wij}{Pama-Nyungan}
\define@key{fams}{wif}{Unattested}
\define@key{fams}{wih}{Pama-Nyungan}
\define@key{fams}{wua}{Pama-Nyungan}
\define@key{fams}{wil}{Worrorran}
\define@key{fams}{wit}{Wintuan}
\define@key{fams}{gdr}{Eastern Trans-Fly}
\define@key{fams}{wrh}{Pama-Nyungan}
\define@key{fams}{wir}{Tupian}
\define@key{fams}{wiu}{Isolate}
\define@key{fams}{xwc}{Siouan}
\define@key{fams}{woc}{Austronesian}
\define@key{fams}{wbw}{Austronesian}
\define@key{fams}{wyi}{Pama-Nyungan}
\define@key{fams}{jod}{Mande}
\define@key{fams}{wod}{Nuclear Trans New Guinea}
\define@key{fams}{wle}{Afro-Asiatic}
\define@key{fams}{wom}{Atlantic-Congo}
\define@key{fams}{wmo}{Nuclear Torricelli}
\define@key{fams}{won}{Atlantic-Congo}
\define@key{fams}{cwd}{Algic}
\define@key{fams}{kda}{Pama-Nyungan}
\define@key{fams}{wor}{Geelvink Bay}
\define@key{fams}{jud}{Mande}
\define@key{fams}{wsv}{Indo-European}
\define@key{fams}{wtw}{Austronesian}
\define@key{fams}{wud}{Atlantic-Congo}
\define@key{fams}{qgu}{Pama-Nyungan}
\define@key{fams}{wlu}{Pama-Nyungan}
\define@key{fams}{wux}{Limilngan-Wulna}
\define@key{fams}{bqm}{Atlantic-Congo}
\define@key{fams}{wum}{Atlantic-Congo}
\define@key{fams}{ywu}{Sino-Tibetan}
\define@key{fams}{bwn}{Hmong-Mien}
\define@key{fams}{wub}{Worrorran}
\define@key{fams}{wur}{Marrku-Wurrugu}
\define@key{fams}{yig}{Sino-Tibetan}
\define@key{fams}{bse}{Atlantic-Congo}
\define@key{fams}{wsi}{Austronesian}
\define@key{fams}{wuh}{Sino-Tibetan}
\define@key{fams}{wut}{Sko}
\define@key{fams}{wuv}{Austronesian}
\define@key{fams}{wym}{Indo-European}
\define@key{fams}{zax}{Otomanguean}
\define@key{fams}{xkr}{Nuclear-Macro-Je}
\define@key{fams}{xan}{Afro-Asiatic}
\define@key{fams}{ztg}{Otomanguean}
\define@key{fams}{axx}{Austronesian}
\define@key{fams}{xeg}{Tuu}
\define@key{fams}{xet}{Tupian}
\define@key{fams}{hsn}{Sino-Tibetan}
\define@key{fams}{sjo}{Tungusic}
\define@key{fams}{asn}{Tupian}
\define@key{fams}{xiy}{Tupian}
\define@key{fams}{xip}{Unattested}
\define@key{fams}{xii}{Khoe-Kwadi}
\define@key{fams}{xoo}{Isolate}
\define@key{fams}{xwe}{Atlantic-Congo}
\define@key{fams}{tyy}{Atlantic-Congo}
\define@key{fams}{muu}{Afro-Asiatic}
\define@key{fams}{yar}{Cariban}
\define@key{fams}{ybn}{Arawakan}
\define@key{fams}{ybm}{Nuclear Trans New Guinea}
\define@key{fams}{ybo}{Nuclear Trans New Guinea}
\define@key{fams}{ekr}{Atlantic-Congo}
\define@key{fams}{rys}{Japonic}
\define@key{fams}{wfg}{Pauwasi}
\define@key{fams}{ygm}{Nuclear Trans New Guinea}
\define@key{fams}{ygw}{Angan}
\define@key{fams}{rhp}{Nuclear Torricelli}
\define@key{fams}{ner}{Konda-Yahadian}
\define@key{fams}{ynu}{Tucanoan}
\define@key{fams}{iyx}{Atlantic-Congo}
\define@key{fams}{ykk}{Austronesian}
\define@key{fams}{ybh}{Sino-Tibetan}
\define@key{fams}{xyl}{Unattested}
\define@key{fams}{yba}{Atlantic-Congo}
\define@key{fams}{jal}{Austronesian}
\define@key{fams}{zpu}{Otomanguean}
\define@key{fams}{yal}{Mande}
\define@key{fams}{ymp}{Austronesian}
\define@key{fams}{yat}{Atlantic-Congo}
\define@key{fams}{ymb}{Nuclear Torricelli}
\define@key{fams}{yme}{Peba-Yagua}
\define@key{fams}{ymn}{Austronesian}
\define@key{fams}{qur}{Quechuan}
\define@key{fams}{yda}{Pama-Nyungan}
\define@key{fams}{dym}{Dogon}
\define@key{fams}{xyb}{Pama-Nyungan}
\define@key{fams}{zyg}{Tai-Kadai}
\define@key{fams}{jng}{Yangmanic}
\define@key{fams}{yng}{Atlantic-Congo}
\define@key{fams}{bsx}{Atlantic-Congo}
\define@key{fams}{yav}{Atlantic-Congo}
\define@key{fams}{ygl}{Nuclear Torricelli}
\define@key{fams}{ymo}{Nuclear Torricelli}
\define@key{fams}{yde}{Nuclear Torricelli}
\define@key{fams}{ynl}{Nuclear Trans New Guinea}
\define@key{fams}{tjj}{Pama-Nyungan}
\define@key{fams}{ysm}{Sign Language}
\define@key{fams}{jay}{Pama-Nyungan}
\define@key{fams}{guu}{Yanomamic}
\define@key{fams}{asy}{Nuclear Trans New Guinea}
\define@key{fams}{yre}{Mande}
\define@key{fams}{yev}{Nuclear Torricelli}
\define@key{fams}{yrw}{Nuclear Trans New Guinea}
\define@key{fams}{zae}{Otomanguean}
\define@key{fams}{yro}{Yanomamic}
\define@key{fams}{yko}{Atlantic-Congo}
\define@key{fams}{zty}{Otomanguean}
\define@key{fams}{yla}{Keram}
\define@key{fams}{yuw}{Nuclear Trans New Guinea}
\define@key{fams}{jau}{Austronesian}
\define@key{fams}{yyu}{Nuclear Torricelli}
\define@key{fams}{zpb}{Otomanguean}
\define@key{fams}{qux}{Quechuan}
\define@key{fams}{yvt}{Arawakan}
\define@key{fams}{yww}{Pama-Nyungan}
\define@key{fams}{ywn}{Pano-Tacanan}
\define@key{fams}{yaw}{Arawakan}
\define@key{fams}{yby}{Nuclear Trans New Guinea}
\define@key{fams}{ybx}{Walioic}
\define@key{fams}{ykr}{Nuclear Trans New Guinea}
\define@key{fams}{yel}{Atlantic-Congo}
\define@key{fams}{ylg}{Ndu}
\define@key{fams}{ynq}{Atlantic-Congo}
\define@key{fams}{yec}{Mixed Language}
\define@key{fams}{yei}{Atlantic-Congo}
\define@key{fams}{yra}{Isolate}
\define@key{fams}{gop}{Austronesian}
\define@key{fams}{yrn}{Tai-Kadai}
\define@key{fams}{yeu}{Dravidian}
\define@key{fams}{yes}{Atlantic-Congo}
\define@key{fams}{yet}{Isolate}
\define@key{fams}{yej}{Indo-European}
\define@key{fams}{ydg}{Indo-European}
\define@key{fams}{yim}{Sino-Tibetan}
\define@key{fams}{kvu}{Sino-Tibetan}
\define@key{fams}{yin}{Austroasiatic}
\define@key{fams}{yil}{Pama-Nyungan}
\define@key{fams}{ywg}{Pama-Nyungan}
\define@key{fams}{kvy}{Sino-Tibetan}
\define@key{fams}{yxm}{Pama-Nyungan}
\define@key{fams}{ljw}{Pama-Nyungan}
\define@key{fams}{yiy}{Pama-Nyungan}
\define@key{fams}{yis}{Nuclear Torricelli}
\define@key{fams}{gek}{Afro-Asiatic}
\define@key{fams}{yob}{Austronesian}
\define@key{fams}{gud}{Kru}
\define@key{fams}{yog}{Austronesian}
\define@key{fams}{ydk}{Nuclear Trans New Guinea}
\define@key{fams}{yki}{Austronesian}
\define@key{fams}{ygs}{Sign Language}
\define@key{fams}{xty}{Otomanguean}
\define@key{fams}{pil}{Atlantic-Congo}
\define@key{fams}{yoi}{Japonic}
\define@key{fams}{sxk}{Kalapuyan}
\define@key{fams}{nru}{Sino-Tibetan}
\define@key{fams}{zyn}{Tai-Kadai}
\define@key{fams}{zyb}{Tai-Kadai}
\define@key{fams}{yno}{Tai-Kadai}
\define@key{fams}{yon}{Nuclear Trans New Guinea}
\define@key{fams}{yut}{Nuclear Trans New Guinea}
\define@key{fams}{mts}{Pano-Tacanan}
\define@key{fams}{yox}{Japonic}
\define@key{fams}{yot}{Atlantic-Congo}
\define@key{fams}{zyj}{Tai-Kadai}
\define@key{fams}{ytw}{Nuclear Trans New Guinea}
\define@key{fams}{yoy}{Tai-Kadai}
\define@key{fams}{nua}{Austronesian}
\define@key{fams}{msd}{Sign Language}
\define@key{fams}{mvg}{Otomanguean}
\define@key{fams}{yub}{Pama-Nyungan}
\define@key{fams}{ysl}{Sign Language}
\define@key{fams}{ygu}{Unattested}
\define@key{fams}{yab}{Naduhup}
\define@key{fams}{omk}{Yukaghir}
\define@key{fams}{ybl}{Atlantic-Congo}
\define@key{fams}{yuq}{Tupian}
\define@key{fams}{ljx}{Pama-Nyungan}
\define@key{fams}{mab}{Otomanguean}
\define@key{fams}{yau}{Isolate}
\define@key{fams}{ztx}{Otomanguean}
\define@key{fams}{kji}{Austronesian}
\define@key{fams}{nhi}{Uto-Aztecan}
\define@key{fams}{ctz}{Otomanguean}
\define@key{fams}{atb}{Sino-Tibetan}
\define@key{fams}{zkr}{Sino-Tibetan}
\define@key{fams}{zsl}{Sign Language}
\define@key{fams}{zak}{Atlantic-Congo}
\define@key{fams}{zau}{Sino-Tibetan}
\define@key{fams}{zna}{Atlantic-Congo}
\define@key{fams}{zah}{Afro-Asiatic}
\define@key{fams}{zpw}{Otomanguean}
\define@key{fams}{zaj}{Atlantic-Congo}
\define@key{fams}{zbu}{Afro-Asiatic}
\define@key{fams}{zaz}{Afro-Asiatic}
\define@key{fams}{zal}{Sino-Tibetan}
\define@key{fams}{kxk}{Sino-Tibetan}
\define@key{fams}{zwa}{Afro-Asiatic}
\define@key{fams}{jaj}{Austronesian}
\define@key{fams}{zua}{Afro-Asiatic}
\define@key{fams}{dhm}{Atlantic-Congo}
\define@key{fams}{zeg}{Austronesian}
\define@key{fams}{czn}{Otomanguean}
\define@key{fams}{zhb}{Sino-Tibetan}
\define@key{fams}{xzh}{Sino-Tibetan}
\define@key{fams}{zhi}{Atlantic-Congo}
\define@key{fams}{zhw}{Atlantic-Congo}
\define@key{fams}{zia}{Nuclear Trans New Guinea}
\define@key{fams}{zil}{Mande}
\define@key{fams}{ziw}{Atlantic-Congo}
\define@key{fams}{zib}{Sign Language}
\define@key{fams}{zmb}{Atlantic-Congo}
\define@key{fams}{zin}{Atlantic-Congo}
\define@key{fams}{sih}{Austronesian}
\define@key{fams}{zrn}{Afro-Asiatic}
\define@key{fams}{ziz}{Afro-Asiatic}
\define@key{fams}{pto}{Tupian}
\define@key{fams}{yzk}{Sino-Tibetan}
\define@key{fams}{gbz}{Indo-European}
\define@key{fams}{czt}{Sino-Tibetan}
\define@key{fams}{zom}{Sino-Tibetan}
\define@key{fams}{zla}{Atlantic-Congo}
\define@key{fams}{gnd}{Afro-Asiatic}
\define@key{fams}{zuy}{Afro-Asiatic}
\define@key{fams}{jmb}{Afro-Asiatic}
\define@key{fams}{zzj}{Tai-Kadai}
\define@key{fams}{zyp}{Sino-Tibetan}
%
}
\DeclareOption{wals}{%
  \def\langnames@cs@prefix{wals}%
  \def\langnames@langs@wals@aaa{Ghotuo}
\def\langnames@langs@wals@aab{Arum}
\def\langnames@langs@wals@aac{Ari}
\def\langnames@langs@wals@aad{Amal}
\def\langnames@langs@wals@aae{Arbëreshë Albanian}
\def\langnames@langs@wals@aaf{Aranadan}
\def\langnames@langs@wals@aag{Ambrak}
\def\langnames@langs@wals@aah{Abu' Arapesh}
\def\langnames@langs@wals@aai{Arifama-Miniafia}
\def\langnames@langs@wals@aak{Ankave}
\def\langnames@langs@wals@aal{Afade}
\def\langnames@langs@wals@aan{Anambé}
\def\langnames@langs@wals@aao{Algerian Saharan Arabic}
\def\langnames@langs@wals@aap{Pará Arára}
\def\langnames@langs@wals@aaq{Eastern Abenaki}
\def\langnames@langs@wals@aar{Afar}
\def\langnames@langs@wals@aas{Aasax}
\def\langnames@langs@wals@aat{Arvanitika Albanian}
\def\langnames@langs@wals@aau{Abau}
\def\langnames@langs@wals@aaw{Solong}
\def\langnames@langs@wals@aax{Mandobo Atas}
\def\langnames@langs@wals@aaz{Amarasi}
\def\langnames@langs@wals@aba{Abé}
\def\langnames@langs@wals@abb{Bankon}
\def\langnames@langs@wals@abc{Ambala Ayta}
\def\langnames@langs@wals@abd{Camarines Norte Agta}
\def\langnames@langs@wals@abe{Western Abenaki}
\def\langnames@langs@wals@abf{Abai Sungai}
\def\langnames@langs@wals@abg{Abaga}
\def\langnames@langs@wals@abh{Tajiki Arabic}
\def\langnames@langs@wals@abi{Abidji}
\def\langnames@langs@wals@abj{Akabea}
\def\langnames@langs@wals@abk{Abkhaz}
\def\langnames@langs@wals@abl{Lampung Nyo}
\def\langnames@langs@wals@abm{Abanyom}
\def\langnames@langs@wals@abn{Abua}
\def\langnames@langs@wals@abo{Abon}
\def\langnames@langs@wals@abp{Abenlen Ayta}
\def\langnames@langs@wals@abq{Abaza}
\def\langnames@langs@wals@abr{Abron}
\def\langnames@langs@wals@abs{Ambonese Malay}
\def\langnames@langs@wals@abt{Ambulas}
\def\langnames@langs@wals@abu{Abure}
\def\langnames@langs@wals@abv{Baharna Arabic}
\def\langnames@langs@wals@abw{Pal}
\def\langnames@langs@wals@abx{Inabaknon}
\def\langnames@langs@wals@aby{Aneme Wake}
\def\langnames@langs@wals@abz{Abui}
\def\langnames@langs@wals@aca{Achagua}
\def\langnames@langs@wals@acd{Gikyode}
\def\langnames@langs@wals@ace{Acehnese}
\def\langnames@langs@wals@acf{Saint Lucian Creole French}
\def\langnames@langs@wals@ach{Acoli}
\def\langnames@langs@wals@aci{Akacari}
\def\langnames@langs@wals@ack{Akakora}
\def\langnames@langs@wals@acl{Akarbale}
\def\langnames@langs@wals@acm{Gilit Mesopotamian Arabic}
\def\langnames@langs@wals@acn{Longchuan Achang}
\def\langnames@langs@wals@acp{Eastern Acipa}
\def\langnames@langs@wals@acq{Ta'izzi-Adeni Arabic}
\def\langnames@langs@wals@acr{Achi}
\def\langnames@langs@wals@acs{Acroá}
\def\langnames@langs@wals@acu{Achuar-Shiwiar}
\def\langnames@langs@wals@acv{Achumawi}
\def\langnames@langs@wals@acw{Hijazi Arabic}
\def\langnames@langs@wals@acx{Omani Arabic}
\def\langnames@langs@wals@acy{Cypriot Arabic}
\def\langnames@langs@wals@acz{Acheron}
\def\langnames@langs@wals@ada{Adangme}
\def\langnames@langs@wals@add{Dzodinka}
\def\langnames@langs@wals@ade{Adele}
\def\langnames@langs@wals@adf{Dhofari Arabic}
\def\langnames@langs@wals@adg{Andegerebinha}
\def\langnames@langs@wals@adh{Adhola}
\def\langnames@langs@wals@adi{Bori-Karko}
\def\langnames@langs@wals@adj{Adioukrou}
\def\langnames@langs@wals@adl{Galo}
\def\langnames@langs@wals@adn{Adang}
\def\langnames@langs@wals@ado{Abu}
\def\langnames@langs@wals@adq{Adangbe}
\def\langnames@langs@wals@adr{Adonara}
\def\langnames@langs@wals@ads{Adamorobe Sign Language}
\def\langnames@langs@wals@adt{Adnyamathanha}
\def\langnames@langs@wals@adw{Amundava}
\def\langnames@langs@wals@adx{Amdo Tibetan}
\def\langnames@langs@wals@ady{Adyghe}
\def\langnames@langs@wals@adz{Adzera}
\def\langnames@langs@wals@aea{Areba}
\def\langnames@langs@wals@aeb{Tunisian Arabic}
\def\langnames@langs@wals@aec{Saidi Arabic}
\def\langnames@langs@wals@aed{Argentine Sign Language}
\def\langnames@langs@wals@aee{Northeast Pashayi}
\def\langnames@langs@wals@aek{Haeke}
\def\langnames@langs@wals@ael{Ambele}
\def\langnames@langs@wals@aem{Arem}
\def\langnames@langs@wals@aen{Armenian Sign Language}
\def\langnames@langs@wals@aeq{Aer}
\def\langnames@langs@wals@aer{Eastern Arrernte}
\def\langnames@langs@wals@aes{Alsea-Yaquina}
\def\langnames@langs@wals@aeu{Akeu}
\def\langnames@langs@wals@aew{Ambakich}
\def\langnames@langs@wals@aey{Amele}
\def\langnames@langs@wals@aez{Aeka}
\def\langnames@langs@wals@afb{Gulf Arabic}
\def\langnames@langs@wals@afd{Andai}
\def\langnames@langs@wals@afe{Utugwang-Irungene-Afrike}
\def\langnames@langs@wals@afg{Afghan Sign Language}
\def\langnames@langs@wals@afh{Afrihili}
\def\langnames@langs@wals@afi{Chini}
\def\langnames@langs@wals@afk{Nanubae-Imangae}
\def\langnames@langs@wals@afn{Defaka}
\def\langnames@langs@wals@afo{Ajiri}
\def\langnames@langs@wals@afp{Tapei}
\def\langnames@langs@wals@afr{Afrikaans}
\def\langnames@langs@wals@afs{Afro-Seminole Creole}
\def\langnames@langs@wals@aft{Afitti}
\def\langnames@langs@wals@afu{Awutu}
\def\langnames@langs@wals@afz{Obokuitai}
\def\langnames@langs@wals@aga{Aguano}
\def\langnames@langs@wals@agb{Legbo}
\def\langnames@langs@wals@agc{Agatu}
\def\langnames@langs@wals@agd{Agarabi}
\def\langnames@langs@wals@age{Angal}
\def\langnames@langs@wals@agf{Arguni}
\def\langnames@langs@wals@agg{Angor}
\def\langnames@langs@wals@agh{Ngelima}
\def\langnames@langs@wals@agi{Agariya}
\def\langnames@langs@wals@agj{Argobba}
\def\langnames@langs@wals@agk{Isarog Agta}
\def\langnames@langs@wals@agl{Fembe}
\def\langnames@langs@wals@agm{Angaataha}
\def\langnames@langs@wals@agn{Agutaynen}
\def\langnames@langs@wals@ago{Tainae}
\def\langnames@langs@wals@agq{Aghem}
\def\langnames@langs@wals@agr{Aguaruna}
\def\langnames@langs@wals@ags{Esimbi}
\def\langnames@langs@wals@agt{Central Cagayan Agta}
\def\langnames@langs@wals@agu{Aguacateco}
\def\langnames@langs@wals@agv{Hatang Kayi}
\def\langnames@langs@wals@agw{Kahua}
\def\langnames@langs@wals@agx{Aghul}
\def\langnames@langs@wals@agy{Southern Alta}
\def\langnames@langs@wals@agz{Mt. Iriga Agta}
\def\langnames@langs@wals@aha{Ahanta}
\def\langnames@langs@wals@ahb{Axamb}
\def\langnames@langs@wals@ahg{Qimant}
\def\langnames@langs@wals@ahh{Aghu}
\def\langnames@langs@wals@ahi{Tiagbamrin Aizi}
\def\langnames@langs@wals@ahk{Akha}
\def\langnames@langs@wals@ahl{Igo}
\def\langnames@langs@wals@ahm{Mobumrin Aizi}
\def\langnames@langs@wals@ahn{Àhàn}
\def\langnames@langs@wals@aho{Ahom}
\def\langnames@langs@wals@ahp{Aproumu Aizi}
\def\langnames@langs@wals@ahs{Ashe}
\def\langnames@langs@wals@aht{Ahtena}
\def\langnames@langs@wals@aia{Arosi}
\def\langnames@langs@wals@aib{Ainu (China)}
\def\langnames@langs@wals@aic{Ainbai}
\def\langnames@langs@wals@aid{Alngith}
\def\langnames@langs@wals@aie{Amara}
\def\langnames@langs@wals@aif{Agi}
\def\langnames@langs@wals@aig{Antigua and Barbuda Creole English}
\def\langnames@langs@wals@aih{Ai-Cham}
\def\langnames@langs@wals@aii{Assyrian Neo-Aramaic}
\def\langnames@langs@wals@aij{Lishanid Noshan}
\def\langnames@langs@wals@aik{Akye}
\def\langnames@langs@wals@ail{Aimele}
\def\langnames@langs@wals@aim{Aimol}
\def\langnames@langs@wals@ain{Hokkaido Ainu}
\def\langnames@langs@wals@aio{Aiton}
\def\langnames@langs@wals@aip{Burumakok}
\def\langnames@langs@wals@aiq{Aimaq}
\def\langnames@langs@wals@air{Airoran}
\def\langnames@langs@wals@ait{Arikem}
\def\langnames@langs@wals@aiw{Aari}
\def\langnames@langs@wals@aix{Aighon}
\def\langnames@langs@wals@aiy{Ali}
\def\langnames@langs@wals@aja{Aja (South Sudan)}
\def\langnames@langs@wals@ajg{Aja (Benin)}
\def\langnames@langs@wals@aji{Ajië}
\def\langnames@langs@wals@ajp{South Levantine Arabic}
\def\langnames@langs@wals@ajs{Ghardaia Sign Language}
\def\langnames@langs@wals@aju{Judeo-Moroccan Arabic}
\def\langnames@langs@wals@ajw{Ajawa}
\def\langnames@langs@wals@ajz{Amri Karbi}
\def\langnames@langs@wals@aka{Akan}
\def\langnames@langs@wals@akb{Batak Angkola}
\def\langnames@langs@wals@akc{Mpur}
\def\langnames@langs@wals@akd{Ukpet-Ehom}
\def\langnames@langs@wals@ake{Akawaio-Ingariko}
\def\langnames@langs@wals@akf{Akpa}
\def\langnames@langs@wals@akg{Anakalangu}
\def\langnames@langs@wals@akh{Angal Heneng}
\def\langnames@langs@wals@aki{Aiome}
\def\langnames@langs@wals@akj{Akajeru}
\def\langnames@langs@wals@akk{Akkadian}
\def\langnames@langs@wals@akl{Aklanon}
\def\langnames@langs@wals@akm{Akabo}
\def\langnames@langs@wals@ako{Akurio}
\def\langnames@langs@wals@akp{Siwu}
\def\langnames@langs@wals@akq{Ak}
\def\langnames@langs@wals@akr{Araki}
\def\langnames@langs@wals@aks{Akaselem}
\def\langnames@langs@wals@akt{Akolet}
\def\langnames@langs@wals@aku{Akum}
\def\langnames@langs@wals@akv{Akhvakh}
\def\langnames@langs@wals@akw{Akwa}
\def\langnames@langs@wals@akx{Akakede}
\def\langnames@langs@wals@aky{Akakol}
\def\langnames@langs@wals@akz{Alabama}
\def\langnames@langs@wals@ala{Alago}
\def\langnames@langs@wals@alc{Qawasqar}
\def\langnames@langs@wals@ald{Alladian}
\def\langnames@langs@wals@ale{Aleut}
\def\langnames@langs@wals@alf{Alege}
\def\langnames@langs@wals@alh{Alawa}
\def\langnames@langs@wals@ali{Amaimon}
\def\langnames@langs@wals@alj{Alangan}
\def\langnames@langs@wals@alk{Alak}
\def\langnames@langs@wals@all{Allar}
\def\langnames@langs@wals@alm{Amblong}
\def\langnames@langs@wals@aln{Gheg Albanian}
\def\langnames@langs@wals@alo{Larike-Wakasihu}
\def\langnames@langs@wals@alp{Alune}
\def\langnames@langs@wals@alq{Algonquin}
\def\langnames@langs@wals@alr{Alutor}
\def\langnames@langs@wals@als{Northern Tosk Albanian}
\def\langnames@langs@wals@alt{Southern Altai}
\def\langnames@langs@wals@alu{'Are'are}
\def\langnames@langs@wals@alw{Alaba-K'abeena}
\def\langnames@langs@wals@alx{Mol}
\def\langnames@langs@wals@aly{Alyawarr}
\def\langnames@langs@wals@alz{Alur}
\def\langnames@langs@wals@ama{Amanayé}
\def\langnames@langs@wals@amb{Ambo}
\def\langnames@langs@wals@amc{Amahuaca}
\def\langnames@langs@wals@ame{Yanesha'}
\def\langnames@langs@wals@amf{Hamer-Banna}
\def\langnames@langs@wals@amg{Amurdak}
\def\langnames@langs@wals@amh{Amharic}
\def\langnames@langs@wals@ami{Amis}
\def\langnames@langs@wals@amj{Amdang}
\def\langnames@langs@wals@amk{Ambai}
\def\langnames@langs@wals@aml{War-Jaintia}
\def\langnames@langs@wals@amm{Ama (Papua New Guinea)}
\def\langnames@langs@wals@amn{Amanab}
\def\langnames@langs@wals@amo{Amo}
\def\langnames@langs@wals@amp{Alamblak}
\def\langnames@langs@wals@amq{Amahai}
\def\langnames@langs@wals@amr{Amarakaeri}
\def\langnames@langs@wals@ams{Southern Amami-Oshima}
\def\langnames@langs@wals@amt{Amto}
\def\langnames@langs@wals@amu{Guerrero Amuzgo}
\def\langnames@langs@wals@amv{Ambelau}
\def\langnames@langs@wals@amw{Western Neo-Aramaic}
\def\langnames@langs@wals@amx{Anmatyerre}
\def\langnames@langs@wals@amy{Ami}
\def\langnames@langs@wals@amz{Atampaya}
\def\langnames@langs@wals@ana{Andaqui}
\def\langnames@langs@wals@anb{Andoa}
\def\langnames@langs@wals@anc{Ngas}
\def\langnames@langs@wals@and{Ansus}
\def\langnames@langs@wals@ane{Xârâcùù}
\def\langnames@langs@wals@anf{Animere}
\def\langnames@langs@wals@ang{Old English (ca. 450-1100)}
\def\langnames@langs@wals@anh{Nend}
\def\langnames@langs@wals@ani{Andi}
\def\langnames@langs@wals@anj{Anor}
\def\langnames@langs@wals@ank{Goemai}
\def\langnames@langs@wals@anl{Anu-Hkongso}
\def\langnames@langs@wals@anm{Anal}
\def\langnames@langs@wals@ann{Obolo}
\def\langnames@langs@wals@ano{Andoque}
\def\langnames@langs@wals@anp{Angika}
\def\langnames@langs@wals@anq{Jarawa (India)}
\def\langnames@langs@wals@ans{Anserma}
\def\langnames@langs@wals@ant{Antakarinya}
\def\langnames@langs@wals@anu{Anuak}
\def\langnames@langs@wals@anv{Denya}
\def\langnames@langs@wals@anw{Anaang}
\def\langnames@langs@wals@anx{Andra-Hus}
\def\langnames@langs@wals@any{Anyin}
\def\langnames@langs@wals@anz{Anem}
\def\langnames@langs@wals@aoa{Angolar}
\def\langnames@langs@wals@aob{Abom}
\def\langnames@langs@wals@aoc{Pemon}
\def\langnames@langs@wals@aod{Andarum}
\def\langnames@langs@wals@aoe{Angal Enen}
\def\langnames@langs@wals@aof{Bragat}
\def\langnames@langs@wals@aog{Angoram}
\def\langnames@langs@wals@aoh{Arma}
\def\langnames@langs@wals@aoi{Anindilyakwa}
\def\langnames@langs@wals@aoj{Mufian}
\def\langnames@langs@wals@aok{Arhö}
\def\langnames@langs@wals@aol{Alorese}
\def\langnames@langs@wals@aom{Ömie}
\def\langnames@langs@wals@aon{Bumbita Arapesh}
\def\langnames@langs@wals@aor{Aore}
\def\langnames@langs@wals@aos{Taikat}
\def\langnames@langs@wals@aot{Atong (India)}
\def\langnames@langs@wals@aou{A'ou}
\def\langnames@langs@wals@aox{Atorada}
\def\langnames@langs@wals@aoz{Uab Meto}
\def\langnames@langs@wals@apb{Sa'a}
\def\langnames@langs@wals@apc{North Levantine Arabic}
\def\langnames@langs@wals@apd{Sudanese Arabic}
\def\langnames@langs@wals@ape{Bukiyip}
\def\langnames@langs@wals@apf{Agta-Pahanan}
\def\langnames@langs@wals@apg{Ampanang}
\def\langnames@langs@wals@aph{Athpariya}
\def\langnames@langs@wals@api{Apiaká}
\def\langnames@langs@wals@apj{Jicarilla Apache}
\def\langnames@langs@wals@apk{Kiowa Apache}
\def\langnames@langs@wals@apl{Lipan Apache}
\def\langnames@langs@wals@apm{Mescalero-Chiricahua Apache}
\def\langnames@langs@wals@apn{Apinayé}
\def\langnames@langs@wals@apo{Apalik}
\def\langnames@langs@wals@app{Apma}
\def\langnames@langs@wals@apq{Apucikwar}
\def\langnames@langs@wals@apr{Arop-Lokep}
\def\langnames@langs@wals@aps{Arop-Sissano}
\def\langnames@langs@wals@apt{Apatani}
\def\langnames@langs@wals@apu{Apurinã}
\def\langnames@langs@wals@apv{Alapmunte}
\def\langnames@langs@wals@apw{Western Apache}
\def\langnames@langs@wals@apx{Aputai}
\def\langnames@langs@wals@apy{Apalaí}
\def\langnames@langs@wals@apz{Safeyoka}
\def\langnames@langs@wals@aqc{Archi}
\def\langnames@langs@wals@aqd{Ampari Dogon}
\def\langnames@langs@wals@aqg{Arigidi}
\def\langnames@langs@wals@aqk{Aninka}
\def\langnames@langs@wals@aqm{Atohwaim}
\def\langnames@langs@wals@aqn{Northern Alta}
\def\langnames@langs@wals@aqp{Atakapa}
\def\langnames@langs@wals@aqr{Arhâ}
\def\langnames@langs@wals@aqt{Angaité}
\def\langnames@langs@wals@aqz{Akuntsu}
\def\langnames@langs@wals@arb{Standard Arabic}
\def\langnames@langs@wals@arc{Imperial Aramaic (700-300 BCE)}
\def\langnames@langs@wals@ard{Arabana}
\def\langnames@langs@wals@are{Western Arrarnta}
\def\langnames@langs@wals@arg{Aragonese}
\def\langnames@langs@wals@arh{Arhuaco}
\def\langnames@langs@wals@ari{Arikara}
\def\langnames@langs@wals@arj{Arapaso}
\def\langnames@langs@wals@ark{Arikapú}
\def\langnames@langs@wals@arl{Arabela}
\def\langnames@langs@wals@arn{Mapudungun}
\def\langnames@langs@wals@aro{Araona}
\def\langnames@langs@wals@arp{Arapaho}
\def\langnames@langs@wals@arq{Algerian Arabic}
\def\langnames@langs@wals@arr{Karo (Brazil)}
\def\langnames@langs@wals@ars{Najdi Arabic}
\def\langnames@langs@wals@aru{Aruá (Amazonas State)}
\def\langnames@langs@wals@arv{Arbore}
\def\langnames@langs@wals@arw{Lokono}
\def\langnames@langs@wals@arx{Aruá (Rondonia State)}
\def\langnames@langs@wals@ary{Moroccan Arabic}
\def\langnames@langs@wals@arz{Egyptian Arabic}
\def\langnames@langs@wals@asa{Asu (Tanzania)}
\def\langnames@langs@wals@asb{Assiniboine}
\def\langnames@langs@wals@asc{Casuarina Coast Asmat}
\def\langnames@langs@wals@ase{American Sign Language}
\def\langnames@langs@wals@asf{Auslan}
\def\langnames@langs@wals@asg{Western-Kambari-Cishingini}
\def\langnames@langs@wals@ash{Aewa}
\def\langnames@langs@wals@asi{Buruwai}
\def\langnames@langs@wals@asj{Nsari}
\def\langnames@langs@wals@ask{Ashkun}
\def\langnames@langs@wals@asl{Asilulu}
\def\langnames@langs@wals@asm{Assamese}
\def\langnames@langs@wals@asn{Xingú Asuriní}
\def\langnames@langs@wals@aso{Dano}
\def\langnames@langs@wals@asp{Algerian Sign Language}
\def\langnames@langs@wals@asq{Austrian Sign Language}
\def\langnames@langs@wals@asr{Asuri}
\def\langnames@langs@wals@ass{Ipulo-Olulu}
\def\langnames@langs@wals@ast{Asturian-Leonese-Cantabrian}
\def\langnames@langs@wals@asu{Tocantins Asurini}
\def\langnames@langs@wals@asv{Asoa}
\def\langnames@langs@wals@asw{Australian Aborigines Sign Language}
\def\langnames@langs@wals@asx{Muratayak}
\def\langnames@langs@wals@asy{Yaosakor Asmat}
\def\langnames@langs@wals@asz{As}
\def\langnames@langs@wals@ata{Pele-Ata}
\def\langnames@langs@wals@atb{Zaiwa}
\def\langnames@langs@wals@atc{Atsahuaca}
\def\langnames@langs@wals@atd{Ata Manobo}
\def\langnames@langs@wals@ate{Mand}
\def\langnames@langs@wals@atg{Ivbie North-Okpela-Arhe}
\def\langnames@langs@wals@ati{Attié}
\def\langnames@langs@wals@atj{Atikamekw}
\def\langnames@langs@wals@atk{Ati}
\def\langnames@langs@wals@atl{Mt. Iraya Agta}
\def\langnames@langs@wals@atm{Ata}
\def\langnames@langs@wals@atn{Ashtiani}
\def\langnames@langs@wals@ato{Atong}
\def\langnames@langs@wals@atp{Pudtol Atta}
\def\langnames@langs@wals@atq{Aralle-Tabulahan}
\def\langnames@langs@wals@atr{Waimiri-Atroari}
\def\langnames@langs@wals@ats{Gros Ventre}
\def\langnames@langs@wals@att{Pamplona Atta}
\def\langnames@langs@wals@atu{Reel}
\def\langnames@langs@wals@atv{Northern Altai}
\def\langnames@langs@wals@atw{Atsugewi}
\def\langnames@langs@wals@atx{Arutani}
\def\langnames@langs@wals@aty{Aneityum}
\def\langnames@langs@wals@atz{Arta}
\def\langnames@langs@wals@aua{Asumboa}
\def\langnames@langs@wals@aub{Alugu}
\def\langnames@langs@wals@auc{Waorani}
\def\langnames@langs@wals@aud{Anuta}
\def\langnames@langs@wals@aug{Aguna}
\def\langnames@langs@wals@auh{Aushi}
\def\langnames@langs@wals@aui{Anuki}
\def\langnames@langs@wals@auj{Awjilah}
\def\langnames@langs@wals@auk{Heyo}
\def\langnames@langs@wals@aul{Aulua}
\def\langnames@langs@wals@aum{Asu (Nigeria)}
\def\langnames@langs@wals@aun{Molmo One}
\def\langnames@langs@wals@auo{Auyokawa}
\def\langnames@langs@wals@aup{Makayam}
\def\langnames@langs@wals@auq{Anus}
\def\langnames@langs@wals@aur{Aruek}
\def\langnames@langs@wals@aut{Austral}
\def\langnames@langs@wals@auu{Auye}
\def\langnames@langs@wals@auw{Awyi}
\def\langnames@langs@wals@aux{Aurê y Aurá}
\def\langnames@langs@wals@auy{Awiyaana}
\def\langnames@langs@wals@auz{Uzbeki Arabic}
\def\langnames@langs@wals@ava{Avar}
\def\langnames@langs@wals@avb{Avau}
\def\langnames@langs@wals@avd{Alviri-Vidari}
\def\langnames@langs@wals@ave{Avestan}
\def\langnames@langs@wals@avi{Avikam}
\def\langnames@langs@wals@avk{Kotava}
\def\langnames@langs@wals@avl{Eastern Egyptian Bedawi Arabic}
\def\langnames@langs@wals@avm{Angkamuthi}
\def\langnames@langs@wals@avn{Avatime}
\def\langnames@langs@wals@avo{Agavotaguerra}
\def\langnames@langs@wals@avs{Aushiri}
\def\langnames@langs@wals@avt{Au}
\def\langnames@langs@wals@avu{Avokaya}
\def\langnames@langs@wals@avv{Avá-Canoeiro}
\def\langnames@langs@wals@awa{Awadhi}
\def\langnames@langs@wals@awb{Awa (Papua New Guinea)}
\def\langnames@langs@wals@awc{Cicipu}
\def\langnames@langs@wals@awe{Awetí}
\def\langnames@langs@wals@awg{Anguthimri}
\def\langnames@langs@wals@awh{Awbono}
\def\langnames@langs@wals@awi{Aekyom}
\def\langnames@langs@wals@awk{Awabakal}
\def\langnames@langs@wals@awm{Arawum}
\def\langnames@langs@wals@awn{Awngi}
\def\langnames@langs@wals@awo{Awak}
\def\langnames@langs@wals@awr{Awera}
\def\langnames@langs@wals@aws{South Awyu}
\def\langnames@langs@wals@awt{Araweté}
\def\langnames@langs@wals@awu{Central Awyu}
\def\langnames@langs@wals@awv{Kia River Awyu}
\def\langnames@langs@wals@aww{Auwon}
\def\langnames@langs@wals@awx{Awara}
\def\langnames@langs@wals@awy{Edera Awyu}
\def\langnames@langs@wals@axb{Abipon}
\def\langnames@langs@wals@axg{Mato Grosso Arára}
\def\langnames@langs@wals@axk{Yaka (Central African Republic)}
\def\langnames@langs@wals@axl{Lower Southern Aranda}
\def\langnames@langs@wals@axx{Xaragure}
\def\langnames@langs@wals@aya{Awar}
\def\langnames@langs@wals@ayb{Ayizo Gbe}
\def\langnames@langs@wals@ayc{Southern Aymara}
\def\langnames@langs@wals@ayd{Yintyinka-Ayabadhu}
\def\langnames@langs@wals@aye{Ayere}
\def\langnames@langs@wals@ayg{Ginyanga}
\def\langnames@langs@wals@ayh{Hadrami Arabic}
\def\langnames@langs@wals@ayi{Leyigha}
\def\langnames@langs@wals@ayk{Akuku}
\def\langnames@langs@wals@ayl{Libyan Arabic}
\def\langnames@langs@wals@ayn{Sanaani Arabic}
\def\langnames@langs@wals@ayo{Ayoreo}
\def\langnames@langs@wals@ayp{North Mesopotamian Arabic}
\def\langnames@langs@wals@ayq{Ayi (Papua New Guinea)}
\def\langnames@langs@wals@ayr{Central Aymara}
\def\langnames@langs@wals@ays{Sorsogon Ayta}
\def\langnames@langs@wals@ayt{Bataan Ayta}
\def\langnames@langs@wals@ayu{Ayu}
\def\langnames@langs@wals@ayy{Tayabas Ayta near Lucena City in Western Quezon}
\def\langnames@langs@wals@ayz{Maybrat-Karon}
\def\langnames@langs@wals@aza{Azha}
\def\langnames@langs@wals@azb{South Azerbaijani}
\def\langnames@langs@wals@azd{Eastern Durango Nahuatl}
\def\langnames@langs@wals@azg{San Pedro Amuzgos Amuzgo}
\def\langnames@langs@wals@azj{North Azerbaijani}
\def\langnames@langs@wals@azm{Ipalapa Amuzgo}
\def\langnames@langs@wals@azn{Western Durango Nahuatl}
\def\langnames@langs@wals@azo{Awing}
\def\langnames@langs@wals@azt{Faire Atta}
\def\langnames@langs@wals@azz{Highland Puebla Nahuatl}
\def\langnames@langs@wals@baa{Babatana}
\def\langnames@langs@wals@bab{Bainounk-Gujaher}
\def\langnames@langs@wals@bac{Badui}
\def\langnames@langs@wals@bae{Baré}
\def\langnames@langs@wals@baf{Nubaca}
\def\langnames@langs@wals@bag{Tuki}
\def\langnames@langs@wals@bah{Bahamas Creole English}
\def\langnames@langs@wals@baj{Barakai}
\def\langnames@langs@wals@bak{Bashkir}
\def\langnames@langs@wals@bam{Bambara}
\def\langnames@langs@wals@ban{Balinese}
\def\langnames@langs@wals@bao{Waimaha}
\def\langnames@langs@wals@bap{Bantawa}
\def\langnames@langs@wals@bar{Bavarian}
\def\langnames@langs@wals@bas{Basa (Cameroon)}
\def\langnames@langs@wals@bau{Bada (Nigeria)}
\def\langnames@langs@wals@bav{Vengo}
\def\langnames@langs@wals@baw{Bambili-Bambui}
\def\langnames@langs@wals@bax{Bamun}
\def\langnames@langs@wals@bay{Batuley}
\def\langnames@langs@wals@bba{Baatonum}
\def\langnames@langs@wals@bbb{Barai}
\def\langnames@langs@wals@bbc{Batak Toba}
\def\langnames@langs@wals@bbd{Bau}
\def\langnames@langs@wals@bbe{Bangba}
\def\langnames@langs@wals@bbf{Baibai}
\def\langnames@langs@wals@bbg{Barama}
\def\langnames@langs@wals@bbh{Bugan}
\def\langnames@langs@wals@bbi{Barombi}
\def\langnames@langs@wals@bbj{Ghomálá'}
\def\langnames@langs@wals@bbk{Babanki}
\def\langnames@langs@wals@bbl{Bats}
\def\langnames@langs@wals@bbm{Babango}
\def\langnames@langs@wals@bbn{Uneapa}
\def\langnames@langs@wals@bbo{Northern Bobo Madaré}
\def\langnames@langs@wals@bbp{West Central Banda}
\def\langnames@langs@wals@bbq{Bamali}
\def\langnames@langs@wals@bbr{Girawa}
\def\langnames@langs@wals@bbs{Bakpinka}
\def\langnames@langs@wals@bbt{Mburku}
\def\langnames@langs@wals@bbu{Kulung (Nigeria)}
\def\langnames@langs@wals@bbv{Karnai}
\def\langnames@langs@wals@bbw{Baba}
\def\langnames@langs@wals@bby{Menchum}
\def\langnames@langs@wals@bca{Central Bai}
\def\langnames@langs@wals@bcc{Southern Balochi}
\def\langnames@langs@wals@bcd{North Babar}
\def\langnames@langs@wals@bce{Bamenyam}
\def\langnames@langs@wals@bcf{Bamu}
\def\langnames@langs@wals@bcg{Pukur}
\def\langnames@langs@wals@bch{Bariai}
\def\langnames@langs@wals@bci{Baoulé}
\def\langnames@langs@wals@bcj{Bardi}
\def\langnames@langs@wals@bck{Bunaba}
\def\langnames@langs@wals@bcl{Coastal-Naga Bikol}
\def\langnames@langs@wals@bcm{Bannoni}
\def\langnames@langs@wals@bcn{Bali (Nigeria)}
\def\langnames@langs@wals@bco{Kaluli}
\def\langnames@langs@wals@bcp{Bali (Democratic Republic of Congo)}
\def\langnames@langs@wals@bcq{Bench}
\def\langnames@langs@wals@bcr{Witsuwit'en-Babine}
\def\langnames@langs@wals@bcs{Kohumono}
\def\langnames@langs@wals@bct{Bendi}
\def\langnames@langs@wals@bcu{Awad Bing}
\def\langnames@langs@wals@bcv{Shoo-Minda-Nye}
\def\langnames@langs@wals@bcw{Bana}
\def\langnames@langs@wals@bcy{Bacama}
\def\langnames@langs@wals@bcz{Bainouk-Gunyaamolo-Gutobor}
\def\langnames@langs@wals@bda{Kugere-Kuxinge}
\def\langnames@langs@wals@bdb{Basap}
\def\langnames@langs@wals@bdc{Emberá-Baudó}
\def\langnames@langs@wals@bdd{Bunama}
\def\langnames@langs@wals@bde{Bade}
\def\langnames@langs@wals@bdf{Biage}
\def\langnames@langs@wals@bdg{Bonggi}
\def\langnames@langs@wals@bdh{Baka (South Sudan)}
\def\langnames@langs@wals@bdi{Northern Burun}
\def\langnames@langs@wals@bdj{Bai}
\def\langnames@langs@wals@bdk{Budukh}
\def\langnames@langs@wals@bdl{Indonesian Bajau}
\def\langnames@langs@wals@bdm{Buduma}
\def\langnames@langs@wals@bdn{Baldemu}
\def\langnames@langs@wals@bdo{Morom}
\def\langnames@langs@wals@bdp{Bende}
\def\langnames@langs@wals@bdq{Bahnar}
\def\langnames@langs@wals@bdr{West Coast Bajau}
\def\langnames@langs@wals@bds{Burunge}
\def\langnames@langs@wals@bdt{Bokoto}
\def\langnames@langs@wals@bdu{Oroko}
\def\langnames@langs@wals@bdv{Bodo Parja}
\def\langnames@langs@wals@bdw{Baham}
\def\langnames@langs@wals@bdx{Budong-Budong}
\def\langnames@langs@wals@bdy{Middle Clarence Bandjalang}
\def\langnames@langs@wals@bdz{Badeshi}
\def\langnames@langs@wals@bea{Beaver}
\def\langnames@langs@wals@beb{Bebele}
\def\langnames@langs@wals@bec{Iceve-Maci}
\def\langnames@langs@wals@bed{Bedoanas}
\def\langnames@langs@wals@bee{Byangsi}
\def\langnames@langs@wals@bef{Benabena}
\def\langnames@langs@wals@beg{Lemeting}
\def\langnames@langs@wals@beh{Biali}
\def\langnames@langs@wals@bei{Riuk Bekati'}
\def\langnames@langs@wals@bej{Beja}
\def\langnames@langs@wals@bek{Bebeli}
\def\langnames@langs@wals@bel{Belarusian}
\def\langnames@langs@wals@bem{Bemba (Zambia)}
\def\langnames@langs@wals@ben{Bengali}
\def\langnames@langs@wals@beo{Beami}
\def\langnames@langs@wals@bep{Besoa}
\def\langnames@langs@wals@beq{Beembe}
\def\langnames@langs@wals@bes{Besme}
\def\langnames@langs@wals@bet{Guiberoua Béte}
\def\langnames@langs@wals@beu{Blagar}
\def\langnames@langs@wals@bev{Daloa Bété}
\def\langnames@langs@wals@bew{Betawi}
\def\langnames@langs@wals@bex{Jur Modo}
\def\langnames@langs@wals@bey{Beli (Papua New Guinea)}
\def\langnames@langs@wals@bez{Bena (Tanzania)}
\def\langnames@langs@wals@bfa{Bari}
\def\langnames@langs@wals@bfb{Pauri Bareli}
\def\langnames@langs@wals@bfc{Northern Bai}
\def\langnames@langs@wals@bfd{Bafut}
\def\langnames@langs@wals@bfe{Betaf}
\def\langnames@langs@wals@bff{Bofi}
\def\langnames@langs@wals@bfg{Busang Kayan}
\def\langnames@langs@wals@bfh{Mblafe-Ránmo}
\def\langnames@langs@wals@bfi{British Sign Language}
\def\langnames@langs@wals@bfj{Bafanji}
\def\langnames@langs@wals@bfk{Ban Khor Sign Language}
\def\langnames@langs@wals@bfl{Banda-Ndélé}
\def\langnames@langs@wals@bfm{Mmen}
\def\langnames@langs@wals@bfn{Bunak}
\def\langnames@langs@wals@bfo{Malba Birifor}
\def\langnames@langs@wals@bfp{Beba}
\def\langnames@langs@wals@bfq{Badaga}
\def\langnames@langs@wals@bfr{Bazigar}
\def\langnames@langs@wals@bfs{Southern Bai}
\def\langnames@langs@wals@bft{Balti}
\def\langnames@langs@wals@bfu{Bunan}
\def\langnames@langs@wals@bfw{Bondo}
\def\langnames@langs@wals@bfx{Bantayanon}
\def\langnames@langs@wals@bfy{Bagheli}
\def\langnames@langs@wals@bfz{Mahasu Pahari}
\def\langnames@langs@wals@bga{Gwamhi-Wuri}
\def\langnames@langs@wals@bgb{Bobongko}
\def\langnames@langs@wals@bgc{Haryanvi}
\def\langnames@langs@wals@bgd{Rathwi Bareli}
\def\langnames@langs@wals@bge{Bauria}
\def\langnames@langs@wals@bgf{Ngombe-Bangandu}
\def\langnames@langs@wals@bgg{Bugun}
\def\langnames@langs@wals@bgi{Giangan}
\def\langnames@langs@wals@bgj{Bangolan}
\def\langnames@langs@wals@bgk{Bit}
\def\langnames@langs@wals@bgn{Western Balochi}
\def\langnames@langs@wals@bgo{Baga Koga}
\def\langnames@langs@wals@bgp{Eastern Balochi}
\def\langnames@langs@wals@bgq{Bagri}
\def\langnames@langs@wals@bgr{Bawm Chin}
\def\langnames@langs@wals@bgs{Tagabawa}
\def\langnames@langs@wals@bgt{Bughotu}
\def\langnames@langs@wals@bgu{Mbongno}
\def\langnames@langs@wals@bgv{Warkay-Bipim}
\def\langnames@langs@wals@bgw{Bhatri}
\def\langnames@langs@wals@bgx{Rumelian Turkish}
\def\langnames@langs@wals@bgy{Benggoi}
\def\langnames@langs@wals@bgz{Banggai}
\def\langnames@langs@wals@bha{Bharia}
\def\langnames@langs@wals@bhb{Bhili}
\def\langnames@langs@wals@bhc{Biga}
\def\langnames@langs@wals@bhd{Bhadrawahi}
\def\langnames@langs@wals@bhe{Bhaya}
\def\langnames@langs@wals@bhf{Odiai}
\def\langnames@langs@wals@bhg{Binandere}
\def\langnames@langs@wals@bhh{Bukharic}
\def\langnames@langs@wals@bhi{Bhilali}
\def\langnames@langs@wals@bhj{Bahing}
\def\langnames@langs@wals@bhk{Inland-Buhi-Daraga Bikol}
\def\langnames@langs@wals@bhl{Bimin}
\def\langnames@langs@wals@bhm{Bathari}
\def\langnames@langs@wals@bhn{Gardabani Bohtan Neo-Aramaic}
\def\langnames@langs@wals@bho{Bhojpuri}
\def\langnames@langs@wals@bhp{Bima}
\def\langnames@langs@wals@bhq{Tukang Besi South}
\def\langnames@langs@wals@bhr{Bara Malagasy}
\def\langnames@langs@wals@bhs{Buwal}
\def\langnames@langs@wals@bht{Bhattiyali}
\def\langnames@langs@wals@bhu{Bhunjia}
\def\langnames@langs@wals@bhv{Bahau}
\def\langnames@langs@wals@bhw{Biak}
\def\langnames@langs@wals@bhy{Bhele}
\def\langnames@langs@wals@bhz{Bada (Indonesia)}
\def\langnames@langs@wals@bia{Badimaya}
\def\langnames@langs@wals@bib{Bissa}
\def\langnames@langs@wals@bid{Bidiyo}
\def\langnames@langs@wals@bie{Bepour}
\def\langnames@langs@wals@bif{Biafada}
\def\langnames@langs@wals@big{Biangai}
\def\langnames@langs@wals@bil{Bile}
\def\langnames@langs@wals@bim{Bimoba}
\def\langnames@langs@wals@bin{Bini}
\def\langnames@langs@wals@bio{Nai}
\def\langnames@langs@wals@bip{Bila}
\def\langnames@langs@wals@biq{Bipi}
\def\langnames@langs@wals@bir{Bisorio}
\def\langnames@langs@wals@bis{Bislama}
\def\langnames@langs@wals@bit{Berinomo}
\def\langnames@langs@wals@biu{Biete}
\def\langnames@langs@wals@biv{Southern Birifor}
\def\langnames@langs@wals@biw{Kol (Cameroon)}
\def\langnames@langs@wals@bix{Bijori}
\def\langnames@langs@wals@biy{Birhor}
\def\langnames@langs@wals@biz{Loi-Likila}
\def\langnames@langs@wals@bja{Budza}
\def\langnames@langs@wals@bjb{Banggarla}
\def\langnames@langs@wals@bjc{Bariji}
\def\langnames@langs@wals@bje{Biao-Jiao Mien}
\def\langnames@langs@wals@bjf{Barzani Jewish Neo-Aramaic}
\def\langnames@langs@wals@bjg{Kanyaki-Kagbaaga-Kajoko Bidyogo}
\def\langnames@langs@wals@bjh{Bahinemo}
\def\langnames@langs@wals@bji{Burji}
\def\langnames@langs@wals@bjj{Kanauji}
\def\langnames@langs@wals@bjk{Barok}
\def\langnames@langs@wals@bjl{Bulu (Papua New Guinea)}
\def\langnames@langs@wals@bjm{Bajelani}
\def\langnames@langs@wals@bjn{Banjar}
\def\langnames@langs@wals@bjo{Mid-Southern Banda}
\def\langnames@langs@wals@bjr{Binumarien}
\def\langnames@langs@wals@bjs{Bajan}
\def\langnames@langs@wals@bjt{Balanta-Ganja}
\def\langnames@langs@wals@bju{Busuu}
\def\langnames@langs@wals@bjv{Nangnda}
\def\langnames@langs@wals@bjw{Bakwé}
\def\langnames@langs@wals@bjx{Banao Itneg}
\def\langnames@langs@wals@bjy{Bayali}
\def\langnames@langs@wals@bjz{Baruga}
\def\langnames@langs@wals@bka{Kyak}
\def\langnames@langs@wals@bkb{Eastern-Southern Bontok}
\def\langnames@langs@wals@bkc{Baka (Cameroon)}
\def\langnames@langs@wals@bkd{Talaandig-Binukid}
\def\langnames@langs@wals@bkf{Beeke}
\def\langnames@langs@wals@bkh{Bakoko}
\def\langnames@langs@wals@bki{Baki}
\def\langnames@langs@wals@bkj{Pande}
\def\langnames@langs@wals@bkk{Brokskat}
\def\langnames@langs@wals@bkl{Berik}
\def\langnames@langs@wals@bkm{Kom (Cameroon)}
\def\langnames@langs@wals@bkn{Bukitan}
\def\langnames@langs@wals@bko{Kwa'}
\def\langnames@langs@wals@bkp{Boko (Democratic Republic of Congo)}
\def\langnames@langs@wals@bkq{Bakairí}
\def\langnames@langs@wals@bkr{Bakumpai}
\def\langnames@langs@wals@bks{Masbate Sorsogon}
\def\langnames@langs@wals@bkt{Boloki}
\def\langnames@langs@wals@bku{Buhid}
\def\langnames@langs@wals@bkv{Bekwarra}
\def\langnames@langs@wals@bkw{Bekwil}
\def\langnames@langs@wals@bkx{Baikeno}
\def\langnames@langs@wals@bky{Bokyi}
\def\langnames@langs@wals@bkz{Bungku}
\def\langnames@langs@wals@bla{Siksika}
\def\langnames@langs@wals@blb{Bilua}
\def\langnames@langs@wals@blc{Bella Coola}
\def\langnames@langs@wals@bld{Bolango}
\def\langnames@langs@wals@ble{Balanta-Kentohe}
\def\langnames@langs@wals@blf{Buol}
\def\langnames@langs@wals@blh{Kuwaa}
\def\langnames@langs@wals@bli{Bolia}
\def\langnames@langs@wals@blj{Bolongan}
\def\langnames@langs@wals@blk{Pa'o Karen}
\def\langnames@langs@wals@bll{Biloxi}
\def\langnames@langs@wals@blm{Beli (South Sudan)}
\def\langnames@langs@wals@bln{Coastal-Virac Bikol}
\def\langnames@langs@wals@blo{Anii}
\def\langnames@langs@wals@blp{Blablanga}
\def\langnames@langs@wals@blq{Paluai}
\def\langnames@langs@wals@blr{Blang}
\def\langnames@langs@wals@bls{Balaesang}
\def\langnames@langs@wals@blt{Tai Dam}
\def\langnames@langs@wals@blv{Kibala}
\def\langnames@langs@wals@blw{Balangao}
\def\langnames@langs@wals@blx{Mag-Indi Ayta}
\def\langnames@langs@wals@bly{Notre}
\def\langnames@langs@wals@blz{Balantak}
\def\langnames@langs@wals@bma{Lame}
\def\langnames@langs@wals@bmb{Bembe}
\def\langnames@langs@wals@bmc{Biem}
\def\langnames@langs@wals@bmd{Baga Manduri}
\def\langnames@langs@wals@bme{Limassa}
\def\langnames@langs@wals@bmf{Bom-Kim}
\def\langnames@langs@wals@bmg{Bamwe}
\def\langnames@langs@wals@bmh{Kein}
\def\langnames@langs@wals@bmi{Bagirmi}
\def\langnames@langs@wals@bmj{Bote}
\def\langnames@langs@wals@bmk{Ghayavi}
\def\langnames@langs@wals@bml{Bomboli-Bozaba}
\def\langnames@langs@wals@bmm{Northern Betsimisaraka Malagasy}
\def\langnames@langs@wals@bmn{Bina (Papua New Guinea)}
\def\langnames@langs@wals@bmo{Bambalang}
\def\langnames@langs@wals@bmp{Bulgebi}
\def\langnames@langs@wals@bmq{Bomu}
\def\langnames@langs@wals@bmr{Muinane}
\def\langnames@langs@wals@bms{Bilma Kanuri}
\def\langnames@langs@wals@bmt{Biao Mon}
\def\langnames@langs@wals@bmu{Burum-Mindik}
\def\langnames@langs@wals@bmv{Bum}
\def\langnames@langs@wals@bmw{Bomwali}
\def\langnames@langs@wals@bmx{Baimak}
\def\langnames@langs@wals@bmz{Baramu}
\def\langnames@langs@wals@bna{Bonerate}
\def\langnames@langs@wals@bnb{Bookan}
\def\langnames@langs@wals@bnd{Banda (Indonesia)}
\def\langnames@langs@wals@bne{Bintauna}
\def\langnames@langs@wals@bnf{Masiwang}
\def\langnames@langs@wals@bng{Benga}
\def\langnames@langs@wals@bni{Bobangi}
\def\langnames@langs@wals@bnj{Eastern Tawbuid}
\def\langnames@langs@wals@bnk{Bierebo}
\def\langnames@langs@wals@bnl{Boon}
\def\langnames@langs@wals@bnm{Batanga}
\def\langnames@langs@wals@bnn{Bunun}
\def\langnames@langs@wals@bno{Bantoanon}
\def\langnames@langs@wals@bnp{Bola}
\def\langnames@langs@wals@bnq{Bantik}
\def\langnames@langs@wals@bnr{Farafi}
\def\langnames@langs@wals@bns{Bundeli}
\def\langnames@langs@wals@bnu{Bentong}
\def\langnames@langs@wals@bnv{Bonerif}
\def\langnames@langs@wals@bnw{Bisis}
\def\langnames@langs@wals@bnx{Bangubangu}
\def\langnames@langs@wals@bny{Bintulu}
\def\langnames@langs@wals@bnz{Beezen}
\def\langnames@langs@wals@boa{Bora}
\def\langnames@langs@wals@bob{Aweer}
\def\langnames@langs@wals@bod{Tibetan}
\def\langnames@langs@wals@boe{Mundabli-Mufu}
\def\langnames@langs@wals@bof{Bolon}
\def\langnames@langs@wals@bog{Langue de Signes Malienne}
\def\langnames@langs@wals@boh{Boma Yumu}
\def\langnames@langs@wals@boi{Barbareño}
\def\langnames@langs@wals@boj{Anjam}
\def\langnames@langs@wals@bok{Impfondo}
\def\langnames@langs@wals@bol{Bole}
\def\langnames@langs@wals@bom{Berom}
\def\langnames@langs@wals@bon{Bine}
\def\langnames@langs@wals@boo{Tiemacèwè Bozo}
\def\langnames@langs@wals@bop{Bonkiman}
\def\langnames@langs@wals@boq{Bogaya}
\def\langnames@langs@wals@bor{Bororo}
\def\langnames@langs@wals@bos{Bosnian}
\def\langnames@langs@wals@bot{Bongo}
\def\langnames@langs@wals@bou{Bondei}
\def\langnames@langs@wals@bov{Tuwuli}
\def\langnames@langs@wals@bow{Rema}
\def\langnames@langs@wals@box{Buamu}
\def\langnames@langs@wals@boy{Bodo (Central African Republic)}
\def\langnames@langs@wals@boz{Tiéyaxo Bozo}
\def\langnames@langs@wals@bpa{Dakaka}
\def\langnames@langs@wals@bpb{Barbacoas}
\def\langnames@langs@wals@bpc{Mbuk}
\def\langnames@langs@wals@bpd{Banda-Banda}
\def\langnames@langs@wals@bpe{Bauni}
\def\langnames@langs@wals@bpg{Bonggo}
\def\langnames@langs@wals@bph{Botlikh}
\def\langnames@langs@wals@bpi{Bagupi}
\def\langnames@langs@wals@bpj{Binji}
\def\langnames@langs@wals@bpk{Orowe}
\def\langnames@langs@wals@bpl{Broome Pearling Lugger Pidgin}
\def\langnames@langs@wals@bpm{Biyom}
\def\langnames@langs@wals@bpn{Dzao Min}
\def\langnames@langs@wals@bpp{Kaure-Narau}
\def\langnames@langs@wals@bpq{Banda Malay}
\def\langnames@langs@wals@bpr{Koronadal Blaan}
\def\langnames@langs@wals@bps{Sarangani Blaan}
\def\langnames@langs@wals@bpt{Barrow Point}
\def\langnames@langs@wals@bpu{Bongu}
\def\langnames@langs@wals@bpv{Bian Marind}
\def\langnames@langs@wals@bpw{Bo (Papua New Guinea)}
\def\langnames@langs@wals@bpx{Palya Bareli}
\def\langnames@langs@wals@bpy{Bishnupriya Manipuri}
\def\langnames@langs@wals@bpz{Bilba}
\def\langnames@langs@wals@bqa{Tchumbuli}
\def\langnames@langs@wals@bqb{Bagusa}
\def\langnames@langs@wals@bqc{Boko (Benin)}
\def\langnames@langs@wals@bqd{Bung}
\def\langnames@langs@wals@bqg{Bago-Kusuntu}
\def\langnames@langs@wals@bqh{Baima}
\def\langnames@langs@wals@bqi{Bakhtiari}
\def\langnames@langs@wals@bqj{Bandial}
\def\langnames@langs@wals@bqk{Banda-Mbrès}
\def\langnames@langs@wals@bql{Karen}
\def\langnames@langs@wals@bqm{Wumboko-Bubia}
\def\langnames@langs@wals@bqn{Bulgarian Sign Language}
\def\langnames@langs@wals@bqo{Balo}
\def\langnames@langs@wals@bqp{Busa}
\def\langnames@langs@wals@bqq{Biritai}
\def\langnames@langs@wals@bqr{Burusu}
\def\langnames@langs@wals@bqs{Bosngun}
\def\langnames@langs@wals@bqt{Bamukumbit}
\def\langnames@langs@wals@bqu{Boguru}
\def\langnames@langs@wals@bqv{Begbere-Ejar}
\def\langnames@langs@wals@bqw{Buru-Angwe}
\def\langnames@langs@wals@bqx{Baangi}
\def\langnames@langs@wals@bqy{Kata Kolok}
\def\langnames@langs@wals@bqz{Bakaka}
\def\langnames@langs@wals@bra{Braj}
\def\langnames@langs@wals@brb{Brao}
\def\langnames@langs@wals@brc{Berbice Creole Dutch}
\def\langnames@langs@wals@brd{Baraamu}
\def\langnames@langs@wals@bre{Breton}
\def\langnames@langs@wals@brf{Bera}
\def\langnames@langs@wals@brg{Baure}
\def\langnames@langs@wals@brh{Brahui}
\def\langnames@langs@wals@bri{Mokpwe}
\def\langnames@langs@wals@brj{Bieria}
\def\langnames@langs@wals@brk{Birked}
\def\langnames@langs@wals@brl{Birwa}
\def\langnames@langs@wals@brm{Barambu}
\def\langnames@langs@wals@brn{Boruca}
\def\langnames@langs@wals@bro{Dur Brokkat}
\def\langnames@langs@wals@brp{Barapasi}
\def\langnames@langs@wals@brq{Breri}
\def\langnames@langs@wals@brr{Birao}
\def\langnames@langs@wals@brs{Baras}
\def\langnames@langs@wals@brt{Bitare}
\def\langnames@langs@wals@bru{Eastern Bru}
\def\langnames@langs@wals@brv{Western Bru}
\def\langnames@langs@wals@brw{Bellari}
\def\langnames@langs@wals@brx{Bodo-Mech}
\def\langnames@langs@wals@bry{Burui}
\def\langnames@langs@wals@brz{Bilibil}
\def\langnames@langs@wals@bsa{Abinomn}
\def\langnames@langs@wals@bsb{Brunei Bisaya-Dusun}
\def\langnames@langs@wals@bsc{Bassari-Tanda}
\def\langnames@langs@wals@bse{Wushi}
\def\langnames@langs@wals@bsf{Bauchi}
\def\langnames@langs@wals@bsg{Bashkardi}
\def\langnames@langs@wals@bsh{Katë}
\def\langnames@langs@wals@bsi{Bassossi}
\def\langnames@langs@wals@bsj{Bangwinji}
\def\langnames@langs@wals@bsk{Burushaski}
\def\langnames@langs@wals@bsl{Basa-Gumna}
\def\langnames@langs@wals@bsm{Busami}
\def\langnames@langs@wals@bsn{Barasana-Eduria}
\def\langnames@langs@wals@bsp{Baga Sitemu}
\def\langnames@langs@wals@bsq{Bassa}
\def\langnames@langs@wals@bsr{Bassa-Kontagora}
\def\langnames@langs@wals@bss{Akoose}
\def\langnames@langs@wals@bst{Basketo}
\def\langnames@langs@wals@bsu{Bahonsuai}
\def\langnames@langs@wals@bsw{Baiso}
\def\langnames@langs@wals@bsx{Yangkam}
\def\langnames@langs@wals@bsy{Sabah Bisaya}
\def\langnames@langs@wals@bta{Bata}
\def\langnames@langs@wals@btc{Bati (Cameroon)}
\def\langnames@langs@wals@btd{Batak Dairi}
\def\langnames@langs@wals@bte{Gamo-Ningi}
\def\langnames@langs@wals@btf{Birgit}
\def\langnames@langs@wals@btg{Gagnoa Bété}
\def\langnames@langs@wals@bth{Biatah Bidayuh}
\def\langnames@langs@wals@bti{Burate}
\def\langnames@langs@wals@btj{Bacanese Malay}
\def\langnames@langs@wals@btm{Batak Mandailing}
\def\langnames@langs@wals@btn{Ratagnon}
\def\langnames@langs@wals@bto{Iriga Bicolano}
\def\langnames@langs@wals@btp{Budibud}
\def\langnames@langs@wals@btq{Batek}
\def\langnames@langs@wals@btr{Baetora}
\def\langnames@langs@wals@bts{Batak Simalungun}
\def\langnames@langs@wals@btt{Bete-Bendi}
\def\langnames@langs@wals@btu{Batu}
\def\langnames@langs@wals@btv{Bateri}
\def\langnames@langs@wals@btw{Butuanon}
\def\langnames@langs@wals@btx{Batak Karo}
\def\langnames@langs@wals@bty{Bobot}
\def\langnames@langs@wals@btz{Batak Alas-Kluet}
\def\langnames@langs@wals@bub{Bua}
\def\langnames@langs@wals@buc{Kibosy Kiantalaotsy-Majunga}
\def\langnames@langs@wals@bud{Ntcham}
\def\langnames@langs@wals@bue{Beothuk}
\def\langnames@langs@wals@buf{Bushoong}
\def\langnames@langs@wals@bug{Buginese}
\def\langnames@langs@wals@buh{Younuo Bunu}
\def\langnames@langs@wals@bui{Bongili}
\def\langnames@langs@wals@buj{Basa-Gurmana}
\def\langnames@langs@wals@buk{Bugawac}
\def\langnames@langs@wals@bul{Bulgarian}
\def\langnames@langs@wals@bum{Bulu (Cameroon)}
\def\langnames@langs@wals@bun{Sherbro}
\def\langnames@langs@wals@buo{Terei}
\def\langnames@langs@wals@bup{Busoa}
\def\langnames@langs@wals@buq{Barem}
\def\langnames@langs@wals@bus{Bokobaru}
\def\langnames@langs@wals@but{Bungain}
\def\langnames@langs@wals@buu{Budu}
\def\langnames@langs@wals@buv{Bun}
\def\langnames@langs@wals@buw{Bubi}
\def\langnames@langs@wals@bux{Boghom}
\def\langnames@langs@wals@buy{Bullom So}
\def\langnames@langs@wals@buz{Bukwen}
\def\langnames@langs@wals@bva{Barain}
\def\langnames@langs@wals@bvb{Bube}
\def\langnames@langs@wals@bvc{Baelelea}
\def\langnames@langs@wals@bvd{Baeggu}
\def\langnames@langs@wals@bve{Berau Malay}
\def\langnames@langs@wals@bvf{Boor}
\def\langnames@langs@wals@bvg{Bonkeng}
\def\langnames@langs@wals@bvh{Bure}
\def\langnames@langs@wals@bvi{Belanda Viri}
\def\langnames@langs@wals@bvj{Baan}
\def\langnames@langs@wals@bvk{Bukat}
\def\langnames@langs@wals@bvl{Bolivian Sign Language}
\def\langnames@langs@wals@bvm{Bamunka}
\def\langnames@langs@wals@bvn{Buna}
\def\langnames@langs@wals@bvo{Bolgo}
\def\langnames@langs@wals@bvq{Birri}
\def\langnames@langs@wals@bvr{Burarra}
\def\langnames@langs@wals@bvt{Bati (Indonesia)}
\def\langnames@langs@wals@bvu{Bukit Malay}
\def\langnames@langs@wals@bvv{Baniva}
\def\langnames@langs@wals@bvw{Boga}
\def\langnames@langs@wals@bvx{Dibole}
\def\langnames@langs@wals@bvy{Baybayanon}
\def\langnames@langs@wals@bvz{Bauzi}
\def\langnames@langs@wals@bwa{Bwatoo}
\def\langnames@langs@wals@bwb{Namosi-Naitasiri-Serua}
\def\langnames@langs@wals@bwc{Bwile}
\def\langnames@langs@wals@bwd{Bwaidoka}
\def\langnames@langs@wals@bwe{Bwe Karen}
\def\langnames@langs@wals@bwf{Boselewa}
\def\langnames@langs@wals@bwg{Barwe}
\def\langnames@langs@wals@bwh{Bishuo}
\def\langnames@langs@wals@bwi{Baniwa do Icana}
\def\langnames@langs@wals@bwj{Láá Láá Bwamu}
\def\langnames@langs@wals@bwk{Bauwaki}
\def\langnames@langs@wals@bwl{Bwela}
\def\langnames@langs@wals@bwm{Biwat}
\def\langnames@langs@wals@bwn{Wunai Bunu}
\def\langnames@langs@wals@bwo{Boro (Ethiopia)}
\def\langnames@langs@wals@bwp{Mandobo Bawah}
\def\langnames@langs@wals@bwq{Southern Bobo Madaré}
\def\langnames@langs@wals@bwr{Bura-Pabir}
\def\langnames@langs@wals@bws{Bomboma}
\def\langnames@langs@wals@bwt{Bafaw-Balong}
\def\langnames@langs@wals@bwu{Buli (Ghana)}
\def\langnames@langs@wals@bww{Bwa}
\def\langnames@langs@wals@bwx{Bu-Nao Bunu}
\def\langnames@langs@wals@bwy{Cwi Bwamu}
\def\langnames@langs@wals@bwz{Bwisi}
\def\langnames@langs@wals@bxa{Bauro}
\def\langnames@langs@wals@bxb{Belanda Bor}
\def\langnames@langs@wals@bxc{Molengue}
\def\langnames@langs@wals@bxd{Pela}
\def\langnames@langs@wals@bxe{Ongota}
\def\langnames@langs@wals@bxf{Bilur}
\def\langnames@langs@wals@bxg{Bangala}
\def\langnames@langs@wals@bxh{Buhutu}
\def\langnames@langs@wals@bxi{Pirlatapa}
\def\langnames@langs@wals@bxj{Bayungu}
\def\langnames@langs@wals@bxk{Bukusu}
\def\langnames@langs@wals@bxl{Jalkunan}
\def\langnames@langs@wals@bxm{Mongolia Buriat}
\def\langnames@langs@wals@bxn{Burduna}
\def\langnames@langs@wals@bxo{Barikanchi}
\def\langnames@langs@wals@bxp{Bebil}
\def\langnames@langs@wals@bxq{Beele}
\def\langnames@langs@wals@bxr{Russia Buriat}
\def\langnames@langs@wals@bxs{Busam}
\def\langnames@langs@wals@bxu{China Buriat}
\def\langnames@langs@wals@bxv{Berakou}
\def\langnames@langs@wals@bxw{Bankagooma}
\def\langnames@langs@wals@bxz{Binahari-Neme}
\def\langnames@langs@wals@bya{Batak}
\def\langnames@langs@wals@byb{Bikya}
\def\langnames@langs@wals@byc{Ubaghara}
\def\langnames@langs@wals@byd{Benyadu'}
\def\langnames@langs@wals@bye{Pouye}
\def\langnames@langs@wals@byf{Bete (Yukubenic)}
\def\langnames@langs@wals@byg{Baygo}
\def\langnames@langs@wals@byh{Bujhyal}
\def\langnames@langs@wals@byi{Buyu}
\def\langnames@langs@wals@byj{Bina (Nigeria)}
\def\langnames@langs@wals@byk{Shidong Biao}
\def\langnames@langs@wals@byl{Bayono}
\def\langnames@langs@wals@bym{Bidyara}
\def\langnames@langs@wals@byn{Bilin}
\def\langnames@langs@wals@byo{Biyo}
\def\langnames@langs@wals@byp{Bumaji}
\def\langnames@langs@wals@byq{Basay}
\def\langnames@langs@wals@byr{Baruya}
\def\langnames@langs@wals@bys{Burak}
\def\langnames@langs@wals@byt{Berti}
\def\langnames@langs@wals@byv{Medumba}
\def\langnames@langs@wals@byw{Belhariya}
\def\langnames@langs@wals@byx{Qaqet}
\def\langnames@langs@wals@byz{Banaro}
\def\langnames@langs@wals@bza{Bandi}
\def\langnames@langs@wals@bzb{Andio}
\def\langnames@langs@wals@bzc{Southern Betsimisaraka Malagasy}
\def\langnames@langs@wals@bzd{Bribri}
\def\langnames@langs@wals@bze{Jenaama Bozo}
\def\langnames@langs@wals@bzf{Boikin}
\def\langnames@langs@wals@bzg{Babuza}
\def\langnames@langs@wals@bzh{Mapos Buang}
\def\langnames@langs@wals@bzi{Bisu}
\def\langnames@langs@wals@bzj{Belize Kriol English}
\def\langnames@langs@wals@bzk{Nicaragua Creole English}
\def\langnames@langs@wals@bzl{Boano (Sulawesi)}
\def\langnames@langs@wals@bzm{Bolondo}
\def\langnames@langs@wals@bzn{Boano (Maluku)}
\def\langnames@langs@wals@bzp{Kemberano}
\def\langnames@langs@wals@bzq{Buli (Indonesia)}
\def\langnames@langs@wals@bzr{Biri}
\def\langnames@langs@wals@bzs{Brazilian Sign Language}
\def\langnames@langs@wals@bzt{Brithenig}
\def\langnames@langs@wals@bzu{Burmeso}
\def\langnames@langs@wals@bzv{Bebe}
\def\langnames@langs@wals@bzw{Basa (Nigeria)}
\def\langnames@langs@wals@bzx{Hainyaxo Bozo}
\def\langnames@langs@wals@bzy{Obanliku}
\def\langnames@langs@wals@bzz{Evant}
\def\langnames@langs@wals@caa{Chortí}
\def\langnames@langs@wals@cab{Garifuna}
\def\langnames@langs@wals@cac{Chuj}
\def\langnames@langs@wals@cad{Caddo}
\def\langnames@langs@wals@cae{Lehar}
\def\langnames@langs@wals@caf{Southern Carrier}
\def\langnames@langs@wals@cag{Nivaclé}
\def\langnames@langs@wals@cah{Cahuarano}
\def\langnames@langs@wals@cak{Kaqchikel}
\def\langnames@langs@wals@cal{Carolinian}
\def\langnames@langs@wals@cam{Cemuhî}
\def\langnames@langs@wals@can{Chambri}
\def\langnames@langs@wals@cao{Chácobo}
\def\langnames@langs@wals@cap{Chipaya}
\def\langnames@langs@wals@caq{Car Nicobarese}
\def\langnames@langs@wals@car{Galibi Carib}
\def\langnames@langs@wals@cas{Mosetén-Chimané}
\def\langnames@langs@wals@cat{Catalan}
\def\langnames@langs@wals@cav{Cavineña}
\def\langnames@langs@wals@caw{Callawalla}
\def\langnames@langs@wals@cax{Lomeriano-Ignaciano Chiquitano}
\def\langnames@langs@wals@cay{Cayuga}
\def\langnames@langs@wals@caz{Canichana}
\def\langnames@langs@wals@cbb{Cabiyarí}
\def\langnames@langs@wals@cbc{Carapana}
\def\langnames@langs@wals@cbd{Carijona}
\def\langnames@langs@wals@cbg{Chimila}
\def\langnames@langs@wals@cbi{Cha'palaa}
\def\langnames@langs@wals@cbj{Ede Cabe}
\def\langnames@langs@wals@cbk{Chavacano}
\def\langnames@langs@wals@cbl{Bualkhaw Chin}
\def\langnames@langs@wals@cbn{Nyahkur}
\def\langnames@langs@wals@cbo{Izora}
\def\langnames@langs@wals@cbq{Cuba}
\def\langnames@langs@wals@cbr{Cashibo-Cacataibo}
\def\langnames@langs@wals@cbs{Cashinahua}
\def\langnames@langs@wals@cbt{Shawi}
\def\langnames@langs@wals@cbu{Candoshi-Shapra}
\def\langnames@langs@wals@cbv{Kakua}
\def\langnames@langs@wals@cbw{Kinabalian}
\def\langnames@langs@wals@cby{Carabayo}
\def\langnames@langs@wals@ccc{Chamicuro}
\def\langnames@langs@wals@ccd{Cafundo}
\def\langnames@langs@wals@cce{Chopi}
\def\langnames@langs@wals@ccg{Samba Daka}
\def\langnames@langs@wals@cch{Atsam}
\def\langnames@langs@wals@ccj{Kasanga}
\def\langnames@langs@wals@ccl{Cutchi-Swahili}
\def\langnames@langs@wals@ccm{Malaccan Creole Malay}
\def\langnames@langs@wals@cco{Comaltepec Chinantec}
\def\langnames@langs@wals@ccp{Chakma}
\def\langnames@langs@wals@ccr{Cacaopera}
\def\langnames@langs@wals@cda{Choni}
\def\langnames@langs@wals@cde{Chenchu}
\def\langnames@langs@wals@cdf{Chiru}
\def\langnames@langs@wals@cdh{Chambeali}
\def\langnames@langs@wals@cdi{Chodri}
\def\langnames@langs@wals@cdj{Churahi}
\def\langnames@langs@wals@cdm{Chepang}
\def\langnames@langs@wals@cdn{Chaudangsi}
\def\langnames@langs@wals@cdo{Min Dong Chinese}
\def\langnames@langs@wals@cdr{Yara}
\def\langnames@langs@wals@cds{Chadian Sign Language}
\def\langnames@langs@wals@cdy{Chadong}
\def\langnames@langs@wals@cdz{Koda}
\def\langnames@langs@wals@cea{Lower Chehalis}
\def\langnames@langs@wals@ceb{Cebuano}
\def\langnames@langs@wals@ceg{Chamacoco}
\def\langnames@langs@wals@cek{Eastern Khumi Chin}
\def\langnames@langs@wals@cen{Cen}
\def\langnames@langs@wals@ces{Czech}
\def\langnames@langs@wals@cet{Jalaa}
\def\langnames@langs@wals@cfa{Dijim-Bwilim}
\def\langnames@langs@wals@cfd{Cara}
\def\langnames@langs@wals@cfg{Como Karim}
\def\langnames@langs@wals@cfm{Falam Chin}
\def\langnames@langs@wals@cga{Changriwa}
\def\langnames@langs@wals@cgc{Kagayanen}
\def\langnames@langs@wals@cgg{Chiga}
\def\langnames@langs@wals@cgk{Chocangacakha}
\def\langnames@langs@wals@cha{Chamorro}
\def\langnames@langs@wals@chb{Chibcha}
\def\langnames@langs@wals@chc{Catawba}
\def\langnames@langs@wals@chd{Highland Oaxaca Chontal}
\def\langnames@langs@wals@che{Chechen}
\def\langnames@langs@wals@chf{Tabasco Chontal}
\def\langnames@langs@wals@chg{Chagatai}
\def\langnames@langs@wals@chh{Clatsop-Shoalwater Chinook}
\def\langnames@langs@wals@chj{Ojitlán Chinantec}
\def\langnames@langs@wals@chk{Chuukese}
\def\langnames@langs@wals@chl{Cahuilla}
\def\langnames@langs@wals@chn{Creolized Grand Ronde Chinook Jargon}
\def\langnames@langs@wals@cho{Choctaw}
\def\langnames@langs@wals@chp{Chipewyan}
\def\langnames@langs@wals@chq{Quiotepec Chinantec}
\def\langnames@langs@wals@chr{Cherokee}
\def\langnames@langs@wals@cht{Cholón}
\def\langnames@langs@wals@chu{Church Slavic}
\def\langnames@langs@wals@chv{Chuvash}
\def\langnames@langs@wals@chw{Chuwabu}
\def\langnames@langs@wals@chx{Chantyal}
\def\langnames@langs@wals@chy{Cheyenne}
\def\langnames@langs@wals@chz{Ozumacín Chinantec}
\def\langnames@langs@wals@cia{Cia-Cia}
\def\langnames@langs@wals@cib{Ci Gbe}
\def\langnames@langs@wals@cic{Chickasaw}
\def\langnames@langs@wals@cid{Chimariko}
\def\langnames@langs@wals@cie{Cineni}
\def\langnames@langs@wals@cih{Chinali}
\def\langnames@langs@wals@cik{Chhitkul-Rakchham}
\def\langnames@langs@wals@cim{Cimbrian}
\def\langnames@langs@wals@cin{Cinta Larga}
\def\langnames@langs@wals@cip{Chiapanec}
\def\langnames@langs@wals@cir{Tiri-Mea}
\def\langnames@langs@wals@ciw{Chippewa}
\def\langnames@langs@wals@ciy{Chaima}
\def\langnames@langs@wals@cja{Western Cham}
\def\langnames@langs@wals@cje{Chru}
\def\langnames@langs@wals@cjh{Upper Chehalis}
\def\langnames@langs@wals@cji{Chamalal}
\def\langnames@langs@wals@cjk{Chokwe}
\def\langnames@langs@wals@cjm{Eastern Cham}
\def\langnames@langs@wals@cjn{Chenapian}
\def\langnames@langs@wals@cjo{Ashéninka Pajonal}
\def\langnames@langs@wals@cjp{Cabécar}
\def\langnames@langs@wals@cjs{Shor}
\def\langnames@langs@wals@cjv{Chuave}
\def\langnames@langs@wals@cjy{Jinyu Chinese}
\def\langnames@langs@wals@ckb{Central Kurdish}
\def\langnames@langs@wals@ckh{Chak}
\def\langnames@langs@wals@ckl{Cibak}
\def\langnames@langs@wals@ckn{Kaang Chin}
\def\langnames@langs@wals@cko{Anufo}
\def\langnames@langs@wals@ckq{Kajakse}
\def\langnames@langs@wals@ckr{Kairak}
\def\langnames@langs@wals@cks{Tayo}
\def\langnames@langs@wals@ckt{Chukchi}
\def\langnames@langs@wals@cku{Koasati}
\def\langnames@langs@wals@ckv{Kavalan}
\def\langnames@langs@wals@ckx{Caka}
\def\langnames@langs@wals@cky{Cakfem-Mushere-Jibyal}
\def\langnames@langs@wals@ckz{Cakchiquel-Quiché Mixed Language}
\def\langnames@langs@wals@cla{Ron}
\def\langnames@langs@wals@clc{Chilcotin-Nicola}
\def\langnames@langs@wals@cld{Chaldean Neo-Aramaic}
\def\langnames@langs@wals@cle{Lealao Chinantec}
\def\langnames@langs@wals@clh{Chilisso}
\def\langnames@langs@wals@cli{Chakali}
\def\langnames@langs@wals@clk{Idu}
\def\langnames@langs@wals@cll{Chala}
\def\langnames@langs@wals@clm{Clallam}
\def\langnames@langs@wals@clo{Lowland Oaxaca Chontal}
\def\langnames@langs@wals@clt{Lautu}
\def\langnames@langs@wals@clu{Caluyanun}
\def\langnames@langs@wals@clw{Chulym Turkic}
\def\langnames@langs@wals@cly{Eastern Highland Chatino}
\def\langnames@langs@wals@cma{Maa}
\def\langnames@langs@wals@cme{Cerma}
\def\langnames@langs@wals@cmi{Emberá-Chamí}
\def\langnames@langs@wals@cml{Campalagian}
\def\langnames@langs@wals@cmn{Mandarin Chinese}
\def\langnames@langs@wals@cmo{Central Mnong}
\def\langnames@langs@wals@cmr{Mro Chin}
\def\langnames@langs@wals@cms{Messapic}
\def\langnames@langs@wals@cmt{Camtho}
\def\langnames@langs@wals@cna{Changthang}
\def\langnames@langs@wals@cnb{Chinbon Chin}
\def\langnames@langs@wals@cnc{Côông}
\def\langnames@langs@wals@cng{Northern Qiang}
\def\langnames@langs@wals@cnh{Haka Chin}
\def\langnames@langs@wals@cni{Asháninka}
\def\langnames@langs@wals@cnk{Khumi Chin}
\def\langnames@langs@wals@cnl{Lalana Chinantec}
\def\langnames@langs@wals@cnp{Northern Pinghua}
\def\langnames@langs@wals@cnq{Chung}
\def\langnames@langs@wals@cns{Central Asmat}
\def\langnames@langs@wals@cnt{Tepetotutla Chinantec}
\def\langnames@langs@wals@cnu{Western Algerian Berber}
\def\langnames@langs@wals@cnw{Ngawn Chin}
\def\langnames@langs@wals@coa{Cocos Islands Malay}
\def\langnames@langs@wals@cob{Chicomuceltec}
\def\langnames@langs@wals@coc{Cocopa}
\def\langnames@langs@wals@cod{Cocama-Cocamilla}
\def\langnames@langs@wals@coe{Koreguaje}
\def\langnames@langs@wals@cof{Tsafiki}
\def\langnames@langs@wals@cog{Chong of Chanthaburi}
\def\langnames@langs@wals@coh{Chonyi-Dzihana-Kauma}
\def\langnames@langs@wals@coj{Cochimi}
\def\langnames@langs@wals@cok{Santa Teresa Cora}
\def\langnames@langs@wals@col{Columbia-Wenatchi}
\def\langnames@langs@wals@com{Comanche}
\def\langnames@langs@wals@con{Cofán}
\def\langnames@langs@wals@coo{Comox}
\def\langnames@langs@wals@cop{Coptic}
\def\langnames@langs@wals@coq{Coquille}
\def\langnames@langs@wals@cor{Cornish}
\def\langnames@langs@wals@cos{Corsican}
\def\langnames@langs@wals@cot{Caquinte}
\def\langnames@langs@wals@cou{Wamey}
\def\langnames@langs@wals@cov{Cao Miao}
\def\langnames@langs@wals@cow{Cowlitz}
\def\langnames@langs@wals@cox{Nanti}
\def\langnames@langs@wals@coz{Chochotec}
\def\langnames@langs@wals@cpa{Palantla Chinantec}
\def\langnames@langs@wals@cpb{Ucayali-Yurúa Ashéninka}
\def\langnames@langs@wals@cpc{Ajyíninka Apurucayali}
\def\langnames@langs@wals@cpg{Cappadocian Greek}
\def\langnames@langs@wals@cpi{Chinese Pidgin English}
\def\langnames@langs@wals@cpn{Cherepon}
\def\langnames@langs@wals@cpo{Kpeego}
\def\langnames@langs@wals@cps{Capiznon}
\def\langnames@langs@wals@cpu{Pichis Ashéninka}
\def\langnames@langs@wals@cpx{Pu-Xian Chinese}
\def\langnames@langs@wals@cpy{South Ucayali Ashéninka}
\def\langnames@langs@wals@cra{Chara}
\def\langnames@langs@wals@crb{Island Carib}
\def\langnames@langs@wals@crc{Lonwolwol}
\def\langnames@langs@wals@crd{Coeur d'Alene}
\def\langnames@langs@wals@crf{Caramanta}
\def\langnames@langs@wals@crg{Michif}
\def\langnames@langs@wals@crh{Crimean Tatar}
\def\langnames@langs@wals@cri{Sãotomense}
\def\langnames@langs@wals@crj{Southern East Cree}
\def\langnames@langs@wals@crk{Plains Cree}
\def\langnames@langs@wals@crl{Northern East Cree}
\def\langnames@langs@wals@crm{Moose Cree}
\def\langnames@langs@wals@crn{El Nayar Cora}
\def\langnames@langs@wals@cro{Crow}
\def\langnames@langs@wals@crq{Iyo'wujwa Chorote}
\def\langnames@langs@wals@crr{Carolina Algonquian}
\def\langnames@langs@wals@crs{Seselwa Creole French}
\def\langnames@langs@wals@crt{Iyojwa'ja Chorote}
\def\langnames@langs@wals@crv{Chaura}
\def\langnames@langs@wals@crw{Chrau}
\def\langnames@langs@wals@crx{Central Carrier}
\def\langnames@langs@wals@cry{Kyoli}
\def\langnames@langs@wals@crz{Cruzeño}
\def\langnames@langs@wals@csa{Chiltepec Chinantec}
\def\langnames@langs@wals@csb{Kashubian}
\def\langnames@langs@wals@csc{Catalan Sign Language}
\def\langnames@langs@wals@csd{Chiangmai Sign Language}
\def\langnames@langs@wals@cse{Czech Sign Language}
\def\langnames@langs@wals@csf{Cuba Sign Language}
\def\langnames@langs@wals@csg{Chilean Sign Language}
\def\langnames@langs@wals@csh{Asho Chin}
\def\langnames@langs@wals@csi{Coast Miwok}
\def\langnames@langs@wals@csk{Jola-Esulalu}
\def\langnames@langs@wals@csl{Chinese Sign Language}
\def\langnames@langs@wals@csm{Central Sierra Miwok}
\def\langnames@langs@wals@csn{Colombian Sign Language}
\def\langnames@langs@wals@cso{Sochiapam Chinantec}
\def\langnames@langs@wals@csp{Southern Pinghua}
\def\langnames@langs@wals@csq{Croatian Sign Language}
\def\langnames@langs@wals@csr{Costa Rican Sign Language}
\def\langnames@langs@wals@css{Mutsun}
\def\langnames@langs@wals@cst{San Francisco Bay Ohlone}
\def\langnames@langs@wals@csv{Sumtu Chin}
\def\langnames@langs@wals@csw{Swampy Cree}
\def\langnames@langs@wals@csx{Cambodian Sign Language}
\def\langnames@langs@wals@csy{Sizang Chin}
\def\langnames@langs@wals@csz{Hanis}
\def\langnames@langs@wals@cta{Tataltepec Chatino}
\def\langnames@langs@wals@ctd{Tedim Chin}
\def\langnames@langs@wals@cte{Tepinapa Chinantec}
\def\langnames@langs@wals@ctg{Chittagonian}
\def\langnames@langs@wals@ctl{Tlacoatzintepec Chinantec}
\def\langnames@langs@wals@ctm{Chitimacha}
\def\langnames@langs@wals@ctn{Chintang}
\def\langnames@langs@wals@cto{Emberá-Catío}
\def\langnames@langs@wals@ctp{Western Highland Chatino}
\def\langnames@langs@wals@cts{Northern Catanduanes Bicolano}
\def\langnames@langs@wals@ctt{Wayanad Chetti}
\def\langnames@langs@wals@ctu{Chol}
\def\langnames@langs@wals@cty{Maundadan Chetti}
\def\langnames@langs@wals@ctz{Zacatepec Chatino}
\def\langnames@langs@wals@cua{Cua}
\def\langnames@langs@wals@cub{Cubeo}
\def\langnames@langs@wals@cuc{Usila Chinantec}
\def\langnames@langs@wals@cuh{Chuka}
\def\langnames@langs@wals@cui{Cuiba}
\def\langnames@langs@wals@cuj{Mashco Piro}
\def\langnames@langs@wals@cuk{San Blas Kuna}
\def\langnames@langs@wals@cul{Culina}
\def\langnames@langs@wals@cuo{Cumanagoto}
\def\langnames@langs@wals@cup{Cupeño}
\def\langnames@langs@wals@cuq{Cun}
\def\langnames@langs@wals@cur{Chhulung}
\def\langnames@langs@wals@cut{Teutila Cuicatec}
\def\langnames@langs@wals@cuu{Tai Ya}
\def\langnames@langs@wals@cuv{Cuvok}
\def\langnames@langs@wals@cuw{Chukwa}
\def\langnames@langs@wals@cux{Tepeuxila Cuicatec}
\def\langnames@langs@wals@cuy{Cuitlatec}
\def\langnames@langs@wals@cvg{Duhumbi}
\def\langnames@langs@wals@cvn{Valle Nacional Chinantec}
\def\langnames@langs@wals@cwa{Kabwa}
\def\langnames@langs@wals@cwb{Maindo}
\def\langnames@langs@wals@cwd{Woods Cree}
\def\langnames@langs@wals@cwe{Kwere}
\def\langnames@langs@wals@cwg{Chewong}
\def\langnames@langs@wals@cwt{Kuwaataay}
\def\langnames@langs@wals@cya{Nopala Chatino}
\def\langnames@langs@wals@cyb{Cayubaba}
\def\langnames@langs@wals@cym{Welsh}
\def\langnames@langs@wals@cyo{Cuyonon}
\def\langnames@langs@wals@czh{Hui Chinese}
\def\langnames@langs@wals@czn{Zenzontepec Chatino}
\def\langnames@langs@wals@czo{Min Zhong Chinese}
\def\langnames@langs@wals@czt{Zotung Chin}
\def\langnames@langs@wals@daa{Dangaleat}
\def\langnames@langs@wals@dac{Dambi}
\def\langnames@langs@wals@dad{Marik}
\def\langnames@langs@wals@dae{Duupa}
\def\langnames@langs@wals@daf{Dan}
\def\langnames@langs@wals@dag{Dagbani}
\def\langnames@langs@wals@dah{Gwahatike}
\def\langnames@langs@wals@dai{Day}
\def\langnames@langs@wals@daj{Dar Fur Daju}
\def\langnames@langs@wals@dak{Dakota}
\def\langnames@langs@wals@dal{Dahalo}
\def\langnames@langs@wals@dam{Damakawa}
\def\langnames@langs@wals@dan{Danish}
\def\langnames@langs@wals@dao{Daai Chin}
\def\langnames@langs@wals@daq{Dandami Maria}
\def\langnames@langs@wals@dar{North-Central Dargwa}
\def\langnames@langs@wals@das{Daho-Doo}
\def\langnames@langs@wals@dau{Dar Sila Daju}
\def\langnames@langs@wals@dav{Taita}
\def\langnames@langs@wals@daw{Davawenyo}
\def\langnames@langs@wals@dax{Dayi}
\def\langnames@langs@wals@daz{Dao}
\def\langnames@langs@wals@dba{Bangime}
\def\langnames@langs@wals@dbb{Deno}
\def\langnames@langs@wals@dbd{Dadiya}
\def\langnames@langs@wals@dbe{Dabe}
\def\langnames@langs@wals@dbf{Edopi}
\def\langnames@langs@wals@dbg{Dogul Dom Dogon}
\def\langnames@langs@wals@dbi{Doka}
\def\langnames@langs@wals@dbj{Ida'an}
\def\langnames@langs@wals@dbl{Dyirbal}
\def\langnames@langs@wals@dbm{Duguri}
\def\langnames@langs@wals@dbn{Duriankere}
\def\langnames@langs@wals@dbo{Dulbu}
\def\langnames@langs@wals@dbp{Duwai}
\def\langnames@langs@wals@dbq{Daba}
\def\langnames@langs@wals@dbr{Dabarre}
\def\langnames@langs@wals@dbt{Ben Tey Dogon}
\def\langnames@langs@wals@dbu{Najamba-Kindige}
\def\langnames@langs@wals@dbv{Dungu}
\def\langnames@langs@wals@dbw{Bankan Tey Dogon}
\def\langnames@langs@wals@dby{Dibiyaso}
\def\langnames@langs@wals@dcr{Negerhollands}
\def\langnames@langs@wals@ddd{Dongotono}
\def\langnames@langs@wals@dde{Doondo}
\def\langnames@langs@wals@ddg{Fataluku}
\def\langnames@langs@wals@ddi{Diodio}
\def\langnames@langs@wals@ddj{Jaru}
\def\langnames@langs@wals@ddn{Dendi (Benin)}
\def\langnames@langs@wals@ddo{Tsez}
\def\langnames@langs@wals@ddr{Dhudhuroa}
\def\langnames@langs@wals@dds{Donno So Dogon}
\def\langnames@langs@wals@ddw{Dawera-Daweloor}
\def\langnames@langs@wals@dec{Dagik}
\def\langnames@langs@wals@ded{Dedua}
\def\langnames@langs@wals@dee{Dewoin}
\def\langnames@langs@wals@def{Dezfuli-Shushtari}
\def\langnames@langs@wals@deg{Degema}
\def\langnames@langs@wals@deh{Dehwari}
\def\langnames@langs@wals@dei{Demisa}
\def\langnames@langs@wals@dek{Dek}
\def\langnames@langs@wals@dem{Dem}
\def\langnames@langs@wals@den{Slave}
\def\langnames@langs@wals@dep{Pidgin Delaware}
\def\langnames@langs@wals@deq{Dendi (Central African Republic)}
\def\langnames@langs@wals@der{Deori}
\def\langnames@langs@wals@des{Desano}
\def\langnames@langs@wals@deu{German}
\def\langnames@langs@wals@dev{Domung}
\def\langnames@langs@wals@dez{Dengese}
\def\langnames@langs@wals@dga{Central Dagaare}
\def\langnames@langs@wals@dgb{Bunoge Dogon}
\def\langnames@langs@wals@dgc{Casiguran-Nagtipunan Agta}
\def\langnames@langs@wals@dgd{Dagaari Dioula}
\def\langnames@langs@wals@dge{Degenan}
\def\langnames@langs@wals@dgg{Doga}
\def\langnames@langs@wals@dgh{Dghwede}
\def\langnames@langs@wals@dgi{Northern Dagara}
\def\langnames@langs@wals@dgk{Dagba}
\def\langnames@langs@wals@dgl{Nubian (Dongolese)}
\def\langnames@langs@wals@dgn{Dagoman}
\def\langnames@langs@wals@dgo{Dogri}
\def\langnames@langs@wals@dgr{Dogrib}
\def\langnames@langs@wals@dgs{Dogoso}
\def\langnames@langs@wals@dgx{Doghoro}
\def\langnames@langs@wals@dgz{Daga}
\def\langnames@langs@wals@dhd{Dhundari}
\def\langnames@langs@wals@dhg{Dhangu}
\def\langnames@langs@wals@dhi{Dhimal}
\def\langnames@langs@wals@dhl{Dhalandji}
\def\langnames@langs@wals@dhm{Zemba}
\def\langnames@langs@wals@dhn{Dhanki}
\def\langnames@langs@wals@dho{Dhodia-Kukna}
\def\langnames@langs@wals@dhr{Dhargari}
\def\langnames@langs@wals@dhs{Dhaiso}
\def\langnames@langs@wals@dhu{Dhurga}
\def\langnames@langs@wals@dhv{Dehu}
\def\langnames@langs@wals@dhw{Kochariya-East Danuwar}
\def\langnames@langs@wals@dia{Alu-Sinagen}
\def\langnames@langs@wals@dib{South Central Dinka}
\def\langnames@langs@wals@dic{Lakota Dida}
\def\langnames@langs@wals@did{Didinga}
\def\langnames@langs@wals@dif{Dieri}
\def\langnames@langs@wals@dig{Digo}
\def\langnames@langs@wals@dih{Tipai}
\def\langnames@langs@wals@dii{Dimbong}
\def\langnames@langs@wals@dij{Dai}
\def\langnames@langs@wals@dik{Southwestern Dinka}
\def\langnames@langs@wals@dil{Dilling}
\def\langnames@langs@wals@dim{Dime}
\def\langnames@langs@wals@din{Dinka}
\def\langnames@langs@wals@dio{Dibo}
\def\langnames@langs@wals@dip{Northeastern Dinka}
\def\langnames@langs@wals@diq{Dimli}
\def\langnames@langs@wals@dir{Dirim}
\def\langnames@langs@wals@dis{Dimasa}
\def\langnames@langs@wals@diu{Diriku-Shambyu}
\def\langnames@langs@wals@div{Dhivehi}
\def\langnames@langs@wals@diw{Northwestern Dinka}
\def\langnames@langs@wals@dix{Dixon Reef}
\def\langnames@langs@wals@diy{Diuwe}
\def\langnames@langs@wals@diz{Ding}
\def\langnames@langs@wals@djb{Djinba}
\def\langnames@langs@wals@djc{Dar Daju Daju}
\def\langnames@langs@wals@djd{Jaminjung-Ngaliwurru}
\def\langnames@langs@wals@dje{Zarma}
\def\langnames@langs@wals@djf{Djangun}
\def\langnames@langs@wals@dji{Djinang}
\def\langnames@langs@wals@djj{Djeebbana}
\def\langnames@langs@wals@djk{Aukan}
\def\langnames@langs@wals@djm{Jamsay Dogon}
\def\langnames@langs@wals@djn{Jawoyn}
\def\langnames@langs@wals@djo{Jangkang}
\def\langnames@langs@wals@djr{Djambarrpuyngu}
\def\langnames@langs@wals@dju{Kapriman}
\def\langnames@langs@wals@djw{Djawi}
\def\langnames@langs@wals@dka{Dakpakha}
\def\langnames@langs@wals@dkg{Kadung}
\def\langnames@langs@wals@dkk{Dakka}
\def\langnames@langs@wals@dkr{Kuijau}
\def\langnames@langs@wals@dks{Southeastern Dinka}
\def\langnames@langs@wals@dkx{Mazagway}
\def\langnames@langs@wals@dlg{Dolgan}
\def\langnames@langs@wals@dlk{Dahalik}
\def\langnames@langs@wals@dlm{Dalmatian}
\def\langnames@langs@wals@dln{Darlong}
\def\langnames@langs@wals@dma{Duma}
\def\langnames@langs@wals@dmb{Mombo Dogon}
\def\langnames@langs@wals@dmc{Gavak}
\def\langnames@langs@wals@dmd{Madimadi}
\def\langnames@langs@wals@dme{Dugwor}
\def\langnames@langs@wals@dmf{Medefidrin}
\def\langnames@langs@wals@dmg{Upper Kinabatangan}
\def\langnames@langs@wals@dmk{Domaaki}
\def\langnames@langs@wals@dml{Dameli}
\def\langnames@langs@wals@dmm{Dama (Cameroon)}
\def\langnames@langs@wals@dmo{Kemezung}
\def\langnames@langs@wals@dmr{East Damar}
\def\langnames@langs@wals@dms{Dampelas}
\def\langnames@langs@wals@dmu{Dubu}
\def\langnames@langs@wals@dmv{Dumpas}
\def\langnames@langs@wals@dmw{Mudburra}
\def\langnames@langs@wals@dmx{Dema}
\def\langnames@langs@wals@dmy{Demta}
\def\langnames@langs@wals@dna{Upper Grand Valley Dani}
\def\langnames@langs@wals@dnd{Daonda}
\def\langnames@langs@wals@dne{Ndendeule}
\def\langnames@langs@wals@dng{Dungan}
\def\langnames@langs@wals@dni{Lower Grand Valley Dani}
\def\langnames@langs@wals@dnj{Dan}
\def\langnames@langs@wals@dnk{Dengka}
\def\langnames@langs@wals@dnn{Dzùùngoo}
\def\langnames@langs@wals@dno{Ndrulo}
\def\langnames@langs@wals@dnr{Danaru}
\def\langnames@langs@wals@dnt{Mid Grand Valley Dani}
\def\langnames@langs@wals@dnu{Danau}
\def\langnames@langs@wals@dnw{Western Dani}
\def\langnames@langs@wals@dny{Deni}
\def\langnames@langs@wals@doa{Dom}
\def\langnames@langs@wals@dob{Dobu}
\def\langnames@langs@wals@doc{Northern Dong}
\def\langnames@langs@wals@doe{Doe}
\def\langnames@langs@wals@dof{Domu}
\def\langnames@langs@wals@doh{Dong}
\def\langnames@langs@wals@dok{Dondo}
\def\langnames@langs@wals@dol{Doso}
\def\langnames@langs@wals@don{Toura (Papua New Guinea)}
\def\langnames@langs@wals@doo{Dongo}
\def\langnames@langs@wals@dop{Lukpa}
\def\langnames@langs@wals@doq{Dominican Sign Language}
\def\langnames@langs@wals@dor{Dori'o}
\def\langnames@langs@wals@dos{Dogosé}
\def\langnames@langs@wals@dot{Dass}
\def\langnames@langs@wals@dov{Toka-Leya-Dombe}
\def\langnames@langs@wals@dow{Doyayo}
\def\langnames@langs@wals@dox{Bussa}
\def\langnames@langs@wals@doy{Dompo}
\def\langnames@langs@wals@doz{Dorze}
\def\langnames@langs@wals@dpp{Papar}
\def\langnames@langs@wals@drb{Dair}
\def\langnames@langs@wals@drc{Minderico}
\def\langnames@langs@wals@drd{Darma}
\def\langnames@langs@wals@dre{Dolpo}
\def\langnames@langs@wals@drg{Rungus}
\def\langnames@langs@wals@dri{C'lela}
\def\langnames@langs@wals@drl{Paakantyi}
\def\langnames@langs@wals@drn{West Damar}
\def\langnames@langs@wals@dro{Daro-Matu Melanau}
\def\langnames@langs@wals@drq{Dura}
\def\langnames@langs@wals@drs{Gedeo}
\def\langnames@langs@wals@dru{Rukai}
\def\langnames@langs@wals@dry{Darai}
\def\langnames@langs@wals@dsb{Lower Sorbian}
\def\langnames@langs@wals@dse{Dutch Sign Language}
\def\langnames@langs@wals@dsh{Daasanach}
\def\langnames@langs@wals@dsi{Dissa-Canton Mufa}
\def\langnames@langs@wals@dsl{Danish Sign Language}
\def\langnames@langs@wals@dsn{Dusner}
\def\langnames@langs@wals@dsq{Tadaksahak}
\def\langnames@langs@wals@dsz{Mardin Sign Language}
\def\langnames@langs@wals@dta{Dagur}
\def\langnames@langs@wals@dtb{Labuk-Kinabatangan Kadazan}
\def\langnames@langs@wals@dtd{Ditidaht}
\def\langnames@langs@wals@dth{Aritinngitigh}
\def\langnames@langs@wals@dti{Ana Tinga Dogon}
\def\langnames@langs@wals@dtk{Tengou-Togo Dogon}
\def\langnames@langs@wals@dtm{Tomo Kan Dogon}
\def\langnames@langs@wals@dtn{Daats'iin}
\def\langnames@langs@wals@dto{Tommo So Dogon}
\def\langnames@langs@wals@dtp{Kadazan Dusun}
\def\langnames@langs@wals@dtr{Lotud}
\def\langnames@langs@wals@dts{Toro So Dogon}
\def\langnames@langs@wals@dtt{Toro Tegu Dogon}
\def\langnames@langs@wals@dtu{Tebul Ure Dogon}
\def\langnames@langs@wals@dty{Dotyali}
\def\langnames@langs@wals@dua{Duala}
\def\langnames@langs@wals@dub{Dubli}
\def\langnames@langs@wals@duc{Duna}
\def\langnames@langs@wals@dud{Hun-Saare}
\def\langnames@langs@wals@due{Umiray Dumaget Agta}
\def\langnames@langs@wals@duf{Dumbea}
\def\langnames@langs@wals@dug{Duruma}
\def\langnames@langs@wals@duh{Dungra Bhil}
\def\langnames@langs@wals@dui{Dumun}
\def\langnames@langs@wals@duj{Dhuwal}
\def\langnames@langs@wals@duk{Duduela}
\def\langnames@langs@wals@dul{Alabat Island Agta}
\def\langnames@langs@wals@dum{Middle Dutch}
\def\langnames@langs@wals@dun{Dusun Deyah}
\def\langnames@langs@wals@duo{Dupaninan Agta}
\def\langnames@langs@wals@dup{Duano}
\def\langnames@langs@wals@duq{Dusun Malang}
\def\langnames@langs@wals@dur{Dii}
\def\langnames@langs@wals@dus{Dumi}
\def\langnames@langs@wals@duu{Drung}
\def\langnames@langs@wals@duv{Duvle}
\def\langnames@langs@wals@duw{Dusun Witu}
\def\langnames@langs@wals@dux{Duungooma}
\def\langnames@langs@wals@duy{Dicamay Agta}
\def\langnames@langs@wals@duz{Duli-Gewe}
\def\langnames@langs@wals@dva{Duau}
\def\langnames@langs@wals@dwa{Diri}
\def\langnames@langs@wals@dwr{Dawro}
\def\langnames@langs@wals@dws{Dutton World Speedwords}
\def\langnames@langs@wals@dwu{Djapu}
\def\langnames@langs@wals@dww{Dawawa}
\def\langnames@langs@wals@dwz{Dewas-Done Danuwar}
\def\langnames@langs@wals@dya{Dyan}
\def\langnames@langs@wals@dyb{Dyaberdyaber}
\def\langnames@langs@wals@dyd{Dyugun}
\def\langnames@langs@wals@dyg{Villa Viciosa Agta}
\def\langnames@langs@wals@dyi{Djimini Senoufo}
\def\langnames@langs@wals@dym{Yanda Dom Dogon}
\def\langnames@langs@wals@dyn{Dyangadi}
\def\langnames@langs@wals@dyo{Jola-Fonyi}
\def\langnames@langs@wals@dyu{Dyula}
\def\langnames@langs@wals@dyy{Dyaabugay}
\def\langnames@langs@wals@dza{Tunzu}
\def\langnames@langs@wals@dzd{Daza}
\def\langnames@langs@wals@dze{Djiwarli}
\def\langnames@langs@wals@dzg{Dazaga}
\def\langnames@langs@wals@dzl{Dzalakha}
\def\langnames@langs@wals@dzn{Dzando}
\def\langnames@langs@wals@dzo{Dzongkha}
\def\langnames@langs@wals@ebg{Ebughu}
\def\langnames@langs@wals@ebo{Teke-Eboo-Nzikou}
\def\langnames@langs@wals@ebr{Ebrié}
\def\langnames@langs@wals@ebu{Embu}
\def\langnames@langs@wals@ecr{Eteocretan}
\def\langnames@langs@wals@ecs{Ecuadorian Sign Language}
\def\langnames@langs@wals@ecy{Eteocypriot}
\def\langnames@langs@wals@eee{E}
\def\langnames@langs@wals@efa{Efai}
\def\langnames@langs@wals@efe{Efe}
\def\langnames@langs@wals@efi{Efik}
\def\langnames@langs@wals@ega{Ega}
\def\langnames@langs@wals@egl{Emiliano}
\def\langnames@langs@wals@ego{Eggon}
\def\langnames@langs@wals@egy{Egyptian (Ancient)}
\def\langnames@langs@wals@ehs{Miyakubo Sign Language}
\def\langnames@langs@wals@ehu{Ehueun}
\def\langnames@langs@wals@eip{Eipomek}
\def\langnames@langs@wals@eit{Eitiep}
\def\langnames@langs@wals@eiv{Askopan}
\def\langnames@langs@wals@eja{Ejamat}
\def\langnames@langs@wals@eka{Ekajuk}
\def\langnames@langs@wals@eke{Ekit}
\def\langnames@langs@wals@ekg{Ekari}
\def\langnames@langs@wals@eki{Eki}
\def\langnames@langs@wals@ekk{Estonian}
\def\langnames@langs@wals@ekl{Kol (Bangladesh)}
\def\langnames@langs@wals@ekm{Elip}
\def\langnames@langs@wals@eko{Koti}
\def\langnames@langs@wals@ekp{Ekpeye}
\def\langnames@langs@wals@ekr{Yace}
\def\langnames@langs@wals@eky{Eastern Kayah}
\def\langnames@langs@wals@ele{Elepi}
\def\langnames@langs@wals@elh{El Hugeirat}
\def\langnames@langs@wals@eli{Nding}
\def\langnames@langs@wals@elk{Elkei}
\def\langnames@langs@wals@ell{Modern Greek}
\def\langnames@langs@wals@elm{Eleme}
\def\langnames@langs@wals@elo{El Molo}
\def\langnames@langs@wals@elu{Elu}
\def\langnames@langs@wals@elx{Elamite}
\def\langnames@langs@wals@ema{Emai-Iuleha-Ora}
\def\langnames@langs@wals@emb{Embaloh}
\def\langnames@langs@wals@eme{Teko}
\def\langnames@langs@wals@emg{Eastern Meohang}
\def\langnames@langs@wals@emi{Mussau-Emira}
\def\langnames@langs@wals@emk{Eastern Maninkakan}
\def\langnames@langs@wals@emn{Eman}
\def\langnames@langs@wals@emp{Northern Emberá}
\def\langnames@langs@wals@emq{Eastern Muya}
\def\langnames@langs@wals@ems{Pacific Gulf Yupik}
\def\langnames@langs@wals@emu{Eastern Muria}
\def\langnames@langs@wals@emw{Emplawas}
\def\langnames@langs@wals@emx{Erromintxela}
\def\langnames@langs@wals@emy{Epigraphic Mayan}
\def\langnames@langs@wals@emz{Mbessa}
\def\langnames@langs@wals@ena{Apali}
\def\langnames@langs@wals@enb{Markweeta}
\def\langnames@langs@wals@enc{En}
\def\langnames@langs@wals@end{Ende}
\def\langnames@langs@wals@enf{Forest Enets}
\def\langnames@langs@wals@eng{English}
\def\langnames@langs@wals@enh{Tundra Enets}
\def\langnames@langs@wals@enl{Enlhet Norte}
\def\langnames@langs@wals@enm{Middle English}
\def\langnames@langs@wals@enn{Egene}
\def\langnames@langs@wals@eno{Enggano}
\def\langnames@langs@wals@enq{Enga}
\def\langnames@langs@wals@enr{Emumu}
\def\langnames@langs@wals@enu{Enu}
\def\langnames@langs@wals@env{Enwan (Edo State)}
\def\langnames@langs@wals@enw{Enwan (Akwa Ibom State)}
\def\langnames@langs@wals@enx{Enxet Sur}
\def\langnames@langs@wals@eot{Beti (Côte d'Ivoire)}
\def\langnames@langs@wals@epi{Epie}
\def\langnames@langs@wals@epo{Esperanto}
\def\langnames@langs@wals@era{Eravallan}
\def\langnames@langs@wals@erg{Sie}
\def\langnames@langs@wals@erh{Eruwa}
\def\langnames@langs@wals@eri{Ogea}
\def\langnames@langs@wals@erk{South Efate}
\def\langnames@langs@wals@ero{Stau-Dgebshes}
\def\langnames@langs@wals@err{Erre}
\def\langnames@langs@wals@ers{Ersu}
\def\langnames@langs@wals@ert{Eritai}
\def\langnames@langs@wals@erw{Erokwanas}
\def\langnames@langs@wals@ese{Ese Ejja}
\def\langnames@langs@wals@esg{Aheri Gondi}
\def\langnames@langs@wals@esh{Eshtehardi}
\def\langnames@langs@wals@esi{North Alaskan Inupiatun}
\def\langnames@langs@wals@esk{Seward Alaska Inupiatun}
\def\langnames@langs@wals@esl{Egypt Sign Language}
\def\langnames@langs@wals@esm{Esuma}
\def\langnames@langs@wals@esn{Salvadoran Sign Language}
\def\langnames@langs@wals@eso{Estonian Sign Language}
\def\langnames@langs@wals@esq{Esselen}
\def\langnames@langs@wals@ess{Central Siberian Yupik}
\def\langnames@langs@wals@esu{Central Alaskan Yupik}
\def\langnames@langs@wals@esy{Eskayan}
\def\langnames@langs@wals@etb{Etebi}
\def\langnames@langs@wals@eth{Ethiopian Sign Language}
\def\langnames@langs@wals@etn{Eton (Vanuatu)}
\def\langnames@langs@wals@eto{Eton-Mengisa}
\def\langnames@langs@wals@etr{Edolo}
\def\langnames@langs@wals@ets{Yekhee}
\def\langnames@langs@wals@ett{Etruscan}
\def\langnames@langs@wals@etu{Ejagham}
\def\langnames@langs@wals@etx{Eten}
\def\langnames@langs@wals@etz{Semimi}
\def\langnames@langs@wals@eus{Basque}
\def\langnames@langs@wals@eve{Even}
\def\langnames@langs@wals@evh{Uvbie}
\def\langnames@langs@wals@evn{Evenki}
\def\langnames@langs@wals@ewe{Ewe}
\def\langnames@langs@wals@ewo{Ewondo}
\def\langnames@langs@wals@ext{Extremaduran}
\def\langnames@langs@wals@eya{Eyak}
\def\langnames@langs@wals@eyo{Keiyo}
\def\langnames@langs@wals@eze{Uzekwe}
\def\langnames@langs@wals@faa{Fasu}
\def\langnames@langs@wals@fab{Annobonese}
\def\langnames@langs@wals@fad{Wagi (Papua New Guinea)}
\def\langnames@langs@wals@faf{Fagani}
\def\langnames@langs@wals@fag{Finongan}
\def\langnames@langs@wals@fah{Baissa Fali}
\def\langnames@langs@wals@fai{Faiwol}
\def\langnames@langs@wals@faj{Kulsab}
\def\langnames@langs@wals@fak{Fang (Cameroon)}
\def\langnames@langs@wals@fal{South Fali}
\def\langnames@langs@wals@fam{Fam}
\def\langnames@langs@wals@fan{Fang (Equatorial Guinea)}
\def\langnames@langs@wals@fao{Faroese}
\def\langnames@langs@wals@fap{Palor}
\def\langnames@langs@wals@far{Fataleka}
\def\langnames@langs@wals@fau{Fayu}
\def\langnames@langs@wals@fax{Fala}
\def\langnames@langs@wals@fay{Fars Dialects}
\def\langnames@langs@wals@fcs{Quebec Sign Language}
\def\langnames@langs@wals@fer{Feroge}
\def\langnames@langs@wals@ffm{Maasina Fulfulde}
\def\langnames@langs@wals@fia{Nobiin}
\def\langnames@langs@wals@fie{Fyer}
\def\langnames@langs@wals@fif{Faifi}
\def\langnames@langs@wals@fij{Fijian}
\def\langnames@langs@wals@fil{Filipino}
\def\langnames@langs@wals@fin{Finnish}
\def\langnames@langs@wals@fip{Fipa}
\def\langnames@langs@wals@fir{Firan}
\def\langnames@langs@wals@fit{Meänkieli}
\def\langnames@langs@wals@fiw{Fiwaga}
\def\langnames@langs@wals@fkk{Kirya-Konzel}
\def\langnames@langs@wals@fkv{Kven Finnish}
\def\langnames@langs@wals@fla{Kalispel-Pend d'Oreille}
\def\langnames@langs@wals@flh{Abawiri}
\def\langnames@langs@wals@fli{Fali}
\def\langnames@langs@wals@fll{North Fali}
\def\langnames@langs@wals@fln{Flinders Island}
\def\langnames@langs@wals@flr{Fuliiru}
\def\langnames@langs@wals@fly{Tsotsitaal}
\def\langnames@langs@wals@fmp{Fe'fe'}
\def\langnames@langs@wals@fmu{Far Western Muria}
\def\langnames@langs@wals@fnb{Orkon-Fanbak}
\def\langnames@langs@wals@fng{Fanagalo}
\def\langnames@langs@wals@fni{Fania}
\def\langnames@langs@wals@fod{Foodo}
\def\langnames@langs@wals@foi{Foi}
\def\langnames@langs@wals@fon{Fon}
\def\langnames@langs@wals@for{Fore}
\def\langnames@langs@wals@fos{Sirayaic}
\def\langnames@langs@wals@fpe{Pichi}
\def\langnames@langs@wals@fqs{Momu-Fas}
\def\langnames@langs@wals@fra{French}
\def\langnames@langs@wals@frc{Cajun French}
\def\langnames@langs@wals@frd{Fordata}
\def\langnames@langs@wals@fro{Old French (842-ca. 1400)}
\def\langnames@langs@wals@frp{Arpitan}
\def\langnames@langs@wals@frq{Forak}
\def\langnames@langs@wals@frr{Northern Frisian}
\def\langnames@langs@wals@frs{German Northern Low Saxon}
\def\langnames@langs@wals@frt{Kiai}
\def\langnames@langs@wals@fry{Western Frisian}
\def\langnames@langs@wals@fse{Finnish Sign Language}
\def\langnames@langs@wals@fsl{French Sign Language}
\def\langnames@langs@wals@fss{Finland-Swedish Sign Language}
\def\langnames@langs@wals@fub{Adamawa Fulfulde}
\def\langnames@langs@wals@fuc{Pulaar}
\def\langnames@langs@wals@fud{East Futuna}
\def\langnames@langs@wals@fue{Borgu Fulfulde}
\def\langnames@langs@wals@fuf{Pular}
\def\langnames@langs@wals@fuh{Western Niger Fulfulde}
\def\langnames@langs@wals@fui{Bagirmi Fulfulde}
\def\langnames@langs@wals@fuj{Ko}
\def\langnames@langs@wals@fun{Fulniô}
\def\langnames@langs@wals@fuq{Central-Eastern Niger Fulfulde}
\def\langnames@langs@wals@fur{Friulian}
\def\langnames@langs@wals@fut{Futuna-Aniwa}
\def\langnames@langs@wals@fuu{Furu}
\def\langnames@langs@wals@fuv{Hausa States Fulfulde}
\def\langnames@langs@wals@fuy{Fuyug}
\def\langnames@langs@wals@fvr{Fur}
\def\langnames@langs@wals@fwa{Fwâi}
\def\langnames@langs@wals@fwe{Fwe}
\def\langnames@langs@wals@gaa{Ga}
\def\langnames@langs@wals@gab{Gabri}
\def\langnames@langs@wals@gad{Gaddang}
\def\langnames@langs@wals@gae{Baniva de Maroa}
\def\langnames@langs@wals@gaf{Gende}
\def\langnames@langs@wals@gag{Gagauz}
\def\langnames@langs@wals@gah{Alekano}
\def\langnames@langs@wals@gai{Borei}
\def\langnames@langs@wals@gaj{Gadsup}
\def\langnames@langs@wals@gak{Gamkonora}
\def\langnames@langs@wals@gal{Galoli-Talur}
\def\langnames@langs@wals@gam{Kandawo}
\def\langnames@langs@wals@gan{Gan Chinese}
\def\langnames@langs@wals@gao{Gants}
\def\langnames@langs@wals@gap{Gal}
\def\langnames@langs@wals@gaq{Gata'}
\def\langnames@langs@wals@gar{Galeya}
\def\langnames@langs@wals@gas{Adiwasi Garasia}
\def\langnames@langs@wals@gat{Kenati}
\def\langnames@langs@wals@gau{Mudhili Gadaba}
\def\langnames@langs@wals@gaw{Nobonob}
\def\langnames@langs@wals@gax{Borana-Arsi-Guji Oromo}
\def\langnames@langs@wals@gay{Gayo}
\def\langnames@langs@wals@gaz{West Central Oromo}
\def\langnames@langs@wals@gbb{Kaytetye}
\def\langnames@langs@wals@gbd{Karadjeri}
\def\langnames@langs@wals@gbe{Niksek}
\def\langnames@langs@wals@gbf{Gaikundi}
\def\langnames@langs@wals@gbg{Gbanziri-Boraka}
\def\langnames@langs@wals@gbh{Defi Gbe}
\def\langnames@langs@wals@gbi{Galela}
\def\langnames@langs@wals@gbj{Bodo Gadaba}
\def\langnames@langs@wals@gbk{Gaddi}
\def\langnames@langs@wals@gbl{Gamit}
\def\langnames@langs@wals@gbm{Garhwali}
\def\langnames@langs@wals@gbn{Mo'da}
\def\langnames@langs@wals@gbo{Northern Grebo}
\def\langnames@langs@wals@gbp{Gbaya-Bossangoa}
\def\langnames@langs@wals@gbq{Gbaya-Bozoum}
\def\langnames@langs@wals@gbr{Gbagyi}
\def\langnames@langs@wals@gbs{Gbesi Gbe}
\def\langnames@langs@wals@gbu{Gaagudju}
\def\langnames@langs@wals@gbv{Gbanu}
\def\langnames@langs@wals@gbw{Kabikabi}
\def\langnames@langs@wals@gbx{Eastern Xwla Gbe}
\def\langnames@langs@wals@gby{Gbari}
\def\langnames@langs@wals@gbz{Zoroastrian Yazdi}
\def\langnames@langs@wals@gcc{Mali}
\def\langnames@langs@wals@gcd{Ganggalida}
\def\langnames@langs@wals@gce{Galice}
\def\langnames@langs@wals@gcf{Guadeloupe-Martinique Creole French}
\def\langnames@langs@wals@gcl{Grenadian Creole English}
\def\langnames@langs@wals@gcn{Gaina}
\def\langnames@langs@wals@gcr{Guianese Creole French}
\def\langnames@langs@wals@gct{Colonia Tovar German}
\def\langnames@langs@wals@gda{Gade Lohar}
\def\langnames@langs@wals@gdb{Pottangi Ollar Gadaba}
\def\langnames@langs@wals@gdc{Gugu Badhun}
\def\langnames@langs@wals@gdd{Gedaged}
\def\langnames@langs@wals@gde{Gude}
\def\langnames@langs@wals@gdf{Guduf-Gava}
\def\langnames@langs@wals@gdg{Ga'dang}
\def\langnames@langs@wals@gdh{Gajirrabeng}
\def\langnames@langs@wals@gdi{Gundi}
\def\langnames@langs@wals@gdj{Gurdjar}
\def\langnames@langs@wals@gdk{Gadang}
\def\langnames@langs@wals@gdl{Dirasha}
\def\langnames@langs@wals@gdm{Laal}
\def\langnames@langs@wals@gdn{Umanakaina}
\def\langnames@langs@wals@gdo{Godoberi}
\def\langnames@langs@wals@gdq{Mehri}
\def\langnames@langs@wals@gdr{Wipi}
\def\langnames@langs@wals@gds{Ghandruk Sign Language}
\def\langnames@langs@wals@gdu{Gudu}
\def\langnames@langs@wals@gdx{Godwari}
\def\langnames@langs@wals@gea{Geruma}
\def\langnames@langs@wals@geb{Kire}
\def\langnames@langs@wals@gec{Gboloo Grebo}
\def\langnames@langs@wals@ged{Gade}
\def\langnames@langs@wals@geh{Hutterite German}
\def\langnames@langs@wals@gei{Gebe}
\def\langnames@langs@wals@gej{Gen}
\def\langnames@langs@wals@gek{Yiwom}
\def\langnames@langs@wals@gel{Ut-Main}
\def\langnames@langs@wals@geq{Geme}
\def\langnames@langs@wals@ges{Geser-Gorom}
\def\langnames@langs@wals@gev{Viya}
\def\langnames@langs@wals@gew{Gera}
\def\langnames@langs@wals@gex{Garre}
\def\langnames@langs@wals@gey{Enya}
\def\langnames@langs@wals@gez{Geez}
\def\langnames@langs@wals@gfk{Patpatar}
\def\langnames@langs@wals@gft{Gafat}
\def\langnames@langs@wals@gga{Gao}
\def\langnames@langs@wals@ggb{Gbii}
\def\langnames@langs@wals@ggd{Gugadj}
\def\langnames@langs@wals@gge{Guragone}
\def\langnames@langs@wals@ggg{Gurgula}
\def\langnames@langs@wals@ggk{Kungarakany}
\def\langnames@langs@wals@ggl{Ganglau}
\def\langnames@langs@wals@ggr{Aghu Tharnggalu}
\def\langnames@langs@wals@ggt{Gitua}
\def\langnames@langs@wals@ggu{Gban}
\def\langnames@langs@wals@ggw{Gogodala}
\def\langnames@langs@wals@gha{Ghadames}
\def\langnames@langs@wals@ghe{Southern Ghale}
\def\langnames@langs@wals@ghh{Northern Ghale}
\def\langnames@langs@wals@ghk{Geko Karen}
\def\langnames@langs@wals@ghl{Uncunwee}
\def\langnames@langs@wals@ghn{Ghanongga}
\def\langnames@langs@wals@gho{Ghomara}
\def\langnames@langs@wals@ghr{Ghera}
\def\langnames@langs@wals@ghs{Guhu-Samane}
\def\langnames@langs@wals@ght{Kutang Ghale}
\def\langnames@langs@wals@gia{Kitja}
\def\langnames@langs@wals@gib{Gibanawa}
\def\langnames@langs@wals@gic{Gail}
\def\langnames@langs@wals@gid{Gidar}
\def\langnames@langs@wals@gie{Gabogbo}
\def\langnames@langs@wals@gig{Goaria}
\def\langnames@langs@wals@gih{Condamine-Upper Clarence Bandjalang}
\def\langnames@langs@wals@gii{Girirra}
\def\langnames@langs@wals@gil{Gilbertese}
\def\langnames@langs@wals@gim{Gimi (Eastern Highlands)}
\def\langnames@langs@wals@gin{Hinuq}
\def\langnames@langs@wals@gip{Gimi (West New Britain)}
\def\langnames@langs@wals@giq{Hagei Gelao}
\def\langnames@langs@wals@gir{Red Gelao}
\def\langnames@langs@wals@gis{North Giziga}
\def\langnames@langs@wals@git{Gitxsan}
\def\langnames@langs@wals@giu{Gelao Mulao}
\def\langnames@langs@wals@giw{Duoluo Gelao}
\def\langnames@langs@wals@gix{Gilima}
\def\langnames@langs@wals@giy{Giyug}
\def\langnames@langs@wals@giz{South Giziga}
\def\langnames@langs@wals@gjk{Kachi Koli}
\def\langnames@langs@wals@gjm{Warrnambool}
\def\langnames@langs@wals@gjn{Gonja}
\def\langnames@langs@wals@gjr{Gurindji Kriol}
\def\langnames@langs@wals@gju{Gujari}
\def\langnames@langs@wals@gka{Guya}
\def\langnames@langs@wals@gkd{Magi}
\def\langnames@langs@wals@gke{Ndai}
\def\langnames@langs@wals@gkn{Gokana}
\def\langnames@langs@wals@gko{Kok-Nar}
\def\langnames@langs@wals@gkp{Guinea Kpelle}
\def\langnames@langs@wals@gku{Danster !Ui}
\def\langnames@langs@wals@gla{Scottish Gaelic}
\def\langnames@langs@wals@glb{Belneng}
\def\langnames@langs@wals@glc{Bon Gula}
\def\langnames@langs@wals@gld{Nanai}
\def\langnames@langs@wals@gle{Irish}
\def\langnames@langs@wals@glg{Galician}
\def\langnames@langs@wals@glh{Northwest Pashayi}
\def\langnames@langs@wals@glj{Gula Iro}
\def\langnames@langs@wals@glk{Gilaki}
\def\langnames@langs@wals@gll{Bulloo River}
\def\langnames@langs@wals@glo{Galambu}
\def\langnames@langs@wals@glr{Glaro-Twabo}
\def\langnames@langs@wals@glu{Gula (Chad)}
\def\langnames@langs@wals@glv{Manx}
\def\langnames@langs@wals@glw{Glavda}
\def\langnames@langs@wals@gly{Gule}
\def\langnames@langs@wals@gma{Gambera}
\def\langnames@langs@wals@gmb{Gula'alaa}
\def\langnames@langs@wals@gmd{Mághdì}
\def\langnames@langs@wals@gmg{Magiyi}
\def\langnames@langs@wals@gmh{Middle High German}
\def\langnames@langs@wals@gml{Middle Low German}
\def\langnames@langs@wals@gmm{Gbaya-Mbodomo}
\def\langnames@langs@wals@gmn{Gimnime}
\def\langnames@langs@wals@gmu{Gumalu}
\def\langnames@langs@wals@gmv{Gamo}
\def\langnames@langs@wals@gmx{Magoma}
\def\langnames@langs@wals@gmy{Mycenaean Greek}
\def\langnames@langs@wals@gna{Kaansa}
\def\langnames@langs@wals@gnb{Gangte}
\def\langnames@langs@wals@gnc{Guanche}
\def\langnames@langs@wals@gnd{Zulgo-Gemzek}
\def\langnames@langs@wals@gne{Ganang}
\def\langnames@langs@wals@gng{Ngangam}
\def\langnames@langs@wals@gnh{Lere}
\def\langnames@langs@wals@gni{Gooniyandi}
\def\langnames@langs@wals@gnj{Ngen of Djonkro}
\def\langnames@langs@wals@gnk{//Gana}
\def\langnames@langs@wals@gnl{Gangulu}
\def\langnames@langs@wals@gnm{Ginuman}
\def\langnames@langs@wals@gnn{Gumatj}
\def\langnames@langs@wals@gno{Northern Gondi}
\def\langnames@langs@wals@gnq{Gana}
\def\langnames@langs@wals@gnr{Gureng Gureng}
\def\langnames@langs@wals@gnt{Warta Thuntai}
\def\langnames@langs@wals@gnu{Gnau}
\def\langnames@langs@wals@gnw{Western Bolivian Guaraní}
\def\langnames@langs@wals@gnz{Ganzi}
\def\langnames@langs@wals@goa{Guro}
\def\langnames@langs@wals@gob{Playero}
\def\langnames@langs@wals@goc{Gorakor}
\def\langnames@langs@wals@god{Godié}
\def\langnames@langs@wals@goe{Gongduk}
\def\langnames@langs@wals@gof{Gofa}
\def\langnames@langs@wals@gog{Gogo}
\def\langnames@langs@wals@goh{Old High German (ca. 750-1050)}
\def\langnames@langs@wals@goi{Gobasi}
\def\langnames@langs@wals@gol{Gola}
\def\langnames@langs@wals@gom{Goan Konkani}
\def\langnames@langs@wals@goo{Gone Dau}
\def\langnames@langs@wals@gop{Yeretuar}
\def\langnames@langs@wals@goq{Gorap}
\def\langnames@langs@wals@gor{Gorontalo}
\def\langnames@langs@wals@gos{Gronings}
\def\langnames@langs@wals@got{Gothic}
\def\langnames@langs@wals@gou{Gavar}
\def\langnames@langs@wals@gov{Goo}
\def\langnames@langs@wals@gow{Gorowa}
\def\langnames@langs@wals@gox{Gobu}
\def\langnames@langs@wals@goy{Goundo}
\def\langnames@langs@wals@goz{Alamuti}
\def\langnames@langs@wals@gpa{Gupa-Abawa}
\def\langnames@langs@wals@gpe{Ghanaian Pidgin English}
\def\langnames@langs@wals@gpn{Taiap}
\def\langnames@langs@wals@gqa{Ga'anda}
\def\langnames@langs@wals@gqi{Guiqiong}
\def\langnames@langs@wals@gqr{Gor}
\def\langnames@langs@wals@gqu{Central Gelao-Qau}
\def\langnames@langs@wals@gra{Rajput Garasia}
\def\langnames@langs@wals@grc{Ancient Greek}
\def\langnames@langs@wals@grd{Guruntum-Mbaaru}
\def\langnames@langs@wals@grg{Madi (Papua New Guinea)}
\def\langnames@langs@wals@grh{Gbiri-Niragu}
\def\langnames@langs@wals@gri{Ghari}
\def\langnames@langs@wals@grj{Southern Grebo}
\def\langnames@langs@wals@grm{Kota Marudu Talantang}
\def\langnames@langs@wals@gro{Groma}
\def\langnames@langs@wals@grq{Gorovu}
\def\langnames@langs@wals@grr{Sud Oranais-Gourara}
\def\langnames@langs@wals@grs{Gresi}
\def\langnames@langs@wals@grt{Garo}
\def\langnames@langs@wals@gru{Kistane}
\def\langnames@langs@wals@grv{Central Grebo}
\def\langnames@langs@wals@grw{Gweda}
\def\langnames@langs@wals@grx{Guriaso}
\def\langnames@langs@wals@gry{Barclayville Grebo}
\def\langnames@langs@wals@grz{Guramalum}
\def\langnames@langs@wals@gse{Ghanaian Sign Language}
\def\langnames@langs@wals@gsg{German Sign Language}
\def\langnames@langs@wals@gsl{Gusilay}
\def\langnames@langs@wals@gsm{Guatemalan Sign Language}
\def\langnames@langs@wals@gsn{Gusan}
\def\langnames@langs@wals@gso{Southwest Gbaya}
\def\langnames@langs@wals@gsp{Wasembo}
\def\langnames@langs@wals@gss{Greek Sign Language}
\def\langnames@langs@wals@gsw{Central Alemannic}
\def\langnames@langs@wals@gta{Guató}
\def\langnames@langs@wals@gua{Shiki}
\def\langnames@langs@wals@gub{Guajajara}
\def\langnames@langs@wals@guc{Wayuu}
\def\langnames@langs@wals@gud{Yocoboué Dida}
\def\langnames@langs@wals@gue{Gurindji}
\def\langnames@langs@wals@guf{Gupapuyngu}
\def\langnames@langs@wals@gug{Paraguayan Guaraní}
\def\langnames@langs@wals@guh{Guahibo}
\def\langnames@langs@wals@gui{Eastern Bolivian Guaraní}
\def\langnames@langs@wals@guj{Gujarati}
\def\langnames@langs@wals@guk{Northern Gumuz}
\def\langnames@langs@wals@gul{Sea Island Creole English}
\def\langnames@langs@wals@gum{Guambiano}
\def\langnames@langs@wals@gun{Mbyá Guaraní}
\def\langnames@langs@wals@guo{Guayabero}
\def\langnames@langs@wals@gup{Bininj Kun-Wok}
\def\langnames@langs@wals@guq{Aché}
\def\langnames@langs@wals@gur{Farefare}
\def\langnames@langs@wals@gus{Guinean Sign Language}
\def\langnames@langs@wals@gut{Maléku Jaíka}
\def\langnames@langs@wals@guu{Yanomamö}
\def\langnames@langs@wals@guw{Gun}
\def\langnames@langs@wals@gux{Gourmanchéma}
\def\langnames@langs@wals@guz{Gusii}
\def\langnames@langs@wals@gva{Guaná (Paraguay)}
\def\langnames@langs@wals@gvc{Kotiria}
\def\langnames@langs@wals@gve{Duwet}
\def\langnames@langs@wals@gvf{Golin}
\def\langnames@langs@wals@gvj{Guajá}
\def\langnames@langs@wals@gvl{Gulay}
\def\langnames@langs@wals@gvm{Gurmana}
\def\langnames@langs@wals@gvn{Kuku-Yalanji}
\def\langnames@langs@wals@gvo{Gavião Do Jiparaná}
\def\langnames@langs@wals@gvp{Pará-Maranhão Gavião}
\def\langnames@langs@wals@gvr{Gurung}
\def\langnames@langs@wals@gvs{Gumawana}
\def\langnames@langs@wals@gvy{Guyani}
\def\langnames@langs@wals@gwa{Mbato}
\def\langnames@langs@wals@gwb{Gwa}
\def\langnames@langs@wals@gwc{Gawri}
\def\langnames@langs@wals@gwd{Ale-Gawwada}
\def\langnames@langs@wals@gwe{Gweno}
\def\langnames@langs@wals@gwf{Gowro}
\def\langnames@langs@wals@gwg{Moo}
\def\langnames@langs@wals@gwi{Gwich'in}
\def\langnames@langs@wals@gwj{/Gwi}
\def\langnames@langs@wals@gwn{Gwandara}
\def\langnames@langs@wals@gwr{Gwere}
\def\langnames@langs@wals@gwt{Gawar-Bati}
\def\langnames@langs@wals@gwu{Guwamu}
\def\langnames@langs@wals@gww{Kwini}
\def\langnames@langs@wals@gwx{Gua}
\def\langnames@langs@wals@gxx{Wè Southern}
\def\langnames@langs@wals@gya{Northwest Gbaya}
\def\langnames@langs@wals@gyb{Garus}
\def\langnames@langs@wals@gyd{Kayardild}
\def\langnames@langs@wals@gye{Gyem}
\def\langnames@langs@wals@gyf{Gungabula}
\def\langnames@langs@wals@gyg{Gbayi}
\def\langnames@langs@wals@gyi{Gyele}
\def\langnames@langs@wals@gyl{Gayil}
\def\langnames@langs@wals@gym{Ngäbere}
\def\langnames@langs@wals@gyn{Guyanese Creole English}
\def\langnames@langs@wals@gyr{Guarayu}
\def\langnames@langs@wals@gyy{Gunya}
\def\langnames@langs@wals@gyz{Gyaazi}
\def\langnames@langs@wals@gza{Ganza}
\def\langnames@langs@wals@gzi{Gazic}
\def\langnames@langs@wals@gzn{Gane}
\def\langnames@langs@wals@haa{Han}
\def\langnames@langs@wals@hab{Hanoi Sign Language}
\def\langnames@langs@wals@hac{Gurani}
\def\langnames@langs@wals@had{Hatam}
\def\langnames@langs@wals@hae{Eastern Oromo}
\def\langnames@langs@wals@haf{Haiphong Sign Language}
\def\langnames@langs@wals@hag{Hanga}
\def\langnames@langs@wals@hah{Hahon}
\def\langnames@langs@wals@hai{Haida}
\def\langnames@langs@wals@haj{Hajong}
\def\langnames@langs@wals@hak{Hakka Chinese}
\def\langnames@langs@wals@hal{Halang}
\def\langnames@langs@wals@ham{Hewa}
\def\langnames@langs@wals@han{Hangaza}
\def\langnames@langs@wals@hao{Hakö}
\def\langnames@langs@wals@hap{Hupla}
\def\langnames@langs@wals@haq{Ha}
\def\langnames@langs@wals@har{Harari}
\def\langnames@langs@wals@has{Haisla}
\def\langnames@langs@wals@hat{Haitian}
\def\langnames@langs@wals@hau{Hausa}
\def\langnames@langs@wals@hav{Havu}
\def\langnames@langs@wals@haw{Hawaiian}
\def\langnames@langs@wals@hax{Southern Haida}
\def\langnames@langs@wals@hay{Haya}
\def\langnames@langs@wals@haz{Hazaragi}
\def\langnames@langs@wals@hba{Hamba de Lomela}
\def\langnames@langs@wals@hbb{Huba}
\def\langnames@langs@wals@hbn{Ebang}
\def\langnames@langs@wals@hbo{Ancient Hebrew}
\def\langnames@langs@wals@hbs{Serbian-Croatian-Bosnian}
\def\langnames@langs@wals@hbu{Habu}
\def\langnames@langs@wals@hca{Andaman Creole Hindi}
\def\langnames@langs@wals@hch{Huichol}
\def\langnames@langs@wals@hdn{Northern Haida}
\def\langnames@langs@wals@hds{Honduras Sign Language}
\def\langnames@langs@wals@hdy{Hadiyya}
\def\langnames@langs@wals@hea{Northern Qiandong Miao}
\def\langnames@langs@wals@heb{Modern Hebrew}
\def\langnames@langs@wals@hed{Herde}
\def\langnames@langs@wals@heg{Helong}
\def\langnames@langs@wals@heh{Hehe}
\def\langnames@langs@wals@hei{Heiltsuk-Oowekyala}
\def\langnames@langs@wals@hem{Hemba-Yazi}
\def\langnames@langs@wals@her{Herero}
\def\langnames@langs@wals@hgm{Hai//om-Akhoe}
\def\langnames@langs@wals@hgw{Haigwai}
\def\langnames@langs@wals@hhi{Hoia Hoia-Ukusi-Koperami}
\def\langnames@langs@wals@hhr{Keerak}
\def\langnames@langs@wals@hhy{Hoyahoya-Matakaia}
\def\langnames@langs@wals@hia{Lamang}
\def\langnames@langs@wals@hib{Hibito}
\def\langnames@langs@wals@hid{Hidatsa}
\def\langnames@langs@wals@hif{Fiji Hindi}
\def\langnames@langs@wals@hig{Kamwe}
\def\langnames@langs@wals@hih{Pamosu}
\def\langnames@langs@wals@hii{Hinduri}
\def\langnames@langs@wals@hij{Hijuk}
\def\langnames@langs@wals@hik{Seit-Kaitetu}
\def\langnames@langs@wals@hil{Hiligaynon}
\def\langnames@langs@wals@hin{Hindi}
\def\langnames@langs@wals@hio{Northern Tshwa}
\def\langnames@langs@wals@hir{Himarimã}
\def\langnames@langs@wals@hit{Hittite}
\def\langnames@langs@wals@hiw{Hiw}
\def\langnames@langs@wals@hix{Hixkaryána}
\def\langnames@langs@wals@hji{Haji}
\def\langnames@langs@wals@hka{Kahe}
\def\langnames@langs@wals@hke{Hunde}
\def\langnames@langs@wals@hkh{Khah}
\def\langnames@langs@wals@hkk{Hunjara-Kaina Ke}
\def\langnames@langs@wals@hkn{Mel-Khaonh}
\def\langnames@langs@wals@hks{Hong Kong-Macau Sign Language}
\def\langnames@langs@wals@hla{Halia}
\def\langnames@langs@wals@hlb{Halbi}
\def\langnames@langs@wals@hld{Halang Doan}
\def\langnames@langs@wals@hle{Hlersu}
\def\langnames@langs@wals@hlt{Nga La}
\def\langnames@langs@wals@hlu{Hieroglyphic Luwian}
\def\langnames@langs@wals@hma{Southern Mashan Hmong}
\def\langnames@langs@wals@hmb{Humburi Senni Songhay}
\def\langnames@langs@wals@hmc{Central Huishui Hmong}
\def\langnames@langs@wals@hmd{Diandongbei-Large Flowery Miao}
\def\langnames@langs@wals@hme{Eastern Huishui Hmong}
\def\langnames@langs@wals@hmf{Hmong Don}
\def\langnames@langs@wals@hmg{Southwestern Guiyang Hmong}
\def\langnames@langs@wals@hmh{Southwestern Huishui Hmong}
\def\langnames@langs@wals@hmi{Northern Huishui Hmong}
\def\langnames@langs@wals@hmj{Ge}
\def\langnames@langs@wals@hml{Luopohe Hmong}
\def\langnames@langs@wals@hmm{Central Mashan Hmong}
\def\langnames@langs@wals@hmo{Hiri Motu}
\def\langnames@langs@wals@hmp{Northern Mashan Hmong}
\def\langnames@langs@wals@hmq{Eastern Qiandong Miao}
\def\langnames@langs@wals@hmr{Hmar}
\def\langnames@langs@wals@hms{Southern Qiandong Miao}
\def\langnames@langs@wals@hmt{Hamtai}
\def\langnames@langs@wals@hmu{Hamap}
\def\langnames@langs@wals@hmv{Hmong Dô}
\def\langnames@langs@wals@hmw{Western Mashan Hmong}
\def\langnames@langs@wals@hmy{Southern Guiyang Hmong}
\def\langnames@langs@wals@hmz{Sinicized Miao}
\def\langnames@langs@wals@hna{Mina (Cameroon)}
\def\langnames@langs@wals@hnd{Southern Hindko}
\def\langnames@langs@wals@hne{Chhattisgarhi}
\def\langnames@langs@wals@hng{Hungu-Pombo}
\def\langnames@langs@wals@hnh{//Ani}
\def\langnames@langs@wals@hni{Hani}
\def\langnames@langs@wals@hnj{Hmong Njua}
\def\langnames@langs@wals@hnn{Hanunoo}
\def\langnames@langs@wals@hno{Northern Hindko}
\def\langnames@langs@wals@hns{Caribbean Hindustani}
\def\langnames@langs@wals@hnu{Hung}
\def\langnames@langs@wals@hoa{Hoava}
\def\langnames@langs@wals@hob{Mari (Madang Province)}
\def\langnames@langs@wals@hoc{Ho}
\def\langnames@langs@wals@hod{Holma}
\def\langnames@langs@wals@hoe{Horom}
\def\langnames@langs@wals@hoh{Hobyót}
\def\langnames@langs@wals@hoi{Holikachuk}
\def\langnames@langs@wals@hoj{Hadothi}
\def\langnames@langs@wals@hol{Holu}
\def\langnames@langs@wals@hom{Homa}
\def\langnames@langs@wals@hoo{Holoholo}
\def\langnames@langs@wals@hop{Hopi}
\def\langnames@langs@wals@hor{Horo}
\def\langnames@langs@wals@hos{Ho Chi Minh City Sign Language}
\def\langnames@langs@wals@hot{Hote}
\def\langnames@langs@wals@hov{Hobongan}
\def\langnames@langs@wals@how{Honi}
\def\langnames@langs@wals@hoy{Holiya}
\def\langnames@langs@wals@hoz{Hozo}
\def\langnames@langs@wals@hpo{Hpon}
\def\langnames@langs@wals@hps{Hawai'i Pidgin Sign Language}
\def\langnames@langs@wals@hra{Hrangkhol}
\def\langnames@langs@wals@hrc{Tangga}
\def\langnames@langs@wals@hre{Hre}
\def\langnames@langs@wals@hrk{Haruku}
\def\langnames@langs@wals@hrm{Horned Miao}
\def\langnames@langs@wals@hro{Haroi}
\def\langnames@langs@wals@hrt{Hertevin}
\def\langnames@langs@wals@hru{Hruso}
\def\langnames@langs@wals@hrx{Hunsrik}
\def\langnames@langs@wals@hrz{Harzani-Kilit}
\def\langnames@langs@wals@hsb{Upper Sorbian}
\def\langnames@langs@wals@hsh{Hungarian Sign Language}
\def\langnames@langs@wals@hsl{Hausa Sign Language}
\def\langnames@langs@wals@hsn{Xiang Chinese}
\def\langnames@langs@wals@hss{Harsusi}
\def\langnames@langs@wals@hti{Hoti of East Seram}
\def\langnames@langs@wals@hto{Minica Huitoto}
\def\langnames@langs@wals@hts{Hadza}
\def\langnames@langs@wals@htu{Hitu}
\def\langnames@langs@wals@hub{Huambisa}
\def\langnames@langs@wals@huc{Amkoe}
\def\langnames@langs@wals@hud{Huaulu}
\def\langnames@langs@wals@hue{San Francisco del Mar Huave}
\def\langnames@langs@wals@huf{Humene}
\def\langnames@langs@wals@hug{Huachipaeri}
\def\langnames@langs@wals@huh{Huilliche}
\def\langnames@langs@wals@hui{Huli}
\def\langnames@langs@wals@huj{Northern Guiyang Hmong}
\def\langnames@langs@wals@huk{Hulung}
\def\langnames@langs@wals@hul{Hula}
\def\langnames@langs@wals@hum{Hungan}
\def\langnames@langs@wals@hun{Hungarian}
\def\langnames@langs@wals@huo{Hu}
\def\langnames@langs@wals@hup{Hupa-Chilula}
\def\langnames@langs@wals@huq{Tsat}
\def\langnames@langs@wals@hur{Halkomelem}
\def\langnames@langs@wals@hus{Huastec}
\def\langnames@langs@wals@hut{Humla}
\def\langnames@langs@wals@huu{Murui Huitoto}
\def\langnames@langs@wals@huv{San Mateo del Mar Huave}
\def\langnames@langs@wals@huw{Hukumina}
\def\langnames@langs@wals@hux{Nüpode Huitoto}
\def\langnames@langs@wals@huy{Hulaulá}
\def\langnames@langs@wals@huz{Hunzib}
\def\langnames@langs@wals@hvc{Haitian Vodoun Culture Language}
\def\langnames@langs@wals@hve{San Dionisio del Mar Huave}
\def\langnames@langs@wals@hvk{Haveke}
\def\langnames@langs@wals@hvn{Hawu}
\def\langnames@langs@wals@hvv{Santa María del Mar Huave}
\def\langnames@langs@wals@hwa{Wané}
\def\langnames@langs@wals@hwc{Hawai'i Creole English}
\def\langnames@langs@wals@hwo{Hwana}
\def\langnames@langs@wals@hya{Hya}
\def\langnames@langs@wals@hye{Eastern Armenian}
\def\langnames@langs@wals@hyw{Western Armenian}
\def\langnames@langs@wals@iai{Iaai}
\def\langnames@langs@wals@ian{Iatmul}
\def\langnames@langs@wals@iar{Purari}
\def\langnames@langs@wals@iba{Iban}
\def\langnames@langs@wals@ibb{Ibibio}
\def\langnames@langs@wals@ibd{Iwaidja}
\def\langnames@langs@wals@ibe{Akpes}
\def\langnames@langs@wals@ibg{Ibanag}
\def\langnames@langs@wals@ibh{Bih}
\def\langnames@langs@wals@ibl{Ibaloi}
\def\langnames@langs@wals@ibm{Agoi}
\def\langnames@langs@wals@ibn{Ibino}
\def\langnames@langs@wals@ibo{Igbo}
\def\langnames@langs@wals@ibr{Ibuoro}
\def\langnames@langs@wals@ibu{Ibu}
\def\langnames@langs@wals@iby{Ibani}
\def\langnames@langs@wals@ica{Ede Ica}
\def\langnames@langs@wals@ich{Etkywan}
\def\langnames@langs@wals@icl{Icelandic Sign Language}
\def\langnames@langs@wals@icr{San Andres Creole English}
\def\langnames@langs@wals@ida{Idakho-Isukha-Tiriki}
\def\langnames@langs@wals@idb{Daman-Diu Portuguese}
\def\langnames@langs@wals@idc{Idon}
\def\langnames@langs@wals@idd{Ede Idaca}
\def\langnames@langs@wals@ide{Idere}
\def\langnames@langs@wals@idi{Idi-Taeme}
\def\langnames@langs@wals@ido{Ido}
\def\langnames@langs@wals@idr{Indri}
\def\langnames@langs@wals@idt{Idaté}
\def\langnames@langs@wals@idu{Idoma}
\def\langnames@langs@wals@ifa{Amganad Ifugao}
\def\langnames@langs@wals@ifb{Batad Ifugao}
\def\langnames@langs@wals@ife{Ifè}
\def\langnames@langs@wals@iff{Ifo}
\def\langnames@langs@wals@ifk{Tuwali Ifugao}
\def\langnames@langs@wals@ifm{Teke-Fuumu}
\def\langnames@langs@wals@ifu{Mayoyao Ifugao}
\def\langnames@langs@wals@ify{Keley-i Kallahan}
\def\langnames@langs@wals@igb{Ebira}
\def\langnames@langs@wals@ige{Igede}
\def\langnames@langs@wals@igg{Igana}
\def\langnames@langs@wals@igl{Igala}
\def\langnames@langs@wals@igm{Kanggape}
\def\langnames@langs@wals@ign{Ignaciano}
\def\langnames@langs@wals@igo{Isebe}
\def\langnames@langs@wals@igs{Interglossa}
\def\langnames@langs@wals@igw{Igwe}
\def\langnames@langs@wals@ihb{Iha-based Pidgin}
\def\langnames@langs@wals@ihp{Iha}
\def\langnames@langs@wals@ihw{Birrdhawal}
\def\langnames@langs@wals@iii{Sichuan Yi}
\def\langnames@langs@wals@iin{Thiin}
\def\langnames@langs@wals@ijc{Izon}
\def\langnames@langs@wals@ije{Biseni}
\def\langnames@langs@wals@ijj{Ede Ije}
\def\langnames@langs@wals@ijn{Kalabari}
\def\langnames@langs@wals@ijs{Southeast Ijo}
\def\langnames@langs@wals@ike{Eastern Canadian Inuktitut}
\def\langnames@langs@wals@iki{Iko}
\def\langnames@langs@wals@ikk{Ika}
\def\langnames@langs@wals@ikl{Ikulu}
\def\langnames@langs@wals@iko{Olulumo-Ikom}
\def\langnames@langs@wals@ikp{Ikpeshi}
\def\langnames@langs@wals@ikr{Ikaranggal}
\def\langnames@langs@wals@iks{Inuit Sign Language}
\def\langnames@langs@wals@ikt{Western Canadian Inuktitut}
\def\langnames@langs@wals@ikv{Iku-Gora-Ankwa}
\def\langnames@langs@wals@ikw{Ikwere}
\def\langnames@langs@wals@ikx{Ik}
\def\langnames@langs@wals@ikz{Ikizu}
\def\langnames@langs@wals@ila{Ile Ape}
\def\langnames@langs@wals@ilb{Ila}
\def\langnames@langs@wals@ile{Interlingue (Occidental)}
\def\langnames@langs@wals@ilg{Garig-Ilgar}
\def\langnames@langs@wals@ili{Ili Turki}
\def\langnames@langs@wals@ilk{Ilongot}
\def\langnames@langs@wals@ill{Iranun}
\def\langnames@langs@wals@ilo{Iloko}
\def\langnames@langs@wals@ils{International Sign}
\def\langnames@langs@wals@ilu{Ili'uun}
\def\langnames@langs@wals@ilv{Ilue}
\def\langnames@langs@wals@ima{Mala Malasar}
\def\langnames@langs@wals@imi{Anamuxra}
\def\langnames@langs@wals@iml{Miluk}
\def\langnames@langs@wals@imn{Imonda}
\def\langnames@langs@wals@imo{Imbongu}
\def\langnames@langs@wals@imr{Imroing}
\def\langnames@langs@wals@imy{Milyan}
\def\langnames@langs@wals@ina{Interlingua (International Auxiliary Language Association)}
\def\langnames@langs@wals@inb{Colombian Inga}
\def\langnames@langs@wals@ind{Standard Indonesian}
\def\langnames@langs@wals@ing{Degexit'an}
\def\langnames@langs@wals@inh{Ingush}
\def\langnames@langs@wals@inl{Jakartan Sign Language}
\def\langnames@langs@wals@inm{Minaean}
\def\langnames@langs@wals@inn{Isinai}
\def\langnames@langs@wals@ino{Inoke-Yate}
\def\langnames@langs@wals@inp{Iñapari}
\def\langnames@langs@wals@ins{Indian Sign Language}
\def\langnames@langs@wals@int{Intha-Danu}
\def\langnames@langs@wals@inz{Ineseño}
\def\langnames@langs@wals@ior{Inoric}
\def\langnames@langs@wals@iou{Tuma-Irumu}
\def\langnames@langs@wals@iow{Iowa-Oto}
\def\langnames@langs@wals@ipi{Ipili}
\def\langnames@langs@wals@ipo{Ipiko}
\def\langnames@langs@wals@iqu{Iquito}
\def\langnames@langs@wals@ire{Yerisiam}
\def\langnames@langs@wals@irh{Irarutu}
\def\langnames@langs@wals@iri{Irigwe}
\def\langnames@langs@wals@irk{Iraqw}
\def\langnames@langs@wals@irn{Irántxe-Münkü}
\def\langnames@langs@wals@iru{Irula of the Nilgiri}
\def\langnames@langs@wals@irx{Kamberau}
\def\langnames@langs@wals@iry{Iraya}
\def\langnames@langs@wals@isa{Isabi}
\def\langnames@langs@wals@isc{Isconahua}
\def\langnames@langs@wals@isd{Isnag}
\def\langnames@langs@wals@ise{Italian Sign Language}
\def\langnames@langs@wals@isg{Irish Sign Language}
\def\langnames@langs@wals@ish{Esan}
\def\langnames@langs@wals@isi{Nkem-Nkum}
\def\langnames@langs@wals@isk{Ishkashimi}
\def\langnames@langs@wals@isl{Icelandic}
\def\langnames@langs@wals@ism{Masimasi}
\def\langnames@langs@wals@isn{Isanzu}
\def\langnames@langs@wals@iso{Isoko}
\def\langnames@langs@wals@isr{Israeli Sign Language}
\def\langnames@langs@wals@ist{Istriot}
\def\langnames@langs@wals@isu{Isu (Menchum Division)}
\def\langnames@langs@wals@ita{Italian}
\def\langnames@langs@wals@itb{Binongan Itneg}
\def\langnames@langs@wals@itd{Southern Tidung}
\def\langnames@langs@wals@ite{Itene}
\def\langnames@langs@wals@iti{Inlaod Itneg}
\def\langnames@langs@wals@itk{Judeo-Italian}
\def\langnames@langs@wals@itl{West Itelmen}
\def\langnames@langs@wals@itm{Itu Mbon Uzo}
\def\langnames@langs@wals@ito{Itonama}
\def\langnames@langs@wals@itr{Iteri}
\def\langnames@langs@wals@its{Isekiri}
\def\langnames@langs@wals@itt{Maeng Itneg}
\def\langnames@langs@wals@itv{Itawit}
\def\langnames@langs@wals@itw{Ito}
\def\langnames@langs@wals@itx{Itik}
\def\langnames@langs@wals@ity{Moyadan Itneg}
\def\langnames@langs@wals@itz{Itzá}
\def\langnames@langs@wals@ium{Iu Mien}
\def\langnames@langs@wals@ivb{Ibatan}
\def\langnames@langs@wals@ivv{Itbayat}
\def\langnames@langs@wals@iwk{I-Wak}
\def\langnames@langs@wals@iwm{Iwam}
\def\langnames@langs@wals@iwo{Morop-Dintere}
\def\langnames@langs@wals@iws{Sepik Iwam}
\def\langnames@langs@wals@ixc{Ixcatec}
\def\langnames@langs@wals@ixl{Ixil}
\def\langnames@langs@wals@iya{Iyayu}
\def\langnames@langs@wals@iyo{Mesaka}
\def\langnames@langs@wals@iyx{Yaka (Congo)}
\def\langnames@langs@wals@izh{Ingrian}
\def\langnames@langs@wals@izi{Izi-Ezaa-Ikwo-Mgbo}
\def\langnames@langs@wals@izr{Izere}
\def\langnames@langs@wals@izz{Izi}
\def\langnames@langs@wals@jaa{Madi}
\def\langnames@langs@wals@jab{Hyam}
\def\langnames@langs@wals@jac{Popti'}
\def\langnames@langs@wals@jad{Jahanka}
\def\langnames@langs@wals@jae{Yabem}
\def\langnames@langs@wals@jaf{Jara}
\def\langnames@langs@wals@jah{Jah Hut}
\def\langnames@langs@wals@jaj{Zazao}
\def\langnames@langs@wals@jak{Jakun}
\def\langnames@langs@wals@jal{Yalahatan-Haruru-Awaiya}
\def\langnames@langs@wals@jam{Jamaican Creole English}
\def\langnames@langs@wals@jao{Yanyuwa}
\def\langnames@langs@wals@jaq{Yaqay}
\def\langnames@langs@wals@jar{Jarawa (Nigeria)}
\def\langnames@langs@wals@jas{New Caledonian Javanese}
\def\langnames@langs@wals@jat{Inku}
\def\langnames@langs@wals@jau{Yaur}
\def\langnames@langs@wals@jav{Javanese}
\def\langnames@langs@wals@jax{Jambi Malay}
\def\langnames@langs@wals@jay{Nhangu}
\def\langnames@langs@wals@jaz{Jawe}
\def\langnames@langs@wals@jbi{Badjirri}
\def\langnames@langs@wals@jbj{Dombano}
\def\langnames@langs@wals@jbk{Barikewa}
\def\langnames@langs@wals@jbm{Bijim}
\def\langnames@langs@wals@jbn{Nafusi}
\def\langnames@langs@wals@jbo{Lojban}
\def\langnames@langs@wals@jbr{Jofotek-Bromnya}
\def\langnames@langs@wals@jbt{Djeoromitxí}
\def\langnames@langs@wals@jbu{Jukun Takum}
\def\langnames@langs@wals@jcs{Jamaican Country Sign Language}
\def\langnames@langs@wals@jct{Krymchak}
\def\langnames@langs@wals@jda{Jad}
\def\langnames@langs@wals@jdg{Jadgali}
\def\langnames@langs@wals@jdt{Judeo-Tat}
\def\langnames@langs@wals@jeb{Jebero}
\def\langnames@langs@wals@jee{Jerung}
\def\langnames@langs@wals@jeh{Jeh}
\def\langnames@langs@wals@jei{Yei}
\def\langnames@langs@wals@jek{Jeli}
\def\langnames@langs@wals@jel{Southern Yelmek}
\def\langnames@langs@wals@jen{Dza}
\def\langnames@langs@wals@jer{Jere}
\def\langnames@langs@wals@jet{Manem}
\def\langnames@langs@wals@jeu{Jonkor Bourmataguil}
\def\langnames@langs@wals@jgb{Ngbee}
\def\langnames@langs@wals@jgo{Ngomba}
\def\langnames@langs@wals@jhi{Jehai}
\def\langnames@langs@wals@jhs{Jhankot Sign Language}
\def\langnames@langs@wals@jia{Jina}
\def\langnames@langs@wals@jib{Jibu}
\def\langnames@langs@wals@jic{Tol}
\def\langnames@langs@wals@jid{Bu}
\def\langnames@langs@wals@jie{Jilbe}
\def\langnames@langs@wals@jig{Jingulu}
\def\langnames@langs@wals@jih{Stodsde}
\def\langnames@langs@wals@jii{Jiiddu}
\def\langnames@langs@wals@jil{Jilim}
\def\langnames@langs@wals@jim{Jimi (Cameroon)}
\def\langnames@langs@wals@jio{Jiamao}
\def\langnames@langs@wals@jiq{Khroskyabs}
\def\langnames@langs@wals@jit{Jita}
\def\langnames@langs@wals@jiu{Youle Jinuo}
\def\langnames@langs@wals@jiv{Shuar}
\def\langnames@langs@wals@jiy{Buyuan Jinuo}
\def\langnames@langs@wals@jje{Jejueo}
\def\langnames@langs@wals@jka{Kaera}
\def\langnames@langs@wals@jkm{Mobwa Karen}
\def\langnames@langs@wals@jko{Kubo}
\def\langnames@langs@wals@jkp{Paku Karen}
\def\langnames@langs@wals@jkr{Koro}
\def\langnames@langs@wals@jks{Amami O Shima Sign Language}
\def\langnames@langs@wals@jku{Labir}
\def\langnames@langs@wals@jle{Ngile}
\def\langnames@langs@wals@jls{Jamaican Sign Language}
\def\langnames@langs@wals@jma{Dima}
\def\langnames@langs@wals@jmb{Zumbun}
\def\langnames@langs@wals@jmc{Machame}
\def\langnames@langs@wals@jmd{Yamdena}
\def\langnames@langs@wals@jmi{Jimi (Nigeria)}
\def\langnames@langs@wals@jml{Jumli}
\def\langnames@langs@wals@jmn{Makuri Naga}
\def\langnames@langs@wals@jmr{Kamara}
\def\langnames@langs@wals@jms{Mashi (Nigeria)}
\def\langnames@langs@wals@jmw{Mouwase}
\def\langnames@langs@wals@jmx{Western Juxtlahuaca Mixtec}
\def\langnames@langs@wals@jna{Jangshung}
\def\langnames@langs@wals@jnd{Jandavra}
\def\langnames@langs@wals@jng{Yangman}
\def\langnames@langs@wals@jni{Janji}
\def\langnames@langs@wals@jnj{Yemsa}
\def\langnames@langs@wals@jnl{Rawat}
\def\langnames@langs@wals@jns{Jaunsari}
\def\langnames@langs@wals@job{Joba}
\def\langnames@langs@wals@jod{Wojenaka}
\def\langnames@langs@wals@jor{Jorá}
\def\langnames@langs@wals@jos{Levantine Arabic Sign Language}
\def\langnames@langs@wals@jow{Jowulu}
\def\langnames@langs@wals@jpn{Japanese}
\def\langnames@langs@wals@jpr{Judeo-Persian}
\def\langnames@langs@wals@jqr{Jaqaru}
\def\langnames@langs@wals@jra{Jarai}
\def\langnames@langs@wals@jrr{Jiru}
\def\langnames@langs@wals@jrt{Jakattoe}
\def\langnames@langs@wals@jru{Japrería}
\def\langnames@langs@wals@jsl{Japanese Sign Language}
\def\langnames@langs@wals@jua{Júma}
\def\langnames@langs@wals@jub{Wannu}
\def\langnames@langs@wals@juc{Jurchen}
\def\langnames@langs@wals@jud{Worodougou}
\def\langnames@langs@wals@juh{Hõne}
\def\langnames@langs@wals@jui{Ngadjuri}
\def\langnames@langs@wals@juk{Wapan}
\def\langnames@langs@wals@jul{Jirel}
\def\langnames@langs@wals@jum{Jumjum}
\def\langnames@langs@wals@jun{Juang}
\def\langnames@langs@wals@juo{Jiba}
\def\langnames@langs@wals@jup{Hup}
\def\langnames@langs@wals@jur{Jurúna}
\def\langnames@langs@wals@jus{Jumla Sign Language}
\def\langnames@langs@wals@jut{Jutish}
\def\langnames@langs@wals@juu{Ju}
\def\langnames@langs@wals@juw{Wãpha}
\def\langnames@langs@wals@juy{Juray}
\def\langnames@langs@wals@jvd{Javindo}
\def\langnames@langs@wals@jvn{Caribbean Javanese}
\def\langnames@langs@wals@jwi{Jwira-Pepesa}
\def\langnames@langs@wals@jya{Jiarong}
\def\langnames@langs@wals@jye{Judeo-Yemeni Arabic}
\def\langnames@langs@wals@jyy{Jaya}
\def\langnames@langs@wals@kaa{Kara-Kalpak}
\def\langnames@langs@wals@kab{Kabyle}
\def\langnames@langs@wals@kac{Southern Jinghpaw}
\def\langnames@langs@wals@kad{Kadara}
\def\langnames@langs@wals@kae{Ketangalan}
\def\langnames@langs@wals@kaf{Katso}
\def\langnames@langs@wals@kag{Kajaman}
\def\langnames@langs@wals@kah{Kara (Central African Republic)}
\def\langnames@langs@wals@kai{Karekare}
\def\langnames@langs@wals@kaj{Jju}
\def\langnames@langs@wals@kak{Ahin-Kayapa Kalanguya}
\def\langnames@langs@wals@kal{Kalaallisut}
\def\langnames@langs@wals@kam{Kamba (Kenya)}
\def\langnames@langs@wals@kan{Kannada}
\def\langnames@langs@wals@kao{Xaasongaxango}
\def\langnames@langs@wals@kap{Bezhta}
\def\langnames@langs@wals@kaq{Capanahua}
\def\langnames@langs@wals@kas{Kashmiri}
\def\langnames@langs@wals@kat{Georgian}
\def\langnames@langs@wals@kaw{Kawi}
\def\langnames@langs@wals@kax{Kao}
\def\langnames@langs@wals@kay{Kamayurá}
\def\langnames@langs@wals@kaz{Kazakh}
\def\langnames@langs@wals@kba{Kalarko-Mirniny}
\def\langnames@langs@wals@kbb{Kaxuiâna}
\def\langnames@langs@wals@kbc{Kadiwéu}
\def\langnames@langs@wals@kbd{Kabardian}
\def\langnames@langs@wals@kbe{Kanju}
\def\langnames@langs@wals@kbg{Khamba}
\def\langnames@langs@wals@kbh{Camsá}
\def\langnames@langs@wals@kbi{Kaptiau}
\def\langnames@langs@wals@kbj{Kari}
\def\langnames@langs@wals@kbk{Grass Koiari}
\def\langnames@langs@wals@kbl{Kanembu}
\def\langnames@langs@wals@kbm{Iwal}
\def\langnames@langs@wals@kbn{Kare (Central African Republic)}
\def\langnames@langs@wals@kbo{Keliko}
\def\langnames@langs@wals@kbp{Kabiyé}
\def\langnames@langs@wals@kbq{Kamano}
\def\langnames@langs@wals@kbr{Kafa}
\def\langnames@langs@wals@kbs{Kande}
\def\langnames@langs@wals@kbt{Abadi}
\def\langnames@langs@wals@kbu{Kabutra}
\def\langnames@langs@wals@kbv{Dera (Indonesia)}
\def\langnames@langs@wals@kbw{Kaiep}
\def\langnames@langs@wals@kbx{Ap Ma}
\def\langnames@langs@wals@kby{Manga Kanuri}
\def\langnames@langs@wals@kbz{Duhwa}
\def\langnames@langs@wals@kca{Kazym-Berezover-Suryskarer Khanty}
\def\langnames@langs@wals@kcb{Kawacha}
\def\langnames@langs@wals@kcc{Lubila}
\def\langnames@langs@wals@kcd{Ngkontar Ngkolmpu}
\def\langnames@langs@wals@kce{Kaivi}
\def\langnames@langs@wals@kcf{Ukaan}
\def\langnames@langs@wals@kcg{Tyap}
\def\langnames@langs@wals@kch{Vono}
\def\langnames@langs@wals@kci{Kamantan}
\def\langnames@langs@wals@kcj{Kobiana}
\def\langnames@langs@wals@kck{Kalanga}
\def\langnames@langs@wals@kcl{Kela (Papua New Guinea)}
\def\langnames@langs@wals@kcm{Gula (Central African Republic)}
\def\langnames@langs@wals@kcn{Nubi}
\def\langnames@langs@wals@kco{Kinalakna}
\def\langnames@langs@wals@kcp{Kanga}
\def\langnames@langs@wals@kcq{Kamo}
\def\langnames@langs@wals@kcr{Katla}
\def\langnames@langs@wals@kcs{Koenoem}
\def\langnames@langs@wals@kct{Kaian}
\def\langnames@langs@wals@kcu{Kami (Tanzania)}
\def\langnames@langs@wals@kcv{Kete}
\def\langnames@langs@wals@kcw{Kabwari}
\def\langnames@langs@wals@kcx{Kachama-Ganjule-Haro}
\def\langnames@langs@wals@kcy{Korandje}
\def\langnames@langs@wals@kcz{Konongo-Ruwila}
\def\langnames@langs@wals@kda{Worimi}
\def\langnames@langs@wals@kdc{Kutu}
\def\langnames@langs@wals@kdd{Yankunytjatjara}
\def\langnames@langs@wals@kde{Makonde}
\def\langnames@langs@wals@kdf{Mamusi}
\def\langnames@langs@wals@kdg{Seba}
\def\langnames@langs@wals@kdh{Tem}
\def\langnames@langs@wals@kdi{Kumam}
\def\langnames@langs@wals@kdj{Karamojong}
\def\langnames@langs@wals@kdk{Numee}
\def\langnames@langs@wals@kdl{Tsikimba}
\def\langnames@langs@wals@kdm{Kagoma}
\def\langnames@langs@wals@kdn{Chikunda}
\def\langnames@langs@wals@kdp{Kaningdon-Nindem}
\def\langnames@langs@wals@kdq{Koch}
\def\langnames@langs@wals@kdr{Karaim}
\def\langnames@langs@wals@kdt{Kuy}
\def\langnames@langs@wals@kdu{Kadaru}
\def\langnames@langs@wals@kdv{Kado}
\def\langnames@langs@wals@kdw{Koneraw}
\def\langnames@langs@wals@kdx{Kam}
\def\langnames@langs@wals@kdy{Keder}
\def\langnames@langs@wals@kdz{Kwaja-Ndaktup}
\def\langnames@langs@wals@kea{Kabuverdianu}
\def\langnames@langs@wals@keb{Kélé}
\def\langnames@langs@wals@kec{Keiga}
\def\langnames@langs@wals@ked{Kerewe}
\def\langnames@langs@wals@kee{Eastern Keres}
\def\langnames@langs@wals@kef{Kpessi}
\def\langnames@langs@wals@keg{Tese}
\def\langnames@langs@wals@keh{Keak}
\def\langnames@langs@wals@kei{Kei}
\def\langnames@langs@wals@kej{Kadar}
\def\langnames@langs@wals@kek{Kekchí}
\def\langnames@langs@wals@kem{Kemak}
\def\langnames@langs@wals@ken{Kenyang}
\def\langnames@langs@wals@keo{Kakwa}
\def\langnames@langs@wals@kep{Kaikadi}
\def\langnames@langs@wals@keq{Kamar}
\def\langnames@langs@wals@ker{Kera}
\def\langnames@langs@wals@kes{Kugbo}
\def\langnames@langs@wals@ket{Ket}
\def\langnames@langs@wals@keu{Akebu}
\def\langnames@langs@wals@kev{Kanikkaran}
\def\langnames@langs@wals@kew{West Kewa}
\def\langnames@langs@wals@kex{Kokni}
\def\langnames@langs@wals@key{Kupia}
\def\langnames@langs@wals@kez{Kukele}
\def\langnames@langs@wals@kfa{Kodava}
\def\langnames@langs@wals@kfb{Northwestern Kolami}
\def\langnames@langs@wals@kfc{Konda-Dora}
\def\langnames@langs@wals@kfd{Korra Koraga}
\def\langnames@langs@wals@kfe{Kota (India)}
\def\langnames@langs@wals@kff{Koya}
\def\langnames@langs@wals@kfg{Kudiya}
\def\langnames@langs@wals@kfh{Kurichiya}
\def\langnames@langs@wals@kfk{Kinnauri}
\def\langnames@langs@wals@kfl{Kung}
\def\langnames@langs@wals@kfm{Khunsaric}
\def\langnames@langs@wals@kfn{Kuk}
\def\langnames@langs@wals@kfo{Koro (Côte d'Ivoire)}
\def\langnames@langs@wals@kfp{Korwa}
\def\langnames@langs@wals@kfq{Korku}
\def\langnames@langs@wals@kfr{Kachchi}
\def\langnames@langs@wals@kfs{Bilaspuri}
\def\langnames@langs@wals@kft{Kanjari}
\def\langnames@langs@wals@kfu{Katkari}
\def\langnames@langs@wals@kfv{Kurmukar}
\def\langnames@langs@wals@kfw{Kharam Naga}
\def\langnames@langs@wals@kfx{Kullu Pahari}
\def\langnames@langs@wals@kfy{Kumaoni}
\def\langnames@langs@wals@kfz{Koromfé}
\def\langnames@langs@wals@kga{Koyaga}
\def\langnames@langs@wals@kgb{Kawe}
\def\langnames@langs@wals@kge{Komering}
\def\langnames@langs@wals@kgf{Kulungtfu-Yuanggeng-Tobo}
\def\langnames@langs@wals@kgg{Kusunda}
\def\langnames@langs@wals@kgi{Selangor Sign Language}
\def\langnames@langs@wals@kgj{Gamale Kham}
\def\langnames@langs@wals@kgk{Kaiwá}
\def\langnames@langs@wals@kgl{Kunggari}
\def\langnames@langs@wals@kgn{Karingani-Kalasuri-Khoynarudi}
\def\langnames@langs@wals@kgo{Krongo}
\def\langnames@langs@wals@kgp{Kaingang}
\def\langnames@langs@wals@kgq{Kamoro}
\def\langnames@langs@wals@kgr{Abun}
\def\langnames@langs@wals@kgs{Kumbainggar}
\def\langnames@langs@wals@kgt{Somyev}
\def\langnames@langs@wals@kgu{Kobol}
\def\langnames@langs@wals@kgv{Kalamang}
\def\langnames@langs@wals@kgx{Kamaru}
\def\langnames@langs@wals@kgy{Kyerung}
\def\langnames@langs@wals@kha{Khasi}
\def\langnames@langs@wals@khb{Lü}
\def\langnames@langs@wals@khc{Tukang Besi North}
\def\langnames@langs@wals@khd{Ngkontar Baedi}
\def\langnames@langs@wals@khe{Korowai}
\def\langnames@langs@wals@khf{Khuen}
\def\langnames@langs@wals@khg{Khams Tibetan}
\def\langnames@langs@wals@khh{Kehu}
\def\langnames@langs@wals@khj{Kuturmi}
\def\langnames@langs@wals@khk{Halh Mongolian}
\def\langnames@langs@wals@khl{Lusi}
\def\langnames@langs@wals@khm{Central Khmer}
\def\langnames@langs@wals@khn{Khandesi}
\def\langnames@langs@wals@kho{Khotanese}
\def\langnames@langs@wals@khp{Kapori}
\def\langnames@langs@wals@khq{Koyra Chiini Songhay}
\def\langnames@langs@wals@khr{Kharia}
\def\langnames@langs@wals@khs{Kasua}
\def\langnames@langs@wals@kht{Khamti}
\def\langnames@langs@wals@khu{Nkhumbi}
\def\langnames@langs@wals@khv{Khwarshi-Inkhoqwari}
\def\langnames@langs@wals@khw{Khowar}
\def\langnames@langs@wals@khx{Kanu}
\def\langnames@langs@wals@khy{Kele-Foma}
\def\langnames@langs@wals@khz{Keapara}
\def\langnames@langs@wals@kia{Kim}
\def\langnames@langs@wals@kib{Koalib-Rere}
\def\langnames@langs@wals@kic{Kickapoo}
\def\langnames@langs@wals@kid{Koshin}
\def\langnames@langs@wals@kie{Kibet}
\def\langnames@langs@wals@kif{Eastern Parbate Kham}
\def\langnames@langs@wals@kig{Kimaama}
\def\langnames@langs@wals@kih{Kilmeri}
\def\langnames@langs@wals@kii{Kitsai}
\def\langnames@langs@wals@kij{Kilivila}
\def\langnames@langs@wals@kik{Kikuyu}
\def\langnames@langs@wals@kil{Kariya}
\def\langnames@langs@wals@kim{Taiga Sayan Turkic}
\def\langnames@langs@wals@kin{Kinyarwanda}
\def\langnames@langs@wals@kio{Kiowa}
\def\langnames@langs@wals@kip{Sheshi Kham}
\def\langnames@langs@wals@kiq{Kosadle}
\def\langnames@langs@wals@kir{Kirghiz}
\def\langnames@langs@wals@kis{Kis}
\def\langnames@langs@wals@kit{Agob-Ende-Kawam}
\def\langnames@langs@wals@kiu{Kirmanjki}
\def\langnames@langs@wals@kiv{Kimbu}
\def\langnames@langs@wals@kiw{Northeast Kiwai}
\def\langnames@langs@wals@kix{Khiamniungan Naga}
\def\langnames@langs@wals@kiy{Kirikiri}
\def\langnames@langs@wals@kiz{Kisi}
\def\langnames@langs@wals@kja{Mlap}
\def\langnames@langs@wals@kjb{Q'anjob'al}
\def\langnames@langs@wals@kjc{Coastal Konjo}
\def\langnames@langs@wals@kjd{Southern Kiwai}
\def\langnames@langs@wals@kje{Kisar}
\def\langnames@langs@wals@kjg{Khmu}
\def\langnames@langs@wals@kjh{Khakas}
\def\langnames@langs@wals@kji{Zabana}
\def\langnames@langs@wals@kjj{Khinalug}
\def\langnames@langs@wals@kjk{Highland Konjo}
\def\langnames@langs@wals@kjl{Western Parbate Kham}
\def\langnames@langs@wals@kjm{Kháng}
\def\langnames@langs@wals@kjn{Kunjen}
\def\langnames@langs@wals@kjo{Indo-Aryan Kinnauri}
\def\langnames@langs@wals@kjp{Pwo Eastern Karen}
\def\langnames@langs@wals@kjq{Western Keres}
\def\langnames@langs@wals@kjr{Kurudu}
\def\langnames@langs@wals@kjs{East Kewa}
\def\langnames@langs@wals@kjt{Phrae Pwo Karen}
\def\langnames@langs@wals@kju{Kashaya}
\def\langnames@langs@wals@kjv{Kajkavian}
\def\langnames@langs@wals@kjx{Ramopa}
\def\langnames@langs@wals@kjy{Erave}
\def\langnames@langs@wals@kjz{Bumthangkha}
\def\langnames@langs@wals@kka{Kakanda}
\def\langnames@langs@wals@kkb{Kwerisa}
\def\langnames@langs@wals@kkc{Odoodee}
\def\langnames@langs@wals@kkd{Kinuku}
\def\langnames@langs@wals@kke{Kakabe}
\def\langnames@langs@wals@kkf{Kalaktang Monpa}
\def\langnames@langs@wals@kkg{Mabaka Valley Kalinga}
\def\langnames@langs@wals@kkh{Khün}
\def\langnames@langs@wals@kki{Kagulu}
\def\langnames@langs@wals@kkj{Kako}
\def\langnames@langs@wals@kkk{Kokota}
\def\langnames@langs@wals@kkl{Kosarek Yale}
\def\langnames@langs@wals@kkm{Kiong}
\def\langnames@langs@wals@kko{Karko}
\def\langnames@langs@wals@kkp{Gugubera}
\def\langnames@langs@wals@kkq{Kaiku}
\def\langnames@langs@wals@kkr{Kir-Balar}
\def\langnames@langs@wals@kks{Giiwo}
\def\langnames@langs@wals@kkt{Koi}
\def\langnames@langs@wals@kku{Tumi}
\def\langnames@langs@wals@kkv{Kangean}
\def\langnames@langs@wals@kkw{Teke-Kukuya}
\def\langnames@langs@wals@kkx{Kohin}
\def\langnames@langs@wals@kky{Guugu Yimidhirr}
\def\langnames@langs@wals@kkz{Kaska}
\def\langnames@langs@wals@kla{Klamath-Modoc}
\def\langnames@langs@wals@klb{Kiliwa}
\def\langnames@langs@wals@klc{Kolbila}
\def\langnames@langs@wals@kld{Yuwaalaraay-Gamilaraay}
\def\langnames@langs@wals@kle{Kulung (Nepal)}
\def\langnames@langs@wals@klf{Kendeje}
\def\langnames@langs@wals@klg{Tagakaulu Kalagan}
\def\langnames@langs@wals@klh{Weliki}
\def\langnames@langs@wals@kli{Kalumpang}
\def\langnames@langs@wals@klj{Turkic Khalaj}
\def\langnames@langs@wals@klk{Kono (Nigeria)}
\def\langnames@langs@wals@kll{Kagan Kalagan}
\def\langnames@langs@wals@klm{Kolom}
\def\langnames@langs@wals@klo{Kapya}
\def\langnames@langs@wals@klp{Kamasa}
\def\langnames@langs@wals@klq{Rumu}
\def\langnames@langs@wals@klr{Khaling}
\def\langnames@langs@wals@kls{Chitral Kalasha}
\def\langnames@langs@wals@klt{Nukna}
\def\langnames@langs@wals@klu{Klao}
\def\langnames@langs@wals@klv{Maskelynes}
\def\langnames@langs@wals@klw{Tado-Lindu}
\def\langnames@langs@wals@klx{Koluwawa}
\def\langnames@langs@wals@kly{Kalao}
\def\langnames@langs@wals@klz{Kabola}
\def\langnames@langs@wals@kma{Konni}
\def\langnames@langs@wals@kmb{Kimbundu}
\def\langnames@langs@wals@kmc{Southern Dong}
\def\langnames@langs@wals@kmd{Madukayang Kalinga}
\def\langnames@langs@wals@kme{Bakole}
\def\langnames@langs@wals@kmf{Kare (Papua New Guinea)}
\def\langnames@langs@wals@kmg{Kâte}
\def\langnames@langs@wals@kmh{Kalam}
\def\langnames@langs@wals@kmi{Kami (Nigeria)}
\def\langnames@langs@wals@kmj{Kumarbhag Paharia}
\def\langnames@langs@wals@kmk{Limos Kalinga}
\def\langnames@langs@wals@kml{Tanudan Kalinga}
\def\langnames@langs@wals@kmm{Kom (India)}
\def\langnames@langs@wals@kmn{Awtuw}
\def\langnames@langs@wals@kmo{Kwoma}
\def\langnames@langs@wals@kmp{Gimme}
\def\langnames@langs@wals@kmq{Gwama}
\def\langnames@langs@wals@kmr{Northern Kurdish}
\def\langnames@langs@wals@kms{Kamasau}
\def\langnames@langs@wals@kmt{Kemtuik}
\def\langnames@langs@wals@kmu{Kanite}
\def\langnames@langs@wals@kmv{Uaçá Creole French}
\def\langnames@langs@wals@kmw{Komo (Democratic Republic of Congo)}
\def\langnames@langs@wals@kmx{Waboda}
\def\langnames@langs@wals@kmy{Koma Ndera}
\def\langnames@langs@wals@kmz{Khorasan Turkic}
\def\langnames@langs@wals@kna{Dera (Nigeria)}
\def\langnames@langs@wals@knb{Lubuagan Kalinga}
\def\langnames@langs@wals@knc{Central Kanuri}
\def\langnames@langs@wals@knd{Yaben (Konda)}
\def\langnames@langs@wals@kne{Kankanaey}
\def\langnames@langs@wals@knf{Mankanya}
\def\langnames@langs@wals@kng{South-Central Koongo}
\def\langnames@langs@wals@kni{Kanufi}
\def\langnames@langs@wals@knj{Akateko}
\def\langnames@langs@wals@knk{Kuranko}
\def\langnames@langs@wals@knl{Keninjal}
\def\langnames@langs@wals@knm{Katukína-Kanamarí}
\def\langnames@langs@wals@knn{Konkan Marathi}
\def\langnames@langs@wals@kno{Kono (Sierra Leone)}
\def\langnames@langs@wals@knp{Kwanja}
\def\langnames@langs@wals@knq{Kintaq}
\def\langnames@langs@wals@knr{Kaningra}
\def\langnames@langs@wals@kns{Kensiu}
\def\langnames@langs@wals@knt{Panoan Katukína}
\def\langnames@langs@wals@knu{Kono (Guinea)}
\def\langnames@langs@wals@knv{Tabo}
\def\langnames@langs@wals@knw{North-Central Ju}
\def\langnames@langs@wals@knx{Kendayan-Belangin}
\def\langnames@langs@wals@kny{Kanyok}
\def\langnames@langs@wals@knz{Kalamsé}
\def\langnames@langs@wals@koa{Konomala}
\def\langnames@langs@wals@koc{Kpati}
\def\langnames@langs@wals@kod{Kodi-Gaura}
\def\langnames@langs@wals@koe{Kacipo-Balesi}
\def\langnames@langs@wals@kof{Kubi}
\def\langnames@langs@wals@kog{Cogui}
\def\langnames@langs@wals@koh{Koyo}
\def\langnames@langs@wals@koi{Komi-Permyak}
\def\langnames@langs@wals@kol{Kol (Papua New Guinea)}
\def\langnames@langs@wals@koo{Konzo}
\def\langnames@langs@wals@kop{Kwato}
\def\langnames@langs@wals@koq{Kota (Gabon)}
\def\langnames@langs@wals@kor{Korean}
\def\langnames@langs@wals@kos{Kosraean}
\def\langnames@langs@wals@kot{Lagwan}
\def\langnames@langs@wals@kou{Koke}
\def\langnames@langs@wals@kov{Kudu-Camo}
\def\langnames@langs@wals@kow{Gengle-Kugama}
\def\langnames@langs@wals@koy{Koyukon}
\def\langnames@langs@wals@koz{Korak}
\def\langnames@langs@wals@kpa{Kutto}
\def\langnames@langs@wals@kpb{Mullu Kurumba}
\def\langnames@langs@wals@kpc{Curripaco}
\def\langnames@langs@wals@kpd{Koba}
\def\langnames@langs@wals@kpf{Komba}
\def\langnames@langs@wals@kpg{Kapingamarangi}
\def\langnames@langs@wals@kph{Kplang}
\def\langnames@langs@wals@kpi{Kofei}
\def\langnames@langs@wals@kpj{Karajá}
\def\langnames@langs@wals@kpk{Kpan}
\def\langnames@langs@wals@kpl{Kpala}
\def\langnames@langs@wals@kpm{Koho}
\def\langnames@langs@wals@kpn{Kepkiriwát}
\def\langnames@langs@wals@kpo{Ikposo}
\def\langnames@langs@wals@kpq{Korupun-Sela}
\def\langnames@langs@wals@kpr{Korafe-Yegha}
\def\langnames@langs@wals@kps{Tehit}
\def\langnames@langs@wals@kpt{Karata-Tukita}
\def\langnames@langs@wals@kpu{Kafoa}
\def\langnames@langs@wals@kpv{Komi-Zyrian}
\def\langnames@langs@wals@kpw{Kobon}
\def\langnames@langs@wals@kpx{Mountain Koiali}
\def\langnames@langs@wals@kpy{Koryak}
\def\langnames@langs@wals@kpz{Kupsabiny}
\def\langnames@langs@wals@kqa{Mum}
\def\langnames@langs@wals@kqb{Kovai}
\def\langnames@langs@wals@kqc{Doromu-Koki}
\def\langnames@langs@wals@kqd{Koy Sanjaq Jewish Neo-Aramaic}
\def\langnames@langs@wals@kqe{Kalagan}
\def\langnames@langs@wals@kqf{Kakabai}
\def\langnames@langs@wals@kqg{Khe}
\def\langnames@langs@wals@kqi{Koitabu}
\def\langnames@langs@wals@kqj{Koromira}
\def\langnames@langs@wals@kqk{Kotafon Gbe}
\def\langnames@langs@wals@kql{Kyenele}
\def\langnames@langs@wals@kqm{Khisa}
\def\langnames@langs@wals@kqn{Kaonde}
\def\langnames@langs@wals@kqo{Konobo-Eastern Krahn}
\def\langnames@langs@wals@kqp{Kimre}
\def\langnames@langs@wals@kqq{Krenak}
\def\langnames@langs@wals@kqr{Kimaragang}
\def\langnames@langs@wals@kqs{Northern Kissi}
\def\langnames@langs@wals@kqt{Klias River Kadazan}
\def\langnames@langs@wals@kqu{Vaal-Orange}
\def\langnames@langs@wals@kqv{Okolod}
\def\langnames@langs@wals@kqw{Kandas}
\def\langnames@langs@wals@kqx{Mser}
\def\langnames@langs@wals@kqy{Koorete}
\def\langnames@langs@wals@kqz{Korana}
\def\langnames@langs@wals@kra{Kumhali}
\def\langnames@langs@wals@krb{Karkin}
\def\langnames@langs@wals@krc{Karachay-Balkar}
\def\langnames@langs@wals@krd{Kairui-Midiki}
\def\langnames@langs@wals@kre{Panará}
\def\langnames@langs@wals@krf{Koro-Olrat}
\def\langnames@langs@wals@krh{Kurama}
\def\langnames@langs@wals@kri{Krio}
\def\langnames@langs@wals@krj{Kinaray-a}
\def\langnames@langs@wals@krk{Kerek}
\def\langnames@langs@wals@krl{Karelian}
\def\langnames@langs@wals@krn{Sapo}
\def\langnames@langs@wals@krp{Korop}
\def\langnames@langs@wals@krs{Kresh-Woro}
\def\langnames@langs@wals@krt{Tumari Kanuri}
\def\langnames@langs@wals@kru{Kurukh}
\def\langnames@langs@wals@krw{Western Krahn}
\def\langnames@langs@wals@krx{Karon}
\def\langnames@langs@wals@kry{Kryz}
\def\langnames@langs@wals@krz{Sota Kanum}
\def\langnames@langs@wals@ksa{Shuwa-Zamani}
\def\langnames@langs@wals@ksb{Shambala}
\def\langnames@langs@wals@ksc{Bangad}
\def\langnames@langs@wals@ksd{Kuanua}
\def\langnames@langs@wals@kse{Kuni}
\def\langnames@langs@wals@ksf{Bafia}
\def\langnames@langs@wals@ksg{Kusaghe-Njela}
\def\langnames@langs@wals@ksh{Kölsch}
\def\langnames@langs@wals@ksi{I'saka}
\def\langnames@langs@wals@ksj{Uare}
\def\langnames@langs@wals@ksk{Kansa}
\def\langnames@langs@wals@ksl{Kumalu}
\def\langnames@langs@wals@ksm{Kumba}
\def\langnames@langs@wals@ksn{Kasiguranin}
\def\langnames@langs@wals@ksp{Kaba}
\def\langnames@langs@wals@ksq{Kwaami}
\def\langnames@langs@wals@ksr{Borong}
\def\langnames@langs@wals@kss{Southern Kisi}
\def\langnames@langs@wals@kst{Winyé}
\def\langnames@langs@wals@ksu{Khamyang}
\def\langnames@langs@wals@ksv{Kusu}
\def\langnames@langs@wals@ksw{S'gaw Karen}
\def\langnames@langs@wals@ksx{Kedang}
\def\langnames@langs@wals@ksy{Kharia Thar}
\def\langnames@langs@wals@ksz{Kodaku}
\def\langnames@langs@wals@kta{Katua}
\def\langnames@langs@wals@ktb{Kambaata}
\def\langnames@langs@wals@ktc{Kholok}
\def\langnames@langs@wals@ktd{Kokata}
\def\langnames@langs@wals@kte{Gyalsumdo-Nubri}
\def\langnames@langs@wals@ktf{Kwami}
\def\langnames@langs@wals@ktg{Kalkutung}
\def\langnames@langs@wals@kth{Karanga}
\def\langnames@langs@wals@kti{North Muyu}
\def\langnames@langs@wals@ktj{Plapo Krumen}
\def\langnames@langs@wals@ktk{Kaniet}
\def\langnames@langs@wals@ktl{Koroshi}
\def\langnames@langs@wals@ktm{Kurti}
\def\langnames@langs@wals@ktn{Karitiâna}
\def\langnames@langs@wals@kto{Kuot}
\def\langnames@langs@wals@ktp{Kaduo}
\def\langnames@langs@wals@ktq{Katabaga}
\def\langnames@langs@wals@kts{South Muyu}
\def\langnames@langs@wals@ktt{Ketum}
\def\langnames@langs@wals@ktu{Kituba (Democratic Republic of Congo)}
\def\langnames@langs@wals@ktv{Eastern Katu}
\def\langnames@langs@wals@ktw{Kato}
\def\langnames@langs@wals@ktx{Kaxararí}
\def\langnames@langs@wals@kty{Kango (Bas-Uélé District)}
\def\langnames@langs@wals@ktz{South-Eastern Ju}
\def\langnames@langs@wals@kua{Kuanyama}
\def\langnames@langs@wals@kub{Kutep}
\def\langnames@langs@wals@kuc{Kwinsu}
\def\langnames@langs@wals@kud{'Auhelawa}
\def\langnames@langs@wals@kue{Kuman}
\def\langnames@langs@wals@kuf{Western Katu}
\def\langnames@langs@wals@kug{Kupa}
\def\langnames@langs@wals@kuh{Kushi}
\def\langnames@langs@wals@kui{Kuikúro-Kalapálo}
\def\langnames@langs@wals@kuj{Kuria}
\def\langnames@langs@wals@kuk{Kepo'}
\def\langnames@langs@wals@kul{Kulere}
\def\langnames@langs@wals@kum{Kumyk}
\def\langnames@langs@wals@kun{Kunama}
\def\langnames@langs@wals@kuo{Kumukio}
\def\langnames@langs@wals@kup{Kunimaipa}
\def\langnames@langs@wals@kuq{Karipúna}
\def\langnames@langs@wals@kus{Kusaal}
\def\langnames@langs@wals@kut{Kutenai}
\def\langnames@langs@wals@kuu{Upper Kuskokwim}
\def\langnames@langs@wals@kuv{Kur}
\def\langnames@langs@wals@kuw{Kpagua}
\def\langnames@langs@wals@kux{Kukatja}
\def\langnames@langs@wals@kuy{Kuuku-Ya'u}
\def\langnames@langs@wals@kuz{Kunza}
\def\langnames@langs@wals@kva{Bagvalal}
\def\langnames@langs@wals@kvb{Kubu}
\def\langnames@langs@wals@kvc{Kove}
\def\langnames@langs@wals@kvd{Kui (Indonesia)}
\def\langnames@langs@wals@kve{Kalabakan}
\def\langnames@langs@wals@kvf{Kabalai}
\def\langnames@langs@wals@kvg{Kuni-Boazi}
\def\langnames@langs@wals@kvh{Komodo}
\def\langnames@langs@wals@kvi{Kwang}
\def\langnames@langs@wals@kvj{Psikye}
\def\langnames@langs@wals@kvk{Korean Sign Language}
\def\langnames@langs@wals@kvl{Brek Karen}
\def\langnames@langs@wals@kvm{Kendem}
\def\langnames@langs@wals@kvn{Border Kuna}
\def\langnames@langs@wals@kvo{Dobel}
\def\langnames@langs@wals@kvp{Kompane}
\def\langnames@langs@wals@kvq{Geba Karen}
\def\langnames@langs@wals@kvr{Kerinci}
\def\langnames@langs@wals@kvu{Yinbaw Karen}
\def\langnames@langs@wals@kvv{Kola}
\def\langnames@langs@wals@kvw{Wersing}
\def\langnames@langs@wals@kvx{Parkari Koli}
\def\langnames@langs@wals@kvy{Yintale Karen}
\def\langnames@langs@wals@kvz{Tsaukambo}
\def\langnames@langs@wals@kwa{Dâw}
\def\langnames@langs@wals@kwb{Baa}
\def\langnames@langs@wals@kwc{Likwala}
\def\langnames@langs@wals@kwd{Kwaio}
\def\langnames@langs@wals@kwe{Kwerba}
\def\langnames@langs@wals@kwf{Kwara'ae}
\def\langnames@langs@wals@kwg{Sara Kaba Deme}
\def\langnames@langs@wals@kwh{Kowiai}
\def\langnames@langs@wals@kwi{Awa-Cuaiquer}
\def\langnames@langs@wals@kwj{Kwanga}
\def\langnames@langs@wals@kwk{Kwak'wala}
\def\langnames@langs@wals@kwl{Pan}
\def\langnames@langs@wals@kwm{Kwambi}
\def\langnames@langs@wals@kwn{Kwangali}
\def\langnames@langs@wals@kwo{Kwomtari}
\def\langnames@langs@wals@kwp{Kodia}
\def\langnames@langs@wals@kwr{Kwer}
\def\langnames@langs@wals@kws{Kwese}
\def\langnames@langs@wals@kwt{Kwesten}
\def\langnames@langs@wals@kwu{Kwakum}
\def\langnames@langs@wals@kwv{Sara Kaba Náà}
\def\langnames@langs@wals@kww{Kwinti}
\def\langnames@langs@wals@kwx{Khirwar}
\def\langnames@langs@wals@kwy{San Salvador Kongo}
\def\langnames@langs@wals@kwz{Kwadi}
\def\langnames@langs@wals@kxa{Kairiru}
\def\langnames@langs@wals@kxb{Krobu}
\def\langnames@langs@wals@kxc{Konso}
\def\langnames@langs@wals@kxd{Brunei}
\def\langnames@langs@wals@kxf{Manumanaw Karen}
\def\langnames@langs@wals@kxh{Karo (Ethiopia)}
\def\langnames@langs@wals@kxi{Keningau Murut}
\def\langnames@langs@wals@kxj{Kulfa}
\def\langnames@langs@wals@kxk{Lahta-Zayein Karen}
\def\langnames@langs@wals@kxm{Northern Khmer}
\def\langnames@langs@wals@kxn{Kanowit-Tanjong Melanau}
\def\langnames@langs@wals@kxo{Kanoê}
\def\langnames@langs@wals@kxp{Wadiyara Koli}
\def\langnames@langs@wals@kxq{Smärky Kanum}
\def\langnames@langs@wals@kxr{Manus Koro}
\def\langnames@langs@wals@kxs{Kangjia}
\def\langnames@langs@wals@kxt{Koiwat}
\def\langnames@langs@wals@kxu{Kui (India)}
\def\langnames@langs@wals@kxv{Kuvi}
\def\langnames@langs@wals@kxw{Konai}
\def\langnames@langs@wals@kxx{Likuba}
\def\langnames@langs@wals@kxy{Kayong}
\def\langnames@langs@wals@kxz{Kerewo}
\def\langnames@langs@wals@kya{Kwaya}
\def\langnames@langs@wals@kyb{Butbut Kalinga}
\def\langnames@langs@wals@kyc{Kyaka}
\def\langnames@langs@wals@kyd{Karey}
\def\langnames@langs@wals@kye{Krache}
\def\langnames@langs@wals@kyf{Kouya}
\def\langnames@langs@wals@kyg{Keyagana}
\def\langnames@langs@wals@kyh{Karok}
\def\langnames@langs@wals@kyi{Kiput}
\def\langnames@langs@wals@kyj{Karao}
\def\langnames@langs@wals@kyk{Kamayo}
\def\langnames@langs@wals@kyl{Central Kalapuya}
\def\langnames@langs@wals@kyn{Northern Binukidnon}
\def\langnames@langs@wals@kyo{Klon}
\def\langnames@langs@wals@kyq{Kenga}
\def\langnames@langs@wals@kyr{Kuruáya}
\def\langnames@langs@wals@kys{Baram Kayan}
\def\langnames@langs@wals@kyt{Kayagar}
\def\langnames@langs@wals@kyu{Western Kayah}
\def\langnames@langs@wals@kyw{Kudmali}
\def\langnames@langs@wals@kyx{Rapoisi}
\def\langnames@langs@wals@kyy{Kambaira}
\def\langnames@langs@wals@kyz{Kayabí}
\def\langnames@langs@wals@kza{Western Karaboro}
\def\langnames@langs@wals@kzb{Kaibobo}
\def\langnames@langs@wals@kzc{Bondoukou Kulango}
\def\langnames@langs@wals@kzd{Kadai}
\def\langnames@langs@wals@kzf{Da'a Kaili}
\def\langnames@langs@wals@kzg{Kikai}
\def\langnames@langs@wals@kzh{Kenuzi-Dongola}
\def\langnames@langs@wals@kzi{Kelabit}
\def\langnames@langs@wals@kzj{Kadazan}
\def\langnames@langs@wals@kzk{Kazukuru}
\def\langnames@langs@wals@kzl{Kayeli}
\def\langnames@langs@wals@kzm{Kais}
\def\langnames@langs@wals@kzn{Kokola}
\def\langnames@langs@wals@kzo{Kaningi}
\def\langnames@langs@wals@kzp{Kaidipang}
\def\langnames@langs@wals@kzq{Kaike}
\def\langnames@langs@wals@kzr{Karang}
\def\langnames@langs@wals@kzs{Sugut Dusun}
\def\langnames@langs@wals@kzu{Kayupulau}
\def\langnames@langs@wals@kzv{Komyandaret}
\def\langnames@langs@wals@kzw{Karirí-Xocó}
\def\langnames@langs@wals@kzx{Kamarian}
\def\langnames@langs@wals@kzy{Kango (Tshopo District)}
\def\langnames@langs@wals@kzz{Kalabra}
\def\langnames@langs@wals@laa{Lapuyan Subanun}
\def\langnames@langs@wals@lac{Lacandon}
\def\langnames@langs@wals@lad{Ladino}
\def\langnames@langs@wals@lae{Pattani}
\def\langnames@langs@wals@laf{Lafofa}
\def\langnames@langs@wals@lag{Langi}
\def\langnames@langs@wals@lai{Lambya}
\def\langnames@langs@wals@laj{Lango (Uganda)}
\def\langnames@langs@wals@lam{Lamba}
\def\langnames@langs@wals@lan{Laru (Nigeria)}
\def\langnames@langs@wals@lao{Lao}
\def\langnames@langs@wals@lap{Laka (Chad)}
\def\langnames@langs@wals@laq{Pubiao-Qabiao}
\def\langnames@langs@wals@lar{Larteh}
\def\langnames@langs@wals@las{Lama (Togo)}
\def\langnames@langs@wals@lat{Latin}
\def\langnames@langs@wals@lav{Latvian}
\def\langnames@langs@wals@law{Lauje}
\def\langnames@langs@wals@lax{Tiwa (India)}
\def\langnames@langs@wals@laz{Aribwatsa}
\def\langnames@langs@wals@lbb{Label}
\def\langnames@langs@wals@lbc{Lakkia}
\def\langnames@langs@wals@lbe{Lak}
\def\langnames@langs@wals@lbf{Tinani}
\def\langnames@langs@wals@lbi{La'bi}
\def\langnames@langs@wals@lbj{Leh Ladakhi}
\def\langnames@langs@wals@lbk{Central Bontoc}
\def\langnames@langs@wals@lbm{Lodhi}
\def\langnames@langs@wals@lbn{Lamet}
\def\langnames@langs@wals@lbo{Laven}
\def\langnames@langs@wals@lbq{Wampar}
\def\langnames@langs@wals@lbr{Lohorung}
\def\langnames@langs@wals@lbs{Libyan Sign Language}
\def\langnames@langs@wals@lbt{Lachi}
\def\langnames@langs@wals@lbu{Labu}
\def\langnames@langs@wals@lbv{Lavatbura-Lamusong}
\def\langnames@langs@wals@lbw{Tolaki}
\def\langnames@langs@wals@lbx{Lawangan}
\def\langnames@langs@wals@lby{Lamalama}
\def\langnames@langs@wals@lbz{Lardil}
\def\langnames@langs@wals@lcc{Legenyem}
\def\langnames@langs@wals@lcd{Lola}
\def\langnames@langs@wals@lce{Loncong}
\def\langnames@langs@wals@lcf{Lubu}
\def\langnames@langs@wals@lch{Luchazi}
\def\langnames@langs@wals@lcl{Lisela}
\def\langnames@langs@wals@lcm{Tungag}
\def\langnames@langs@wals@lcp{Western Lawa}
\def\langnames@langs@wals@lcq{Luhu-Piru}
\def\langnames@langs@wals@lcs{Lisabata-Nuniali}
\def\langnames@langs@wals@ldb{Dũya}
\def\langnames@langs@wals@ldd{Luri}
\def\langnames@langs@wals@ldg{Lenyima}
\def\langnames@langs@wals@ldh{Lamja-Dengsa-Tola}
\def\langnames@langs@wals@ldi{Laari}
\def\langnames@langs@wals@ldj{Lemoro}
\def\langnames@langs@wals@ldk{Leelau}
\def\langnames@langs@wals@ldl{Kaan}
\def\langnames@langs@wals@ldm{Landoma}
\def\langnames@langs@wals@ldn{Láadan}
\def\langnames@langs@wals@ldo{Loo}
\def\langnames@langs@wals@ldp{Tso}
\def\langnames@langs@wals@ldq{Lufu}
\def\langnames@langs@wals@lea{Lega-Shabunda}
\def\langnames@langs@wals@leb{Lala-Bisa}
\def\langnames@langs@wals@lec{Leco}
\def\langnames@langs@wals@led{Lendu}
\def\langnames@langs@wals@lee{Lyélé}
\def\langnames@langs@wals@lef{Lelemi}
\def\langnames@langs@wals@leh{Lenje}
\def\langnames@langs@wals@lei{Lemio}
\def\langnames@langs@wals@lej{Lengola}
\def\langnames@langs@wals@lek{Leipon}
\def\langnames@langs@wals@lel{Lele (Democratic Republic of Congo)}
\def\langnames@langs@wals@lem{Nomaande}
\def\langnames@langs@wals@leo{Leti (Cameroon)}
\def\langnames@langs@wals@lep{Lepcha}
\def\langnames@langs@wals@leq{Lembena}
\def\langnames@langs@wals@ler{Lenkau}
\def\langnames@langs@wals@les{Lese}
\def\langnames@langs@wals@let{Lesing-Gelimi}
\def\langnames@langs@wals@leu{Kara (Papua New Guinea)}
\def\langnames@langs@wals@lev{Western Pantar}
\def\langnames@langs@wals@lew{Ledo Kaili}
\def\langnames@langs@wals@lex{Luang}
\def\langnames@langs@wals@ley{Lemolang}
\def\langnames@langs@wals@lez{Lezgian}
\def\langnames@langs@wals@lfa{Lefa}
\def\langnames@langs@wals@lfn{Lingua Franca Nova}
\def\langnames@langs@wals@lga{Lungga}
\def\langnames@langs@wals@lgb{Laghu}
\def\langnames@langs@wals@lgg{Lugbara}
\def\langnames@langs@wals@lgh{Laghuu}
\def\langnames@langs@wals@lgi{Lengilu}
\def\langnames@langs@wals@lgk{Neverver}
\def\langnames@langs@wals@lgl{Wala}
\def\langnames@langs@wals@lgm{Lega-Mwenga}
\def\langnames@langs@wals@lgn{Opo}
\def\langnames@langs@wals@lgq{Ikpana}
\def\langnames@langs@wals@lgr{Lengo}
\def\langnames@langs@wals@lgt{Pahi}
\def\langnames@langs@wals@lgu{Longgu}
\def\langnames@langs@wals@lgz{Ligenza}
\def\langnames@langs@wals@lha{Laha (Viet Nam)}
\def\langnames@langs@wals@lhh{Laha (Indonesia)}
\def\langnames@langs@wals@lhi{Lahu Shi}
\def\langnames@langs@wals@lhl{Lahul Lohar}
\def\langnames@langs@wals@lhm{Lhomi}
\def\langnames@langs@wals@lhn{Lahanan}
\def\langnames@langs@wals@lhp{Lhokpu}
\def\langnames@langs@wals@lhs{Mlahsô}
\def\langnames@langs@wals@lht{Lo-Toga}
\def\langnames@langs@wals@lhu{Lahu}
\def\langnames@langs@wals@lia{West-Central Limba}
\def\langnames@langs@wals@lib{Likum}
\def\langnames@langs@wals@lic{Hlai}
\def\langnames@langs@wals@lid{Nyindrou}
\def\langnames@langs@wals@lie{Balobo}
\def\langnames@langs@wals@lif{Limbu}
\def\langnames@langs@wals@lig{Ligbi}
\def\langnames@langs@wals@lih{Lihir}
\def\langnames@langs@wals@lij{Ligurian}
\def\langnames@langs@wals@lik{Liko}
\def\langnames@langs@wals@lil{Lillooet}
\def\langnames@langs@wals@lim{Limburgan}
\def\langnames@langs@wals@lin{Kinshasa Lingala}
\def\langnames@langs@wals@lio{Liki}
\def\langnames@langs@wals@lip{Sekpele}
\def\langnames@langs@wals@liq{Libido}
\def\langnames@langs@wals@lir{Kru Pidgin English}
\def\langnames@langs@wals@lis{Lisu}
\def\langnames@langs@wals@lit{Lithuanian}
\def\langnames@langs@wals@liu{Logorik}
\def\langnames@langs@wals@liv{Liv}
\def\langnames@langs@wals@liw{Col}
\def\langnames@langs@wals@lix{Liabuku}
\def\langnames@langs@wals@liy{Banda-Bambari}
\def\langnames@langs@wals@liz{Libinza}
\def\langnames@langs@wals@lje{Rampi}
\def\langnames@langs@wals@lji{Laiyolo}
\def\langnames@langs@wals@ljl{Li'o}
\def\langnames@langs@wals@ljp{Lampung Api}
\def\langnames@langs@wals@ljw{Yirandhali}
\def\langnames@langs@wals@ljx{Yuru}
\def\langnames@langs@wals@lka{Lakalei}
\def\langnames@langs@wals@lkb{Kabras}
\def\langnames@langs@wals@lkc{Kucong}
\def\langnames@langs@wals@lkd{Lakondê}
\def\langnames@langs@wals@lke{Kenyi}
\def\langnames@langs@wals@lkh{Lakha}
\def\langnames@langs@wals@lki{Laki}
\def\langnames@langs@wals@lkj{Remun}
\def\langnames@langs@wals@lkl{Laeko-Libuat}
\def\langnames@langs@wals@lkm{Kalaamaya}
\def\langnames@langs@wals@lkn{Lakon}
\def\langnames@langs@wals@lko{Khayo}
\def\langnames@langs@wals@lkr{Päri}
\def\langnames@langs@wals@lks{Kisa}
\def\langnames@langs@wals@lkt{Lakota}
\def\langnames@langs@wals@lku{Kuungkari of Barcoo River}
\def\langnames@langs@wals@lky{Lokoya}
\def\langnames@langs@wals@lla{Lala-Roba}
\def\langnames@langs@wals@llb{Lolo}
\def\langnames@langs@wals@llc{Lele (Guinea)}
\def\langnames@langs@wals@lld{Ladin}
\def\langnames@langs@wals@lle{Lele (Papua New Guinea)}
\def\langnames@langs@wals@llf{Hermit}
\def\langnames@langs@wals@llg{Lole}
\def\langnames@langs@wals@llh{Lamu}
\def\langnames@langs@wals@lli{Teke-Laali}
\def\langnames@langs@wals@llk{Lelak}
\def\langnames@langs@wals@lll{Lilau}
\def\langnames@langs@wals@llm{Lasalimu}
\def\langnames@langs@wals@lln{Lele (Chad)}
\def\langnames@langs@wals@llp{North Efate}
\def\langnames@langs@wals@llq{Lolak}
\def\langnames@langs@wals@lls{Lithuanian Sign Language}
\def\langnames@langs@wals@llu{Lau}
\def\langnames@langs@wals@llx{Lauan}
\def\langnames@langs@wals@lma{East Limba}
\def\langnames@langs@wals@lmb{Merei}
\def\langnames@langs@wals@lmc{Limilngan}
\def\langnames@langs@wals@lmd{Lumun}
\def\langnames@langs@wals@lme{Peve}
\def\langnames@langs@wals@lmf{Eastern Atadei}
\def\langnames@langs@wals@lmg{Lamogai}
\def\langnames@langs@wals@lmi{Lombi}
\def\langnames@langs@wals@lmj{West Lembata}
\def\langnames@langs@wals@lmk{Lamkang}
\def\langnames@langs@wals@lml{Hano}
\def\langnames@langs@wals@lmn{Lambadi}
\def\langnames@langs@wals@lmo{Lombard}
\def\langnames@langs@wals@lmp{Limbum}
\def\langnames@langs@wals@lmq{Lamatuka}
\def\langnames@langs@wals@lmr{Peripheral Lembata}
\def\langnames@langs@wals@lmu{Lamenu}
\def\langnames@langs@wals@lmv{Lomaiviti}
\def\langnames@langs@wals@lmw{Lake Miwok}
\def\langnames@langs@wals@lmx{Laimbue}
\def\langnames@langs@wals@lmy{Lamboya}
\def\langnames@langs@wals@lmz{Lumbee}
\def\langnames@langs@wals@lna{Langbashe}
\def\langnames@langs@wals@lnb{Central Wambo}
\def\langnames@langs@wals@lnd{Lundayeh}
\def\langnames@langs@wals@lnh{Lanoh}
\def\langnames@langs@wals@lni{Daantanai'}
\def\langnames@langs@wals@lnj{Linngithigh}
\def\langnames@langs@wals@lnl{South Central Banda}
\def\langnames@langs@wals@lnm{Pondi}
\def\langnames@langs@wals@lnn{Nethalp}
\def\langnames@langs@wals@lno{Lango-Logire-Logir}
\def\langnames@langs@wals@lns{Lamnso'}
\def\langnames@langs@wals@lnu{Longuda}
\def\langnames@langs@wals@loa{Loloda-Laba}
\def\langnames@langs@wals@lob{Lobi}
\def\langnames@langs@wals@loc{Inonhan}
\def\langnames@langs@wals@loe{Saluan}
\def\langnames@langs@wals@lof{Logol}
\def\langnames@langs@wals@log{Logo}
\def\langnames@langs@wals@loh{Narim}
\def\langnames@langs@wals@loi{Loma (Côte d'Ivoire)}
\def\langnames@langs@wals@loj{Lou}
\def\langnames@langs@wals@lok{Loko}
\def\langnames@langs@wals@lol{Mongo}
\def\langnames@langs@wals@lom{Loma (Liberia)}
\def\langnames@langs@wals@lon{Malawi Lomwe}
\def\langnames@langs@wals@loo{Lombo}
\def\langnames@langs@wals@lop{Lopa}
\def\langnames@langs@wals@loq{Lobala}
\def\langnames@langs@wals@lor{Téén}
\def\langnames@langs@wals@los{Loniu}
\def\langnames@langs@wals@lot{Otuho}
\def\langnames@langs@wals@lou{Louisiana Creole French}
\def\langnames@langs@wals@low{Tampias Lobu}
\def\langnames@langs@wals@lox{Loun}
\def\langnames@langs@wals@loy{Lowa}
\def\langnames@langs@wals@loz{Lozi}
\def\langnames@langs@wals@lpa{Lelepa}
\def\langnames@langs@wals@lpe{Lepki}
\def\langnames@langs@wals@lpn{Long Phuri Naga}
\def\langnames@langs@wals@lpo{Lipo}
\def\langnames@langs@wals@lpx{Lopit}
\def\langnames@langs@wals@lra{Rara Bakati'}
\def\langnames@langs@wals@lrc{Northern Luri}
\def\langnames@langs@wals@lre{Laurentian}
\def\langnames@langs@wals@lrg{Laragia}
\def\langnames@langs@wals@lri{Marachi}
\def\langnames@langs@wals@lrl{Larestani}
\def\langnames@langs@wals@lrm{Marama}
\def\langnames@langs@wals@lrn{Lorang}
\def\langnames@langs@wals@lro{Laru (North Sudan)}
\def\langnames@langs@wals@lrr{Southern Yamphu}
\def\langnames@langs@wals@lrt{Larantuka Malay}
\def\langnames@langs@wals@lrv{Larevat}
\def\langnames@langs@wals@lrz{Lemerig}
\def\langnames@langs@wals@lsa{Lasgerdi}
\def\langnames@langs@wals@lsc{Albarradas Sign Language}
\def\langnames@langs@wals@lsd{Lishana Deni}
\def\langnames@langs@wals@lse{Lusengo}
\def\langnames@langs@wals@lsh{Khispi}
\def\langnames@langs@wals@lsi{Lashi}
\def\langnames@langs@wals@lsl{Latvian Sign Language}
\def\langnames@langs@wals@lsm{Saamia}
\def\langnames@langs@wals@lsn{Tibetan Sign Language}
\def\langnames@langs@wals@lsp{Panamanian Sign Language}
\def\langnames@langs@wals@lsr{Srenge}
\def\langnames@langs@wals@lss{Lasi}
\def\langnames@langs@wals@lst{Trinidad and Tobago Sign Language}
\def\langnames@langs@wals@lsv{Sivia Sign Language}
\def\langnames@langs@wals@lsw{Seychelles Sign Language}
\def\langnames@langs@wals@lsy{Mauritian Sign Language}
\def\langnames@langs@wals@ltc{Middle Chinese}
\def\langnames@langs@wals@lti{Leti (Indonesia)}
\def\langnames@langs@wals@ltn{Latundê}
\def\langnames@langs@wals@lto{Tsotso}
\def\langnames@langs@wals@lts{Tachoni}
\def\langnames@langs@wals@ltu{Latu}
\def\langnames@langs@wals@ltz{Moselle Franconian}
\def\langnames@langs@wals@lua{Luba-Lulua}
\def\langnames@langs@wals@lub{Luba-Katanga}
\def\langnames@langs@wals@luc{Aringa}
\def\langnames@langs@wals@lud{Ludian}
\def\langnames@langs@wals@lue{Luvale}
\def\langnames@langs@wals@luf{Laua}
\def\langnames@langs@wals@lug{Ganda}
\def\langnames@langs@wals@lui{Luiseno-Juaneño}
\def\langnames@langs@wals@luj{Luna}
\def\langnames@langs@wals@luk{Lunanakha}
\def\langnames@langs@wals@lul{Olu'bo}
\def\langnames@langs@wals@lum{Luimbi}
\def\langnames@langs@wals@lun{Lunda}
\def\langnames@langs@wals@luo{Luo (Kenya and Tanzania)}
\def\langnames@langs@wals@lup{Lumbu}
\def\langnames@langs@wals@luq{Lucumi}
\def\langnames@langs@wals@lus{Mizo}
\def\langnames@langs@wals@lut{Northern Lushootseed}
\def\langnames@langs@wals@luv{Luwati}
\def\langnames@langs@wals@luw{Luo (Cameroon)}
\def\langnames@langs@wals@luy{Luyia}
\def\langnames@langs@wals@luz{Southern Luri}
\def\langnames@langs@wals@lva{Maku'a}
\def\langnames@langs@wals@lvi{Lawi}
\def\langnames@langs@wals@lvk{Lavukaleve}
\def\langnames@langs@wals@lvu{Central Lembata-Lewokukun}
\def\langnames@langs@wals@lwa{Lwalu}
\def\langnames@langs@wals@lwe{Lewo Eleng}
\def\langnames@langs@wals@lwg{Wanga}
\def\langnames@langs@wals@lwh{White Lachi}
\def\langnames@langs@wals@lwl{Eastern Lawa}
\def\langnames@langs@wals@lwm{Laomian}
\def\langnames@langs@wals@lwo{Luwo}
\def\langnames@langs@wals@lws{Malawian Sign Language}
\def\langnames@langs@wals@lwt{Lewotobi}
\def\langnames@langs@wals@lwu{Lawu}
\def\langnames@langs@wals@lww{Lewo}
\def\langnames@langs@wals@lxm{Lakuramau}
\def\langnames@langs@wals@lya{Layakha}
\def\langnames@langs@wals@lyg{India Lyngam}
\def\langnames@langs@wals@lyn{Luyi}
\def\langnames@langs@wals@lzh{Literary Chinese}
\def\langnames@langs@wals@lzl{Naman}
\def\langnames@langs@wals@lzn{Leinong Naga}
\def\langnames@langs@wals@lzz{Laz}
\def\langnames@langs@wals@maa{San Jerónimo Tecóatl Mazatec}
\def\langnames@langs@wals@mab{Yutanduchi Mixtec}
\def\langnames@langs@wals@mad{Madurese}
\def\langnames@langs@wals@mae{Bo-Rukul}
\def\langnames@langs@wals@maf{Mafa}
\def\langnames@langs@wals@mag{Magahi}
\def\langnames@langs@wals@mah{Marshallese}
\def\langnames@langs@wals@mai{Maithili}
\def\langnames@langs@wals@maj{Jalapa De Díaz Mazatec}
\def\langnames@langs@wals@mak{Makasar}
\def\langnames@langs@wals@mal{Malayalam}
\def\langnames@langs@wals@mam{Mam}
\def\langnames@langs@wals@maq{Chiquihuitlán Mazatec}
\def\langnames@langs@wals@mar{Marathi}
\def\langnames@langs@wals@mas{Masai}
\def\langnames@langs@wals@mat{San Francisco Matlatzinca}
\def\langnames@langs@wals@mau{Huautla Mazatec}
\def\langnames@langs@wals@mav{Sateré-Mawé}
\def\langnames@langs@wals@maw{Mampruli}
\def\langnames@langs@wals@max{North Moluccan Malay}
\def\langnames@langs@wals@maz{Central Mazahua}
\def\langnames@langs@wals@mba{Higaonon}
\def\langnames@langs@wals@mbb{Western Bukidnon Manobo}
\def\langnames@langs@wals@mbc{Macushi}
\def\langnames@langs@wals@mbd{Dibabawon Manobo}
\def\langnames@langs@wals@mbe{Molale}
\def\langnames@langs@wals@mbf{Baba Malay}
\def\langnames@langs@wals@mbh{Mangseng}
\def\langnames@langs@wals@mbi{Ilianen Manobo}
\def\langnames@langs@wals@mbj{Nadëb}
\def\langnames@langs@wals@mbk{Malol}
\def\langnames@langs@wals@mbl{Maxakalí}
\def\langnames@langs@wals@mbn{Macaguán}
\def\langnames@langs@wals@mbo{Mbo (Cameroon)}
\def\langnames@langs@wals@mbp{Malayo}
\def\langnames@langs@wals@mbq{Maisin}
\def\langnames@langs@wals@mbr{Nukak Makú}
\def\langnames@langs@wals@mbs{Sarangani Manobo}
\def\langnames@langs@wals@mbt{Matigsalug Manobo}
\def\langnames@langs@wals@mbu{Mbula-Bwazza}
\def\langnames@langs@wals@mbv{Mbulungish}
\def\langnames@langs@wals@mbw{Maring}
\def\langnames@langs@wals@mbx{Mari (East Sepik Province)}
\def\langnames@langs@wals@mby{Memoni}
\def\langnames@langs@wals@mbz{Amoltepec Mixtec}
\def\langnames@langs@wals@mca{Maca}
\def\langnames@langs@wals@mcb{Machiguenga}
\def\langnames@langs@wals@mcc{Bitur}
\def\langnames@langs@wals@mcd{Sharanahua}
\def\langnames@langs@wals@mce{Itundujia Mixtec}
\def\langnames@langs@wals@mcf{Matsés}
\def\langnames@langs@wals@mcg{Mapoyo}
\def\langnames@langs@wals@mch{Ye'kwana}
\def\langnames@langs@wals@mci{Mese}
\def\langnames@langs@wals@mcj{Mvano}
\def\langnames@langs@wals@mck{Mbunda}
\def\langnames@langs@wals@mcl{Macaguaje}
\def\langnames@langs@wals@mcm{Malacca-Batavia Portuguese Creole}
\def\langnames@langs@wals@mcn{Masana}
\def\langnames@langs@wals@mco{Coatlán Mixe}
\def\langnames@langs@wals@mcp{Makaa}
\def\langnames@langs@wals@mcq{Ese}
\def\langnames@langs@wals@mcr{Menya}
\def\langnames@langs@wals@mcs{Mambai}
\def\langnames@langs@wals@mcu{Donga Mambila}
\def\langnames@langs@wals@mcv{Minanibai-Foia Foia}
\def\langnames@langs@wals@mcw{Mawa (Chad)}
\def\langnames@langs@wals@mcx{Mpiemo}
\def\langnames@langs@wals@mcy{South Watut}
\def\langnames@langs@wals@mcz{Mawan}
\def\langnames@langs@wals@mda{Mada (Nigeria)}
\def\langnames@langs@wals@mdb{Morigi}
\def\langnames@langs@wals@mdc{Male (Papua New Guinea)}
\def\langnames@langs@wals@mdd{Mbum}
\def\langnames@langs@wals@mde{Maba (Chad)}
\def\langnames@langs@wals@mdf{Moksha}
\def\langnames@langs@wals@mdg{Massalat}
\def\langnames@langs@wals@mdh{Maguindanao}
\def\langnames@langs@wals@mdi{Mamvu}
\def\langnames@langs@wals@mdj{Mangbetu}
\def\langnames@langs@wals@mdk{Mangbutu}
\def\langnames@langs@wals@mdl{Maltese Sign Language}
\def\langnames@langs@wals@mdm{Mayogo}
\def\langnames@langs@wals@mdn{Mbati}
\def\langnames@langs@wals@mdp{Mbala}
\def\langnames@langs@wals@mdq{Mbole}
\def\langnames@langs@wals@mdr{Mandar}
\def\langnames@langs@wals@mds{Maria (Papua New Guinea)}
\def\langnames@langs@wals@mdt{Mbere-Mbamba}
\def\langnames@langs@wals@mdu{Mboko}
\def\langnames@langs@wals@mdv{Santa Lucía Monteverde Mixtec}
\def\langnames@langs@wals@mdw{Mbosi}
\def\langnames@langs@wals@mdx{Dizin}
\def\langnames@langs@wals@mdy{Male (Ethiopia)}
\def\langnames@langs@wals@mdz{Suruí Do Pará}
\def\langnames@langs@wals@mea{Menka}
\def\langnames@langs@wals@meb{Ikobi}
\def\langnames@langs@wals@mec{Marra}
\def\langnames@langs@wals@med{Melpa}
\def\langnames@langs@wals@mee{Mengen}
\def\langnames@langs@wals@mef{Bangladesh Lyngam}
\def\langnames@langs@wals@meh{Southwestern Tlaxiaco Mixtec}
\def\langnames@langs@wals@mei{Midob}
\def\langnames@langs@wals@mej{Meyah}
\def\langnames@langs@wals@mek{Mekeo}
\def\langnames@langs@wals@mel{Central Melanau}
\def\langnames@langs@wals@mem{Mangala}
\def\langnames@langs@wals@men{Mende (Sierra Leone)}
\def\langnames@langs@wals@meo{Kedah-Perak Malay}
\def\langnames@langs@wals@mep{Miriwung}
\def\langnames@langs@wals@meq{Merey}
\def\langnames@langs@wals@mer{Meru}
\def\langnames@langs@wals@mes{Masmaje}
\def\langnames@langs@wals@met{Mato}
\def\langnames@langs@wals@meu{Motu}
\def\langnames@langs@wals@mev{Mann}
\def\langnames@langs@wals@mew{Maaka}
\def\langnames@langs@wals@mey{Hassaniyya}
\def\langnames@langs@wals@mez{Menominee}
\def\langnames@langs@wals@mfa{Kelantan-Pattani Malay}
\def\langnames@langs@wals@mfb{Bangka}
\def\langnames@langs@wals@mfc{Mba}
\def\langnames@langs@wals@mfd{Mendankwe-Nkwen}
\def\langnames@langs@wals@mfe{Morisyen}
\def\langnames@langs@wals@mff{Naki}
\def\langnames@langs@wals@mfg{Mixifore}
\def\langnames@langs@wals@mfh{Matal}
\def\langnames@langs@wals@mfi{Wandala}
\def\langnames@langs@wals@mfj{Mefele}
\def\langnames@langs@wals@mfk{North Mofu}
\def\langnames@langs@wals@mfl{Putai}
\def\langnames@langs@wals@mfm{Marghi South}
\def\langnames@langs@wals@mfn{Cross River Mbembe}
\def\langnames@langs@wals@mfo{Mbe}
\def\langnames@langs@wals@mfp{Makassar Malay}
\def\langnames@langs@wals@mfq{Moba}
\def\langnames@langs@wals@mfr{Marithiel}
\def\langnames@langs@wals@mfs{Mexican Sign Language}
\def\langnames@langs@wals@mft{Mokerang}
\def\langnames@langs@wals@mfu{Mbwela}
\def\langnames@langs@wals@mfv{Mandjak}
\def\langnames@langs@wals@mfw{Mulaha}
\def\langnames@langs@wals@mfx{Melo}
\def\langnames@langs@wals@mfy{Mayo}
\def\langnames@langs@wals@mfz{Mabaan}
\def\langnames@langs@wals@mgb{Mararit}
\def\langnames@langs@wals@mgc{Morokodo}
\def\langnames@langs@wals@mgd{Moru}
\def\langnames@langs@wals@mge{Mango}
\def\langnames@langs@wals@mgf{Maklew}
\def\langnames@langs@wals@mgg{Mpongmpong}
\def\langnames@langs@wals@mgh{Makhuwa-Meetto}
\def\langnames@langs@wals@mgi{Lijili}
\def\langnames@langs@wals@mgj{Abureni}
\def\langnames@langs@wals@mgk{Mawes}
\def\langnames@langs@wals@mgl{Maleu-Kilenge}
\def\langnames@langs@wals@mgm{Mambae}
\def\langnames@langs@wals@mgn{Mbangi}
\def\langnames@langs@wals@mgo{Meta'}
\def\langnames@langs@wals@mgp{Eastern Magar}
\def\langnames@langs@wals@mgq{Malila}
\def\langnames@langs@wals@mgr{Mambwe-Lungu}
\def\langnames@langs@wals@mgs{Manda-Matumba}
\def\langnames@langs@wals@mgt{Mwakai}
\def\langnames@langs@wals@mgu{Mailu}
\def\langnames@langs@wals@mgv{Matengo}
\def\langnames@langs@wals@mgw{Matumbi}
\def\langnames@langs@wals@mgy{Mbunga}
\def\langnames@langs@wals@mgz{Mbugwe}
\def\langnames@langs@wals@mha{Manda (India)}
\def\langnames@langs@wals@mhb{Mahongwe}
\def\langnames@langs@wals@mhc{Mocho}
\def\langnames@langs@wals@mhd{Mbugu}
\def\langnames@langs@wals@mhe{Besisi}
\def\langnames@langs@wals@mhf{Mamaa}
\def\langnames@langs@wals@mhg{Margu}
\def\langnames@langs@wals@mhi{Ma'di}
\def\langnames@langs@wals@mhj{Mogholi}
\def\langnames@langs@wals@mhk{Mungaka}
\def\langnames@langs@wals@mhl{Mauwake}
\def\langnames@langs@wals@mhm{Makhuwa-Moniga}
\def\langnames@langs@wals@mhn{Mòcheno}
\def\langnames@langs@wals@mho{Mashi (Zambia)}
\def\langnames@langs@wals@mhp{Balinese Malay}
\def\langnames@langs@wals@mhq{Mandan}
\def\langnames@langs@wals@mhr{Eastern Mari}
\def\langnames@langs@wals@mhs{Buru (Indonesia)}
\def\langnames@langs@wals@mht{Mandahuaca}
\def\langnames@langs@wals@mhu{Tawra}
\def\langnames@langs@wals@mhw{Mbukushu}
\def\langnames@langs@wals@mhx{Maru}
\def\langnames@langs@wals@mhy{Ma'anyan}
\def\langnames@langs@wals@mhz{Mor (Mor Islands)}
\def\langnames@langs@wals@mia{Miami}
\def\langnames@langs@wals@mib{Atatláhuca Mixtec}
\def\langnames@langs@wals@mic{Mi'kmaq}
\def\langnames@langs@wals@mid{Neo-Mandaic}
\def\langnames@langs@wals@mie{Ocotepec Mixtec}
\def\langnames@langs@wals@mif{Mofu-Gudur}
\def\langnames@langs@wals@mig{San Miguel El Grande Mixtec}
\def\langnames@langs@wals@mih{Chayuco Mixtec}
\def\langnames@langs@wals@mii{Chigmecatitlán Mixtec}
\def\langnames@langs@wals@mij{Mungbam}
\def\langnames@langs@wals@mik{Mikasuki}
\def\langnames@langs@wals@mil{Peñoles Mixtec}
\def\langnames@langs@wals@mim{Alacatlatzala Mixtec}
\def\langnames@langs@wals@min{Minangkabau}
\def\langnames@langs@wals@mio{Pinotepa Nacional Mixtec}
\def\langnames@langs@wals@mip{Apasco-Apoala Mixtec}
\def\langnames@langs@wals@miq{Mískito}
\def\langnames@langs@wals@mir{Isthmus Mixe}
\def\langnames@langs@wals@mit{Southern Puebla Mixtec}
\def\langnames@langs@wals@miu{Cacaloxtepec Mixtec}
\def\langnames@langs@wals@miw{Akoye}
\def\langnames@langs@wals@mix{Mixtepec Mixtec}
\def\langnames@langs@wals@miy{Ayutla Mixtec}
\def\langnames@langs@wals@miz{Coatzospan Mixtec}
\def\langnames@langs@wals@mjc{San Juan Colorado Mixtec}
\def\langnames@langs@wals@mjd{Northwest Maidu}
\def\langnames@langs@wals@mje{Muskum}
\def\langnames@langs@wals@mjg{Mongghul}
\def\langnames@langs@wals@mjh{Mwera (Nyasa)}
\def\langnames@langs@wals@mji{Kim Mun}
\def\langnames@langs@wals@mjj{Mawak}
\def\langnames@langs@wals@mjk{Matukar}
\def\langnames@langs@wals@mjl{Mandeali}
\def\langnames@langs@wals@mjm{Medebur}
\def\langnames@langs@wals@mjn{Ma (Papua New Guinea)}
\def\langnames@langs@wals@mjo{Malankuravan}
\def\langnames@langs@wals@mjp{Malapandaram}
\def\langnames@langs@wals@mjq{Malaryan}
\def\langnames@langs@wals@mjr{Malavedan}
\def\langnames@langs@wals@mjs{Miship}
\def\langnames@langs@wals@mjt{Sauria Paharia}
\def\langnames@langs@wals@mjv{Mannan}
\def\langnames@langs@wals@mjw{Hills Karbi}
\def\langnames@langs@wals@mjx{Mahali}
\def\langnames@langs@wals@mjy{Mohican}
\def\langnames@langs@wals@mjz{Majhi}
\def\langnames@langs@wals@mka{Mbre}
\def\langnames@langs@wals@mkb{Mar Paharia of Dumka}
\def\langnames@langs@wals@mkc{Siliput}
\def\langnames@langs@wals@mkd{Macedonian}
\def\langnames@langs@wals@mke{Mawchi}
\def\langnames@langs@wals@mkf{Miya}
\def\langnames@langs@wals@mkg{Mak (China)}
\def\langnames@langs@wals@mki{Dhatki}
\def\langnames@langs@wals@mkj{Mokilese}
\def\langnames@langs@wals@mkk{Byep-Besep}
\def\langnames@langs@wals@mkl{Mokole}
\def\langnames@langs@wals@mkm{Moklen}
\def\langnames@langs@wals@mkn{Kupang Malay}
\def\langnames@langs@wals@mko{Mingang Doso}
\def\langnames@langs@wals@mkp{Moikodi}
\def\langnames@langs@wals@mkq{Bay Miwok}
\def\langnames@langs@wals@mkr{Manep}
\def\langnames@langs@wals@mks{Silacayoapan Mixtec}
\def\langnames@langs@wals@mkt{Vamale}
\def\langnames@langs@wals@mku{Konyanka Maninka}
\def\langnames@langs@wals@mkv{Mafea}
\def\langnames@langs@wals@mkw{Kituba (Congo)}
\def\langnames@langs@wals@mkx{Cinamiguin Manobo}
\def\langnames@langs@wals@mky{East Makian}
\def\langnames@langs@wals@mkz{Makasae-Makalero}
\def\langnames@langs@wals@mla{Tamambo}
\def\langnames@langs@wals@mlb{Mbule}
\def\langnames@langs@wals@mlc{Cao Lan}
\def\langnames@langs@wals@mle{Manambu}
\def\langnames@langs@wals@mlf{Mal}
\def\langnames@langs@wals@mlh{Mape}
\def\langnames@langs@wals@mli{Malimpung}
\def\langnames@langs@wals@mlj{Miltu}
\def\langnames@langs@wals@mlk{Ilwana}
\def\langnames@langs@wals@mll{Malua Bay}
\def\langnames@langs@wals@mlm{Mulam}
\def\langnames@langs@wals@mln{Malango}
\def\langnames@langs@wals@mlo{Mlomp}
\def\langnames@langs@wals@mlp{Bargam}
\def\langnames@langs@wals@mlq{Western Maninkakan}
\def\langnames@langs@wals@mlr{Vame}
\def\langnames@langs@wals@mls{Masalit}
\def\langnames@langs@wals@mlt{Maltese}
\def\langnames@langs@wals@mlu{To'abaita}
\def\langnames@langs@wals@mlv{Mwotlap}
\def\langnames@langs@wals@mlw{Moloko}
\def\langnames@langs@wals@mlx{Na'ahai}
\def\langnames@langs@wals@mma{Mama}
\def\langnames@langs@wals@mmb{Momina}
\def\langnames@langs@wals@mmc{Michoacán Mazahua}
\def\langnames@langs@wals@mmd{Maonan}
\def\langnames@langs@wals@mme{Tirax}
\def\langnames@langs@wals@mmf{Mindat}
\def\langnames@langs@wals@mmg{North Ambrym}
\def\langnames@langs@wals@mmh{Mehináku}
\def\langnames@langs@wals@mmi{Hember Avu}
\def\langnames@langs@wals@mmj{Majhwar}
\def\langnames@langs@wals@mmk{Mukha-Dora}
\def\langnames@langs@wals@mml{Man Met}
\def\langnames@langs@wals@mmm{Maii}
\def\langnames@langs@wals@mmn{Mamanwa}
\def\langnames@langs@wals@mmo{Mangga Buang}
\def\langnames@langs@wals@mmp{Siawi}
\def\langnames@langs@wals@mmq{Aisi}
\def\langnames@langs@wals@mmr{Western Xiangxi Miao}
\def\langnames@langs@wals@mmt{Malalamai}
\def\langnames@langs@wals@mmu{Mmaala}
\def\langnames@langs@wals@mmv{Miriti}
\def\langnames@langs@wals@mmw{Emae}
\def\langnames@langs@wals@mmx{Madak}
\def\langnames@langs@wals@mmy{Migaama}
\def\langnames@langs@wals@mmz{Mabaale}
\def\langnames@langs@wals@mna{Mbula}
\def\langnames@langs@wals@mnb{Muna}
\def\langnames@langs@wals@mnc{Manchu}
\def\langnames@langs@wals@mnd{Salamãi}
\def\langnames@langs@wals@mne{Naba}
\def\langnames@langs@wals@mnf{Mundani}
\def\langnames@langs@wals@mng{Eastern Mnong}
\def\langnames@langs@wals@mnh{Mono (Democratic Republic of Congo)}
\def\langnames@langs@wals@mni{Manipuri}
\def\langnames@langs@wals@mnj{Munji}
\def\langnames@langs@wals@mnk{Mandinka}
\def\langnames@langs@wals@mnl{Tiale}
\def\langnames@langs@wals@mnm{Mapena}
\def\langnames@langs@wals@mnn{Southern Mnong}
\def\langnames@langs@wals@mnp{Min Bei Chinese}
\def\langnames@langs@wals@mnq{Minriq}
\def\langnames@langs@wals@mnr{Mono (USA)}
\def\langnames@langs@wals@mns{Northern Mansi}
\def\langnames@langs@wals@mnt{Maykulan}
\def\langnames@langs@wals@mnu{Mer}
\def\langnames@langs@wals@mnv{Rennell-Bellona}
\def\langnames@langs@wals@mnw{Mon}
\def\langnames@langs@wals@mnx{Sougb}
\def\langnames@langs@wals@mny{Manyawa}
\def\langnames@langs@wals@mnz{Moni}
\def\langnames@langs@wals@moa{Mwan}
\def\langnames@langs@wals@moc{Mocoví}
\def\langnames@langs@wals@mod{Mobilian}
\def\langnames@langs@wals@moe{Montagnais}
\def\langnames@langs@wals@mof{Mohegan-Montauk-Narragansett}
\def\langnames@langs@wals@mog{Mongondow}
\def\langnames@langs@wals@moh{Mohawk}
\def\langnames@langs@wals@moi{Mboi}
\def\langnames@langs@wals@moj{Monzombo}
\def\langnames@langs@wals@mok{Marori}
\def\langnames@langs@wals@mom{Mangue}
\def\langnames@langs@wals@moo{Monom}
\def\langnames@langs@wals@mop{Mopán Maya}
\def\langnames@langs@wals@moq{Mor (Bomberai Peninsula)}
\def\langnames@langs@wals@mor{Moro}
\def\langnames@langs@wals@mos{Mossi}
\def\langnames@langs@wals@mot{Barí}
\def\langnames@langs@wals@mou{Mogum}
\def\langnames@langs@wals@mov{Mohave}
\def\langnames@langs@wals@mow{Moi (Congo)}
\def\langnames@langs@wals@mox{Molima}
\def\langnames@langs@wals@moy{Shekkacho}
\def\langnames@langs@wals@moz{Mukulu}
\def\langnames@langs@wals@mpa{Mpoto}
\def\langnames@langs@wals@mpb{Mullukmulluk}
\def\langnames@langs@wals@mpc{Mangarrayi}
\def\langnames@langs@wals@mpd{Machinere}
\def\langnames@langs@wals@mpe{Majang}
\def\langnames@langs@wals@mpg{Marba}
\def\langnames@langs@wals@mph{Mawng}
\def\langnames@langs@wals@mpi{Mpade}
\def\langnames@langs@wals@mpj{Martu Wangka}
\def\langnames@langs@wals@mpk{Mbara (Chad)}
\def\langnames@langs@wals@mpl{Middle Watut}
\def\langnames@langs@wals@mpm{Yosondúa Mixtec}
\def\langnames@langs@wals@mpn{Mindiri}
\def\langnames@langs@wals@mpo{Miu}
\def\langnames@langs@wals@mpp{Migabac}
\def\langnames@langs@wals@mpq{Matís}
\def\langnames@langs@wals@mpr{Vangunu}
\def\langnames@langs@wals@mps{Dadibi}
\def\langnames@langs@wals@mpt{Mian}
\def\langnames@langs@wals@mpu{Makuráp}
\def\langnames@langs@wals@mpv{Mungkip}
\def\langnames@langs@wals@mpw{Mapidian-Mawayana}
\def\langnames@langs@wals@mpx{Misima-Paneati}
\def\langnames@langs@wals@mpy{Mapia}
\def\langnames@langs@wals@mpz{Mpi}
\def\langnames@langs@wals@mqa{Maba (Indonesia)}
\def\langnames@langs@wals@mqb{Mbuko}
\def\langnames@langs@wals@mqe{Matepi}
\def\langnames@langs@wals@mqf{Momuna}
\def\langnames@langs@wals@mqg{Kota Bangun Kutai Malay}
\def\langnames@langs@wals@mqh{Tlazoyaltepec Mixtec}
\def\langnames@langs@wals@mqi{Mariri}
\def\langnames@langs@wals@mqj{Mamasa}
\def\langnames@langs@wals@mqk{Rajah Kabunsuwan Manobo}
\def\langnames@langs@wals@mql{Mbelime}
\def\langnames@langs@wals@mqm{South Marquesan}
\def\langnames@langs@wals@mqn{Moronene}
\def\langnames@langs@wals@mqo{Modole}
\def\langnames@langs@wals@mqp{Manipa}
\def\langnames@langs@wals@mqq{Minokok}
\def\langnames@langs@wals@mqr{Mander}
\def\langnames@langs@wals@mqs{West Makian}
\def\langnames@langs@wals@mqt{Mok}
\def\langnames@langs@wals@mqu{Mandari}
\def\langnames@langs@wals@mqv{Mosimo}
\def\langnames@langs@wals@mqw{Murupi}
\def\langnames@langs@wals@mqx{Mamuju}
\def\langnames@langs@wals@mqy{Manggarai}
\def\langnames@langs@wals@mqz{Malasanga}
\def\langnames@langs@wals@mra{Mlabri}
\def\langnames@langs@wals@mrb{Sunwadia}
\def\langnames@langs@wals@mrc{Maricopa}
\def\langnames@langs@wals@mrd{Western Magar}
\def\langnames@langs@wals@mre{Martha's Vineyard Sign Language}
\def\langnames@langs@wals@mrf{Elseng}
\def\langnames@langs@wals@mrg{Mising-Padam-Miri-Minyong}
\def\langnames@langs@wals@mrh{Mara Chin}
\def\langnames@langs@wals@mri{Maori}
\def\langnames@langs@wals@mrj{Western Mari}
\def\langnames@langs@wals@mrk{Hmwaveke}
\def\langnames@langs@wals@mrl{Mortlockese}
\def\langnames@langs@wals@mrm{Merlav}
\def\langnames@langs@wals@mrn{Cheke Holo}
\def\langnames@langs@wals@mro{Mru}
\def\langnames@langs@wals@mrp{Morouas}
\def\langnames@langs@wals@mrq{North Marquesan}
\def\langnames@langs@wals@mrr{Maria (India)}
\def\langnames@langs@wals@mrs{Maragus}
\def\langnames@langs@wals@mrt{Marghi Central}
\def\langnames@langs@wals@mru{Mono (Cameroon)}
\def\langnames@langs@wals@mrv{Mangareva}
\def\langnames@langs@wals@mrw{Maranao}
\def\langnames@langs@wals@mrx{Maremgi}
\def\langnames@langs@wals@mry{Mandaya}
\def\langnames@langs@wals@mrz{Marind}
\def\langnames@langs@wals@msb{Masbatenyo}
\def\langnames@langs@wals@msc{Sankaran Maninka}
\def\langnames@langs@wals@msd{Yucatec Maya Sign Language}
\def\langnames@langs@wals@mse{Musey}
\def\langnames@langs@wals@msf{Mekwei}
\def\langnames@langs@wals@msg{Moraid}
\def\langnames@langs@wals@msh{Masikoro Malagasy}
\def\langnames@langs@wals@msi{Sabah Malay}
\def\langnames@langs@wals@msj{Ma (Democratic Republic of Congo)}
\def\langnames@langs@wals@msk{Mansaka}
\def\langnames@langs@wals@msl{Molof}
\def\langnames@langs@wals@msm{Agusan Manobo}
\def\langnames@langs@wals@msn{Vurës}
\def\langnames@langs@wals@mso{Mombum}
\def\langnames@langs@wals@msp{Maritsauá}
\def\langnames@langs@wals@msq{Caac}
\def\langnames@langs@wals@msr{Mongolian Sign Language}
\def\langnames@langs@wals@mss{West Masela}
\def\langnames@langs@wals@msu{Musom}
\def\langnames@langs@wals@msv{Maslam}
\def\langnames@langs@wals@msw{Mansoanka}
\def\langnames@langs@wals@msx{Moresada}
\def\langnames@langs@wals@msy{Aruamu}
\def\langnames@langs@wals@msz{Momare}
\def\langnames@langs@wals@mta{Cotabato Manobo}
\def\langnames@langs@wals@mtb{Anyin Morofo}
\def\langnames@langs@wals@mtc{Munit}
\def\langnames@langs@wals@mtd{Mualang}
\def\langnames@langs@wals@mte{Mono-Alu}
\def\langnames@langs@wals@mtf{Murik (Papua New Guinea)}
\def\langnames@langs@wals@mtg{Una}
\def\langnames@langs@wals@mth{Munggui}
\def\langnames@langs@wals@mti{Maiwa (Papua New Guinea)}
\def\langnames@langs@wals@mtj{Moskona}
\def\langnames@langs@wals@mtk{Mbe'}
\def\langnames@langs@wals@mtl{Montol}
\def\langnames@langs@wals@mtm{Mator-Taigi-Karagas}
\def\langnames@langs@wals@mtn{Matagalpa}
\def\langnames@langs@wals@mto{Totontepec Mixe}
\def\langnames@langs@wals@mtp{Wichí Lhamtés Nocten}
\def\langnames@langs@wals@mtq{Muong}
\def\langnames@langs@wals@mtr{Mewari}
\def\langnames@langs@wals@mts{Yora}
\def\langnames@langs@wals@mtt{Mota}
\def\langnames@langs@wals@mtu{Tututepec Mixtec}
\def\langnames@langs@wals@mtv{Asaro'o}
\def\langnames@langs@wals@mtw{Southern Binukidnon}
\def\langnames@langs@wals@mtx{Tidaá Mixtec}
\def\langnames@langs@wals@mty{Nabi-Metan}
\def\langnames@langs@wals@mua{Mundang}
\def\langnames@langs@wals@mub{Mubi}
\def\langnames@langs@wals@muc{Ajumbu}
\def\langnames@langs@wals@mud{Mednyj Aleut}
\def\langnames@langs@wals@mue{Media Lengua}
\def\langnames@langs@wals@mug{Musgu}
\def\langnames@langs@wals@muh{Mündü}
\def\langnames@langs@wals@mui{Musi}
\def\langnames@langs@wals@muj{Mabire}
\def\langnames@langs@wals@muk{Mugom}
\def\langnames@langs@wals@mum{Maiwala}
\def\langnames@langs@wals@muo{Nyong}
\def\langnames@langs@wals@mup{Malvi}
\def\langnames@langs@wals@muq{Eastern Xiangxi Miao}
\def\langnames@langs@wals@mur{Murle}
\def\langnames@langs@wals@mus{Creek}
\def\langnames@langs@wals@mut{Western Muria}
\def\langnames@langs@wals@muu{Yaaku}
\def\langnames@langs@wals@muv{Muthuvan}
\def\langnames@langs@wals@mux{Bo-Ung}
\def\langnames@langs@wals@muy{Muyang}
\def\langnames@langs@wals@muz{Mursi}
\def\langnames@langs@wals@mva{Manam}
\def\langnames@langs@wals@mvb{Mattole-Bear River}
\def\langnames@langs@wals@mvd{Mamboru}
\def\langnames@langs@wals@mve{Marwari (Pakistan)}
\def\langnames@langs@wals@mvf{Peripheral Mongolian}
\def\langnames@langs@wals@mvg{Yucuañe Mixtec}
\def\langnames@langs@wals@mvh{Mire}
\def\langnames@langs@wals@mvi{Miyako}
\def\langnames@langs@wals@mvk{Mekmek}
\def\langnames@langs@wals@mvl{Mbara-Yanga}
\def\langnames@langs@wals@mvn{Minaveha}
\def\langnames@langs@wals@mvo{Marovo}
\def\langnames@langs@wals@mvp{Duri}
\def\langnames@langs@wals@mvq{Moere}
\def\langnames@langs@wals@mvr{Marau}
\def\langnames@langs@wals@mvs{Massep}
\def\langnames@langs@wals@mvt{Mpotovoro}
\def\langnames@langs@wals@mvu{Marfa}
\def\langnames@langs@wals@mvv{Tagal Murut}
\def\langnames@langs@wals@mvw{Machinga}
\def\langnames@langs@wals@mvx{Meoswar}
\def\langnames@langs@wals@mvy{Indus Kohistani}
\def\langnames@langs@wals@mvz{Mesqan}
\def\langnames@langs@wals@mwa{Mwatebu}
\def\langnames@langs@wals@mwb{Juwal}
\def\langnames@langs@wals@mwc{Are}
\def\langnames@langs@wals@mwe{Mwera (Chimwera)}
\def\langnames@langs@wals@mwf{Murriny Patha}
\def\langnames@langs@wals@mwg{Aiklep}
\def\langnames@langs@wals@mwh{Mouk-Aria}
\def\langnames@langs@wals@mwi{Ninde}
\def\langnames@langs@wals@mwk{Kita Maninkakan}
\def\langnames@langs@wals@mwl{Mirandese}
\def\langnames@langs@wals@mwm{Sar}
\def\langnames@langs@wals@mwn{Nyamwanga}
\def\langnames@langs@wals@mwo{Central Maewo}
\def\langnames@langs@wals@mwp{Kala Lagaw Ya}
\def\langnames@langs@wals@mwq{Mün Chin}
\def\langnames@langs@wals@mws{Mwimbi-Muthambi}
\def\langnames@langs@wals@mwt{Moken}
\def\langnames@langs@wals@mwu{Mittu}
\def\langnames@langs@wals@mwv{Mentawai}
\def\langnames@langs@wals@mww{Hmong Daw}
\def\langnames@langs@wals@mwy{Akie}
\def\langnames@langs@wals@mwz{Moingi}
\def\langnames@langs@wals@mxa{Northwest Oaxaca Mixtec}
\def\langnames@langs@wals@mxb{Tezoatlán Mixtec}
\def\langnames@langs@wals@mxc{Manyika}
\def\langnames@langs@wals@mxd{Modang}
\def\langnames@langs@wals@mxe{Mele-Fila}
\def\langnames@langs@wals@mxf{Malgbe}
\def\langnames@langs@wals@mxg{Mbangala}
\def\langnames@langs@wals@mxh{Mvuba}
\def\langnames@langs@wals@mxi{Mozarabic}
\def\langnames@langs@wals@mxj{Kman}
\def\langnames@langs@wals@mxk{Monumbo}
\def\langnames@langs@wals@mxl{Maxi Gbe}
\def\langnames@langs@wals@mxm{Meramera}
\def\langnames@langs@wals@mxn{Moi (Indonesia)}
\def\langnames@langs@wals@mxo{Mbowe}
\def\langnames@langs@wals@mxp{Tlahuitoltepec Mixe}
\def\langnames@langs@wals@mxq{Juquila Mixe}
\def\langnames@langs@wals@mxr{Murik (Malaysia)}
\def\langnames@langs@wals@mxs{Huitepec Mixtec}
\def\langnames@langs@wals@mxt{Jamiltepec Mixtec}
\def\langnames@langs@wals@mxu{Mada (Cameroon)}
\def\langnames@langs@wals@mxv{Metlatónoc Mixtec}
\def\langnames@langs@wals@mxw{Namo}
\def\langnames@langs@wals@mxx{Mahou}
\def\langnames@langs@wals@mxy{Southeastern Nochixtlán Mixtec}
\def\langnames@langs@wals@mxz{Central Masela}
\def\langnames@langs@wals@mya{Burmese}
\def\langnames@langs@wals@myb{Mbay}
\def\langnames@langs@wals@mye{Myene}
\def\langnames@langs@wals@myf{Bambassi}
\def\langnames@langs@wals@myg{Manta}
\def\langnames@langs@wals@myh{Makah}
\def\langnames@langs@wals@myj{Mangayat}
\def\langnames@langs@wals@myk{Mamara Senoufo}
\def\langnames@langs@wals@myl{Moma}
\def\langnames@langs@wals@mym{Me'en}
\def\langnames@langs@wals@myo{Anfillo}
\def\langnames@langs@wals@myp{Pirahã}
\def\langnames@langs@wals@myr{Muniche}
\def\langnames@langs@wals@mys{Mesmes}
\def\langnames@langs@wals@myu{Mundurukú}
\def\langnames@langs@wals@myv{Erzya}
\def\langnames@langs@wals@myw{Muyuw}
\def\langnames@langs@wals@myx{Masaaba}
\def\langnames@langs@wals@myy{Macuna}
\def\langnames@langs@wals@myz{Classical Mandaic}
\def\langnames@langs@wals@mza{Santa María Zacatepec Mixtec}
\def\langnames@langs@wals@mzb{Tumzabt}
\def\langnames@langs@wals@mzc{Madagascar Sign Language}
\def\langnames@langs@wals@mzd{Malimba}
\def\langnames@langs@wals@mze{Morawa}
\def\langnames@langs@wals@mzg{Monastic Sign Language}
\def\langnames@langs@wals@mzh{Wichí Lhamtés Güisnay}
\def\langnames@langs@wals@mzi{Ixcatlán Mazatec}
\def\langnames@langs@wals@mzj{Manya}
\def\langnames@langs@wals@mzk{Western Mambila}
\def\langnames@langs@wals@mzl{Mazatlán Mixe}
\def\langnames@langs@wals@mzm{Mumuye}
\def\langnames@langs@wals@mzn{Mazanderani}
\def\langnames@langs@wals@mzo{Matipuhy}
\def\langnames@langs@wals@mzp{Movima}
\def\langnames@langs@wals@mzq{Mori Atas}
\def\langnames@langs@wals@mzr{Marúbo}
\def\langnames@langs@wals@mzs{Macanese}
\def\langnames@langs@wals@mzt{Mintil}
\def\langnames@langs@wals@mzu{Itutang-Inapang}
\def\langnames@langs@wals@mzv{Manza}
\def\langnames@langs@wals@mzw{Deg}
\def\langnames@langs@wals@mzy{Mozambican Sign Language}
\def\langnames@langs@wals@mzz{Maiadomu}
\def\langnames@langs@wals@naa{Namla}
\def\langnames@langs@wals@nab{Southern Nambikuára}
\def\langnames@langs@wals@nac{Narak}
\def\langnames@langs@wals@nad{Palyku}
\def\langnames@langs@wals@nae{Naka'ela}
\def\langnames@langs@wals@naf{Nabak}
\def\langnames@langs@wals@nag{Naga Pidgin}
\def\langnames@langs@wals@naj{Nalu}
\def\langnames@langs@wals@nak{Nakanai}
\def\langnames@langs@wals@nal{Nalik}
\def\langnames@langs@wals@nam{Nangikurrunggurr}
\def\langnames@langs@wals@nao{Naaba}
\def\langnames@langs@wals@nap{Continental Southern Italian}
\def\langnames@langs@wals@naq{Nama (Namibia)}
\def\langnames@langs@wals@nar{Iguta}
\def\langnames@langs@wals@nas{Naasioi}
\def\langnames@langs@wals@nat{Hungworo}
\def\langnames@langs@wals@nau{Nauru}
\def\langnames@langs@wals@nav{Navajo}
\def\langnames@langs@wals@naw{Nawuri}
\def\langnames@langs@wals@nax{Nakwi}
\def\langnames@langs@wals@nay{Narrinyeri}
\def\langnames@langs@wals@naz{Coatepec Nahuatl}
\def\langnames@langs@wals@nba{Nyemba}
\def\langnames@langs@wals@nbb{Ndoe}
\def\langnames@langs@wals@nbc{Chang Naga}
\def\langnames@langs@wals@nbd{Ngbinda-Mayeka}
\def\langnames@langs@wals@nbe{Konyak Naga}
\def\langnames@langs@wals@nbg{Nagarchal}
\def\langnames@langs@wals@nbh{Ngamo}
\def\langnames@langs@wals@nbi{Mao Naga}
\def\langnames@langs@wals@nbj{Ngarinman}
\def\langnames@langs@wals@nbk{Nake}
\def\langnames@langs@wals@nbm{Ngbaka Ma'bo}
\def\langnames@langs@wals@nbn{Nabi}
\def\langnames@langs@wals@nbo{Nkukoli}
\def\langnames@langs@wals@nbp{Nnam}
\def\langnames@langs@wals@nbq{Nggem}
\def\langnames@langs@wals@nbr{Numana}
\def\langnames@langs@wals@nbs{Namibian Sign Language}
\def\langnames@langs@wals@nbt{Na}
\def\langnames@langs@wals@nbu{Rongmei Naga}
\def\langnames@langs@wals@nbv{Ngamambo}
\def\langnames@langs@wals@nbw{Southern Ngbandi}
\def\langnames@langs@wals@nbx{Wilson River (Grey Range)}
\def\langnames@langs@wals@nby{Ningera}
\def\langnames@langs@wals@nca{Iyo}
\def\langnames@langs@wals@ncb{Central Nicobarese}
\def\langnames@langs@wals@ncc{Ponam}
\def\langnames@langs@wals@ncd{Nachering}
\def\langnames@langs@wals@nce{Yale}
\def\langnames@langs@wals@ncf{Notsi}
\def\langnames@langs@wals@ncg{Nisga'a}
\def\langnames@langs@wals@nch{Central Huasteca Nahuatl}
\def\langnames@langs@wals@nci{Classical Nahuatl}
\def\langnames@langs@wals@ncj{Northern Puebla Nahuatl}
\def\langnames@langs@wals@nck{Nakara}
\def\langnames@langs@wals@ncl{Michoacán Nahuatl}
\def\langnames@langs@wals@ncm{Nambo}
\def\langnames@langs@wals@ncn{Nauna}
\def\langnames@langs@wals@nco{Sibe (Nasioi)}
\def\langnames@langs@wals@ncq{Northern Katang}
\def\langnames@langs@wals@ncr{Ncane-Mungong}
\def\langnames@langs@wals@ncs{Nicaraguan Sign Language}
\def\langnames@langs@wals@nct{Chothe}
\def\langnames@langs@wals@ncu{Chumburung}
\def\langnames@langs@wals@ncx{Central Puebla Nahuatl}
\def\langnames@langs@wals@ncz{Natchez}
\def\langnames@langs@wals@nda{Ndasa}
\def\langnames@langs@wals@ndb{Kenswei Nsei}
\def\langnames@langs@wals@ndc{Ndau}
\def\langnames@langs@wals@ndd{Nde-Nsele-Nta}
\def\langnames@langs@wals@nde{Zimbabwean Ndebele}
\def\langnames@langs@wals@ndg{Ndengereko}
\def\langnames@langs@wals@ndh{Ndali}
\def\langnames@langs@wals@ndi{Samba Leko}
\def\langnames@langs@wals@ndj{Ndamba}
\def\langnames@langs@wals@ndk{Ndaka}
\def\langnames@langs@wals@ndl{Ndolo}
\def\langnames@langs@wals@ndm{Ndam}
\def\langnames@langs@wals@ndn{Ngundi}
\def\langnames@langs@wals@ndo{Ndonga}
\def\langnames@langs@wals@ndp{Ndo}
\def\langnames@langs@wals@ndq{Ndombe}
\def\langnames@langs@wals@ndr{Ndoola}
\def\langnames@langs@wals@nds{Eastern Low German}
\def\langnames@langs@wals@ndt{Ndunga}
\def\langnames@langs@wals@ndu{Dugun}
\def\langnames@langs@wals@ndv{Ndut}
\def\langnames@langs@wals@ndw{Ndobo}
\def\langnames@langs@wals@ndx{Nduga}
\def\langnames@langs@wals@ndy{Lutos}
\def\langnames@langs@wals@ndz{Ndogo}
\def\langnames@langs@wals@neb{Toura (Côte d'Ivoire)}
\def\langnames@langs@wals@nec{Klamu}
\def\langnames@langs@wals@nee{Nêlêmwa-Nixumwak}
\def\langnames@langs@wals@nef{Nefamese}
\def\langnames@langs@wals@neg{Negidal}
\def\langnames@langs@wals@neh{Upper Mangdep}
\def\langnames@langs@wals@nej{Neko}
\def\langnames@langs@wals@nek{Neku}
\def\langnames@langs@wals@nem{Nemi}
\def\langnames@langs@wals@nen{Nengone}
\def\langnames@langs@wals@neo{Ná-Meo}
\def\langnames@langs@wals@neq{North Central Mixe}
\def\langnames@langs@wals@ner{Yahadian}
\def\langnames@langs@wals@nes{Bhoti Kinnauri}
\def\langnames@langs@wals@net{Nete}
\def\langnames@langs@wals@neu{Neo (Artificial Language)}
\def\langnames@langs@wals@nev{Nyaheun}
\def\langnames@langs@wals@new{Kathmandu Valley Newari}
\def\langnames@langs@wals@nex{Neme}
\def\langnames@langs@wals@ney{Neyo}
\def\langnames@langs@wals@nez{Nez Perce}
\def\langnames@langs@wals@nfa{Dhao}
\def\langnames@langs@wals@nfd{Ndunic}
\def\langnames@langs@wals@nfl{Äiwoo}
\def\langnames@langs@wals@nfr{Nafanan}
\def\langnames@langs@wals@nfu{Southern Mfumte}
\def\langnames@langs@wals@nga{Ngbaka Minagende}
\def\langnames@langs@wals@ngb{Northern Ngbandi}
\def\langnames@langs@wals@ngc{Ngombe (Democratic Republic of Congo)}
\def\langnames@langs@wals@ngd{Ngando (Central African Republic)}
\def\langnames@langs@wals@nge{Ngemba}
\def\langnames@langs@wals@ngg{Ngbaka Manza}
\def\langnames@langs@wals@ngh{N||ng}
\def\langnames@langs@wals@ngi{Ngizim}
\def\langnames@langs@wals@ngj{Ngie}
\def\langnames@langs@wals@ngk{Ngalkbun}
\def\langnames@langs@wals@ngl{Mozambique Lomwe}
\def\langnames@langs@wals@ngm{Ngatik Men's Creole}
\def\langnames@langs@wals@ngn{Ngwo}
\def\langnames@langs@wals@ngo{Ngoni}
\def\langnames@langs@wals@ngp{Ngulu}
\def\langnames@langs@wals@ngq{Ngoreme}
\def\langnames@langs@wals@ngr{Nanggu}
\def\langnames@langs@wals@ngs{Gvoko}
\def\langnames@langs@wals@ngt{Kriang-Khlor}
\def\langnames@langs@wals@ngu{Central Guerrero Nahuatl}
\def\langnames@langs@wals@ngv{Nagumi}
\def\langnames@langs@wals@ngw{Ngwaba}
\def\langnames@langs@wals@ngx{Nggwahyi}
\def\langnames@langs@wals@ngy{Tibea}
\def\langnames@langs@wals@ngz{Ngungwel}
\def\langnames@langs@wals@nha{Nhanda}
\def\langnames@langs@wals@nhb{Beng}
\def\langnames@langs@wals@nhc{Tabasco Nahuatl}
\def\langnames@langs@wals@nhd{Chiripá}
\def\langnames@langs@wals@nhe{Eastern Huasteca Nahuatl}
\def\langnames@langs@wals@nhf{Nhuwala}
\def\langnames@langs@wals@nhg{Tetelcingo Nahuatl}
\def\langnames@langs@wals@nhh{Nahari}
\def\langnames@langs@wals@nhi{Zacatlán-Ahuacatlán-Tepetzintla Nahuatl}
\def\langnames@langs@wals@nhk{Isthmus-Cosoleacaque Nahuatl}
\def\langnames@langs@wals@nhm{Morelos Nahuatl}
\def\langnames@langs@wals@nhn{Tlaxcala-Puebla-Central Nahuatl}
\def\langnames@langs@wals@nho{Takuu}
\def\langnames@langs@wals@nhp{Isthmus-Pajapan Nahuatl}
\def\langnames@langs@wals@nhq{Huaxcaleca Nahuatl}
\def\langnames@langs@wals@nhr{Naro}
\def\langnames@langs@wals@nht{Ometepec Nahuatl}
\def\langnames@langs@wals@nhu{Noone}
\def\langnames@langs@wals@nhv{Temascaltepec Nahuatl}
\def\langnames@langs@wals@nhw{Western Huasteca Nahuatl}
\def\langnames@langs@wals@nhx{Isthmus-Mecayapan Nahuatl}
\def\langnames@langs@wals@nhy{Northern Oaxaca Nahuatl}
\def\langnames@langs@wals@nhz{Santa María La Alta Nahuatl}
\def\langnames@langs@wals@nia{Nias}
\def\langnames@langs@wals@nib{Nakama}
\def\langnames@langs@wals@nid{Ngandi}
\def\langnames@langs@wals@nie{Niellim}
\def\langnames@langs@wals@nif{Nek}
\def\langnames@langs@wals@nig{Ngalakgan}
\def\langnames@langs@wals@nih{Nyiha (Tanzania)}
\def\langnames@langs@wals@nii{Nii}
\def\langnames@langs@wals@nij{Ngaju}
\def\langnames@langs@wals@nik{Southern Nicobarese}
\def\langnames@langs@wals@nil{Nila}
\def\langnames@langs@wals@nim{Nilamba}
\def\langnames@langs@wals@nin{Ninzo}
\def\langnames@langs@wals@nio{Nganasan}
\def\langnames@langs@wals@niq{Nandi}
\def\langnames@langs@wals@nir{Nimboran}
\def\langnames@langs@wals@nis{Nimi}
\def\langnames@langs@wals@nit{Southeastern Kolami}
\def\langnames@langs@wals@niu{Niuean}
\def\langnames@langs@wals@niv{Amur Nivkh}
\def\langnames@langs@wals@niw{Nimo}
\def\langnames@langs@wals@nix{Hema}
\def\langnames@langs@wals@niy{Ngiti}
\def\langnames@langs@wals@niz{Ningil}
\def\langnames@langs@wals@nja{Nzanyi}
\def\langnames@langs@wals@njb{Nocte Naga}
\def\langnames@langs@wals@njh{Lotha Naga}
\def\langnames@langs@wals@nji{Ngarnka}
\def\langnames@langs@wals@njj{Njen}
\def\langnames@langs@wals@njl{Njalgulgule}
\def\langnames@langs@wals@njm{Angami Naga}
\def\langnames@langs@wals@njn{Liangmai Naga}
\def\langnames@langs@wals@njo{Ao Naga}
\def\langnames@langs@wals@njr{Njerep}
\def\langnames@langs@wals@njs{Nisa-Anasi}
\def\langnames@langs@wals@njt{Ndyuka-Trio Pidgin}
\def\langnames@langs@wals@nju{Ngadjunmaya}
\def\langnames@langs@wals@njx{Kunyi}
\def\langnames@langs@wals@njy{Njyem}
\def\langnames@langs@wals@njz{Nyishi-Hill Miri}
\def\langnames@langs@wals@nka{Nkoya}
\def\langnames@langs@wals@nkb{Khoibu}
\def\langnames@langs@wals@nkc{Nkongho}
\def\langnames@langs@wals@nkd{Koireng}
\def\langnames@langs@wals@nke{Duke}
\def\langnames@langs@wals@nkg{Nekgini}
\def\langnames@langs@wals@nkh{Khezha Naga}
\def\langnames@langs@wals@nki{Thangal Naga}
\def\langnames@langs@wals@nkj{Nakai}
\def\langnames@langs@wals@nkk{Nokuku}
\def\langnames@langs@wals@nkm{Namat}
\def\langnames@langs@wals@nkn{Nkangala}
\def\langnames@langs@wals@nko{Nkonya}
\def\langnames@langs@wals@nkp{Niuatoputapu}
\def\langnames@langs@wals@nkq{Nkami}
\def\langnames@langs@wals@nkr{Nukuoro}
\def\langnames@langs@wals@nks{Momogo-Pupis-Irogo}
\def\langnames@langs@wals@nkt{Nyika (Tanzania)}
\def\langnames@langs@wals@nku{Bouna Kulango}
\def\langnames@langs@wals@nkv{Nyika (Malawi and Zambia)}
\def\langnames@langs@wals@nkw{Nkutu}
\def\langnames@langs@wals@nkx{Nkoroo}
\def\langnames@langs@wals@nkz{Nkari}
\def\langnames@langs@wals@nla{Ngombale}
\def\langnames@langs@wals@nlc{Nalca}
\def\langnames@langs@wals@nld{Dutch}
\def\langnames@langs@wals@nle{East Nyala}
\def\langnames@langs@wals@nlg{Gela}
\def\langnames@langs@wals@nli{Grangali-Ningalami}
\def\langnames@langs@wals@nlj{Nyali}
\def\langnames@langs@wals@nlk{Ninia Yali}
\def\langnames@langs@wals@nll{Nihali}
\def\langnames@langs@wals@nlm{Mankiyali}
\def\langnames@langs@wals@nlo{Ngwii}
\def\langnames@langs@wals@nlu{Nchumbulu}
\def\langnames@langs@wals@nlv{Orizaba Nahuatl}
\def\langnames@langs@wals@nlx{Nahali-Baglani}
\def\langnames@langs@wals@nly{Nyamal}
\def\langnames@langs@wals@nlz{Nalögo}
\def\langnames@langs@wals@nma{Maram Naga}
\def\langnames@langs@wals@nmb{Big Nambas}
\def\langnames@langs@wals@nmc{Ngam}
\def\langnames@langs@wals@nmd{Ndumu}
\def\langnames@langs@wals@nme{Mzieme Naga}
\def\langnames@langs@wals@nmf{East-Central Tangkhul Naga}
\def\langnames@langs@wals@nmg{Kwasio}
\def\langnames@langs@wals@nmh{Monsang Naga}
\def\langnames@langs@wals@nmi{Nyam}
\def\langnames@langs@wals@nmk{Namakura}
\def\langnames@langs@wals@nml{Ndemli}
\def\langnames@langs@wals@nmm{Manange}
\def\langnames@langs@wals@nmn{East Taa}
\def\langnames@langs@wals@nmo{Moyon}
\def\langnames@langs@wals@nmp{Nimanbur}
\def\langnames@langs@wals@nmq{Nambya}
\def\langnames@langs@wals@nmr{Nimbari}
\def\langnames@langs@wals@nms{Letemboi-Repanbitip}
\def\langnames@langs@wals@nmt{Namonuito}
\def\langnames@langs@wals@nmu{Northeast Maidu}
\def\langnames@langs@wals@nmv{Ngamini-Yarluyandi-Karangura}
\def\langnames@langs@wals@nmw{Nimoa}
\def\langnames@langs@wals@nmx{Nama (Papua New Guinea)}
\def\langnames@langs@wals@nmy{Namuyi}
\def\langnames@langs@wals@nmz{Nawdm}
\def\langnames@langs@wals@nna{Nyangumarta}
\def\langnames@langs@wals@nnb{Nande}
\def\langnames@langs@wals@nnc{Nancere}
\def\langnames@langs@wals@nnd{West Ambae}
\def\langnames@langs@wals@nne{Ngandyera}
\def\langnames@langs@wals@nnf{Ngaing}
\def\langnames@langs@wals@nng{Maring Naga}
\def\langnames@langs@wals@nnh{Ngiemboon}
\def\langnames@langs@wals@nni{North Nuaulu}
\def\langnames@langs@wals@nnj{Nyangatom}
\def\langnames@langs@wals@nnk{Nankina}
\def\langnames@langs@wals@nnl{Northern Rengma Naga}
\def\langnames@langs@wals@nnm{Namia}
\def\langnames@langs@wals@nnn{Ngete}
\def\langnames@langs@wals@nnp{Wancho Naga}
\def\langnames@langs@wals@nnq{Ngindo}
\def\langnames@langs@wals@nnr{Narungga}
\def\langnames@langs@wals@nnt{Nanticoke}
\def\langnames@langs@wals@nnu{Dwang}
\def\langnames@langs@wals@nnv{Nugunu (Australia)}
\def\langnames@langs@wals@nnw{Southern Nuni}
\def\langnames@langs@wals@nny{Yangkaal}
\def\langnames@langs@wals@nnz{Nda'nda'}
\def\langnames@langs@wals@noa{Woun Meu}
\def\langnames@langs@wals@noc{Nuk}
\def\langnames@langs@wals@nod{Northern Thai}
\def\langnames@langs@wals@noe{Nimadi}
\def\langnames@langs@wals@nof{Nomane}
\def\langnames@langs@wals@nog{Nogai}
\def\langnames@langs@wals@noh{Nomu}
\def\langnames@langs@wals@noi{Noiri}
\def\langnames@langs@wals@noj{Nonuya}
\def\langnames@langs@wals@nok{Nooksack}
\def\langnames@langs@wals@non{Old Norse}
\def\langnames@langs@wals@nop{Numanggang}
\def\langnames@langs@wals@noq{Ngongo}
\def\langnames@langs@wals@nor{Norwegian}
\def\langnames@langs@wals@nos{Eastern Nisu}
\def\langnames@langs@wals@not{Nomatsiguenga}
\def\langnames@langs@wals@nou{Ewage-Notu}
\def\langnames@langs@wals@nov{Novial}
\def\langnames@langs@wals@now{Nyambo}
\def\langnames@langs@wals@noy{Noy}
\def\langnames@langs@wals@noz{Nayi}
\def\langnames@langs@wals@npa{Nar Phu}
\def\langnames@langs@wals@nph{Phom Naga}
\def\langnames@langs@wals@npi{Nepali}
\def\langnames@langs@wals@npl{Nahuatl, Southeastern Puebla}
\def\langnames@langs@wals@npn{Mondropolon}
\def\langnames@langs@wals@npo{Pochuri Naga}
\def\langnames@langs@wals@nps{Nipsan}
\def\langnames@langs@wals@npy{Napu}
\def\langnames@langs@wals@nqg{Ede Nago}
\def\langnames@langs@wals@nqk{Kura Ede Nago}
\def\langnames@langs@wals@nql{Ngendelengo}
\def\langnames@langs@wals@nqm{Ndom}
\def\langnames@langs@wals@nqn{Nen}
\def\langnames@langs@wals@nqo{N'Ko}
\def\langnames@langs@wals@nqt{Nteng}
\def\langnames@langs@wals@nra{Ngom}
\def\langnames@langs@wals@nrb{Nara}
\def\langnames@langs@wals@nrc{Noric}
\def\langnames@langs@wals@nre{Southern Rengma Naga}
\def\langnames@langs@wals@nrg{Narango}
\def\langnames@langs@wals@nri{Chokri Naga}
\def\langnames@langs@wals@nrk{Ngarla}
\def\langnames@langs@wals@nrl{Ngarluma}
\def\langnames@langs@wals@nrm{Narom}
\def\langnames@langs@wals@nrp{North Picene}
\def\langnames@langs@wals@nrt{Tualatin-Yamhill}
\def\langnames@langs@wals@nru{Narua}
\def\langnames@langs@wals@nrx{Ngurmbur}
\def\langnames@langs@wals@nrz{Lala}
\def\langnames@langs@wals@nsa{Sangtam Naga}
\def\langnames@langs@wals@nsb{Lower-Nosop}
\def\langnames@langs@wals@nsc{Nshi}
\def\langnames@langs@wals@nsd{Southern Nisu}
\def\langnames@langs@wals@nse{Nsenga}
\def\langnames@langs@wals@nsf{Far Northwestern Nisu}
\def\langnames@langs@wals@nsg{Ngasa}
\def\langnames@langs@wals@nsh{Ngoshie}
\def\langnames@langs@wals@nsi{Nigerian Sign Language}
\def\langnames@langs@wals@nsk{Naskapi}
\def\langnames@langs@wals@nsl{Norwegian Sign Language}
\def\langnames@langs@wals@nsm{Sumi Naga}
\def\langnames@langs@wals@nsn{Nehan}
\def\langnames@langs@wals@nso{Pedi}
\def\langnames@langs@wals@nsp{Nepalese Sign Language}
\def\langnames@langs@wals@nsq{Northern Sierra Miwok}
\def\langnames@langs@wals@nsr{Maritime Sign Language}
\def\langnames@langs@wals@nss{Nali}
\def\langnames@langs@wals@nst{Pangwa Naga}
\def\langnames@langs@wals@nsu{Sierra Negra Nahuatl}
\def\langnames@langs@wals@nsw{Navut}
\def\langnames@langs@wals@nsx{Nsongo}
\def\langnames@langs@wals@nsy{Nasal}
\def\langnames@langs@wals@nsz{Nisenan}
\def\langnames@langs@wals@ntd{Northern Tidung}
\def\langnames@langs@wals@nte{Nathembo}
\def\langnames@langs@wals@nti{Natioro}
\def\langnames@langs@wals@ntj{Ngaanyatjarra}
\def\langnames@langs@wals@ntk{Ikoma-Nata}
\def\langnames@langs@wals@ntm{Nateni}
\def\langnames@langs@wals@nto{Ntomba}
\def\langnames@langs@wals@ntp{Northern Tepehuan}
\def\langnames@langs@wals@ntr{Delo}
\def\langnames@langs@wals@ntu{Natügu}
\def\langnames@langs@wals@ntw{Nottoway}
\def\langnames@langs@wals@nty{Mantsi}
\def\langnames@langs@wals@ntz{Natanzic}
\def\langnames@langs@wals@nua{Yuaga}
\def\langnames@langs@wals@nuc{Nukuini}
\def\langnames@langs@wals@nud{Ngala}
\def\langnames@langs@wals@nue{Ngundu}
\def\langnames@langs@wals@nuf{Nusu}
\def\langnames@langs@wals@nug{Nungali}
\def\langnames@langs@wals@nuh{Ndunda}
\def\langnames@langs@wals@nui{Ngumbi}
\def\langnames@langs@wals@nuj{Nyole}
\def\langnames@langs@wals@nuk{Nuu-chah-nulth}
\def\langnames@langs@wals@nul{Nusa Laut}
\def\langnames@langs@wals@num{Niuafo'ou}
\def\langnames@langs@wals@nun{Nung (Myanmar)}
\def\langnames@langs@wals@nuo{Nguôn}
\def\langnames@langs@wals@nup{Nupe-Nupe-Tako}
\def\langnames@langs@wals@nuq{Nukumanu}
\def\langnames@langs@wals@nur{Nukuria}
\def\langnames@langs@wals@nus{Nuer}
\def\langnames@langs@wals@nut{Nung (Viet Nam)}
\def\langnames@langs@wals@nuu{Ngbundu}
\def\langnames@langs@wals@nuv{Northern Nuni}
\def\langnames@langs@wals@nuw{Nguluwan}
\def\langnames@langs@wals@nux{Mehek}
\def\langnames@langs@wals@nuy{Wubuy}
\def\langnames@langs@wals@nuz{Tlamacazapa Nahuatl}
\def\langnames@langs@wals@nvh{Nasarian}
\def\langnames@langs@wals@nvm{Namiae}
\def\langnames@langs@wals@nvo{Nyokon}
\def\langnames@langs@wals@nwa{Nawathinehena}
\def\langnames@langs@wals@nwb{Nyabwa}
\def\langnames@langs@wals@nwe{Ngwe}
\def\langnames@langs@wals@nwi{Southwest Tanna}
\def\langnames@langs@wals@nwm{Nyamusa-Molo}
\def\langnames@langs@wals@nwo{Nauo}
\def\langnames@langs@wals@nwr{Nawaru}
\def\langnames@langs@wals@nww{Ndwewe}
\def\langnames@langs@wals@nxa{Nauete}
\def\langnames@langs@wals@nxd{Ngando-Lalia}
\def\langnames@langs@wals@nxe{Nage}
\def\langnames@langs@wals@nxg{Ngad'a}
\def\langnames@langs@wals@nxi{Nindi}
\def\langnames@langs@wals@nxl{South Nuaulu}
\def\langnames@langs@wals@nxm{Numidian}
\def\langnames@langs@wals@nxn{Ngawun}
\def\langnames@langs@wals@nxo{Ndambomo}
\def\langnames@langs@wals@nxq{Naxi}
\def\langnames@langs@wals@nxr{Ninggerum}
\def\langnames@langs@wals@nxx{Nafri}
\def\langnames@langs@wals@nya{Nyanja}
\def\langnames@langs@wals@nyb{Nyangbo}
\def\langnames@langs@wals@nyc{Nyanga-li}
\def\langnames@langs@wals@nyd{Nyore}
\def\langnames@langs@wals@nye{Nyengo}
\def\langnames@langs@wals@nyf{Giryama}
\def\langnames@langs@wals@nyg{Nyindu}
\def\langnames@langs@wals@nyh{Nyigina}
\def\langnames@langs@wals@nyi{Ama (Sudan)}
\def\langnames@langs@wals@nyj{Nyanga}
\def\langnames@langs@wals@nyk{Nyaneka}
\def\langnames@langs@wals@nyl{Nyeu}
\def\langnames@langs@wals@nym{Nyamwezi}
\def\langnames@langs@wals@nyn{Nyankole}
\def\langnames@langs@wals@nyo{Nyoro}
\def\langnames@langs@wals@nyp{Nyang'i}
\def\langnames@langs@wals@nyq{Nayinic}
\def\langnames@langs@wals@nyr{Nyiha (Malawi)}
\def\langnames@langs@wals@nys{Nyunga}
\def\langnames@langs@wals@nyt{Nyawaygi}
\def\langnames@langs@wals@nyu{Nyungwe}
\def\langnames@langs@wals@nyv{Nyulnyul}
\def\langnames@langs@wals@nyx{Nganyaywana}
\def\langnames@langs@wals@nyy{Nyakyusa-Ngonde}
\def\langnames@langs@wals@nza{Tigon Mbembe}
\def\langnames@langs@wals@nzb{Njebi}
\def\langnames@langs@wals@nzd{Nzadi}
\def\langnames@langs@wals@nzi{Nzima}
\def\langnames@langs@wals@nzk{Nzakara}
\def\langnames@langs@wals@nzm{Zeme Naga}
\def\langnames@langs@wals@nzs{New Zealand Sign Language}
\def\langnames@langs@wals@nzy{Nzakambay}
\def\langnames@langs@wals@nzz{Nanga}
\def\langnames@langs@wals@oaa{Orok}
\def\langnames@langs@wals@oac{Oroch}
\def\langnames@langs@wals@oar{Old Aramaic-Sam'alian}
\def\langnames@langs@wals@obi{Obispeño}
\def\langnames@langs@wals@obl{Oblo}
\def\langnames@langs@wals@obo{Obo Manobo}
\def\langnames@langs@wals@obr{Old Burmese}
\def\langnames@langs@wals@obu{Obulom-Ochichi}
\def\langnames@langs@wals@oca{Ocaina}
\def\langnames@langs@wals@och{Old Chinese}
\def\langnames@langs@wals@oci{Occitan}
\def\langnames@langs@wals@ocu{Atzingo Matlatzinca}
\def\langnames@langs@wals@odk{Od}
\def\langnames@langs@wals@odt{Old Dutch-Old Frankish}
\def\langnames@langs@wals@odu{Odual}
\def\langnames@langs@wals@ofo{Ofo}
\def\langnames@langs@wals@ofs{Old Frisian}
\def\langnames@langs@wals@ofu{Efutop}
\def\langnames@langs@wals@ogb{Ogbia}
\def\langnames@langs@wals@ogc{Ogbah}
\def\langnames@langs@wals@oge{Old Georgian}
\def\langnames@langs@wals@ogg{Ogbogolo}
\def\langnames@langs@wals@ogo{Khana}
\def\langnames@langs@wals@ogu{Ogbronuagum}
\def\langnames@langs@wals@oia{Oirata}
\def\langnames@langs@wals@oie{Okolie}
\def\langnames@langs@wals@oin{Inebu One}
\def\langnames@langs@wals@ojb{Northwestern Ojibwa}
\def\langnames@langs@wals@ojc{Central Ojibwa}
\def\langnames@langs@wals@ojg{Eastern Ojibwa}
\def\langnames@langs@wals@ojp{Old Japanese}
\def\langnames@langs@wals@ojs{Severn Ojibwa}
\def\langnames@langs@wals@ojv{Luangiua}
\def\langnames@langs@wals@ojw{Western Ojibwa}
\def\langnames@langs@wals@oka{Okanagan}
\def\langnames@langs@wals@okb{Okobo}
\def\langnames@langs@wals@okd{Okodia}
\def\langnames@langs@wals@oke{Okpe (Southwestern Edo)}
\def\langnames@langs@wals@okh{Karanic}
\def\langnames@langs@wals@oki{Okiek}
\def\langnames@langs@wals@okj{Okojuwoi}
\def\langnames@langs@wals@okk{Kwamtim One}
\def\langnames@langs@wals@okl{Old Kentish Sign Language}
\def\langnames@langs@wals@okn{Oki-No-Erabu}
\def\langnames@langs@wals@okr{Kirike}
\def\langnames@langs@wals@oks{Oko-Eni-Osayen}
\def\langnames@langs@wals@oku{Oku}
\def\langnames@langs@wals@okv{Orokaiva}
\def\langnames@langs@wals@okx{Okpe (Northwestern Edo)}
\def\langnames@langs@wals@ola{Walungge}
\def\langnames@langs@wals@old{Mochi}
\def\langnames@langs@wals@ole{Olekha}
\def\langnames@langs@wals@olm{Oloma}
\def\langnames@langs@wals@olo{Livvi}
\def\langnames@langs@wals@olu{Kuvale}
\def\langnames@langs@wals@oma{Omaha-Ponca}
\def\langnames@langs@wals@omb{East Ambae}
\def\langnames@langs@wals@omc{Mochica}
\def\langnames@langs@wals@omg{Omagua}
\def\langnames@langs@wals@omi{Omi}
\def\langnames@langs@wals@omk{Malyj Anjuj Omok}
\def\langnames@langs@wals@oml{Ombo}
\def\langnames@langs@wals@omn{Minoan}
\def\langnames@langs@wals@omo{Utarmbung}
\def\langnames@langs@wals@omr{Old Marathi}
\def\langnames@langs@wals@omt{Omotik}
\def\langnames@langs@wals@omu{Omurano}
\def\langnames@langs@wals@omw{South Tairora}
\def\langnames@langs@wals@omx{Old Mon}
\def\langnames@langs@wals@ona{Selk'nam}
\def\langnames@langs@wals@onb{Western Ong-Be}
\def\langnames@langs@wals@one{Oneida}
\def\langnames@langs@wals@ong{Olo}
\def\langnames@langs@wals@oni{Onin}
\def\langnames@langs@wals@onj{Onjob}
\def\langnames@langs@wals@onk{Kabore One}
\def\langnames@langs@wals@onn{Onobasulu}
\def\langnames@langs@wals@ono{Onondaga}
\def\langnames@langs@wals@onp{Sartang}
\def\langnames@langs@wals@onr{Northern One}
\def\langnames@langs@wals@ons{Ono}
\def\langnames@langs@wals@onu{Unua}
\def\langnames@langs@wals@onw{Old Nubian}
\def\langnames@langs@wals@onx{Onin Pidgin}
\def\langnames@langs@wals@ood{Tohono O'odham}
\def\langnames@langs@wals@oog{Ong-Ir}
\def\langnames@langs@wals@oon{Önge}
\def\langnames@langs@wals@oor{Oorlams}
\def\langnames@langs@wals@oos{Old Ossetic}
\def\langnames@langs@wals@opa{Okpamheri}
\def\langnames@langs@wals@opk{Kopkaka}
\def\langnames@langs@wals@opm{Oksapmin}
\def\langnames@langs@wals@opo{Opao}
\def\langnames@langs@wals@opt{Teguima}
\def\langnames@langs@wals@opy{Ofayé}
\def\langnames@langs@wals@ora{Oroha}
\def\langnames@langs@wals@orc{Orma}
\def\langnames@langs@wals@ore{Maijiki}
\def\langnames@langs@wals@org{Oring}
\def\langnames@langs@wals@orh{Oroqen}
\def\langnames@langs@wals@orn{Orang Kanaq}
\def\langnames@langs@wals@oro{Orokolo}
\def\langnames@langs@wals@orr{Oruma}
\def\langnames@langs@wals@ors{Orang Seletar}
\def\langnames@langs@wals@ort{Kotia-Adivasi Oriya-Desiya}
\def\langnames@langs@wals@oru{Ormuri}
\def\langnames@langs@wals@orv{Old Russian}
\def\langnames@langs@wals@orw{Oro Win}
\def\langnames@langs@wals@orx{Oro}
\def\langnames@langs@wals@ory{Odia}
\def\langnames@langs@wals@orz{Ormu}
\def\langnames@langs@wals@osa{Osage}
\def\langnames@langs@wals@osc{Oscan}
\def\langnames@langs@wals@osi{Osing}
\def\langnames@langs@wals@oso{Ososo}
\def\langnames@langs@wals@osp{Old Spanish}
\def\langnames@langs@wals@oss{Iron Ossetian}
\def\langnames@langs@wals@ost{Osatu}
\def\langnames@langs@wals@osu{Southern One}
\def\langnames@langs@wals@osx{Old Saxon}
\def\langnames@langs@wals@otd{Ot Danum}
\def\langnames@langs@wals@ote{Mezquital Otomi}
\def\langnames@langs@wals@oti{Oti}
\def\langnames@langs@wals@otl{Tilapa Otomi}
\def\langnames@langs@wals@otm{Eastern Highland Otomi}
\def\langnames@langs@wals@otn{Tenango Otomi}
\def\langnames@langs@wals@otq{Querétaro Otomi}
\def\langnames@langs@wals@otr{Otoro}
\def\langnames@langs@wals@ots{Estado de México Otomi}
\def\langnames@langs@wals@ott{Temoaya Otomi}
\def\langnames@langs@wals@otu{Otuke}
\def\langnames@langs@wals@otw{Ottawa}
\def\langnames@langs@wals@otx{Texcatepec Otomi}
\def\langnames@langs@wals@oty{Old Tamil}
\def\langnames@langs@wals@otz{Ixtenco Otomi}
\def\langnames@langs@wals@oua{Ouargli}
\def\langnames@langs@wals@oub{Glio-Oubi}
\def\langnames@langs@wals@oue{Ounge}
\def\langnames@langs@wals@oui{Old Turkic}
\def\langnames@langs@wals@oum{Ouma}
\def\langnames@langs@wals@owi{Owiniga}
\def\langnames@langs@wals@owl{Old-Middle Welsh}
\def\langnames@langs@wals@oyb{Oy}
\def\langnames@langs@wals@oyd{Oyda}
\def\langnames@langs@wals@oym{Wayampi}
\def\langnames@langs@wals@oyy{Oya'oya}
\def\langnames@langs@wals@ozm{Koonzime}
\def\langnames@langs@wals@pab{Parecís}
\def\langnames@langs@wals@pac{Pacoh}
\def\langnames@langs@wals@pad{Paumari}
\def\langnames@langs@wals@pae{Pagibete}
\def\langnames@langs@wals@paf{Paranawát}
\def\langnames@langs@wals@pag{Pangasinan}
\def\langnames@langs@wals@pah{Tenharim-Parintintin-Diahoi}
\def\langnames@langs@wals@pai{Pye}
\def\langnames@langs@wals@pak{Parakanã}
\def\langnames@langs@wals@pal{Pahlavi}
\def\langnames@langs@wals@pam{Pampanga}
\def\langnames@langs@wals@pan{Eastern Panjabi}
\def\langnames@langs@wals@pao{Northern Paiute}
\def\langnames@langs@wals@pap{Papiamento}
\def\langnames@langs@wals@paq{Parya}
\def\langnames@langs@wals@par{Panamint}
\def\langnames@langs@wals@pas{Papasena}
\def\langnames@langs@wals@pau{Palauan}
\def\langnames@langs@wals@pav{Wari'}
\def\langnames@langs@wals@paw{Pawnee}
\def\langnames@langs@wals@pax{Pankararé}
\def\langnames@langs@wals@pay{Pech}
\def\langnames@langs@wals@paz{Pankararú}
\def\langnames@langs@wals@pbb{Páez}
\def\langnames@langs@wals@pbc{Patamona}
\def\langnames@langs@wals@pbe{Mezontla Popoloca}
\def\langnames@langs@wals@pbf{Coyotepec Popoloca}
\def\langnames@langs@wals@pbg{Paraujano}
\def\langnames@langs@wals@pbh{Panare}
\def\langnames@langs@wals@pbi{Parkwa}
\def\langnames@langs@wals@pbl{Mak (Nigeria)}
\def\langnames@langs@wals@pbm{Puebla and Northeastern Mazatec}
\def\langnames@langs@wals@pbn{Kpasam}
\def\langnames@langs@wals@pbo{Papel}
\def\langnames@langs@wals@pbp{Jaad-Badyara}
\def\langnames@langs@wals@pbr{Pangwa}
\def\langnames@langs@wals@pbs{Central Pame}
\def\langnames@langs@wals@pbt{Southern Pashto}
\def\langnames@langs@wals@pbu{Northern Pashto}
\def\langnames@langs@wals@pbv{Pnar}
\def\langnames@langs@wals@pby{Pyu}
\def\langnames@langs@wals@pca{Santa Inés Ahuatempan Popoloca}
\def\langnames@langs@wals@pcb{Pear}
\def\langnames@langs@wals@pcc{Bouyei}
\def\langnames@langs@wals@pcd{Picard}
\def\langnames@langs@wals@pce{Ruching Palaung}
\def\langnames@langs@wals@pcf{Paliyan}
\def\langnames@langs@wals@pcg{Paniya}
\def\langnames@langs@wals@pch{Pardhan}
\def\langnames@langs@wals@pci{Duruwa}
\def\langnames@langs@wals@pcj{Gorum-Parenga}
\def\langnames@langs@wals@pck{Paite Chin}
\def\langnames@langs@wals@pcl{Pardhi}
\def\langnames@langs@wals@pcm{Nigerian Pidgin}
\def\langnames@langs@wals@pcn{Piti}
\def\langnames@langs@wals@pcp{Pacahuara}
\def\langnames@langs@wals@pcw{Pyapun}
\def\langnames@langs@wals@pda{Anam}
\def\langnames@langs@wals@pdc{Pennsylvania German}
\def\langnames@langs@wals@pdi{Pa Di}
\def\langnames@langs@wals@pdn{Podena}
\def\langnames@langs@wals@pdo{Padoe}
\def\langnames@langs@wals@pdt{Plautdietsch}
\def\langnames@langs@wals@pdu{Kayan Lahwi}
\def\langnames@langs@wals@pea{Peranakan Indonesian}
\def\langnames@langs@wals@peb{Eastern Pomo}
\def\langnames@langs@wals@ped{Mala (Papua New Guinea)}
\def\langnames@langs@wals@pee{Taje}
\def\langnames@langs@wals@pef{Northeastern Russian River Pomo}
\def\langnames@langs@wals@peg{Pengo}
\def\langnames@langs@wals@peh{Bonan}
\def\langnames@langs@wals@pei{Chichimeca-Jonaz}
\def\langnames@langs@wals@pej{Northern Pomo}
\def\langnames@langs@wals@pek{Penchal}
\def\langnames@langs@wals@pel{Pekal}
\def\langnames@langs@wals@pem{Phende}
\def\langnames@langs@wals@peo{Old Persian (ca. 600-400 B.C.)}
\def\langnames@langs@wals@pep{Kánchá}
\def\langnames@langs@wals@peq{Southern Pomo}
\def\langnames@langs@wals@pes{Western Farsi}
\def\langnames@langs@wals@pev{Pémono}
\def\langnames@langs@wals@pex{Petats}
\def\langnames@langs@wals@pey{Petjo}
\def\langnames@langs@wals@pez{Eastern Penan}
\def\langnames@langs@wals@pfa{Pááfang}
\def\langnames@langs@wals@pfe{Peere}
\def\langnames@langs@wals@pfl{Pfaelzisch-Lothringisch}
\def\langnames@langs@wals@pga{South Sudanese Creole Arabic}
\def\langnames@langs@wals@pgd{Gandhari}
\def\langnames@langs@wals@pgg{Pangwali}
\def\langnames@langs@wals@pgi{Pagi}
\def\langnames@langs@wals@pgk{Rerep}
\def\langnames@langs@wals@pgs{Pangseng}
\def\langnames@langs@wals@pgu{Pagu}
\def\langnames@langs@wals@pgz{Papua New Guinean Sign Language}
\def\langnames@langs@wals@pha{Pa-Hng}
\def\langnames@langs@wals@phd{Phudagi}
\def\langnames@langs@wals@phg{Phuong}
\def\langnames@langs@wals@phh{Phukha}
\def\langnames@langs@wals@phj{Pahari Newari}
\def\langnames@langs@wals@phk{Phake}
\def\langnames@langs@wals@phl{Palula}
\def\langnames@langs@wals@phm{Phimbi}
\def\langnames@langs@wals@phn{Phoenician}
\def\langnames@langs@wals@pho{Phunoi}
\def\langnames@langs@wals@phq{Phana'}
\def\langnames@langs@wals@phr{Pahari Potwari}
\def\langnames@langs@wals@pht{Phu Thai}
\def\langnames@langs@wals@phu{Phuan}
\def\langnames@langs@wals@phv{Pahlavani}
\def\langnames@langs@wals@pia{Pima Bajo}
\def\langnames@langs@wals@pib{Yine}
\def\langnames@langs@wals@pic{Pinji}
\def\langnames@langs@wals@pid{Piaroa}
\def\langnames@langs@wals@pie{Piro}
\def\langnames@langs@wals@pif{Pingelapese}
\def\langnames@langs@wals@pig{Pisabo}
\def\langnames@langs@wals@pih{Pitcairn-Norfolk}
\def\langnames@langs@wals@pij{Pijao}
\def\langnames@langs@wals@pil{Yom}
\def\langnames@langs@wals@pim{Powhatan}
\def\langnames@langs@wals@pin{Piame}
\def\langnames@langs@wals@pio{Piapoco}
\def\langnames@langs@wals@pip{Pero}
\def\langnames@langs@wals@pir{Wa'ikhana}
\def\langnames@langs@wals@pis{Pijin}
\def\langnames@langs@wals@pit{Pitta Pitta}
\def\langnames@langs@wals@piu{Pintupi-Luritja}
\def\langnames@langs@wals@piv{Vaeakau-Taumako}
\def\langnames@langs@wals@piw{Pimbwe}
\def\langnames@langs@wals@pix{Piu}
\def\langnames@langs@wals@piy{Piya-Kwonci}
\def\langnames@langs@wals@piz{Pije}
\def\langnames@langs@wals@pjt{Pitjantjatjara}
\def\langnames@langs@wals@pkb{Pokomo}
\def\langnames@langs@wals@pkc{Paekche}
\def\langnames@langs@wals@pkg{Pak-Tong}
\def\langnames@langs@wals@pkh{Pangkhua}
\def\langnames@langs@wals@pkn{Pakanha}
\def\langnames@langs@wals@pko{Pökoot}
\def\langnames@langs@wals@pkp{Pukapuka}
\def\langnames@langs@wals@pkr{Attapady Kurumba}
\def\langnames@langs@wals@pks{Pakistan Sign Language}
\def\langnames@langs@wals@pkt{Maleng}
\def\langnames@langs@wals@pku{Paku}
\def\langnames@langs@wals@pla{Miani}
\def\langnames@langs@wals@plb{Polonombauk}
\def\langnames@langs@wals@plc{Central Palawano}
\def\langnames@langs@wals@pld{Polari}
\def\langnames@langs@wals@ple{Palu'e}
\def\langnames@langs@wals@plg{Pilagá}
\def\langnames@langs@wals@plh{Paulohi}
\def\langnames@langs@wals@pli{Pali}
\def\langnames@langs@wals@plj{Pesse}
\def\langnames@langs@wals@plk{Kohistani Shina}
\def\langnames@langs@wals@pll{Shwe Palaung}
\def\langnames@langs@wals@pln{Palenquero}
\def\langnames@langs@wals@plo{Oluta Popoluca}
\def\langnames@langs@wals@plq{Palaic}
\def\langnames@langs@wals@plr{Palaka Senoufo}
\def\langnames@langs@wals@pls{San Marcos Tlalcoyalco Popoloca}
\def\langnames@langs@wals@plt{Plateau Malagasy}
\def\langnames@langs@wals@plu{Palikúr}
\def\langnames@langs@wals@plv{Southwest Palawano}
\def\langnames@langs@wals@plw{Brooke's Point Palawano}
\def\langnames@langs@wals@ply{Bolyu}
\def\langnames@langs@wals@plz{Paluan}
\def\langnames@langs@wals@pma{Paama}
\def\langnames@langs@wals@pmb{Pambia}
\def\langnames@langs@wals@pmc{Palumata}
\def\langnames@langs@wals@pmd{Pallanganmiddang}
\def\langnames@langs@wals@pme{Pwaamei}
\def\langnames@langs@wals@pmf{Pamona}
\def\langnames@langs@wals@pmh{Maharastri Prakrit}
\def\langnames@langs@wals@pmi{Northern Pumi}
\def\langnames@langs@wals@pmj{Southern Pumi}
\def\langnames@langs@wals@pml{Mediterranean Lingua Franca}
\def\langnames@langs@wals@pmm{Pol}
\def\langnames@langs@wals@pmn{Pam (Cameroon)}
\def\langnames@langs@wals@pmo{Pom}
\def\langnames@langs@wals@pmq{Northern Pame}
\def\langnames@langs@wals@pmr{Manat}
\def\langnames@langs@wals@pms{Piemontese}
\def\langnames@langs@wals@pmt{Tuamotuan}
\def\langnames@langs@wals@pmw{Plains Miwok}
\def\langnames@langs@wals@pmx{Poumei Naga}
\def\langnames@langs@wals@pmy{Papuan Malay}
\def\langnames@langs@wals@pmz{Southern Pame}
\def\langnames@langs@wals@pna{Punan Bah-Biau}
\def\langnames@langs@wals@pnb{Western Panjabi}
\def\langnames@langs@wals@pnc{Pannei}
\def\langnames@langs@wals@pnd{Mpinda}
\def\langnames@langs@wals@pne{Western Penan}
\def\langnames@langs@wals@png{Pongu}
\def\langnames@langs@wals@pnh{Māngarongaro}
\def\langnames@langs@wals@pni{Aoheng-Seputan}
\def\langnames@langs@wals@pnk{Paunaka}
\def\langnames@langs@wals@pnl{Palen}
\def\langnames@langs@wals@pnm{Punan Batu 1}
\def\langnames@langs@wals@pnn{Pinai-Hagahai}
\def\langnames@langs@wals@pno{Panobo}
\def\langnames@langs@wals@pnp{Pancana}
\def\langnames@langs@wals@pnq{Pana (Burkina Faso)}
\def\langnames@langs@wals@pnr{Panim}
\def\langnames@langs@wals@pns{Ponosakan}
\def\langnames@langs@wals@pnt{Pontic}
\def\langnames@langs@wals@pnu{Jiongnai Bunu}
\def\langnames@langs@wals@pnv{Pinigura}
\def\langnames@langs@wals@pnw{Panytyima}
\def\langnames@langs@wals@pnx{Phong-Kniang}
\def\langnames@langs@wals@pny{Pinyin}
\def\langnames@langs@wals@pnz{Pana (Central African Republic)}
\def\langnames@langs@wals@poc{Poqomam}
\def\langnames@langs@wals@poe{San Juan Atzingo Popoloca}
\def\langnames@langs@wals@pof{Poke}
\def\langnames@langs@wals@poh{Poqomchi'}
\def\langnames@langs@wals@poi{Highland Popoluca}
\def\langnames@langs@wals@pol{Polish}
\def\langnames@langs@wals@pom{Southeastern Pomo}
\def\langnames@langs@wals@pon{Pohnpeian}
\def\langnames@langs@wals@poo{Central Pomo}
\def\langnames@langs@wals@pop{Pwapwa}
\def\langnames@langs@wals@poq{Texistepec Popoluca}
\def\langnames@langs@wals@por{Portuguese}
\def\langnames@langs@wals@pos{Sayula Popoluca}
\def\langnames@langs@wals@pot{Potawatomi}
\def\langnames@langs@wals@pov{Upper Guinea Crioulo}
\def\langnames@langs@wals@pow{San Felipe Otlaltepec Popoloca}
\def\langnames@langs@wals@pox{Polabian}
\def\langnames@langs@wals@poy{Pogolo}
\def\langnames@langs@wals@ppe{Papi}
\def\langnames@langs@wals@ppi{Paipai}
\def\langnames@langs@wals@ppk{Uma}
\def\langnames@langs@wals@ppl{Pipil}
\def\langnames@langs@wals@ppm{Papuma}
\def\langnames@langs@wals@ppn{Papapana}
\def\langnames@langs@wals@ppo{Folopa}
\def\langnames@langs@wals@ppq{Pei}
\def\langnames@langs@wals@pps{San Luís Temalacayuca Popoloca}
\def\langnames@langs@wals@ppt{Pare}
\def\langnames@langs@wals@ppu{Papora-Hoanya}
\def\langnames@langs@wals@ppv{Papavô}
\def\langnames@langs@wals@pqa{Pa'a}
\def\langnames@langs@wals@pqm{Malecite-Passamaquoddy}
\def\langnames@langs@wals@prc{Parachi}
\def\langnames@langs@wals@pre{Principense}
\def\langnames@langs@wals@prf{Paranan}
\def\langnames@langs@wals@prg{Old Prussian}
\def\langnames@langs@wals@prh{Porohanon}
\def\langnames@langs@wals@pri{Paicî}
\def\langnames@langs@wals@prk{South Wa}
\def\langnames@langs@wals@prl{Peruvian Sign Language}
\def\langnames@langs@wals@prm{Kibiri}
\def\langnames@langs@wals@prn{Prasun}
\def\langnames@langs@wals@pro{Old Provençal}
\def\langnames@langs@wals@prq{Ashéninka Perené}
\def\langnames@langs@wals@prr{Puri}
\def\langnames@langs@wals@prs{Dari}
\def\langnames@langs@wals@prt{Prai}
\def\langnames@langs@wals@pru{Puragi}
\def\langnames@langs@wals@prw{Parawen}
\def\langnames@langs@wals@prx{Purik-Sham-Nubra}
\def\langnames@langs@wals@prz{Providencia Sign Language}
\def\langnames@langs@wals@psa{Asue Awyu}
\def\langnames@langs@wals@psc{Zaban Eshareh Irani}
\def\langnames@langs@wals@psd{Plains Indian Sign Language}
\def\langnames@langs@wals@pse{South Barisan Malay}
\def\langnames@langs@wals@psg{Penang Sign Language}
\def\langnames@langs@wals@psh{Southwest Pashayi}
\def\langnames@langs@wals@psi{Southeast Pashayi}
\def\langnames@langs@wals@psl{Puerto Rican Sign Language}
\def\langnames@langs@wals@psm{Warázu}
\def\langnames@langs@wals@psn{Panasuan}
\def\langnames@langs@wals@pso{Polish Sign Language}
\def\langnames@langs@wals@psp{Philippine Sign Language}
\def\langnames@langs@wals@psq{Pasi}
\def\langnames@langs@wals@psr{Portuguese Sign Language}
\def\langnames@langs@wals@pss{Kaulong}
\def\langnames@langs@wals@pst{Central Pashto}
\def\langnames@langs@wals@psu{Sauraseni Prakrit}
\def\langnames@langs@wals@psw{Port Sandwich}
\def\langnames@langs@wals@psy{Piscataway}
\def\langnames@langs@wals@pta{Pai Tavytera}
\def\langnames@langs@wals@pth{Pataxó Hã-Ha-Hãe}
\def\langnames@langs@wals@pti{Pintiini}
\def\langnames@langs@wals@ptn{Patani}
\def\langnames@langs@wals@pto{Zo'é}
\def\langnames@langs@wals@ptp{Patep}
\def\langnames@langs@wals@ptq{Pattapu}
\def\langnames@langs@wals@ptr{Piamatsina}
\def\langnames@langs@wals@ptt{Enrekang}
\def\langnames@langs@wals@ptu{Bambam}
\def\langnames@langs@wals@ptv{Daakie}
\def\langnames@langs@wals@ptw{Pentlatch}
\def\langnames@langs@wals@pty{Pathiya}
\def\langnames@langs@wals@pua{Western Highland Purepecha}
\def\langnames@langs@wals@pub{Purum}
\def\langnames@langs@wals@pud{Punan Aput}
\def\langnames@langs@wals@pue{Puelche}
\def\langnames@langs@wals@puf{Punan Merah}
\def\langnames@langs@wals@pug{Phuie}
\def\langnames@langs@wals@pui{Puinave}
\def\langnames@langs@wals@puj{Punan Tubu}
\def\langnames@langs@wals@pum{Puma}
\def\langnames@langs@wals@puo{Ksingmul}
\def\langnames@langs@wals@pup{Pulabu}
\def\langnames@langs@wals@puq{Puquina}
\def\langnames@langs@wals@pur{Puruborá}
\def\langnames@langs@wals@puu{Punu}
\def\langnames@langs@wals@puw{Puluwatese}
\def\langnames@langs@wals@pux{Puare}
\def\langnames@langs@wals@puy{Purisimeño}
\def\langnames@langs@wals@pwa{Pawaia}
\def\langnames@langs@wals@pwb{Panawa}
\def\langnames@langs@wals@pwg{Gapapaiwa}
\def\langnames@langs@wals@pwi{Patwin}
\def\langnames@langs@wals@pwm{Molbog}
\def\langnames@langs@wals@pwn{Paiwan}
\def\langnames@langs@wals@pwo{Pwo Western Karen}
\def\langnames@langs@wals@pwr{Powari}
\def\langnames@langs@wals@pww{Pwo Northern Karen}
\def\langnames@langs@wals@pye{Pye Krumen}
\def\langnames@langs@wals@pym{Fyam}
\def\langnames@langs@wals@pyn{Poyanáwa}
\def\langnames@langs@wals@pys{Paraguayan Sign Language}
\def\langnames@langs@wals@pyu{Puyuma}
\def\langnames@langs@wals@pyx{Burma Pyu}
\def\langnames@langs@wals@pyy{Pyen}
\def\langnames@langs@wals@pzh{Pazeh-Kahabu}
\def\langnames@langs@wals@pzn{Jejara Naga}
\def\langnames@langs@wals@qbb{Old Latin}
\def\langnames@langs@wals@qcs{Tapachultec}
\def\langnames@langs@wals@qer{Dalecarlian}
\def\langnames@langs@wals@qgu{Wulguru}
\def\langnames@langs@wals@qhr{Old Sabellic}
\def\langnames@langs@wals@qkn{Old Kannada}
\def\langnames@langs@wals@qlm{Limonese Creole}
\def\langnames@langs@wals@qok{Old Khmer}
\def\langnames@langs@wals@qpp{Paisaci Prakrit}
\def\langnames@langs@wals@qua{Quapaw}
\def\langnames@langs@wals@qub{Huallaga Huánuco Quechua}
\def\langnames@langs@wals@quc{K'iche'}
\def\langnames@langs@wals@qud{Calderón Highland Quichua}
\def\langnames@langs@wals@quf{Lambayeque Quechua}
\def\langnames@langs@wals@qug{Bolivar-North Chimborazo Highland Quichua}
\def\langnames@langs@wals@quh{South Bolivian Quechua}
\def\langnames@langs@wals@qui{Quileute}
\def\langnames@langs@wals@quk{Chachapoyas Quechua}
\def\langnames@langs@wals@qul{North Bolivian Quechua}
\def\langnames@langs@wals@qum{Sipacapense}
\def\langnames@langs@wals@qun{Quinault}
\def\langnames@langs@wals@qup{Southern Pastaza Quechua}
\def\langnames@langs@wals@quq{Quinqui}
\def\langnames@langs@wals@qur{Chaupihuaranga Quechua}
\def\langnames@langs@wals@qus{Santiago del Estero Quichua}
\def\langnames@langs@wals@quv{Sacapulteco}
\def\langnames@langs@wals@quw{Tena Lowland Quichua}
\def\langnames@langs@wals@qux{Yauyos Quechua}
\def\langnames@langs@wals@quy{Ayacucho Quechua}
\def\langnames@langs@wals@quz{Cusco Quechua}
\def\langnames@langs@wals@qva{Ambo-Pasco Quechua}
\def\langnames@langs@wals@qvc{Cajamarca Quechua}
\def\langnames@langs@wals@qve{Eastern Apurímac Quechua}
\def\langnames@langs@wals@qvh{Huamalíes-Dos de Mayo Huánuco Quechua}
\def\langnames@langs@wals@qvi{Imbabura Highland Quichua}
\def\langnames@langs@wals@qvj{Loja Highland Quichua}
\def\langnames@langs@wals@qvl{Cajatambo North Lima Quechua}
\def\langnames@langs@wals@qvm{Margos-Yarowilca-Lauricocha Quechua}
\def\langnames@langs@wals@qvn{North Junín Quechua}
\def\langnames@langs@wals@qvo{Napo Lowland Quechua}
\def\langnames@langs@wals@qvp{Pacaraos Quechua}
\def\langnames@langs@wals@qvs{San Martín Quechua}
\def\langnames@langs@wals@qvw{Huaylla Wanca Quechua}
\def\langnames@langs@wals@qvy{Queyu}
\def\langnames@langs@wals@qvz{Northern Pastaza Quichua}
\def\langnames@langs@wals@qwa{Corongo Ancash Quechua}
\def\langnames@langs@wals@qwc{Classical Quechua}
\def\langnames@langs@wals@qwh{Huaylas Ancash Quechua}
\def\langnames@langs@wals@qws{Sihuas Ancash Quechua}
\def\langnames@langs@wals@qwt{Kwalhioqua-Clatskanie}
\def\langnames@langs@wals@qxa{Chiquián Ancash Quechua}
\def\langnames@langs@wals@qxc{Chincha Quechua}
\def\langnames@langs@wals@qxh{Panao Huánuco Quechua}
\def\langnames@langs@wals@qxl{Tungurahua Highland Quichua}
\def\langnames@langs@wals@qxn{Northern Conchucos Ancash Quechua}
\def\langnames@langs@wals@qxo{Southern Conchucos Ancash Quechua}
\def\langnames@langs@wals@qxp{Puno Quechua}
\def\langnames@langs@wals@qxq{Qashqa'i}
\def\langnames@langs@wals@qxr{Cañar-Azuay-South Chimborazo Highland Quichua}
\def\langnames@langs@wals@qxs{Southern Qiang}
\def\langnames@langs@wals@qxu{Arequipa-La Unión Quechua}
\def\langnames@langs@wals@qxw{Jauja Wanca Quechua}
\def\langnames@langs@wals@qya{Quenya}
\def\langnames@langs@wals@qyp{Wampano}
\def\langnames@langs@wals@raa{Dungmali}
\def\langnames@langs@wals@rab{Camling}
\def\langnames@langs@wals@rac{Rasawa}
\def\langnames@langs@wals@rad{Rade}
\def\langnames@langs@wals@raf{Western Meohang}
\def\langnames@langs@wals@rag{Logooli}
\def\langnames@langs@wals@rah{Rabha}
\def\langnames@langs@wals@rai{Ramoaaina}
\def\langnames@langs@wals@rak{Tulu-Bohuai}
\def\langnames@langs@wals@ral{Ralte}
\def\langnames@langs@wals@ram{Canela-Krahô}
\def\langnames@langs@wals@ran{Riantana}
\def\langnames@langs@wals@rao{Rao}
\def\langnames@langs@wals@rap{Rapanui}
\def\langnames@langs@wals@raq{Saam}
\def\langnames@langs@wals@rar{Southern Cook Island Maori}
\def\langnames@langs@wals@ras{Tegali}
\def\langnames@langs@wals@rat{Razajerdi}
\def\langnames@langs@wals@rau{Raute}
\def\langnames@langs@wals@rav{Sampang}
\def\langnames@langs@wals@raw{Rawang}
\def\langnames@langs@wals@rax{Rang}
\def\langnames@langs@wals@ray{Mangaia-Old Rapa}
\def\langnames@langs@wals@raz{Rahambuu}
\def\langnames@langs@wals@rbb{Rumai Palaung}
\def\langnames@langs@wals@rcf{Réunion Creole French}
\def\langnames@langs@wals@rdb{Rudbari}
\def\langnames@langs@wals@rea{Rerau}
\def\langnames@langs@wals@reb{Rembong-Wangka}
\def\langnames@langs@wals@ree{Rejang Kayan}
\def\langnames@langs@wals@reg{Kara (Tanzania)}
\def\langnames@langs@wals@rei{Reli}
\def\langnames@langs@wals@rej{Rejang}
\def\langnames@langs@wals@rel{Rendille}
\def\langnames@langs@wals@rem{Remo of the Moa river}
\def\langnames@langs@wals@ren{Rengao}
\def\langnames@langs@wals@rer{Rer Bare}
\def\langnames@langs@wals@res{Reshe}
\def\langnames@langs@wals@ret{Reta}
\def\langnames@langs@wals@rey{Reyesano}
\def\langnames@langs@wals@rga{Mores}
\def\langnames@langs@wals@rge{Romano-Greek}
\def\langnames@langs@wals@rgk{Rangkas}
\def\langnames@langs@wals@rgn{Romagnol}
\def\langnames@langs@wals@rgr{Resígaro}
\def\langnames@langs@wals@rgs{Southern Roglai}
\def\langnames@langs@wals@rgu{Ringgou}
\def\langnames@langs@wals@rhg{Rohingya}
\def\langnames@langs@wals@rhp{Yahang}
\def\langnames@langs@wals@ria{Riang (India)}
\def\langnames@langs@wals@rib{Bribri Sign Language}
\def\langnames@langs@wals@rif{Tarifiyt-Beni-Iznasen-Eastern Middle Atlas Berber}
\def\langnames@langs@wals@ril{Riang (Myanmar)}
\def\langnames@langs@wals@rim{Nyaturu}
\def\langnames@langs@wals@rin{Nungu}
\def\langnames@langs@wals@rir{Ribun}
\def\langnames@langs@wals@rit{Ritarungo}
\def\langnames@langs@wals@riu{Riung}
\def\langnames@langs@wals@rjg{Rajong}
\def\langnames@langs@wals@rji{Raji}
\def\langnames@langs@wals@rjs{Rajbanshi}
\def\langnames@langs@wals@rka{Kraol}
\def\langnames@langs@wals@rkb{Rikbaktsa}
\def\langnames@langs@wals@rkh{Rakahanga-Manihiki}
\def\langnames@langs@wals@rki{Rakhine}
\def\langnames@langs@wals@rkm{Marka}
\def\langnames@langs@wals@rkt{Central-Eastern Kamta}
\def\langnames@langs@wals@rma{Rama}
\def\langnames@langs@wals@rmb{Rembarrnga}
\def\langnames@langs@wals@rmc{Central Romani}
\def\langnames@langs@wals@rmd{Traveller Danish}
\def\langnames@langs@wals@rme{Archaic Angloromani}
\def\langnames@langs@wals@rmf{Kalo Finnish Romani}
\def\langnames@langs@wals@rmg{Traveller Norwegian}
\def\langnames@langs@wals@rmh{Murkim}
\def\langnames@langs@wals@rmi{Lomavren}
\def\langnames@langs@wals@rmk{Romkun}
\def\langnames@langs@wals@rml{Baltic Romani}
\def\langnames@langs@wals@rmm{Roma}
\def\langnames@langs@wals@rmn{Balkan Romani}
\def\langnames@langs@wals@rmo{Sinte-Manus Romani}
\def\langnames@langs@wals@rmp{Rempi}
\def\langnames@langs@wals@rmq{Caló}
\def\langnames@langs@wals@rms{Romanian Sign Language}
\def\langnames@langs@wals@rmt{Domari}
\def\langnames@langs@wals@rmu{Tavringer Romani}
\def\langnames@langs@wals@rmv{Romanova}
\def\langnames@langs@wals@rmw{Welsh Romani}
\def\langnames@langs@wals@rmx{Romam}
\def\langnames@langs@wals@rmy{Vlax Romani}
\def\langnames@langs@wals@rmz{Marma}
\def\langnames@langs@wals@rna{Runa}
\def\langnames@langs@wals@rnb{Brunca Sign Language}
\def\langnames@langs@wals@rnd{Ruund}
\def\langnames@langs@wals@rng{Ronga}
\def\langnames@langs@wals@rnl{Halam}
\def\langnames@langs@wals@rnn{Roon}
\def\langnames@langs@wals@rnp{Rongpo}
\def\langnames@langs@wals@rnw{Rungwa}
\def\langnames@langs@wals@rob{Tae'}
\def\langnames@langs@wals@roc{Cacgia Roglai}
\def\langnames@langs@wals@rod{Rogo}
\def\langnames@langs@wals@roe{Ronji}
\def\langnames@langs@wals@rof{Rombo}
\def\langnames@langs@wals@rog{Northern Roglai}
\def\langnames@langs@wals@roh{Romansh}
\def\langnames@langs@wals@rol{Romblomanon}
\def\langnames@langs@wals@ron{Romanian}
\def\langnames@langs@wals@roo{Rotokas}
\def\langnames@langs@wals@rop{Kriol}
\def\langnames@langs@wals@ror{Rongga}
\def\langnames@langs@wals@rou{Runga}
\def\langnames@langs@wals@row{Dela-Oenale}
\def\langnames@langs@wals@rpt{Rapting}
\def\langnames@langs@wals@rri{Ririo}
\def\langnames@langs@wals@rro{Waima}
\def\langnames@langs@wals@rsi{Rennellese Sign Language}
\def\langnames@langs@wals@rsl{Russian-Tajik Sign Language}
\def\langnames@langs@wals@rsm{Miriwoong Sign Language}
\def\langnames@langs@wals@rsn{Rwandan Sign Language}
\def\langnames@langs@wals@rth{Ratahan}
\def\langnames@langs@wals@rtm{Rotuman}
\def\langnames@langs@wals@rtw{Rathawi}
\def\langnames@langs@wals@rub{Gungu}
\def\langnames@langs@wals@ruc{Ruuli}
\def\langnames@langs@wals@rue{Rusyn}
\def\langnames@langs@wals@ruf{Luguru}
\def\langnames@langs@wals@rug{Roviana}
\def\langnames@langs@wals@ruh{Ruga}
\def\langnames@langs@wals@ruk{Che}
\def\langnames@langs@wals@run{Rundi}
\def\langnames@langs@wals@ruo{Istro Romanian}
\def\langnames@langs@wals@rup{Aromanian}
\def\langnames@langs@wals@ruq{Megleno Romanian}
\def\langnames@langs@wals@rus{Russian}
\def\langnames@langs@wals@rut{Rutul}
\def\langnames@langs@wals@ruu{Lanas Lobu}
\def\langnames@langs@wals@ruy{Mala (Nigeria)}
\def\langnames@langs@wals@ruz{Ruma}
\def\langnames@langs@wals@rwa{Rawo}
\def\langnames@langs@wals@rwk{Rwa}
\def\langnames@langs@wals@rwm{Amba (Uganda)}
\def\langnames@langs@wals@rwo{Rawa}
\def\langnames@langs@wals@rwr{Marwari (India)}
\def\langnames@langs@wals@rxd{Ngardi}
\def\langnames@langs@wals@rxw{Karruwali}
\def\langnames@langs@wals@ryn{Northern Amami-Oshima}
\def\langnames@langs@wals@rys{Yaeyama}
\def\langnames@langs@wals@ryu{Central Okinawan}
\def\langnames@langs@wals@rzh{Jabal Razih}
\def\langnames@langs@wals@saa{Saba}
\def\langnames@langs@wals@sab{Buglere}
\def\langnames@langs@wals@sac{Meskwaki}
\def\langnames@langs@wals@sad{Sandawe}
\def\langnames@langs@wals@sae{Sabanê}
\def\langnames@langs@wals@saf{Safaliba}
\def\langnames@langs@wals@sag{Sango}
\def\langnames@langs@wals@sah{Sakha}
\def\langnames@langs@wals@saj{Sahu}
\def\langnames@langs@wals@sak{Sake}
\def\langnames@langs@wals@san{Sanskrit}
\def\langnames@langs@wals@sao{Sause}
\def\langnames@langs@wals@saq{Samburu}
\def\langnames@langs@wals@sar{Saraveca}
\def\langnames@langs@wals@sas{Sasak}
\def\langnames@langs@wals@sat{Santali}
\def\langnames@langs@wals@sau{Saleman}
\def\langnames@langs@wals@sav{Saafi-Saafi}
\def\langnames@langs@wals@saw{Sawi}
\def\langnames@langs@wals@sax{Sa}
\def\langnames@langs@wals@say{Saya}
\def\langnames@langs@wals@saz{Saurashtra}
\def\langnames@langs@wals@sba{Ngambay}
\def\langnames@langs@wals@sbb{Simbo}
\def\langnames@langs@wals@sbc{Kele (Papua New Guinea)}
\def\langnames@langs@wals@sbd{Southern Samo}
\def\langnames@langs@wals@sbe{Saliba}
\def\langnames@langs@wals@sbf{Shabo}
\def\langnames@langs@wals@sbg{Seget}
\def\langnames@langs@wals@sbh{Sori-Harengan}
\def\langnames@langs@wals@sbi{Seti}
\def\langnames@langs@wals@sbj{Surbakhal}
\def\langnames@langs@wals@sbk{Safwa}
\def\langnames@langs@wals@sbl{Botolan Sambal}
\def\langnames@langs@wals@sbm{Sagala}
\def\langnames@langs@wals@sbn{Sindhi Bhil}
\def\langnames@langs@wals@sbo{Sabüm}
\def\langnames@langs@wals@sbp{Sangu (Tanzania)}
\def\langnames@langs@wals@sbq{Sirva}
\def\langnames@langs@wals@sbr{Sembakung Murut}
\def\langnames@langs@wals@sbs{Subiya}
\def\langnames@langs@wals@sbt{Kimki}
\def\langnames@langs@wals@sbu{Stod Bhoti}
\def\langnames@langs@wals@sbw{Simba}
\def\langnames@langs@wals@sbx{Seberuang}
\def\langnames@langs@wals@sby{Soli}
\def\langnames@langs@wals@sbz{Sara Kaba}
\def\langnames@langs@wals@scb{Chut}
\def\langnames@langs@wals@sce{Dongxiang}
\def\langnames@langs@wals@scg{Sanggau}
\def\langnames@langs@wals@sch{Sakachep-Chorei}
\def\langnames@langs@wals@sci{Sri Lanka Malay}
\def\langnames@langs@wals@sck{Sadri}
\def\langnames@langs@wals@scl{Shina}
\def\langnames@langs@wals@scn{Sicilian}
\def\langnames@langs@wals@sco{Scots}
\def\langnames@langs@wals@scp{Lamjung-Melamchi Yolmo}
\def\langnames@langs@wals@scq{Sa'och}
\def\langnames@langs@wals@scs{North Slavey}
\def\langnames@langs@wals@sct{Southern Katang}
\def\langnames@langs@wals@scu{Shumcho}
\def\langnames@langs@wals@scv{Sheni-Ziriya}
\def\langnames@langs@wals@scw{Sya}
\def\langnames@langs@wals@scx{Sicula}
\def\langnames@langs@wals@sda{Toraja-Sa'dan}
\def\langnames@langs@wals@sdb{Shabaki}
\def\langnames@langs@wals@sdc{Sassarese Sardinian}
\def\langnames@langs@wals@sde{Vori}
\def\langnames@langs@wals@sdg{Savi}
\def\langnames@langs@wals@sdh{Southern Kurdish}
\def\langnames@langs@wals@sdj{Suundi}
\def\langnames@langs@wals@sdk{Sos Kundi}
\def\langnames@langs@wals@sdl{Saudi Arabian Sign Language}
\def\langnames@langs@wals@sdm{Onya Darat}
\def\langnames@langs@wals@sdn{Gallurese Sardinian}
\def\langnames@langs@wals@sdo{Bukar-Sadung Bidayuh}
\def\langnames@langs@wals@sdp{Sherdukpen}
\def\langnames@langs@wals@sdr{Oraon Sadri}
\def\langnames@langs@wals@sds{Sened}
\def\langnames@langs@wals@sdu{Sarudu}
\def\langnames@langs@wals@sdx{Sibu Melanau}
\def\langnames@langs@wals@sea{Semai}
\def\langnames@langs@wals@sec{Sechelt}
\def\langnames@langs@wals@sed{Sedang}
\def\langnames@langs@wals@see{Seneca}
\def\langnames@langs@wals@sef{Senari}
\def\langnames@langs@wals@seg{Segeju}
\def\langnames@langs@wals@seh{Sena}
\def\langnames@langs@wals@sei{Seri}
\def\langnames@langs@wals@sej{Sene}
\def\langnames@langs@wals@sek{Sekani}
\def\langnames@langs@wals@sel{Selkup}
\def\langnames@langs@wals@sen{Nanerigé Sénoufo}
\def\langnames@langs@wals@seo{Asabano}
\def\langnames@langs@wals@sep{Sìcìté Sénoufo}
\def\langnames@langs@wals@seq{Senar de Kankalaba}
\def\langnames@langs@wals@ser{Serrano}
\def\langnames@langs@wals@ses{Koyraboro Senni Songhai}
\def\langnames@langs@wals@set{Sentani}
\def\langnames@langs@wals@seu{Serui-Laut}
\def\langnames@langs@wals@sev{Nyarafolo Senoufo}
\def\langnames@langs@wals@sew{Sewa Bay}
\def\langnames@langs@wals@sey{Secoya}
\def\langnames@langs@wals@sez{Senthang Chin}
\def\langnames@langs@wals@sfb{Langue des signes de Belgique Francophone}
\def\langnames@langs@wals@sfe{Eastern Subanen}
\def\langnames@langs@wals@sfm{Gha-mu}
\def\langnames@langs@wals@sfs{South African Sign Language}
\def\langnames@langs@wals@sfw{Sehwi}
\def\langnames@langs@wals@sga{Early Irish}
\def\langnames@langs@wals@sgb{Mag-Anchi Ayta}
\def\langnames@langs@wals@sgc{Kipsigis}
\def\langnames@langs@wals@sgd{Surigaonon}
\def\langnames@langs@wals@sge{Segai}
\def\langnames@langs@wals@sgg{Swiss-German Sign Language}
\def\langnames@langs@wals@sgh{Shughni}
\def\langnames@langs@wals@sgi{Nizaa}
\def\langnames@langs@wals@sgk{Sangkong}
\def\langnames@langs@wals@sgm{Singa}
\def\langnames@langs@wals@sgp{Northern Jinghpaw}
\def\langnames@langs@wals@sgr{Sangisari}
\def\langnames@langs@wals@sgt{Brokpake}
\def\langnames@langs@wals@sgu{Salas}
\def\langnames@langs@wals@sgw{Sebat Bet Gurage}
\def\langnames@langs@wals@sgx{Sierra Leone Sign Language}
\def\langnames@langs@wals@sgy{Sanglechi}
\def\langnames@langs@wals@sgz{Sursurunga}
\def\langnames@langs@wals@sha{Shall-Zwall}
\def\langnames@langs@wals@shb{Ninam}
\def\langnames@langs@wals@shc{Sonde}
\def\langnames@langs@wals@shd{Kundal Shahi}
\def\langnames@langs@wals@she{Sheko}
\def\langnames@langs@wals@shg{Shua}
\def\langnames@langs@wals@shh{Shoshoni}
\def\langnames@langs@wals@shi{Tachelhit}
\def\langnames@langs@wals@shj{Shatt}
\def\langnames@langs@wals@shk{Shilluk}
\def\langnames@langs@wals@shl{Shendu}
\def\langnames@langs@wals@shm{Shahrudi-Southern Talysh}
\def\langnames@langs@wals@shn{Shan}
\def\langnames@langs@wals@sho{Shanga}
\def\langnames@langs@wals@shp{Shipibo-Conibo}
\def\langnames@langs@wals@shq{Sala}
\def\langnames@langs@wals@shr{Shi}
\def\langnames@langs@wals@shs{Shuswap}
\def\langnames@langs@wals@sht{Shasta}
\def\langnames@langs@wals@shu{Chadian Arabic}
\def\langnames@langs@wals@shv{Jibbali}
\def\langnames@langs@wals@shw{Shwai}
\def\langnames@langs@wals@shx{She}
\def\langnames@langs@wals@shy{Chaouia of the Aures}
\def\langnames@langs@wals@sia{Akkala Saami}
\def\langnames@langs@wals@sib{Sebop}
\def\langnames@langs@wals@sid{Sidamo}
\def\langnames@langs@wals@sie{Simaa}
\def\langnames@langs@wals@sif{Siamou}
\def\langnames@langs@wals@sig{Paasaal}
\def\langnames@langs@wals@sih{Zire}
\def\langnames@langs@wals@sii{Shom Peng}
\def\langnames@langs@wals@sij{Numbami}
\def\langnames@langs@wals@sil{Tumulung Sisaala}
\def\langnames@langs@wals@sim{Mende (Papua New Guinea)}
\def\langnames@langs@wals@sin{Sinhala}
\def\langnames@langs@wals@sip{Sikkimese}
\def\langnames@langs@wals@siq{Sonia}
\def\langnames@langs@wals@sir{Siri}
\def\langnames@langs@wals@sis{Siuslaw}
\def\langnames@langs@wals@siu{Galu}
\def\langnames@langs@wals@siv{Sumariup}
\def\langnames@langs@wals@siw{Siwai}
\def\langnames@langs@wals@six{Sumau}
\def\langnames@langs@wals@siy{Sivandi}
\def\langnames@langs@wals@siz{Siwi}
\def\langnames@langs@wals@sja{Epena}
\def\langnames@langs@wals@sjb{Sajau-Latti}
\def\langnames@langs@wals@sjd{Kildin Saami}
\def\langnames@langs@wals@sje{Pite Saami}
\def\langnames@langs@wals@sjg{Assangori}
\def\langnames@langs@wals@sjk{Kemi Saami}
\def\langnames@langs@wals@sjl{Sajolang}
\def\langnames@langs@wals@sjm{Mapun}
\def\langnames@langs@wals@sjn{Sindarin}
\def\langnames@langs@wals@sjo{Xibe}
\def\langnames@langs@wals@sjp{Surjapuri}
\def\langnames@langs@wals@sjr{Siar-Lak}
\def\langnames@langs@wals@sjs{Senhaja De Srair}
\def\langnames@langs@wals@sjt{Ter Saami}
\def\langnames@langs@wals@sju{Ume Saami}
\def\langnames@langs@wals@sjw{Shawnee}
\def\langnames@langs@wals@skb{Saek}
\def\langnames@langs@wals@skc{Ma Manda}
\def\langnames@langs@wals@skd{Southern Sierra Miwok}
\def\langnames@langs@wals@ske{Seke (Vanuatu)}
\def\langnames@langs@wals@skf{Mekens}
\def\langnames@langs@wals@skg{West Malagasy Sakalava}
\def\langnames@langs@wals@skh{Sikule}
\def\langnames@langs@wals@ski{Sika}
\def\langnames@langs@wals@skj{Seke (Nepal)}
\def\langnames@langs@wals@skm{Sakam}
\def\langnames@langs@wals@skn{Kolibugan Subanon}
\def\langnames@langs@wals@sko{Seko Tengah}
\def\langnames@langs@wals@skp{Sekapan}
\def\langnames@langs@wals@skq{Sininkere}
\def\langnames@langs@wals@skr{Saraiki}
\def\langnames@langs@wals@sks{Maia}
\def\langnames@langs@wals@skt{Sakata}
\def\langnames@langs@wals@sku{Wanohe}
\def\langnames@langs@wals@skv{Skou}
\def\langnames@langs@wals@skw{Skepi Creole Dutch}
\def\langnames@langs@wals@skx{Seko Padang}
\def\langnames@langs@wals@sky{Sikaiana}
\def\langnames@langs@wals@skz{Sekar}
\def\langnames@langs@wals@slc{Sáliba}
\def\langnames@langs@wals@sld{Sissala of Burkina Faso}
\def\langnames@langs@wals@sle{Sholaga}
\def\langnames@langs@wals@slf{Swiss-Italian Sign Language}
\def\langnames@langs@wals@slg{Selungai Murut}
\def\langnames@langs@wals@slh{Southern Puget Sound Salish}
\def\langnames@langs@wals@sli{Lower Silesian}
\def\langnames@langs@wals@slk{Slovak}
\def\langnames@langs@wals@sll{Salt-Yui}
\def\langnames@langs@wals@slm{Pangutaran Sama}
\def\langnames@langs@wals@sln{Salinan}
\def\langnames@langs@wals@slp{Lamaholot}
\def\langnames@langs@wals@slq{Salchuq}
\def\langnames@langs@wals@slr{Salar}
\def\langnames@langs@wals@slt{Sila}
\def\langnames@langs@wals@slu{Selaru}
\def\langnames@langs@wals@slv{Slovenian}
\def\langnames@langs@wals@slw{Sialum}
\def\langnames@langs@wals@slx{Salampasu}
\def\langnames@langs@wals@sly{Selayar}
\def\langnames@langs@wals@slz{Misool-Salawati Ma'ya}
\def\langnames@langs@wals@sma{South Saami}
\def\langnames@langs@wals@smb{Simbari}
\def\langnames@langs@wals@smc{Som}
\def\langnames@langs@wals@sme{North Saami}
\def\langnames@langs@wals@smf{Auwe}
\def\langnames@langs@wals@smg{Simbali}
\def\langnames@langs@wals@smh{Samei}
\def\langnames@langs@wals@smj{Lule Saami}
\def\langnames@langs@wals@smk{Bolinao}
\def\langnames@langs@wals@sml{Central Sama}
\def\langnames@langs@wals@smm{Musasa}
\def\langnames@langs@wals@smn{Inari Saami}
\def\langnames@langs@wals@smo{Samoan}
\def\langnames@langs@wals@smp{Samaritan}
\def\langnames@langs@wals@smq{Samo}
\def\langnames@langs@wals@smr{Simeulue}
\def\langnames@langs@wals@sms{Skolt Saami}
\def\langnames@langs@wals@smt{Simte}
\def\langnames@langs@wals@smu{Somray of Battambang-Somre of Siem Reap}
\def\langnames@langs@wals@smv{Samvedi}
\def\langnames@langs@wals@smw{Sumbawa}
\def\langnames@langs@wals@smx{Samba}
\def\langnames@langs@wals@smy{Semnani-Biyabuneki}
\def\langnames@langs@wals@smz{Simeku}
\def\langnames@langs@wals@sna{Shona}
\def\langnames@langs@wals@snc{Sinaugoro}
\def\langnames@langs@wals@snd{Sindhi}
\def\langnames@langs@wals@sne{Bau-Jagoi Bidayuh}
\def\langnames@langs@wals@snf{Noon}
\def\langnames@langs@wals@sng{Sanga (Democratic Republic of Congo)}
\def\langnames@langs@wals@snh{Shinabo}
\def\langnames@langs@wals@sni{Sensi}
\def\langnames@langs@wals@snj{Riverain Sango}
\def\langnames@langs@wals@snk{Soninke}
\def\langnames@langs@wals@snl{Sangil}
\def\langnames@langs@wals@snm{Southern Ma'di}
\def\langnames@langs@wals@snn{Siona-Tetete}
\def\langnames@langs@wals@snp{Siane}
\def\langnames@langs@wals@snq{Sangu (Gabon)}
\def\langnames@langs@wals@snr{Sihan (Gum)}
\def\langnames@langs@wals@sns{Nahavaq}
\def\langnames@langs@wals@snu{Senggi}
\def\langnames@langs@wals@snv{Sa'ban}
\def\langnames@langs@wals@snw{Selee}
\def\langnames@langs@wals@snx{Sam}
\def\langnames@langs@wals@sny{Saniyo-Hiyewe}
\def\langnames@langs@wals@snz{Kou}
\def\langnames@langs@wals@soa{Thai Song}
\def\langnames@langs@wals@sob{Sobei}
\def\langnames@langs@wals@soc{So (Democratic Republic of Congo)}
\def\langnames@langs@wals@sod{Songoora}
\def\langnames@langs@wals@soe{Ohendo}
\def\langnames@langs@wals@sog{Sogdian}
\def\langnames@langs@wals@soh{Aka}
\def\langnames@langs@wals@soi{Sonha}
\def\langnames@langs@wals@soj{Soic}
\def\langnames@langs@wals@sok{Sokoro}
\def\langnames@langs@wals@sol{Solos}
\def\langnames@langs@wals@som{Somali}
\def\langnames@langs@wals@soo{Nsong-Mpiin}
\def\langnames@langs@wals@sop{Songe}
\def\langnames@langs@wals@soq{Kanasi}
\def\langnames@langs@wals@sor{Somrai}
\def\langnames@langs@wals@sos{Seeku}
\def\langnames@langs@wals@sot{Southern Sotho}
\def\langnames@langs@wals@sou{Southern Thai}
\def\langnames@langs@wals@sov{Sonsorol}
\def\langnames@langs@wals@sow{Sowanda}
\def\langnames@langs@wals@sox{So (Cameroon)}
\def\langnames@langs@wals@soy{Miyobe}
\def\langnames@langs@wals@soz{Temi}
\def\langnames@langs@wals@spa{Spanish}
\def\langnames@langs@wals@spb{Sepa (Indonesia)}
\def\langnames@langs@wals@spc{Sapé}
\def\langnames@langs@wals@spd{Saep}
\def\langnames@langs@wals@spe{Sepa (Papua New Guinea)}
\def\langnames@langs@wals@spg{Sihan}
\def\langnames@langs@wals@spi{Saponi}
\def\langnames@langs@wals@spk{Sengo}
\def\langnames@langs@wals@spl{Selepet}
\def\langnames@langs@wals@spm{Sepen}
\def\langnames@langs@wals@spn{Sanapaná}
\def\langnames@langs@wals@spo{Spokane}
\def\langnames@langs@wals@spp{Supyire Senoufo}
\def\langnames@langs@wals@spq{Peruvian Amazonian Spanish}
\def\langnames@langs@wals@spr{Saparua}
\def\langnames@langs@wals@sps{Saposa}
\def\langnames@langs@wals@spt{Spiti Bhoti}
\def\langnames@langs@wals@spu{Sapuan}
\def\langnames@langs@wals@spv{Sambalpuri}
\def\langnames@langs@wals@spy{Sabaot}
\def\langnames@langs@wals@sqa{Shama-Sambuga}
\def\langnames@langs@wals@sqh{Shau}
\def\langnames@langs@wals@sqi{Albanian}
\def\langnames@langs@wals@sqk{Albanian Sign Language}
\def\langnames@langs@wals@sqm{Suma}
\def\langnames@langs@wals@sqn{Susquehannock}
\def\langnames@langs@wals@sqo{Sorkhei-Aftari}
\def\langnames@langs@wals@sqq{Sou}
\def\langnames@langs@wals@sqs{Sri Lankan Sign Language}
\def\langnames@langs@wals@sqt{Soqotri}
\def\langnames@langs@wals@squ{Squamish}
\def\langnames@langs@wals@sqx{Kafr Qasem Sign Language}
\def\langnames@langs@wals@sra{Saruga}
\def\langnames@langs@wals@srb{Sora}
\def\langnames@langs@wals@src{Logudorese Sardinian}
\def\langnames@langs@wals@sre{Sara Bakati'}
\def\langnames@langs@wals@srf{Nafi}
\def\langnames@langs@wals@srg{Sulod}
\def\langnames@langs@wals@srh{Sarikoli}
\def\langnames@langs@wals@sri{Siriano}
\def\langnames@langs@wals@srk{Serudung Murut}
\def\langnames@langs@wals@srl{Isirawa}
\def\langnames@langs@wals@srm{Saramaccan}
\def\langnames@langs@wals@srn{Sranan Tongo}
\def\langnames@langs@wals@sro{Campidanese Sardinian}
\def\langnames@langs@wals@srq{Sirionó}
\def\langnames@langs@wals@srr{Sereer}
\def\langnames@langs@wals@srs{Sarsi}
\def\langnames@langs@wals@srt{Sauri}
\def\langnames@langs@wals@sru{Suruí}
\def\langnames@langs@wals@srv{Waray Sorsogon}
\def\langnames@langs@wals@srw{Serua}
\def\langnames@langs@wals@srx{Sirmauri}
\def\langnames@langs@wals@sry{Sera}
\def\langnames@langs@wals@srz{Shahmirzadi}
\def\langnames@langs@wals@ssb{Southern Sama}
\def\langnames@langs@wals@ssc{Suba-Simbiti}
\def\langnames@langs@wals@ssd{Siroi}
\def\langnames@langs@wals@sse{Balangingi}
\def\langnames@langs@wals@ssf{Thao}
\def\langnames@langs@wals@ssg{Seimat}
\def\langnames@langs@wals@ssh{Shihhi Arabic}
\def\langnames@langs@wals@ssi{Sansi}
\def\langnames@langs@wals@ssj{Sausi}
\def\langnames@langs@wals@ssk{Sunam}
\def\langnames@langs@wals@ssl{Western Sisaala}
\def\langnames@langs@wals@ssm{Semnam}
\def\langnames@langs@wals@ssn{Waata}
\def\langnames@langs@wals@sso{Sissano}
\def\langnames@langs@wals@ssp{Spanish Sign Language}
\def\langnames@langs@wals@ssr{Swiss-French Sign Language}
\def\langnames@langs@wals@sss{Sô}
\def\langnames@langs@wals@sst{Sinasina}
\def\langnames@langs@wals@ssu{Susuami}
\def\langnames@langs@wals@ssv{Ngen}
\def\langnames@langs@wals@ssw{Swati}
\def\langnames@langs@wals@ssx{Samberigi}
\def\langnames@langs@wals@ssy{Saho}
\def\langnames@langs@wals@ssz{Sengseng}
\def\langnames@langs@wals@sta{KiSetla}
\def\langnames@langs@wals@stb{Northern Subanen}
\def\langnames@langs@wals@std{Sentinel}
\def\langnames@langs@wals@ste{Liana-Seti}
\def\langnames@langs@wals@stf{Seta}
\def\langnames@langs@wals@stg{Trieng}
\def\langnames@langs@wals@sth{Shelta}
\def\langnames@langs@wals@sti{Bulo Stieng}
\def\langnames@langs@wals@stj{Matya Samo}
\def\langnames@langs@wals@stk{Arammba}
\def\langnames@langs@wals@stm{Setaman}
\def\langnames@langs@wals@stn{Owa}
\def\langnames@langs@wals@sto{Stoney}
\def\langnames@langs@wals@stp{Southeastern Tepehuan}
\def\langnames@langs@wals@stq{Ems-Weser Frisian}
\def\langnames@langs@wals@str{Northern Straits Salish}
\def\langnames@langs@wals@sts{Shumashti}
\def\langnames@langs@wals@stt{Budeh Stieng}
\def\langnames@langs@wals@stu{Samtao}
\def\langnames@langs@wals@stv{Silt'e}
\def\langnames@langs@wals@stw{Satawalese}
\def\langnames@langs@wals@sty{Siberian Tatar}
\def\langnames@langs@wals@sua{Sulka}
\def\langnames@langs@wals@sub{Suku}
\def\langnames@langs@wals@suc{Western Subanon}
\def\langnames@langs@wals@sue{Suena}
\def\langnames@langs@wals@sug{Suganga}
\def\langnames@langs@wals@sui{Suki}
\def\langnames@langs@wals@suj{Shubi}
\def\langnames@langs@wals@suk{Sukuma}
\def\langnames@langs@wals@sun{Sundanese}
\def\langnames@langs@wals@suo{Bouni-Bobe}
\def\langnames@langs@wals@suq{Tirma-Chai}
\def\langnames@langs@wals@sur{Mwaghavul}
\def\langnames@langs@wals@sus{Susu}
\def\langnames@langs@wals@sut{Subtiaba}
\def\langnames@langs@wals@suv{Eastern Puroik}
\def\langnames@langs@wals@suw{Sumbwa}
\def\langnames@langs@wals@sux{Sumerian}
\def\langnames@langs@wals@suy{Suyá}
\def\langnames@langs@wals@suz{Sunwar}
\def\langnames@langs@wals@sva{Svan}
\def\langnames@langs@wals@svb{Ulau-Suain}
\def\langnames@langs@wals@svc{Vincentian Creole English}
\def\langnames@langs@wals@sve{Serili}
\def\langnames@langs@wals@svk{Slovakian Sign Language}
\def\langnames@langs@wals@svm{Slavomolisano}
\def\langnames@langs@wals@svs{Savosavo}
\def\langnames@langs@wals@swb{Maore Comorian}
\def\langnames@langs@wals@swc{Congo Swahili}
\def\langnames@langs@wals@swe{Swedish}
\def\langnames@langs@wals@swf{Sere}
\def\langnames@langs@wals@swg{Swabian}
\def\langnames@langs@wals@swh{Swahili}
\def\langnames@langs@wals@swi{Sui}
\def\langnames@langs@wals@swj{Sira}
\def\langnames@langs@wals@swk{Malawi Sena}
\def\langnames@langs@wals@swl{Swedish Sign Language}
\def\langnames@langs@wals@swm{Samosa}
\def\langnames@langs@wals@swn{Sawknah-Fogaha}
\def\langnames@langs@wals@swo{Shanenawa}
\def\langnames@langs@wals@swp{Suau}
\def\langnames@langs@wals@swq{Sharwa}
\def\langnames@langs@wals@swr{Saweru}
\def\langnames@langs@wals@sws{Seluwasan}
\def\langnames@langs@wals@swt{Sawila}
\def\langnames@langs@wals@swu{Suwawa}
\def\langnames@langs@wals@swv{Shekhawati}
\def\langnames@langs@wals@sww{Sowa}
\def\langnames@langs@wals@swx{Suruahá}
\def\langnames@langs@wals@swy{Sarua}
\def\langnames@langs@wals@sxb{Suba}
\def\langnames@langs@wals@sxc{Sicana}
\def\langnames@langs@wals@sxe{Sighu}
\def\langnames@langs@wals@sxg{Shixing}
\def\langnames@langs@wals@sxk{Yoncalla}
\def\langnames@langs@wals@sxn{Sangir}
\def\langnames@langs@wals@sxr{Saaroa}
\def\langnames@langs@wals@sxs{Sasaru}
\def\langnames@langs@wals@sxu{Central East Middle German}
\def\langnames@langs@wals@sxw{Saxwe Gbe}
\def\langnames@langs@wals@sya{Siang}
\def\langnames@langs@wals@syb{Central Subanen}
\def\langnames@langs@wals@syc{Classical Syriac}
\def\langnames@langs@wals@syi{Seki}
\def\langnames@langs@wals@syk{Sukur}
\def\langnames@langs@wals@syl{Sylheti}
\def\langnames@langs@wals@sym{Maya Samo}
\def\langnames@langs@wals@syo{Suoy}
\def\langnames@langs@wals@sys{Sinyar}
\def\langnames@langs@wals@syw{Kagate}
\def\langnames@langs@wals@syx{Osamayi}
\def\langnames@langs@wals@syy{Al-Sayyid Bedouin Sign Language}
\def\langnames@langs@wals@sza{Semelai}
\def\langnames@langs@wals@szb{Ngalum}
\def\langnames@langs@wals@szc{Semaq Beri}
\def\langnames@langs@wals@sze{Seze}
\def\langnames@langs@wals@szg{Sengele}
\def\langnames@langs@wals@szl{Silesian}
\def\langnames@langs@wals@szn{Sula}
\def\langnames@langs@wals@szp{Suabo}
\def\langnames@langs@wals@szs{Solomon Islands Sign Language}
\def\langnames@langs@wals@szv{Isu (Fako Division)}
\def\langnames@langs@wals@szw{Sawai}
\def\langnames@langs@wals@szy{Sakizaya}
\def\langnames@langs@wals@taa{Lower Tanana}
\def\langnames@langs@wals@tab{Tabasaran}
\def\langnames@langs@wals@tac{Lowland Tarahumara}
\def\langnames@langs@wals@tad{Tause}
\def\langnames@langs@wals@tae{Tariana}
\def\langnames@langs@wals@taf{Tapirapé}
\def\langnames@langs@wals@tag{Tagoi}
\def\langnames@langs@wals@tah{Tahitian}
\def\langnames@langs@wals@taj{Eastern Tamang}
\def\langnames@langs@wals@tak{Tala}
\def\langnames@langs@wals@tal{Tal}
\def\langnames@langs@wals@tam{Tamil}
\def\langnames@langs@wals@tan{Tangale}
\def\langnames@langs@wals@tao{Yami}
\def\langnames@langs@wals@tap{Taabwa}
\def\langnames@langs@wals@taq{Tamasheq}
\def\langnames@langs@wals@tar{Central Tarahumara}
\def\langnames@langs@wals@tas{Tay Boi}
\def\langnames@langs@wals@tat{Tatar}
\def\langnames@langs@wals@tau{Upper Tanana}
\def\langnames@langs@wals@tav{Tatuyo}
\def\langnames@langs@wals@taw{Tai}
\def\langnames@langs@wals@tax{Tamki}
\def\langnames@langs@wals@tay{Atayal}
\def\langnames@langs@wals@taz{Tocho}
\def\langnames@langs@wals@tba{Aikanã}
\def\langnames@langs@wals@tbc{Takia}
\def\langnames@langs@wals@tbd{Kaki Ae}
\def\langnames@langs@wals@tbe{Tanimbili}
\def\langnames@langs@wals@tbf{Mandara}
\def\langnames@langs@wals@tbg{North Tairora}
\def\langnames@langs@wals@tbh{Thurawal}
\def\langnames@langs@wals@tbi{Gaam}
\def\langnames@langs@wals@tbj{Tiang}
\def\langnames@langs@wals@tbk{Calamian Tagbanwa}
\def\langnames@langs@wals@tbl{Tboli}
\def\langnames@langs@wals@tbm{Tagbu}
\def\langnames@langs@wals@tbn{Barro Negro Tunebo}
\def\langnames@langs@wals@tbo{Tawala}
\def\langnames@langs@wals@tbp{Taworta}
\def\langnames@langs@wals@tbr{Tumtum}
\def\langnames@langs@wals@tbs{Tanguat}
\def\langnames@langs@wals@tbt{Tembo (Kitembo)}
\def\langnames@langs@wals@tbu{Tubar}
\def\langnames@langs@wals@tbw{Tagbanwa}
\def\langnames@langs@wals@tbx{Kapin}
\def\langnames@langs@wals@tby{Tabaru}
\def\langnames@langs@wals@tbz{Ditammari}
\def\langnames@langs@wals@tca{Ticuna}
\def\langnames@langs@wals@tcb{Tanacross}
\def\langnames@langs@wals@tcc{Barabayiiga-Gisamjanga}
\def\langnames@langs@wals@tcd{Tafi}
\def\langnames@langs@wals@tce{Southern Tutchone}
\def\langnames@langs@wals@tcf{Malinaltepec Me'phaa}
\def\langnames@langs@wals@tcg{Tamagario}
\def\langnames@langs@wals@tch{Turks And Caicos Creole English}
\def\langnames@langs@wals@tci{Anta-Komnzo-Wára-Wérè-Kémä}
\def\langnames@langs@wals@tck{Tchitchege}
\def\langnames@langs@wals@tcl{Taman (Myanmar)}
\def\langnames@langs@wals@tcm{Tanahmerah}
\def\langnames@langs@wals@tcn{Tichurong}
\def\langnames@langs@wals@tco{Taungyo}
\def\langnames@langs@wals@tcp{Laamtuk Thet}
\def\langnames@langs@wals@tcq{Kaiy}
\def\langnames@langs@wals@tcs{Torres Strait-Lockhart River Creole}
\def\langnames@langs@wals@tct{T'en}
\def\langnames@langs@wals@tcu{Southeastern Tarahumara}
\def\langnames@langs@wals@tcw{Tecpatlán Totonac}
\def\langnames@langs@wals@tcx{Toda}
\def\langnames@langs@wals@tcy{Tulu}
\def\langnames@langs@wals@tcz{Thado Chin}
\def\langnames@langs@wals@tda{Tagdal}
\def\langnames@langs@wals@tdb{Panchpargania}
\def\langnames@langs@wals@tdc{Emberá-Tadó}
\def\langnames@langs@wals@tdd{Tai Nüa}
\def\langnames@langs@wals@tde{Tiranige Diga Dogon}
\def\langnames@langs@wals@tdf{Talieng}
\def\langnames@langs@wals@tdg{Western Tamang}
\def\langnames@langs@wals@tdh{Thulung}
\def\langnames@langs@wals@tdi{Tomadino}
\def\langnames@langs@wals@tdj{Tajio}
\def\langnames@langs@wals@tdk{Tambas}
\def\langnames@langs@wals@tdl{Sur}
\def\langnames@langs@wals@tdm{Taruma}
\def\langnames@langs@wals@tdn{Tondano}
\def\langnames@langs@wals@tdo{Teme}
\def\langnames@langs@wals@tdq{Tita}
\def\langnames@langs@wals@tdr{Todrah}
\def\langnames@langs@wals@tds{Doutai}
\def\langnames@langs@wals@tdt{Tetun Dili}
\def\langnames@langs@wals@tdv{Toro}
\def\langnames@langs@wals@tdx{Tandroy Malagasy}
\def\langnames@langs@wals@tdy{Tadyawan}
\def\langnames@langs@wals@tea{Temiar}
\def\langnames@langs@wals@tec{Terik}
\def\langnames@langs@wals@ted{Tepo Krumen}
\def\langnames@langs@wals@tee{Huehuetla Tepehua}
\def\langnames@langs@wals@tef{Teressa}
\def\langnames@langs@wals@teg{Latege}
\def\langnames@langs@wals@teh{Tehuelche}
\def\langnames@langs@wals@tei{Aro}
\def\langnames@langs@wals@tek{Kwa South}
\def\langnames@langs@wals@tel{Telugu}
\def\langnames@langs@wals@tem{Timne}
\def\langnames@langs@wals@ten{Tama (Colombia)}
\def\langnames@langs@wals@teo{Teso}
\def\langnames@langs@wals@tep{Tepecano}
\def\langnames@langs@wals@teq{Temein}
\def\langnames@langs@wals@ter{Terena-Kinikinao-Chane}
\def\langnames@langs@wals@tes{Tengger}
\def\langnames@langs@wals@tet{Tetum}
\def\langnames@langs@wals@teu{Soo}
\def\langnames@langs@wals@tev{Teor}
\def\langnames@langs@wals@tew{Rio Grande Tewa}
\def\langnames@langs@wals@tex{Tennet}
\def\langnames@langs@wals@tey{Tulishi}
\def\langnames@langs@wals@tez{Tetserret}
\def\langnames@langs@wals@tfi{Tofin Gbe}
\def\langnames@langs@wals@tfn{Dena'ina}
\def\langnames@langs@wals@tfo{Tefaro}
\def\langnames@langs@wals@tfr{Teribe}
\def\langnames@langs@wals@tft{Ternate}
\def\langnames@langs@wals@tga{Sagalla}
\def\langnames@langs@wals@tgb{Tobilung}
\def\langnames@langs@wals@tgc{Tigak}
\def\langnames@langs@wals@tgd{Ciwogai}
\def\langnames@langs@wals@tge{Eastern Gorkha Tamang}
\def\langnames@langs@wals@tgf{Chalikha}
\def\langnames@langs@wals@tgg{Tangga}
\def\langnames@langs@wals@tgh{Tobagonian Creole English}
\def\langnames@langs@wals@tgi{Lawunuia}
\def\langnames@langs@wals@tgj{Tagin}
\def\langnames@langs@wals@tgk{Tajik}
\def\langnames@langs@wals@tgl{Tagalog}
\def\langnames@langs@wals@tgn{Tandaganon}
\def\langnames@langs@wals@tgo{Sudest}
\def\langnames@langs@wals@tgp{Movono}
\def\langnames@langs@wals@tgq{Tring}
\def\langnames@langs@wals@tgs{Nume}
\def\langnames@langs@wals@tgt{Central Tagbanwa}
\def\langnames@langs@wals@tgu{Tanggu}
\def\langnames@langs@wals@tgw{Tagwana Senoufo}
\def\langnames@langs@wals@tgx{Tagish}
\def\langnames@langs@wals@tgy{Togoyo}
\def\langnames@langs@wals@tgz{Tagalaka}
\def\langnames@langs@wals@tha{Thai}
\def\langnames@langs@wals@thd{Thayore}
\def\langnames@langs@wals@the{Chitwania Tharu}
\def\langnames@langs@wals@thf{Thangmi}
\def\langnames@langs@wals@thh{Northern Tarahumara}
\def\langnames@langs@wals@thi{Tai Long}
\def\langnames@langs@wals@thk{Tharaka}
\def\langnames@langs@wals@thl{Dangaura Tharu}
\def\langnames@langs@wals@thm{Thavung}
\def\langnames@langs@wals@thn{Thachanadan}
\def\langnames@langs@wals@thp{Thompson}
\def\langnames@langs@wals@thq{Kochila Tharu}
\def\langnames@langs@wals@thr{Rana Tharu}
\def\langnames@langs@wals@ths{Thakali}
\def\langnames@langs@wals@tht{Tahltan}
\def\langnames@langs@wals@thu{Thuri}
\def\langnames@langs@wals@thv{Tahaggart Tamahaq}
\def\langnames@langs@wals@thy{Tha}
\def\langnames@langs@wals@thz{Tayart Tamajeq}
\def\langnames@langs@wals@tia{Tidikelt-Tuat Tamazight}
\def\langnames@langs@wals@tic{Tira}
\def\langnames@langs@wals@tif{Tifal}
\def\langnames@langs@wals@tig{Tigre}
\def\langnames@langs@wals@tih{Timugon Murut}
\def\langnames@langs@wals@tii{Tiene}
\def\langnames@langs@wals@tij{Tilung}
\def\langnames@langs@wals@tik{Tikar}
\def\langnames@langs@wals@til{Tillamook}
\def\langnames@langs@wals@tim{Timbe}
\def\langnames@langs@wals@tin{Tindi}
\def\langnames@langs@wals@tio{Teop}
\def\langnames@langs@wals@tip{Trimuris}
\def\langnames@langs@wals@tiq{Tiefo-Daramandugu}
\def\langnames@langs@wals@tir{Tigrinya}
\def\langnames@langs@wals@tis{Masadiit Itneg}
\def\langnames@langs@wals@tit{Tinigua}
\def\langnames@langs@wals@tiu{Adasen}
\def\langnames@langs@wals@tiv{Tiv}
\def\langnames@langs@wals@tiw{Tiwi}
\def\langnames@langs@wals@tix{Southern Tiwa}
\def\langnames@langs@wals@tiy{Tiruray}
\def\langnames@langs@wals@tiz{Tai Hongjin}
\def\langnames@langs@wals@tja{Tajuasohn}
\def\langnames@langs@wals@tjg{Tunjung}
\def\langnames@langs@wals@tji{Northern Tujia}
\def\langnames@langs@wals@tjj{Yangathimri}
\def\langnames@langs@wals@tjl{Tai Laing}
\def\langnames@langs@wals@tjm{Timucua}
\def\langnames@langs@wals@tjn{Tonjon}
\def\langnames@langs@wals@tjo{Oued Righ}
\def\langnames@langs@wals@tjp{Lake Carnegie Western Desert}
\def\langnames@langs@wals@tjs{Southern Tujia}
\def\langnames@langs@wals@tju{Tjurruru}
\def\langnames@langs@wals@tka{Truká}
\def\langnames@langs@wals@tkb{Buksa}
\def\langnames@langs@wals@tkd{Tukudede}
\def\langnames@langs@wals@tke{Takwane}
\def\langnames@langs@wals@tkf{Tukumanféd}
\def\langnames@langs@wals@tkg{Tesaka Malagasy}
\def\langnames@langs@wals@tkl{Tokelau}
\def\langnames@langs@wals@tkm{Takelma}
\def\langnames@langs@wals@tkn{Toku-No-Shima}
\def\langnames@langs@wals@tkp{Tikopia}
\def\langnames@langs@wals@tkq{Tee}
\def\langnames@langs@wals@tkr{Tsakhur}
\def\langnames@langs@wals@tks{Takestani}
\def\langnames@langs@wals@tkt{Kathoriya Tharu}
\def\langnames@langs@wals@tku{Upper Necaxa Totonac}
\def\langnames@langs@wals@tkv{Mur Pano}
\def\langnames@langs@wals@tkw{Teanu}
\def\langnames@langs@wals@tkx{Tangko}
\def\langnames@langs@wals@tkz{Takua}
\def\langnames@langs@wals@tla{Southwestern Tepehuan}
\def\langnames@langs@wals@tlb{Tobelo}
\def\langnames@langs@wals@tlc{Yecuatla Totonac}
\def\langnames@langs@wals@tld{Talaud}
\def\langnames@langs@wals@tlf{Telefol}
\def\langnames@langs@wals@tlg{Tofanma}
\def\langnames@langs@wals@tlh{Klingon}
\def\langnames@langs@wals@tli{Tlingit}
\def\langnames@langs@wals@tlj{Talinga-Bwisi}
\def\langnames@langs@wals@tlk{Taloki}
\def\langnames@langs@wals@tll{Tetela}
\def\langnames@langs@wals@tlm{Tolomako}
\def\langnames@langs@wals@tln{Talondo'}
\def\langnames@langs@wals@tlo{Tasomi-Tata}
\def\langnames@langs@wals@tlp{Filomeno Mata Totonac}
\def\langnames@langs@wals@tlq{Muak}
\def\langnames@langs@wals@tlr{Talise}
\def\langnames@langs@wals@tls{Tambotalo}
\def\langnames@langs@wals@tlt{Teluti}
\def\langnames@langs@wals@tlu{Tulehu}
\def\langnames@langs@wals@tlv{Taliabu}
\def\langnames@langs@wals@tlx{Khehek}
\def\langnames@langs@wals@tly{North-Central Talysh}
\def\langnames@langs@wals@tma{Tama (Chad)}
\def\langnames@langs@wals@tmb{Avava}
\def\langnames@langs@wals@tmc{Tumak}
\def\langnames@langs@wals@tmd{Haruai}
\def\langnames@langs@wals@tme{Tremembé}
\def\langnames@langs@wals@tmf{Toba-Enenlhet}
\def\langnames@langs@wals@tmg{Ternateño}
\def\langnames@langs@wals@tmi{Tutuba}
\def\langnames@langs@wals@tmj{Samarokena}
\def\langnames@langs@wals@tml{Tamnim Citak}
\def\langnames@langs@wals@tmm{Tai Thanh}
\def\langnames@langs@wals@tmn{Taman (Indonesia)}
\def\langnames@langs@wals@tmo{Temoq}
\def\langnames@langs@wals@tmq{Tumleo}
\def\langnames@langs@wals@tmr{Jewish Babylonian Aramaic (ca. 200-1200 CE)}
\def\langnames@langs@wals@tms{Tima}
\def\langnames@langs@wals@tmt{Tasmate}
\def\langnames@langs@wals@tmu{Iau}
\def\langnames@langs@wals@tmv{Motembo-Kunda}
\def\langnames@langs@wals@tmw{Temuan}
\def\langnames@langs@wals@tmy{Tami}
\def\langnames@langs@wals@tmz{Tamanaku}
\def\langnames@langs@wals@tna{Tacana}
\def\langnames@langs@wals@tnb{Western Tunebo}
\def\langnames@langs@wals@tnc{Tanimuca-Retuarã}
\def\langnames@langs@wals@tnd{Angosturas Tunebo}
\def\langnames@langs@wals@tng{Tobanga}
\def\langnames@langs@wals@tnh{Maiani}
\def\langnames@langs@wals@tni{Tandia}
\def\langnames@langs@wals@tnk{Kwamera}
\def\langnames@langs@wals@tnl{Lenakel}
\def\langnames@langs@wals@tnm{Tabla}
\def\langnames@langs@wals@tnn{North Tanna}
\def\langnames@langs@wals@tno{Toromono}
\def\langnames@langs@wals@tnp{Whitesands}
\def\langnames@langs@wals@tnq{Taino}
\def\langnames@langs@wals@tnr{Bedik}
\def\langnames@langs@wals@tns{Tenis}
\def\langnames@langs@wals@tnt{Tontemboan}
\def\langnames@langs@wals@tnu{Tay Khang}
\def\langnames@langs@wals@tnv{Tangchangya}
\def\langnames@langs@wals@tnw{Tonsawang}
\def\langnames@langs@wals@tnx{Tanema}
\def\langnames@langs@wals@tny{Tongwe}
\def\langnames@langs@wals@tnz{Maniq}
\def\langnames@langs@wals@tob{Toba}
\def\langnames@langs@wals@toc{Coyutla Totonac}
\def\langnames@langs@wals@tod{Toma}
\def\langnames@langs@wals@tof{Gizrra}
\def\langnames@langs@wals@tog{Tonga (Nyasa)}
\def\langnames@langs@wals@toh{Gitonga}
\def\langnames@langs@wals@toi{Tonga (Zambia)}
\def\langnames@langs@wals@toj{Tojolabal}
\def\langnames@langs@wals@tok{Toki Pona}
\def\langnames@langs@wals@tol{Tolowa-Chetco}
\def\langnames@langs@wals@tom{Tombulu}
\def\langnames@langs@wals@ton{Tonga (Tonga Islands)}
\def\langnames@langs@wals@too{Xicotepec De Juárez Totonac}
\def\langnames@langs@wals@top{Papantla Totonac}
\def\langnames@langs@wals@toq{Toposa}
\def\langnames@langs@wals@tor{Togbo-Vara Banda}
\def\langnames@langs@wals@tos{Highland Totonac}
\def\langnames@langs@wals@tou{Tho}
\def\langnames@langs@wals@tov{Upper Taromi}
\def\langnames@langs@wals@tow{Towa}
\def\langnames@langs@wals@tox{Tobian}
\def\langnames@langs@wals@toy{Topoiyo}
\def\langnames@langs@wals@toz{To}
\def\langnames@langs@wals@tpa{Taupota}
\def\langnames@langs@wals@tpc{Azoyú Me'phaa}
\def\langnames@langs@wals@tpe{Tippera}
\def\langnames@langs@wals@tpf{Tarpia}
\def\langnames@langs@wals@tpg{Kula (Indonesia)}
\def\langnames@langs@wals@tpi{Tok Pisin}
\def\langnames@langs@wals@tpj{Tapieté}
\def\langnames@langs@wals@tpl{Tlacoapa Me'phaa}
\def\langnames@langs@wals@tpm{Tampulma}
\def\langnames@langs@wals@tpn{Tupinambá}
\def\langnames@langs@wals@tpo{Tai Pao}
\def\langnames@langs@wals@tpp{Pisaflores Tepehua}
\def\langnames@langs@wals@tpq{Nyamkad}
\def\langnames@langs@wals@tpr{Tuparí}
\def\langnames@langs@wals@tpt{Tlachichilco Tepehua}
\def\langnames@langs@wals@tpu{Tampuan}
\def\langnames@langs@wals@tpv{Tanapag}
\def\langnames@langs@wals@tpw{Lingua Geral Paulista}
\def\langnames@langs@wals@tpx{Acatepec Me'phaa}
\def\langnames@langs@wals@tpy{Trumai}
\def\langnames@langs@wals@tpz{Tinputz}
\def\langnames@langs@wals@tqb{Tenetehara}
\def\langnames@langs@wals@tql{Lehali}
\def\langnames@langs@wals@tqm{Turumsa}
\def\langnames@langs@wals@tqn{Tenino}
\def\langnames@langs@wals@tqo{Toaripi}
\def\langnames@langs@wals@tqp{Tomoip}
\def\langnames@langs@wals@tqq{Tunni}
\def\langnames@langs@wals@tqr{Torona}
\def\langnames@langs@wals@tqt{Ozumatlán Totonac}
\def\langnames@langs@wals@tqu{Touo}
\def\langnames@langs@wals@tqw{Tonkawa}
\def\langnames@langs@wals@tra{Tirahi}
\def\langnames@langs@wals@trb{Terebu}
\def\langnames@langs@wals@trc{Copala Triqui}
\def\langnames@langs@wals@trd{Turi}
\def\langnames@langs@wals@tre{East Tarangan}
\def\langnames@langs@wals@trf{Trinidadian Creole English}
\def\langnames@langs@wals@trg{Lishán Didán}
\def\langnames@langs@wals@trh{Turaka}
\def\langnames@langs@wals@tri{Trió}
\def\langnames@langs@wals@trj{Toram}
\def\langnames@langs@wals@trl{Traveller Scottish}
\def\langnames@langs@wals@trm{Tregami}
\def\langnames@langs@wals@trn{Trinitario-Javeriano-Loretano}
\def\langnames@langs@wals@tro{Tarao}
\def\langnames@langs@wals@trp{Kok Borok}
\def\langnames@langs@wals@trq{San Martín Itunyoso Triqui}
\def\langnames@langs@wals@trr{Taushiro}
\def\langnames@langs@wals@trs{Chicahuaxtla Triqui}
\def\langnames@langs@wals@trt{Tunggare}
\def\langnames@langs@wals@tru{Turoyo}
\def\langnames@langs@wals@trv{Seediq}
\def\langnames@langs@wals@trw{Torwali}
\def\langnames@langs@wals@trx{Tringgus-Sembaan Bidayuh}
\def\langnames@langs@wals@try{Turung}
\def\langnames@langs@wals@trz{Torá}
\def\langnames@langs@wals@tsa{Tsaangi}
\def\langnames@langs@wals@tsb{Tsamai}
\def\langnames@langs@wals@tsc{Tswa}
\def\langnames@langs@wals@tsd{Tsakonian}
\def\langnames@langs@wals@tse{Tunisian Sign Language}
\def\langnames@langs@wals@tsg{Tausug}
\def\langnames@langs@wals@tsh{Tsuvan}
\def\langnames@langs@wals@tsi{Southern-Coastal Tsimshian}
\def\langnames@langs@wals@tsj{Tshangla}
\def\langnames@langs@wals@tsk{Tseku}
\def\langnames@langs@wals@tsl{Ts'ün-Lao}
\def\langnames@langs@wals@tsm{Turkish Sign Language}
\def\langnames@langs@wals@tsn{Tswana}
\def\langnames@langs@wals@tso{Tsonga}
\def\langnames@langs@wals@tsp{Northern Toussian}
\def\langnames@langs@wals@tsq{Thai Sign Language}
\def\langnames@langs@wals@tsr{Akei}
\def\langnames@langs@wals@tss{Taiwan Sign Language}
\def\langnames@langs@wals@tst{Tondi Songway Kiini}
\def\langnames@langs@wals@tsu{Tsou}
\def\langnames@langs@wals@tsv{Tsogo}
\def\langnames@langs@wals@tsw{Salka-Tsishingini}
\def\langnames@langs@wals@tsx{Mubami}
\def\langnames@langs@wals@tsy{Tebul Sign Language}
\def\langnames@langs@wals@tsz{Purepecha}
\def\langnames@langs@wals@tta{Tutelo}
\def\langnames@langs@wals@ttb{Gaa}
\def\langnames@langs@wals@ttc{Tektiteko}
\def\langnames@langs@wals@ttd{Tauade}
\def\langnames@langs@wals@tte{Bwanabwana}
\def\langnames@langs@wals@ttf{Tuotomb}
\def\langnames@langs@wals@ttg{Tutong}
\def\langnames@langs@wals@tth{Upper Ta'oih}
\def\langnames@langs@wals@tti{Tobati}
\def\langnames@langs@wals@ttj{Tooro}
\def\langnames@langs@wals@ttk{Totoro}
\def\langnames@langs@wals@ttl{Totela}
\def\langnames@langs@wals@ttm{Northern Tutchone}
\def\langnames@langs@wals@ttn{Towei}
\def\langnames@langs@wals@tto{Lower Ta'oih}
\def\langnames@langs@wals@ttp{Tombelala}
\def\langnames@langs@wals@ttq{Tawallammat Tamajaq}
\def\langnames@langs@wals@ttr{Tera}
\def\langnames@langs@wals@tts{Northeastern Thai}
\def\langnames@langs@wals@ttt{Muslim Tat}
\def\langnames@langs@wals@ttu{Torau}
\def\langnames@langs@wals@ttv{Titan}
\def\langnames@langs@wals@ttw{Western Lowland Kenyah}
\def\langnames@langs@wals@tty{Sikaritai}
\def\langnames@langs@wals@ttz{Tsum}
\def\langnames@langs@wals@tua{Wiarumus}
\def\langnames@langs@wals@tub{Tübatulabal}
\def\langnames@langs@wals@tuc{Mutu}
\def\langnames@langs@wals@tud{Tuxá}
\def\langnames@langs@wals@tue{Tuyuca}
\def\langnames@langs@wals@tuf{Central Tunebo}
\def\langnames@langs@wals@tug{Tunia}
\def\langnames@langs@wals@tuh{Taulil}
\def\langnames@langs@wals@tui{Tupuri}
\def\langnames@langs@wals@tuj{Tugutil}
\def\langnames@langs@wals@tuk{Turkmen}
\def\langnames@langs@wals@tul{Tula}
\def\langnames@langs@wals@tum{Tumbuka}
\def\langnames@langs@wals@tun{Tunica}
\def\langnames@langs@wals@tuo{Tucano}
\def\langnames@langs@wals@tuq{Tedaga}
\def\langnames@langs@wals@tur{Turkish}
\def\langnames@langs@wals@tus{Tuscarora}
\def\langnames@langs@wals@tuu{Tututni}
\def\langnames@langs@wals@tuv{Turkana}
\def\langnames@langs@wals@tux{Tuxináwa}
\def\langnames@langs@wals@tuy{Tugen}
\def\langnames@langs@wals@tuz{Turka}
\def\langnames@langs@wals@tva{Vaghua}
\def\langnames@langs@wals@tvd{Tsuvadi}
\def\langnames@langs@wals@tve{Te'un}
\def\langnames@langs@wals@tvk{Southeast Ambrym}
\def\langnames@langs@wals@tvl{Tuvalu}
\def\langnames@langs@wals@tvm{Tela-Masbuar}
\def\langnames@langs@wals@tvn{Tavoyan}
\def\langnames@langs@wals@tvo{Tidore}
\def\langnames@langs@wals@tvs{Taveta}
\def\langnames@langs@wals@tvt{Tutsa Naga}
\def\langnames@langs@wals@tvu{Tunen}
\def\langnames@langs@wals@tvw{Sedoa}
\def\langnames@langs@wals@tvy{Timor Pidgin}
\def\langnames@langs@wals@twa{Twana}
\def\langnames@langs@wals@twb{Western Tawbuid}
\def\langnames@langs@wals@twc{Teshenawa}
\def\langnames@langs@wals@twe{Teiwa}
\def\langnames@langs@wals@twf{Taos Northern Tiwa}
\def\langnames@langs@wals@twg{Tereweng}
\def\langnames@langs@wals@twh{Tai Dón}
\def\langnames@langs@wals@twl{Tawara}
\def\langnames@langs@wals@twn{Cambap-Langa}
\def\langnames@langs@wals@two{Tswapong}
\def\langnames@langs@wals@twp{Ere}
\def\langnames@langs@wals@twq{Tasawaq}
\def\langnames@langs@wals@twr{Southwestern Tarahumara}
\def\langnames@langs@wals@twt{Turiwára}
\def\langnames@langs@wals@twu{Termanu}
\def\langnames@langs@wals@tww{Tuwari}
\def\langnames@langs@wals@twx{Tewe}
\def\langnames@langs@wals@twy{Tawoyan}
\def\langnames@langs@wals@txa{Tombonuo}
\def\langnames@langs@wals@txb{Tokharian B}
\def\langnames@langs@wals@txc{Tsetsaut}
\def\langnames@langs@wals@txe{Totoli}
\def\langnames@langs@wals@txg{Tangut}
\def\langnames@langs@wals@txh{Thracian}
\def\langnames@langs@wals@txi{Ikpeng}
\def\langnames@langs@wals@txj{Tarjumo}
\def\langnames@langs@wals@txm{Tomini}
\def\langnames@langs@wals@txn{West Tarangan}
\def\langnames@langs@wals@txo{Toto}
\def\langnames@langs@wals@txq{Tii}
\def\langnames@langs@wals@txr{Tartessian}
\def\langnames@langs@wals@txs{Tonsea}
\def\langnames@langs@wals@txt{Citak}
\def\langnames@langs@wals@txu{Kayapó}
\def\langnames@langs@wals@txx{Tatana}
\def\langnames@langs@wals@txy{Tanosy Malagasy}
\def\langnames@langs@wals@tya{Tauya}
\def\langnames@langs@wals@tye{Kyenga}
\def\langnames@langs@wals@tyh{O'du}
\def\langnames@langs@wals@tyi{Teke-Tsaayi}
\def\langnames@langs@wals@tyj{Tai Do-Mene-Yo}
\def\langnames@langs@wals@tyn{Kombai}
\def\langnames@langs@wals@typ{Thaypan}
\def\langnames@langs@wals@tyr{Tai Daeng-Meuay}
\def\langnames@langs@wals@tys{Tày Sa Pa}
\def\langnames@langs@wals@tyt{Tày Tac}
\def\langnames@langs@wals@tyu{Southern Tshwa}
\def\langnames@langs@wals@tyv{Tuvinian}
\def\langnames@langs@wals@tyx{Teke-Tyee}
\def\langnames@langs@wals@tyy{Tiyaa}
\def\langnames@langs@wals@tyz{Tày}
\def\langnames@langs@wals@tza{Tanzanian Sign Language}
\def\langnames@langs@wals@tzh{Tzeltal}
\def\langnames@langs@wals@tzj{Tz'utujil}
\def\langnames@langs@wals@tzl{Talossan}
\def\langnames@langs@wals@tzm{Central Moroccan Berber}
\def\langnames@langs@wals@tzn{Tugun}
\def\langnames@langs@wals@tzo{Tzotzil}
\def\langnames@langs@wals@tzx{Tabriak}
\def\langnames@langs@wals@uam{Uamué}
\def\langnames@langs@wals@uan{Kuan}
\def\langnames@langs@wals@uar{Tairuma}
\def\langnames@langs@wals@uba{Ubang}
\def\langnames@langs@wals@ubi{Ubi}
\def\langnames@langs@wals@ubr{Ubir}
\def\langnames@langs@wals@ubu{Umbu-Ungu}
\def\langnames@langs@wals@uby{Ubykh}
\def\langnames@langs@wals@uda{Uda}
\def\langnames@langs@wals@ude{Udihe}
\def\langnames@langs@wals@udg{Muduga}
\def\langnames@langs@wals@udi{Udi}
\def\langnames@langs@wals@udj{Ujir}
\def\langnames@langs@wals@udl{Wuzlam}
\def\langnames@langs@wals@udm{Udmurt}
\def\langnames@langs@wals@udu{Uduk}
\def\langnames@langs@wals@ues{Kioko}
\def\langnames@langs@wals@ufi{Ufim}
\def\langnames@langs@wals@uga{Ugaritic}
\def\langnames@langs@wals@ugb{Kuku-Ugbanh}
\def\langnames@langs@wals@uge{Ughele}
\def\langnames@langs@wals@ugh{Kubachi}
\def\langnames@langs@wals@ugn{Ugandan Sign Language}
\def\langnames@langs@wals@ugo{Ugong}
\def\langnames@langs@wals@ugy{Uruguayan Sign Language}
\def\langnames@langs@wals@uha{Uhami}
\def\langnames@langs@wals@uhn{Damal}
\def\langnames@langs@wals@uig{Uighur}
\def\langnames@langs@wals@uis{Uisai}
\def\langnames@langs@wals@uiv{Iyive}
\def\langnames@langs@wals@uji{Rjili}
\def\langnames@langs@wals@uka{Kaburi}
\def\langnames@langs@wals@ukg{Ukuriguma}
\def\langnames@langs@wals@ukh{Ukhwejo}
\def\langnames@langs@wals@ukl{Ukrainian Sign Language}
\def\langnames@langs@wals@ukp{Ukpe-Bayobiri}
\def\langnames@langs@wals@ukq{Ukwa}
\def\langnames@langs@wals@ukr{Ukrainian}
\def\langnames@langs@wals@uks{Urubú-Kaapor Sign Language}
\def\langnames@langs@wals@uku{Ukue}
\def\langnames@langs@wals@ukv{Kuku}
\def\langnames@langs@wals@ukw{Ukwuani-Aboh-Ndoni}
\def\langnames@langs@wals@uky{Kuuk-Yak}
\def\langnames@langs@wals@ula{Fungwa}
\def\langnames@langs@wals@ulb{Ulukwumi}
\def\langnames@langs@wals@ulc{Ulch}
\def\langnames@langs@wals@ule{Lule}
\def\langnames@langs@wals@ulf{Usku}
\def\langnames@langs@wals@uli{Ulithian}
\def\langnames@langs@wals@ulk{Meriam}
\def\langnames@langs@wals@ull{Ullatan}
\def\langnames@langs@wals@ulm{Ulumanda'}
\def\langnames@langs@wals@uln{Unserdeutsch}
\def\langnames@langs@wals@ulu{Uma' Lung}
\def\langnames@langs@wals@ulw{Ulwa}
\def\langnames@langs@wals@uma{Umatilla}
\def\langnames@langs@wals@umb{Umbundu}
\def\langnames@langs@wals@umd{Umbindhamu}
\def\langnames@langs@wals@umg{Umbuygamu}
\def\langnames@langs@wals@umi{Ukit}
\def\langnames@langs@wals@umm{Umon}
\def\langnames@langs@wals@umn{Makyan Naga}
\def\langnames@langs@wals@umo{Umotína}
\def\langnames@langs@wals@ump{Umpila}
\def\langnames@langs@wals@umr{Umbugarla}
\def\langnames@langs@wals@ums{Pendau}
\def\langnames@langs@wals@umu{Munsee}
\def\langnames@langs@wals@una{North Watut}
\def\langnames@langs@wals@une{Uneme}
\def\langnames@langs@wals@ung{Ngarinyin}
\def\langnames@langs@wals@uni{Uni}
\def\langnames@langs@wals@unk{Enawené-Nawé}
\def\langnames@langs@wals@unm{Unami}
\def\langnames@langs@wals@unn{Ganai}
\def\langnames@langs@wals@unr{Mundari}
\def\langnames@langs@wals@unu{Unubahe}
\def\langnames@langs@wals@unz{Unde Kaili}
\def\langnames@langs@wals@uon{Kulon}
\def\langnames@langs@wals@upi{Umeda-Punda}
\def\langnames@langs@wals@upv{Uripiv-Wala-Rano-Atchin}
\def\langnames@langs@wals@ura{Urarina}
\def\langnames@langs@wals@urb{Urubú-Kaapor}
\def\langnames@langs@wals@urc{Urningangg}
\def\langnames@langs@wals@urd{Urdu}
\def\langnames@langs@wals@ure{Uru}
\def\langnames@langs@wals@urf{Uradhi}
\def\langnames@langs@wals@urg{Urigina}
\def\langnames@langs@wals@urh{Urhobo}
\def\langnames@langs@wals@uri{Urim}
\def\langnames@langs@wals@urk{Urak Lawoi'}
\def\langnames@langs@wals@url{Urali of Idukki}
\def\langnames@langs@wals@urm{Urapmin}
\def\langnames@langs@wals@urn{Uruangnirin}
\def\langnames@langs@wals@uro{Ura (Papua New Guinea)}
\def\langnames@langs@wals@urp{Uru-Pa-In}
\def\langnames@langs@wals@urr{Lehalurup}
\def\langnames@langs@wals@urt{Urat}
\def\langnames@langs@wals@uru{Urumi}
\def\langnames@langs@wals@urv{Uruava}
\def\langnames@langs@wals@urw{Sop}
\def\langnames@langs@wals@urx{Urimo}
\def\langnames@langs@wals@ury{Orya}
\def\langnames@langs@wals@urz{Uru-Eu-Wau-Wau}
\def\langnames@langs@wals@usa{Usarufa}
\def\langnames@langs@wals@ush{Ushojo}
\def\langnames@langs@wals@usi{Usui}
\def\langnames@langs@wals@usk{Usaghade}
\def\langnames@langs@wals@usp{Uspanteco}
\def\langnames@langs@wals@usu{Uya}
\def\langnames@langs@wals@uta{Otank}
\def\langnames@langs@wals@ute{Ute-Southern Paiute}
\def\langnames@langs@wals@utp{Amba (Solomon Islands)}
\def\langnames@langs@wals@utr{Etulo}
\def\langnames@langs@wals@utu{Utu}
\def\langnames@langs@wals@uum{Urum}
\def\langnames@langs@wals@uur{Ura (Vanuatu)}
\def\langnames@langs@wals@uuu{U}
\def\langnames@langs@wals@uve{West Uvean}
\def\langnames@langs@wals@uvh{Uri}
\def\langnames@langs@wals@uvl{Lote}
\def\langnames@langs@wals@uwa{Kuku-Uwanh}
\def\langnames@langs@wals@uya{Doko-Uyanga}
\def\langnames@langs@wals@uzn{Northern Uzbek}
\def\langnames@langs@wals@uzs{Southern Uzbek}
\def\langnames@langs@wals@vaa{Vaagri Booli}
\def\langnames@langs@wals@vae{Vale}
\def\langnames@langs@wals@vaf{Vafsi}
\def\langnames@langs@wals@vag{Vagla}
\def\langnames@langs@wals@vah{Varhadi-Nagpuri}
\def\langnames@langs@wals@vai{Vai}
\def\langnames@langs@wals@vaj{Northern Ju}
\def\langnames@langs@wals@val{Vehes}
\def\langnames@langs@wals@vam{Vanimo}
\def\langnames@langs@wals@van{Walman}
\def\langnames@langs@wals@vao{Vao}
\def\langnames@langs@wals@vap{Vaiphei}
\def\langnames@langs@wals@var{Huarijio}
\def\langnames@langs@wals@vas{Vasavi}
\def\langnames@langs@wals@vau{Vanuma}
\def\langnames@langs@wals@vav{Dungar Varli}
\def\langnames@langs@wals@vay{Wayu}
\def\langnames@langs@wals@vbb{Southeast Babar}
\def\langnames@langs@wals@vec{Venetian}
\def\langnames@langs@wals@ved{Veddah}
\def\langnames@langs@wals@vem{Vemgo-Mabas}
\def\langnames@langs@wals@ven{Venda}
\def\langnames@langs@wals@veo{Ventureño}
\def\langnames@langs@wals@vep{Veps}
\def\langnames@langs@wals@ver{Mom Jango}
\def\langnames@langs@wals@vgr{Vaghri}
\def\langnames@langs@wals@vgt{Vlaamse Gebarentaal}
\def\langnames@langs@wals@vic{Virgin Islands Creole English}
\def\langnames@langs@wals@vid{Vidunda}
\def\langnames@langs@wals@vie{Vietnamese}
\def\langnames@langs@wals@vif{Vili}
\def\langnames@langs@wals@vig{Viemo}
\def\langnames@langs@wals@vil{Vilela}
\def\langnames@langs@wals@vin{Vinza}
\def\langnames@langs@wals@vis{Vishavan}
\def\langnames@langs@wals@vit{Viti}
\def\langnames@langs@wals@viv{Iduna}
\def\langnames@langs@wals@vka{Kariyarra}
\def\langnames@langs@wals@vkj{Kujarge}
\def\langnames@langs@wals@vkk{Kaur}
\def\langnames@langs@wals@vkl{Kulisusu}
\def\langnames@langs@wals@vkm{Kamakan}
\def\langnames@langs@wals@vkn{Koro Nulu}
\def\langnames@langs@wals@vko{Kodeoha}
\def\langnames@langs@wals@vkp{Korlai Portuguese}
\def\langnames@langs@wals@vkt{Tenggarong Kutai Malay}
\def\langnames@langs@wals@vku{Kurrama}
\def\langnames@langs@wals@vkz{Koro Zuba}
\def\langnames@langs@wals@vlp{Valpei}
\def\langnames@langs@wals@vls{Western Flemish}
\def\langnames@langs@wals@vma{Martuthunira}
\def\langnames@langs@wals@vmb{Mbabaram}
\def\langnames@langs@wals@vmc{Juxtlahuaca Mixtec}
\def\langnames@langs@wals@vmd{Mudu Koraga}
\def\langnames@langs@wals@vme{East Masela}
\def\langnames@langs@wals@vmf{Ostfränkisch}
\def\langnames@langs@wals@vmg{Minigir}
\def\langnames@langs@wals@vmh{Maraghei}
\def\langnames@langs@wals@vmi{Miwa}
\def\langnames@langs@wals@vmj{Ixtayutla Mixtec}
\def\langnames@langs@wals@vmk{Makhuwa-Shirima}
\def\langnames@langs@wals@vml{Malgana}
\def\langnames@langs@wals@vmm{Mitlatongo Mixtec}
\def\langnames@langs@wals@vmp{Soyaltepec Mazatec}
\def\langnames@langs@wals@vmq{Soyaltepec Mixtec}
\def\langnames@langs@wals@vmr{Marenje}
\def\langnames@langs@wals@vms{Moksela}
\def\langnames@langs@wals@vmu{Muluridyi}
\def\langnames@langs@wals@vmv{Valley Maidu}
\def\langnames@langs@wals@vmw{Makhuwa}
\def\langnames@langs@wals@vmx{Tamazola Mixtec}
\def\langnames@langs@wals@vmy{Ayautla Mazatec}
\def\langnames@langs@wals@vmz{Mazatlán Mazatec}
\def\langnames@langs@wals@vnk{Lovono}
\def\langnames@langs@wals@vnm{Neve'ei}
\def\langnames@langs@wals@vnp{Vunapu}
\def\langnames@langs@wals@vol{Volapük}
\def\langnames@langs@wals@vor{Voro}
\def\langnames@langs@wals@vot{Votic}
\def\langnames@langs@wals@vra{Vera'a}
\def\langnames@langs@wals@vro{South Estonian}
\def\langnames@langs@wals@vrs{Varisi}
\def\langnames@langs@wals@vrt{Burmbar}
\def\langnames@langs@wals@vsi{Moldova Sign Language}
\def\langnames@langs@wals@vsl{Venezuelan Sign Language}
\def\langnames@langs@wals@vsv{Valencian Sign Language}
\def\langnames@langs@wals@vto{Vitou}
\def\langnames@langs@wals@vum{Vumbu}
\def\langnames@langs@wals@vun{Vunjo}
\def\langnames@langs@wals@vut{Vute}
\def\langnames@langs@wals@vwa{Lavia-Awalai-Damangnuo Awa}
\def\langnames@langs@wals@waa{Northeast Sahaptin}
\def\langnames@langs@wals@wab{Wab}
\def\langnames@langs@wals@wac{Upper Chinook}
\def\langnames@langs@wals@wad{Wandamen}
\def\langnames@langs@wals@wae{Walser}
\def\langnames@langs@wals@waf{Wakoná}
\def\langnames@langs@wals@wag{Wa'ema}
\def\langnames@langs@wals@wah{Watubela}
\def\langnames@langs@wals@wai{Wares}
\def\langnames@langs@wals@waj{Waffa}
\def\langnames@langs@wals@wal{Wolaytta}
\def\langnames@langs@wals@wam{Wampanoag}
\def\langnames@langs@wals@wan{Wan}
\def\langnames@langs@wals@wao{Wappo}
\def\langnames@langs@wals@wap{Wapishana}
\def\langnames@langs@wals@waq{Wageman}
\def\langnames@langs@wals@war{Waray (Philippines)}
\def\langnames@langs@wals@was{Washo}
\def\langnames@langs@wals@wat{Kaninuwa}
\def\langnames@langs@wals@wau{Waurá}
\def\langnames@langs@wals@wav{Waka}
\def\langnames@langs@wals@waw{Waiwai}
\def\langnames@langs@wals@wax{Watam}
\def\langnames@langs@wals@way{Wayana}
\def\langnames@langs@wals@waz{Wampur}
\def\langnames@langs@wals@wba{Warao}
\def\langnames@langs@wals@wbb{Wabo}
\def\langnames@langs@wals@wbe{Waritai}
\def\langnames@langs@wals@wbf{Samue}
\def\langnames@langs@wals@wbh{Wanda}
\def\langnames@langs@wals@wbi{Vwanji}
\def\langnames@langs@wals@wbj{Alagwa}
\def\langnames@langs@wals@wbk{Nuristani Kalasha}
\def\langnames@langs@wals@wbl{Wakhi}
\def\langnames@langs@wals@wbm{Zhenkang Wa}
\def\langnames@langs@wals@wbp{Warlpiri}
\def\langnames@langs@wals@wbq{Waddar}
\def\langnames@langs@wals@wbr{Wagdi}
\def\langnames@langs@wals@wbt{Wanman}
\def\langnames@langs@wals@wbv{Wajarri}
\def\langnames@langs@wals@wbw{Woi}
\def\langnames@langs@wals@wca{Yanomám}
\def\langnames@langs@wals@wci{Waci Gbe}
\def\langnames@langs@wals@wdd{Wandji}
\def\langnames@langs@wals@wdg{Wadaginam}
\def\langnames@langs@wals@wdj{Wadjiginy}
\def\langnames@langs@wals@wdu{Wadjigu}
\def\langnames@langs@wals@wea{Wewaw}
\def\langnames@langs@wals@wec{Wè Western}
\def\langnames@langs@wals@wed{Wedau}
\def\langnames@langs@wals@weh{Weh}
\def\langnames@langs@wals@wei{Were}
\def\langnames@langs@wals@wem{Weme Gbe}
\def\langnames@langs@wals@weo{Wemale}
\def\langnames@langs@wals@wep{Westphalic}
\def\langnames@langs@wals@wer{Weri}
\def\langnames@langs@wals@wes{Cameroon Pidgin}
\def\langnames@langs@wals@wet{Perai}
\def\langnames@langs@wals@wew{Wewewa}
\def\langnames@langs@wals@wfg{Yafi}
\def\langnames@langs@wals@wga{Wagaya}
\def\langnames@langs@wals@wgb{Wagawaga}
\def\langnames@langs@wals@wgg{Wangganguru}
\def\langnames@langs@wals@wgi{Wahgi}
\def\langnames@langs@wals@wgo{Waigeo}
\def\langnames@langs@wals@wgu{Wirangu}
\def\langnames@langs@wals@wgy{Warrgamay}
\def\langnames@langs@wals@wha{Manusela}
\def\langnames@langs@wals@whg{North Wahgi}
\def\langnames@langs@wals@whk{Eastern Lowland Kenyah}
\def\langnames@langs@wals@wib{Southern Toussian}
\def\langnames@langs@wals@wic{Wichita}
\def\langnames@langs@wals@wie{Wik-Epa}
\def\langnames@langs@wals@wif{Wik-Keyangan}
\def\langnames@langs@wals@wig{Wik-Ngathana}
\def\langnames@langs@wals@wih{Wik-Me'anha}
\def\langnames@langs@wals@wii{Minidien}
\def\langnames@langs@wals@wij{Wik-Iiyanh}
\def\langnames@langs@wals@wik{Wikalkan}
\def\langnames@langs@wals@wil{Wilawila}
\def\langnames@langs@wals@wim{Wik-Mungkan}
\def\langnames@langs@wals@win{Ho-Chunk}
\def\langnames@langs@wals@wir{Wiraféd}
\def\langnames@langs@wals@wit{Wintu}
\def\langnames@langs@wals@wiu{Wiru}
\def\langnames@langs@wals@wiv{Muduapa}
\def\langnames@langs@wals@wiy{Wiyot}
\def\langnames@langs@wals@wja{Waja}
\def\langnames@langs@wals@wji{Warji}
\def\langnames@langs@wals@wka{Kw'adza}
\def\langnames@langs@wals@wkd{Wakde}
\def\langnames@langs@wals@wkl{Kalanadi}
\def\langnames@langs@wals@wku{Kunduvadi}
\def\langnames@langs@wals@wkw{Wakawaka}
\def\langnames@langs@wals@wla{Walio}
\def\langnames@langs@wals@wlc{Mwali Comorian}
\def\langnames@langs@wals@wle{Wolane}
\def\langnames@langs@wals@wlg{Kunbarlang}
\def\langnames@langs@wals@wlh{Welaun}
\def\langnames@langs@wals@wli{Waioli}
\def\langnames@langs@wals@wlk{Eel River Athabaskan}
\def\langnames@langs@wals@wll{Wali (Sudan)}
\def\langnames@langs@wals@wln{Walloon}
\def\langnames@langs@wals@wlo{Wolio}
\def\langnames@langs@wals@wlr{Ale}
\def\langnames@langs@wals@wls{East Uvean}
\def\langnames@langs@wals@wlu{Wuliwuli}
\def\langnames@langs@wals@wlv{Wichí Lhamtés Vejoz}
\def\langnames@langs@wals@wlw{Walak}
\def\langnames@langs@wals@wlx{Wali (Ghana)}
\def\langnames@langs@wals@wly{Waling}
\def\langnames@langs@wals@wma{Mawa (Nigeria)}
\def\langnames@langs@wals@wmb{Wambayan}
\def\langnames@langs@wals@wmc{Wamas}
\def\langnames@langs@wals@wmd{Mamaindé}
\def\langnames@langs@wals@wme{Wambule}
\def\langnames@langs@wals@wmg{Western Muya}
\def\langnames@langs@wals@wmh{Waima'a}
\def\langnames@langs@wals@wmi{Wamin}
\def\langnames@langs@wals@wmm{Maiwa (Indonesia)}
\def\langnames@langs@wals@wmn{Waamwang}
\def\langnames@langs@wals@wmo{Wom (Papua New Guinea)}
\def\langnames@langs@wals@wms{Wambon}
\def\langnames@langs@wals@wmt{Walmajarri}
\def\langnames@langs@wals@wmw{Mwani}
\def\langnames@langs@wals@wmx{Womo-Sumararu}
\def\langnames@langs@wals@wnb{Mokati}
\def\langnames@langs@wals@wnc{Wantoat}
\def\langnames@langs@wals@wnd{Wandarang}
\def\langnames@langs@wals@wne{Waneci}
\def\langnames@langs@wals@wng{Wanggom}
\def\langnames@langs@wals@wni{Ndzwani Comorian}
\def\langnames@langs@wals@wnk{Wanukaka}
\def\langnames@langs@wals@wnm{Wanggamala}
\def\langnames@langs@wals@wno{Wano}
\def\langnames@langs@wals@wnp{Wanap}
\def\langnames@langs@wals@wnu{Usan}
\def\langnames@langs@wals@wnw{Wintu}
\def\langnames@langs@wals@wny{Wanyi}
\def\langnames@langs@wals@woa{Tyaraity}
\def\langnames@langs@wals@wob{Wobe-Wè Northern}
\def\langnames@langs@wals@woc{Wogeo}
\def\langnames@langs@wals@wod{Wolani}
\def\langnames@langs@wals@woe{Woleaian}
\def\langnames@langs@wals@wof{Gambian Wolof}
\def\langnames@langs@wals@wog{Wogamusin}
\def\langnames@langs@wals@woi{Kamang}
\def\langnames@langs@wals@wok{Longto}
\def\langnames@langs@wals@wol{Wolof}
\def\langnames@langs@wals@wom{Wom (Nigeria)}
\def\langnames@langs@wals@won{Wongo}
\def\langnames@langs@wals@woo{Manombai}
\def\langnames@langs@wals@wor{Woria}
\def\langnames@langs@wals@wos{Hanga Hundi}
\def\langnames@langs@wals@wow{Wawonii}
\def\langnames@langs@wals@woy{Weyto}
\def\langnames@langs@wals@wpc{Maco}
\def\langnames@langs@wals@wrb{Warluwara}
\def\langnames@langs@wals@wre{Ware}
\def\langnames@langs@wals@wrg{Warrongo}
\def\langnames@langs@wals@wrh{Wiradhuri}
\def\langnames@langs@wals@wri{Wariyangga}
\def\langnames@langs@wals@wrk{Garrwa}
\def\langnames@langs@wals@wrl{Warlmanpa}
\def\langnames@langs@wals@wrm{Warumungu}
\def\langnames@langs@wals@wrn{Warnang}
\def\langnames@langs@wals@wro{Worrorra}
\def\langnames@langs@wals@wrp{Waropen}
\def\langnames@langs@wals@wrr{Wardaman}
\def\langnames@langs@wals@wrs{Waris}
\def\langnames@langs@wals@wru{Waru}
\def\langnames@langs@wals@wrv{Waruna}
\def\langnames@langs@wals@wrw{Roth's Gugu Warra}
\def\langnames@langs@wals@wrx{Kolor}
\def\langnames@langs@wals@wry{Merwari}
\def\langnames@langs@wals@wrz{Warray}
\def\langnames@langs@wals@wsa{Warembori}
\def\langnames@langs@wals@wsg{Adilabad Gondi}
\def\langnames@langs@wals@wsi{Kula (Vanuatu)}
\def\langnames@langs@wals@wsk{Waskia}
\def\langnames@langs@wals@wsr{Oweina}
\def\langnames@langs@wals@wss{Wasa}
\def\langnames@langs@wals@wsu{Wasu}
\def\langnames@langs@wals@wsv{Wotapuri-Katarqalai}
\def\langnames@langs@wals@wtf{Dumpu}
\def\langnames@langs@wals@wth{Wathawurrung}
\def\langnames@langs@wals@wti{Berta}
\def\langnames@langs@wals@wtk{Watakataui}
\def\langnames@langs@wals@wtm{Mewati}
\def\langnames@langs@wals@wtw{Wotu}
\def\langnames@langs@wals@wua{Wikngenchera}
\def\langnames@langs@wals@wub{Wunambal}
\def\langnames@langs@wals@wud{Wudu}
\def\langnames@langs@wals@wuh{Wutunhua}
\def\langnames@langs@wals@wul{Silimo}
\def\langnames@langs@wals@wum{Wumbvu}
\def\langnames@langs@wals@wun{Bungu}
\def\langnames@langs@wals@wur{Wurrugu}
\def\langnames@langs@wals@wut{Wutung}
\def\langnames@langs@wals@wuu{Wu Chinese}
\def\langnames@langs@wals@wuv{Wuvulu-Aua}
\def\langnames@langs@wals@wux{Wulna}
\def\langnames@langs@wals@wuy{Wauyai}
\def\langnames@langs@wals@wwa{Waama}
\def\langnames@langs@wals@wwb{Wakabunga}
\def\langnames@langs@wals@wwo{Dorig}
\def\langnames@langs@wals@wwr{Warrwa}
\def\langnames@langs@wals@www{Wawa}
\def\langnames@langs@wals@wxa{Waxianghua}
\def\langnames@langs@wals@wya{Huron-Wyandot}
\def\langnames@langs@wals@wyb{Ngiyambaa}
\def\langnames@langs@wals@wyi{Woiwurrung-Thagungwurrung}
\def\langnames@langs@wals@wym{Wymysorys}
\def\langnames@langs@wals@wyr{Wayoró}
\def\langnames@langs@wals@wyy{Western Fijian}
\def\langnames@langs@wals@xaa{Andalusian Arabic}
\def\langnames@langs@wals@xab{Sambe}
\def\langnames@langs@wals@xac{Kachari}
\def\langnames@langs@wals@xad{Adai}
\def\langnames@langs@wals@xag{Aghwan}
\def\langnames@langs@wals@xai{Kaimbé}
\def\langnames@langs@wals@xak{Máku}
\def\langnames@langs@wals@xal{Oirad-Kalmyk-Darkhat}
\def\langnames@langs@wals@xam{/Xam}
\def\langnames@langs@wals@xan{Xamtanga}
\def\langnames@langs@wals@xap{Apalachee}
\def\langnames@langs@wals@xar{Karami}
\def\langnames@langs@wals@xas{Kamas-Koibal}
\def\langnames@langs@wals@xat{Katawixi}
\def\langnames@langs@wals@xau{Kauwera}
\def\langnames@langs@wals@xav{Xavánte}
\def\langnames@langs@wals@xaw{Kawaiisu}
\def\langnames@langs@wals@xay{Kayan Mahakam}
\def\langnames@langs@wals@xbc{Bactrian}
\def\langnames@langs@wals@xbe{Bigambal}
\def\langnames@langs@wals@xbg{Bunganditj}
\def\langnames@langs@wals@xbi{Kombio}
\def\langnames@langs@wals@xbn{Kenaboi}
\def\langnames@langs@wals@xbo{Bolgarian}
\def\langnames@langs@wals@xbr{Kambera}
\def\langnames@langs@wals@xbw{Kambiwá}
\def\langnames@langs@wals@xcc{Camunic}
\def\langnames@langs@wals@xce{Celtiberian}
\def\langnames@langs@wals@xcg{Cisalpine Gaulish}
\def\langnames@langs@wals@xch{Chimakum}
\def\langnames@langs@wals@xcl{Classical-Middle Armenian}
\def\langnames@langs@wals@xcm{Comecrudan}
\def\langnames@langs@wals@xcn{Cotoname}
\def\langnames@langs@wals@xco{Khwarezmian}
\def\langnames@langs@wals@xcr{Carian}
\def\langnames@langs@wals@xct{Classical Tibetan}
\def\langnames@langs@wals@xcv{Chuvantsy}
\def\langnames@langs@wals@xcw{Coahuilteco}
\def\langnames@langs@wals@xcy{Cayuse}
\def\langnames@langs@wals@xda{Hawkesbury}
\def\langnames@langs@wals@xdc{Dacian}
\def\langnames@langs@wals@xdk{Sydney}
\def\langnames@langs@wals@xdo{Kwandu}
\def\langnames@langs@wals@xdq{Kajtak}
\def\langnames@langs@wals@xdy{Malayic Dayak}
\def\langnames@langs@wals@xeb{Eblaite}
\def\langnames@langs@wals@xed{Hdi}
\def\langnames@langs@wals@xeg{//Xegwi}
\def\langnames@langs@wals@xel{Kelo}
\def\langnames@langs@wals@xem{Mateq}
\def\langnames@langs@wals@xer{Xerénte}
\def\langnames@langs@wals@xes{Kesawai}
\def\langnames@langs@wals@xet{Xetá}
\def\langnames@langs@wals@xeu{Keoru-Ahia}
\def\langnames@langs@wals@xfa{Faliscan}
\def\langnames@langs@wals@xga{Galatian}
\def\langnames@langs@wals@xgb{Gbin}
\def\langnames@langs@wals@xgd{Gudang}
\def\langnames@langs@wals@xgf{Tongva}
\def\langnames@langs@wals@xgm{Dharumbal}
\def\langnames@langs@wals@xgu{Unggumi}
\def\langnames@langs@wals@xgw{Guwa}
\def\langnames@langs@wals@xhd{Hadrami}
\def\langnames@langs@wals@xhe{Khetrani}
\def\langnames@langs@wals@xho{Xhosa}
\def\langnames@langs@wals@xht{Hattic}
\def\langnames@langs@wals@xhu{Hurrian}
\def\langnames@langs@wals@xib{Iberian}
\def\langnames@langs@wals@xii{Xiri}
\def\langnames@langs@wals@xil{Illyrian}
\def\langnames@langs@wals@xip{Xipináwa}
\def\langnames@langs@wals@xir{Xiriâna}
\def\langnames@langs@wals@xiv{Harappan}
\def\langnames@langs@wals@xiy{Xipaya}
\def\langnames@langs@wals@xjb{Tweed-Albert}
\def\langnames@langs@wals@xka{Kalkoti}
\def\langnames@langs@wals@xkb{Manigri-Kambolé Ede Nago}
\def\langnames@langs@wals@xkc{Kho'ini}
\def\langnames@langs@wals@xkd{Mendalam Kayan}
\def\langnames@langs@wals@xke{Kereho}
\def\langnames@langs@wals@xkf{Khengkha}
\def\langnames@langs@wals@xkg{Kagoro}
\def\langnames@langs@wals@xkh{Karahawyana}
\def\langnames@langs@wals@xki{Kenya-Somali Sign Language}
\def\langnames@langs@wals@xkj{Kajali}
\def\langnames@langs@wals@xkk{Kaco'}
\def\langnames@langs@wals@xkl{Usun Apau Kenyah}
\def\langnames@langs@wals@xkn{Kayan River Kayan}
\def\langnames@langs@wals@xkp{Kabatei}
\def\langnames@langs@wals@xkq{Koroni}
\def\langnames@langs@wals@xkr{Xakriabá}
\def\langnames@langs@wals@xks{Kumbewaha}
\def\langnames@langs@wals@xkt{Kantosi}
\def\langnames@langs@wals@xku{Kaamba}
\def\langnames@langs@wals@xkv{Kgalagadi}
\def\langnames@langs@wals@xkw{Kembra}
\def\langnames@langs@wals@xkx{Karore}
\def\langnames@langs@wals@xky{Uma' Lasan}
\def\langnames@langs@wals@xkz{Kurtokha}
\def\langnames@langs@wals@xla{Kamula}
\def\langnames@langs@wals@xlc{Lycian A}
\def\langnames@langs@wals@xld{Lydian}
\def\langnames@langs@wals@xle{Lemnian}
\def\langnames@langs@wals@xlg{Ancient Ligurian}
\def\langnames@langs@wals@xlo{Loup A}
\def\langnames@langs@wals@xlp{Lepontic}
\def\langnames@langs@wals@xls{Lusitanian}
\def\langnames@langs@wals@xlu{Cuneiform Luwian}
\def\langnames@langs@wals@xly{Elymian}
\def\langnames@langs@wals@xmb{Mbonga}
\def\langnames@langs@wals@xmc{Makhuwa-Marrevone}
\def\langnames@langs@wals@xmd{Mbedam}
\def\langnames@langs@wals@xmf{Mingrelian}
\def\langnames@langs@wals@xmg{Mengaka}
\def\langnames@langs@wals@xmh{Kuku-Muminh}
\def\langnames@langs@wals@xmi{Miarrã}
\def\langnames@langs@wals@xmj{Majera}
\def\langnames@langs@wals@xml{Malaysian Sign Language}
\def\langnames@langs@wals@xmm{Manado Malay}
\def\langnames@langs@wals@xmo{Morerebi}
\def\langnames@langs@wals@xmp{Kuku-Mu'inh}
\def\langnames@langs@wals@xmr{Meroitic}
\def\langnames@langs@wals@xms{Moroccan Sign Language}
\def\langnames@langs@wals@xmt{Matbat}
\def\langnames@langs@wals@xmu{Kamu}
\def\langnames@langs@wals@xmv{Antankarana Malagasy}
\def\langnames@langs@wals@xmw{Tsimihety Malagasy}
\def\langnames@langs@wals@xmx{Salawati}
\def\langnames@langs@wals@xmy{Mayaguduna}
\def\langnames@langs@wals@xmz{Mori Bawah}
\def\langnames@langs@wals@xna{Ancient North Arabian}
\def\langnames@langs@wals@xnb{Kanakanavu}
\def\langnames@langs@wals@xng{Middle Mongol}
\def\langnames@langs@wals@xnj{Tanzanian Ngoni}
\def\langnames@langs@wals@xnm{Ngumbarl}
\def\langnames@langs@wals@xnn{Northern Kankanay}
\def\langnames@langs@wals@xnq{Mozambican Ngoni}
\def\langnames@langs@wals@xnr{Kangri}
\def\langnames@langs@wals@xns{Kanashi}
\def\langnames@langs@wals@xny{Nyiyaparli-Palyku}
\def\langnames@langs@wals@xnz{Nubian (Kunuz)}
\def\langnames@langs@wals@xod{Kokoda}
\def\langnames@langs@wals@xog{Soga}
\def\langnames@langs@wals@xoi{Kominimung}
\def\langnames@langs@wals@xok{Xokleng}
\def\langnames@langs@wals@xom{Komo (Sudan-Ethiopia)}
\def\langnames@langs@wals@xon{Konkomba}
\def\langnames@langs@wals@xoo{Xukurú}
\def\langnames@langs@wals@xop{Kopar}
\def\langnames@langs@wals@xor{Korubo}
\def\langnames@langs@wals@xow{Kowaki}
\def\langnames@langs@wals@xpa{Pirriya}
\def\langnames@langs@wals@xpc{Pecheneg}
\def\langnames@langs@wals@xpe{Liberia Kpelle}
\def\langnames@langs@wals@xpg{Phrygian}
\def\langnames@langs@wals@xpi{Pictish}
\def\langnames@langs@wals@xpk{Kulina Pano}
\def\langnames@langs@wals@xpm{Pumpokol}
\def\langnames@langs@wals@xpn{Kapinawá}
\def\langnames@langs@wals@xpo{Pochutec}
\def\langnames@langs@wals@xpq{Mahican}
\def\langnames@langs@wals@xpr{Parthian}
\def\langnames@langs@wals@xps{Pisidian}
\def\langnames@langs@wals@xpu{Punic}
\def\langnames@langs@wals@xqt{Qatabanian}
\def\langnames@langs@wals@xra{Krahô}
\def\langnames@langs@wals@xrb{Eastern Karaboro}
\def\langnames@langs@wals@xre{Northeastern Timbira}
\def\langnames@langs@wals@xrn{Arin}
\def\langnames@langs@wals@xrr{Raetic}
\def\langnames@langs@wals@xrt{Aranama}
\def\langnames@langs@wals@xru{Marriammu}
\def\langnames@langs@wals@xrw{Karawa}
\def\langnames@langs@wals@xsa{Sabaic}
\def\langnames@langs@wals@xsb{Tinà Sambal}
\def\langnames@langs@wals@xsd{Sidetic}
\def\langnames@langs@wals@xse{Sempan}
\def\langnames@langs@wals@xsh{Shamang}
\def\langnames@langs@wals@xsi{Sio}
\def\langnames@langs@wals@xsl{South Slavey}
\def\langnames@langs@wals@xsm{Kasem}
\def\langnames@langs@wals@xsn{Sanga (Nigeria)}
\def\langnames@langs@wals@xso{San Francisco Solano}
\def\langnames@langs@wals@xsp{Silopi}
\def\langnames@langs@wals@xsq{Makhuwa-Saka}
\def\langnames@langs@wals@xsr{Solu-Khumbu Sherpa}
\def\langnames@langs@wals@xsu{Sanumá}
\def\langnames@langs@wals@xsy{Saisiyat}
\def\langnames@langs@wals@xta{Alcozauca Mixtec}
\def\langnames@langs@wals@xtb{Chazumba Mixtec}
\def\langnames@langs@wals@xtc{Katcha-Kadugli-Miri}
\def\langnames@langs@wals@xtd{Diuxi-Tilantongo Mixtec}
\def\langnames@langs@wals@xte{Ketengban}
\def\langnames@langs@wals@xtg{Transalpine Gaulish}
\def\langnames@langs@wals@xti{Sinicahua Mixtec}
\def\langnames@langs@wals@xtj{San Juan Teita Mixtec}
\def\langnames@langs@wals@xtl{Tijaltepec Mixtec}
\def\langnames@langs@wals@xtm{Magdalena Peñasco Mixtec}
\def\langnames@langs@wals@xtn{Northern Tlaxiaco Mixtec}
\def\langnames@langs@wals@xto{Tokharian A}
\def\langnames@langs@wals@xtp{San Miguel Piedras Mixtec}
\def\langnames@langs@wals@xtq{Tumshuqese}
\def\langnames@langs@wals@xts{Sindihui Mixtec}
\def\langnames@langs@wals@xtt{Tacahua-Yolotepec Mixtec}
\def\langnames@langs@wals@xtu{Cuyamecalco Mixtec}
\def\langnames@langs@wals@xtv{Southern Coastal Yuin}
\def\langnames@langs@wals@xtw{Tawandê}
\def\langnames@langs@wals@xty{Yoloxochitl Mixtec}
\def\langnames@langs@wals@xua{Alu Kurumba}
\def\langnames@langs@wals@xub{Betta Kurumba}
\def\langnames@langs@wals@xug{Kunigami}
\def\langnames@langs@wals@xuj{Jennu Kurumba}
\def\langnames@langs@wals@xum{Umbrian}
\def\langnames@langs@wals@xuo{Kuo}
\def\langnames@langs@wals@xup{Upper Umpqua}
\def\langnames@langs@wals@xur{Urartian}
\def\langnames@langs@wals@xut{Kuthant}
\def\langnames@langs@wals@xuu{Kxoe}
\def\langnames@langs@wals@xve{Venetic}
\def\langnames@langs@wals@xwa{Kwaza}
\def\langnames@langs@wals@xwc{Woccon}
\def\langnames@langs@wals@xwe{Xwela Gbe}
\def\langnames@langs@wals@xwg{Kwegu}
\def\langnames@langs@wals@xwk{Wangkumara}
\def\langnames@langs@wals@xwl{Western Xwla Gbe}
\def\langnames@langs@wals@xwr{Kwerba Mamberamo}
\def\langnames@langs@wals@xww{Wembawemba}
\def\langnames@langs@wals@xxb{Boro}
\def\langnames@langs@wals@xxk{Kéo}
\def\langnames@langs@wals@xxm{Minkin}
\def\langnames@langs@wals@xxr{Koropó}
\def\langnames@langs@wals@xxt{Tambora}
\def\langnames@langs@wals@xya{Yaygir}
\def\langnames@langs@wals@xyb{Yandjibara}
\def\langnames@langs@wals@xyj{Mayi-Yapi}
\def\langnames@langs@wals@xyl{Yalakalore}
\def\langnames@langs@wals@xyy{Yorta Yorta}
\def\langnames@langs@wals@xzh{Zhangzhung}
\def\langnames@langs@wals@yaa{Yaminahua}
\def\langnames@langs@wals@yab{Yuhup}
\def\langnames@langs@wals@yac{Pass Valley Yali}
\def\langnames@langs@wals@yad{Yagua}
\def\langnames@langs@wals@yae{Pumé}
\def\langnames@langs@wals@yaf{Yaka-Pelende-Lonzo}
\def\langnames@langs@wals@yag{Yámana}
\def\langnames@langs@wals@yah{Yazgulyam}
\def\langnames@langs@wals@yai{Yagnobi}
\def\langnames@langs@wals@yaj{Banda-Yangere}
\def\langnames@langs@wals@yak{Northwest Sahaptin}
\def\langnames@langs@wals@yal{Yalunka}
\def\langnames@langs@wals@yam{Yamba}
\def\langnames@langs@wals@yan{Mayangna}
\def\langnames@langs@wals@yao{Yao}
\def\langnames@langs@wals@yap{Yapese}
\def\langnames@langs@wals@yaq{Yaqui}
\def\langnames@langs@wals@yar{Yabarana}
\def\langnames@langs@wals@yas{Nugunu (Cameroon)}
\def\langnames@langs@wals@yat{Yambeta}
\def\langnames@langs@wals@yau{Hoti}
\def\langnames@langs@wals@yav{Yangben}
\def\langnames@langs@wals@yaw{Yawalapití}
\def\langnames@langs@wals@yay{Agwagwune}
\def\langnames@langs@wals@yaz{Lokaa}
\def\langnames@langs@wals@yba{Yala}
\def\langnames@langs@wals@ybb{Yemba}
\def\langnames@langs@wals@ybe{West Yugur}
\def\langnames@langs@wals@ybh{Yakkha}
\def\langnames@langs@wals@ybi{Yamphu}
\def\langnames@langs@wals@ybj{Hasha}
\def\langnames@langs@wals@ybk{Bokha}
\def\langnames@langs@wals@ybl{Yukuben}
\def\langnames@langs@wals@ybm{Yaben}
\def\langnames@langs@wals@ybn{Yabaâna-Mainatari}
\def\langnames@langs@wals@ybo{Yabong}
\def\langnames@langs@wals@ybx{Yawiyo}
\def\langnames@langs@wals@yby{Yaweyuha}
\def\langnames@langs@wals@ych{Chesu}
\def\langnames@langs@wals@ycl{Lolopo}
\def\langnames@langs@wals@ycn{Yucuna}
\def\langnames@langs@wals@ycp{Chepya}
\def\langnames@langs@wals@yda{Yanda}
\def\langnames@langs@wals@ydd{Eastern Yiddish}
\def\langnames@langs@wals@yde{Yangum Dey}
\def\langnames@langs@wals@ydg{Yidgha}
\def\langnames@langs@wals@ydk{Yoidik}
\def\langnames@langs@wals@yea{Ravula}
\def\langnames@langs@wals@yec{Yeniche}
\def\langnames@langs@wals@yee{Yimas}
\def\langnames@langs@wals@yei{Yeni}
\def\langnames@langs@wals@yej{Yevanic}
\def\langnames@langs@wals@yel{Yela-Kela}
\def\langnames@langs@wals@yer{Tarok}
\def\langnames@langs@wals@yes{Yeskwa}
\def\langnames@langs@wals@yet{Yetfa}
\def\langnames@langs@wals@yeu{Yerukula}
\def\langnames@langs@wals@yev{Yeri}
\def\langnames@langs@wals@yey{Yeyi}
\def\langnames@langs@wals@ygl{Yangum Gel}
\def\langnames@langs@wals@ygm{Yagomi}
\def\langnames@langs@wals@ygp{Gepo}
\def\langnames@langs@wals@ygr{Yagaria}
\def\langnames@langs@wals@ygs{Yolngu Sign Language}
\def\langnames@langs@wals@ygu{Yugul}
\def\langnames@langs@wals@ygw{Yagwoia}
\def\langnames@langs@wals@yha{Baha Buyang}
\def\langnames@langs@wals@yhd{Judeo-Iraqi Arabic}
\def\langnames@langs@wals@yhl{Hlepho Phowa}
\def\langnames@langs@wals@yia{Yinggarda}
\def\langnames@langs@wals@yif{Ache}
\def\langnames@langs@wals@yig{Wusa Nasu}
\def\langnames@langs@wals@yih{Western Yiddish}
\def\langnames@langs@wals@yii{Yidiñ}
\def\langnames@langs@wals@yij{Yindjibarndi}
\def\langnames@langs@wals@yik{Dongshanba Lalo}
\def\langnames@langs@wals@yil{Yindjilandji}
\def\langnames@langs@wals@yim{Yimchungru Naga}
\def\langnames@langs@wals@yin{Yinchia}
\def\langnames@langs@wals@yip{Pholo}
\def\langnames@langs@wals@yiq{Miqie}
\def\langnames@langs@wals@yir{North Awyu}
\def\langnames@langs@wals@yis{Yis}
\def\langnames@langs@wals@yit{Eastern Lalu}
\def\langnames@langs@wals@yiu{Southern Awu (Lope)}
\def\langnames@langs@wals@yiv{Northern Nisu}
\def\langnames@langs@wals@yix{Axi Yi}
\def\langnames@langs@wals@yiy{Yir-Yoront}
\def\langnames@langs@wals@yiz{Azhe}
\def\langnames@langs@wals@yka{Yakan}
\def\langnames@langs@wals@ykg{Northern Yukaghir}
\def\langnames@langs@wals@yki{Yoke}
\def\langnames@langs@wals@ykk{Yakaikeke}
\def\langnames@langs@wals@ykl{Khlula}
\def\langnames@langs@wals@ykm{Kap}
\def\langnames@langs@wals@ykn{Kua-nsi}
\def\langnames@langs@wals@yko{Yasa}
\def\langnames@langs@wals@ykr{Yekora}
\def\langnames@langs@wals@ykt{Thou-Kathu}
\def\langnames@langs@wals@yku{Kuamasi}
\def\langnames@langs@wals@yky{Yakoma}
\def\langnames@langs@wals@yla{Ulwa (Papua New Guinea)}
\def\langnames@langs@wals@yle{Yele}
\def\langnames@langs@wals@ylg{Yalaku}
\def\langnames@langs@wals@yli{Angguruk Yali}
\def\langnames@langs@wals@yll{Yil}
\def\langnames@langs@wals@ylm{Limi}
\def\langnames@langs@wals@yln{Langnian Buyang}
\def\langnames@langs@wals@ylo{Naluo Yi}
\def\langnames@langs@wals@ylr{Yalarnnga}
\def\langnames@langs@wals@ylu{Aribwaung}
\def\langnames@langs@wals@yly{Belep}
\def\langnames@langs@wals@ymb{Yambes}
\def\langnames@langs@wals@ymc{Southern Muji}
\def\langnames@langs@wals@ymd{Muda}
\def\langnames@langs@wals@yme{Yameo}
\def\langnames@langs@wals@ymh{Mili}
\def\langnames@langs@wals@ymi{Moji}
\def\langnames@langs@wals@ymk{Makwe}
\def\langnames@langs@wals@yml{Iamalele}
\def\langnames@langs@wals@ymm{Maay}
\def\langnames@langs@wals@ymn{Yamna}
\def\langnames@langs@wals@ymo{Yangum Mon}
\def\langnames@langs@wals@ymp{Yamap}
\def\langnames@langs@wals@ymq{Qila Muji}
\def\langnames@langs@wals@ymr{Malasar}
\def\langnames@langs@wals@ymx{Northern Muji}
\def\langnames@langs@wals@ymz{Muzi}
\def\langnames@langs@wals@yna{Aluo}
\def\langnames@langs@wals@ynd{Yandruwandha}
\def\langnames@langs@wals@yng{Yango}
\def\langnames@langs@wals@ynk{Naukan Yupik}
\def\langnames@langs@wals@ynl{Yangulam}
\def\langnames@langs@wals@ynn{Yana}
\def\langnames@langs@wals@yno{Yong}
\def\langnames@langs@wals@ynq{Yendang}
\def\langnames@langs@wals@yns{Yansi}
\def\langnames@langs@wals@ynu{Yahuna}
\def\langnames@langs@wals@yob{Yoba}
\def\langnames@langs@wals@yog{Yogad}
\def\langnames@langs@wals@yoi{Yonaguni}
\def\langnames@langs@wals@yok{Northern Yokuts}
\def\langnames@langs@wals@yol{Irish Anglo-Norman}
\def\langnames@langs@wals@yom{Yombe}
\def\langnames@langs@wals@yon{Yonggom}
\def\langnames@langs@wals@yor{Yoruba}
\def\langnames@langs@wals@yot{Yotti}
\def\langnames@langs@wals@yox{Yoron}
\def\langnames@langs@wals@yoy{Yoy}
\def\langnames@langs@wals@ypa{Phala}
\def\langnames@langs@wals@ypb{Labo Phowa}
\def\langnames@langs@wals@ypg{Phola}
\def\langnames@langs@wals@yph{Phupha}
\def\langnames@langs@wals@ypm{Phuma}
\def\langnames@langs@wals@ypn{Ani Phowa}
\def\langnames@langs@wals@ypo{Alo Phola}
\def\langnames@langs@wals@ypp{Phupa}
\def\langnames@langs@wals@ypz{Phuza}
\def\langnames@langs@wals@yra{Yerakai}
\def\langnames@langs@wals@yrb{Yareba}
\def\langnames@langs@wals@yre{Yaouré}
\def\langnames@langs@wals@yrk{Tundra Nenets}
\def\langnames@langs@wals@yrl{Nhengatu}
\def\langnames@langs@wals@yrn{Yerong-Southern Buyang}
\def\langnames@langs@wals@yro{Yaroame}
\def\langnames@langs@wals@yrw{Yarawata}
\def\langnames@langs@wals@ysd{Samatao}
\def\langnames@langs@wals@ysg{Sonaga}
\def\langnames@langs@wals@ysl{Yugoslavian Sign Language}
\def\langnames@langs@wals@ysm{Yangon Myanmar Sign Language}
\def\langnames@langs@wals@ysn{Sani}
\def\langnames@langs@wals@yso{Nisi (China)}
\def\langnames@langs@wals@ysr{Old Sirenik}
\def\langnames@langs@wals@yss{Yessan-Mayo}
\def\langnames@langs@wals@ysy{Sanie}
\def\langnames@langs@wals@yta{Lavu-Yongsheng-Talu}
\def\langnames@langs@wals@ytl{Tanglang-Toloza}
\def\langnames@langs@wals@ytp{Thopho}
\def\langnames@langs@wals@ytw{Yout Wam}
\def\langnames@langs@wals@yua{Yucatec Maya}
\def\langnames@langs@wals@yub{Yugambal}
\def\langnames@langs@wals@yuc{Yuchi}
\def\langnames@langs@wals@yud{Judeo-Tripolitanian Arabic}
\def\langnames@langs@wals@yue{Yue Chinese}
\def\langnames@langs@wals@yuf{Havasupai-Walapai-Yavapai}
\def\langnames@langs@wals@yug{Yugh}
\def\langnames@langs@wals@yui{Yurutí}
\def\langnames@langs@wals@yuj{Karkar-Yuri}
\def\langnames@langs@wals@yuk{Northern Yukian}
\def\langnames@langs@wals@yul{Yulu-Binga}
\def\langnames@langs@wals@yum{Quechan}
\def\langnames@langs@wals@yun{Bena (Nigeria)}
\def\langnames@langs@wals@yup{Yukpa}
\def\langnames@langs@wals@yuq{Yuqui}
\def\langnames@langs@wals@yur{Yurok}
\def\langnames@langs@wals@yut{Yopno}
\def\langnames@langs@wals@yuw{Yau-Nungon}
\def\langnames@langs@wals@yux{Southern Yukaghir}
\def\langnames@langs@wals@yuy{East Yugur}
\def\langnames@langs@wals@yuz{Yuracaré}
\def\langnames@langs@wals@yva{Yawa}
\def\langnames@langs@wals@yvt{Yavitero-Pareni}
\def\langnames@langs@wals@ywa{Kalou}
\def\langnames@langs@wals@ywg{Yinhawangka}
\def\langnames@langs@wals@ywl{Western Lalu}
\def\langnames@langs@wals@ywn{Yawanawa}
\def\langnames@langs@wals@ywq{Wuding-Luquan Yi}
\def\langnames@langs@wals@ywr{Yawuru}
\def\langnames@langs@wals@ywt{Xishanba Lalo}
\def\langnames@langs@wals@ywu{Wumeng Nasu}
\def\langnames@langs@wals@yww{Yawarawarga}
\def\langnames@langs@wals@yxm{Yinwum}
\def\langnames@langs@wals@yyr{Yir Yoront}
\def\langnames@langs@wals@yyu{Yau (Sandaun Province)}
\def\langnames@langs@wals@yyz{Ayizi}
\def\langnames@langs@wals@yzg{E'ma Buyang}
\def\langnames@langs@wals@yzk{Zokhuo}
\def\langnames@langs@wals@zaa{Sierra de Juárez Zapotec}
\def\langnames@langs@wals@zab{Western Tlacolula Valley Zapotec}
\def\langnames@langs@wals@zac{Ocotlán Zapotec}
\def\langnames@langs@wals@zad{Cajonos Zapotec}
\def\langnames@langs@wals@zae{Yareni Zapotec}
\def\langnames@langs@wals@zaf{Ayoquesco Zapotec}
\def\langnames@langs@wals@zag{Beria}
\def\langnames@langs@wals@zah{Zangwal}
\def\langnames@langs@wals@zai{Isthmus Zapotec}
\def\langnames@langs@wals@zaj{Zaramo}
\def\langnames@langs@wals@zak{Zanaki}
\def\langnames@langs@wals@zal{Zauzou}
\def\langnames@langs@wals@zam{Cuixtla-Xitla Zapotec}
\def\langnames@langs@wals@zao{Ozolotepec Zapotec}
\def\langnames@langs@wals@zaq{Aloápam Zapotec}
\def\langnames@langs@wals@zar{Rincón Zapotec}
\def\langnames@langs@wals@zas{Santo Domingo Albarradas Zapotec}
\def\langnames@langs@wals@zat{Tabaa Zapotec}
\def\langnames@langs@wals@zau{Zangskari}
\def\langnames@langs@wals@zav{Yatzachi Zapotec}
\def\langnames@langs@wals@zaw{Mitla Zapotec}
\def\langnames@langs@wals@zax{Xadani Zapotec}
\def\langnames@langs@wals@zay{Zayse-Zergulla}
\def\langnames@langs@wals@zaz{Zari}
\def\langnames@langs@wals@zbc{Central Berawan}
\def\langnames@langs@wals@zbe{East Berawan}
\def\langnames@langs@wals@zbl{Blissymbols}
\def\langnames@langs@wals@zbt{Batui}
\def\langnames@langs@wals@zbu{Bu (Zaranda)}
\def\langnames@langs@wals@zbw{West Berawan}
\def\langnames@langs@wals@zca{Coatecas Altas Zapotec}
\def\langnames@langs@wals@zch{Central Hongshuihe Zhuang}
\def\langnames@langs@wals@zdj{Ngazidja Comorian}
\def\langnames@langs@wals@zea{Zeeuws}
\def\langnames@langs@wals@zeg{Zenag}
\def\langnames@langs@wals@zeh{Eastern Hongshuihe Zhuang}
\def\langnames@langs@wals@zen{Zenaga}
\def\langnames@langs@wals@zga{Kinga}
\def\langnames@langs@wals@zgb{Guibei Zhuang}
\def\langnames@langs@wals@zgm{Minz Zhuang}
\def\langnames@langs@wals@zgn{Guibian Zhuang}
\def\langnames@langs@wals@zgr{Magori}
\def\langnames@langs@wals@zhb{Zhaba}
\def\langnames@langs@wals@zhd{Dai Zhuang}
\def\langnames@langs@wals@zhi{Zhire}
\def\langnames@langs@wals@zhn{Nong Zhuang}
\def\langnames@langs@wals@zhw{Zhoa}
\def\langnames@langs@wals@zia{Zia}
\def\langnames@langs@wals@zib{Zimbabwe Sign Language}
\def\langnames@langs@wals@zik{Zimakani}
\def\langnames@langs@wals@zil{Zialo}
\def\langnames@langs@wals@zim{Mesme}
\def\langnames@langs@wals@zin{Zinza}
\def\langnames@langs@wals@ziw{Zigula-Mushungulu}
\def\langnames@langs@wals@ziz{Zizilivakan}
\def\langnames@langs@wals@zka{Kaimbulawa}
\def\langnames@langs@wals@zkg{Koguryo}
\def\langnames@langs@wals@zkk{Karankawa}
\def\langnames@langs@wals@zko{Kott-Assan}
\def\langnames@langs@wals@zkp{São Paulo Kaingáng}
\def\langnames@langs@wals@zkr{Zakhring}
\def\langnames@langs@wals@zkt{Kitan}
\def\langnames@langs@wals@zku{Kaurna}
\def\langnames@langs@wals@zla{Zula}
\def\langnames@langs@wals@zlj{Liujiang Zhuang}
\def\langnames@langs@wals@zlm{Central Malay}
\def\langnames@langs@wals@zln{Lianshan Zhuang}
\def\langnames@langs@wals@zlq{Liuqian Zhuang}
\def\langnames@langs@wals@zma{Manda (Australia)}
\def\langnames@langs@wals@zmb{Zimba}
\def\langnames@langs@wals@zmc{Margany}
\def\langnames@langs@wals@zmd{Maridan}
\def\langnames@langs@wals@zme{Mangerr}
\def\langnames@langs@wals@zmf{Mfinu}
\def\langnames@langs@wals@zmg{Marti Ke}
\def\langnames@langs@wals@zmh{Makolkol}
\def\langnames@langs@wals@zmi{Negeri Sembilan Malay}
\def\langnames@langs@wals@zmj{Maridjabin}
\def\langnames@langs@wals@zmk{Mandandanyi}
\def\langnames@langs@wals@zml{Madngele}
\def\langnames@langs@wals@zmm{Marimanindji}
\def\langnames@langs@wals@zmn{Mbangwe}
\def\langnames@langs@wals@zmo{Molo}
\def\langnames@langs@wals@zmp{Mbuun}
\def\langnames@langs@wals@zmq{Mituku}
\def\langnames@langs@wals@zmr{Maranunggu}
\def\langnames@langs@wals@zms{Mbesa}
\def\langnames@langs@wals@zmt{Maringarr}
\def\langnames@langs@wals@zmu{Muruwari}
\def\langnames@langs@wals@zmv{Rimanggudhinma}
\def\langnames@langs@wals@zmw{Mbo (Democratic Republic of Congo)}
\def\langnames@langs@wals@zmx{Bomitaba}
\def\langnames@langs@wals@zmy{Mariyedi}
\def\langnames@langs@wals@zmz{Mbandja}
\def\langnames@langs@wals@zna{Zan Gula}
\def\langnames@langs@wals@zne{Zande}
\def\langnames@langs@wals@zng{Mang}
\def\langnames@langs@wals@znk{Manangkari}
\def\langnames@langs@wals@zns{Mangas}
\def\langnames@langs@wals@zoc{Copainalá Zoque}
\def\langnames@langs@wals@zoh{Chimalapa Zoque}
\def\langnames@langs@wals@zom{Zou}
\def\langnames@langs@wals@zoo{Asunción Mixtepec Zapotec}
\def\langnames@langs@wals@zoq{Tabasco Zoque}
\def\langnames@langs@wals@zor{Rayón Zoque}
\def\langnames@langs@wals@zos{Francisco León Zoque}
\def\langnames@langs@wals@zpa{Lachiguiri Zapotec}
\def\langnames@langs@wals@zpb{Yautepec Zapotec}
\def\langnames@langs@wals@zpc{Choapan Zapotec}
\def\langnames@langs@wals@zpd{Southeastern Ixtlán Zapotec}
\def\langnames@langs@wals@zpe{Petapa Zapotec}
\def\langnames@langs@wals@zpf{San Pedro Quiatoni Zapotec}
\def\langnames@langs@wals@zpg{Guevea De Humboldt Zapotec}
\def\langnames@langs@wals@zph{Totomachapan Zapotec}
\def\langnames@langs@wals@zpi{Santa María Quiegolani Zapotec}
\def\langnames@langs@wals@zpj{Quiavicuzas Zapotec}
\def\langnames@langs@wals@zpk{Tlacolulita Zapotec}
\def\langnames@langs@wals@zpl{Lachixío Zapotec}
\def\langnames@langs@wals@zpm{Mixtepec Zapotec}
\def\langnames@langs@wals@zpn{Santa Inés Yatzechi Zapotec}
\def\langnames@langs@wals@zpo{Amatlán Zapotec}
\def\langnames@langs@wals@zpp{El Alto Zapotec}
\def\langnames@langs@wals@zpq{Zoogocho Zapotec}
\def\langnames@langs@wals@zpr{Santiago Xanica Zapotec}
\def\langnames@langs@wals@zps{Coatlán Zapotec}
\def\langnames@langs@wals@zpt{San Vicente Coatlán Zapotec}
\def\langnames@langs@wals@zpu{Yalálag Zapotec}
\def\langnames@langs@wals@zpv{Chichicapan Zapotec}
\def\langnames@langs@wals@zpw{Zaniza Zapotec}
\def\langnames@langs@wals@zpx{San Baltazar Loxicha Zapotec}
\def\langnames@langs@wals@zpy{Mazaltepec Zapotec}
\def\langnames@langs@wals@zpz{Texmelucan Zapotec}
\def\langnames@langs@wals@zqe{Qiubei Zhuang}
\def\langnames@langs@wals@zrn{Zirenkel}
\def\langnames@langs@wals@zro{Záparo}
\def\langnames@langs@wals@zrs{Mairasi}
\def\langnames@langs@wals@zsa{Sarasira}
\def\langnames@langs@wals@zsl{Zambian Sign Language}
\def\langnames@langs@wals@zsm{Standard Malay}
\def\langnames@langs@wals@zsu{Sukurum}
\def\langnames@langs@wals@zte{Elotepec Zapotec}
\def\langnames@langs@wals@ztg{Xanaguía Zapotec}
\def\langnames@langs@wals@ztl{Lapaguía-Guivini Zapotec}
\def\langnames@langs@wals@ztm{San Agustín Mixtepec Zapotec}
\def\langnames@langs@wals@ztn{Santa Catarina Albarradas Zapotec}
\def\langnames@langs@wals@ztp{Loxicha Zapotec}
\def\langnames@langs@wals@ztq{Quioquitani-Quieri Zapotec}
\def\langnames@langs@wals@zts{Tilquiapan Zapotec}
\def\langnames@langs@wals@ztt{Tejalapan Zapotec}
\def\langnames@langs@wals@ztu{Güilá Zapotec}
\def\langnames@langs@wals@ztx{Zaachila Zapotec}
\def\langnames@langs@wals@zty{Yatee Zapotec}
\def\langnames@langs@wals@zua{Zeem}
\def\langnames@langs@wals@zuh{Tokano}
\def\langnames@langs@wals@zul{Zulu}
\def\langnames@langs@wals@zum{Kumzari}
\def\langnames@langs@wals@zun{Zuni}
\def\langnames@langs@wals@zuy{Zumaya}
\def\langnames@langs@wals@zwa{Zay}
\def\langnames@langs@wals@zyb{Yongbei Zhuang}
\def\langnames@langs@wals@zyg{Yang Zhuang}
\def\langnames@langs@wals@zyj{Youjiang Zhuang}
\def\langnames@langs@wals@zyn{Yongnan Zhuang}
\def\langnames@langs@wals@zyp{Zyphe}
\def\langnames@langs@wals@zzj{Zuojiang Zhuang}
%
  \def\langnames@fams@wals@aaa{Atlantic-Congo}
\def\langnames@fams@wals@aab{Atlantic-Congo}
\def\langnames@fams@wals@aac{Suki-Gogodala}
\def\langnames@fams@wals@aad{Sepik}
\def\langnames@fams@wals@aae{Indo-European}
\def\langnames@fams@wals@aaf{Dravidian}
\def\langnames@fams@wals@aag{Nuclear Torricelli}
\def\langnames@fams@wals@aah{Nuclear Torricelli}
\def\langnames@fams@wals@aai{Austronesian}
\def\langnames@fams@wals@aak{Angan}
\def\langnames@fams@wals@aal{Afro-Asiatic}
\def\langnames@fams@wals@aan{Tupian}
\def\langnames@fams@wals@aao{Afro-Asiatic}
\def\langnames@fams@wals@aap{Cariban}
\def\langnames@fams@wals@aaq{Algic}
\def\langnames@fams@wals@aar{Afro-Asiatic}
\def\langnames@fams@wals@aas{Afro-Asiatic}
\def\langnames@fams@wals@aat{Indo-European}
\def\langnames@fams@wals@aau{Sepik}
\def\langnames@fams@wals@aaw{Austronesian}
\def\langnames@fams@wals@aax{Nuclear Trans New Guinea}
\def\langnames@fams@wals@aaz{Austronesian}
\def\langnames@fams@wals@aba{Atlantic-Congo}
\def\langnames@fams@wals@abb{Atlantic-Congo}
\def\langnames@fams@wals@abc{Austronesian}
\def\langnames@fams@wals@abd{Austronesian}
\def\langnames@fams@wals@abe{Algic}
\def\langnames@fams@wals@abf{Austronesian}
\def\langnames@fams@wals@abg{Nuclear Trans New Guinea}
\def\langnames@fams@wals@abh{Afro-Asiatic}
\def\langnames@fams@wals@abi{Atlantic-Congo}
\def\langnames@fams@wals@abj{Great Andamanese}
\def\langnames@fams@wals@abk{Abkhaz-Adyge}
\def\langnames@fams@wals@abl{Austronesian}
\def\langnames@fams@wals@abm{Atlantic-Congo}
\def\langnames@fams@wals@abn{Atlantic-Congo}
\def\langnames@fams@wals@abo{Atlantic-Congo}
\def\langnames@fams@wals@abp{Austronesian}
\def\langnames@fams@wals@abq{Abkhaz-Adyge}
\def\langnames@fams@wals@abr{Atlantic-Congo}
\def\langnames@fams@wals@abs{Austronesian}
\def\langnames@fams@wals@abt{Ndu}
\def\langnames@fams@wals@abu{Atlantic-Congo}
\def\langnames@fams@wals@abv{Afro-Asiatic}
\def\langnames@fams@wals@abw{Nuclear Trans New Guinea}
\def\langnames@fams@wals@abx{Austronesian}
\def\langnames@fams@wals@aby{Yareban}
\def\langnames@fams@wals@abz{Timor-Alor-Pantar}
\def\langnames@fams@wals@aca{Arawakan}
\def\langnames@fams@wals@acd{Atlantic-Congo}
\def\langnames@fams@wals@ace{Austronesian}
\def\langnames@fams@wals@acf{Indo-European}
\def\langnames@fams@wals@ach{Nilotic}
\def\langnames@fams@wals@aci{Great Andamanese}
\def\langnames@fams@wals@ack{Great Andamanese}
\def\langnames@fams@wals@acl{Great Andamanese}
\def\langnames@fams@wals@acm{Afro-Asiatic}
\def\langnames@fams@wals@acn{Sino-Tibetan}
\def\langnames@fams@wals@acp{Atlantic-Congo}
\def\langnames@fams@wals@acq{Afro-Asiatic}
\def\langnames@fams@wals@acr{Mayan}
\def\langnames@fams@wals@acs{Nuclear-Macro-Je}
\def\langnames@fams@wals@acu{Chicham}
\def\langnames@fams@wals@acv{Palaihnihan}
\def\langnames@fams@wals@acw{Afro-Asiatic}
\def\langnames@fams@wals@acx{Afro-Asiatic}
\def\langnames@fams@wals@acy{Afro-Asiatic}
\def\langnames@fams@wals@acz{Narrow Talodi}
\def\langnames@fams@wals@ada{Atlantic-Congo}
\def\langnames@fams@wals@add{Atlantic-Congo}
\def\langnames@fams@wals@ade{Atlantic-Congo}
\def\langnames@fams@wals@adf{Afro-Asiatic}
\def\langnames@fams@wals@adg{Pama-Nyungan}
\def\langnames@fams@wals@adh{Nilotic}
\def\langnames@fams@wals@adi{Sino-Tibetan}
\def\langnames@fams@wals@adj{Atlantic-Congo}
\def\langnames@fams@wals@adl{Sino-Tibetan}
\def\langnames@fams@wals@adn{Timor-Alor-Pantar}
\def\langnames@fams@wals@ado{Lower Sepik-Ramu}
\def\langnames@fams@wals@adq{Atlantic-Congo}
\def\langnames@fams@wals@adr{Austronesian}
\def\langnames@fams@wals@ads{Sign Language}
\def\langnames@fams@wals@adt{Pama-Nyungan}
\def\langnames@fams@wals@adw{Tupian}
\def\langnames@fams@wals@adx{Sino-Tibetan}
\def\langnames@fams@wals@ady{Abkhaz-Adyge}
\def\langnames@fams@wals@adz{Austronesian}
\def\langnames@fams@wals@aea{Pama-Nyungan}
\def\langnames@fams@wals@aeb{Afro-Asiatic}
\def\langnames@fams@wals@aec{Afro-Asiatic}
\def\langnames@fams@wals@aed{Sign Language}
\def\langnames@fams@wals@aee{Indo-European}
\def\langnames@fams@wals@aek{Austronesian}
\def\langnames@fams@wals@ael{Atlantic-Congo}
\def\langnames@fams@wals@aem{Austroasiatic}
\def\langnames@fams@wals@aen{Sign Language}
\def\langnames@fams@wals@aeq{Indo-European}
\def\langnames@fams@wals@aer{Pama-Nyungan}
\def\langnames@fams@wals@aes{Isolate}
\def\langnames@fams@wals@aeu{Sino-Tibetan}
\def\langnames@fams@wals@aew{Keram}
\def\langnames@fams@wals@aey{Nuclear Trans New Guinea}
\def\langnames@fams@wals@aez{Nuclear Trans New Guinea}
\def\langnames@fams@wals@afb{Afro-Asiatic}
\def\langnames@fams@wals@afd{Arafundi}
\def\langnames@fams@wals@afe{Atlantic-Congo}
\def\langnames@fams@wals@afg{Sign Language}
\def\langnames@fams@wals@afh{Artificial Language}
\def\langnames@fams@wals@afi{Lower Sepik-Ramu}
\def\langnames@fams@wals@afk{Arafundi}
\def\langnames@fams@wals@afn{Ijoid}
\def\langnames@fams@wals@afo{Atlantic-Congo}
\def\langnames@fams@wals@afp{Arafundi}
\def\langnames@fams@wals@afr{Indo-European}
\def\langnames@fams@wals@afs{Indo-European}
\def\langnames@fams@wals@aft{Nyimang}
\def\langnames@fams@wals@afu{Atlantic-Congo}
\def\langnames@fams@wals@afz{Lakes Plain}
\def\langnames@fams@wals@aga{Unattested}
\def\langnames@fams@wals@agb{Atlantic-Congo}
\def\langnames@fams@wals@agc{Atlantic-Congo}
\def\langnames@fams@wals@agd{Nuclear Trans New Guinea}
\def\langnames@fams@wals@age{Nuclear Trans New Guinea}
\def\langnames@fams@wals@agf{Austronesian}
\def\langnames@fams@wals@agg{Senagi}
\def\langnames@fams@wals@agh{Atlantic-Congo}
\def\langnames@fams@wals@agi{Unattested}
\def\langnames@fams@wals@agj{Afro-Asiatic}
\def\langnames@fams@wals@agk{Austronesian}
\def\langnames@fams@wals@agl{East Strickland}
\def\langnames@fams@wals@agm{Angan}
\def\langnames@fams@wals@agn{Austronesian}
\def\langnames@fams@wals@ago{Angan}
\def\langnames@fams@wals@agq{Atlantic-Congo}
\def\langnames@fams@wals@agr{Chicham}
\def\langnames@fams@wals@ags{Atlantic-Congo}
\def\langnames@fams@wals@agt{Austronesian}
\def\langnames@fams@wals@agu{Mayan}
\def\langnames@fams@wals@agv{Austronesian}
\def\langnames@fams@wals@agw{Austronesian}
\def\langnames@fams@wals@agx{Nakh-Daghestanian}
\def\langnames@fams@wals@agy{Austronesian}
\def\langnames@fams@wals@agz{Austronesian}
\def\langnames@fams@wals@aha{Atlantic-Congo}
\def\langnames@fams@wals@ahb{Austronesian}
\def\langnames@fams@wals@ahg{Afro-Asiatic}
\def\langnames@fams@wals@ahh{Nuclear Trans New Guinea}
\def\langnames@fams@wals@ahi{Kru}
\def\langnames@fams@wals@ahk{Sino-Tibetan}
\def\langnames@fams@wals@ahl{Atlantic-Congo}
\def\langnames@fams@wals@ahm{Kru}
\def\langnames@fams@wals@ahn{Atlantic-Congo}
\def\langnames@fams@wals@aho{Tai-Kadai}
\def\langnames@fams@wals@ahp{Atlantic-Congo}
\def\langnames@fams@wals@ahs{Atlantic-Congo}
\def\langnames@fams@wals@aht{Athabaskan-Eyak-Tlingit}
\def\langnames@fams@wals@aia{Austronesian}
\def\langnames@fams@wals@aib{Turkic}
\def\langnames@fams@wals@aic{Border}
\def\langnames@fams@wals@aid{Pama-Nyungan}
\def\langnames@fams@wals@aie{Austronesian}
\def\langnames@fams@wals@aif{Nuclear Torricelli}
\def\langnames@fams@wals@aig{Indo-European}
\def\langnames@fams@wals@aih{Tai-Kadai}
\def\langnames@fams@wals@aii{Afro-Asiatic}
\def\langnames@fams@wals@aij{Afro-Asiatic}
\def\langnames@fams@wals@aik{Atlantic-Congo}
\def\langnames@fams@wals@ail{Bosavi}
\def\langnames@fams@wals@aim{Sino-Tibetan}
\def\langnames@fams@wals@ain{Ainu}
\def\langnames@fams@wals@aio{Tai-Kadai}
\def\langnames@fams@wals@aip{Nuclear Trans New Guinea}
\def\langnames@fams@wals@aiq{Indo-European}
\def\langnames@fams@wals@air{Greater Kwerba}
\def\langnames@fams@wals@ait{Tupian}
\def\langnames@fams@wals@aiw{South Omotic}
\def\langnames@fams@wals@aix{Austronesian}
\def\langnames@fams@wals@aiy{Atlantic-Congo}
\def\langnames@fams@wals@aja{Kresh-Aja}
\def\langnames@fams@wals@ajg{Atlantic-Congo}
\def\langnames@fams@wals@aji{Austronesian}
\def\langnames@fams@wals@ajp{Afro-Asiatic}
\def\langnames@fams@wals@ajs{Sign Language}
\def\langnames@fams@wals@aju{Afro-Asiatic}
\def\langnames@fams@wals@ajw{Afro-Asiatic}
\def\langnames@fams@wals@ajz{Sino-Tibetan}
\def\langnames@fams@wals@aka{Atlantic-Congo}
\def\langnames@fams@wals@akb{Austronesian}
\def\langnames@fams@wals@akc{Isolate}
\def\langnames@fams@wals@akd{Atlantic-Congo}
\def\langnames@fams@wals@ake{Cariban}
\def\langnames@fams@wals@akf{Atlantic-Congo}
\def\langnames@fams@wals@akg{Austronesian}
\def\langnames@fams@wals@akh{Nuclear Trans New Guinea}
\def\langnames@fams@wals@aki{Lower Sepik-Ramu}
\def\langnames@fams@wals@akj{Great Andamanese}
\def\langnames@fams@wals@akk{Afro-Asiatic}
\def\langnames@fams@wals@akl{Austronesian}
\def\langnames@fams@wals@akm{Great Andamanese}
\def\langnames@fams@wals@ako{Cariban}
\def\langnames@fams@wals@akp{Atlantic-Congo}
\def\langnames@fams@wals@akq{Sepik}
\def\langnames@fams@wals@akr{Austronesian}
\def\langnames@fams@wals@aks{Atlantic-Congo}
\def\langnames@fams@wals@akt{Austronesian}
\def\langnames@fams@wals@aku{Atlantic-Congo}
\def\langnames@fams@wals@akv{Nakh-Daghestanian}
\def\langnames@fams@wals@akw{Atlantic-Congo}
\def\langnames@fams@wals@akx{Great Andamanese}
\def\langnames@fams@wals@aky{Great Andamanese}
\def\langnames@fams@wals@akz{Muskogean}
\def\langnames@fams@wals@ala{Atlantic-Congo}
\def\langnames@fams@wals@alc{Kawesqar}
\def\langnames@fams@wals@ald{Atlantic-Congo}
\def\langnames@fams@wals@ale{Eskimo-Aleut}
\def\langnames@fams@wals@alf{Atlantic-Congo}
\def\langnames@fams@wals@alh{Mangarrayi-Maran}
\def\langnames@fams@wals@ali{Nuclear Trans New Guinea}
\def\langnames@fams@wals@alj{Austronesian}
\def\langnames@fams@wals@alk{Austroasiatic}
\def\langnames@fams@wals@all{Dravidian}
\def\langnames@fams@wals@alm{Austronesian}
\def\langnames@fams@wals@aln{Indo-European}
\def\langnames@fams@wals@alo{Austronesian}
\def\langnames@fams@wals@alp{Austronesian}
\def\langnames@fams@wals@alq{Algic}
\def\langnames@fams@wals@alr{Chukotko-Kamchatkan}
\def\langnames@fams@wals@als{Indo-European}
\def\langnames@fams@wals@alt{Turkic}
\def\langnames@fams@wals@alu{Austronesian}
\def\langnames@fams@wals@alw{Afro-Asiatic}
\def\langnames@fams@wals@alx{Nuclear Torricelli}
\def\langnames@fams@wals@aly{Pama-Nyungan}
\def\langnames@fams@wals@alz{Nilotic}
\def\langnames@fams@wals@ama{Tupian}
\def\langnames@fams@wals@amb{Atlantic-Congo}
\def\langnames@fams@wals@amc{Pano-Tacanan}
\def\langnames@fams@wals@ame{Arawakan}
\def\langnames@fams@wals@amf{South Omotic}
\def\langnames@fams@wals@amg{Iwaidjan Proper}
\def\langnames@fams@wals@amh{Afro-Asiatic}
\def\langnames@fams@wals@ami{Austronesian}
\def\langnames@fams@wals@amj{Furan}
\def\langnames@fams@wals@amk{Austronesian}
\def\langnames@fams@wals@aml{Austroasiatic}
\def\langnames@fams@wals@amm{Left May}
\def\langnames@fams@wals@amn{Border}
\def\langnames@fams@wals@amo{Atlantic-Congo}
\def\langnames@fams@wals@amp{Sepik}
\def\langnames@fams@wals@amq{Austronesian}
\def\langnames@fams@wals@amr{Harakmbut}
\def\langnames@fams@wals@ams{Japonic}
\def\langnames@fams@wals@amt{Amto-Musan}
\def\langnames@fams@wals@amu{Otomanguean}
\def\langnames@fams@wals@amv{Austronesian}
\def\langnames@fams@wals@amw{Afro-Asiatic}
\def\langnames@fams@wals@amx{Pama-Nyungan}
\def\langnames@fams@wals@amy{Western Daly}
\def\langnames@fams@wals@amz{Pama-Nyungan}
\def\langnames@fams@wals@ana{Isolate}
\def\langnames@fams@wals@anb{Zaparoan}
\def\langnames@fams@wals@anc{Afro-Asiatic}
\def\langnames@fams@wals@and{Austronesian}
\def\langnames@fams@wals@ane{Austronesian}
\def\langnames@fams@wals@anf{Atlantic-Congo}
\def\langnames@fams@wals@ang{Indo-European}
\def\langnames@fams@wals@anh{Nuclear Trans New Guinea}
\def\langnames@fams@wals@ani{Nakh-Daghestanian}
\def\langnames@fams@wals@anj{Lower Sepik-Ramu}
\def\langnames@fams@wals@ank{Afro-Asiatic}
\def\langnames@fams@wals@anl{Sino-Tibetan}
\def\langnames@fams@wals@anm{Sino-Tibetan}
\def\langnames@fams@wals@ann{Atlantic-Congo}
\def\langnames@fams@wals@ano{Isolate}
\def\langnames@fams@wals@anp{Indo-European}
\def\langnames@fams@wals@anq{Jarawa-Onge}
\def\langnames@fams@wals@ans{Chocoan}
\def\langnames@fams@wals@ant{Pama-Nyungan}
\def\langnames@fams@wals@anu{Nilotic}
\def\langnames@fams@wals@anv{Atlantic-Congo}
\def\langnames@fams@wals@anw{Atlantic-Congo}
\def\langnames@fams@wals@anx{Austronesian}
\def\langnames@fams@wals@any{Atlantic-Congo}
\def\langnames@fams@wals@anz{Isolate}
\def\langnames@fams@wals@aoa{Indo-European}
\def\langnames@fams@wals@aob{Anim}
\def\langnames@fams@wals@aoc{Cariban}
\def\langnames@fams@wals@aod{Lower Sepik-Ramu}
\def\langnames@fams@wals@aoe{Nuclear Trans New Guinea}
\def\langnames@fams@wals@aof{Nuclear Torricelli}
\def\langnames@fams@wals@aog{Lower Sepik-Ramu}
\def\langnames@fams@wals@aoh{Unattested}
\def\langnames@fams@wals@aoi{Gunwinyguan}
\def\langnames@fams@wals@aoj{Nuclear Torricelli}
\def\langnames@fams@wals@aok{Austronesian}
\def\langnames@fams@wals@aol{Austronesian}
\def\langnames@fams@wals@aom{Koiarian}
\def\langnames@fams@wals@aon{Nuclear Torricelli}
\def\langnames@fams@wals@aor{Austronesian}
\def\langnames@fams@wals@aos{Border}
\def\langnames@fams@wals@aot{Sino-Tibetan}
\def\langnames@fams@wals@aou{Tai-Kadai}
\def\langnames@fams@wals@aox{Arawakan}
\def\langnames@fams@wals@aoz{Austronesian}
\def\langnames@fams@wals@apb{Austronesian}
\def\langnames@fams@wals@apc{Afro-Asiatic}
\def\langnames@fams@wals@apd{Afro-Asiatic}
\def\langnames@fams@wals@ape{Nuclear Torricelli}
\def\langnames@fams@wals@apf{Austronesian}
\def\langnames@fams@wals@apg{Austronesian}
\def\langnames@fams@wals@aph{Sino-Tibetan}
\def\langnames@fams@wals@api{Tupian}
\def\langnames@fams@wals@apj{Athabaskan-Eyak-Tlingit}
\def\langnames@fams@wals@apk{Athabaskan-Eyak-Tlingit}
\def\langnames@fams@wals@apl{Athabaskan-Eyak-Tlingit}
\def\langnames@fams@wals@apm{Athabaskan-Eyak-Tlingit}
\def\langnames@fams@wals@apn{Nuclear-Macro-Je}
\def\langnames@fams@wals@apo{Austronesian}
\def\langnames@fams@wals@app{Austronesian}
\def\langnames@fams@wals@apq{Great Andamanese}
\def\langnames@fams@wals@apr{Austronesian}
\def\langnames@fams@wals@aps{Austronesian}
\def\langnames@fams@wals@apt{Sino-Tibetan}
\def\langnames@fams@wals@apu{Arawakan}
\def\langnames@fams@wals@apv{Unattested}
\def\langnames@fams@wals@apw{Athabaskan-Eyak-Tlingit}
\def\langnames@fams@wals@apx{Austronesian}
\def\langnames@fams@wals@apy{Cariban}
\def\langnames@fams@wals@apz{Angan}
\def\langnames@fams@wals@aqc{Nakh-Daghestanian}
\def\langnames@fams@wals@aqd{Dogon}
\def\langnames@fams@wals@aqg{Atlantic-Congo}
\def\langnames@fams@wals@aqk{Atlantic-Congo}
\def\langnames@fams@wals@aqm{Kayagaric}
\def\langnames@fams@wals@aqn{Austronesian}
\def\langnames@fams@wals@aqp{Isolate}
\def\langnames@fams@wals@aqr{Austronesian}
\def\langnames@fams@wals@aqt{Lengua-Mascoy}
\def\langnames@fams@wals@aqz{Tupian}
\def\langnames@fams@wals@arb{Afro-Asiatic}
\def\langnames@fams@wals@arc{Afro-Asiatic}
\def\langnames@fams@wals@ard{Pama-Nyungan}
\def\langnames@fams@wals@are{Pama-Nyungan}
\def\langnames@fams@wals@arg{Indo-European}
\def\langnames@fams@wals@arh{Chibchan}
\def\langnames@fams@wals@ari{Caddoan}
\def\langnames@fams@wals@arj{Tucanoan}
\def\langnames@fams@wals@ark{Nuclear-Macro-Je}
\def\langnames@fams@wals@arl{Zaparoan}
\def\langnames@fams@wals@arn{Araucanian}
\def\langnames@fams@wals@aro{Pano-Tacanan}
\def\langnames@fams@wals@arp{Algic}
\def\langnames@fams@wals@arq{Afro-Asiatic}
\def\langnames@fams@wals@arr{Tupian}
\def\langnames@fams@wals@ars{Afro-Asiatic}
\def\langnames@fams@wals@aru{Arawan}
\def\langnames@fams@wals@arv{Afro-Asiatic}
\def\langnames@fams@wals@arw{Arawakan}
\def\langnames@fams@wals@arx{Tupian}
\def\langnames@fams@wals@ary{Afro-Asiatic}
\def\langnames@fams@wals@arz{Afro-Asiatic}
\def\langnames@fams@wals@asa{Atlantic-Congo}
\def\langnames@fams@wals@asb{Siouan}
\def\langnames@fams@wals@asc{Nuclear Trans New Guinea}
\def\langnames@fams@wals@ase{Sign Language}
\def\langnames@fams@wals@asf{Sign Language}
\def\langnames@fams@wals@asg{Atlantic-Congo}
\def\langnames@fams@wals@ash{Isolate}
\def\langnames@fams@wals@asi{Nuclear Trans New Guinea}
\def\langnames@fams@wals@asj{Atlantic-Congo}
\def\langnames@fams@wals@ask{Indo-European}
\def\langnames@fams@wals@asl{Austronesian}
\def\langnames@fams@wals@asm{Indo-European}
\def\langnames@fams@wals@asn{Tupian}
\def\langnames@fams@wals@aso{Nuclear Trans New Guinea}
\def\langnames@fams@wals@asp{Sign Language}
\def\langnames@fams@wals@asq{Sign Language}
\def\langnames@fams@wals@asr{Austroasiatic}
\def\langnames@fams@wals@ass{Atlantic-Congo}
\def\langnames@fams@wals@ast{Indo-European}
\def\langnames@fams@wals@asu{Tupian}
\def\langnames@fams@wals@asv{Central Sudanic}
\def\langnames@fams@wals@asw{Sign Language}
\def\langnames@fams@wals@asx{Nuclear Trans New Guinea}
\def\langnames@fams@wals@asy{Nuclear Trans New Guinea}
\def\langnames@fams@wals@asz{Austronesian}
\def\langnames@fams@wals@ata{Isolate}
\def\langnames@fams@wals@atb{Sino-Tibetan}
\def\langnames@fams@wals@atc{Pano-Tacanan}
\def\langnames@fams@wals@atd{Austronesian}
\def\langnames@fams@wals@ate{Nuclear Trans New Guinea}
\def\langnames@fams@wals@atg{Atlantic-Congo}
\def\langnames@fams@wals@ati{Atlantic-Congo}
\def\langnames@fams@wals@atj{Algic}
\def\langnames@fams@wals@atk{Austronesian}
\def\langnames@fams@wals@atl{Austronesian}
\def\langnames@fams@wals@atm{Austronesian}
\def\langnames@fams@wals@atn{Indo-European}
\def\langnames@fams@wals@ato{Atlantic-Congo}
\def\langnames@fams@wals@atp{Austronesian}
\def\langnames@fams@wals@atq{Austronesian}
\def\langnames@fams@wals@atr{Cariban}
\def\langnames@fams@wals@ats{Algic}
\def\langnames@fams@wals@att{Austronesian}
\def\langnames@fams@wals@atu{Nilotic}
\def\langnames@fams@wals@atv{Turkic}
\def\langnames@fams@wals@atw{Palaihnihan}
\def\langnames@fams@wals@atx{Isolate}
\def\langnames@fams@wals@aty{Austronesian}
\def\langnames@fams@wals@atz{Austronesian}
\def\langnames@fams@wals@aua{Austronesian}
\def\langnames@fams@wals@aub{Sino-Tibetan}
\def\langnames@fams@wals@auc{Isolate}
\def\langnames@fams@wals@aud{Austronesian}
\def\langnames@fams@wals@aug{Atlantic-Congo}
\def\langnames@fams@wals@auh{Atlantic-Congo}
\def\langnames@fams@wals@aui{Austronesian}
\def\langnames@fams@wals@auj{Afro-Asiatic}
\def\langnames@fams@wals@auk{Nuclear Torricelli}
\def\langnames@fams@wals@aul{Austronesian}
\def\langnames@fams@wals@aum{Atlantic-Congo}
\def\langnames@fams@wals@aun{Nuclear Torricelli}
\def\langnames@fams@wals@auo{Afro-Asiatic}
\def\langnames@fams@wals@aup{Anim}
\def\langnames@fams@wals@auq{Austronesian}
\def\langnames@fams@wals@aur{Nuclear Torricelli}
\def\langnames@fams@wals@aut{Austronesian}
\def\langnames@fams@wals@auu{Nuclear Trans New Guinea}
\def\langnames@fams@wals@auw{Border}
\def\langnames@fams@wals@aux{Tupian}
\def\langnames@fams@wals@auy{Nuclear Trans New Guinea}
\def\langnames@fams@wals@auz{Afro-Asiatic}
\def\langnames@fams@wals@ava{Nakh-Daghestanian}
\def\langnames@fams@wals@avb{Austronesian}
\def\langnames@fams@wals@avd{Indo-European}
\def\langnames@fams@wals@ave{Indo-European}
\def\langnames@fams@wals@avi{Atlantic-Congo}
\def\langnames@fams@wals@avk{Artificial Language}
\def\langnames@fams@wals@avl{Afro-Asiatic}
\def\langnames@fams@wals@avm{Pama-Nyungan}
\def\langnames@fams@wals@avn{Atlantic-Congo}
\def\langnames@fams@wals@avo{Unattested}
\def\langnames@fams@wals@avs{Zaparoan}
\def\langnames@fams@wals@avt{Nuclear Torricelli}
\def\langnames@fams@wals@avu{Central Sudanic}
\def\langnames@fams@wals@avv{Tupian}
\def\langnames@fams@wals@awa{Indo-European}
\def\langnames@fams@wals@awb{Nuclear Trans New Guinea}
\def\langnames@fams@wals@awc{Atlantic-Congo}
\def\langnames@fams@wals@awe{Tupian}
\def\langnames@fams@wals@awg{Pama-Nyungan}
\def\langnames@fams@wals@awh{Bayono-Awbono}
\def\langnames@fams@wals@awi{Kamula-Elevala}
\def\langnames@fams@wals@awk{Pama-Nyungan}
\def\langnames@fams@wals@awm{Nuclear Trans New Guinea}
\def\langnames@fams@wals@awn{Afro-Asiatic}
\def\langnames@fams@wals@awo{Atlantic-Congo}
\def\langnames@fams@wals@awr{Lakes Plain}
\def\langnames@fams@wals@aws{Nuclear Trans New Guinea}
\def\langnames@fams@wals@awt{Tupian}
\def\langnames@fams@wals@awu{Nuclear Trans New Guinea}
\def\langnames@fams@wals@awv{Nuclear Trans New Guinea}
\def\langnames@fams@wals@aww{Sepik}
\def\langnames@fams@wals@awx{Nuclear Trans New Guinea}
\def\langnames@fams@wals@awy{Nuclear Trans New Guinea}
\def\langnames@fams@wals@axb{Guaicuruan}
\def\langnames@fams@wals@axg{Isolate}
\def\langnames@fams@wals@axk{Atlantic-Congo}
\def\langnames@fams@wals@axl{Pama-Nyungan}
\def\langnames@fams@wals@axx{Austronesian}
\def\langnames@fams@wals@aya{Lower Sepik-Ramu}
\def\langnames@fams@wals@ayb{Atlantic-Congo}
\def\langnames@fams@wals@ayc{Aymaran}
\def\langnames@fams@wals@ayd{Pama-Nyungan}
\def\langnames@fams@wals@aye{Atlantic-Congo}
\def\langnames@fams@wals@ayg{Atlantic-Congo}
\def\langnames@fams@wals@ayh{Afro-Asiatic}
\def\langnames@fams@wals@ayi{Atlantic-Congo}
\def\langnames@fams@wals@ayk{Atlantic-Congo}
\def\langnames@fams@wals@ayl{Afro-Asiatic}
\def\langnames@fams@wals@ayn{Afro-Asiatic}
\def\langnames@fams@wals@ayo{Zamucoan}
\def\langnames@fams@wals@ayp{Afro-Asiatic}
\def\langnames@fams@wals@ayq{Sepik}
\def\langnames@fams@wals@ayr{Aymaran}
\def\langnames@fams@wals@ays{Unattested}
\def\langnames@fams@wals@ayt{Austronesian}
\def\langnames@fams@wals@ayu{Atlantic-Congo}
\def\langnames@fams@wals@ayy{Unattested}
\def\langnames@fams@wals@ayz{Isolate}
\def\langnames@fams@wals@aza{Sino-Tibetan}
\def\langnames@fams@wals@azb{Turkic}
\def\langnames@fams@wals@azd{Uto-Aztecan}
\def\langnames@fams@wals@azg{Otomanguean}
\def\langnames@fams@wals@azj{Turkic}
\def\langnames@fams@wals@azm{Otomanguean}
\def\langnames@fams@wals@azn{Uto-Aztecan}
\def\langnames@fams@wals@azo{Atlantic-Congo}
\def\langnames@fams@wals@azt{Austronesian}
\def\langnames@fams@wals@azz{Uto-Aztecan}
\def\langnames@fams@wals@baa{Austronesian}
\def\langnames@fams@wals@bab{Atlantic-Congo}
\def\langnames@fams@wals@bac{Austronesian}
\def\langnames@fams@wals@bae{Arawakan}
\def\langnames@fams@wals@baf{Atlantic-Congo}
\def\langnames@fams@wals@bag{Atlantic-Congo}
\def\langnames@fams@wals@bah{Indo-European}
\def\langnames@fams@wals@baj{Austronesian}
\def\langnames@fams@wals@bak{Turkic}
\def\langnames@fams@wals@bam{Mande}
\def\langnames@fams@wals@ban{Austronesian}
\def\langnames@fams@wals@bao{Tucanoan}
\def\langnames@fams@wals@bap{Sino-Tibetan}
\def\langnames@fams@wals@bar{Indo-European}
\def\langnames@fams@wals@bas{Atlantic-Congo}
\def\langnames@fams@wals@bau{Atlantic-Congo}
\def\langnames@fams@wals@bav{Atlantic-Congo}
\def\langnames@fams@wals@baw{Atlantic-Congo}
\def\langnames@fams@wals@bax{Atlantic-Congo}
\def\langnames@fams@wals@bay{Austronesian}
\def\langnames@fams@wals@bba{Atlantic-Congo}
\def\langnames@fams@wals@bbb{Koiarian}
\def\langnames@fams@wals@bbc{Austronesian}
\def\langnames@fams@wals@bbd{Nuclear Trans New Guinea}
\def\langnames@fams@wals@bbe{Atlantic-Congo}
\def\langnames@fams@wals@bbf{Baibai-Fas}
\def\langnames@fams@wals@bbg{Atlantic-Congo}
\def\langnames@fams@wals@bbh{Austroasiatic}
\def\langnames@fams@wals@bbi{Atlantic-Congo}
\def\langnames@fams@wals@bbj{Atlantic-Congo}
\def\langnames@fams@wals@bbk{Atlantic-Congo}
\def\langnames@fams@wals@bbl{Nakh-Daghestanian}
\def\langnames@fams@wals@bbm{Atlantic-Congo}
\def\langnames@fams@wals@bbn{Austronesian}
\def\langnames@fams@wals@bbo{Mande}
\def\langnames@fams@wals@bbp{Atlantic-Congo}
\def\langnames@fams@wals@bbq{Atlantic-Congo}
\def\langnames@fams@wals@bbr{Nuclear Trans New Guinea}
\def\langnames@fams@wals@bbs{Atlantic-Congo}
\def\langnames@fams@wals@bbt{Afro-Asiatic}
\def\langnames@fams@wals@bbu{Atlantic-Congo}
\def\langnames@fams@wals@bbv{Austronesian}
\def\langnames@fams@wals@bbw{Atlantic-Congo}
\def\langnames@fams@wals@bby{Atlantic-Congo}
\def\langnames@fams@wals@bca{Sino-Tibetan}
\def\langnames@fams@wals@bcc{Indo-European}
\def\langnames@fams@wals@bcd{Austronesian}
\def\langnames@fams@wals@bce{Atlantic-Congo}
\def\langnames@fams@wals@bcf{Kiwaian}
\def\langnames@fams@wals@bcg{Atlantic-Congo}
\def\langnames@fams@wals@bch{Austronesian}
\def\langnames@fams@wals@bci{Atlantic-Congo}
\def\langnames@fams@wals@bcj{Nyulnyulan}
\def\langnames@fams@wals@bck{Bunaban}
\def\langnames@fams@wals@bcl{Austronesian}
\def\langnames@fams@wals@bcm{Austronesian}
\def\langnames@fams@wals@bcn{Atlantic-Congo}
\def\langnames@fams@wals@bco{Bosavi}
\def\langnames@fams@wals@bcp{Atlantic-Congo}
\def\langnames@fams@wals@bcq{Ta-Ne-Omotic}
\def\langnames@fams@wals@bcr{Athabaskan-Eyak-Tlingit}
\def\langnames@fams@wals@bcs{Atlantic-Congo}
\def\langnames@fams@wals@bct{Central Sudanic}
\def\langnames@fams@wals@bcu{Austronesian}
\def\langnames@fams@wals@bcv{Atlantic-Congo}
\def\langnames@fams@wals@bcw{Afro-Asiatic}
\def\langnames@fams@wals@bcy{Afro-Asiatic}
\def\langnames@fams@wals@bcz{Atlantic-Congo}
\def\langnames@fams@wals@bda{Atlantic-Congo}
\def\langnames@fams@wals@bdb{Austronesian}
\def\langnames@fams@wals@bdc{Chocoan}
\def\langnames@fams@wals@bdd{Austronesian}
\def\langnames@fams@wals@bde{Afro-Asiatic}
\def\langnames@fams@wals@bdf{Koiarian}
\def\langnames@fams@wals@bdg{Austronesian}
\def\langnames@fams@wals@bdh{Central Sudanic}
\def\langnames@fams@wals@bdi{Nilotic}
\def\langnames@fams@wals@bdj{Atlantic-Congo}
\def\langnames@fams@wals@bdk{Nakh-Daghestanian}
\def\langnames@fams@wals@bdl{Austronesian}
\def\langnames@fams@wals@bdm{Afro-Asiatic}
\def\langnames@fams@wals@bdn{Afro-Asiatic}
\def\langnames@fams@wals@bdo{Central Sudanic}
\def\langnames@fams@wals@bdp{Atlantic-Congo}
\def\langnames@fams@wals@bdq{Austroasiatic}
\def\langnames@fams@wals@bdr{Austronesian}
\def\langnames@fams@wals@bds{Afro-Asiatic}
\def\langnames@fams@wals@bdt{Atlantic-Congo}
\def\langnames@fams@wals@bdu{Atlantic-Congo}
\def\langnames@fams@wals@bdv{Indo-European}
\def\langnames@fams@wals@bdw{West Bomberai}
\def\langnames@fams@wals@bdx{Austronesian}
\def\langnames@fams@wals@bdy{Pama-Nyungan}
\def\langnames@fams@wals@bdz{Unattested}
\def\langnames@fams@wals@bea{Athabaskan-Eyak-Tlingit}
\def\langnames@fams@wals@beb{Atlantic-Congo}
\def\langnames@fams@wals@bec{Atlantic-Congo}
\def\langnames@fams@wals@bed{Austronesian}
\def\langnames@fams@wals@bee{Sino-Tibetan}
\def\langnames@fams@wals@bef{Nuclear Trans New Guinea}
\def\langnames@fams@wals@beg{Austronesian}
\def\langnames@fams@wals@beh{Atlantic-Congo}
\def\langnames@fams@wals@bei{Austronesian}
\def\langnames@fams@wals@bej{Afro-Asiatic}
\def\langnames@fams@wals@bek{Austronesian}
\def\langnames@fams@wals@bel{Indo-European}
\def\langnames@fams@wals@bem{Atlantic-Congo}
\def\langnames@fams@wals@ben{Indo-European}
\def\langnames@fams@wals@beo{Bosavi}
\def\langnames@fams@wals@bep{Austronesian}
\def\langnames@fams@wals@beq{Atlantic-Congo}
\def\langnames@fams@wals@bes{Atlantic-Congo}
\def\langnames@fams@wals@bet{Kru}
\def\langnames@fams@wals@beu{Timor-Alor-Pantar}
\def\langnames@fams@wals@bev{Kru}
\def\langnames@fams@wals@bew{Austronesian}
\def\langnames@fams@wals@bex{Central Sudanic}
\def\langnames@fams@wals@bey{Nuclear Torricelli}
\def\langnames@fams@wals@bez{Atlantic-Congo}
\def\langnames@fams@wals@bfa{Nilotic}
\def\langnames@fams@wals@bfb{Indo-European}
\def\langnames@fams@wals@bfc{Sino-Tibetan}
\def\langnames@fams@wals@bfd{Atlantic-Congo}
\def\langnames@fams@wals@bfe{Tor-Orya}
\def\langnames@fams@wals@bff{Atlantic-Congo}
\def\langnames@fams@wals@bfg{Austronesian}
\def\langnames@fams@wals@bfh{Yam}
\def\langnames@fams@wals@bfi{Sign Language}
\def\langnames@fams@wals@bfj{Atlantic-Congo}
\def\langnames@fams@wals@bfk{Sign Language}
\def\langnames@fams@wals@bfl{Atlantic-Congo}
\def\langnames@fams@wals@bfm{Atlantic-Congo}
\def\langnames@fams@wals@bfn{Timor-Alor-Pantar}
\def\langnames@fams@wals@bfo{Atlantic-Congo}
\def\langnames@fams@wals@bfp{Atlantic-Congo}
\def\langnames@fams@wals@bfq{Dravidian}
\def\langnames@fams@wals@bfr{Unclassifiable}
\def\langnames@fams@wals@bfs{Sino-Tibetan}
\def\langnames@fams@wals@bft{Sino-Tibetan}
\def\langnames@fams@wals@bfu{Sino-Tibetan}
\def\langnames@fams@wals@bfw{Austroasiatic}
\def\langnames@fams@wals@bfx{Austronesian}
\def\langnames@fams@wals@bfy{Indo-European}
\def\langnames@fams@wals@bfz{Indo-European}
\def\langnames@fams@wals@bga{Atlantic-Congo}
\def\langnames@fams@wals@bgb{Austronesian}
\def\langnames@fams@wals@bgc{Indo-European}
\def\langnames@fams@wals@bgd{Indo-European}
\def\langnames@fams@wals@bge{Indo-European}
\def\langnames@fams@wals@bgf{Atlantic-Congo}
\def\langnames@fams@wals@bgg{Sino-Tibetan}
\def\langnames@fams@wals@bgi{Austronesian}
\def\langnames@fams@wals@bgj{Atlantic-Congo}
\def\langnames@fams@wals@bgk{Austroasiatic}
\def\langnames@fams@wals@bgn{Indo-European}
\def\langnames@fams@wals@bgo{Atlantic-Congo}
\def\langnames@fams@wals@bgp{Indo-European}
\def\langnames@fams@wals@bgq{Indo-European}
\def\langnames@fams@wals@bgr{Sino-Tibetan}
\def\langnames@fams@wals@bgs{Austronesian}
\def\langnames@fams@wals@bgt{Austronesian}
\def\langnames@fams@wals@bgu{Atlantic-Congo}
\def\langnames@fams@wals@bgv{Anim}
\def\langnames@fams@wals@bgw{Indo-European}
\def\langnames@fams@wals@bgx{Turkic}
\def\langnames@fams@wals@bgy{Austronesian}
\def\langnames@fams@wals@bgz{Austronesian}
\def\langnames@fams@wals@bha{Indo-European}
\def\langnames@fams@wals@bhb{Indo-European}
\def\langnames@fams@wals@bhc{Austronesian}
\def\langnames@fams@wals@bhd{Indo-European}
\def\langnames@fams@wals@bhe{Indo-European}
\def\langnames@fams@wals@bhf{Isolate}
\def\langnames@fams@wals@bhg{Nuclear Trans New Guinea}
\def\langnames@fams@wals@bhh{Indo-European}
\def\langnames@fams@wals@bhi{Indo-European}
\def\langnames@fams@wals@bhj{Sino-Tibetan}
\def\langnames@fams@wals@bhk{Austronesian}
\def\langnames@fams@wals@bhl{Nuclear Trans New Guinea}
\def\langnames@fams@wals@bhm{Afro-Asiatic}
\def\langnames@fams@wals@bhn{Afro-Asiatic}
\def\langnames@fams@wals@bho{Indo-European}
\def\langnames@fams@wals@bhp{Austronesian}
\def\langnames@fams@wals@bhq{Austronesian}
\def\langnames@fams@wals@bhr{Austronesian}
\def\langnames@fams@wals@bhs{Afro-Asiatic}
\def\langnames@fams@wals@bht{Indo-European}
\def\langnames@fams@wals@bhu{Indo-European}
\def\langnames@fams@wals@bhv{Austronesian}
\def\langnames@fams@wals@bhw{Austronesian}
\def\langnames@fams@wals@bhy{Atlantic-Congo}
\def\langnames@fams@wals@bhz{Austronesian}
\def\langnames@fams@wals@bia{Pama-Nyungan}
\def\langnames@fams@wals@bib{Mande}
\def\langnames@fams@wals@bid{Afro-Asiatic}
\def\langnames@fams@wals@bie{Nuclear Trans New Guinea}
\def\langnames@fams@wals@bif{Atlantic-Congo}
\def\langnames@fams@wals@big{Goilalan}
\def\langnames@fams@wals@bil{Atlantic-Congo}
\def\langnames@fams@wals@bim{Atlantic-Congo}
\def\langnames@fams@wals@bin{Atlantic-Congo}
\def\langnames@fams@wals@bio{Kwomtari-Nai}
\def\langnames@fams@wals@bip{Atlantic-Congo}
\def\langnames@fams@wals@biq{Austronesian}
\def\langnames@fams@wals@bir{Nuclear Trans New Guinea}
\def\langnames@fams@wals@bis{Indo-European}
\def\langnames@fams@wals@bit{Sepik}
\def\langnames@fams@wals@biu{Sino-Tibetan}
\def\langnames@fams@wals@biv{Atlantic-Congo}
\def\langnames@fams@wals@biw{Atlantic-Congo}
\def\langnames@fams@wals@bix{Austroasiatic}
\def\langnames@fams@wals@biy{Austroasiatic}
\def\langnames@fams@wals@biz{Atlantic-Congo}
\def\langnames@fams@wals@bja{Atlantic-Congo}
\def\langnames@fams@wals@bjb{Pama-Nyungan}
\def\langnames@fams@wals@bjc{Yareban}
\def\langnames@fams@wals@bje{Hmong-Mien}
\def\langnames@fams@wals@bjf{Afro-Asiatic}
\def\langnames@fams@wals@bjg{Atlantic-Congo}
\def\langnames@fams@wals@bjh{Sepik}
\def\langnames@fams@wals@bji{Afro-Asiatic}
\def\langnames@fams@wals@bjj{Indo-European}
\def\langnames@fams@wals@bjk{Austronesian}
\def\langnames@fams@wals@bjl{Austronesian}
\def\langnames@fams@wals@bjm{Indo-European}
\def\langnames@fams@wals@bjn{Austronesian}
\def\langnames@fams@wals@bjo{Atlantic-Congo}
\def\langnames@fams@wals@bjr{Nuclear Trans New Guinea}
\def\langnames@fams@wals@bjs{Indo-European}
\def\langnames@fams@wals@bjt{Atlantic-Congo}
\def\langnames@fams@wals@bju{Atlantic-Congo}
\def\langnames@fams@wals@bjv{Central Sudanic}
\def\langnames@fams@wals@bjw{Kru}
\def\langnames@fams@wals@bjx{Austronesian}
\def\langnames@fams@wals@bjy{Pama-Nyungan}
\def\langnames@fams@wals@bjz{Nuclear Trans New Guinea}
\def\langnames@fams@wals@bka{Atlantic-Congo}
\def\langnames@fams@wals@bkb{Austronesian}
\def\langnames@fams@wals@bkc{Atlantic-Congo}
\def\langnames@fams@wals@bkd{Austronesian}
\def\langnames@fams@wals@bkf{Atlantic-Congo}
\def\langnames@fams@wals@bkh{Atlantic-Congo}
\def\langnames@fams@wals@bki{Austronesian}
\def\langnames@fams@wals@bkj{Atlantic-Congo}
\def\langnames@fams@wals@bkk{Indo-European}
\def\langnames@fams@wals@bkl{Tor-Orya}
\def\langnames@fams@wals@bkm{Atlantic-Congo}
\def\langnames@fams@wals@bkn{Austronesian}
\def\langnames@fams@wals@bko{Atlantic-Congo}
\def\langnames@fams@wals@bkp{Atlantic-Congo}
\def\langnames@fams@wals@bkq{Cariban}
\def\langnames@fams@wals@bkr{Austronesian}
\def\langnames@fams@wals@bks{Austronesian}
\def\langnames@fams@wals@bkt{Atlantic-Congo}
\def\langnames@fams@wals@bku{Austronesian}
\def\langnames@fams@wals@bkv{Atlantic-Congo}
\def\langnames@fams@wals@bkw{Atlantic-Congo}
\def\langnames@fams@wals@bkx{Austronesian}
\def\langnames@fams@wals@bky{Atlantic-Congo}
\def\langnames@fams@wals@bkz{Austronesian}
\def\langnames@fams@wals@bla{Algic}
\def\langnames@fams@wals@blb{Isolate}
\def\langnames@fams@wals@blc{Salishan}
\def\langnames@fams@wals@bld{Austronesian}
\def\langnames@fams@wals@ble{Atlantic-Congo}
\def\langnames@fams@wals@blf{Austronesian}
\def\langnames@fams@wals@blh{Kru}
\def\langnames@fams@wals@bli{Atlantic-Congo}
\def\langnames@fams@wals@blj{Austronesian}
\def\langnames@fams@wals@blk{Sino-Tibetan}
\def\langnames@fams@wals@bll{Siouan}
\def\langnames@fams@wals@blm{Central Sudanic}
\def\langnames@fams@wals@bln{Austronesian}
\def\langnames@fams@wals@blo{Atlantic-Congo}
\def\langnames@fams@wals@blp{Austronesian}
\def\langnames@fams@wals@blq{Austronesian}
\def\langnames@fams@wals@blr{Austroasiatic}
\def\langnames@fams@wals@bls{Austronesian}
\def\langnames@fams@wals@blt{Tai-Kadai}
\def\langnames@fams@wals@blv{Atlantic-Congo}
\def\langnames@fams@wals@blw{Austronesian}
\def\langnames@fams@wals@blx{Austronesian}
\def\langnames@fams@wals@bly{Atlantic-Congo}
\def\langnames@fams@wals@blz{Austronesian}
\def\langnames@fams@wals@bma{Atlantic-Congo}
\def\langnames@fams@wals@bmb{Atlantic-Congo}
\def\langnames@fams@wals@bmc{Austronesian}
\def\langnames@fams@wals@bmd{Atlantic-Congo}
\def\langnames@fams@wals@bme{Atlantic-Congo}
\def\langnames@fams@wals@bmf{Atlantic-Congo}
\def\langnames@fams@wals@bmg{Atlantic-Congo}
\def\langnames@fams@wals@bmh{Nuclear Trans New Guinea}
\def\langnames@fams@wals@bmi{Central Sudanic}
\def\langnames@fams@wals@bmj{Indo-European}
\def\langnames@fams@wals@bmk{Austronesian}
\def\langnames@fams@wals@bml{Atlantic-Congo}
\def\langnames@fams@wals@bmm{Austronesian}
\def\langnames@fams@wals@bmn{Austronesian}
\def\langnames@fams@wals@bmo{Atlantic-Congo}
\def\langnames@fams@wals@bmp{Nuclear Trans New Guinea}
\def\langnames@fams@wals@bmq{Atlantic-Congo}
\def\langnames@fams@wals@bmr{Boran}
\def\langnames@fams@wals@bms{Saharan}
\def\langnames@fams@wals@bmt{Hmong-Mien}
\def\langnames@fams@wals@bmu{Nuclear Trans New Guinea}
\def\langnames@fams@wals@bmv{Atlantic-Congo}
\def\langnames@fams@wals@bmw{Atlantic-Congo}
\def\langnames@fams@wals@bmx{Nuclear Trans New Guinea}
\def\langnames@fams@wals@bmz{Anim}
\def\langnames@fams@wals@bna{Austronesian}
\def\langnames@fams@wals@bnb{Austronesian}
\def\langnames@fams@wals@bnd{Austronesian}
\def\langnames@fams@wals@bne{Austronesian}
\def\langnames@fams@wals@bnf{Austronesian}
\def\langnames@fams@wals@bng{Atlantic-Congo}
\def\langnames@fams@wals@bni{Atlantic-Congo}
\def\langnames@fams@wals@bnj{Austronesian}
\def\langnames@fams@wals@bnk{Austronesian}
\def\langnames@fams@wals@bnl{Afro-Asiatic}
\def\langnames@fams@wals@bnm{Atlantic-Congo}
\def\langnames@fams@wals@bnn{Austronesian}
\def\langnames@fams@wals@bno{Austronesian}
\def\langnames@fams@wals@bnp{Austronesian}
\def\langnames@fams@wals@bnq{Austronesian}
\def\langnames@fams@wals@bnr{Austronesian}
\def\langnames@fams@wals@bns{Indo-European}
\def\langnames@fams@wals@bnu{Austronesian}
\def\langnames@fams@wals@bnv{Tor-Orya}
\def\langnames@fams@wals@bnw{Sepik}
\def\langnames@fams@wals@bnx{Atlantic-Congo}
\def\langnames@fams@wals@bny{Austronesian}
\def\langnames@fams@wals@bnz{Atlantic-Congo}
\def\langnames@fams@wals@boa{Boran}
\def\langnames@fams@wals@bob{Afro-Asiatic}
\def\langnames@fams@wals@bod{Sino-Tibetan}
\def\langnames@fams@wals@boe{Atlantic-Congo}
\def\langnames@fams@wals@bof{Mande}
\def\langnames@fams@wals@bog{Sign Language}
\def\langnames@fams@wals@boh{Atlantic-Congo}
\def\langnames@fams@wals@boi{Chumashan}
\def\langnames@fams@wals@boj{Nuclear Trans New Guinea}
\def\langnames@fams@wals@bok{Atlantic-Congo}
\def\langnames@fams@wals@bol{Afro-Asiatic}
\def\langnames@fams@wals@bom{Atlantic-Congo}
\def\langnames@fams@wals@bon{Eastern Trans-Fly}
\def\langnames@fams@wals@boo{Mande}
\def\langnames@fams@wals@bop{Nuclear Trans New Guinea}
\def\langnames@fams@wals@boq{Isolate}
\def\langnames@fams@wals@bor{Bororoan}
\def\langnames@fams@wals@bot{Central Sudanic}
\def\langnames@fams@wals@bou{Atlantic-Congo}
\def\langnames@fams@wals@bov{Atlantic-Congo}
\def\langnames@fams@wals@bow{Yam}
\def\langnames@fams@wals@box{Atlantic-Congo}
\def\langnames@fams@wals@boy{Atlantic-Congo}
\def\langnames@fams@wals@boz{Mande}
\def\langnames@fams@wals@bpa{Austronesian}
\def\langnames@fams@wals@bpb{Unattested}
\def\langnames@fams@wals@bpc{Atlantic-Congo}
\def\langnames@fams@wals@bpd{Atlantic-Congo}
\def\langnames@fams@wals@bpe{Sko}
\def\langnames@fams@wals@bpg{Austronesian}
\def\langnames@fams@wals@bph{Nakh-Daghestanian}
\def\langnames@fams@wals@bpi{Nuclear Trans New Guinea}
\def\langnames@fams@wals@bpj{Atlantic-Congo}
\def\langnames@fams@wals@bpk{Austronesian}
\def\langnames@fams@wals@bpl{Pidgin}
\def\langnames@fams@wals@bpm{Nuclear Trans New Guinea}
\def\langnames@fams@wals@bpn{Hmong-Mien}
\def\langnames@fams@wals@bpp{Kaure-Kosare}
\def\langnames@fams@wals@bpq{Austronesian}
\def\langnames@fams@wals@bpr{Austronesian}
\def\langnames@fams@wals@bps{Austronesian}
\def\langnames@fams@wals@bpt{Pama-Nyungan}
\def\langnames@fams@wals@bpu{Nuclear Trans New Guinea}
\def\langnames@fams@wals@bpv{Anim}
\def\langnames@fams@wals@bpw{Left May}
\def\langnames@fams@wals@bpx{Indo-European}
\def\langnames@fams@wals@bpy{Indo-European}
\def\langnames@fams@wals@bpz{Austronesian}
\def\langnames@fams@wals@bqa{Atlantic-Congo}
\def\langnames@fams@wals@bqb{Greater Kwerba}
\def\langnames@fams@wals@bqc{Mande}
\def\langnames@fams@wals@bqd{Atlantic-Congo}
\def\langnames@fams@wals@bqg{Atlantic-Congo}
\def\langnames@fams@wals@bqh{Sino-Tibetan}
\def\langnames@fams@wals@bqi{Indo-European}
\def\langnames@fams@wals@bqj{Atlantic-Congo}
\def\langnames@fams@wals@bqk{Atlantic-Congo}
\def\langnames@fams@wals@bql{Nuclear Trans New Guinea}
\def\langnames@fams@wals@bqm{Atlantic-Congo}
\def\langnames@fams@wals@bqn{Sign Language}
\def\langnames@fams@wals@bqo{Atlantic-Congo}
\def\langnames@fams@wals@bqp{Mande}
\def\langnames@fams@wals@bqq{Lakes Plain}
\def\langnames@fams@wals@bqr{Austronesian}
\def\langnames@fams@wals@bqs{Lower Sepik-Ramu}
\def\langnames@fams@wals@bqt{Atlantic-Congo}
\def\langnames@fams@wals@bqu{Atlantic-Congo}
\def\langnames@fams@wals@bqv{Atlantic-Congo}
\def\langnames@fams@wals@bqw{Atlantic-Congo}
\def\langnames@fams@wals@bqx{Atlantic-Congo}
\def\langnames@fams@wals@bqy{Sign Language}
\def\langnames@fams@wals@bqz{Atlantic-Congo}
\def\langnames@fams@wals@bra{Indo-European}
\def\langnames@fams@wals@brb{Austroasiatic}
\def\langnames@fams@wals@brc{Indo-European}
\def\langnames@fams@wals@brd{Sino-Tibetan}
\def\langnames@fams@wals@bre{Indo-European}
\def\langnames@fams@wals@brf{Atlantic-Congo}
\def\langnames@fams@wals@brg{Arawakan}
\def\langnames@fams@wals@brh{Dravidian}
\def\langnames@fams@wals@bri{Atlantic-Congo}
\def\langnames@fams@wals@brj{Austronesian}
\def\langnames@fams@wals@brk{Nubian}
\def\langnames@fams@wals@brl{Atlantic-Congo}
\def\langnames@fams@wals@brm{Atlantic-Congo}
\def\langnames@fams@wals@brn{Chibchan}
\def\langnames@fams@wals@bro{Sino-Tibetan}
\def\langnames@fams@wals@brp{Geelvink Bay}
\def\langnames@fams@wals@brq{Lower Sepik-Ramu}
\def\langnames@fams@wals@brr{Austronesian}
\def\langnames@fams@wals@brs{Austronesian}
\def\langnames@fams@wals@brt{Atlantic-Congo}
\def\langnames@fams@wals@bru{Austroasiatic}
\def\langnames@fams@wals@brv{Austroasiatic}
\def\langnames@fams@wals@brw{Dravidian}
\def\langnames@fams@wals@brx{Sino-Tibetan}
\def\langnames@fams@wals@bry{Ndu}
\def\langnames@fams@wals@brz{Austronesian}
\def\langnames@fams@wals@bsa{Isolate}
\def\langnames@fams@wals@bsb{Austronesian}
\def\langnames@fams@wals@bsc{Atlantic-Congo}
\def\langnames@fams@wals@bse{Atlantic-Congo}
\def\langnames@fams@wals@bsf{Atlantic-Congo}
\def\langnames@fams@wals@bsg{Indo-European}
\def\langnames@fams@wals@bsh{Indo-European}
\def\langnames@fams@wals@bsi{Atlantic-Congo}
\def\langnames@fams@wals@bsj{Atlantic-Congo}
\def\langnames@fams@wals@bsk{Isolate}
\def\langnames@fams@wals@bsl{Atlantic-Congo}
\def\langnames@fams@wals@bsm{Austronesian}
\def\langnames@fams@wals@bsn{Tucanoan}
\def\langnames@fams@wals@bsp{Atlantic-Congo}
\def\langnames@fams@wals@bsq{Kru}
\def\langnames@fams@wals@bsr{Atlantic-Congo}
\def\langnames@fams@wals@bss{Atlantic-Congo}
\def\langnames@fams@wals@bst{Ta-Ne-Omotic}
\def\langnames@fams@wals@bsu{Austronesian}
\def\langnames@fams@wals@bsw{Afro-Asiatic}
\def\langnames@fams@wals@bsx{Atlantic-Congo}
\def\langnames@fams@wals@bsy{Austronesian}
\def\langnames@fams@wals@bta{Afro-Asiatic}
\def\langnames@fams@wals@btc{Atlantic-Congo}
\def\langnames@fams@wals@btd{Austronesian}
\def\langnames@fams@wals@bte{Atlantic-Congo}
\def\langnames@fams@wals@btf{Afro-Asiatic}
\def\langnames@fams@wals@btg{Kru}
\def\langnames@fams@wals@bth{Austronesian}
\def\langnames@fams@wals@bti{Geelvink Bay}
\def\langnames@fams@wals@btj{Austronesian}
\def\langnames@fams@wals@btm{Austronesian}
\def\langnames@fams@wals@btn{Austronesian}
\def\langnames@fams@wals@bto{Austronesian}
\def\langnames@fams@wals@btp{Austronesian}
\def\langnames@fams@wals@btq{Austroasiatic}
\def\langnames@fams@wals@btr{Austronesian}
\def\langnames@fams@wals@bts{Austronesian}
\def\langnames@fams@wals@btt{Atlantic-Congo}
\def\langnames@fams@wals@btu{Atlantic-Congo}
\def\langnames@fams@wals@btv{Indo-European}
\def\langnames@fams@wals@btw{Austronesian}
\def\langnames@fams@wals@btx{Austronesian}
\def\langnames@fams@wals@bty{Austronesian}
\def\langnames@fams@wals@btz{Austronesian}
\def\langnames@fams@wals@bub{Atlantic-Congo}
\def\langnames@fams@wals@buc{Austronesian}
\def\langnames@fams@wals@bud{Atlantic-Congo}
\def\langnames@fams@wals@bue{Isolate}
\def\langnames@fams@wals@buf{Atlantic-Congo}
\def\langnames@fams@wals@bug{Austronesian}
\def\langnames@fams@wals@buh{Hmong-Mien}
\def\langnames@fams@wals@bui{Atlantic-Congo}
\def\langnames@fams@wals@buj{Atlantic-Congo}
\def\langnames@fams@wals@buk{Austronesian}
\def\langnames@fams@wals@bul{Indo-European}
\def\langnames@fams@wals@bum{Atlantic-Congo}
\def\langnames@fams@wals@bun{Atlantic-Congo}
\def\langnames@fams@wals@buo{South Bougainville}
\def\langnames@fams@wals@bup{Austronesian}
\def\langnames@fams@wals@buq{Nuclear Trans New Guinea}
\def\langnames@fams@wals@bus{Mande}
\def\langnames@fams@wals@but{Nuclear Torricelli}
\def\langnames@fams@wals@buu{Atlantic-Congo}
\def\langnames@fams@wals@buv{Yuat}
\def\langnames@fams@wals@buw{Atlantic-Congo}
\def\langnames@fams@wals@bux{Afro-Asiatic}
\def\langnames@fams@wals@buy{Atlantic-Congo}
\def\langnames@fams@wals@buz{Atlantic-Congo}
\def\langnames@fams@wals@bva{Afro-Asiatic}
\def\langnames@fams@wals@bvb{Atlantic-Congo}
\def\langnames@fams@wals@bvc{Austronesian}
\def\langnames@fams@wals@bvd{Austronesian}
\def\langnames@fams@wals@bve{Austronesian}
\def\langnames@fams@wals@bvf{Afro-Asiatic}
\def\langnames@fams@wals@bvg{Atlantic-Congo}
\def\langnames@fams@wals@bvh{Afro-Asiatic}
\def\langnames@fams@wals@bvi{Atlantic-Congo}
\def\langnames@fams@wals@bvj{Atlantic-Congo}
\def\langnames@fams@wals@bvk{Austronesian}
\def\langnames@fams@wals@bvl{Sign Language}
\def\langnames@fams@wals@bvm{Atlantic-Congo}
\def\langnames@fams@wals@bvn{Nuclear Torricelli}
\def\langnames@fams@wals@bvo{Atlantic-Congo}
\def\langnames@fams@wals@bvq{Central Sudanic}
\def\langnames@fams@wals@bvr{Maningrida}
\def\langnames@fams@wals@bvt{Austronesian}
\def\langnames@fams@wals@bvu{Austronesian}
\def\langnames@fams@wals@bvw{Afro-Asiatic}
\def\langnames@fams@wals@bvx{Atlantic-Congo}
\def\langnames@fams@wals@bvy{Austronesian}
\def\langnames@fams@wals@bvz{Geelvink Bay}
\def\langnames@fams@wals@bwa{Austronesian}
\def\langnames@fams@wals@bwb{Austronesian}
\def\langnames@fams@wals@bwc{Atlantic-Congo}
\def\langnames@fams@wals@bwd{Austronesian}
\def\langnames@fams@wals@bwe{Sino-Tibetan}
\def\langnames@fams@wals@bwf{Austronesian}
\def\langnames@fams@wals@bwg{Atlantic-Congo}
\def\langnames@fams@wals@bwh{Atlantic-Congo}
\def\langnames@fams@wals@bwi{Arawakan}
\def\langnames@fams@wals@bwj{Atlantic-Congo}
\def\langnames@fams@wals@bwk{Mailuan}
\def\langnames@fams@wals@bwl{Atlantic-Congo}
\def\langnames@fams@wals@bwm{Yuat}
\def\langnames@fams@wals@bwn{Hmong-Mien}
\def\langnames@fams@wals@bwo{Ta-Ne-Omotic}
\def\langnames@fams@wals@bwp{Nuclear Trans New Guinea}
\def\langnames@fams@wals@bwq{Mande}
\def\langnames@fams@wals@bwr{Afro-Asiatic}
\def\langnames@fams@wals@bws{Atlantic-Congo}
\def\langnames@fams@wals@bwt{Atlantic-Congo}
\def\langnames@fams@wals@bwu{Atlantic-Congo}
\def\langnames@fams@wals@bww{Atlantic-Congo}
\def\langnames@fams@wals@bwx{Hmong-Mien}
\def\langnames@fams@wals@bwy{Atlantic-Congo}
\def\langnames@fams@wals@bwz{Atlantic-Congo}
\def\langnames@fams@wals@bxa{Austronesian}
\def\langnames@fams@wals@bxb{Nilotic}
\def\langnames@fams@wals@bxc{Atlantic-Congo}
\def\langnames@fams@wals@bxd{Sino-Tibetan}
\def\langnames@fams@wals@bxe{Isolate}
\def\langnames@fams@wals@bxf{Austronesian}
\def\langnames@fams@wals@bxg{Atlantic-Congo}
\def\langnames@fams@wals@bxh{Austronesian}
\def\langnames@fams@wals@bxi{Pama-Nyungan}
\def\langnames@fams@wals@bxj{Pama-Nyungan}
\def\langnames@fams@wals@bxk{Atlantic-Congo}
\def\langnames@fams@wals@bxl{Mande}
\def\langnames@fams@wals@bxm{Mongolic-Khitan}
\def\langnames@fams@wals@bxn{Pama-Nyungan}
\def\langnames@fams@wals@bxo{Pidgin}
\def\langnames@fams@wals@bxp{Atlantic-Congo}
\def\langnames@fams@wals@bxq{Afro-Asiatic}
\def\langnames@fams@wals@bxr{Mongolic-Khitan}
\def\langnames@fams@wals@bxs{Atlantic-Congo}
\def\langnames@fams@wals@bxu{Mongolic-Khitan}
\def\langnames@fams@wals@bxv{Central Sudanic}
\def\langnames@fams@wals@bxw{Mande}
\def\langnames@fams@wals@bxz{Mailuan}
\def\langnames@fams@wals@bya{Austronesian}
\def\langnames@fams@wals@byb{Atlantic-Congo}
\def\langnames@fams@wals@byc{Atlantic-Congo}
\def\langnames@fams@wals@byd{Austronesian}
\def\langnames@fams@wals@bye{Sepik}
\def\langnames@fams@wals@byf{Atlantic-Congo}
\def\langnames@fams@wals@byg{Dajuic}
\def\langnames@fams@wals@byh{Sino-Tibetan}
\def\langnames@fams@wals@byi{Atlantic-Congo}
\def\langnames@fams@wals@byj{Atlantic-Congo}
\def\langnames@fams@wals@byk{Tai-Kadai}
\def\langnames@fams@wals@byl{Bayono-Awbono}
\def\langnames@fams@wals@bym{Pama-Nyungan}
\def\langnames@fams@wals@byn{Afro-Asiatic}
\def\langnames@fams@wals@byo{Sino-Tibetan}
\def\langnames@fams@wals@byp{Atlantic-Congo}
\def\langnames@fams@wals@byq{Austronesian}
\def\langnames@fams@wals@byr{Angan}
\def\langnames@fams@wals@bys{Atlantic-Congo}
\def\langnames@fams@wals@byt{Saharan}
\def\langnames@fams@wals@byv{Atlantic-Congo}
\def\langnames@fams@wals@byw{Sino-Tibetan}
\def\langnames@fams@wals@byx{Baining}
\def\langnames@fams@wals@byz{Lower Sepik-Ramu}
\def\langnames@fams@wals@bza{Mande}
\def\langnames@fams@wals@bzb{Austronesian}
\def\langnames@fams@wals@bzc{Austronesian}
\def\langnames@fams@wals@bzd{Chibchan}
\def\langnames@fams@wals@bze{Mande}
\def\langnames@fams@wals@bzf{Ndu}
\def\langnames@fams@wals@bzg{Austronesian}
\def\langnames@fams@wals@bzh{Austronesian}
\def\langnames@fams@wals@bzi{Sino-Tibetan}
\def\langnames@fams@wals@bzj{Indo-European}
\def\langnames@fams@wals@bzk{Indo-European}
\def\langnames@fams@wals@bzl{Austronesian}
\def\langnames@fams@wals@bzm{Atlantic-Congo}
\def\langnames@fams@wals@bzn{Austronesian}
\def\langnames@fams@wals@bzp{South Bird's Head Family}
\def\langnames@fams@wals@bzq{Austronesian}
\def\langnames@fams@wals@bzr{Pama-Nyungan}
\def\langnames@fams@wals@bzs{Sign Language}
\def\langnames@fams@wals@bzt{Artificial Language}
\def\langnames@fams@wals@bzu{Isolate}
\def\langnames@fams@wals@bzv{Atlantic-Congo}
\def\langnames@fams@wals@bzw{Atlantic-Congo}
\def\langnames@fams@wals@bzx{Mande}
\def\langnames@fams@wals@bzy{Atlantic-Congo}
\def\langnames@fams@wals@bzz{Atlantic-Congo}
\def\langnames@fams@wals@caa{Mayan}
\def\langnames@fams@wals@cab{Arawakan}
\def\langnames@fams@wals@cac{Mayan}
\def\langnames@fams@wals@cad{Caddoan}
\def\langnames@fams@wals@cae{Atlantic-Congo}
\def\langnames@fams@wals@caf{Athabaskan-Eyak-Tlingit}
\def\langnames@fams@wals@cag{Matacoan}
\def\langnames@fams@wals@cah{Zaparoan}
\def\langnames@fams@wals@cak{Mayan}
\def\langnames@fams@wals@cal{Austronesian}
\def\langnames@fams@wals@cam{Austronesian}
\def\langnames@fams@wals@can{Lower Sepik-Ramu}
\def\langnames@fams@wals@cao{Pano-Tacanan}
\def\langnames@fams@wals@cap{Uru-Chipaya}
\def\langnames@fams@wals@caq{Austroasiatic}
\def\langnames@fams@wals@car{Cariban}
\def\langnames@fams@wals@cas{Isolate}
\def\langnames@fams@wals@cat{Indo-European}
\def\langnames@fams@wals@cav{Pano-Tacanan}
\def\langnames@fams@wals@caw{Speech Register}
\def\langnames@fams@wals@cax{Chiquitano}
\def\langnames@fams@wals@cay{Iroquoian}
\def\langnames@fams@wals@caz{Isolate}
\def\langnames@fams@wals@cbb{Arawakan}
\def\langnames@fams@wals@cbc{Tucanoan}
\def\langnames@fams@wals@cbd{Cariban}
\def\langnames@fams@wals@cbg{Chibchan}
\def\langnames@fams@wals@cbi{Barbacoan}
\def\langnames@fams@wals@cbj{Atlantic-Congo}
\def\langnames@fams@wals@cbk{Indo-European}
\def\langnames@fams@wals@cbl{Sino-Tibetan}
\def\langnames@fams@wals@cbn{Austroasiatic}
\def\langnames@fams@wals@cbo{Atlantic-Congo}
\def\langnames@fams@wals@cbq{Atlantic-Congo}
\def\langnames@fams@wals@cbr{Pano-Tacanan}
\def\langnames@fams@wals@cbs{Pano-Tacanan}
\def\langnames@fams@wals@cbt{Cahuapanan}
\def\langnames@fams@wals@cbu{Isolate}
\def\langnames@fams@wals@cbv{Kakua-Nukak}
\def\langnames@fams@wals@cbw{Austronesian}
\def\langnames@fams@wals@cby{Unclassifiable}
\def\langnames@fams@wals@ccc{Arawakan}
\def\langnames@fams@wals@ccd{Indo-European}
\def\langnames@fams@wals@cce{Atlantic-Congo}
\def\langnames@fams@wals@ccg{Atlantic-Congo}
\def\langnames@fams@wals@cch{Atlantic-Congo}
\def\langnames@fams@wals@ccj{Atlantic-Congo}
\def\langnames@fams@wals@ccl{Atlantic-Congo}
\def\langnames@fams@wals@ccm{Austronesian}
\def\langnames@fams@wals@cco{Otomanguean}
\def\langnames@fams@wals@ccp{Indo-European}
\def\langnames@fams@wals@ccr{Misumalpan}
\def\langnames@fams@wals@cda{Sino-Tibetan}
\def\langnames@fams@wals@cde{Dravidian}
\def\langnames@fams@wals@cdf{Sino-Tibetan}
\def\langnames@fams@wals@cdh{Indo-European}
\def\langnames@fams@wals@cdi{Indo-European}
\def\langnames@fams@wals@cdj{Indo-European}
\def\langnames@fams@wals@cdm{Sino-Tibetan}
\def\langnames@fams@wals@cdn{Sino-Tibetan}
\def\langnames@fams@wals@cdo{Sino-Tibetan}
\def\langnames@fams@wals@cdr{Atlantic-Congo}
\def\langnames@fams@wals@cds{Sign Language}
\def\langnames@fams@wals@cdy{Tai-Kadai}
\def\langnames@fams@wals@cdz{Austroasiatic}
\def\langnames@fams@wals@cea{Salishan}
\def\langnames@fams@wals@ceb{Austronesian}
\def\langnames@fams@wals@ceg{Zamucoan}
\def\langnames@fams@wals@cek{Sino-Tibetan}
\def\langnames@fams@wals@cen{Atlantic-Congo}
\def\langnames@fams@wals@ces{Indo-European}
\def\langnames@fams@wals@cet{Isolate}
\def\langnames@fams@wals@cfa{Atlantic-Congo}
\def\langnames@fams@wals@cfd{Atlantic-Congo}
\def\langnames@fams@wals@cfg{Atlantic-Congo}
\def\langnames@fams@wals@cfm{Sino-Tibetan}
\def\langnames@fams@wals@cga{Yuat}
\def\langnames@fams@wals@cgc{Austronesian}
\def\langnames@fams@wals@cgg{Atlantic-Congo}
\def\langnames@fams@wals@cgk{Sino-Tibetan}
\def\langnames@fams@wals@cha{Austronesian}
\def\langnames@fams@wals@chb{Chibchan}
\def\langnames@fams@wals@chc{Siouan}
\def\langnames@fams@wals@chd{Tequistlatecan}
\def\langnames@fams@wals@che{Nakh-Daghestanian}
\def\langnames@fams@wals@chf{Mayan}
\def\langnames@fams@wals@chg{Turkic}
\def\langnames@fams@wals@chh{Chinookan}
\def\langnames@fams@wals@chj{Otomanguean}
\def\langnames@fams@wals@chk{Austronesian}
\def\langnames@fams@wals@chl{Uto-Aztecan}
\def\langnames@fams@wals@chn{Chinookan}
\def\langnames@fams@wals@cho{Muskogean}
\def\langnames@fams@wals@chp{Athabaskan-Eyak-Tlingit}
\def\langnames@fams@wals@chq{Otomanguean}
\def\langnames@fams@wals@chr{Iroquoian}
\def\langnames@fams@wals@cht{Hibito-Cholon}
\def\langnames@fams@wals@chu{Indo-European}
\def\langnames@fams@wals@chv{Turkic}
\def\langnames@fams@wals@chw{Atlantic-Congo}
\def\langnames@fams@wals@chx{Sino-Tibetan}
\def\langnames@fams@wals@chy{Algic}
\def\langnames@fams@wals@chz{Otomanguean}
\def\langnames@fams@wals@cia{Austronesian}
\def\langnames@fams@wals@cib{Atlantic-Congo}
\def\langnames@fams@wals@cic{Muskogean}
\def\langnames@fams@wals@cid{Isolate}
\def\langnames@fams@wals@cie{Afro-Asiatic}
\def\langnames@fams@wals@cih{Indo-European}
\def\langnames@fams@wals@cik{Sino-Tibetan}
\def\langnames@fams@wals@cim{Indo-European}
\def\langnames@fams@wals@cin{Tupian}
\def\langnames@fams@wals@cip{Otomanguean}
\def\langnames@fams@wals@cir{Austronesian}
\def\langnames@fams@wals@ciw{Algic}
\def\langnames@fams@wals@ciy{Cariban}
\def\langnames@fams@wals@cja{Austronesian}
\def\langnames@fams@wals@cje{Austronesian}
\def\langnames@fams@wals@cjh{Salishan}
\def\langnames@fams@wals@cji{Nakh-Daghestanian}
\def\langnames@fams@wals@cjk{Atlantic-Congo}
\def\langnames@fams@wals@cjm{Austronesian}
\def\langnames@fams@wals@cjn{Sepik}
\def\langnames@fams@wals@cjo{Arawakan}
\def\langnames@fams@wals@cjp{Chibchan}
\def\langnames@fams@wals@cjs{Turkic}
\def\langnames@fams@wals@cjv{Nuclear Trans New Guinea}
\def\langnames@fams@wals@cjy{Sino-Tibetan}
\def\langnames@fams@wals@ckb{Indo-European}
\def\langnames@fams@wals@ckh{Sino-Tibetan}
\def\langnames@fams@wals@ckl{Afro-Asiatic}
\def\langnames@fams@wals@ckn{Sino-Tibetan}
\def\langnames@fams@wals@cko{Atlantic-Congo}
\def\langnames@fams@wals@ckq{Afro-Asiatic}
\def\langnames@fams@wals@ckr{Baining}
\def\langnames@fams@wals@cks{Indo-European}
\def\langnames@fams@wals@ckt{Chukotko-Kamchatkan}
\def\langnames@fams@wals@cku{Muskogean}
\def\langnames@fams@wals@ckv{Austronesian}
\def\langnames@fams@wals@ckx{Atlantic-Congo}
\def\langnames@fams@wals@cky{Afro-Asiatic}
\def\langnames@fams@wals@ckz{Mixed Language}
\def\langnames@fams@wals@cla{Afro-Asiatic}
\def\langnames@fams@wals@clc{Athabaskan-Eyak-Tlingit}
\def\langnames@fams@wals@cld{Afro-Asiatic}
\def\langnames@fams@wals@cle{Otomanguean}
\def\langnames@fams@wals@clh{Indo-European}
\def\langnames@fams@wals@cli{Atlantic-Congo}
\def\langnames@fams@wals@clk{Sino-Tibetan}
\def\langnames@fams@wals@cll{Atlantic-Congo}
\def\langnames@fams@wals@clm{Salishan}
\def\langnames@fams@wals@clo{Tequistlatecan}
\def\langnames@fams@wals@clt{Sino-Tibetan}
\def\langnames@fams@wals@clu{Austronesian}
\def\langnames@fams@wals@clw{Turkic}
\def\langnames@fams@wals@cly{Otomanguean}
\def\langnames@fams@wals@cma{Austroasiatic}
\def\langnames@fams@wals@cme{Atlantic-Congo}
\def\langnames@fams@wals@cmi{Chocoan}
\def\langnames@fams@wals@cml{Austronesian}
\def\langnames@fams@wals@cmn{Sino-Tibetan}
\def\langnames@fams@wals@cmo{Austroasiatic}
\def\langnames@fams@wals@cmr{Sino-Tibetan}
\def\langnames@fams@wals@cms{Indo-European}
\def\langnames@fams@wals@cmt{Speech Register}
\def\langnames@fams@wals@cna{Sino-Tibetan}
\def\langnames@fams@wals@cnb{Sino-Tibetan}
\def\langnames@fams@wals@cnc{Sino-Tibetan}
\def\langnames@fams@wals@cng{Sino-Tibetan}
\def\langnames@fams@wals@cnh{Sino-Tibetan}
\def\langnames@fams@wals@cni{Arawakan}
\def\langnames@fams@wals@cnk{Sino-Tibetan}
\def\langnames@fams@wals@cnl{Otomanguean}
\def\langnames@fams@wals@cnp{Sino-Tibetan}
\def\langnames@fams@wals@cnq{Atlantic-Congo}
\def\langnames@fams@wals@cns{Nuclear Trans New Guinea}
\def\langnames@fams@wals@cnt{Otomanguean}
\def\langnames@fams@wals@cnu{Afro-Asiatic}
\def\langnames@fams@wals@cnw{Sino-Tibetan}
\def\langnames@fams@wals@coa{Austronesian}
\def\langnames@fams@wals@cob{Mayan}
\def\langnames@fams@wals@coc{Cochimi-Yuman}
\def\langnames@fams@wals@cod{Tupian}
\def\langnames@fams@wals@coe{Tucanoan}
\def\langnames@fams@wals@cof{Barbacoan}
\def\langnames@fams@wals@cog{Austroasiatic}
\def\langnames@fams@wals@coh{Atlantic-Congo}
\def\langnames@fams@wals@coj{Cochimi-Yuman}
\def\langnames@fams@wals@cok{Uto-Aztecan}
\def\langnames@fams@wals@col{Salishan}
\def\langnames@fams@wals@com{Uto-Aztecan}
\def\langnames@fams@wals@con{Isolate}
\def\langnames@fams@wals@coo{Salishan}
\def\langnames@fams@wals@cop{Afro-Asiatic}
\def\langnames@fams@wals@coq{Athabaskan-Eyak-Tlingit}
\def\langnames@fams@wals@cor{Indo-European}
\def\langnames@fams@wals@cos{Indo-European}
\def\langnames@fams@wals@cot{Arawakan}
\def\langnames@fams@wals@cou{Atlantic-Congo}
\def\langnames@fams@wals@cov{Tai-Kadai}
\def\langnames@fams@wals@cow{Salishan}
\def\langnames@fams@wals@cox{Arawakan}
\def\langnames@fams@wals@coz{Otomanguean}
\def\langnames@fams@wals@cpa{Otomanguean}
\def\langnames@fams@wals@cpb{Arawakan}
\def\langnames@fams@wals@cpc{Arawakan}
\def\langnames@fams@wals@cpg{Indo-European}
\def\langnames@fams@wals@cpi{Pidgin}
\def\langnames@fams@wals@cpn{Atlantic-Congo}
\def\langnames@fams@wals@cpo{Mande}
\def\langnames@fams@wals@cps{Austronesian}
\def\langnames@fams@wals@cpu{Arawakan}
\def\langnames@fams@wals@cpx{Sino-Tibetan}
\def\langnames@fams@wals@cpy{Arawakan}
\def\langnames@fams@wals@cra{Ta-Ne-Omotic}
\def\langnames@fams@wals@crb{Arawakan}
\def\langnames@fams@wals@crc{Austronesian}
\def\langnames@fams@wals@crd{Salishan}
\def\langnames@fams@wals@crf{Chocoan}
\def\langnames@fams@wals@crg{Algic}
\def\langnames@fams@wals@crh{Turkic}
\def\langnames@fams@wals@cri{Indo-European}
\def\langnames@fams@wals@crj{Algic}
\def\langnames@fams@wals@crk{Algic}
\def\langnames@fams@wals@crl{Algic}
\def\langnames@fams@wals@crm{Algic}
\def\langnames@fams@wals@crn{Uto-Aztecan}
\def\langnames@fams@wals@cro{Siouan}
\def\langnames@fams@wals@crq{Matacoan}
\def\langnames@fams@wals@crr{Algic}
\def\langnames@fams@wals@crs{Indo-European}
\def\langnames@fams@wals@crt{Matacoan}
\def\langnames@fams@wals@crv{Austroasiatic}
\def\langnames@fams@wals@crw{Austroasiatic}
\def\langnames@fams@wals@crx{Athabaskan-Eyak-Tlingit}
\def\langnames@fams@wals@cry{Atlantic-Congo}
\def\langnames@fams@wals@crz{Chumashan}
\def\langnames@fams@wals@csa{Otomanguean}
\def\langnames@fams@wals@csb{Indo-European}
\def\langnames@fams@wals@csc{Sign Language}
\def\langnames@fams@wals@csd{Sign Language}
\def\langnames@fams@wals@cse{Sign Language}
\def\langnames@fams@wals@csf{Sign Language}
\def\langnames@fams@wals@csg{Sign Language}
\def\langnames@fams@wals@csh{Sino-Tibetan}
\def\langnames@fams@wals@csi{Miwok-Costanoan}
\def\langnames@fams@wals@csk{Atlantic-Congo}
\def\langnames@fams@wals@csl{Sign Language}
\def\langnames@fams@wals@csm{Miwok-Costanoan}
\def\langnames@fams@wals@csn{Sign Language}
\def\langnames@fams@wals@cso{Otomanguean}
\def\langnames@fams@wals@csp{Sino-Tibetan}
\def\langnames@fams@wals@csq{Sign Language}
\def\langnames@fams@wals@csr{Sign Language}
\def\langnames@fams@wals@css{Miwok-Costanoan}
\def\langnames@fams@wals@cst{Miwok-Costanoan}
\def\langnames@fams@wals@csv{Sino-Tibetan}
\def\langnames@fams@wals@csw{Algic}
\def\langnames@fams@wals@csx{Sign Language}
\def\langnames@fams@wals@csy{Sino-Tibetan}
\def\langnames@fams@wals@csz{Coosan}
\def\langnames@fams@wals@cta{Otomanguean}
\def\langnames@fams@wals@ctd{Sino-Tibetan}
\def\langnames@fams@wals@cte{Otomanguean}
\def\langnames@fams@wals@ctg{Indo-European}
\def\langnames@fams@wals@ctl{Otomanguean}
\def\langnames@fams@wals@ctm{Isolate}
\def\langnames@fams@wals@ctn{Sino-Tibetan}
\def\langnames@fams@wals@cto{Chocoan}
\def\langnames@fams@wals@ctp{Otomanguean}
\def\langnames@fams@wals@cts{Austronesian}
\def\langnames@fams@wals@ctt{Dravidian}
\def\langnames@fams@wals@ctu{Mayan}
\def\langnames@fams@wals@cty{Dravidian}
\def\langnames@fams@wals@ctz{Otomanguean}
\def\langnames@fams@wals@cua{Austroasiatic}
\def\langnames@fams@wals@cub{Tucanoan}
\def\langnames@fams@wals@cuc{Otomanguean}
\def\langnames@fams@wals@cuh{Atlantic-Congo}
\def\langnames@fams@wals@cui{Guahiboan}
\def\langnames@fams@wals@cuj{Arawakan}
\def\langnames@fams@wals@cuk{Chibchan}
\def\langnames@fams@wals@cul{Arawan}
\def\langnames@fams@wals@cuo{Cariban}
\def\langnames@fams@wals@cup{Uto-Aztecan}
\def\langnames@fams@wals@cuq{Tai-Kadai}
\def\langnames@fams@wals@cur{Sino-Tibetan}
\def\langnames@fams@wals@cut{Otomanguean}
\def\langnames@fams@wals@cuu{Tai-Kadai}
\def\langnames@fams@wals@cuv{Afro-Asiatic}
\def\langnames@fams@wals@cuw{Sino-Tibetan}
\def\langnames@fams@wals@cux{Otomanguean}
\def\langnames@fams@wals@cuy{Isolate}
\def\langnames@fams@wals@cvg{Sino-Tibetan}
\def\langnames@fams@wals@cvn{Otomanguean}
\def\langnames@fams@wals@cwa{Atlantic-Congo}
\def\langnames@fams@wals@cwb{Atlantic-Congo}
\def\langnames@fams@wals@cwd{Algic}
\def\langnames@fams@wals@cwe{Atlantic-Congo}
\def\langnames@fams@wals@cwg{Austroasiatic}
\def\langnames@fams@wals@cwt{Atlantic-Congo}
\def\langnames@fams@wals@cya{Otomanguean}
\def\langnames@fams@wals@cyb{Isolate}
\def\langnames@fams@wals@cym{Indo-European}
\def\langnames@fams@wals@cyo{Austronesian}
\def\langnames@fams@wals@czh{Sino-Tibetan}
\def\langnames@fams@wals@czn{Otomanguean}
\def\langnames@fams@wals@czo{Sino-Tibetan}
\def\langnames@fams@wals@czt{Sino-Tibetan}
\def\langnames@fams@wals@daa{Afro-Asiatic}
\def\langnames@fams@wals@dac{Austronesian}
\def\langnames@fams@wals@dad{Austronesian}
\def\langnames@fams@wals@dae{Atlantic-Congo}
\def\langnames@fams@wals@daf{Mande}
\def\langnames@fams@wals@dag{Atlantic-Congo}
\def\langnames@fams@wals@dah{Nuclear Trans New Guinea}
\def\langnames@fams@wals@dai{Atlantic-Congo}
\def\langnames@fams@wals@daj{Dajuic}
\def\langnames@fams@wals@dak{Siouan}
\def\langnames@fams@wals@dal{Afro-Asiatic}
\def\langnames@fams@wals@dam{Atlantic-Congo}
\def\langnames@fams@wals@dan{Indo-European}
\def\langnames@fams@wals@dao{Sino-Tibetan}
\def\langnames@fams@wals@daq{Dravidian}
\def\langnames@fams@wals@dar{Nakh-Daghestanian}
\def\langnames@fams@wals@das{Kru}
\def\langnames@fams@wals@dau{Dajuic}
\def\langnames@fams@wals@dav{Atlantic-Congo}
\def\langnames@fams@wals@daw{Austronesian}
\def\langnames@fams@wals@dax{Pama-Nyungan}
\def\langnames@fams@wals@daz{Nuclear Trans New Guinea}
\def\langnames@fams@wals@dba{Isolate}
\def\langnames@fams@wals@dbb{Afro-Asiatic}
\def\langnames@fams@wals@dbd{Atlantic-Congo}
\def\langnames@fams@wals@dbe{Tor-Orya}
\def\langnames@fams@wals@dbf{Lakes Plain}
\def\langnames@fams@wals@dbg{Dogon}
\def\langnames@fams@wals@dbi{Atlantic-Congo}
\def\langnames@fams@wals@dbj{Austronesian}
\def\langnames@fams@wals@dbl{Pama-Nyungan}
\def\langnames@fams@wals@dbm{Atlantic-Congo}
\def\langnames@fams@wals@dbn{Inanwatan}
\def\langnames@fams@wals@dbo{Atlantic-Congo}
\def\langnames@fams@wals@dbp{Afro-Asiatic}
\def\langnames@fams@wals@dbq{Afro-Asiatic}
\def\langnames@fams@wals@dbr{Afro-Asiatic}
\def\langnames@fams@wals@dbt{Dogon}
\def\langnames@fams@wals@dbu{Dogon}
\def\langnames@fams@wals@dbv{Unattested}
\def\langnames@fams@wals@dbw{Dogon}
\def\langnames@fams@wals@dby{Isolate}
\def\langnames@fams@wals@dcr{Indo-European}
\def\langnames@fams@wals@ddd{Nilotic}
\def\langnames@fams@wals@dde{Atlantic-Congo}
\def\langnames@fams@wals@ddg{Timor-Alor-Pantar}
\def\langnames@fams@wals@ddi{Austronesian}
\def\langnames@fams@wals@ddj{Pama-Nyungan}
\def\langnames@fams@wals@ddn{Songhay}
\def\langnames@fams@wals@ddo{Nakh-Daghestanian}
\def\langnames@fams@wals@ddr{Pama-Nyungan}
\def\langnames@fams@wals@dds{Dogon}
\def\langnames@fams@wals@ddw{Austronesian}
\def\langnames@fams@wals@dec{Narrow Talodi}
\def\langnames@fams@wals@ded{Nuclear Trans New Guinea}
\def\langnames@fams@wals@dee{Kru}
\def\langnames@fams@wals@def{Indo-European}
\def\langnames@fams@wals@deg{Atlantic-Congo}
\def\langnames@fams@wals@deh{Indo-European}
\def\langnames@fams@wals@dei{Geelvink Bay}
\def\langnames@fams@wals@dek{Unattested}
\def\langnames@fams@wals@dem{Isolate}
\def\langnames@fams@wals@dep{Pidgin}
\def\langnames@fams@wals@deq{Atlantic-Congo}
\def\langnames@fams@wals@der{Sino-Tibetan}
\def\langnames@fams@wals@des{Tucanoan}
\def\langnames@fams@wals@deu{Indo-European}
\def\langnames@fams@wals@dev{Nuclear Trans New Guinea}
\def\langnames@fams@wals@dez{Atlantic-Congo}
\def\langnames@fams@wals@dga{Atlantic-Congo}
\def\langnames@fams@wals@dgb{Dogon}
\def\langnames@fams@wals@dgc{Austronesian}
\def\langnames@fams@wals@dgd{Atlantic-Congo}
\def\langnames@fams@wals@dge{Nuclear Trans New Guinea}
\def\langnames@fams@wals@dgg{Austronesian}
\def\langnames@fams@wals@dgh{Afro-Asiatic}
\def\langnames@fams@wals@dgi{Atlantic-Congo}
\def\langnames@fams@wals@dgk{Central Sudanic}
\def\langnames@fams@wals@dgn{Yangmanic}
\def\langnames@fams@wals@dgo{Indo-European}
\def\langnames@fams@wals@dgr{Athabaskan-Eyak-Tlingit}
\def\langnames@fams@wals@dgs{Atlantic-Congo}
\def\langnames@fams@wals@dgx{Nuclear Trans New Guinea}
\def\langnames@fams@wals@dgz{Dagan}
\def\langnames@fams@wals@dhd{Indo-European}
\def\langnames@fams@wals@dhg{Pama-Nyungan}
\def\langnames@fams@wals@dhi{Sino-Tibetan}
\def\langnames@fams@wals@dhl{Pama-Nyungan}
\def\langnames@fams@wals@dhm{Atlantic-Congo}
\def\langnames@fams@wals@dhn{Indo-European}
\def\langnames@fams@wals@dho{Indo-European}
\def\langnames@fams@wals@dhr{Pama-Nyungan}
\def\langnames@fams@wals@dhs{Atlantic-Congo}
\def\langnames@fams@wals@dhu{Pama-Nyungan}
\def\langnames@fams@wals@dhv{Austronesian}
\def\langnames@fams@wals@dhw{Indo-European}
\def\langnames@fams@wals@dia{Nuclear Torricelli}
\def\langnames@fams@wals@dib{Nilotic}
\def\langnames@fams@wals@dic{Kru}
\def\langnames@fams@wals@did{Surmic}
\def\langnames@fams@wals@dif{Pama-Nyungan}
\def\langnames@fams@wals@dig{Atlantic-Congo}
\def\langnames@fams@wals@dih{Cochimi-Yuman}
\def\langnames@fams@wals@dii{Atlantic-Congo}
\def\langnames@fams@wals@dij{Austronesian}
\def\langnames@fams@wals@dik{Nilotic}
\def\langnames@fams@wals@dil{Nubian}
\def\langnames@fams@wals@dim{South Omotic}
\def\langnames@fams@wals@dio{Atlantic-Congo}
\def\langnames@fams@wals@dip{Nilotic}
\def\langnames@fams@wals@diq{Indo-European}
\def\langnames@fams@wals@dir{Atlantic-Congo}
\def\langnames@fams@wals@dis{Sino-Tibetan}
\def\langnames@fams@wals@diu{Atlantic-Congo}
\def\langnames@fams@wals@div{Indo-European}
\def\langnames@fams@wals@diw{Nilotic}
\def\langnames@fams@wals@dix{Austronesian}
\def\langnames@fams@wals@diy{Nuclear Trans New Guinea}
\def\langnames@fams@wals@diz{Atlantic-Congo}
\def\langnames@fams@wals@djb{Pama-Nyungan}
\def\langnames@fams@wals@djc{Dajuic}
\def\langnames@fams@wals@djd{Mirndi}
\def\langnames@fams@wals@dje{Songhay}
\def\langnames@fams@wals@djf{Pama-Nyungan}
\def\langnames@fams@wals@dji{Pama-Nyungan}
\def\langnames@fams@wals@djj{Maningrida}
\def\langnames@fams@wals@djk{Indo-European}
\def\langnames@fams@wals@djm{Dogon}
\def\langnames@fams@wals@djn{Gunwinyguan}
\def\langnames@fams@wals@djo{Austronesian}
\def\langnames@fams@wals@djr{Pama-Nyungan}
\def\langnames@fams@wals@dju{Sepik}
\def\langnames@fams@wals@djw{Nyulnyulan}
\def\langnames@fams@wals@dka{Sino-Tibetan}
\def\langnames@fams@wals@dkg{Atlantic-Congo}
\def\langnames@fams@wals@dkk{Austronesian}
\def\langnames@fams@wals@dkr{Austronesian}
\def\langnames@fams@wals@dks{Nilotic}
\def\langnames@fams@wals@dkx{Afro-Asiatic}
\def\langnames@fams@wals@dlg{Turkic}
\def\langnames@fams@wals@dlk{Afro-Asiatic}
\def\langnames@fams@wals@dlm{Indo-European}
\def\langnames@fams@wals@dln{Sino-Tibetan}
\def\langnames@fams@wals@dma{Atlantic-Congo}
\def\langnames@fams@wals@dmb{Dogon}
\def\langnames@fams@wals@dmc{Nuclear Trans New Guinea}
\def\langnames@fams@wals@dme{Afro-Asiatic}
\def\langnames@fams@wals@dmf{Speech Register}
\def\langnames@fams@wals@dmg{Austronesian}
\def\langnames@fams@wals@dmk{Indo-European}
\def\langnames@fams@wals@dml{Indo-European}
\def\langnames@fams@wals@dmm{Atlantic-Congo}
\def\langnames@fams@wals@dmo{Atlantic-Congo}
\def\langnames@fams@wals@dmr{Austronesian}
\def\langnames@fams@wals@dms{Austronesian}
\def\langnames@fams@wals@dmu{Pauwasi}
\def\langnames@fams@wals@dmv{Austronesian}
\def\langnames@fams@wals@dmw{Pama-Nyungan}
\def\langnames@fams@wals@dmx{Atlantic-Congo}
\def\langnames@fams@wals@dmy{Sentanic}
\def\langnames@fams@wals@dna{Nuclear Trans New Guinea}
\def\langnames@fams@wals@dnd{Border}
\def\langnames@fams@wals@dne{Atlantic-Congo}
\def\langnames@fams@wals@dng{Sino-Tibetan}
\def\langnames@fams@wals@dni{Nuclear Trans New Guinea}
\def\langnames@fams@wals@dnk{Austronesian}
\def\langnames@fams@wals@dnn{Mande}
\def\langnames@fams@wals@dno{Central Sudanic}
\def\langnames@fams@wals@dnr{Nuclear Trans New Guinea}
\def\langnames@fams@wals@dnt{Nuclear Trans New Guinea}
\def\langnames@fams@wals@dnu{Austroasiatic}
\def\langnames@fams@wals@dnw{Nuclear Trans New Guinea}
\def\langnames@fams@wals@dny{Arawan}
\def\langnames@fams@wals@doa{Nuclear Trans New Guinea}
\def\langnames@fams@wals@dob{Austronesian}
\def\langnames@fams@wals@doc{Tai-Kadai}
\def\langnames@fams@wals@doe{Atlantic-Congo}
\def\langnames@fams@wals@dof{Mailuan}
\def\langnames@fams@wals@doh{Atlantic-Congo}
\def\langnames@fams@wals@dok{Austronesian}
\def\langnames@fams@wals@dol{Doso-Turumsa}
\def\langnames@fams@wals@don{Austronesian}
\def\langnames@fams@wals@doo{Atlantic-Congo}
\def\langnames@fams@wals@dop{Atlantic-Congo}
\def\langnames@fams@wals@doq{Sign Language}
\def\langnames@fams@wals@dor{Austronesian}
\def\langnames@fams@wals@dos{Atlantic-Congo}
\def\langnames@fams@wals@dot{Afro-Asiatic}
\def\langnames@fams@wals@dov{Atlantic-Congo}
\def\langnames@fams@wals@dow{Atlantic-Congo}
\def\langnames@fams@wals@dox{Afro-Asiatic}
\def\langnames@fams@wals@doy{Atlantic-Congo}
\def\langnames@fams@wals@doz{Ta-Ne-Omotic}
\def\langnames@fams@wals@dpp{Austronesian}
\def\langnames@fams@wals@drb{Nubian}
\def\langnames@fams@wals@drc{Indo-European}
\def\langnames@fams@wals@drd{Sino-Tibetan}
\def\langnames@fams@wals@dre{Sino-Tibetan}
\def\langnames@fams@wals@drg{Austronesian}
\def\langnames@fams@wals@dri{Atlantic-Congo}
\def\langnames@fams@wals@drl{Pama-Nyungan}
\def\langnames@fams@wals@drn{Austronesian}
\def\langnames@fams@wals@dro{Austronesian}
\def\langnames@fams@wals@drq{Sino-Tibetan}
\def\langnames@fams@wals@drs{Afro-Asiatic}
\def\langnames@fams@wals@dru{Austronesian}
\def\langnames@fams@wals@dry{Indo-European}
\def\langnames@fams@wals@dsb{Indo-European}
\def\langnames@fams@wals@dse{Sign Language}
\def\langnames@fams@wals@dsh{Afro-Asiatic}
\def\langnames@fams@wals@dsi{Central Sudanic}
\def\langnames@fams@wals@dsl{Sign Language}
\def\langnames@fams@wals@dsn{Austronesian}
\def\langnames@fams@wals@dsq{Songhay}
\def\langnames@fams@wals@dsz{Sign Language}
\def\langnames@fams@wals@dta{Mongolic-Khitan}
\def\langnames@fams@wals@dtb{Austronesian}
\def\langnames@fams@wals@dtd{Wakashan}
\def\langnames@fams@wals@dth{Pama-Nyungan}
\def\langnames@fams@wals@dti{Dogon}
\def\langnames@fams@wals@dtk{Dogon}
\def\langnames@fams@wals@dtm{Dogon}
\def\langnames@fams@wals@dtn{Gumuz}
\def\langnames@fams@wals@dto{Dogon}
\def\langnames@fams@wals@dtp{Austronesian}
\def\langnames@fams@wals@dtr{Austronesian}
\def\langnames@fams@wals@dts{Dogon}
\def\langnames@fams@wals@dtt{Dogon}
\def\langnames@fams@wals@dtu{Dogon}
\def\langnames@fams@wals@dty{Indo-European}
\def\langnames@fams@wals@dua{Atlantic-Congo}
\def\langnames@fams@wals@dub{Indo-European}
\def\langnames@fams@wals@duc{Isolate}
\def\langnames@fams@wals@dud{Atlantic-Congo}
\def\langnames@fams@wals@due{Austronesian}
\def\langnames@fams@wals@duf{Austronesian}
\def\langnames@fams@wals@dug{Atlantic-Congo}
\def\langnames@fams@wals@duh{Indo-European}
\def\langnames@fams@wals@dui{Nuclear Trans New Guinea}
\def\langnames@fams@wals@duj{Pama-Nyungan}
\def\langnames@fams@wals@duk{Nuclear Trans New Guinea}
\def\langnames@fams@wals@dul{Austronesian}
\def\langnames@fams@wals@dum{Indo-European}
\def\langnames@fams@wals@dun{Austronesian}
\def\langnames@fams@wals@duo{Austronesian}
\def\langnames@fams@wals@dup{Austronesian}
\def\langnames@fams@wals@duq{Austronesian}
\def\langnames@fams@wals@dur{Atlantic-Congo}
\def\langnames@fams@wals@dus{Sino-Tibetan}
\def\langnames@fams@wals@duu{Sino-Tibetan}
\def\langnames@fams@wals@duv{Lakes Plain}
\def\langnames@fams@wals@duw{Austronesian}
\def\langnames@fams@wals@dux{Mande}
\def\langnames@fams@wals@duy{Austronesian}
\def\langnames@fams@wals@duz{Atlantic-Congo}
\def\langnames@fams@wals@dva{Austronesian}
\def\langnames@fams@wals@dwa{Afro-Asiatic}
\def\langnames@fams@wals@dwr{Ta-Ne-Omotic}
\def\langnames@fams@wals@dws{Artificial Language}
\def\langnames@fams@wals@dww{Austronesian}
\def\langnames@fams@wals@dwz{Indo-European}
\def\langnames@fams@wals@dya{Atlantic-Congo}
\def\langnames@fams@wals@dyb{Nyulnyulan}
\def\langnames@fams@wals@dyd{Nyulnyulan}
\def\langnames@fams@wals@dyg{Unattested}
\def\langnames@fams@wals@dyi{Atlantic-Congo}
\def\langnames@fams@wals@dym{Dogon}
\def\langnames@fams@wals@dyn{Pama-Nyungan}
\def\langnames@fams@wals@dyo{Atlantic-Congo}
\def\langnames@fams@wals@dyu{Mande}
\def\langnames@fams@wals@dyy{Pama-Nyungan}
\def\langnames@fams@wals@dza{Atlantic-Congo}
\def\langnames@fams@wals@dzd{Unattested}
\def\langnames@fams@wals@dze{Pama-Nyungan}
\def\langnames@fams@wals@dzg{Saharan}
\def\langnames@fams@wals@dzl{Sino-Tibetan}
\def\langnames@fams@wals@dzn{Atlantic-Congo}
\def\langnames@fams@wals@dzo{Sino-Tibetan}
\def\langnames@fams@wals@ebg{Atlantic-Congo}
\def\langnames@fams@wals@ebo{Atlantic-Congo}
\def\langnames@fams@wals@ebr{Atlantic-Congo}
\def\langnames@fams@wals@ebu{Atlantic-Congo}
\def\langnames@fams@wals@ecr{Unclassifiable}
\def\langnames@fams@wals@ecs{Sign Language}
\def\langnames@fams@wals@ecy{Unclassifiable}
\def\langnames@fams@wals@eee{Tai-Kadai}
\def\langnames@fams@wals@efa{Atlantic-Congo}
\def\langnames@fams@wals@efe{Central Sudanic}
\def\langnames@fams@wals@efi{Atlantic-Congo}
\def\langnames@fams@wals@ega{Atlantic-Congo}
\def\langnames@fams@wals@egl{Indo-European}
\def\langnames@fams@wals@ego{Atlantic-Congo}
\def\langnames@fams@wals@egy{Afro-Asiatic}
\def\langnames@fams@wals@ehs{Sign Language}
\def\langnames@fams@wals@ehu{Atlantic-Congo}
\def\langnames@fams@wals@eip{Nuclear Trans New Guinea}
\def\langnames@fams@wals@eit{Nuclear Torricelli}
\def\langnames@fams@wals@eiv{North Bougainville}
\def\langnames@fams@wals@eja{Atlantic-Congo}
\def\langnames@fams@wals@eka{Atlantic-Congo}
\def\langnames@fams@wals@eke{Atlantic-Congo}
\def\langnames@fams@wals@ekg{Nuclear Trans New Guinea}
\def\langnames@fams@wals@eki{Atlantic-Congo}
\def\langnames@fams@wals@ekk{Uralic}
\def\langnames@fams@wals@ekl{Austroasiatic}
\def\langnames@fams@wals@ekm{Atlantic-Congo}
\def\langnames@fams@wals@eko{Atlantic-Congo}
\def\langnames@fams@wals@ekp{Atlantic-Congo}
\def\langnames@fams@wals@ekr{Atlantic-Congo}
\def\langnames@fams@wals@eky{Sino-Tibetan}
\def\langnames@fams@wals@ele{Nuclear Torricelli}
\def\langnames@fams@wals@elh{Nubian}
\def\langnames@fams@wals@eli{Narrow Talodi}
\def\langnames@fams@wals@elk{Nuclear Torricelli}
\def\langnames@fams@wals@ell{Indo-European}
\def\langnames@fams@wals@elm{Atlantic-Congo}
\def\langnames@fams@wals@elo{Afro-Asiatic}
\def\langnames@fams@wals@elu{Austronesian}
\def\langnames@fams@wals@elx{Isolate}
\def\langnames@fams@wals@ema{Atlantic-Congo}
\def\langnames@fams@wals@emb{Austronesian}
\def\langnames@fams@wals@eme{Tupian}
\def\langnames@fams@wals@emg{Sino-Tibetan}
\def\langnames@fams@wals@emi{Austronesian}
\def\langnames@fams@wals@emk{Mande}
\def\langnames@fams@wals@emn{Atlantic-Congo}
\def\langnames@fams@wals@emp{Chocoan}
\def\langnames@fams@wals@emq{Sino-Tibetan}
\def\langnames@fams@wals@ems{Eskimo-Aleut}
\def\langnames@fams@wals@emu{Dravidian}
\def\langnames@fams@wals@emw{Austronesian}
\def\langnames@fams@wals@emx{Speech Register}
\def\langnames@fams@wals@emy{Mayan}
\def\langnames@fams@wals@emz{Atlantic-Congo}
\def\langnames@fams@wals@ena{Nuclear Trans New Guinea}
\def\langnames@fams@wals@enb{Nilotic}
\def\langnames@fams@wals@enc{Tai-Kadai}
\def\langnames@fams@wals@end{Austronesian}
\def\langnames@fams@wals@enf{Uralic}
\def\langnames@fams@wals@eng{Indo-European}
\def\langnames@fams@wals@enh{Uralic}
\def\langnames@fams@wals@enl{Lengua-Mascoy}
\def\langnames@fams@wals@enm{Indo-European}
\def\langnames@fams@wals@enn{Atlantic-Congo}
\def\langnames@fams@wals@eno{Austronesian}
\def\langnames@fams@wals@enq{Nuclear Trans New Guinea}
\def\langnames@fams@wals@enr{Pauwasi}
\def\langnames@fams@wals@enu{Sino-Tibetan}
\def\langnames@fams@wals@env{Atlantic-Congo}
\def\langnames@fams@wals@enw{Atlantic-Congo}
\def\langnames@fams@wals@enx{Lengua-Mascoy}
\def\langnames@fams@wals@eot{Atlantic-Congo}
\def\langnames@fams@wals@epi{Atlantic-Congo}
\def\langnames@fams@wals@epo{Artificial Language}
\def\langnames@fams@wals@era{Dravidian}
\def\langnames@fams@wals@erg{Austronesian}
\def\langnames@fams@wals@erh{Atlantic-Congo}
\def\langnames@fams@wals@eri{Nuclear Trans New Guinea}
\def\langnames@fams@wals@erk{Austronesian}
\def\langnames@fams@wals@ero{Sino-Tibetan}
\def\langnames@fams@wals@err{Giimbiyu}
\def\langnames@fams@wals@ers{Sino-Tibetan}
\def\langnames@fams@wals@ert{Lakes Plain}
\def\langnames@fams@wals@erw{Austronesian}
\def\langnames@fams@wals@ese{Pano-Tacanan}
\def\langnames@fams@wals@esg{Dravidian}
\def\langnames@fams@wals@esh{Indo-European}
\def\langnames@fams@wals@esi{Eskimo-Aleut}
\def\langnames@fams@wals@esk{Eskimo-Aleut}
\def\langnames@fams@wals@esl{Sign Language}
\def\langnames@fams@wals@esm{Unattested}
\def\langnames@fams@wals@esn{Sign Language}
\def\langnames@fams@wals@eso{Sign Language}
\def\langnames@fams@wals@esq{Isolate}
\def\langnames@fams@wals@ess{Eskimo-Aleut}
\def\langnames@fams@wals@esu{Eskimo-Aleut}
\def\langnames@fams@wals@esy{Artificial Language}
\def\langnames@fams@wals@etb{Atlantic-Congo}
\def\langnames@fams@wals@eth{Sign Language}
\def\langnames@fams@wals@etn{Austronesian}
\def\langnames@fams@wals@eto{Atlantic-Congo}
\def\langnames@fams@wals@etr{Bosavi}
\def\langnames@fams@wals@ets{Atlantic-Congo}
\def\langnames@fams@wals@ett{Isolate}
\def\langnames@fams@wals@etu{Atlantic-Congo}
\def\langnames@fams@wals@etx{Atlantic-Congo}
\def\langnames@fams@wals@etz{Mairasic}
\def\langnames@fams@wals@eus{Isolate}
\def\langnames@fams@wals@eve{Tungusic}
\def\langnames@fams@wals@evh{Atlantic-Congo}
\def\langnames@fams@wals@evn{Tungusic}
\def\langnames@fams@wals@ewe{Atlantic-Congo}
\def\langnames@fams@wals@ewo{Atlantic-Congo}
\def\langnames@fams@wals@ext{Indo-European}
\def\langnames@fams@wals@eya{Athabaskan-Eyak-Tlingit}
\def\langnames@fams@wals@eyo{Nilotic}
\def\langnames@fams@wals@eze{Atlantic-Congo}
\def\langnames@fams@wals@faa{Isolate}
\def\langnames@fams@wals@fab{Indo-European}
\def\langnames@fams@wals@fad{Nuclear Trans New Guinea}
\def\langnames@fams@wals@faf{Austronesian}
\def\langnames@fams@wals@fag{Nuclear Trans New Guinea}
\def\langnames@fams@wals@fah{Atlantic-Congo}
\def\langnames@fams@wals@fai{Nuclear Trans New Guinea}
\def\langnames@fams@wals@faj{Nuclear Trans New Guinea}
\def\langnames@fams@wals@fak{Atlantic-Congo}
\def\langnames@fams@wals@fal{Atlantic-Congo}
\def\langnames@fams@wals@fam{Atlantic-Congo}
\def\langnames@fams@wals@fan{Atlantic-Congo}
\def\langnames@fams@wals@fao{Indo-European}
\def\langnames@fams@wals@fap{Atlantic-Congo}
\def\langnames@fams@wals@far{Austronesian}
\def\langnames@fams@wals@fau{Lakes Plain}
\def\langnames@fams@wals@fax{Indo-European}
\def\langnames@fams@wals@fay{Indo-European}
\def\langnames@fams@wals@fcs{Sign Language}
\def\langnames@fams@wals@fer{Atlantic-Congo}
\def\langnames@fams@wals@ffm{Atlantic-Congo}
\def\langnames@fams@wals@fia{Nubian}
\def\langnames@fams@wals@fie{Afro-Asiatic}
\def\langnames@fams@wals@fif{Afro-Asiatic}
\def\langnames@fams@wals@fij{Austronesian}
\def\langnames@fams@wals@fil{Austronesian}
\def\langnames@fams@wals@fin{Uralic}
\def\langnames@fams@wals@fip{Atlantic-Congo}
\def\langnames@fams@wals@fir{Atlantic-Congo}
\def\langnames@fams@wals@fit{Uralic}
\def\langnames@fams@wals@fiw{East Kutubu}
\def\langnames@fams@wals@fkk{Afro-Asiatic}
\def\langnames@fams@wals@fkv{Uralic}
\def\langnames@fams@wals@fla{Salishan}
\def\langnames@fams@wals@flh{Lakes Plain}
\def\langnames@fams@wals@fli{Afro-Asiatic}
\def\langnames@fams@wals@fll{Atlantic-Congo}
\def\langnames@fams@wals@fln{Pama-Nyungan}
\def\langnames@fams@wals@flr{Atlantic-Congo}
\def\langnames@fams@wals@fly{Speech Register}
\def\langnames@fams@wals@fmp{Atlantic-Congo}
\def\langnames@fams@wals@fmu{Dravidian}
\def\langnames@fams@wals@fnb{Austronesian}
\def\langnames@fams@wals@fng{Pidgin}
\def\langnames@fams@wals@fni{Atlantic-Congo}
\def\langnames@fams@wals@fod{Atlantic-Congo}
\def\langnames@fams@wals@foi{East Kutubu}
\def\langnames@fams@wals@fon{Atlantic-Congo}
\def\langnames@fams@wals@for{Nuclear Trans New Guinea}
\def\langnames@fams@wals@fos{Austronesian}
\def\langnames@fams@wals@fpe{Indo-European}
\def\langnames@fams@wals@fqs{Baibai-Fas}
\def\langnames@fams@wals@fra{Indo-European}
\def\langnames@fams@wals@frc{Indo-European}
\def\langnames@fams@wals@frd{Austronesian}
\def\langnames@fams@wals@fro{Indo-European}
\def\langnames@fams@wals@frp{Indo-European}
\def\langnames@fams@wals@frq{Nuclear Trans New Guinea}
\def\langnames@fams@wals@frr{Indo-European}
\def\langnames@fams@wals@frs{Indo-European}
\def\langnames@fams@wals@frt{Austronesian}
\def\langnames@fams@wals@fry{Indo-European}
\def\langnames@fams@wals@fse{Sign Language}
\def\langnames@fams@wals@fsl{Sign Language}
\def\langnames@fams@wals@fss{Sign Language}
\def\langnames@fams@wals@fub{Atlantic-Congo}
\def\langnames@fams@wals@fuc{Atlantic-Congo}
\def\langnames@fams@wals@fud{Austronesian}
\def\langnames@fams@wals@fue{Atlantic-Congo}
\def\langnames@fams@wals@fuf{Atlantic-Congo}
\def\langnames@fams@wals@fuh{Atlantic-Congo}
\def\langnames@fams@wals@fui{Atlantic-Congo}
\def\langnames@fams@wals@fuj{Heibanic}
\def\langnames@fams@wals@fun{Isolate}
\def\langnames@fams@wals@fuq{Atlantic-Congo}
\def\langnames@fams@wals@fur{Indo-European}
\def\langnames@fams@wals@fut{Austronesian}
\def\langnames@fams@wals@fuu{Central Sudanic}
\def\langnames@fams@wals@fuv{Atlantic-Congo}
\def\langnames@fams@wals@fuy{Goilalan}
\def\langnames@fams@wals@fvr{Furan}
\def\langnames@fams@wals@fwa{Austronesian}
\def\langnames@fams@wals@fwe{Atlantic-Congo}
\def\langnames@fams@wals@gaa{Atlantic-Congo}
\def\langnames@fams@wals@gab{Afro-Asiatic}
\def\langnames@fams@wals@gad{Austronesian}
\def\langnames@fams@wals@gae{Arawakan}
\def\langnames@fams@wals@gaf{Nuclear Trans New Guinea}
\def\langnames@fams@wals@gag{Turkic}
\def\langnames@fams@wals@gah{Nuclear Trans New Guinea}
\def\langnames@fams@wals@gai{Lower Sepik-Ramu}
\def\langnames@fams@wals@gaj{Nuclear Trans New Guinea}
\def\langnames@fams@wals@gak{North Halmahera}
\def\langnames@fams@wals@gal{Austronesian}
\def\langnames@fams@wals@gam{Nuclear Trans New Guinea}
\def\langnames@fams@wals@gan{Sino-Tibetan}
\def\langnames@fams@wals@gao{Nuclear Trans New Guinea}
\def\langnames@fams@wals@gap{Nuclear Trans New Guinea}
\def\langnames@fams@wals@gaq{Austroasiatic}
\def\langnames@fams@wals@gar{Austronesian}
\def\langnames@fams@wals@gas{Indo-European}
\def\langnames@fams@wals@gat{Nuclear Trans New Guinea}
\def\langnames@fams@wals@gau{Dravidian}
\def\langnames@fams@wals@gaw{Nuclear Trans New Guinea}
\def\langnames@fams@wals@gax{Afro-Asiatic}
\def\langnames@fams@wals@gay{Austronesian}
\def\langnames@fams@wals@gaz{Afro-Asiatic}
\def\langnames@fams@wals@gbb{Pama-Nyungan}
\def\langnames@fams@wals@gbd{Pama-Nyungan}
\def\langnames@fams@wals@gbe{Sepik}
\def\langnames@fams@wals@gbf{Ndu}
\def\langnames@fams@wals@gbg{Atlantic-Congo}
\def\langnames@fams@wals@gbh{Atlantic-Congo}
\def\langnames@fams@wals@gbi{North Halmahera}
\def\langnames@fams@wals@gbj{Austroasiatic}
\def\langnames@fams@wals@gbk{Indo-European}
\def\langnames@fams@wals@gbl{Indo-European}
\def\langnames@fams@wals@gbm{Indo-European}
\def\langnames@fams@wals@gbn{Central Sudanic}
\def\langnames@fams@wals@gbo{Kru}
\def\langnames@fams@wals@gbp{Atlantic-Congo}
\def\langnames@fams@wals@gbq{Atlantic-Congo}
\def\langnames@fams@wals@gbr{Atlantic-Congo}
\def\langnames@fams@wals@gbs{Atlantic-Congo}
\def\langnames@fams@wals@gbu{Isolate}
\def\langnames@fams@wals@gbv{Atlantic-Congo}
\def\langnames@fams@wals@gbw{Pama-Nyungan}
\def\langnames@fams@wals@gbx{Atlantic-Congo}
\def\langnames@fams@wals@gby{Atlantic-Congo}
\def\langnames@fams@wals@gbz{Indo-European}
\def\langnames@fams@wals@gcc{Baining}
\def\langnames@fams@wals@gcd{Tangkic}
\def\langnames@fams@wals@gce{Athabaskan-Eyak-Tlingit}
\def\langnames@fams@wals@gcf{Indo-European}
\def\langnames@fams@wals@gcl{Indo-European}
\def\langnames@fams@wals@gcn{Nuclear Trans New Guinea}
\def\langnames@fams@wals@gcr{Indo-European}
\def\langnames@fams@wals@gct{Indo-European}
\def\langnames@fams@wals@gda{Indo-European}
\def\langnames@fams@wals@gdb{Dravidian}
\def\langnames@fams@wals@gdc{Pama-Nyungan}
\def\langnames@fams@wals@gdd{Austronesian}
\def\langnames@fams@wals@gde{Afro-Asiatic}
\def\langnames@fams@wals@gdf{Afro-Asiatic}
\def\langnames@fams@wals@gdg{Austronesian}
\def\langnames@fams@wals@gdh{Jarrakan}
\def\langnames@fams@wals@gdi{Atlantic-Congo}
\def\langnames@fams@wals@gdj{Pama-Nyungan}
\def\langnames@fams@wals@gdk{Afro-Asiatic}
\def\langnames@fams@wals@gdl{Afro-Asiatic}
\def\langnames@fams@wals@gdm{Isolate}
\def\langnames@fams@wals@gdn{Dagan}
\def\langnames@fams@wals@gdo{Nakh-Daghestanian}
\def\langnames@fams@wals@gdq{Afro-Asiatic}
\def\langnames@fams@wals@gdr{Eastern Trans-Fly}
\def\langnames@fams@wals@gds{Sign Language}
\def\langnames@fams@wals@gdu{Afro-Asiatic}
\def\langnames@fams@wals@gdx{Indo-European}
\def\langnames@fams@wals@gea{Afro-Asiatic}
\def\langnames@fams@wals@geb{Lower Sepik-Ramu}
\def\langnames@fams@wals@gec{Kru}
\def\langnames@fams@wals@ged{Atlantic-Congo}
\def\langnames@fams@wals@geh{Indo-European}
\def\langnames@fams@wals@gei{Austronesian}
\def\langnames@fams@wals@gej{Atlantic-Congo}
\def\langnames@fams@wals@gek{Afro-Asiatic}
\def\langnames@fams@wals@gel{Atlantic-Congo}
\def\langnames@fams@wals@geq{Atlantic-Congo}
\def\langnames@fams@wals@ges{Austronesian}
\def\langnames@fams@wals@gev{Atlantic-Congo}
\def\langnames@fams@wals@gew{Afro-Asiatic}
\def\langnames@fams@wals@gex{Afro-Asiatic}
\def\langnames@fams@wals@gey{Atlantic-Congo}
\def\langnames@fams@wals@gez{Afro-Asiatic}
\def\langnames@fams@wals@gfk{Austronesian}
\def\langnames@fams@wals@gft{Afro-Asiatic}
\def\langnames@fams@wals@gga{Austronesian}
\def\langnames@fams@wals@ggb{Kru}
\def\langnames@fams@wals@ggd{Pama-Nyungan}
\def\langnames@fams@wals@gge{Maningrida}
\def\langnames@fams@wals@ggg{Indo-European}
\def\langnames@fams@wals@ggk{Isolate}
\def\langnames@fams@wals@ggl{Nuclear Trans New Guinea}
\def\langnames@fams@wals@ggr{Pama-Nyungan}
\def\langnames@fams@wals@ggt{Austronesian}
\def\langnames@fams@wals@ggu{Mande}
\def\langnames@fams@wals@ggw{Suki-Gogodala}
\def\langnames@fams@wals@gha{Afro-Asiatic}
\def\langnames@fams@wals@ghe{Sino-Tibetan}
\def\langnames@fams@wals@ghh{Sino-Tibetan}
\def\langnames@fams@wals@ghk{Sino-Tibetan}
\def\langnames@fams@wals@ghl{Nubian}
\def\langnames@fams@wals@ghn{Austronesian}
\def\langnames@fams@wals@gho{Afro-Asiatic}
\def\langnames@fams@wals@ghr{Indo-European}
\def\langnames@fams@wals@ghs{Nuclear Trans New Guinea}
\def\langnames@fams@wals@ght{Sino-Tibetan}
\def\langnames@fams@wals@gia{Jarrakan}
\def\langnames@fams@wals@gib{Pidgin}
\def\langnames@fams@wals@gic{Unclassifiable}
\def\langnames@fams@wals@gid{Afro-Asiatic}
\def\langnames@fams@wals@gie{Kru}
\def\langnames@fams@wals@gig{Indo-European}
\def\langnames@fams@wals@gih{Pama-Nyungan}
\def\langnames@fams@wals@gii{Afro-Asiatic}
\def\langnames@fams@wals@gil{Austronesian}
\def\langnames@fams@wals@gim{Nuclear Trans New Guinea}
\def\langnames@fams@wals@gin{Nakh-Daghestanian}
\def\langnames@fams@wals@gip{Austronesian}
\def\langnames@fams@wals@giq{Tai-Kadai}
\def\langnames@fams@wals@gir{Tai-Kadai}
\def\langnames@fams@wals@gis{Afro-Asiatic}
\def\langnames@fams@wals@git{Tsimshian}
\def\langnames@fams@wals@giu{Tai-Kadai}
\def\langnames@fams@wals@giw{Tai-Kadai}
\def\langnames@fams@wals@gix{Atlantic-Congo}
\def\langnames@fams@wals@giy{Unattested}
\def\langnames@fams@wals@giz{Afro-Asiatic}
\def\langnames@fams@wals@gjk{Indo-European}
\def\langnames@fams@wals@gjm{Pama-Nyungan}
\def\langnames@fams@wals@gjn{Atlantic-Congo}
\def\langnames@fams@wals@gjr{Mixed Language}
\def\langnames@fams@wals@gju{Indo-European}
\def\langnames@fams@wals@gka{Nuclear Trans New Guinea}
\def\langnames@fams@wals@gkd{Nuclear Trans New Guinea}
\def\langnames@fams@wals@gke{Atlantic-Congo}
\def\langnames@fams@wals@gkn{Atlantic-Congo}
\def\langnames@fams@wals@gko{Pama-Nyungan}
\def\langnames@fams@wals@gkp{Mande}
\def\langnames@fams@wals@gku{Tuu}
\def\langnames@fams@wals@gla{Indo-European}
\def\langnames@fams@wals@glb{Afro-Asiatic}
\def\langnames@fams@wals@glc{Atlantic-Congo}
\def\langnames@fams@wals@gld{Tungusic}
\def\langnames@fams@wals@gle{Indo-European}
\def\langnames@fams@wals@glg{Indo-European}
\def\langnames@fams@wals@glh{Indo-European}
\def\langnames@fams@wals@glj{Atlantic-Congo}
\def\langnames@fams@wals@glk{Indo-European}
\def\langnames@fams@wals@gll{Pama-Nyungan}
\def\langnames@fams@wals@glo{Afro-Asiatic}
\def\langnames@fams@wals@glr{Kru}
\def\langnames@fams@wals@glu{Central Sudanic}
\def\langnames@fams@wals@glv{Indo-European}
\def\langnames@fams@wals@glw{Afro-Asiatic}
\def\langnames@fams@wals@gly{Isolate}
\def\langnames@fams@wals@gma{Worrorran}
\def\langnames@fams@wals@gmb{Austronesian}
\def\langnames@fams@wals@gmd{Atlantic-Congo}
\def\langnames@fams@wals@gmg{Nuclear Trans New Guinea}
\def\langnames@fams@wals@gmh{Indo-European}
\def\langnames@fams@wals@gml{Indo-European}
\def\langnames@fams@wals@gmm{Atlantic-Congo}
\def\langnames@fams@wals@gmn{Atlantic-Congo}
\def\langnames@fams@wals@gmu{Nuclear Trans New Guinea}
\def\langnames@fams@wals@gmv{Ta-Ne-Omotic}
\def\langnames@fams@wals@gmx{Atlantic-Congo}
\def\langnames@fams@wals@gmy{Indo-European}
\def\langnames@fams@wals@gna{Atlantic-Congo}
\def\langnames@fams@wals@gnb{Sino-Tibetan}
\def\langnames@fams@wals@gnc{Afro-Asiatic}
\def\langnames@fams@wals@gnd{Afro-Asiatic}
\def\langnames@fams@wals@gne{Atlantic-Congo}
\def\langnames@fams@wals@gng{Atlantic-Congo}
\def\langnames@fams@wals@gnh{Atlantic-Congo}
\def\langnames@fams@wals@gni{Bunaban}
\def\langnames@fams@wals@gnj{Mande}
\def\langnames@fams@wals@gnk{Khoe-Kwadi}
\def\langnames@fams@wals@gnl{Pama-Nyungan}
\def\langnames@fams@wals@gnm{Dagan}
\def\langnames@fams@wals@gnn{Pama-Nyungan}
\def\langnames@fams@wals@gno{Dravidian}
\def\langnames@fams@wals@gnq{Austronesian}
\def\langnames@fams@wals@gnr{Pama-Nyungan}
\def\langnames@fams@wals@gnt{Yam}
\def\langnames@fams@wals@gnu{Nuclear Torricelli}
\def\langnames@fams@wals@gnw{Tupian}
\def\langnames@fams@wals@gnz{Atlantic-Congo}
\def\langnames@fams@wals@goa{Mande}
\def\langnames@fams@wals@gob{Guahiboan}
\def\langnames@fams@wals@goc{Austronesian}
\def\langnames@fams@wals@god{Kru}
\def\langnames@fams@wals@goe{Sino-Tibetan}
\def\langnames@fams@wals@gof{Ta-Ne-Omotic}
\def\langnames@fams@wals@gog{Atlantic-Congo}
\def\langnames@fams@wals@goh{Indo-European}
\def\langnames@fams@wals@goi{East Strickland}
\def\langnames@fams@wals@gol{Atlantic-Congo}
\def\langnames@fams@wals@gom{Indo-European}
\def\langnames@fams@wals@goo{Austronesian}
\def\langnames@fams@wals@gop{Austronesian}
\def\langnames@fams@wals@goq{Austronesian}
\def\langnames@fams@wals@gor{Austronesian}
\def\langnames@fams@wals@gos{Indo-European}
\def\langnames@fams@wals@got{Indo-European}
\def\langnames@fams@wals@gou{Afro-Asiatic}
\def\langnames@fams@wals@gov{Mande}
\def\langnames@fams@wals@gow{Afro-Asiatic}
\def\langnames@fams@wals@gox{Atlantic-Congo}
\def\langnames@fams@wals@goy{Atlantic-Congo}
\def\langnames@fams@wals@goz{Indo-European}
\def\langnames@fams@wals@gpa{Atlantic-Congo}
\def\langnames@fams@wals@gpe{Indo-European}
\def\langnames@fams@wals@gpn{Isolate}
\def\langnames@fams@wals@gqa{Afro-Asiatic}
\def\langnames@fams@wals@gqi{Sino-Tibetan}
\def\langnames@fams@wals@gqr{Central Sudanic}
\def\langnames@fams@wals@gqu{Tai-Kadai}
\def\langnames@fams@wals@gra{Indo-European}
\def\langnames@fams@wals@grc{Indo-European}
\def\langnames@fams@wals@grd{Afro-Asiatic}
\def\langnames@fams@wals@grg{Nuclear Trans New Guinea}
\def\langnames@fams@wals@grh{Atlantic-Congo}
\def\langnames@fams@wals@gri{Austronesian}
\def\langnames@fams@wals@grj{Kru}
\def\langnames@fams@wals@grm{Austronesian}
\def\langnames@fams@wals@gro{Sino-Tibetan}
\def\langnames@fams@wals@grq{Lower Sepik-Ramu}
\def\langnames@fams@wals@grr{Afro-Asiatic}
\def\langnames@fams@wals@grs{Nimboranic}
\def\langnames@fams@wals@grt{Sino-Tibetan}
\def\langnames@fams@wals@gru{Afro-Asiatic}
\def\langnames@fams@wals@grv{Kru}
\def\langnames@fams@wals@grw{Austronesian}
\def\langnames@fams@wals@grx{Isolate}
\def\langnames@fams@wals@gry{Kru}
\def\langnames@fams@wals@grz{Austronesian}
\def\langnames@fams@wals@gse{Sign Language}
\def\langnames@fams@wals@gsg{Sign Language}
\def\langnames@fams@wals@gsl{Atlantic-Congo}
\def\langnames@fams@wals@gsm{Sign Language}
\def\langnames@fams@wals@gsn{Nuclear Trans New Guinea}
\def\langnames@fams@wals@gso{Atlantic-Congo}
\def\langnames@fams@wals@gsp{Nuclear Trans New Guinea}
\def\langnames@fams@wals@gss{Sign Language}
\def\langnames@fams@wals@gsw{Indo-European}
\def\langnames@fams@wals@gta{Isolate}
\def\langnames@fams@wals@gua{Atlantic-Congo}
\def\langnames@fams@wals@guc{Arawakan}
\def\langnames@fams@wals@gud{Kru}
\def\langnames@fams@wals@gue{Pama-Nyungan}
\def\langnames@fams@wals@guf{Pama-Nyungan}
\def\langnames@fams@wals@gug{Tupian}
\def\langnames@fams@wals@guh{Guahiboan}
\def\langnames@fams@wals@gui{Tupian}
\def\langnames@fams@wals@guj{Indo-European}
\def\langnames@fams@wals@guk{Gumuz}
\def\langnames@fams@wals@gul{Indo-European}
\def\langnames@fams@wals@gum{Barbacoan}
\def\langnames@fams@wals@gun{Tupian}
\def\langnames@fams@wals@guo{Guahiboan}
\def\langnames@fams@wals@gup{Gunwinyguan}
\def\langnames@fams@wals@guq{Tupian}
\def\langnames@fams@wals@gur{Atlantic-Congo}
\def\langnames@fams@wals@gus{Sign Language}
\def\langnames@fams@wals@gut{Chibchan}
\def\langnames@fams@wals@guu{Yanomamic}
\def\langnames@fams@wals@guw{Atlantic-Congo}
\def\langnames@fams@wals@gux{Atlantic-Congo}
\def\langnames@fams@wals@guz{Atlantic-Congo}
\def\langnames@fams@wals@gva{Lengua-Mascoy}
\def\langnames@fams@wals@gvc{Tucanoan}
\def\langnames@fams@wals@gve{Austronesian}
\def\langnames@fams@wals@gvf{Nuclear Trans New Guinea}
\def\langnames@fams@wals@gvj{Tupian}
\def\langnames@fams@wals@gvl{Central Sudanic}
\def\langnames@fams@wals@gvm{Atlantic-Congo}
\def\langnames@fams@wals@gvn{Pama-Nyungan}
\def\langnames@fams@wals@gvo{Tupian}
\def\langnames@fams@wals@gvp{Nuclear-Macro-Je}
\def\langnames@fams@wals@gvr{Sino-Tibetan}
\def\langnames@fams@wals@gvs{Austronesian}
\def\langnames@fams@wals@gvy{Pama-Nyungan}
\def\langnames@fams@wals@gwa{Atlantic-Congo}
\def\langnames@fams@wals@gwb{Atlantic-Congo}
\def\langnames@fams@wals@gwc{Indo-European}
\def\langnames@fams@wals@gwd{Afro-Asiatic}
\def\langnames@fams@wals@gwe{Atlantic-Congo}
\def\langnames@fams@wals@gwf{Indo-European}
\def\langnames@fams@wals@gwg{Atlantic-Congo}
\def\langnames@fams@wals@gwi{Athabaskan-Eyak-Tlingit}
\def\langnames@fams@wals@gwj{Khoe-Kwadi}
\def\langnames@fams@wals@gwn{Afro-Asiatic}
\def\langnames@fams@wals@gwr{Atlantic-Congo}
\def\langnames@fams@wals@gwt{Indo-European}
\def\langnames@fams@wals@gwu{Pama-Nyungan}
\def\langnames@fams@wals@gww{Worrorran}
\def\langnames@fams@wals@gwx{Atlantic-Congo}
\def\langnames@fams@wals@gxx{Kru}
\def\langnames@fams@wals@gya{Atlantic-Congo}
\def\langnames@fams@wals@gyb{Nuclear Trans New Guinea}
\def\langnames@fams@wals@gyd{Tangkic}
\def\langnames@fams@wals@gye{Atlantic-Congo}
\def\langnames@fams@wals@gyf{Pama-Nyungan}
\def\langnames@fams@wals@gyg{Atlantic-Congo}
\def\langnames@fams@wals@gyi{Atlantic-Congo}
\def\langnames@fams@wals@gyl{South Omotic}
\def\langnames@fams@wals@gym{Chibchan}
\def\langnames@fams@wals@gyn{Indo-European}
\def\langnames@fams@wals@gyr{Tupian}
\def\langnames@fams@wals@gyy{Pama-Nyungan}
\def\langnames@fams@wals@gyz{Afro-Asiatic}
\def\langnames@fams@wals@gza{Blue Nile Mao}
\def\langnames@fams@wals@gzi{Indo-European}
\def\langnames@fams@wals@gzn{Austronesian}
\def\langnames@fams@wals@haa{Athabaskan-Eyak-Tlingit}
\def\langnames@fams@wals@hab{Sign Language}
\def\langnames@fams@wals@hac{Indo-European}
\def\langnames@fams@wals@had{Hatam-Mansim}
\def\langnames@fams@wals@hae{Afro-Asiatic}
\def\langnames@fams@wals@haf{Sign Language}
\def\langnames@fams@wals@hag{Atlantic-Congo}
\def\langnames@fams@wals@hah{Austronesian}
\def\langnames@fams@wals@haj{Indo-European}
\def\langnames@fams@wals@hak{Sino-Tibetan}
\def\langnames@fams@wals@hal{Austroasiatic}
\def\langnames@fams@wals@ham{Sepik}
\def\langnames@fams@wals@han{Atlantic-Congo}
\def\langnames@fams@wals@hao{Austronesian}
\def\langnames@fams@wals@hap{Nuclear Trans New Guinea}
\def\langnames@fams@wals@haq{Atlantic-Congo}
\def\langnames@fams@wals@har{Afro-Asiatic}
\def\langnames@fams@wals@has{Wakashan}
\def\langnames@fams@wals@hat{Indo-European}
\def\langnames@fams@wals@hau{Afro-Asiatic}
\def\langnames@fams@wals@hav{Atlantic-Congo}
\def\langnames@fams@wals@haw{Austronesian}
\def\langnames@fams@wals@hax{Haida}
\def\langnames@fams@wals@hay{Atlantic-Congo}
\def\langnames@fams@wals@haz{Indo-European}
\def\langnames@fams@wals@hba{Atlantic-Congo}
\def\langnames@fams@wals@hbb{Afro-Asiatic}
\def\langnames@fams@wals@hbn{Heibanic}
\def\langnames@fams@wals@hbo{Afro-Asiatic}
\def\langnames@fams@wals@hbs{Indo-European}
\def\langnames@fams@wals@hbu{Austronesian}
\def\langnames@fams@wals@hca{Indo-European}
\def\langnames@fams@wals@hch{Uto-Aztecan}
\def\langnames@fams@wals@hdn{Haida}
\def\langnames@fams@wals@hds{Sign Language}
\def\langnames@fams@wals@hdy{Afro-Asiatic}
\def\langnames@fams@wals@hea{Hmong-Mien}
\def\langnames@fams@wals@heb{Afro-Asiatic}
\def\langnames@fams@wals@hed{Afro-Asiatic}
\def\langnames@fams@wals@heg{Austronesian}
\def\langnames@fams@wals@heh{Atlantic-Congo}
\def\langnames@fams@wals@hei{Wakashan}
\def\langnames@fams@wals@hem{Atlantic-Congo}
\def\langnames@fams@wals@her{Atlantic-Congo}
\def\langnames@fams@wals@hgm{Khoe-Kwadi}
\def\langnames@fams@wals@hgw{Austronesian}
\def\langnames@fams@wals@hhi{Anim}
\def\langnames@fams@wals@hhr{Atlantic-Congo}
\def\langnames@fams@wals@hhy{Anim}
\def\langnames@fams@wals@hia{Afro-Asiatic}
\def\langnames@fams@wals@hib{Hibito-Cholon}
\def\langnames@fams@wals@hid{Siouan}
\def\langnames@fams@wals@hif{Indo-European}
\def\langnames@fams@wals@hig{Afro-Asiatic}
\def\langnames@fams@wals@hih{Nuclear Trans New Guinea}
\def\langnames@fams@wals@hii{Indo-European}
\def\langnames@fams@wals@hij{Atlantic-Congo}
\def\langnames@fams@wals@hik{Austronesian}
\def\langnames@fams@wals@hil{Austronesian}
\def\langnames@fams@wals@hin{Indo-European}
\def\langnames@fams@wals@hio{Khoe-Kwadi}
\def\langnames@fams@wals@hir{Unattested}
\def\langnames@fams@wals@hit{Indo-European}
\def\langnames@fams@wals@hiw{Austronesian}
\def\langnames@fams@wals@hix{Cariban}
\def\langnames@fams@wals@hji{Austronesian}
\def\langnames@fams@wals@hka{Atlantic-Congo}
\def\langnames@fams@wals@hke{Atlantic-Congo}
\def\langnames@fams@wals@hkh{Indo-European}
\def\langnames@fams@wals@hkk{Nuclear Trans New Guinea}
\def\langnames@fams@wals@hkn{Austroasiatic}
\def\langnames@fams@wals@hks{Sign Language}
\def\langnames@fams@wals@hla{Austronesian}
\def\langnames@fams@wals@hlb{Indo-European}
\def\langnames@fams@wals@hld{Austroasiatic}
\def\langnames@fams@wals@hle{Sino-Tibetan}
\def\langnames@fams@wals@hlt{Sino-Tibetan}
\def\langnames@fams@wals@hlu{Indo-European}
\def\langnames@fams@wals@hma{Hmong-Mien}
\def\langnames@fams@wals@hmb{Songhay}
\def\langnames@fams@wals@hmc{Hmong-Mien}
\def\langnames@fams@wals@hmd{Hmong-Mien}
\def\langnames@fams@wals@hme{Hmong-Mien}
\def\langnames@fams@wals@hmf{Hmong-Mien}
\def\langnames@fams@wals@hmg{Hmong-Mien}
\def\langnames@fams@wals@hmh{Hmong-Mien}
\def\langnames@fams@wals@hmi{Hmong-Mien}
\def\langnames@fams@wals@hmj{Hmong-Mien}
\def\langnames@fams@wals@hml{Hmong-Mien}
\def\langnames@fams@wals@hmm{Hmong-Mien}
\def\langnames@fams@wals@hmo{Pidgin}
\def\langnames@fams@wals@hmp{Hmong-Mien}
\def\langnames@fams@wals@hmq{Hmong-Mien}
\def\langnames@fams@wals@hmr{Sino-Tibetan}
\def\langnames@fams@wals@hms{Hmong-Mien}
\def\langnames@fams@wals@hmt{Angan}
\def\langnames@fams@wals@hmu{Timor-Alor-Pantar}
\def\langnames@fams@wals@hmv{Hmong-Mien}
\def\langnames@fams@wals@hmw{Hmong-Mien}
\def\langnames@fams@wals@hmy{Hmong-Mien}
\def\langnames@fams@wals@hmz{Hmong-Mien}
\def\langnames@fams@wals@hna{Afro-Asiatic}
\def\langnames@fams@wals@hnd{Indo-European}
\def\langnames@fams@wals@hne{Indo-European}
\def\langnames@fams@wals@hng{Atlantic-Congo}
\def\langnames@fams@wals@hnh{Khoe-Kwadi}
\def\langnames@fams@wals@hni{Sino-Tibetan}
\def\langnames@fams@wals@hnj{Hmong-Mien}
\def\langnames@fams@wals@hnn{Austronesian}
\def\langnames@fams@wals@hno{Indo-European}
\def\langnames@fams@wals@hns{Indo-European}
\def\langnames@fams@wals@hnu{Austroasiatic}
\def\langnames@fams@wals@hoa{Austronesian}
\def\langnames@fams@wals@hob{Austronesian}
\def\langnames@fams@wals@hoc{Austroasiatic}
\def\langnames@fams@wals@hod{Afro-Asiatic}
\def\langnames@fams@wals@hoe{Atlantic-Congo}
\def\langnames@fams@wals@hoh{Afro-Asiatic}
\def\langnames@fams@wals@hoi{Athabaskan-Eyak-Tlingit}
\def\langnames@fams@wals@hoj{Indo-European}
\def\langnames@fams@wals@hol{Atlantic-Congo}
\def\langnames@fams@wals@hom{Atlantic-Congo}
\def\langnames@fams@wals@hoo{Atlantic-Congo}
\def\langnames@fams@wals@hop{Uto-Aztecan}
\def\langnames@fams@wals@hor{Central Sudanic}
\def\langnames@fams@wals@hos{Sign Language}
\def\langnames@fams@wals@hot{Austronesian}
\def\langnames@fams@wals@hov{Austronesian}
\def\langnames@fams@wals@how{Sino-Tibetan}
\def\langnames@fams@wals@hoy{Dravidian}
\def\langnames@fams@wals@hoz{Blue Nile Mao}
\def\langnames@fams@wals@hpo{Sino-Tibetan}
\def\langnames@fams@wals@hps{Sign Language}
\def\langnames@fams@wals@hra{Sino-Tibetan}
\def\langnames@fams@wals@hre{Austroasiatic}
\def\langnames@fams@wals@hrk{Austronesian}
\def\langnames@fams@wals@hrm{Hmong-Mien}
\def\langnames@fams@wals@hro{Austronesian}
\def\langnames@fams@wals@hrt{Afro-Asiatic}
\def\langnames@fams@wals@hru{Isolate}
\def\langnames@fams@wals@hrx{Indo-European}
\def\langnames@fams@wals@hrz{Indo-European}
\def\langnames@fams@wals@hsb{Indo-European}
\def\langnames@fams@wals@hsh{Sign Language}
\def\langnames@fams@wals@hsl{Sign Language}
\def\langnames@fams@wals@hsn{Sino-Tibetan}
\def\langnames@fams@wals@hss{Afro-Asiatic}
\def\langnames@fams@wals@hti{Austronesian}
\def\langnames@fams@wals@hto{Huitotoan}
\def\langnames@fams@wals@hts{Isolate}
\def\langnames@fams@wals@htu{Austronesian}
\def\langnames@fams@wals@hub{Chicham}
\def\langnames@fams@wals@huc{Kxa}
\def\langnames@fams@wals@hud{Austronesian}
\def\langnames@fams@wals@hue{Huavean}
\def\langnames@fams@wals@huf{Kwalean}
\def\langnames@fams@wals@hug{Harakmbut}
\def\langnames@fams@wals@huh{Araucanian}
\def\langnames@fams@wals@hui{Nuclear Trans New Guinea}
\def\langnames@fams@wals@huj{Hmong-Mien}
\def\langnames@fams@wals@huk{Austronesian}
\def\langnames@fams@wals@hul{Austronesian}
\def\langnames@fams@wals@hum{Atlantic-Congo}
\def\langnames@fams@wals@hun{Uralic}
\def\langnames@fams@wals@huo{Austroasiatic}
\def\langnames@fams@wals@hup{Athabaskan-Eyak-Tlingit}
\def\langnames@fams@wals@huq{Austronesian}
\def\langnames@fams@wals@hur{Salishan}
\def\langnames@fams@wals@hus{Mayan}
\def\langnames@fams@wals@hut{Sino-Tibetan}
\def\langnames@fams@wals@huu{Huitotoan}
\def\langnames@fams@wals@huv{Huavean}
\def\langnames@fams@wals@huw{Austronesian}
\def\langnames@fams@wals@hux{Huitotoan}
\def\langnames@fams@wals@huy{Afro-Asiatic}
\def\langnames@fams@wals@huz{Nakh-Daghestanian}
\def\langnames@fams@wals@hvc{Unclassifiable}
\def\langnames@fams@wals@hve{Huavean}
\def\langnames@fams@wals@hvk{Austronesian}
\def\langnames@fams@wals@hvn{Austronesian}
\def\langnames@fams@wals@hvv{Huavean}
\def\langnames@fams@wals@hwa{Kru}
\def\langnames@fams@wals@hwc{Indo-European}
\def\langnames@fams@wals@hwo{Afro-Asiatic}
\def\langnames@fams@wals@hya{Afro-Asiatic}
\def\langnames@fams@wals@hye{Indo-European}
\def\langnames@fams@wals@hyw{Indo-European}
\def\langnames@fams@wals@iai{Austronesian}
\def\langnames@fams@wals@ian{Ndu}
\def\langnames@fams@wals@iar{Isolate}
\def\langnames@fams@wals@iba{Austronesian}
\def\langnames@fams@wals@ibb{Atlantic-Congo}
\def\langnames@fams@wals@ibd{Iwaidjan Proper}
\def\langnames@fams@wals@ibe{Atlantic-Congo}
\def\langnames@fams@wals@ibg{Austronesian}
\def\langnames@fams@wals@ibh{Austronesian}
\def\langnames@fams@wals@ibl{Austronesian}
\def\langnames@fams@wals@ibm{Atlantic-Congo}
\def\langnames@fams@wals@ibn{Atlantic-Congo}
\def\langnames@fams@wals@ibo{Atlantic-Congo}
\def\langnames@fams@wals@ibr{Atlantic-Congo}
\def\langnames@fams@wals@ibu{North Halmahera}
\def\langnames@fams@wals@iby{Ijoid}
\def\langnames@fams@wals@ica{Atlantic-Congo}
\def\langnames@fams@wals@ich{Atlantic-Congo}
\def\langnames@fams@wals@icl{Sign Language}
\def\langnames@fams@wals@icr{Indo-European}
\def\langnames@fams@wals@ida{Atlantic-Congo}
\def\langnames@fams@wals@idb{Indo-European}
\def\langnames@fams@wals@idc{Atlantic-Congo}
\def\langnames@fams@wals@idd{Atlantic-Congo}
\def\langnames@fams@wals@ide{Atlantic-Congo}
\def\langnames@fams@wals@idi{Pahoturi}
\def\langnames@fams@wals@ido{Artificial Language}
\def\langnames@fams@wals@idr{Atlantic-Congo}
\def\langnames@fams@wals@idt{Austronesian}
\def\langnames@fams@wals@idu{Atlantic-Congo}
\def\langnames@fams@wals@ifa{Austronesian}
\def\langnames@fams@wals@ifb{Austronesian}
\def\langnames@fams@wals@ife{Atlantic-Congo}
\def\langnames@fams@wals@iff{Austronesian}
\def\langnames@fams@wals@ifk{Austronesian}
\def\langnames@fams@wals@ifm{Atlantic-Congo}
\def\langnames@fams@wals@ifu{Austronesian}
\def\langnames@fams@wals@ify{Austronesian}
\def\langnames@fams@wals@igb{Atlantic-Congo}
\def\langnames@fams@wals@ige{Atlantic-Congo}
\def\langnames@fams@wals@igg{Lower Sepik-Ramu}
\def\langnames@fams@wals@igl{Atlantic-Congo}
\def\langnames@fams@wals@igm{Lower Sepik-Ramu}
\def\langnames@fams@wals@ign{Arawakan}
\def\langnames@fams@wals@igo{Nuclear Trans New Guinea}
\def\langnames@fams@wals@igs{Artificial Language}
\def\langnames@fams@wals@igw{Atlantic-Congo}
\def\langnames@fams@wals@ihb{Pidgin}
\def\langnames@fams@wals@ihp{West Bomberai}
\def\langnames@fams@wals@ihw{Pama-Nyungan}
\def\langnames@fams@wals@iii{Sino-Tibetan}
\def\langnames@fams@wals@iin{Pama-Nyungan}
\def\langnames@fams@wals@ijc{Ijoid}
\def\langnames@fams@wals@ije{Ijoid}
\def\langnames@fams@wals@ijj{Atlantic-Congo}
\def\langnames@fams@wals@ijn{Ijoid}
\def\langnames@fams@wals@ijs{Ijoid}
\def\langnames@fams@wals@ike{Eskimo-Aleut}
\def\langnames@fams@wals@iki{Atlantic-Congo}
\def\langnames@fams@wals@ikk{Atlantic-Congo}
\def\langnames@fams@wals@ikl{Atlantic-Congo}
\def\langnames@fams@wals@iko{Atlantic-Congo}
\def\langnames@fams@wals@ikp{Atlantic-Congo}
\def\langnames@fams@wals@ikr{Pama-Nyungan}
\def\langnames@fams@wals@iks{Sign Language}
\def\langnames@fams@wals@ikt{Eskimo-Aleut}
\def\langnames@fams@wals@ikv{Atlantic-Congo}
\def\langnames@fams@wals@ikw{Atlantic-Congo}
\def\langnames@fams@wals@ikx{Kuliak}
\def\langnames@fams@wals@ikz{Atlantic-Congo}
\def\langnames@fams@wals@ila{Austronesian}
\def\langnames@fams@wals@ilb{Atlantic-Congo}
\def\langnames@fams@wals@ile{Artificial Language}
\def\langnames@fams@wals@ilg{Iwaidjan Proper}
\def\langnames@fams@wals@ili{Turkic}
\def\langnames@fams@wals@ilk{Austronesian}
\def\langnames@fams@wals@ill{Austronesian}
\def\langnames@fams@wals@ilo{Austronesian}
\def\langnames@fams@wals@ils{Sign Language}
\def\langnames@fams@wals@ilu{Austronesian}
\def\langnames@fams@wals@ilv{Atlantic-Congo}
\def\langnames@fams@wals@ima{Dravidian}
\def\langnames@fams@wals@imi{Nuclear Trans New Guinea}
\def\langnames@fams@wals@iml{Coosan}
\def\langnames@fams@wals@imn{Border}
\def\langnames@fams@wals@imo{Nuclear Trans New Guinea}
\def\langnames@fams@wals@imr{Austronesian}
\def\langnames@fams@wals@imy{Indo-European}
\def\langnames@fams@wals@ina{Artificial Language}
\def\langnames@fams@wals@inb{Quechuan}
\def\langnames@fams@wals@ind{Austronesian}
\def\langnames@fams@wals@ing{Athabaskan-Eyak-Tlingit}
\def\langnames@fams@wals@inh{Nakh-Daghestanian}
\def\langnames@fams@wals@inl{Sign Language}
\def\langnames@fams@wals@inm{Afro-Asiatic}
\def\langnames@fams@wals@inn{Austronesian}
\def\langnames@fams@wals@ino{Nuclear Trans New Guinea}
\def\langnames@fams@wals@inp{Arawakan}
\def\langnames@fams@wals@ins{Sign Language}
\def\langnames@fams@wals@int{Sino-Tibetan}
\def\langnames@fams@wals@inz{Chumashan}
\def\langnames@fams@wals@ior{Afro-Asiatic}
\def\langnames@fams@wals@iou{Nuclear Trans New Guinea}
\def\langnames@fams@wals@iow{Siouan}
\def\langnames@fams@wals@ipi{Nuclear Trans New Guinea}
\def\langnames@fams@wals@ipo{Anim}
\def\langnames@fams@wals@iqu{Zaparoan}
\def\langnames@fams@wals@ire{Austronesian}
\def\langnames@fams@wals@irh{Austronesian}
\def\langnames@fams@wals@iri{Atlantic-Congo}
\def\langnames@fams@wals@irk{Afro-Asiatic}
\def\langnames@fams@wals@irn{Isolate}
\def\langnames@fams@wals@iru{Dravidian}
\def\langnames@fams@wals@irx{Nuclear Trans New Guinea}
\def\langnames@fams@wals@iry{Austronesian}
\def\langnames@fams@wals@isa{Nuclear Trans New Guinea}
\def\langnames@fams@wals@isc{Pano-Tacanan}
\def\langnames@fams@wals@isd{Austronesian}
\def\langnames@fams@wals@ise{Sign Language}
\def\langnames@fams@wals@isg{Sign Language}
\def\langnames@fams@wals@ish{Atlantic-Congo}
\def\langnames@fams@wals@isi{Atlantic-Congo}
\def\langnames@fams@wals@isk{Indo-European}
\def\langnames@fams@wals@isl{Indo-European}
\def\langnames@fams@wals@ism{Austronesian}
\def\langnames@fams@wals@isn{Atlantic-Congo}
\def\langnames@fams@wals@iso{Atlantic-Congo}
\def\langnames@fams@wals@isr{Sign Language}
\def\langnames@fams@wals@ist{Indo-European}
\def\langnames@fams@wals@isu{Atlantic-Congo}
\def\langnames@fams@wals@ita{Indo-European}
\def\langnames@fams@wals@itb{Austronesian}
\def\langnames@fams@wals@itd{Austronesian}
\def\langnames@fams@wals@ite{Chapacuran}
\def\langnames@fams@wals@iti{Austronesian}
\def\langnames@fams@wals@itk{Indo-European}
\def\langnames@fams@wals@itl{Chukotko-Kamchatkan}
\def\langnames@fams@wals@itm{Atlantic-Congo}
\def\langnames@fams@wals@ito{Isolate}
\def\langnames@fams@wals@itr{Left May}
\def\langnames@fams@wals@its{Atlantic-Congo}
\def\langnames@fams@wals@itt{Austronesian}
\def\langnames@fams@wals@itv{Austronesian}
\def\langnames@fams@wals@itw{Atlantic-Congo}
\def\langnames@fams@wals@itx{Tor-Orya}
\def\langnames@fams@wals@ity{Austronesian}
\def\langnames@fams@wals@itz{Mayan}
\def\langnames@fams@wals@ium{Hmong-Mien}
\def\langnames@fams@wals@ivb{Austronesian}
\def\langnames@fams@wals@ivv{Austronesian}
\def\langnames@fams@wals@iwk{Austronesian}
\def\langnames@fams@wals@iwm{Sepik}
\def\langnames@fams@wals@iwo{Nuclear Trans New Guinea}
\def\langnames@fams@wals@iws{Sepik}
\def\langnames@fams@wals@ixc{Otomanguean}
\def\langnames@fams@wals@ixl{Mayan}
\def\langnames@fams@wals@iya{Atlantic-Congo}
\def\langnames@fams@wals@iyo{Atlantic-Congo}
\def\langnames@fams@wals@iyx{Atlantic-Congo}
\def\langnames@fams@wals@izh{Uralic}
\def\langnames@fams@wals@izi{Atlantic-Congo}
\def\langnames@fams@wals@izr{Atlantic-Congo}
\def\langnames@fams@wals@jaa{Arawan}
\def\langnames@fams@wals@jab{Atlantic-Congo}
\def\langnames@fams@wals@jac{Mayan}
\def\langnames@fams@wals@jad{Mande}
\def\langnames@fams@wals@jae{Austronesian}
\def\langnames@fams@wals@jaf{Afro-Asiatic}
\def\langnames@fams@wals@jah{Austroasiatic}
\def\langnames@fams@wals@jaj{Austronesian}
\def\langnames@fams@wals@jak{Austronesian}
\def\langnames@fams@wals@jal{Austronesian}
\def\langnames@fams@wals@jam{Indo-European}
\def\langnames@fams@wals@jao{Pama-Nyungan}
\def\langnames@fams@wals@jaq{Anim}
\def\langnames@fams@wals@jar{Atlantic-Congo}
\def\langnames@fams@wals@jas{Austronesian}
\def\langnames@fams@wals@jat{Indo-European}
\def\langnames@fams@wals@jau{Austronesian}
\def\langnames@fams@wals@jav{Austronesian}
\def\langnames@fams@wals@jax{Austronesian}
\def\langnames@fams@wals@jay{Pama-Nyungan}
\def\langnames@fams@wals@jaz{Austronesian}
\def\langnames@fams@wals@jbi{Pama-Nyungan}
\def\langnames@fams@wals@jbj{South Bird's Head Family}
\def\langnames@fams@wals@jbk{Turama-Kikori}
\def\langnames@fams@wals@jbm{Atlantic-Congo}
\def\langnames@fams@wals@jbn{Afro-Asiatic}
\def\langnames@fams@wals@jbo{Artificial Language}
\def\langnames@fams@wals@jbr{Tor-Orya}
\def\langnames@fams@wals@jbt{Nuclear-Macro-Je}
\def\langnames@fams@wals@jbu{Atlantic-Congo}
\def\langnames@fams@wals@jcs{Sign Language}
\def\langnames@fams@wals@jct{Turkic}
\def\langnames@fams@wals@jda{Sino-Tibetan}
\def\langnames@fams@wals@jdg{Indo-European}
\def\langnames@fams@wals@jdt{Indo-European}
\def\langnames@fams@wals@jeb{Cahuapanan}
\def\langnames@fams@wals@jee{Sino-Tibetan}
\def\langnames@fams@wals@jeh{Austroasiatic}
\def\langnames@fams@wals@jei{Yam}
\def\langnames@fams@wals@jek{Mande}
\def\langnames@fams@wals@jel{Bulaka River}
\def\langnames@fams@wals@jen{Atlantic-Congo}
\def\langnames@fams@wals@jer{Atlantic-Congo}
\def\langnames@fams@wals@jet{Border}
\def\langnames@fams@wals@jeu{Afro-Asiatic}
\def\langnames@fams@wals@jgb{Atlantic-Congo}
\def\langnames@fams@wals@jgo{Atlantic-Congo}
\def\langnames@fams@wals@jhi{Austroasiatic}
\def\langnames@fams@wals@jhs{Sign Language}
\def\langnames@fams@wals@jia{Afro-Asiatic}
\def\langnames@fams@wals@jib{Atlantic-Congo}
\def\langnames@fams@wals@jic{Jicaquean}
\def\langnames@fams@wals@jid{Atlantic-Congo}
\def\langnames@fams@wals@jie{Afro-Asiatic}
\def\langnames@fams@wals@jig{Mirndi}
\def\langnames@fams@wals@jih{Sino-Tibetan}
\def\langnames@fams@wals@jii{Afro-Asiatic}
\def\langnames@fams@wals@jil{Nuclear Trans New Guinea}
\def\langnames@fams@wals@jim{Afro-Asiatic}
\def\langnames@fams@wals@jio{Tai-Kadai}
\def\langnames@fams@wals@jiq{Sino-Tibetan}
\def\langnames@fams@wals@jit{Atlantic-Congo}
\def\langnames@fams@wals@jiu{Sino-Tibetan}
\def\langnames@fams@wals@jiv{Chicham}
\def\langnames@fams@wals@jiy{Sino-Tibetan}
\def\langnames@fams@wals@jje{Koreanic}
\def\langnames@fams@wals@jka{Timor-Alor-Pantar}
\def\langnames@fams@wals@jkm{Sino-Tibetan}
\def\langnames@fams@wals@jko{East Strickland}
\def\langnames@fams@wals@jkp{Sino-Tibetan}
\def\langnames@fams@wals@jkr{Sino-Tibetan}
\def\langnames@fams@wals@jks{Sign Language}
\def\langnames@fams@wals@jku{Atlantic-Congo}
\def\langnames@fams@wals@jle{Narrow Talodi}
\def\langnames@fams@wals@jls{Sign Language}
\def\langnames@fams@wals@jma{Dagan}
\def\langnames@fams@wals@jmb{Afro-Asiatic}
\def\langnames@fams@wals@jmc{Atlantic-Congo}
\def\langnames@fams@wals@jmd{Austronesian}
\def\langnames@fams@wals@jmi{Afro-Asiatic}
\def\langnames@fams@wals@jml{Indo-European}
\def\langnames@fams@wals@jmn{Sino-Tibetan}
\def\langnames@fams@wals@jmr{Atlantic-Congo}
\def\langnames@fams@wals@jms{Atlantic-Congo}
\def\langnames@fams@wals@jmw{Turama-Kikori}
\def\langnames@fams@wals@jmx{Otomanguean}
\def\langnames@fams@wals@jna{Sino-Tibetan}
\def\langnames@fams@wals@jnd{Indo-European}
\def\langnames@fams@wals@jng{Yangmanic}
\def\langnames@fams@wals@jni{Atlantic-Congo}
\def\langnames@fams@wals@jnj{Ta-Ne-Omotic}
\def\langnames@fams@wals@jnl{Sino-Tibetan}
\def\langnames@fams@wals@jns{Indo-European}
\def\langnames@fams@wals@job{Atlantic-Congo}
\def\langnames@fams@wals@jod{Mande}
\def\langnames@fams@wals@jor{Tupian}
\def\langnames@fams@wals@jos{Sign Language}
\def\langnames@fams@wals@jow{Mande}
\def\langnames@fams@wals@jpn{Japonic}
\def\langnames@fams@wals@jpr{Indo-European}
\def\langnames@fams@wals@jqr{Aymaran}
\def\langnames@fams@wals@jra{Austronesian}
\def\langnames@fams@wals@jrr{Atlantic-Congo}
\def\langnames@fams@wals@jrt{Afro-Asiatic}
\def\langnames@fams@wals@jru{Cariban}
\def\langnames@fams@wals@jsl{Sign Language}
\def\langnames@fams@wals@jua{Tupian}
\def\langnames@fams@wals@jub{Atlantic-Congo}
\def\langnames@fams@wals@juc{Tungusic}
\def\langnames@fams@wals@jud{Mande}
\def\langnames@fams@wals@juh{Atlantic-Congo}
\def\langnames@fams@wals@jui{Pama-Nyungan}
\def\langnames@fams@wals@juk{Atlantic-Congo}
\def\langnames@fams@wals@jul{Sino-Tibetan}
\def\langnames@fams@wals@jum{Nilotic}
\def\langnames@fams@wals@jun{Austroasiatic}
\def\langnames@fams@wals@juo{Atlantic-Congo}
\def\langnames@fams@wals@jup{Naduhup}
\def\langnames@fams@wals@jur{Tupian}
\def\langnames@fams@wals@jus{Sign Language}
\def\langnames@fams@wals@jut{Indo-European}
\def\langnames@fams@wals@juu{Afro-Asiatic}
\def\langnames@fams@wals@juw{Atlantic-Congo}
\def\langnames@fams@wals@juy{Austroasiatic}
\def\langnames@fams@wals@jvd{Indo-European}
\def\langnames@fams@wals@jvn{Austronesian}
\def\langnames@fams@wals@jwi{Atlantic-Congo}
\def\langnames@fams@wals@jye{Afro-Asiatic}
\def\langnames@fams@wals@jyy{Central Sudanic}
\def\langnames@fams@wals@kaa{Turkic}
\def\langnames@fams@wals@kab{Afro-Asiatic}
\def\langnames@fams@wals@kac{Sino-Tibetan}
\def\langnames@fams@wals@kad{Atlantic-Congo}
\def\langnames@fams@wals@kae{Austronesian}
\def\langnames@fams@wals@kaf{Sino-Tibetan}
\def\langnames@fams@wals@kag{Austronesian}
\def\langnames@fams@wals@kah{Central Sudanic}
\def\langnames@fams@wals@kai{Afro-Asiatic}
\def\langnames@fams@wals@kaj{Atlantic-Congo}
\def\langnames@fams@wals@kak{Austronesian}
\def\langnames@fams@wals@kal{Eskimo-Aleut}
\def\langnames@fams@wals@kam{Atlantic-Congo}
\def\langnames@fams@wals@kan{Dravidian}
\def\langnames@fams@wals@kao{Mande}
\def\langnames@fams@wals@kap{Nakh-Daghestanian}
\def\langnames@fams@wals@kaq{Pano-Tacanan}
\def\langnames@fams@wals@kas{Indo-European}
\def\langnames@fams@wals@kat{Kartvelian}
\def\langnames@fams@wals@kaw{Austronesian}
\def\langnames@fams@wals@kax{North Halmahera}
\def\langnames@fams@wals@kay{Tupian}
\def\langnames@fams@wals@kaz{Turkic}
\def\langnames@fams@wals@kba{Pama-Nyungan}
\def\langnames@fams@wals@kbb{Cariban}
\def\langnames@fams@wals@kbc{Guaicuruan}
\def\langnames@fams@wals@kbd{Abkhaz-Adyge}
\def\langnames@fams@wals@kbe{Pama-Nyungan}
\def\langnames@fams@wals@kbg{Sino-Tibetan}
\def\langnames@fams@wals@kbh{Isolate}
\def\langnames@fams@wals@kbi{Austronesian}
\def\langnames@fams@wals@kbj{Atlantic-Congo}
\def\langnames@fams@wals@kbk{Koiarian}
\def\langnames@fams@wals@kbl{Saharan}
\def\langnames@fams@wals@kbm{Austronesian}
\def\langnames@fams@wals@kbn{Atlantic-Congo}
\def\langnames@fams@wals@kbo{Central Sudanic}
\def\langnames@fams@wals@kbp{Atlantic-Congo}
\def\langnames@fams@wals@kbq{Nuclear Trans New Guinea}
\def\langnames@fams@wals@kbr{Ta-Ne-Omotic}
\def\langnames@fams@wals@kbs{Atlantic-Congo}
\def\langnames@fams@wals@kbt{Austronesian}
\def\langnames@fams@wals@kbu{Indo-European}
\def\langnames@fams@wals@kbv{Senagi}
\def\langnames@fams@wals@kbw{Austronesian}
\def\langnames@fams@wals@kbx{Keram}
\def\langnames@fams@wals@kby{Saharan}
\def\langnames@fams@wals@kbz{Afro-Asiatic}
\def\langnames@fams@wals@kca{Uralic}
\def\langnames@fams@wals@kcb{Angan}
\def\langnames@fams@wals@kcc{Atlantic-Congo}
\def\langnames@fams@wals@kcd{Yam}
\def\langnames@fams@wals@kce{Unattested}
\def\langnames@fams@wals@kcf{Atlantic-Congo}
\def\langnames@fams@wals@kcg{Atlantic-Congo}
\def\langnames@fams@wals@kch{Unattested}
\def\langnames@fams@wals@kci{Atlantic-Congo}
\def\langnames@fams@wals@kcj{Atlantic-Congo}
\def\langnames@fams@wals@kck{Atlantic-Congo}
\def\langnames@fams@wals@kcl{Austronesian}
\def\langnames@fams@wals@kcm{Central Sudanic}
\def\langnames@fams@wals@kcn{Afro-Asiatic}
\def\langnames@fams@wals@kco{Nuclear Trans New Guinea}
\def\langnames@fams@wals@kcp{Kadugli-Krongo}
\def\langnames@fams@wals@kcq{Atlantic-Congo}
\def\langnames@fams@wals@kcr{Katla-Tima}
\def\langnames@fams@wals@kcs{Afro-Asiatic}
\def\langnames@fams@wals@kct{Lower Sepik-Ramu}
\def\langnames@fams@wals@kcu{Atlantic-Congo}
\def\langnames@fams@wals@kcv{Atlantic-Congo}
\def\langnames@fams@wals@kcw{Atlantic-Congo}
\def\langnames@fams@wals@kcx{Ta-Ne-Omotic}
\def\langnames@fams@wals@kcy{Songhay}
\def\langnames@fams@wals@kcz{Atlantic-Congo}
\def\langnames@fams@wals@kda{Pama-Nyungan}
\def\langnames@fams@wals@kdc{Atlantic-Congo}
\def\langnames@fams@wals@kdd{Pama-Nyungan}
\def\langnames@fams@wals@kde{Atlantic-Congo}
\def\langnames@fams@wals@kdf{Austronesian}
\def\langnames@fams@wals@kdg{Atlantic-Congo}
\def\langnames@fams@wals@kdh{Atlantic-Congo}
\def\langnames@fams@wals@kdi{Nilotic}
\def\langnames@fams@wals@kdj{Nilotic}
\def\langnames@fams@wals@kdk{Austronesian}
\def\langnames@fams@wals@kdl{Atlantic-Congo}
\def\langnames@fams@wals@kdm{Atlantic-Congo}
\def\langnames@fams@wals@kdn{Atlantic-Congo}
\def\langnames@fams@wals@kdp{Atlantic-Congo}
\def\langnames@fams@wals@kdq{Sino-Tibetan}
\def\langnames@fams@wals@kdr{Turkic}
\def\langnames@fams@wals@kdt{Austroasiatic}
\def\langnames@fams@wals@kdu{Nubian}
\def\langnames@fams@wals@kdv{Sino-Tibetan}
\def\langnames@fams@wals@kdw{Mombum-Koneraw}
\def\langnames@fams@wals@kdx{Atlantic-Congo}
\def\langnames@fams@wals@kdy{Tor-Orya}
\def\langnames@fams@wals@kdz{Atlantic-Congo}
\def\langnames@fams@wals@kea{Indo-European}
\def\langnames@fams@wals@keb{Atlantic-Congo}
\def\langnames@fams@wals@kec{Kadugli-Krongo}
\def\langnames@fams@wals@ked{Atlantic-Congo}
\def\langnames@fams@wals@kee{Keresan}
\def\langnames@fams@wals@kef{Atlantic-Congo}
\def\langnames@fams@wals@keg{Temeinic}
\def\langnames@fams@wals@keh{Ndu}
\def\langnames@fams@wals@kei{Austronesian}
\def\langnames@fams@wals@kej{Dravidian}
\def\langnames@fams@wals@kek{Mayan}
\def\langnames@fams@wals@kem{Austronesian}
\def\langnames@fams@wals@ken{Atlantic-Congo}
\def\langnames@fams@wals@keo{Nilotic}
\def\langnames@fams@wals@kep{Dravidian}
\def\langnames@fams@wals@keq{Indo-European}
\def\langnames@fams@wals@ker{Afro-Asiatic}
\def\langnames@fams@wals@kes{Atlantic-Congo}
\def\langnames@fams@wals@ket{Yeniseian}
\def\langnames@fams@wals@keu{Atlantic-Congo}
\def\langnames@fams@wals@kev{Dravidian}
\def\langnames@fams@wals@kew{Nuclear Trans New Guinea}
\def\langnames@fams@wals@key{Indo-European}
\def\langnames@fams@wals@kez{Atlantic-Congo}
\def\langnames@fams@wals@kfa{Dravidian}
\def\langnames@fams@wals@kfb{Dravidian}
\def\langnames@fams@wals@kfc{Dravidian}
\def\langnames@fams@wals@kfd{Dravidian}
\def\langnames@fams@wals@kfe{Dravidian}
\def\langnames@fams@wals@kff{Dravidian}
\def\langnames@fams@wals@kfg{Dravidian}
\def\langnames@fams@wals@kfh{Dravidian}
\def\langnames@fams@wals@kfk{Sino-Tibetan}
\def\langnames@fams@wals@kfl{Atlantic-Congo}
\def\langnames@fams@wals@kfm{Indo-European}
\def\langnames@fams@wals@kfn{Atlantic-Congo}
\def\langnames@fams@wals@kfo{Mande}
\def\langnames@fams@wals@kfp{Austroasiatic}
\def\langnames@fams@wals@kfq{Austroasiatic}
\def\langnames@fams@wals@kfr{Indo-European}
\def\langnames@fams@wals@kfs{Indo-European}
\def\langnames@fams@wals@kft{Indo-European}
\def\langnames@fams@wals@kfu{Indo-European}
\def\langnames@fams@wals@kfv{Indo-European}
\def\langnames@fams@wals@kfw{Sino-Tibetan}
\def\langnames@fams@wals@kfx{Indo-European}
\def\langnames@fams@wals@kfy{Indo-European}
\def\langnames@fams@wals@kfz{Atlantic-Congo}
\def\langnames@fams@wals@kga{Mande}
\def\langnames@fams@wals@kgb{Austronesian}
\def\langnames@fams@wals@kge{Austronesian}
\def\langnames@fams@wals@kgf{Nuclear Trans New Guinea}
\def\langnames@fams@wals@kgg{Isolate}
\def\langnames@fams@wals@kgi{Sign Language}
\def\langnames@fams@wals@kgj{Sino-Tibetan}
\def\langnames@fams@wals@kgk{Tupian}
\def\langnames@fams@wals@kgl{Pama-Nyungan}
\def\langnames@fams@wals@kgn{Indo-European}
\def\langnames@fams@wals@kgo{Kadugli-Krongo}
\def\langnames@fams@wals@kgp{Nuclear-Macro-Je}
\def\langnames@fams@wals@kgq{Nuclear Trans New Guinea}
\def\langnames@fams@wals@kgr{Isolate}
\def\langnames@fams@wals@kgs{Pama-Nyungan}
\def\langnames@fams@wals@kgt{Atlantic-Congo}
\def\langnames@fams@wals@kgu{Nuclear Trans New Guinea}
\def\langnames@fams@wals@kgv{West Bomberai}
\def\langnames@fams@wals@kgx{Austronesian}
\def\langnames@fams@wals@kgy{Sino-Tibetan}
\def\langnames@fams@wals@kha{Austroasiatic}
\def\langnames@fams@wals@khb{Tai-Kadai}
\def\langnames@fams@wals@khc{Austronesian}
\def\langnames@fams@wals@khd{Yam}
\def\langnames@fams@wals@khe{Nuclear Trans New Guinea}
\def\langnames@fams@wals@khf{Austroasiatic}
\def\langnames@fams@wals@khg{Sino-Tibetan}
\def\langnames@fams@wals@khh{Isolate}
\def\langnames@fams@wals@khj{Atlantic-Congo}
\def\langnames@fams@wals@khk{Mongolic-Khitan}
\def\langnames@fams@wals@khl{Austronesian}
\def\langnames@fams@wals@khm{Austroasiatic}
\def\langnames@fams@wals@khn{Indo-European}
\def\langnames@fams@wals@kho{Indo-European}
\def\langnames@fams@wals@khp{Isolate}
\def\langnames@fams@wals@khq{Songhay}
\def\langnames@fams@wals@khr{Austroasiatic}
\def\langnames@fams@wals@khs{Bosavi}
\def\langnames@fams@wals@kht{Tai-Kadai}
\def\langnames@fams@wals@khu{Atlantic-Congo}
\def\langnames@fams@wals@khv{Nakh-Daghestanian}
\def\langnames@fams@wals@khw{Indo-European}
\def\langnames@fams@wals@khx{Atlantic-Congo}
\def\langnames@fams@wals@khy{Atlantic-Congo}
\def\langnames@fams@wals@khz{Austronesian}
\def\langnames@fams@wals@kia{Atlantic-Congo}
\def\langnames@fams@wals@kib{Heibanic}
\def\langnames@fams@wals@kic{Algic}
\def\langnames@fams@wals@kid{Atlantic-Congo}
\def\langnames@fams@wals@kie{Maban}
\def\langnames@fams@wals@kif{Sino-Tibetan}
\def\langnames@fams@wals@kig{Kolopom}
\def\langnames@fams@wals@kih{Border}
\def\langnames@fams@wals@kii{Caddoan}
\def\langnames@fams@wals@kij{Austronesian}
\def\langnames@fams@wals@kik{Atlantic-Congo}
\def\langnames@fams@wals@kil{Afro-Asiatic}
\def\langnames@fams@wals@kim{Turkic}
\def\langnames@fams@wals@kin{Atlantic-Congo}
\def\langnames@fams@wals@kio{Kiowa-Tanoan}
\def\langnames@fams@wals@kip{Sino-Tibetan}
\def\langnames@fams@wals@kiq{Kaure-Kosare}
\def\langnames@fams@wals@kir{Turkic}
\def\langnames@fams@wals@kis{Austronesian}
\def\langnames@fams@wals@kit{Pahoturi}
\def\langnames@fams@wals@kiu{Indo-European}
\def\langnames@fams@wals@kiv{Atlantic-Congo}
\def\langnames@fams@wals@kiw{Kiwaian}
\def\langnames@fams@wals@kix{Sino-Tibetan}
\def\langnames@fams@wals@kiy{Lakes Plain}
\def\langnames@fams@wals@kiz{Atlantic-Congo}
\def\langnames@fams@wals@kja{Nimboranic}
\def\langnames@fams@wals@kjb{Mayan}
\def\langnames@fams@wals@kjc{Austronesian}
\def\langnames@fams@wals@kjd{Kiwaian}
\def\langnames@fams@wals@kje{Austronesian}
\def\langnames@fams@wals@kjg{Austroasiatic}
\def\langnames@fams@wals@kjh{Turkic}
\def\langnames@fams@wals@kji{Austronesian}
\def\langnames@fams@wals@kjj{Nakh-Daghestanian}
\def\langnames@fams@wals@kjk{Austronesian}
\def\langnames@fams@wals@kjl{Sino-Tibetan}
\def\langnames@fams@wals@kjm{Austroasiatic}
\def\langnames@fams@wals@kjn{Pama-Nyungan}
\def\langnames@fams@wals@kjo{Indo-European}
\def\langnames@fams@wals@kjp{Sino-Tibetan}
\def\langnames@fams@wals@kjq{Keresan}
\def\langnames@fams@wals@kjr{Austronesian}
\def\langnames@fams@wals@kjs{Nuclear Trans New Guinea}
\def\langnames@fams@wals@kjt{Sino-Tibetan}
\def\langnames@fams@wals@kju{Pomoan}
\def\langnames@fams@wals@kjv{Indo-European}
\def\langnames@fams@wals@kjx{North Bougainville}
\def\langnames@fams@wals@kjy{Nuclear Trans New Guinea}
\def\langnames@fams@wals@kjz{Sino-Tibetan}
\def\langnames@fams@wals@kka{Atlantic-Congo}
\def\langnames@fams@wals@kkb{Lakes Plain}
\def\langnames@fams@wals@kkc{East Strickland}
\def\langnames@fams@wals@kkd{Atlantic-Congo}
\def\langnames@fams@wals@kke{Mande}
\def\langnames@fams@wals@kkf{Sino-Tibetan}
\def\langnames@fams@wals@kkg{Austronesian}
\def\langnames@fams@wals@kkh{Tai-Kadai}
\def\langnames@fams@wals@kki{Atlantic-Congo}
\def\langnames@fams@wals@kkj{Atlantic-Congo}
\def\langnames@fams@wals@kkk{Austronesian}
\def\langnames@fams@wals@kkl{Nuclear Trans New Guinea}
\def\langnames@fams@wals@kkm{Atlantic-Congo}
\def\langnames@fams@wals@kko{Nubian}
\def\langnames@fams@wals@kkp{Pama-Nyungan}
\def\langnames@fams@wals@kkq{Atlantic-Congo}
\def\langnames@fams@wals@kkr{Afro-Asiatic}
\def\langnames@fams@wals@kks{Afro-Asiatic}
\def\langnames@fams@wals@kkt{Sino-Tibetan}
\def\langnames@fams@wals@kku{Unattested}
\def\langnames@fams@wals@kkv{Austronesian}
\def\langnames@fams@wals@kkw{Atlantic-Congo}
\def\langnames@fams@wals@kkx{Austronesian}
\def\langnames@fams@wals@kky{Pama-Nyungan}
\def\langnames@fams@wals@kkz{Athabaskan-Eyak-Tlingit}
\def\langnames@fams@wals@kla{Isolate}
\def\langnames@fams@wals@klb{Cochimi-Yuman}
\def\langnames@fams@wals@klc{Atlantic-Congo}
\def\langnames@fams@wals@kld{Pama-Nyungan}
\def\langnames@fams@wals@kle{Sino-Tibetan}
\def\langnames@fams@wals@klf{Maban}
\def\langnames@fams@wals@klg{Austronesian}
\def\langnames@fams@wals@klh{Nuclear Trans New Guinea}
\def\langnames@fams@wals@kli{Austronesian}
\def\langnames@fams@wals@klj{Turkic}
\def\langnames@fams@wals@klk{Atlantic-Congo}
\def\langnames@fams@wals@kll{Austronesian}
\def\langnames@fams@wals@klm{Nuclear Trans New Guinea}
\def\langnames@fams@wals@klo{Atlantic-Congo}
\def\langnames@fams@wals@klp{Angan}
\def\langnames@fams@wals@klq{Turama-Kikori}
\def\langnames@fams@wals@klr{Sino-Tibetan}
\def\langnames@fams@wals@kls{Indo-European}
\def\langnames@fams@wals@klt{Nuclear Trans New Guinea}
\def\langnames@fams@wals@klu{Kru}
\def\langnames@fams@wals@klv{Austronesian}
\def\langnames@fams@wals@klw{Austronesian}
\def\langnames@fams@wals@klx{Austronesian}
\def\langnames@fams@wals@kly{Austronesian}
\def\langnames@fams@wals@klz{Timor-Alor-Pantar}
\def\langnames@fams@wals@kma{Atlantic-Congo}
\def\langnames@fams@wals@kmb{Atlantic-Congo}
\def\langnames@fams@wals@kmc{Tai-Kadai}
\def\langnames@fams@wals@kmd{Austronesian}
\def\langnames@fams@wals@kme{Atlantic-Congo}
\def\langnames@fams@wals@kmf{Nuclear Trans New Guinea}
\def\langnames@fams@wals@kmg{Nuclear Trans New Guinea}
\def\langnames@fams@wals@kmh{Nuclear Trans New Guinea}
\def\langnames@fams@wals@kmi{Atlantic-Congo}
\def\langnames@fams@wals@kmj{Dravidian}
\def\langnames@fams@wals@kmk{Austronesian}
\def\langnames@fams@wals@kml{Austronesian}
\def\langnames@fams@wals@kmm{Sino-Tibetan}
\def\langnames@fams@wals@kmn{Sepik}
\def\langnames@fams@wals@kmo{Sepik}
\def\langnames@fams@wals@kmp{Atlantic-Congo}
\def\langnames@fams@wals@kmq{Koman}
\def\langnames@fams@wals@kmr{Indo-European}
\def\langnames@fams@wals@kms{Nuclear Torricelli}
\def\langnames@fams@wals@kmt{Nimboranic}
\def\langnames@fams@wals@kmu{Nuclear Trans New Guinea}
\def\langnames@fams@wals@kmv{Indo-European}
\def\langnames@fams@wals@kmw{Atlantic-Congo}
\def\langnames@fams@wals@kmx{Kiwaian}
\def\langnames@fams@wals@kmy{Atlantic-Congo}
\def\langnames@fams@wals@kmz{Turkic}
\def\langnames@fams@wals@kna{Afro-Asiatic}
\def\langnames@fams@wals@knb{Austronesian}
\def\langnames@fams@wals@knc{Saharan}
\def\langnames@fams@wals@knd{Konda-Yahadian}
\def\langnames@fams@wals@kne{Austronesian}
\def\langnames@fams@wals@knf{Atlantic-Congo}
\def\langnames@fams@wals@kng{Atlantic-Congo}
\def\langnames@fams@wals@kni{Atlantic-Congo}
\def\langnames@fams@wals@knj{Mayan}
\def\langnames@fams@wals@knk{Mande}
\def\langnames@fams@wals@knl{Austronesian}
\def\langnames@fams@wals@knm{Katukinan}
\def\langnames@fams@wals@knn{Indo-European}
\def\langnames@fams@wals@kno{Mande}
\def\langnames@fams@wals@knp{Atlantic-Congo}
\def\langnames@fams@wals@knq{Austroasiatic}
\def\langnames@fams@wals@knr{Sepik}
\def\langnames@fams@wals@kns{Austroasiatic}
\def\langnames@fams@wals@knt{Pano-Tacanan}
\def\langnames@fams@wals@knu{Mande}
\def\langnames@fams@wals@knv{Isolate}
\def\langnames@fams@wals@knw{Kxa}
\def\langnames@fams@wals@knx{Austronesian}
\def\langnames@fams@wals@kny{Atlantic-Congo}
\def\langnames@fams@wals@knz{Atlantic-Congo}
\def\langnames@fams@wals@koa{Austronesian}
\def\langnames@fams@wals@koc{Atlantic-Congo}
\def\langnames@fams@wals@kod{Austronesian}
\def\langnames@fams@wals@koe{Surmic}
\def\langnames@fams@wals@kof{Afro-Asiatic}
\def\langnames@fams@wals@kog{Chibchan}
\def\langnames@fams@wals@koh{Atlantic-Congo}
\def\langnames@fams@wals@koi{Uralic}
\def\langnames@fams@wals@kol{Isolate}
\def\langnames@fams@wals@koo{Atlantic-Congo}
\def\langnames@fams@wals@kop{Nuclear Trans New Guinea}
\def\langnames@fams@wals@koq{Atlantic-Congo}
\def\langnames@fams@wals@kor{Koreanic}
\def\langnames@fams@wals@kos{Austronesian}
\def\langnames@fams@wals@kot{Afro-Asiatic}
\def\langnames@fams@wals@kou{Atlantic-Congo}
\def\langnames@fams@wals@kov{Atlantic-Congo}
\def\langnames@fams@wals@kow{Atlantic-Congo}
\def\langnames@fams@wals@koy{Athabaskan-Eyak-Tlingit}
\def\langnames@fams@wals@koz{Nuclear Trans New Guinea}
\def\langnames@fams@wals@kpa{Afro-Asiatic}
\def\langnames@fams@wals@kpb{Dravidian}
\def\langnames@fams@wals@kpc{Arawakan}
\def\langnames@fams@wals@kpd{Austronesian}
\def\langnames@fams@wals@kpf{Nuclear Trans New Guinea}
\def\langnames@fams@wals@kpg{Austronesian}
\def\langnames@fams@wals@kph{Atlantic-Congo}
\def\langnames@fams@wals@kpi{Geelvink Bay}
\def\langnames@fams@wals@kpj{Nuclear-Macro-Je}
\def\langnames@fams@wals@kpk{Atlantic-Congo}
\def\langnames@fams@wals@kpl{Atlantic-Congo}
\def\langnames@fams@wals@kpm{Austroasiatic}
\def\langnames@fams@wals@kpn{Tupian}
\def\langnames@fams@wals@kpo{Atlantic-Congo}
\def\langnames@fams@wals@kpq{Nuclear Trans New Guinea}
\def\langnames@fams@wals@kpr{Nuclear Trans New Guinea}
\def\langnames@fams@wals@kps{West Bird's Head}
\def\langnames@fams@wals@kpt{Nakh-Daghestanian}
\def\langnames@fams@wals@kpu{Timor-Alor-Pantar}
\def\langnames@fams@wals@kpv{Uralic}
\def\langnames@fams@wals@kpw{Nuclear Trans New Guinea}
\def\langnames@fams@wals@kpx{Koiarian}
\def\langnames@fams@wals@kpy{Chukotko-Kamchatkan}
\def\langnames@fams@wals@kpz{Nilotic}
\def\langnames@fams@wals@kqa{Nuclear Trans New Guinea}
\def\langnames@fams@wals@kqb{Nuclear Trans New Guinea}
\def\langnames@fams@wals@kqc{Manubaran}
\def\langnames@fams@wals@kqd{Afro-Asiatic}
\def\langnames@fams@wals@kqe{Austronesian}
\def\langnames@fams@wals@kqf{Austronesian}
\def\langnames@fams@wals@kqg{Atlantic-Congo}
\def\langnames@fams@wals@kqi{Koiarian}
\def\langnames@fams@wals@kqj{South Bougainville}
\def\langnames@fams@wals@kqk{Atlantic-Congo}
\def\langnames@fams@wals@kql{Yuat}
\def\langnames@fams@wals@kqm{Atlantic-Congo}
\def\langnames@fams@wals@kqn{Atlantic-Congo}
\def\langnames@fams@wals@kqo{Kru}
\def\langnames@fams@wals@kqp{Afro-Asiatic}
\def\langnames@fams@wals@kqq{Nuclear-Macro-Je}
\def\langnames@fams@wals@kqr{Austronesian}
\def\langnames@fams@wals@kqs{Atlantic-Congo}
\def\langnames@fams@wals@kqt{Austronesian}
\def\langnames@fams@wals@kqu{Tuu}
\def\langnames@fams@wals@kqv{Austronesian}
\def\langnames@fams@wals@kqw{Austronesian}
\def\langnames@fams@wals@kqx{Afro-Asiatic}
\def\langnames@fams@wals@kqy{Ta-Ne-Omotic}
\def\langnames@fams@wals@kqz{Khoe-Kwadi}
\def\langnames@fams@wals@kra{Indo-European}
\def\langnames@fams@wals@krb{Miwok-Costanoan}
\def\langnames@fams@wals@krc{Turkic}
\def\langnames@fams@wals@krd{Austronesian}
\def\langnames@fams@wals@kre{Nuclear-Macro-Je}
\def\langnames@fams@wals@krf{Austronesian}
\def\langnames@fams@wals@krh{Atlantic-Congo}
\def\langnames@fams@wals@kri{Indo-European}
\def\langnames@fams@wals@krj{Austronesian}
\def\langnames@fams@wals@krk{Chukotko-Kamchatkan}
\def\langnames@fams@wals@krl{Uralic}
\def\langnames@fams@wals@krn{Kru}
\def\langnames@fams@wals@krp{Atlantic-Congo}
\def\langnames@fams@wals@krs{Kresh-Aja}
\def\langnames@fams@wals@krt{Saharan}
\def\langnames@fams@wals@kru{Dravidian}
\def\langnames@fams@wals@krw{Kru}
\def\langnames@fams@wals@krx{Atlantic-Congo}
\def\langnames@fams@wals@kry{Nakh-Daghestanian}
\def\langnames@fams@wals@krz{Yam}
\def\langnames@fams@wals@ksa{Unattested}
\def\langnames@fams@wals@ksb{Atlantic-Congo}
\def\langnames@fams@wals@ksc{Austronesian}
\def\langnames@fams@wals@ksd{Austronesian}
\def\langnames@fams@wals@kse{Austronesian}
\def\langnames@fams@wals@ksf{Atlantic-Congo}
\def\langnames@fams@wals@ksg{Austronesian}
\def\langnames@fams@wals@ksh{Indo-European}
\def\langnames@fams@wals@ksi{Sko}
\def\langnames@fams@wals@ksj{Kwalean}
\def\langnames@fams@wals@ksk{Siouan}
\def\langnames@fams@wals@ksl{Austronesian}
\def\langnames@fams@wals@ksm{Atlantic-Congo}
\def\langnames@fams@wals@ksn{Austronesian}
\def\langnames@fams@wals@ksp{Central Sudanic}
\def\langnames@fams@wals@ksq{Afro-Asiatic}
\def\langnames@fams@wals@ksr{Nuclear Trans New Guinea}
\def\langnames@fams@wals@kss{Atlantic-Congo}
\def\langnames@fams@wals@kst{Atlantic-Congo}
\def\langnames@fams@wals@ksu{Tai-Kadai}
\def\langnames@fams@wals@ksv{Atlantic-Congo}
\def\langnames@fams@wals@ksw{Sino-Tibetan}
\def\langnames@fams@wals@ksx{Austronesian}
\def\langnames@fams@wals@ksy{Indo-European}
\def\langnames@fams@wals@ksz{Austroasiatic}
\def\langnames@fams@wals@kta{Austroasiatic}
\def\langnames@fams@wals@ktb{Afro-Asiatic}
\def\langnames@fams@wals@ktc{Afro-Asiatic}
\def\langnames@fams@wals@ktd{Pama-Nyungan}
\def\langnames@fams@wals@kte{Sino-Tibetan}
\def\langnames@fams@wals@ktf{Atlantic-Congo}
\def\langnames@fams@wals@ktg{Pama-Nyungan}
\def\langnames@fams@wals@kth{Maban}
\def\langnames@fams@wals@kti{Nuclear Trans New Guinea}
\def\langnames@fams@wals@ktj{Kru}
\def\langnames@fams@wals@ktk{Austronesian}
\def\langnames@fams@wals@ktl{Indo-European}
\def\langnames@fams@wals@ktm{Austronesian}
\def\langnames@fams@wals@ktn{Tupian}
\def\langnames@fams@wals@kto{Isolate}
\def\langnames@fams@wals@ktp{Sino-Tibetan}
\def\langnames@fams@wals@ktq{Unclassifiable}
\def\langnames@fams@wals@kts{Nuclear Trans New Guinea}
\def\langnames@fams@wals@ktt{Nuclear Trans New Guinea}
\def\langnames@fams@wals@ktu{Atlantic-Congo}
\def\langnames@fams@wals@ktv{Austroasiatic}
\def\langnames@fams@wals@ktw{Athabaskan-Eyak-Tlingit}
\def\langnames@fams@wals@ktx{Pano-Tacanan}
\def\langnames@fams@wals@kty{Atlantic-Congo}
\def\langnames@fams@wals@ktz{Kxa}
\def\langnames@fams@wals@kua{Atlantic-Congo}
\def\langnames@fams@wals@kub{Atlantic-Congo}
\def\langnames@fams@wals@kuc{Tor-Orya}
\def\langnames@fams@wals@kud{Austronesian}
\def\langnames@fams@wals@kue{Nuclear Trans New Guinea}
\def\langnames@fams@wals@kuf{Austroasiatic}
\def\langnames@fams@wals@kug{Atlantic-Congo}
\def\langnames@fams@wals@kuh{Afro-Asiatic}
\def\langnames@fams@wals@kui{Cariban}
\def\langnames@fams@wals@kuj{Atlantic-Congo}
\def\langnames@fams@wals@kuk{Austronesian}
\def\langnames@fams@wals@kul{Afro-Asiatic}
\def\langnames@fams@wals@kum{Turkic}
\def\langnames@fams@wals@kun{Isolate}
\def\langnames@fams@wals@kuo{Nuclear Trans New Guinea}
\def\langnames@fams@wals@kup{Goilalan}
\def\langnames@fams@wals@kuq{Tupian}
\def\langnames@fams@wals@kus{Atlantic-Congo}
\def\langnames@fams@wals@kut{Isolate}
\def\langnames@fams@wals@kuu{Athabaskan-Eyak-Tlingit}
\def\langnames@fams@wals@kuv{Austronesian}
\def\langnames@fams@wals@kuw{Atlantic-Congo}
\def\langnames@fams@wals@kux{Pama-Nyungan}
\def\langnames@fams@wals@kuy{Pama-Nyungan}
\def\langnames@fams@wals@kuz{Isolate}
\def\langnames@fams@wals@kva{Nakh-Daghestanian}
\def\langnames@fams@wals@kvb{Austronesian}
\def\langnames@fams@wals@kvc{Austronesian}
\def\langnames@fams@wals@kvd{Timor-Alor-Pantar}
\def\langnames@fams@wals@kve{Austronesian}
\def\langnames@fams@wals@kvf{Afro-Asiatic}
\def\langnames@fams@wals@kvg{Anim}
\def\langnames@fams@wals@kvh{Austronesian}
\def\langnames@fams@wals@kvi{Afro-Asiatic}
\def\langnames@fams@wals@kvj{Afro-Asiatic}
\def\langnames@fams@wals@kvk{Sign Language}
\def\langnames@fams@wals@kvl{Sino-Tibetan}
\def\langnames@fams@wals@kvm{Atlantic-Congo}
\def\langnames@fams@wals@kvn{Chibchan}
\def\langnames@fams@wals@kvo{Austronesian}
\def\langnames@fams@wals@kvp{Austronesian}
\def\langnames@fams@wals@kvq{Sino-Tibetan}
\def\langnames@fams@wals@kvr{Austronesian}
\def\langnames@fams@wals@kvu{Sino-Tibetan}
\def\langnames@fams@wals@kvv{Austronesian}
\def\langnames@fams@wals@kvw{Timor-Alor-Pantar}
\def\langnames@fams@wals@kvx{Indo-European}
\def\langnames@fams@wals@kvy{Sino-Tibetan}
\def\langnames@fams@wals@kvz{Nuclear Trans New Guinea}
\def\langnames@fams@wals@kwa{Naduhup}
\def\langnames@fams@wals@kwb{Atlantic-Congo}
\def\langnames@fams@wals@kwc{Atlantic-Congo}
\def\langnames@fams@wals@kwd{Austronesian}
\def\langnames@fams@wals@kwe{Greater Kwerba}
\def\langnames@fams@wals@kwf{Austronesian}
\def\langnames@fams@wals@kwg{Central Sudanic}
\def\langnames@fams@wals@kwh{Austronesian}
\def\langnames@fams@wals@kwi{Barbacoan}
\def\langnames@fams@wals@kwj{Sepik}
\def\langnames@fams@wals@kwk{Wakashan}
\def\langnames@fams@wals@kwl{Afro-Asiatic}
\def\langnames@fams@wals@kwm{Atlantic-Congo}
\def\langnames@fams@wals@kwn{Atlantic-Congo}
\def\langnames@fams@wals@kwo{Kwomtari-Nai}
\def\langnames@fams@wals@kwp{Kru}
\def\langnames@fams@wals@kwr{Nuclear Trans New Guinea}
\def\langnames@fams@wals@kws{Atlantic-Congo}
\def\langnames@fams@wals@kwt{Tor-Orya}
\def\langnames@fams@wals@kwu{Atlantic-Congo}
\def\langnames@fams@wals@kwv{Central Sudanic}
\def\langnames@fams@wals@kww{Indo-European}
\def\langnames@fams@wals@kwx{Dravidian}
\def\langnames@fams@wals@kwy{Atlantic-Congo}
\def\langnames@fams@wals@kwz{Khoe-Kwadi}
\def\langnames@fams@wals@kxa{Austronesian}
\def\langnames@fams@wals@kxb{Atlantic-Congo}
\def\langnames@fams@wals@kxc{Afro-Asiatic}
\def\langnames@fams@wals@kxd{Austronesian}
\def\langnames@fams@wals@kxf{Sino-Tibetan}
\def\langnames@fams@wals@kxh{South Omotic}
\def\langnames@fams@wals@kxi{Austronesian}
\def\langnames@fams@wals@kxj{Central Sudanic}
\def\langnames@fams@wals@kxk{Sino-Tibetan}
\def\langnames@fams@wals@kxm{Austroasiatic}
\def\langnames@fams@wals@kxn{Austronesian}
\def\langnames@fams@wals@kxo{Isolate}
\def\langnames@fams@wals@kxp{Indo-European}
\def\langnames@fams@wals@kxq{Yam}
\def\langnames@fams@wals@kxr{Austronesian}
\def\langnames@fams@wals@kxs{Mongolic-Khitan}
\def\langnames@fams@wals@kxt{Ndu}
\def\langnames@fams@wals@kxu{Dravidian}
\def\langnames@fams@wals@kxv{Dravidian}
\def\langnames@fams@wals@kxw{East Strickland}
\def\langnames@fams@wals@kxx{Atlantic-Congo}
\def\langnames@fams@wals@kxy{Austroasiatic}
\def\langnames@fams@wals@kxz{Kiwaian}
\def\langnames@fams@wals@kya{Atlantic-Congo}
\def\langnames@fams@wals@kyb{Austronesian}
\def\langnames@fams@wals@kyc{Nuclear Trans New Guinea}
\def\langnames@fams@wals@kyd{Austronesian}
\def\langnames@fams@wals@kye{Atlantic-Congo}
\def\langnames@fams@wals@kyf{Kru}
\def\langnames@fams@wals@kyg{Nuclear Trans New Guinea}
\def\langnames@fams@wals@kyh{Isolate}
\def\langnames@fams@wals@kyi{Austronesian}
\def\langnames@fams@wals@kyj{Austronesian}
\def\langnames@fams@wals@kyk{Austronesian}
\def\langnames@fams@wals@kyl{Kalapuyan}
\def\langnames@fams@wals@kyn{Austronesian}
\def\langnames@fams@wals@kyo{Timor-Alor-Pantar}
\def\langnames@fams@wals@kyq{Central Sudanic}
\def\langnames@fams@wals@kyr{Tupian}
\def\langnames@fams@wals@kys{Austronesian}
\def\langnames@fams@wals@kyt{Kayagaric}
\def\langnames@fams@wals@kyu{Sino-Tibetan}
\def\langnames@fams@wals@kyw{Indo-European}
\def\langnames@fams@wals@kyx{North Bougainville}
\def\langnames@fams@wals@kyy{Nuclear Trans New Guinea}
\def\langnames@fams@wals@kyz{Tupian}
\def\langnames@fams@wals@kza{Atlantic-Congo}
\def\langnames@fams@wals@kzb{Austronesian}
\def\langnames@fams@wals@kzc{Atlantic-Congo}
\def\langnames@fams@wals@kzd{Austronesian}
\def\langnames@fams@wals@kzf{Austronesian}
\def\langnames@fams@wals@kzg{Japonic}
\def\langnames@fams@wals@kzh{Nubian}
\def\langnames@fams@wals@kzi{Austronesian}
\def\langnames@fams@wals@kzk{Austronesian}
\def\langnames@fams@wals@kzl{Austronesian}
\def\langnames@fams@wals@kzm{South Bird's Head Family}
\def\langnames@fams@wals@kzn{Atlantic-Congo}
\def\langnames@fams@wals@kzo{Atlantic-Congo}
\def\langnames@fams@wals@kzp{Austronesian}
\def\langnames@fams@wals@kzq{Sino-Tibetan}
\def\langnames@fams@wals@kzr{Atlantic-Congo}
\def\langnames@fams@wals@kzs{Austronesian}
\def\langnames@fams@wals@kzu{Austronesian}
\def\langnames@fams@wals@kzv{Nuclear Trans New Guinea}
\def\langnames@fams@wals@kzw{Unclassifiable}
\def\langnames@fams@wals@kzx{Austronesian}
\def\langnames@fams@wals@kzy{Atlantic-Congo}
\def\langnames@fams@wals@kzz{West Bird's Head}
\def\langnames@fams@wals@laa{Austronesian}
\def\langnames@fams@wals@lac{Mayan}
\def\langnames@fams@wals@lad{Indo-European}
\def\langnames@fams@wals@lae{Sino-Tibetan}
\def\langnames@fams@wals@laf{Isolate}
\def\langnames@fams@wals@lag{Atlantic-Congo}
\def\langnames@fams@wals@lai{Atlantic-Congo}
\def\langnames@fams@wals@laj{Nilotic}
\def\langnames@fams@wals@lam{Atlantic-Congo}
\def\langnames@fams@wals@lan{Atlantic-Congo}
\def\langnames@fams@wals@lao{Tai-Kadai}
\def\langnames@fams@wals@lap{Central Sudanic}
\def\langnames@fams@wals@laq{Tai-Kadai}
\def\langnames@fams@wals@lar{Atlantic-Congo}
\def\langnames@fams@wals@las{Atlantic-Congo}
\def\langnames@fams@wals@lat{Indo-European}
\def\langnames@fams@wals@lav{Indo-European}
\def\langnames@fams@wals@law{Austronesian}
\def\langnames@fams@wals@lax{Sino-Tibetan}
\def\langnames@fams@wals@laz{Austronesian}
\def\langnames@fams@wals@lbb{Austronesian}
\def\langnames@fams@wals@lbc{Tai-Kadai}
\def\langnames@fams@wals@lbe{Nakh-Daghestanian}
\def\langnames@fams@wals@lbf{Sino-Tibetan}
\def\langnames@fams@wals@lbi{Speech Register}
\def\langnames@fams@wals@lbj{Sino-Tibetan}
\def\langnames@fams@wals@lbk{Austronesian}
\def\langnames@fams@wals@lbm{Indo-European}
\def\langnames@fams@wals@lbn{Austroasiatic}
\def\langnames@fams@wals@lbo{Austroasiatic}
\def\langnames@fams@wals@lbq{Austronesian}
\def\langnames@fams@wals@lbr{Sino-Tibetan}
\def\langnames@fams@wals@lbs{Sign Language}
\def\langnames@fams@wals@lbt{Tai-Kadai}
\def\langnames@fams@wals@lbu{Austronesian}
\def\langnames@fams@wals@lbv{Austronesian}
\def\langnames@fams@wals@lbw{Austronesian}
\def\langnames@fams@wals@lbx{Austronesian}
\def\langnames@fams@wals@lby{Pama-Nyungan}
\def\langnames@fams@wals@lbz{Tangkic}
\def\langnames@fams@wals@lcc{Austronesian}
\def\langnames@fams@wals@lcd{Austronesian}
\def\langnames@fams@wals@lce{Austronesian}
\def\langnames@fams@wals@lcf{Austronesian}
\def\langnames@fams@wals@lch{Atlantic-Congo}
\def\langnames@fams@wals@lcl{Austronesian}
\def\langnames@fams@wals@lcm{Austronesian}
\def\langnames@fams@wals@lcp{Austroasiatic}
\def\langnames@fams@wals@lcq{Austronesian}
\def\langnames@fams@wals@lcs{Austronesian}
\def\langnames@fams@wals@ldb{Atlantic-Congo}
\def\langnames@fams@wals@ldd{Afro-Asiatic}
\def\langnames@fams@wals@ldg{Atlantic-Congo}
\def\langnames@fams@wals@ldh{Atlantic-Congo}
\def\langnames@fams@wals@ldi{Atlantic-Congo}
\def\langnames@fams@wals@ldj{Atlantic-Congo}
\def\langnames@fams@wals@ldk{Atlantic-Congo}
\def\langnames@fams@wals@ldl{Atlantic-Congo}
\def\langnames@fams@wals@ldm{Atlantic-Congo}
\def\langnames@fams@wals@ldn{Artificial Language}
\def\langnames@fams@wals@ldo{Atlantic-Congo}
\def\langnames@fams@wals@ldp{Atlantic-Congo}
\def\langnames@fams@wals@ldq{Atlantic-Congo}
\def\langnames@fams@wals@lea{Atlantic-Congo}
\def\langnames@fams@wals@leb{Atlantic-Congo}
\def\langnames@fams@wals@lec{Isolate}
\def\langnames@fams@wals@led{Central Sudanic}
\def\langnames@fams@wals@lee{Atlantic-Congo}
\def\langnames@fams@wals@lef{Atlantic-Congo}
\def\langnames@fams@wals@leh{Atlantic-Congo}
\def\langnames@fams@wals@lei{Nuclear Trans New Guinea}
\def\langnames@fams@wals@lej{Atlantic-Congo}
\def\langnames@fams@wals@lek{Austronesian}
\def\langnames@fams@wals@lel{Atlantic-Congo}
\def\langnames@fams@wals@lem{Atlantic-Congo}
\def\langnames@fams@wals@leo{Atlantic-Congo}
\def\langnames@fams@wals@lep{Sino-Tibetan}
\def\langnames@fams@wals@leq{Nuclear Trans New Guinea}
\def\langnames@fams@wals@ler{Austronesian}
\def\langnames@fams@wals@les{Central Sudanic}
\def\langnames@fams@wals@let{Austronesian}
\def\langnames@fams@wals@leu{Austronesian}
\def\langnames@fams@wals@lev{Timor-Alor-Pantar}
\def\langnames@fams@wals@lew{Austronesian}
\def\langnames@fams@wals@lex{Austronesian}
\def\langnames@fams@wals@ley{Austronesian}
\def\langnames@fams@wals@lez{Nakh-Daghestanian}
\def\langnames@fams@wals@lfa{Atlantic-Congo}
\def\langnames@fams@wals@lfn{Artificial Language}
\def\langnames@fams@wals@lga{Austronesian}
\def\langnames@fams@wals@lgb{Austronesian}
\def\langnames@fams@wals@lgg{Central Sudanic}
\def\langnames@fams@wals@lgh{Sino-Tibetan}
\def\langnames@fams@wals@lgi{Austronesian}
\def\langnames@fams@wals@lgk{Austronesian}
\def\langnames@fams@wals@lgl{Austronesian}
\def\langnames@fams@wals@lgm{Atlantic-Congo}
\def\langnames@fams@wals@lgn{Koman}
\def\langnames@fams@wals@lgq{Atlantic-Congo}
\def\langnames@fams@wals@lgr{Austronesian}
\def\langnames@fams@wals@lgt{Sepik}
\def\langnames@fams@wals@lgu{Austronesian}
\def\langnames@fams@wals@lgz{Atlantic-Congo}
\def\langnames@fams@wals@lha{Tai-Kadai}
\def\langnames@fams@wals@lhh{Austronesian}
\def\langnames@fams@wals@lhi{Sino-Tibetan}
\def\langnames@fams@wals@lhl{Indo-European}
\def\langnames@fams@wals@lhm{Sino-Tibetan}
\def\langnames@fams@wals@lhn{Austronesian}
\def\langnames@fams@wals@lhp{Sino-Tibetan}
\def\langnames@fams@wals@lhs{Afro-Asiatic}
\def\langnames@fams@wals@lht{Austronesian}
\def\langnames@fams@wals@lhu{Sino-Tibetan}
\def\langnames@fams@wals@lia{Atlantic-Congo}
\def\langnames@fams@wals@lib{Austronesian}
\def\langnames@fams@wals@lic{Tai-Kadai}
\def\langnames@fams@wals@lid{Austronesian}
\def\langnames@fams@wals@lie{Atlantic-Congo}
\def\langnames@fams@wals@lif{Sino-Tibetan}
\def\langnames@fams@wals@lig{Mande}
\def\langnames@fams@wals@lih{Austronesian}
\def\langnames@fams@wals@lij{Indo-European}
\def\langnames@fams@wals@lik{Atlantic-Congo}
\def\langnames@fams@wals@lil{Salishan}
\def\langnames@fams@wals@lim{Indo-European}
\def\langnames@fams@wals@lin{Atlantic-Congo}
\def\langnames@fams@wals@lio{Austronesian}
\def\langnames@fams@wals@lip{Atlantic-Congo}
\def\langnames@fams@wals@liq{Afro-Asiatic}
\def\langnames@fams@wals@lir{Pidgin}
\def\langnames@fams@wals@lis{Sino-Tibetan}
\def\langnames@fams@wals@lit{Indo-European}
\def\langnames@fams@wals@liu{Dajuic}
\def\langnames@fams@wals@liv{Uralic}
\def\langnames@fams@wals@liw{Austronesian}
\def\langnames@fams@wals@lix{Austronesian}
\def\langnames@fams@wals@liy{Atlantic-Congo}
\def\langnames@fams@wals@liz{Atlantic-Congo}
\def\langnames@fams@wals@lje{Austronesian}
\def\langnames@fams@wals@lji{Austronesian}
\def\langnames@fams@wals@ljl{Austronesian}
\def\langnames@fams@wals@ljp{Austronesian}
\def\langnames@fams@wals@ljw{Pama-Nyungan}
\def\langnames@fams@wals@ljx{Pama-Nyungan}
\def\langnames@fams@wals@lka{Austronesian}
\def\langnames@fams@wals@lkb{Atlantic-Congo}
\def\langnames@fams@wals@lkc{Sino-Tibetan}
\def\langnames@fams@wals@lkd{Nambiquaran}
\def\langnames@fams@wals@lke{Atlantic-Congo}
\def\langnames@fams@wals@lkh{Sino-Tibetan}
\def\langnames@fams@wals@lki{Indo-European}
\def\langnames@fams@wals@lkj{Austronesian}
\def\langnames@fams@wals@lkl{Nuclear Torricelli}
\def\langnames@fams@wals@lkm{Pama-Nyungan}
\def\langnames@fams@wals@lkn{Austronesian}
\def\langnames@fams@wals@lko{Atlantic-Congo}
\def\langnames@fams@wals@lkr{Nilotic}
\def\langnames@fams@wals@lks{Atlantic-Congo}
\def\langnames@fams@wals@lkt{Siouan}
\def\langnames@fams@wals@lku{Pama-Nyungan}
\def\langnames@fams@wals@lky{Nilotic}
\def\langnames@fams@wals@lla{Atlantic-Congo}
\def\langnames@fams@wals@llb{Atlantic-Congo}
\def\langnames@fams@wals@llc{Mande}
\def\langnames@fams@wals@lld{Indo-European}
\def\langnames@fams@wals@lle{Austronesian}
\def\langnames@fams@wals@llf{Austronesian}
\def\langnames@fams@wals@llg{Austronesian}
\def\langnames@fams@wals@llh{Sino-Tibetan}
\def\langnames@fams@wals@lli{Atlantic-Congo}
\def\langnames@fams@wals@llk{Austronesian}
\def\langnames@fams@wals@lll{Bogia}
\def\langnames@fams@wals@llm{Austronesian}
\def\langnames@fams@wals@lln{Afro-Asiatic}
\def\langnames@fams@wals@llp{Austronesian}
\def\langnames@fams@wals@llq{Austronesian}
\def\langnames@fams@wals@lls{Sign Language}
\def\langnames@fams@wals@llu{Austronesian}
\def\langnames@fams@wals@llx{Austronesian}
\def\langnames@fams@wals@lma{Atlantic-Congo}
\def\langnames@fams@wals@lmb{Austronesian}
\def\langnames@fams@wals@lmc{Limilngan-Wulna}
\def\langnames@fams@wals@lmd{Narrow Talodi}
\def\langnames@fams@wals@lme{Afro-Asiatic}
\def\langnames@fams@wals@lmf{Austronesian}
\def\langnames@fams@wals@lmg{Austronesian}
\def\langnames@fams@wals@lmi{Central Sudanic}
\def\langnames@fams@wals@lmj{Austronesian}
\def\langnames@fams@wals@lmk{Sino-Tibetan}
\def\langnames@fams@wals@lml{Austronesian}
\def\langnames@fams@wals@lmn{Indo-European}
\def\langnames@fams@wals@lmo{Indo-European}
\def\langnames@fams@wals@lmp{Atlantic-Congo}
\def\langnames@fams@wals@lmq{Austronesian}
\def\langnames@fams@wals@lmr{Austronesian}
\def\langnames@fams@wals@lmu{Austronesian}
\def\langnames@fams@wals@lmv{Austronesian}
\def\langnames@fams@wals@lmw{Miwok-Costanoan}
\def\langnames@fams@wals@lmx{Atlantic-Congo}
\def\langnames@fams@wals@lmy{Austronesian}
\def\langnames@fams@wals@lmz{Unattested}
\def\langnames@fams@wals@lna{Atlantic-Congo}
\def\langnames@fams@wals@lnb{Atlantic-Congo}
\def\langnames@fams@wals@lnd{Austronesian}
\def\langnames@fams@wals@lnh{Austroasiatic}
\def\langnames@fams@wals@lni{South Bougainville}
\def\langnames@fams@wals@lnj{Pama-Nyungan}
\def\langnames@fams@wals@lnl{Atlantic-Congo}
\def\langnames@fams@wals@lnm{Keram}
\def\langnames@fams@wals@lnn{Austronesian}
\def\langnames@fams@wals@lno{Nilotic}
\def\langnames@fams@wals@lns{Atlantic-Congo}
\def\langnames@fams@wals@lnu{Atlantic-Congo}
\def\langnames@fams@wals@loa{North Halmahera}
\def\langnames@fams@wals@lob{Atlantic-Congo}
\def\langnames@fams@wals@loc{Austronesian}
\def\langnames@fams@wals@loe{Austronesian}
\def\langnames@fams@wals@lof{Heibanic}
\def\langnames@fams@wals@log{Central Sudanic}
\def\langnames@fams@wals@loh{Surmic}
\def\langnames@fams@wals@loi{Atlantic-Congo}
\def\langnames@fams@wals@loj{Austronesian}
\def\langnames@fams@wals@lok{Mande}
\def\langnames@fams@wals@lol{Atlantic-Congo}
\def\langnames@fams@wals@lom{Mande}
\def\langnames@fams@wals@lon{Atlantic-Congo}
\def\langnames@fams@wals@loo{Atlantic-Congo}
\def\langnames@fams@wals@lop{Atlantic-Congo}
\def\langnames@fams@wals@loq{Atlantic-Congo}
\def\langnames@fams@wals@lor{Atlantic-Congo}
\def\langnames@fams@wals@los{Austronesian}
\def\langnames@fams@wals@lot{Nilotic}
\def\langnames@fams@wals@lou{Indo-European}
\def\langnames@fams@wals@low{Austronesian}
\def\langnames@fams@wals@lox{Austronesian}
\def\langnames@fams@wals@loy{Sino-Tibetan}
\def\langnames@fams@wals@loz{Atlantic-Congo}
\def\langnames@fams@wals@lpa{Austronesian}
\def\langnames@fams@wals@lpe{Lepki-Murkim-Kembra}
\def\langnames@fams@wals@lpn{Sino-Tibetan}
\def\langnames@fams@wals@lpo{Sino-Tibetan}
\def\langnames@fams@wals@lpx{Nilotic}
\def\langnames@fams@wals@lra{Austronesian}
\def\langnames@fams@wals@lrc{Indo-European}
\def\langnames@fams@wals@lre{Iroquoian}
\def\langnames@fams@wals@lrg{Isolate}
\def\langnames@fams@wals@lri{Atlantic-Congo}
\def\langnames@fams@wals@lrl{Indo-European}
\def\langnames@fams@wals@lrm{Atlantic-Congo}
\def\langnames@fams@wals@lrn{Austronesian}
\def\langnames@fams@wals@lro{Heibanic}
\def\langnames@fams@wals@lrr{Sino-Tibetan}
\def\langnames@fams@wals@lrt{Austronesian}
\def\langnames@fams@wals@lrv{Austronesian}
\def\langnames@fams@wals@lrz{Austronesian}
\def\langnames@fams@wals@lsa{Indo-European}
\def\langnames@fams@wals@lsc{Sign Language}
\def\langnames@fams@wals@lsd{Afro-Asiatic}
\def\langnames@fams@wals@lse{Atlantic-Congo}
\def\langnames@fams@wals@lsh{Sino-Tibetan}
\def\langnames@fams@wals@lsi{Sino-Tibetan}
\def\langnames@fams@wals@lsl{Sign Language}
\def\langnames@fams@wals@lsm{Atlantic-Congo}
\def\langnames@fams@wals@lsn{Sign Language}
\def\langnames@fams@wals@lsp{Sign Language}
\def\langnames@fams@wals@lsr{Nuclear Torricelli}
\def\langnames@fams@wals@lss{Indo-European}
\def\langnames@fams@wals@lst{Sign Language}
\def\langnames@fams@wals@lsv{Sign Language}
\def\langnames@fams@wals@lsw{Sign Language}
\def\langnames@fams@wals@lsy{Sign Language}
\def\langnames@fams@wals@ltc{Sino-Tibetan}
\def\langnames@fams@wals@lti{Austronesian}
\def\langnames@fams@wals@ltn{Nambiquaran}
\def\langnames@fams@wals@lto{Atlantic-Congo}
\def\langnames@fams@wals@lts{Atlantic-Congo}
\def\langnames@fams@wals@ltu{Austronesian}
\def\langnames@fams@wals@ltz{Indo-European}
\def\langnames@fams@wals@lua{Atlantic-Congo}
\def\langnames@fams@wals@lub{Atlantic-Congo}
\def\langnames@fams@wals@luc{Central Sudanic}
\def\langnames@fams@wals@lud{Uralic}
\def\langnames@fams@wals@lue{Atlantic-Congo}
\def\langnames@fams@wals@luf{Mailuan}
\def\langnames@fams@wals@lug{Atlantic-Congo}
\def\langnames@fams@wals@lui{Uto-Aztecan}
\def\langnames@fams@wals@luj{Atlantic-Congo}
\def\langnames@fams@wals@luk{Sino-Tibetan}
\def\langnames@fams@wals@lul{Central Sudanic}
\def\langnames@fams@wals@lum{Atlantic-Congo}
\def\langnames@fams@wals@lun{Atlantic-Congo}
\def\langnames@fams@wals@luo{Nilotic}
\def\langnames@fams@wals@lup{Atlantic-Congo}
\def\langnames@fams@wals@luq{Atlantic-Congo}
\def\langnames@fams@wals@lus{Sino-Tibetan}
\def\langnames@fams@wals@lut{Salishan}
\def\langnames@fams@wals@luv{Indo-European}
\def\langnames@fams@wals@luw{Atlantic-Congo}
\def\langnames@fams@wals@luz{Indo-European}
\def\langnames@fams@wals@lva{Austronesian}
\def\langnames@fams@wals@lvi{Austroasiatic}
\def\langnames@fams@wals@lvk{Isolate}
\def\langnames@fams@wals@lvu{Austronesian}
\def\langnames@fams@wals@lwa{Atlantic-Congo}
\def\langnames@fams@wals@lwe{Austronesian}
\def\langnames@fams@wals@lwg{Atlantic-Congo}
\def\langnames@fams@wals@lwh{Tai-Kadai}
\def\langnames@fams@wals@lwl{Austroasiatic}
\def\langnames@fams@wals@lwm{Sino-Tibetan}
\def\langnames@fams@wals@lwo{Nilotic}
\def\langnames@fams@wals@lws{Artificial Language}
\def\langnames@fams@wals@lwt{Austronesian}
\def\langnames@fams@wals@lwu{Sino-Tibetan}
\def\langnames@fams@wals@lww{Austronesian}
\def\langnames@fams@wals@lxm{Austronesian}
\def\langnames@fams@wals@lya{Sino-Tibetan}
\def\langnames@fams@wals@lyg{Austroasiatic}
\def\langnames@fams@wals@lyn{Atlantic-Congo}
\def\langnames@fams@wals@lzh{Sino-Tibetan}
\def\langnames@fams@wals@lzl{Austronesian}
\def\langnames@fams@wals@lzn{Sino-Tibetan}
\def\langnames@fams@wals@lzz{Kartvelian}
\def\langnames@fams@wals@maa{Otomanguean}
\def\langnames@fams@wals@mab{Otomanguean}
\def\langnames@fams@wals@mad{Austronesian}
\def\langnames@fams@wals@mae{Atlantic-Congo}
\def\langnames@fams@wals@maf{Afro-Asiatic}
\def\langnames@fams@wals@mag{Indo-European}
\def\langnames@fams@wals@mah{Austronesian}
\def\langnames@fams@wals@mai{Indo-European}
\def\langnames@fams@wals@maj{Otomanguean}
\def\langnames@fams@wals@mak{Austronesian}
\def\langnames@fams@wals@mal{Dravidian}
\def\langnames@fams@wals@mam{Mayan}
\def\langnames@fams@wals@maq{Otomanguean}
\def\langnames@fams@wals@mar{Indo-European}
\def\langnames@fams@wals@mas{Nilotic}
\def\langnames@fams@wals@mat{Otomanguean}
\def\langnames@fams@wals@mau{Otomanguean}
\def\langnames@fams@wals@mav{Tupian}
\def\langnames@fams@wals@maw{Atlantic-Congo}
\def\langnames@fams@wals@max{Austronesian}
\def\langnames@fams@wals@maz{Otomanguean}
\def\langnames@fams@wals@mba{Austronesian}
\def\langnames@fams@wals@mbb{Austronesian}
\def\langnames@fams@wals@mbc{Cariban}
\def\langnames@fams@wals@mbd{Austronesian}
\def\langnames@fams@wals@mbe{Isolate}
\def\langnames@fams@wals@mbf{Austronesian}
\def\langnames@fams@wals@mbh{Austronesian}
\def\langnames@fams@wals@mbi{Austronesian}
\def\langnames@fams@wals@mbj{Naduhup}
\def\langnames@fams@wals@mbk{Austronesian}
\def\langnames@fams@wals@mbl{Nuclear-Macro-Je}
\def\langnames@fams@wals@mbn{Guahiboan}
\def\langnames@fams@wals@mbo{Atlantic-Congo}
\def\langnames@fams@wals@mbp{Chibchan}
\def\langnames@fams@wals@mbq{Austronesian}
\def\langnames@fams@wals@mbr{Kakua-Nukak}
\def\langnames@fams@wals@mbs{Austronesian}
\def\langnames@fams@wals@mbt{Austronesian}
\def\langnames@fams@wals@mbu{Atlantic-Congo}
\def\langnames@fams@wals@mbv{Atlantic-Congo}
\def\langnames@fams@wals@mbw{Nuclear Trans New Guinea}
\def\langnames@fams@wals@mbx{Sepik}
\def\langnames@fams@wals@mby{Indo-European}
\def\langnames@fams@wals@mbz{Otomanguean}
\def\langnames@fams@wals@mca{Matacoan}
\def\langnames@fams@wals@mcb{Arawakan}
\def\langnames@fams@wals@mcc{Anim}
\def\langnames@fams@wals@mcd{Pano-Tacanan}
\def\langnames@fams@wals@mce{Otomanguean}
\def\langnames@fams@wals@mcf{Pano-Tacanan}
\def\langnames@fams@wals@mcg{Cariban}
\def\langnames@fams@wals@mch{Cariban}
\def\langnames@fams@wals@mci{Nuclear Trans New Guinea}
\def\langnames@fams@wals@mcj{Atlantic-Congo}
\def\langnames@fams@wals@mck{Atlantic-Congo}
\def\langnames@fams@wals@mcl{Tucanoan}
\def\langnames@fams@wals@mcm{Indo-European}
\def\langnames@fams@wals@mcn{Afro-Asiatic}
\def\langnames@fams@wals@mco{Mixe-Zoque}
\def\langnames@fams@wals@mcp{Atlantic-Congo}
\def\langnames@fams@wals@mcq{Koiarian}
\def\langnames@fams@wals@mcr{Angan}
\def\langnames@fams@wals@mcs{Atlantic-Congo}
\def\langnames@fams@wals@mcu{Atlantic-Congo}
\def\langnames@fams@wals@mcv{Anim}
\def\langnames@fams@wals@mcw{Afro-Asiatic}
\def\langnames@fams@wals@mcx{Atlantic-Congo}
\def\langnames@fams@wals@mcy{Austronesian}
\def\langnames@fams@wals@mcz{Nuclear Trans New Guinea}
\def\langnames@fams@wals@mda{Atlantic-Congo}
\def\langnames@fams@wals@mdb{Kiwaian}
\def\langnames@fams@wals@mdc{Nuclear Trans New Guinea}
\def\langnames@fams@wals@mdd{Atlantic-Congo}
\def\langnames@fams@wals@mde{Maban}
\def\langnames@fams@wals@mdf{Uralic}
\def\langnames@fams@wals@mdg{Maban}
\def\langnames@fams@wals@mdh{Austronesian}
\def\langnames@fams@wals@mdi{Central Sudanic}
\def\langnames@fams@wals@mdj{Central Sudanic}
\def\langnames@fams@wals@mdk{Central Sudanic}
\def\langnames@fams@wals@mdl{Sign Language}
\def\langnames@fams@wals@mdm{Atlantic-Congo}
\def\langnames@fams@wals@mdn{Atlantic-Congo}
\def\langnames@fams@wals@mdp{Atlantic-Congo}
\def\langnames@fams@wals@mdq{Atlantic-Congo}
\def\langnames@fams@wals@mdr{Austronesian}
\def\langnames@fams@wals@mds{Manubaran}
\def\langnames@fams@wals@mdt{Atlantic-Congo}
\def\langnames@fams@wals@mdu{Atlantic-Congo}
\def\langnames@fams@wals@mdv{Otomanguean}
\def\langnames@fams@wals@mdw{Atlantic-Congo}
\def\langnames@fams@wals@mdx{Dizoid}
\def\langnames@fams@wals@mdy{Ta-Ne-Omotic}
\def\langnames@fams@wals@mdz{Tupian}
\def\langnames@fams@wals@mea{Atlantic-Congo}
\def\langnames@fams@wals@meb{Turama-Kikori}
\def\langnames@fams@wals@mec{Mangarrayi-Maran}
\def\langnames@fams@wals@med{Nuclear Trans New Guinea}
\def\langnames@fams@wals@mee{Austronesian}
\def\langnames@fams@wals@mef{Austroasiatic}
\def\langnames@fams@wals@meh{Otomanguean}
\def\langnames@fams@wals@mei{Nubian}
\def\langnames@fams@wals@mej{East Bird's Head}
\def\langnames@fams@wals@mek{Austronesian}
\def\langnames@fams@wals@mel{Austronesian}
\def\langnames@fams@wals@mem{Pama-Nyungan}
\def\langnames@fams@wals@men{Mande}
\def\langnames@fams@wals@meo{Austronesian}
\def\langnames@fams@wals@mep{Jarrakan}
\def\langnames@fams@wals@meq{Afro-Asiatic}
\def\langnames@fams@wals@mer{Atlantic-Congo}
\def\langnames@fams@wals@mes{Afro-Asiatic}
\def\langnames@fams@wals@met{Austronesian}
\def\langnames@fams@wals@meu{Austronesian}
\def\langnames@fams@wals@mev{Mande}
\def\langnames@fams@wals@mew{Afro-Asiatic}
\def\langnames@fams@wals@mey{Afro-Asiatic}
\def\langnames@fams@wals@mez{Algic}
\def\langnames@fams@wals@mfa{Austronesian}
\def\langnames@fams@wals@mfb{Austronesian}
\def\langnames@fams@wals@mfc{Atlantic-Congo}
\def\langnames@fams@wals@mfd{Atlantic-Congo}
\def\langnames@fams@wals@mfe{Indo-European}
\def\langnames@fams@wals@mff{Atlantic-Congo}
\def\langnames@fams@wals@mfg{Mande}
\def\langnames@fams@wals@mfh{Afro-Asiatic}
\def\langnames@fams@wals@mfi{Afro-Asiatic}
\def\langnames@fams@wals@mfj{Afro-Asiatic}
\def\langnames@fams@wals@mfk{Afro-Asiatic}
\def\langnames@fams@wals@mfl{Afro-Asiatic}
\def\langnames@fams@wals@mfm{Afro-Asiatic}
\def\langnames@fams@wals@mfn{Atlantic-Congo}
\def\langnames@fams@wals@mfo{Atlantic-Congo}
\def\langnames@fams@wals@mfp{Austronesian}
\def\langnames@fams@wals@mfq{Atlantic-Congo}
\def\langnames@fams@wals@mfr{Western Daly}
\def\langnames@fams@wals@mfs{Sign Language}
\def\langnames@fams@wals@mft{Austronesian}
\def\langnames@fams@wals@mfu{Atlantic-Congo}
\def\langnames@fams@wals@mfv{Atlantic-Congo}
\def\langnames@fams@wals@mfw{Kwalean}
\def\langnames@fams@wals@mfx{Ta-Ne-Omotic}
\def\langnames@fams@wals@mfy{Uto-Aztecan}
\def\langnames@fams@wals@mfz{Nilotic}
\def\langnames@fams@wals@mgb{Tamaic}
\def\langnames@fams@wals@mgc{Central Sudanic}
\def\langnames@fams@wals@mgd{Central Sudanic}
\def\langnames@fams@wals@mge{Central Sudanic}
\def\langnames@fams@wals@mgf{Bulaka River}
\def\langnames@fams@wals@mgg{Atlantic-Congo}
\def\langnames@fams@wals@mgh{Atlantic-Congo}
\def\langnames@fams@wals@mgi{Atlantic-Congo}
\def\langnames@fams@wals@mgj{Atlantic-Congo}
\def\langnames@fams@wals@mgk{Isolate}
\def\langnames@fams@wals@mgl{Austronesian}
\def\langnames@fams@wals@mgm{Austronesian}
\def\langnames@fams@wals@mgn{Atlantic-Congo}
\def\langnames@fams@wals@mgo{Atlantic-Congo}
\def\langnames@fams@wals@mgp{Sino-Tibetan}
\def\langnames@fams@wals@mgq{Atlantic-Congo}
\def\langnames@fams@wals@mgr{Atlantic-Congo}
\def\langnames@fams@wals@mgs{Atlantic-Congo}
\def\langnames@fams@wals@mgt{Keram}
\def\langnames@fams@wals@mgu{Mailuan}
\def\langnames@fams@wals@mgv{Atlantic-Congo}
\def\langnames@fams@wals@mgw{Atlantic-Congo}
\def\langnames@fams@wals@mgy{Atlantic-Congo}
\def\langnames@fams@wals@mgz{Atlantic-Congo}
\def\langnames@fams@wals@mha{Dravidian}
\def\langnames@fams@wals@mhb{Atlantic-Congo}
\def\langnames@fams@wals@mhc{Mayan}
\def\langnames@fams@wals@mhd{Atlantic-Congo}
\def\langnames@fams@wals@mhe{Austroasiatic}
\def\langnames@fams@wals@mhf{Nuclear Trans New Guinea}
\def\langnames@fams@wals@mhg{Marrku-Wurrugu}
\def\langnames@fams@wals@mhi{Central Sudanic}
\def\langnames@fams@wals@mhj{Mongolic-Khitan}
\def\langnames@fams@wals@mhk{Atlantic-Congo}
\def\langnames@fams@wals@mhl{Nuclear Trans New Guinea}
\def\langnames@fams@wals@mhm{Atlantic-Congo}
\def\langnames@fams@wals@mhn{Indo-European}
\def\langnames@fams@wals@mho{Atlantic-Congo}
\def\langnames@fams@wals@mhp{Austronesian}
\def\langnames@fams@wals@mhq{Siouan}
\def\langnames@fams@wals@mhr{Uralic}
\def\langnames@fams@wals@mhs{Austronesian}
\def\langnames@fams@wals@mht{Arawakan}
\def\langnames@fams@wals@mhu{Sino-Tibetan}
\def\langnames@fams@wals@mhw{Atlantic-Congo}
\def\langnames@fams@wals@mhx{Sino-Tibetan}
\def\langnames@fams@wals@mhy{Austronesian}
\def\langnames@fams@wals@mhz{Austronesian}
\def\langnames@fams@wals@mia{Algic}
\def\langnames@fams@wals@mib{Otomanguean}
\def\langnames@fams@wals@mic{Algic}
\def\langnames@fams@wals@mid{Afro-Asiatic}
\def\langnames@fams@wals@mie{Otomanguean}
\def\langnames@fams@wals@mif{Afro-Asiatic}
\def\langnames@fams@wals@mig{Otomanguean}
\def\langnames@fams@wals@mih{Otomanguean}
\def\langnames@fams@wals@mii{Otomanguean}
\def\langnames@fams@wals@mij{Atlantic-Congo}
\def\langnames@fams@wals@mik{Muskogean}
\def\langnames@fams@wals@mil{Otomanguean}
\def\langnames@fams@wals@mim{Otomanguean}
\def\langnames@fams@wals@min{Austronesian}
\def\langnames@fams@wals@mio{Otomanguean}
\def\langnames@fams@wals@mip{Otomanguean}
\def\langnames@fams@wals@miq{Misumalpan}
\def\langnames@fams@wals@mir{Mixe-Zoque}
\def\langnames@fams@wals@mit{Otomanguean}
\def\langnames@fams@wals@miu{Otomanguean}
\def\langnames@fams@wals@miw{Angan}
\def\langnames@fams@wals@mix{Otomanguean}
\def\langnames@fams@wals@miy{Otomanguean}
\def\langnames@fams@wals@miz{Otomanguean}
\def\langnames@fams@wals@mjc{Otomanguean}
\def\langnames@fams@wals@mjd{Maiduan}
\def\langnames@fams@wals@mje{Afro-Asiatic}
\def\langnames@fams@wals@mjg{Mongolic-Khitan}
\def\langnames@fams@wals@mjh{Atlantic-Congo}
\def\langnames@fams@wals@mji{Hmong-Mien}
\def\langnames@fams@wals@mjj{Nuclear Trans New Guinea}
\def\langnames@fams@wals@mjk{Austronesian}
\def\langnames@fams@wals@mjl{Indo-European}
\def\langnames@fams@wals@mjm{Austronesian}
\def\langnames@fams@wals@mjn{Nuclear Trans New Guinea}
\def\langnames@fams@wals@mjo{Dravidian}
\def\langnames@fams@wals@mjp{Dravidian}
\def\langnames@fams@wals@mjq{Dravidian}
\def\langnames@fams@wals@mjr{Dravidian}
\def\langnames@fams@wals@mjs{Afro-Asiatic}
\def\langnames@fams@wals@mjt{Dravidian}
\def\langnames@fams@wals@mjv{Dravidian}
\def\langnames@fams@wals@mjw{Sino-Tibetan}
\def\langnames@fams@wals@mjx{Austroasiatic}
\def\langnames@fams@wals@mjy{Algic}
\def\langnames@fams@wals@mjz{Indo-European}
\def\langnames@fams@wals@mka{Atlantic-Congo}
\def\langnames@fams@wals@mkb{Indo-European}
\def\langnames@fams@wals@mkc{Nuclear Torricelli}
\def\langnames@fams@wals@mkd{Indo-European}
\def\langnames@fams@wals@mke{Indo-European}
\def\langnames@fams@wals@mkf{Afro-Asiatic}
\def\langnames@fams@wals@mkg{Tai-Kadai}
\def\langnames@fams@wals@mki{Indo-European}
\def\langnames@fams@wals@mkj{Austronesian}
\def\langnames@fams@wals@mkk{Atlantic-Congo}
\def\langnames@fams@wals@mkl{Atlantic-Congo}
\def\langnames@fams@wals@mkm{Austronesian}
\def\langnames@fams@wals@mkn{Austronesian}
\def\langnames@fams@wals@mko{Atlantic-Congo}
\def\langnames@fams@wals@mkp{Yareban}
\def\langnames@fams@wals@mkq{Miwok-Costanoan}
\def\langnames@fams@wals@mkr{Nuclear Trans New Guinea}
\def\langnames@fams@wals@mks{Otomanguean}
\def\langnames@fams@wals@mkt{Austronesian}
\def\langnames@fams@wals@mku{Mande}
\def\langnames@fams@wals@mkv{Austronesian}
\def\langnames@fams@wals@mkw{Atlantic-Congo}
\def\langnames@fams@wals@mkx{Austronesian}
\def\langnames@fams@wals@mky{Austronesian}
\def\langnames@fams@wals@mkz{Timor-Alor-Pantar}
\def\langnames@fams@wals@mla{Austronesian}
\def\langnames@fams@wals@mlb{Atlantic-Congo}
\def\langnames@fams@wals@mlc{Tai-Kadai}
\def\langnames@fams@wals@mle{Ndu}
\def\langnames@fams@wals@mlf{Austroasiatic}
\def\langnames@fams@wals@mlh{Nuclear Trans New Guinea}
\def\langnames@fams@wals@mli{Austronesian}
\def\langnames@fams@wals@mlj{Afro-Asiatic}
\def\langnames@fams@wals@mlk{Atlantic-Congo}
\def\langnames@fams@wals@mll{Austronesian}
\def\langnames@fams@wals@mlm{Tai-Kadai}
\def\langnames@fams@wals@mln{Austronesian}
\def\langnames@fams@wals@mlo{Atlantic-Congo}
\def\langnames@fams@wals@mlp{Nuclear Trans New Guinea}
\def\langnames@fams@wals@mlq{Mande}
\def\langnames@fams@wals@mlr{Afro-Asiatic}
\def\langnames@fams@wals@mls{Maban}
\def\langnames@fams@wals@mlt{Afro-Asiatic}
\def\langnames@fams@wals@mlu{Austronesian}
\def\langnames@fams@wals@mlv{Austronesian}
\def\langnames@fams@wals@mlw{Afro-Asiatic}
\def\langnames@fams@wals@mlx{Austronesian}
\def\langnames@fams@wals@mma{Atlantic-Congo}
\def\langnames@fams@wals@mmb{Somahai}
\def\langnames@fams@wals@mmc{Otomanguean}
\def\langnames@fams@wals@mmd{Tai-Kadai}
\def\langnames@fams@wals@mme{Austronesian}
\def\langnames@fams@wals@mmf{Afro-Asiatic}
\def\langnames@fams@wals@mmg{Austronesian}
\def\langnames@fams@wals@mmh{Arawakan}
\def\langnames@fams@wals@mmi{Nuclear Trans New Guinea}
\def\langnames@fams@wals@mmj{Austroasiatic}
\def\langnames@fams@wals@mmk{Dravidian}
\def\langnames@fams@wals@mml{Austroasiatic}
\def\langnames@fams@wals@mmm{Austronesian}
\def\langnames@fams@wals@mmn{Austronesian}
\def\langnames@fams@wals@mmo{Austronesian}
\def\langnames@fams@wals@mmp{Amto-Musan}
\def\langnames@fams@wals@mmq{Nuclear Trans New Guinea}
\def\langnames@fams@wals@mmr{Hmong-Mien}
\def\langnames@fams@wals@mmt{Austronesian}
\def\langnames@fams@wals@mmu{Atlantic-Congo}
\def\langnames@fams@wals@mmv{Tucanoan}
\def\langnames@fams@wals@mmw{Austronesian}
\def\langnames@fams@wals@mmx{Austronesian}
\def\langnames@fams@wals@mmy{Afro-Asiatic}
\def\langnames@fams@wals@mmz{Atlantic-Congo}
\def\langnames@fams@wals@mna{Austronesian}
\def\langnames@fams@wals@mnb{Austronesian}
\def\langnames@fams@wals@mnc{Tungusic}
\def\langnames@fams@wals@mnd{Tupian}
\def\langnames@fams@wals@mne{Central Sudanic}
\def\langnames@fams@wals@mnf{Atlantic-Congo}
\def\langnames@fams@wals@mng{Austroasiatic}
\def\langnames@fams@wals@mnh{Atlantic-Congo}
\def\langnames@fams@wals@mni{Sino-Tibetan}
\def\langnames@fams@wals@mnj{Indo-European}
\def\langnames@fams@wals@mnk{Mande}
\def\langnames@fams@wals@mnl{Austronesian}
\def\langnames@fams@wals@mnm{Dagan}
\def\langnames@fams@wals@mnn{Austroasiatic}
\def\langnames@fams@wals@mnp{Sino-Tibetan}
\def\langnames@fams@wals@mnq{Austroasiatic}
\def\langnames@fams@wals@mnr{Uto-Aztecan}
\def\langnames@fams@wals@mns{Uralic}
\def\langnames@fams@wals@mnt{Pama-Nyungan}
\def\langnames@fams@wals@mnu{Mairasic}
\def\langnames@fams@wals@mnv{Austronesian}
\def\langnames@fams@wals@mnw{Austroasiatic}
\def\langnames@fams@wals@mnx{East Bird's Head}
\def\langnames@fams@wals@mny{Atlantic-Congo}
\def\langnames@fams@wals@mnz{Nuclear Trans New Guinea}
\def\langnames@fams@wals@moa{Mande}
\def\langnames@fams@wals@moc{Guaicuruan}
\def\langnames@fams@wals@mod{Pidgin}
\def\langnames@fams@wals@moe{Algic}
\def\langnames@fams@wals@mof{Algic}
\def\langnames@fams@wals@mog{Austronesian}
\def\langnames@fams@wals@moh{Iroquoian}
\def\langnames@fams@wals@moi{Atlantic-Congo}
\def\langnames@fams@wals@moj{Atlantic-Congo}
\def\langnames@fams@wals@mok{Isolate}
\def\langnames@fams@wals@mom{Otomanguean}
\def\langnames@fams@wals@moo{Austroasiatic}
\def\langnames@fams@wals@mop{Mayan}
\def\langnames@fams@wals@moq{Isolate}
\def\langnames@fams@wals@mor{Heibanic}
\def\langnames@fams@wals@mos{Atlantic-Congo}
\def\langnames@fams@wals@mot{Chibchan}
\def\langnames@fams@wals@mou{Afro-Asiatic}
\def\langnames@fams@wals@mov{Cochimi-Yuman}
\def\langnames@fams@wals@mow{Atlantic-Congo}
\def\langnames@fams@wals@mox{Austronesian}
\def\langnames@fams@wals@moy{Ta-Ne-Omotic}
\def\langnames@fams@wals@moz{Afro-Asiatic}
\def\langnames@fams@wals@mpa{Atlantic-Congo}
\def\langnames@fams@wals@mpb{Northern Daly}
\def\langnames@fams@wals@mpc{Mangarrayi-Maran}
\def\langnames@fams@wals@mpd{Arawakan}
\def\langnames@fams@wals@mpe{Surmic}
\def\langnames@fams@wals@mpg{Afro-Asiatic}
\def\langnames@fams@wals@mph{Iwaidjan Proper}
\def\langnames@fams@wals@mpi{Afro-Asiatic}
\def\langnames@fams@wals@mpj{Pama-Nyungan}
\def\langnames@fams@wals@mpk{Afro-Asiatic}
\def\langnames@fams@wals@mpl{Austronesian}
\def\langnames@fams@wals@mpm{Otomanguean}
\def\langnames@fams@wals@mpn{Austronesian}
\def\langnames@fams@wals@mpo{Austronesian}
\def\langnames@fams@wals@mpp{Nuclear Trans New Guinea}
\def\langnames@fams@wals@mpq{Pano-Tacanan}
\def\langnames@fams@wals@mpr{Austronesian}
\def\langnames@fams@wals@mps{Teberan}
\def\langnames@fams@wals@mpt{Nuclear Trans New Guinea}
\def\langnames@fams@wals@mpu{Tupian}
\def\langnames@fams@wals@mpv{Nuclear Trans New Guinea}
\def\langnames@fams@wals@mpw{Arawakan}
\def\langnames@fams@wals@mpx{Austronesian}
\def\langnames@fams@wals@mpy{Austronesian}
\def\langnames@fams@wals@mpz{Sino-Tibetan}
\def\langnames@fams@wals@mqa{Austronesian}
\def\langnames@fams@wals@mqb{Afro-Asiatic}
\def\langnames@fams@wals@mqe{Nuclear Trans New Guinea}
\def\langnames@fams@wals@mqf{Somahai}
\def\langnames@fams@wals@mqg{Austronesian}
\def\langnames@fams@wals@mqh{Otomanguean}
\def\langnames@fams@wals@mqi{Austronesian}
\def\langnames@fams@wals@mqj{Austronesian}
\def\langnames@fams@wals@mqk{Austronesian}
\def\langnames@fams@wals@mql{Atlantic-Congo}
\def\langnames@fams@wals@mqm{Austronesian}
\def\langnames@fams@wals@mqn{Austronesian}
\def\langnames@fams@wals@mqo{North Halmahera}
\def\langnames@fams@wals@mqp{Austronesian}
\def\langnames@fams@wals@mqq{Austronesian}
\def\langnames@fams@wals@mqr{Tor-Orya}
\def\langnames@fams@wals@mqs{North Halmahera}
\def\langnames@fams@wals@mqt{Austroasiatic}
\def\langnames@fams@wals@mqu{Nilotic}
\def\langnames@fams@wals@mqv{Nuclear Trans New Guinea}
\def\langnames@fams@wals@mqw{Nuclear Trans New Guinea}
\def\langnames@fams@wals@mqx{Austronesian}
\def\langnames@fams@wals@mqy{Austronesian}
\def\langnames@fams@wals@mqz{Austronesian}
\def\langnames@fams@wals@mra{Austroasiatic}
\def\langnames@fams@wals@mrb{Austronesian}
\def\langnames@fams@wals@mrc{Cochimi-Yuman}
\def\langnames@fams@wals@mrd{Sino-Tibetan}
\def\langnames@fams@wals@mre{Sign Language}
\def\langnames@fams@wals@mrf{Isolate}
\def\langnames@fams@wals@mrg{Sino-Tibetan}
\def\langnames@fams@wals@mrh{Sino-Tibetan}
\def\langnames@fams@wals@mri{Austronesian}
\def\langnames@fams@wals@mrj{Uralic}
\def\langnames@fams@wals@mrk{Austronesian}
\def\langnames@fams@wals@mrl{Austronesian}
\def\langnames@fams@wals@mrm{Austronesian}
\def\langnames@fams@wals@mrn{Austronesian}
\def\langnames@fams@wals@mro{Sino-Tibetan}
\def\langnames@fams@wals@mrp{Austronesian}
\def\langnames@fams@wals@mrq{Austronesian}
\def\langnames@fams@wals@mrr{Dravidian}
\def\langnames@fams@wals@mrs{Austronesian}
\def\langnames@fams@wals@mrt{Afro-Asiatic}
\def\langnames@fams@wals@mru{Atlantic-Congo}
\def\langnames@fams@wals@mrv{Austronesian}
\def\langnames@fams@wals@mrw{Austronesian}
\def\langnames@fams@wals@mrx{Tor-Orya}
\def\langnames@fams@wals@mry{Austronesian}
\def\langnames@fams@wals@mrz{Anim}
\def\langnames@fams@wals@msb{Austronesian}
\def\langnames@fams@wals@msc{Mande}
\def\langnames@fams@wals@msd{Sign Language}
\def\langnames@fams@wals@mse{Afro-Asiatic}
\def\langnames@fams@wals@msf{Nimboranic}
\def\langnames@fams@wals@msg{West Bird's Head}
\def\langnames@fams@wals@msh{Austronesian}
\def\langnames@fams@wals@msi{Austronesian}
\def\langnames@fams@wals@msj{Atlantic-Congo}
\def\langnames@fams@wals@msk{Austronesian}
\def\langnames@fams@wals@msl{Isolate}
\def\langnames@fams@wals@msm{Austronesian}
\def\langnames@fams@wals@msn{Austronesian}
\def\langnames@fams@wals@mso{Mombum-Koneraw}
\def\langnames@fams@wals@msp{Tupian}
\def\langnames@fams@wals@msq{Austronesian}
\def\langnames@fams@wals@msr{Sign Language}
\def\langnames@fams@wals@mss{Austronesian}
\def\langnames@fams@wals@msu{Austronesian}
\def\langnames@fams@wals@msv{Afro-Asiatic}
\def\langnames@fams@wals@msw{Atlantic-Congo}
\def\langnames@fams@wals@msx{Nuclear Trans New Guinea}
\def\langnames@fams@wals@msy{Lower Sepik-Ramu}
\def\langnames@fams@wals@msz{Nuclear Trans New Guinea}
\def\langnames@fams@wals@mta{Austronesian}
\def\langnames@fams@wals@mtb{Atlantic-Congo}
\def\langnames@fams@wals@mtc{Nuclear Trans New Guinea}
\def\langnames@fams@wals@mtd{Austronesian}
\def\langnames@fams@wals@mte{Austronesian}
\def\langnames@fams@wals@mtf{Lower Sepik-Ramu}
\def\langnames@fams@wals@mtg{Nuclear Trans New Guinea}
\def\langnames@fams@wals@mth{Austronesian}
\def\langnames@fams@wals@mti{Dagan}
\def\langnames@fams@wals@mtj{East Bird's Head}
\def\langnames@fams@wals@mtk{Atlantic-Congo}
\def\langnames@fams@wals@mtl{Afro-Asiatic}
\def\langnames@fams@wals@mtm{Uralic}
\def\langnames@fams@wals@mtn{Misumalpan}
\def\langnames@fams@wals@mto{Mixe-Zoque}
\def\langnames@fams@wals@mtp{Matacoan}
\def\langnames@fams@wals@mtq{Austroasiatic}
\def\langnames@fams@wals@mtr{Indo-European}
\def\langnames@fams@wals@mts{Pano-Tacanan}
\def\langnames@fams@wals@mtt{Austronesian}
\def\langnames@fams@wals@mtu{Otomanguean}
\def\langnames@fams@wals@mtv{Nuclear Trans New Guinea}
\def\langnames@fams@wals@mtw{Austronesian}
\def\langnames@fams@wals@mtx{Otomanguean}
\def\langnames@fams@wals@mty{Nuclear Torricelli}
\def\langnames@fams@wals@mua{Atlantic-Congo}
\def\langnames@fams@wals@mub{Afro-Asiatic}
\def\langnames@fams@wals@muc{Atlantic-Congo}
\def\langnames@fams@wals@mud{Eskimo-Aleut}
\def\langnames@fams@wals@mue{Mixed Language}
\def\langnames@fams@wals@mug{Afro-Asiatic}
\def\langnames@fams@wals@muh{Atlantic-Congo}
\def\langnames@fams@wals@mui{Austronesian}
\def\langnames@fams@wals@muj{Afro-Asiatic}
\def\langnames@fams@wals@muk{Sino-Tibetan}
\def\langnames@fams@wals@mum{Austronesian}
\def\langnames@fams@wals@muo{Atlantic-Congo}
\def\langnames@fams@wals@mup{Indo-European}
\def\langnames@fams@wals@muq{Hmong-Mien}
\def\langnames@fams@wals@mur{Surmic}
\def\langnames@fams@wals@mus{Muskogean}
\def\langnames@fams@wals@mut{Dravidian}
\def\langnames@fams@wals@muu{Afro-Asiatic}
\def\langnames@fams@wals@muv{Dravidian}
\def\langnames@fams@wals@mux{Nuclear Trans New Guinea}
\def\langnames@fams@wals@muy{Afro-Asiatic}
\def\langnames@fams@wals@muz{Surmic}
\def\langnames@fams@wals@mva{Austronesian}
\def\langnames@fams@wals@mvb{Athabaskan-Eyak-Tlingit}
\def\langnames@fams@wals@mvd{Austronesian}
\def\langnames@fams@wals@mve{Indo-European}
\def\langnames@fams@wals@mvf{Mongolic-Khitan}
\def\langnames@fams@wals@mvg{Otomanguean}
\def\langnames@fams@wals@mvh{Afro-Asiatic}
\def\langnames@fams@wals@mvi{Japonic}
\def\langnames@fams@wals@mvk{Yuat}
\def\langnames@fams@wals@mvl{Pama-Nyungan}
\def\langnames@fams@wals@mvn{Austronesian}
\def\langnames@fams@wals@mvo{Austronesian}
\def\langnames@fams@wals@mvp{Austronesian}
\def\langnames@fams@wals@mvq{Nuclear Trans New Guinea}
\def\langnames@fams@wals@mvr{Austronesian}
\def\langnames@fams@wals@mvs{Isolate}
\def\langnames@fams@wals@mvt{Austronesian}
\def\langnames@fams@wals@mvu{Maban}
\def\langnames@fams@wals@mvv{Austronesian}
\def\langnames@fams@wals@mvw{Atlantic-Congo}
\def\langnames@fams@wals@mvx{Austronesian}
\def\langnames@fams@wals@mvy{Indo-European}
\def\langnames@fams@wals@mvz{Afro-Asiatic}
\def\langnames@fams@wals@mwa{Austronesian}
\def\langnames@fams@wals@mwb{Nuclear Torricelli}
\def\langnames@fams@wals@mwc{Austronesian}
\def\langnames@fams@wals@mwe{Atlantic-Congo}
\def\langnames@fams@wals@mwf{Southern Daly}
\def\langnames@fams@wals@mwg{Austronesian}
\def\langnames@fams@wals@mwh{Austronesian}
\def\langnames@fams@wals@mwi{Austronesian}
\def\langnames@fams@wals@mwk{Mande}
\def\langnames@fams@wals@mwl{Indo-European}
\def\langnames@fams@wals@mwm{Central Sudanic}
\def\langnames@fams@wals@mwn{Atlantic-Congo}
\def\langnames@fams@wals@mwo{Austronesian}
\def\langnames@fams@wals@mwp{Pama-Nyungan}
\def\langnames@fams@wals@mwq{Sino-Tibetan}
\def\langnames@fams@wals@mws{Atlantic-Congo}
\def\langnames@fams@wals@mwt{Austronesian}
\def\langnames@fams@wals@mwu{Central Sudanic}
\def\langnames@fams@wals@mwv{Austronesian}
\def\langnames@fams@wals@mww{Hmong-Mien}
\def\langnames@fams@wals@mwy{Nilotic}
\def\langnames@fams@wals@mwz{Atlantic-Congo}
\def\langnames@fams@wals@mxa{Otomanguean}
\def\langnames@fams@wals@mxb{Otomanguean}
\def\langnames@fams@wals@mxc{Atlantic-Congo}
\def\langnames@fams@wals@mxd{Austronesian}
\def\langnames@fams@wals@mxe{Austronesian}
\def\langnames@fams@wals@mxf{Afro-Asiatic}
\def\langnames@fams@wals@mxg{Atlantic-Congo}
\def\langnames@fams@wals@mxh{Central Sudanic}
\def\langnames@fams@wals@mxi{Indo-European}
\def\langnames@fams@wals@mxj{Sino-Tibetan}
\def\langnames@fams@wals@mxk{Bogia}
\def\langnames@fams@wals@mxl{Atlantic-Congo}
\def\langnames@fams@wals@mxm{Austronesian}
\def\langnames@fams@wals@mxn{West Bird's Head}
\def\langnames@fams@wals@mxo{Atlantic-Congo}
\def\langnames@fams@wals@mxp{Mixe-Zoque}
\def\langnames@fams@wals@mxq{Mixe-Zoque}
\def\langnames@fams@wals@mxr{Austronesian}
\def\langnames@fams@wals@mxs{Otomanguean}
\def\langnames@fams@wals@mxt{Otomanguean}
\def\langnames@fams@wals@mxu{Afro-Asiatic}
\def\langnames@fams@wals@mxv{Otomanguean}
\def\langnames@fams@wals@mxw{Yam}
\def\langnames@fams@wals@mxx{Mande}
\def\langnames@fams@wals@mxy{Otomanguean}
\def\langnames@fams@wals@mxz{Austronesian}
\def\langnames@fams@wals@mya{Sino-Tibetan}
\def\langnames@fams@wals@myb{Central Sudanic}
\def\langnames@fams@wals@mye{Atlantic-Congo}
\def\langnames@fams@wals@myf{Blue Nile Mao}
\def\langnames@fams@wals@myg{Atlantic-Congo}
\def\langnames@fams@wals@myh{Wakashan}
\def\langnames@fams@wals@myj{Atlantic-Congo}
\def\langnames@fams@wals@myk{Atlantic-Congo}
\def\langnames@fams@wals@myl{Austronesian}
\def\langnames@fams@wals@mym{Surmic}
\def\langnames@fams@wals@myo{Ta-Ne-Omotic}
\def\langnames@fams@wals@myp{Isolate}
\def\langnames@fams@wals@myr{Isolate}
\def\langnames@fams@wals@mys{Afro-Asiatic}
\def\langnames@fams@wals@myu{Tupian}
\def\langnames@fams@wals@myv{Uralic}
\def\langnames@fams@wals@myw{Austronesian}
\def\langnames@fams@wals@myx{Atlantic-Congo}
\def\langnames@fams@wals@myy{Tucanoan}
\def\langnames@fams@wals@myz{Afro-Asiatic}
\def\langnames@fams@wals@mza{Otomanguean}
\def\langnames@fams@wals@mzb{Afro-Asiatic}
\def\langnames@fams@wals@mzc{Sign Language}
\def\langnames@fams@wals@mzd{Atlantic-Congo}
\def\langnames@fams@wals@mze{Mailuan}
\def\langnames@fams@wals@mzg{Sign Language}
\def\langnames@fams@wals@mzh{Matacoan}
\def\langnames@fams@wals@mzi{Otomanguean}
\def\langnames@fams@wals@mzj{Mande}
\def\langnames@fams@wals@mzk{Atlantic-Congo}
\def\langnames@fams@wals@mzl{Mixe-Zoque}
\def\langnames@fams@wals@mzm{Atlantic-Congo}
\def\langnames@fams@wals@mzn{Indo-European}
\def\langnames@fams@wals@mzo{Cariban}
\def\langnames@fams@wals@mzp{Isolate}
\def\langnames@fams@wals@mzq{Austronesian}
\def\langnames@fams@wals@mzr{Pano-Tacanan}
\def\langnames@fams@wals@mzs{Indo-European}
\def\langnames@fams@wals@mzt{Austroasiatic}
\def\langnames@fams@wals@mzu{Lower Sepik-Ramu}
\def\langnames@fams@wals@mzv{Atlantic-Congo}
\def\langnames@fams@wals@mzw{Atlantic-Congo}
\def\langnames@fams@wals@mzy{Sign Language}
\def\langnames@fams@wals@mzz{Austronesian}
\def\langnames@fams@wals@naa{Namla-Tofanma}
\def\langnames@fams@wals@nab{Nambiquaran}
\def\langnames@fams@wals@nac{Nuclear Trans New Guinea}
\def\langnames@fams@wals@nae{Austronesian}
\def\langnames@fams@wals@naf{Nuclear Trans New Guinea}
\def\langnames@fams@wals@nag{Indo-European}
\def\langnames@fams@wals@naj{Atlantic-Congo}
\def\langnames@fams@wals@nak{Austronesian}
\def\langnames@fams@wals@nal{Austronesian}
\def\langnames@fams@wals@nam{Southern Daly}
\def\langnames@fams@wals@nao{Sino-Tibetan}
\def\langnames@fams@wals@nap{Indo-European}
\def\langnames@fams@wals@naq{Khoe-Kwadi}
\def\langnames@fams@wals@nar{Atlantic-Congo}
\def\langnames@fams@wals@nas{South Bougainville}
\def\langnames@fams@wals@nat{Atlantic-Congo}
\def\langnames@fams@wals@nau{Austronesian}
\def\langnames@fams@wals@nav{Athabaskan-Eyak-Tlingit}
\def\langnames@fams@wals@naw{Atlantic-Congo}
\def\langnames@fams@wals@nax{Left May}
\def\langnames@fams@wals@nay{Pama-Nyungan}
\def\langnames@fams@wals@naz{Uto-Aztecan}
\def\langnames@fams@wals@nba{Atlantic-Congo}
\def\langnames@fams@wals@nbb{Atlantic-Congo}
\def\langnames@fams@wals@nbc{Sino-Tibetan}
\def\langnames@fams@wals@nbd{Atlantic-Congo}
\def\langnames@fams@wals@nbe{Sino-Tibetan}
\def\langnames@fams@wals@nbg{Unattested}
\def\langnames@fams@wals@nbh{Afro-Asiatic}
\def\langnames@fams@wals@nbi{Sino-Tibetan}
\def\langnames@fams@wals@nbj{Pama-Nyungan}
\def\langnames@fams@wals@nbk{Nuclear Trans New Guinea}
\def\langnames@fams@wals@nbm{Atlantic-Congo}
\def\langnames@fams@wals@nbn{Austronesian}
\def\langnames@fams@wals@nbo{Atlantic-Congo}
\def\langnames@fams@wals@nbp{Atlantic-Congo}
\def\langnames@fams@wals@nbq{Nuclear Trans New Guinea}
\def\langnames@fams@wals@nbr{Atlantic-Congo}
\def\langnames@fams@wals@nbs{Sign Language}
\def\langnames@fams@wals@nbt{Sino-Tibetan}
\def\langnames@fams@wals@nbu{Sino-Tibetan}
\def\langnames@fams@wals@nbv{Atlantic-Congo}
\def\langnames@fams@wals@nbw{Atlantic-Congo}
\def\langnames@fams@wals@nbx{Pama-Nyungan}
\def\langnames@fams@wals@nby{Border}
\def\langnames@fams@wals@nca{Nuclear Trans New Guinea}
\def\langnames@fams@wals@ncb{Austroasiatic}
\def\langnames@fams@wals@ncc{Austronesian}
\def\langnames@fams@wals@ncd{Sino-Tibetan}
\def\langnames@fams@wals@nce{Isolate}
\def\langnames@fams@wals@ncf{Austronesian}
\def\langnames@fams@wals@ncg{Tsimshian}
\def\langnames@fams@wals@nch{Uto-Aztecan}
\def\langnames@fams@wals@nci{Uto-Aztecan}
\def\langnames@fams@wals@ncj{Uto-Aztecan}
\def\langnames@fams@wals@nck{Maningrida}
\def\langnames@fams@wals@ncl{Uto-Aztecan}
\def\langnames@fams@wals@ncm{Yam}
\def\langnames@fams@wals@ncn{Austronesian}
\def\langnames@fams@wals@nco{South Bougainville}
\def\langnames@fams@wals@ncq{Austroasiatic}
\def\langnames@fams@wals@ncr{Atlantic-Congo}
\def\langnames@fams@wals@ncs{Sign Language}
\def\langnames@fams@wals@nct{Sino-Tibetan}
\def\langnames@fams@wals@ncu{Atlantic-Congo}
\def\langnames@fams@wals@ncx{Uto-Aztecan}
\def\langnames@fams@wals@ncz{Isolate}
\def\langnames@fams@wals@nda{Atlantic-Congo}
\def\langnames@fams@wals@ndb{Atlantic-Congo}
\def\langnames@fams@wals@ndc{Atlantic-Congo}
\def\langnames@fams@wals@ndd{Atlantic-Congo}
\def\langnames@fams@wals@nde{Atlantic-Congo}
\def\langnames@fams@wals@ndg{Atlantic-Congo}
\def\langnames@fams@wals@ndh{Atlantic-Congo}
\def\langnames@fams@wals@ndi{Atlantic-Congo}
\def\langnames@fams@wals@ndj{Atlantic-Congo}
\def\langnames@fams@wals@ndk{Atlantic-Congo}
\def\langnames@fams@wals@ndl{Atlantic-Congo}
\def\langnames@fams@wals@ndm{Afro-Asiatic}
\def\langnames@fams@wals@ndn{Atlantic-Congo}
\def\langnames@fams@wals@ndo{Atlantic-Congo}
\def\langnames@fams@wals@ndp{Central Sudanic}
\def\langnames@fams@wals@ndq{Atlantic-Congo}
\def\langnames@fams@wals@ndr{Atlantic-Congo}
\def\langnames@fams@wals@nds{Indo-European}
\def\langnames@fams@wals@ndt{Atlantic-Congo}
\def\langnames@fams@wals@ndu{Atlantic-Congo}
\def\langnames@fams@wals@ndv{Atlantic-Congo}
\def\langnames@fams@wals@ndw{Atlantic-Congo}
\def\langnames@fams@wals@ndx{Nuclear Trans New Guinea}
\def\langnames@fams@wals@ndy{Central Sudanic}
\def\langnames@fams@wals@ndz{Atlantic-Congo}
\def\langnames@fams@wals@neb{Mande}
\def\langnames@fams@wals@nec{Timor-Alor-Pantar}
\def\langnames@fams@wals@nee{Austronesian}
\def\langnames@fams@wals@nef{Pidgin}
\def\langnames@fams@wals@neg{Tungusic}
\def\langnames@fams@wals@neh{Sino-Tibetan}
\def\langnames@fams@wals@nej{Nuclear Trans New Guinea}
\def\langnames@fams@wals@nek{Austronesian}
\def\langnames@fams@wals@nem{Austronesian}
\def\langnames@fams@wals@nen{Austronesian}
\def\langnames@fams@wals@neo{Unclassifiable}
\def\langnames@fams@wals@neq{Mixe-Zoque}
\def\langnames@fams@wals@ner{Konda-Yahadian}
\def\langnames@fams@wals@nes{Sino-Tibetan}
\def\langnames@fams@wals@net{Nuclear Trans New Guinea}
\def\langnames@fams@wals@neu{Artificial Language}
\def\langnames@fams@wals@nev{Austroasiatic}
\def\langnames@fams@wals@new{Sino-Tibetan}
\def\langnames@fams@wals@nex{Yam}
\def\langnames@fams@wals@ney{Kru}
\def\langnames@fams@wals@nez{Sahaptian}
\def\langnames@fams@wals@nfa{Austronesian}
\def\langnames@fams@wals@nfd{Atlantic-Congo}
\def\langnames@fams@wals@nfl{Austronesian}
\def\langnames@fams@wals@nfr{Atlantic-Congo}
\def\langnames@fams@wals@nfu{Atlantic-Congo}
\def\langnames@fams@wals@nga{Atlantic-Congo}
\def\langnames@fams@wals@ngb{Atlantic-Congo}
\def\langnames@fams@wals@ngc{Atlantic-Congo}
\def\langnames@fams@wals@ngd{Atlantic-Congo}
\def\langnames@fams@wals@nge{Atlantic-Congo}
\def\langnames@fams@wals@ngg{Atlantic-Congo}
\def\langnames@fams@wals@ngh{Tuu}
\def\langnames@fams@wals@ngi{Afro-Asiatic}
\def\langnames@fams@wals@ngj{Atlantic-Congo}
\def\langnames@fams@wals@ngk{Gunwinyguan}
\def\langnames@fams@wals@ngl{Atlantic-Congo}
\def\langnames@fams@wals@ngm{Speech Register}
\def\langnames@fams@wals@ngn{Atlantic-Congo}
\def\langnames@fams@wals@ngp{Atlantic-Congo}
\def\langnames@fams@wals@ngq{Atlantic-Congo}
\def\langnames@fams@wals@ngr{Austronesian}
\def\langnames@fams@wals@ngs{Afro-Asiatic}
\def\langnames@fams@wals@ngt{Austroasiatic}
\def\langnames@fams@wals@ngu{Uto-Aztecan}
\def\langnames@fams@wals@ngv{Atlantic-Congo}
\def\langnames@fams@wals@ngw{Afro-Asiatic}
\def\langnames@fams@wals@ngx{Afro-Asiatic}
\def\langnames@fams@wals@ngy{Atlantic-Congo}
\def\langnames@fams@wals@ngz{Atlantic-Congo}
\def\langnames@fams@wals@nha{Pama-Nyungan}
\def\langnames@fams@wals@nhb{Mande}
\def\langnames@fams@wals@nhc{Uto-Aztecan}
\def\langnames@fams@wals@nhd{Tupian}
\def\langnames@fams@wals@nhe{Uto-Aztecan}
\def\langnames@fams@wals@nhf{Pama-Nyungan}
\def\langnames@fams@wals@nhg{Uto-Aztecan}
\def\langnames@fams@wals@nhh{Indo-European}
\def\langnames@fams@wals@nhi{Uto-Aztecan}
\def\langnames@fams@wals@nhk{Uto-Aztecan}
\def\langnames@fams@wals@nhm{Uto-Aztecan}
\def\langnames@fams@wals@nhn{Uto-Aztecan}
\def\langnames@fams@wals@nho{Austronesian}
\def\langnames@fams@wals@nhp{Uto-Aztecan}
\def\langnames@fams@wals@nhq{Uto-Aztecan}
\def\langnames@fams@wals@nhr{Khoe-Kwadi}
\def\langnames@fams@wals@nht{Uto-Aztecan}
\def\langnames@fams@wals@nhu{Atlantic-Congo}
\def\langnames@fams@wals@nhv{Uto-Aztecan}
\def\langnames@fams@wals@nhw{Uto-Aztecan}
\def\langnames@fams@wals@nhx{Uto-Aztecan}
\def\langnames@fams@wals@nhy{Uto-Aztecan}
\def\langnames@fams@wals@nhz{Uto-Aztecan}
\def\langnames@fams@wals@nia{Austronesian}
\def\langnames@fams@wals@nib{Nuclear Trans New Guinea}
\def\langnames@fams@wals@nid{Gunwinyguan}
\def\langnames@fams@wals@nie{Atlantic-Congo}
\def\langnames@fams@wals@nif{Nuclear Trans New Guinea}
\def\langnames@fams@wals@nig{Gunwinyguan}
\def\langnames@fams@wals@nih{Atlantic-Congo}
\def\langnames@fams@wals@nii{Nuclear Trans New Guinea}
\def\langnames@fams@wals@nij{Austronesian}
\def\langnames@fams@wals@nik{Austroasiatic}
\def\langnames@fams@wals@nil{Austronesian}
\def\langnames@fams@wals@nim{Atlantic-Congo}
\def\langnames@fams@wals@nin{Atlantic-Congo}
\def\langnames@fams@wals@nio{Uralic}
\def\langnames@fams@wals@niq{Nilotic}
\def\langnames@fams@wals@nir{Nimboranic}
\def\langnames@fams@wals@nis{Nuclear Trans New Guinea}
\def\langnames@fams@wals@nit{Dravidian}
\def\langnames@fams@wals@niu{Austronesian}
\def\langnames@fams@wals@niv{Nivkh}
\def\langnames@fams@wals@niw{Left May}
\def\langnames@fams@wals@nix{Atlantic-Congo}
\def\langnames@fams@wals@niy{Central Sudanic}
\def\langnames@fams@wals@niz{Nuclear Torricelli}
\def\langnames@fams@wals@nja{Afro-Asiatic}
\def\langnames@fams@wals@njb{Sino-Tibetan}
\def\langnames@fams@wals@njh{Sino-Tibetan}
\def\langnames@fams@wals@nji{Mirndi}
\def\langnames@fams@wals@njj{Atlantic-Congo}
\def\langnames@fams@wals@njl{Dajuic}
\def\langnames@fams@wals@njm{Sino-Tibetan}
\def\langnames@fams@wals@njn{Sino-Tibetan}
\def\langnames@fams@wals@njo{Sino-Tibetan}
\def\langnames@fams@wals@njr{Atlantic-Congo}
\def\langnames@fams@wals@njs{Geelvink Bay}
\def\langnames@fams@wals@njt{Pidgin}
\def\langnames@fams@wals@nju{Pama-Nyungan}
\def\langnames@fams@wals@njx{Atlantic-Congo}
\def\langnames@fams@wals@njy{Atlantic-Congo}
\def\langnames@fams@wals@njz{Sino-Tibetan}
\def\langnames@fams@wals@nka{Atlantic-Congo}
\def\langnames@fams@wals@nkb{Sino-Tibetan}
\def\langnames@fams@wals@nkc{Atlantic-Congo}
\def\langnames@fams@wals@nkd{Sino-Tibetan}
\def\langnames@fams@wals@nke{Austronesian}
\def\langnames@fams@wals@nkg{Nuclear Trans New Guinea}
\def\langnames@fams@wals@nkh{Sino-Tibetan}
\def\langnames@fams@wals@nki{Sino-Tibetan}
\def\langnames@fams@wals@nkj{Nuclear Trans New Guinea}
\def\langnames@fams@wals@nkk{Austronesian}
\def\langnames@fams@wals@nkm{Yam}
\def\langnames@fams@wals@nkn{Atlantic-Congo}
\def\langnames@fams@wals@nko{Atlantic-Congo}
\def\langnames@fams@wals@nkp{Austronesian}
\def\langnames@fams@wals@nkq{Atlantic-Congo}
\def\langnames@fams@wals@nkr{Austronesian}
\def\langnames@fams@wals@nks{Nuclear Trans New Guinea}
\def\langnames@fams@wals@nkt{Atlantic-Congo}
\def\langnames@fams@wals@nku{Atlantic-Congo}
\def\langnames@fams@wals@nkv{Atlantic-Congo}
\def\langnames@fams@wals@nkw{Atlantic-Congo}
\def\langnames@fams@wals@nkx{Ijoid}
\def\langnames@fams@wals@nkz{Atlantic-Congo}
\def\langnames@fams@wals@nla{Atlantic-Congo}
\def\langnames@fams@wals@nlc{Nuclear Trans New Guinea}
\def\langnames@fams@wals@nld{Indo-European}
\def\langnames@fams@wals@nle{Atlantic-Congo}
\def\langnames@fams@wals@nlg{Austronesian}
\def\langnames@fams@wals@nli{Indo-European}
\def\langnames@fams@wals@nlj{Atlantic-Congo}
\def\langnames@fams@wals@nlk{Nuclear Trans New Guinea}
\def\langnames@fams@wals@nll{Isolate}
\def\langnames@fams@wals@nlm{Indo-European}
\def\langnames@fams@wals@nlo{Atlantic-Congo}
\def\langnames@fams@wals@nlu{Atlantic-Congo}
\def\langnames@fams@wals@nlv{Uto-Aztecan}
\def\langnames@fams@wals@nlx{Indo-European}
\def\langnames@fams@wals@nly{Pama-Nyungan}
\def\langnames@fams@wals@nlz{Austronesian}
\def\langnames@fams@wals@nma{Sino-Tibetan}
\def\langnames@fams@wals@nmb{Austronesian}
\def\langnames@fams@wals@nmc{Central Sudanic}
\def\langnames@fams@wals@nmd{Atlantic-Congo}
\def\langnames@fams@wals@nme{Sino-Tibetan}
\def\langnames@fams@wals@nmf{Sino-Tibetan}
\def\langnames@fams@wals@nmg{Atlantic-Congo}
\def\langnames@fams@wals@nmh{Sino-Tibetan}
\def\langnames@fams@wals@nmi{Afro-Asiatic}
\def\langnames@fams@wals@nmk{Austronesian}
\def\langnames@fams@wals@nml{Atlantic-Congo}
\def\langnames@fams@wals@nmm{Sino-Tibetan}
\def\langnames@fams@wals@nmn{Tuu}
\def\langnames@fams@wals@nmo{Sino-Tibetan}
\def\langnames@fams@wals@nmp{Nyulnyulan}
\def\langnames@fams@wals@nmq{Atlantic-Congo}
\def\langnames@fams@wals@nmr{Atlantic-Congo}
\def\langnames@fams@wals@nms{Austronesian}
\def\langnames@fams@wals@nmt{Austronesian}
\def\langnames@fams@wals@nmu{Maiduan}
\def\langnames@fams@wals@nmv{Pama-Nyungan}
\def\langnames@fams@wals@nmw{Austronesian}
\def\langnames@fams@wals@nmx{Yam}
\def\langnames@fams@wals@nmy{Sino-Tibetan}
\def\langnames@fams@wals@nmz{Atlantic-Congo}
\def\langnames@fams@wals@nna{Pama-Nyungan}
\def\langnames@fams@wals@nnb{Atlantic-Congo}
\def\langnames@fams@wals@nnc{Afro-Asiatic}
\def\langnames@fams@wals@nnd{Austronesian}
\def\langnames@fams@wals@nne{Atlantic-Congo}
\def\langnames@fams@wals@nnf{Nuclear Trans New Guinea}
\def\langnames@fams@wals@nng{Sino-Tibetan}
\def\langnames@fams@wals@nnh{Atlantic-Congo}
\def\langnames@fams@wals@nni{Austronesian}
\def\langnames@fams@wals@nnj{Nilotic}
\def\langnames@fams@wals@nnk{Nuclear Trans New Guinea}
\def\langnames@fams@wals@nnl{Sino-Tibetan}
\def\langnames@fams@wals@nnm{Sepik}
\def\langnames@fams@wals@nnn{Afro-Asiatic}
\def\langnames@fams@wals@nnp{Sino-Tibetan}
\def\langnames@fams@wals@nnq{Atlantic-Congo}
\def\langnames@fams@wals@nnr{Pama-Nyungan}
\def\langnames@fams@wals@nnt{Algic}
\def\langnames@fams@wals@nnu{Atlantic-Congo}
\def\langnames@fams@wals@nnv{Pama-Nyungan}
\def\langnames@fams@wals@nnw{Atlantic-Congo}
\def\langnames@fams@wals@nny{Tangkic}
\def\langnames@fams@wals@nnz{Atlantic-Congo}
\def\langnames@fams@wals@noa{Chocoan}
\def\langnames@fams@wals@noc{Nuclear Trans New Guinea}
\def\langnames@fams@wals@nod{Tai-Kadai}
\def\langnames@fams@wals@noe{Indo-European}
\def\langnames@fams@wals@nof{Nuclear Trans New Guinea}
\def\langnames@fams@wals@nog{Turkic}
\def\langnames@fams@wals@noh{Nuclear Trans New Guinea}
\def\langnames@fams@wals@noi{Indo-European}
\def\langnames@fams@wals@noj{Huitotoan}
\def\langnames@fams@wals@nok{Salishan}
\def\langnames@fams@wals@non{Indo-European}
\def\langnames@fams@wals@nop{Nuclear Trans New Guinea}
\def\langnames@fams@wals@noq{Atlantic-Congo}
\def\langnames@fams@wals@nor{Indo-European}
\def\langnames@fams@wals@nos{Sino-Tibetan}
\def\langnames@fams@wals@not{Arawakan}
\def\langnames@fams@wals@nou{Nuclear Trans New Guinea}
\def\langnames@fams@wals@nov{Artificial Language}
\def\langnames@fams@wals@now{Atlantic-Congo}
\def\langnames@fams@wals@noy{Atlantic-Congo}
\def\langnames@fams@wals@noz{Dizoid}
\def\langnames@fams@wals@npa{Sino-Tibetan}
\def\langnames@fams@wals@nph{Sino-Tibetan}
\def\langnames@fams@wals@npi{Indo-European}
\def\langnames@fams@wals@npl{Uto-Aztecan}
\def\langnames@fams@wals@npn{Austronesian}
\def\langnames@fams@wals@npo{Sino-Tibetan}
\def\langnames@fams@wals@nps{Nuclear Trans New Guinea}
\def\langnames@fams@wals@npy{Austronesian}
\def\langnames@fams@wals@nqg{Atlantic-Congo}
\def\langnames@fams@wals@nqk{Atlantic-Congo}
\def\langnames@fams@wals@nql{Atlantic-Congo}
\def\langnames@fams@wals@nqm{Kolopom}
\def\langnames@fams@wals@nqn{Yam}
\def\langnames@fams@wals@nqo{Artificial Language}
\def\langnames@fams@wals@nqt{Afro-Asiatic}
\def\langnames@fams@wals@nra{Atlantic-Congo}
\def\langnames@fams@wals@nrb{Isolate}
\def\langnames@fams@wals@nrc{Indo-European}
\def\langnames@fams@wals@nre{Sino-Tibetan}
\def\langnames@fams@wals@nrg{Austronesian}
\def\langnames@fams@wals@nri{Sino-Tibetan}
\def\langnames@fams@wals@nrk{Pama-Nyungan}
\def\langnames@fams@wals@nrl{Pama-Nyungan}
\def\langnames@fams@wals@nrm{Austronesian}
\def\langnames@fams@wals@nrp{Unclassifiable}
\def\langnames@fams@wals@nrt{Kalapuyan}
\def\langnames@fams@wals@nru{Sino-Tibetan}
\def\langnames@fams@wals@nrx{Unattested}
\def\langnames@fams@wals@nrz{Austronesian}
\def\langnames@fams@wals@nsa{Sino-Tibetan}
\def\langnames@fams@wals@nsb{Tuu}
\def\langnames@fams@wals@nsc{Unattested}
\def\langnames@fams@wals@nsd{Sino-Tibetan}
\def\langnames@fams@wals@nse{Atlantic-Congo}
\def\langnames@fams@wals@nsf{Sino-Tibetan}
\def\langnames@fams@wals@nsg{Nilotic}
\def\langnames@fams@wals@nsh{Atlantic-Congo}
\def\langnames@fams@wals@nsi{Sign Language}
\def\langnames@fams@wals@nsk{Algic}
\def\langnames@fams@wals@nsl{Sign Language}
\def\langnames@fams@wals@nsm{Sino-Tibetan}
\def\langnames@fams@wals@nsn{Austronesian}
\def\langnames@fams@wals@nso{Atlantic-Congo}
\def\langnames@fams@wals@nsp{Sign Language}
\def\langnames@fams@wals@nsq{Miwok-Costanoan}
\def\langnames@fams@wals@nsr{Sign Language}
\def\langnames@fams@wals@nss{Austronesian}
\def\langnames@fams@wals@nst{Sino-Tibetan}
\def\langnames@fams@wals@nsu{Uto-Aztecan}
\def\langnames@fams@wals@nsw{Austronesian}
\def\langnames@fams@wals@nsx{Atlantic-Congo}
\def\langnames@fams@wals@nsy{Austronesian}
\def\langnames@fams@wals@nsz{Maiduan}
\def\langnames@fams@wals@ntd{Austronesian}
\def\langnames@fams@wals@nte{Atlantic-Congo}
\def\langnames@fams@wals@nti{Atlantic-Congo}
\def\langnames@fams@wals@ntj{Pama-Nyungan}
\def\langnames@fams@wals@ntk{Atlantic-Congo}
\def\langnames@fams@wals@ntm{Atlantic-Congo}
\def\langnames@fams@wals@nto{Atlantic-Congo}
\def\langnames@fams@wals@ntp{Uto-Aztecan}
\def\langnames@fams@wals@ntr{Atlantic-Congo}
\def\langnames@fams@wals@ntu{Austronesian}
\def\langnames@fams@wals@ntw{Iroquoian}
\def\langnames@fams@wals@nty{Sino-Tibetan}
\def\langnames@fams@wals@ntz{Indo-European}
\def\langnames@fams@wals@nua{Austronesian}
\def\langnames@fams@wals@nuc{Pano-Tacanan}
\def\langnames@fams@wals@nud{Ndu}
\def\langnames@fams@wals@nue{Atlantic-Congo}
\def\langnames@fams@wals@nuf{Sino-Tibetan}
\def\langnames@fams@wals@nug{Mirndi}
\def\langnames@fams@wals@nuh{Atlantic-Congo}
\def\langnames@fams@wals@nui{Atlantic-Congo}
\def\langnames@fams@wals@nuj{Atlantic-Congo}
\def\langnames@fams@wals@nuk{Wakashan}
\def\langnames@fams@wals@nul{Austronesian}
\def\langnames@fams@wals@num{Austronesian}
\def\langnames@fams@wals@nun{Sino-Tibetan}
\def\langnames@fams@wals@nuo{Austroasiatic}
\def\langnames@fams@wals@nup{Atlantic-Congo}
\def\langnames@fams@wals@nuq{Austronesian}
\def\langnames@fams@wals@nur{Austronesian}
\def\langnames@fams@wals@nus{Nilotic}
\def\langnames@fams@wals@nut{Tai-Kadai}
\def\langnames@fams@wals@nuu{Atlantic-Congo}
\def\langnames@fams@wals@nuv{Atlantic-Congo}
\def\langnames@fams@wals@nuw{Austronesian}
\def\langnames@fams@wals@nux{Sepik}
\def\langnames@fams@wals@nuy{Gunwinyguan}
\def\langnames@fams@wals@nuz{Uto-Aztecan}
\def\langnames@fams@wals@nvh{Austronesian}
\def\langnames@fams@wals@nvm{Koiarian}
\def\langnames@fams@wals@nvo{Atlantic-Congo}
\def\langnames@fams@wals@nwa{Algic}
\def\langnames@fams@wals@nwb{Kru}
\def\langnames@fams@wals@nwe{Atlantic-Congo}
\def\langnames@fams@wals@nwi{Austronesian}
\def\langnames@fams@wals@nwm{Central Sudanic}
\def\langnames@fams@wals@nwo{Pama-Nyungan}
\def\langnames@fams@wals@nwr{Yareban}
\def\langnames@fams@wals@nww{Atlantic-Congo}
\def\langnames@fams@wals@nxa{Austronesian}
\def\langnames@fams@wals@nxd{Atlantic-Congo}
\def\langnames@fams@wals@nxe{Austronesian}
\def\langnames@fams@wals@nxg{Austronesian}
\def\langnames@fams@wals@nxi{Atlantic-Congo}
\def\langnames@fams@wals@nxl{Austronesian}
\def\langnames@fams@wals@nxm{Unclassifiable}
\def\langnames@fams@wals@nxn{Pama-Nyungan}
\def\langnames@fams@wals@nxo{Atlantic-Congo}
\def\langnames@fams@wals@nxq{Sino-Tibetan}
\def\langnames@fams@wals@nxr{Nuclear Trans New Guinea}
\def\langnames@fams@wals@nxx{Sentanic}
\def\langnames@fams@wals@nya{Atlantic-Congo}
\def\langnames@fams@wals@nyb{Atlantic-Congo}
\def\langnames@fams@wals@nyc{Atlantic-Congo}
\def\langnames@fams@wals@nyd{Atlantic-Congo}
\def\langnames@fams@wals@nye{Atlantic-Congo}
\def\langnames@fams@wals@nyf{Atlantic-Congo}
\def\langnames@fams@wals@nyg{Atlantic-Congo}
\def\langnames@fams@wals@nyh{Nyulnyulan}
\def\langnames@fams@wals@nyi{Nyimang}
\def\langnames@fams@wals@nyj{Atlantic-Congo}
\def\langnames@fams@wals@nyk{Atlantic-Congo}
\def\langnames@fams@wals@nyl{Austroasiatic}
\def\langnames@fams@wals@nym{Atlantic-Congo}
\def\langnames@fams@wals@nyn{Atlantic-Congo}
\def\langnames@fams@wals@nyo{Atlantic-Congo}
\def\langnames@fams@wals@nyp{Kuliak}
\def\langnames@fams@wals@nyq{Indo-European}
\def\langnames@fams@wals@nyr{Atlantic-Congo}
\def\langnames@fams@wals@nys{Pama-Nyungan}
\def\langnames@fams@wals@nyt{Pama-Nyungan}
\def\langnames@fams@wals@nyu{Atlantic-Congo}
\def\langnames@fams@wals@nyv{Nyulnyulan}
\def\langnames@fams@wals@nyx{Pama-Nyungan}
\def\langnames@fams@wals@nyy{Atlantic-Congo}
\def\langnames@fams@wals@nza{Atlantic-Congo}
\def\langnames@fams@wals@nzb{Atlantic-Congo}
\def\langnames@fams@wals@nzd{Atlantic-Congo}
\def\langnames@fams@wals@nzi{Atlantic-Congo}
\def\langnames@fams@wals@nzk{Atlantic-Congo}
\def\langnames@fams@wals@nzm{Sino-Tibetan}
\def\langnames@fams@wals@nzs{Sign Language}
\def\langnames@fams@wals@nzy{Atlantic-Congo}
\def\langnames@fams@wals@nzz{Dogon}
\def\langnames@fams@wals@oaa{Tungusic}
\def\langnames@fams@wals@oac{Tungusic}
\def\langnames@fams@wals@oar{Afro-Asiatic}
\def\langnames@fams@wals@obi{Chumashan}
\def\langnames@fams@wals@obl{Atlantic-Congo}
\def\langnames@fams@wals@obo{Austronesian}
\def\langnames@fams@wals@obr{Sino-Tibetan}
\def\langnames@fams@wals@obu{Atlantic-Congo}
\def\langnames@fams@wals@oca{Huitotoan}
\def\langnames@fams@wals@och{Sino-Tibetan}
\def\langnames@fams@wals@oci{Indo-European}
\def\langnames@fams@wals@ocu{Otomanguean}
\def\langnames@fams@wals@odk{Indo-European}
\def\langnames@fams@wals@odt{Indo-European}
\def\langnames@fams@wals@odu{Atlantic-Congo}
\def\langnames@fams@wals@ofo{Siouan}
\def\langnames@fams@wals@ofs{Indo-European}
\def\langnames@fams@wals@ofu{Atlantic-Congo}
\def\langnames@fams@wals@ogb{Atlantic-Congo}
\def\langnames@fams@wals@ogc{Atlantic-Congo}
\def\langnames@fams@wals@oge{Kartvelian}
\def\langnames@fams@wals@ogg{Atlantic-Congo}
\def\langnames@fams@wals@ogo{Atlantic-Congo}
\def\langnames@fams@wals@ogu{Atlantic-Congo}
\def\langnames@fams@wals@oia{Timor-Alor-Pantar}
\def\langnames@fams@wals@oie{Nilotic}
\def\langnames@fams@wals@oin{Nuclear Torricelli}
\def\langnames@fams@wals@ojb{Algic}
\def\langnames@fams@wals@ojc{Algic}
\def\langnames@fams@wals@ojg{Algic}
\def\langnames@fams@wals@ojp{Japonic}
\def\langnames@fams@wals@ojs{Algic}
\def\langnames@fams@wals@ojv{Austronesian}
\def\langnames@fams@wals@ojw{Algic}
\def\langnames@fams@wals@oka{Salishan}
\def\langnames@fams@wals@okb{Atlantic-Congo}
\def\langnames@fams@wals@okd{Ijoid}
\def\langnames@fams@wals@oke{Atlantic-Congo}
\def\langnames@fams@wals@okh{Indo-European}
\def\langnames@fams@wals@oki{Nilotic}
\def\langnames@fams@wals@okj{Great Andamanese}
\def\langnames@fams@wals@okk{Nuclear Torricelli}
\def\langnames@fams@wals@okl{Sign Language}
\def\langnames@fams@wals@okn{Japonic}
\def\langnames@fams@wals@okr{Ijoid}
\def\langnames@fams@wals@oks{Atlantic-Congo}
\def\langnames@fams@wals@oku{Atlantic-Congo}
\def\langnames@fams@wals@okv{Nuclear Trans New Guinea}
\def\langnames@fams@wals@okx{Atlantic-Congo}
\def\langnames@fams@wals@ola{Sino-Tibetan}
\def\langnames@fams@wals@old{Atlantic-Congo}
\def\langnames@fams@wals@ole{Sino-Tibetan}
\def\langnames@fams@wals@olm{Atlantic-Congo}
\def\langnames@fams@wals@olo{Uralic}
\def\langnames@fams@wals@olu{Atlantic-Congo}
\def\langnames@fams@wals@oma{Siouan}
\def\langnames@fams@wals@omb{Austronesian}
\def\langnames@fams@wals@omc{Isolate}
\def\langnames@fams@wals@omg{Tupian}
\def\langnames@fams@wals@omi{Central Sudanic}
\def\langnames@fams@wals@omk{Yukaghir}
\def\langnames@fams@wals@oml{Atlantic-Congo}
\def\langnames@fams@wals@omn{Unclassifiable}
\def\langnames@fams@wals@omo{Nuclear Trans New Guinea}
\def\langnames@fams@wals@omr{Indo-European}
\def\langnames@fams@wals@omt{Nilotic}
\def\langnames@fams@wals@omu{Isolate}
\def\langnames@fams@wals@omw{Nuclear Trans New Guinea}
\def\langnames@fams@wals@omx{Austroasiatic}
\def\langnames@fams@wals@ona{Chonan}
\def\langnames@fams@wals@onb{Tai-Kadai}
\def\langnames@fams@wals@one{Iroquoian}
\def\langnames@fams@wals@ong{Nuclear Torricelli}
\def\langnames@fams@wals@oni{Austronesian}
\def\langnames@fams@wals@onj{Dagan}
\def\langnames@fams@wals@onk{Nuclear Torricelli}
\def\langnames@fams@wals@onn{Bosavi}
\def\langnames@fams@wals@ono{Iroquoian}
\def\langnames@fams@wals@onp{Sino-Tibetan}
\def\langnames@fams@wals@onr{Nuclear Torricelli}
\def\langnames@fams@wals@ons{Nuclear Trans New Guinea}
\def\langnames@fams@wals@onu{Austronesian}
\def\langnames@fams@wals@onw{Nubian}
\def\langnames@fams@wals@onx{Pidgin}
\def\langnames@fams@wals@ood{Uto-Aztecan}
\def\langnames@fams@wals@oog{Austroasiatic}
\def\langnames@fams@wals@oon{Jarawa-Onge}
\def\langnames@fams@wals@oor{Indo-European}
\def\langnames@fams@wals@oos{Indo-European}
\def\langnames@fams@wals@opa{Atlantic-Congo}
\def\langnames@fams@wals@opk{Nuclear Trans New Guinea}
\def\langnames@fams@wals@opm{Nuclear Trans New Guinea}
\def\langnames@fams@wals@opo{Eleman}
\def\langnames@fams@wals@opt{Uto-Aztecan}
\def\langnames@fams@wals@opy{Nuclear-Macro-Je}
\def\langnames@fams@wals@ora{Austronesian}
\def\langnames@fams@wals@orc{Afro-Asiatic}
\def\langnames@fams@wals@ore{Tucanoan}
\def\langnames@fams@wals@org{Atlantic-Congo}
\def\langnames@fams@wals@orh{Tungusic}
\def\langnames@fams@wals@orn{Austronesian}
\def\langnames@fams@wals@oro{Eleman}
\def\langnames@fams@wals@orr{Ijoid}
\def\langnames@fams@wals@ors{Austronesian}
\def\langnames@fams@wals@ort{Indo-European}
\def\langnames@fams@wals@oru{Indo-European}
\def\langnames@fams@wals@orv{Indo-European}
\def\langnames@fams@wals@orw{Chapacuran}
\def\langnames@fams@wals@orx{Atlantic-Congo}
\def\langnames@fams@wals@ory{Indo-European}
\def\langnames@fams@wals@orz{Austronesian}
\def\langnames@fams@wals@osa{Siouan}
\def\langnames@fams@wals@osc{Indo-European}
\def\langnames@fams@wals@osi{Austronesian}
\def\langnames@fams@wals@oso{Atlantic-Congo}
\def\langnames@fams@wals@osp{Indo-European}
\def\langnames@fams@wals@oss{Indo-European}
\def\langnames@fams@wals@ost{Atlantic-Congo}
\def\langnames@fams@wals@osu{Nuclear Torricelli}
\def\langnames@fams@wals@osx{Indo-European}
\def\langnames@fams@wals@otd{Austronesian}
\def\langnames@fams@wals@ote{Otomanguean}
\def\langnames@fams@wals@oti{Isolate}
\def\langnames@fams@wals@otl{Otomanguean}
\def\langnames@fams@wals@otm{Otomanguean}
\def\langnames@fams@wals@otn{Otomanguean}
\def\langnames@fams@wals@otq{Otomanguean}
\def\langnames@fams@wals@otr{Heibanic}
\def\langnames@fams@wals@ots{Otomanguean}
\def\langnames@fams@wals@ott{Otomanguean}
\def\langnames@fams@wals@otu{Bororoan}
\def\langnames@fams@wals@otw{Algic}
\def\langnames@fams@wals@otx{Otomanguean}
\def\langnames@fams@wals@oty{Dravidian}
\def\langnames@fams@wals@otz{Otomanguean}
\def\langnames@fams@wals@oua{Afro-Asiatic}
\def\langnames@fams@wals@oub{Kru}
\def\langnames@fams@wals@oue{South Bougainville}
\def\langnames@fams@wals@oui{Turkic}
\def\langnames@fams@wals@oum{Austronesian}
\def\langnames@fams@wals@owi{Left May}
\def\langnames@fams@wals@owl{Indo-European}
\def\langnames@fams@wals@oyb{Austroasiatic}
\def\langnames@fams@wals@oyd{Ta-Ne-Omotic}
\def\langnames@fams@wals@oym{Tupian}
\def\langnames@fams@wals@oyy{Austronesian}
\def\langnames@fams@wals@ozm{Atlantic-Congo}
\def\langnames@fams@wals@pab{Arawakan}
\def\langnames@fams@wals@pac{Austroasiatic}
\def\langnames@fams@wals@pad{Arawan}
\def\langnames@fams@wals@pae{Atlantic-Congo}
\def\langnames@fams@wals@paf{Tupian}
\def\langnames@fams@wals@pag{Austronesian}
\def\langnames@fams@wals@pah{Tupian}
\def\langnames@fams@wals@pai{Atlantic-Congo}
\def\langnames@fams@wals@pak{Tupian}
\def\langnames@fams@wals@pal{Indo-European}
\def\langnames@fams@wals@pam{Austronesian}
\def\langnames@fams@wals@pan{Indo-European}
\def\langnames@fams@wals@pao{Uto-Aztecan}
\def\langnames@fams@wals@pap{Indo-European}
\def\langnames@fams@wals@paq{Indo-European}
\def\langnames@fams@wals@par{Uto-Aztecan}
\def\langnames@fams@wals@pas{Lakes Plain}
\def\langnames@fams@wals@pau{Austronesian}
\def\langnames@fams@wals@pav{Chapacuran}
\def\langnames@fams@wals@paw{Caddoan}
\def\langnames@fams@wals@pax{Unattested}
\def\langnames@fams@wals@pay{Chibchan}
\def\langnames@fams@wals@paz{Isolate}
\def\langnames@fams@wals@pbb{Isolate}
\def\langnames@fams@wals@pbc{Cariban}
\def\langnames@fams@wals@pbe{Otomanguean}
\def\langnames@fams@wals@pbf{Otomanguean}
\def\langnames@fams@wals@pbg{Arawakan}
\def\langnames@fams@wals@pbh{Cariban}
\def\langnames@fams@wals@pbi{Afro-Asiatic}
\def\langnames@fams@wals@pbl{Atlantic-Congo}
\def\langnames@fams@wals@pbm{Otomanguean}
\def\langnames@fams@wals@pbn{Atlantic-Congo}
\def\langnames@fams@wals@pbo{Atlantic-Congo}
\def\langnames@fams@wals@pbp{Atlantic-Congo}
\def\langnames@fams@wals@pbr{Atlantic-Congo}
\def\langnames@fams@wals@pbs{Otomanguean}
\def\langnames@fams@wals@pbt{Indo-European}
\def\langnames@fams@wals@pbu{Indo-European}
\def\langnames@fams@wals@pbv{Austroasiatic}
\def\langnames@fams@wals@pby{Isolate}
\def\langnames@fams@wals@pca{Otomanguean}
\def\langnames@fams@wals@pcb{Austroasiatic}
\def\langnames@fams@wals@pcc{Tai-Kadai}
\def\langnames@fams@wals@pcd{Indo-European}
\def\langnames@fams@wals@pce{Austroasiatic}
\def\langnames@fams@wals@pcf{Dravidian}
\def\langnames@fams@wals@pcg{Dravidian}
\def\langnames@fams@wals@pch{Unattested}
\def\langnames@fams@wals@pci{Dravidian}
\def\langnames@fams@wals@pcj{Austroasiatic}
\def\langnames@fams@wals@pck{Sino-Tibetan}
\def\langnames@fams@wals@pcl{Indo-European}
\def\langnames@fams@wals@pcm{Indo-European}
\def\langnames@fams@wals@pcn{Atlantic-Congo}
\def\langnames@fams@wals@pcp{Pano-Tacanan}
\def\langnames@fams@wals@pcw{Afro-Asiatic}
\def\langnames@fams@wals@pda{Nuclear Trans New Guinea}
\def\langnames@fams@wals@pdc{Indo-European}
\def\langnames@fams@wals@pdi{Tai-Kadai}
\def\langnames@fams@wals@pdn{Austronesian}
\def\langnames@fams@wals@pdo{Austronesian}
\def\langnames@fams@wals@pdt{Indo-European}
\def\langnames@fams@wals@pdu{Sino-Tibetan}
\def\langnames@fams@wals@pea{Austronesian}
\def\langnames@fams@wals@peb{Pomoan}
\def\langnames@fams@wals@ped{Nuclear Trans New Guinea}
\def\langnames@fams@wals@pee{Austronesian}
\def\langnames@fams@wals@pef{Pomoan}
\def\langnames@fams@wals@peg{Dravidian}
\def\langnames@fams@wals@peh{Mongolic-Khitan}
\def\langnames@fams@wals@pei{Otomanguean}
\def\langnames@fams@wals@pej{Pomoan}
\def\langnames@fams@wals@pek{Austronesian}
\def\langnames@fams@wals@pel{Austronesian}
\def\langnames@fams@wals@pem{Atlantic-Congo}
\def\langnames@fams@wals@peo{Indo-European}
\def\langnames@fams@wals@pep{Yam}
\def\langnames@fams@wals@peq{Pomoan}
\def\langnames@fams@wals@pes{Indo-European}
\def\langnames@fams@wals@pev{Cariban}
\def\langnames@fams@wals@pex{Austronesian}
\def\langnames@fams@wals@pey{Indo-European}
\def\langnames@fams@wals@pez{Austronesian}
\def\langnames@fams@wals@pfa{Austronesian}
\def\langnames@fams@wals@pfe{Atlantic-Congo}
\def\langnames@fams@wals@pfl{Indo-European}
\def\langnames@fams@wals@pga{Afro-Asiatic}
\def\langnames@fams@wals@pgd{Indo-European}
\def\langnames@fams@wals@pgg{Indo-European}
\def\langnames@fams@wals@pgi{Border}
\def\langnames@fams@wals@pgk{Austronesian}
\def\langnames@fams@wals@pgs{Atlantic-Congo}
\def\langnames@fams@wals@pgu{North Halmahera}
\def\langnames@fams@wals@pgz{Sign Language}
\def\langnames@fams@wals@pha{Hmong-Mien}
\def\langnames@fams@wals@phd{Indo-European}
\def\langnames@fams@wals@phg{Austroasiatic}
\def\langnames@fams@wals@phh{Sino-Tibetan}
\def\langnames@fams@wals@phj{Sino-Tibetan}
\def\langnames@fams@wals@phk{Tai-Kadai}
\def\langnames@fams@wals@phl{Indo-European}
\def\langnames@fams@wals@phm{Atlantic-Congo}
\def\langnames@fams@wals@phn{Afro-Asiatic}
\def\langnames@fams@wals@pho{Sino-Tibetan}
\def\langnames@fams@wals@phq{Sino-Tibetan}
\def\langnames@fams@wals@phr{Indo-European}
\def\langnames@fams@wals@pht{Tai-Kadai}
\def\langnames@fams@wals@phu{Tai-Kadai}
\def\langnames@fams@wals@phv{Indo-European}
\def\langnames@fams@wals@pia{Uto-Aztecan}
\def\langnames@fams@wals@pib{Arawakan}
\def\langnames@fams@wals@pic{Atlantic-Congo}
\def\langnames@fams@wals@pid{Saliban}
\def\langnames@fams@wals@pie{Kiowa-Tanoan}
\def\langnames@fams@wals@pif{Austronesian}
\def\langnames@fams@wals@pig{Unattested}
\def\langnames@fams@wals@pih{Indo-European}
\def\langnames@fams@wals@pij{Unclassifiable}
\def\langnames@fams@wals@pil{Atlantic-Congo}
\def\langnames@fams@wals@pim{Algic}
\def\langnames@fams@wals@pin{Sepik}
\def\langnames@fams@wals@pio{Arawakan}
\def\langnames@fams@wals@pip{Afro-Asiatic}
\def\langnames@fams@wals@pir{Tucanoan}
\def\langnames@fams@wals@pis{Indo-European}
\def\langnames@fams@wals@pit{Pama-Nyungan}
\def\langnames@fams@wals@piu{Pama-Nyungan}
\def\langnames@fams@wals@piv{Austronesian}
\def\langnames@fams@wals@piw{Atlantic-Congo}
\def\langnames@fams@wals@pix{Austronesian}
\def\langnames@fams@wals@piy{Afro-Asiatic}
\def\langnames@fams@wals@piz{Austronesian}
\def\langnames@fams@wals@pjt{Pama-Nyungan}
\def\langnames@fams@wals@pkb{Atlantic-Congo}
\def\langnames@fams@wals@pkc{Unclassifiable}
\def\langnames@fams@wals@pkg{Austronesian}
\def\langnames@fams@wals@pkh{Sino-Tibetan}
\def\langnames@fams@wals@pkn{Pama-Nyungan}
\def\langnames@fams@wals@pko{Nilotic}
\def\langnames@fams@wals@pkp{Austronesian}
\def\langnames@fams@wals@pkr{Dravidian}
\def\langnames@fams@wals@pks{Sign Language}
\def\langnames@fams@wals@pkt{Austroasiatic}
\def\langnames@fams@wals@pku{Austronesian}
\def\langnames@fams@wals@pla{Nuclear Trans New Guinea}
\def\langnames@fams@wals@plb{Austronesian}
\def\langnames@fams@wals@plc{Austronesian}
\def\langnames@fams@wals@pld{Unclassifiable}
\def\langnames@fams@wals@ple{Austronesian}
\def\langnames@fams@wals@plg{Guaicuruan}
\def\langnames@fams@wals@plh{Austronesian}
\def\langnames@fams@wals@pli{Indo-European}
\def\langnames@fams@wals@plj{Afro-Asiatic}
\def\langnames@fams@wals@plk{Indo-European}
\def\langnames@fams@wals@pll{Austroasiatic}
\def\langnames@fams@wals@pln{Indo-European}
\def\langnames@fams@wals@plo{Mixe-Zoque}
\def\langnames@fams@wals@plq{Indo-European}
\def\langnames@fams@wals@plr{Atlantic-Congo}
\def\langnames@fams@wals@pls{Otomanguean}
\def\langnames@fams@wals@plt{Austronesian}
\def\langnames@fams@wals@plu{Arawakan}
\def\langnames@fams@wals@plv{Austronesian}
\def\langnames@fams@wals@plw{Austronesian}
\def\langnames@fams@wals@ply{Austroasiatic}
\def\langnames@fams@wals@plz{Austronesian}
\def\langnames@fams@wals@pma{Austronesian}
\def\langnames@fams@wals@pmb{Atlantic-Congo}
\def\langnames@fams@wals@pmc{Unattested}
\def\langnames@fams@wals@pmd{Pama-Nyungan}
\def\langnames@fams@wals@pme{Austronesian}
\def\langnames@fams@wals@pmf{Austronesian}
\def\langnames@fams@wals@pmh{Indo-European}
\def\langnames@fams@wals@pmi{Sino-Tibetan}
\def\langnames@fams@wals@pmj{Sino-Tibetan}
\def\langnames@fams@wals@pml{Pidgin}
\def\langnames@fams@wals@pmm{Atlantic-Congo}
\def\langnames@fams@wals@pmn{Atlantic-Congo}
\def\langnames@fams@wals@pmo{Austronesian}
\def\langnames@fams@wals@pmq{Otomanguean}
\def\langnames@fams@wals@pmr{Nuclear Trans New Guinea}
\def\langnames@fams@wals@pms{Indo-European}
\def\langnames@fams@wals@pmt{Austronesian}
\def\langnames@fams@wals@pmw{Miwok-Costanoan}
\def\langnames@fams@wals@pmx{Sino-Tibetan}
\def\langnames@fams@wals@pmy{Austronesian}
\def\langnames@fams@wals@pmz{Otomanguean}
\def\langnames@fams@wals@pna{Austronesian}
\def\langnames@fams@wals@pnb{Indo-European}
\def\langnames@fams@wals@pnc{Austronesian}
\def\langnames@fams@wals@pnd{Atlantic-Congo}
\def\langnames@fams@wals@pne{Austronesian}
\def\langnames@fams@wals@png{Atlantic-Congo}
\def\langnames@fams@wals@pnh{Austronesian}
\def\langnames@fams@wals@pni{Austronesian}
\def\langnames@fams@wals@pnk{Arawakan}
\def\langnames@fams@wals@pnl{Atlantic-Congo}
\def\langnames@fams@wals@pnm{Austronesian}
\def\langnames@fams@wals@pnn{Piawi}
\def\langnames@fams@wals@pno{Pano-Tacanan}
\def\langnames@fams@wals@pnp{Austronesian}
\def\langnames@fams@wals@pnq{Atlantic-Congo}
\def\langnames@fams@wals@pnr{Nuclear Trans New Guinea}
\def\langnames@fams@wals@pns{Austronesian}
\def\langnames@fams@wals@pnt{Indo-European}
\def\langnames@fams@wals@pnu{Hmong-Mien}
\def\langnames@fams@wals@pnv{Pama-Nyungan}
\def\langnames@fams@wals@pnw{Pama-Nyungan}
\def\langnames@fams@wals@pnx{Austroasiatic}
\def\langnames@fams@wals@pny{Atlantic-Congo}
\def\langnames@fams@wals@pnz{Atlantic-Congo}
\def\langnames@fams@wals@poc{Mayan}
\def\langnames@fams@wals@poe{Otomanguean}
\def\langnames@fams@wals@pof{Atlantic-Congo}
\def\langnames@fams@wals@poh{Mayan}
\def\langnames@fams@wals@poi{Mixe-Zoque}
\def\langnames@fams@wals@pol{Indo-European}
\def\langnames@fams@wals@pom{Pomoan}
\def\langnames@fams@wals@pon{Austronesian}
\def\langnames@fams@wals@poo{Pomoan}
\def\langnames@fams@wals@pop{Austronesian}
\def\langnames@fams@wals@poq{Mixe-Zoque}
\def\langnames@fams@wals@por{Indo-European}
\def\langnames@fams@wals@pos{Mixe-Zoque}
\def\langnames@fams@wals@pot{Algic}
\def\langnames@fams@wals@pov{Indo-European}
\def\langnames@fams@wals@pow{Otomanguean}
\def\langnames@fams@wals@pox{Indo-European}
\def\langnames@fams@wals@poy{Atlantic-Congo}
\def\langnames@fams@wals@ppe{Isolate}
\def\langnames@fams@wals@ppi{Cochimi-Yuman}
\def\langnames@fams@wals@ppk{Austronesian}
\def\langnames@fams@wals@ppl{Uto-Aztecan}
\def\langnames@fams@wals@ppm{Austronesian}
\def\langnames@fams@wals@ppn{Austronesian}
\def\langnames@fams@wals@ppo{Teberan}
\def\langnames@fams@wals@ppq{Walioic}
\def\langnames@fams@wals@pps{Otomanguean}
\def\langnames@fams@wals@ppt{Kamula-Elevala}
\def\langnames@fams@wals@ppu{Austronesian}
\def\langnames@fams@wals@ppv{Unattested}
\def\langnames@fams@wals@pqa{Afro-Asiatic}
\def\langnames@fams@wals@pqm{Algic}
\def\langnames@fams@wals@prc{Indo-European}
\def\langnames@fams@wals@pre{Indo-European}
\def\langnames@fams@wals@prf{Austronesian}
\def\langnames@fams@wals@prg{Indo-European}
\def\langnames@fams@wals@prh{Austronesian}
\def\langnames@fams@wals@pri{Austronesian}
\def\langnames@fams@wals@prk{Austroasiatic}
\def\langnames@fams@wals@prl{Sign Language}
\def\langnames@fams@wals@prm{Isolate}
\def\langnames@fams@wals@prn{Indo-European}
\def\langnames@fams@wals@pro{Indo-European}
\def\langnames@fams@wals@prq{Arawakan}
\def\langnames@fams@wals@prr{Puri-Coroado}
\def\langnames@fams@wals@prs{Indo-European}
\def\langnames@fams@wals@prt{Austroasiatic}
\def\langnames@fams@wals@pru{South Bird's Head Family}
\def\langnames@fams@wals@prw{Nuclear Trans New Guinea}
\def\langnames@fams@wals@prx{Sino-Tibetan}
\def\langnames@fams@wals@prz{Sign Language}
\def\langnames@fams@wals@psa{Nuclear Trans New Guinea}
\def\langnames@fams@wals@psc{Sign Language}
\def\langnames@fams@wals@psd{Sign Language}
\def\langnames@fams@wals@pse{Austronesian}
\def\langnames@fams@wals@psg{Sign Language}
\def\langnames@fams@wals@psh{Indo-European}
\def\langnames@fams@wals@psi{Indo-European}
\def\langnames@fams@wals@psl{Sign Language}
\def\langnames@fams@wals@psm{Tupian}
\def\langnames@fams@wals@psn{Austronesian}
\def\langnames@fams@wals@pso{Sign Language}
\def\langnames@fams@wals@psp{Sign Language}
\def\langnames@fams@wals@psq{Sepik}
\def\langnames@fams@wals@psr{Sign Language}
\def\langnames@fams@wals@pss{Austronesian}
\def\langnames@fams@wals@pst{Indo-European}
\def\langnames@fams@wals@psu{Indo-European}
\def\langnames@fams@wals@psw{Austronesian}
\def\langnames@fams@wals@psy{Algic}
\def\langnames@fams@wals@pta{Tupian}
\def\langnames@fams@wals@pth{Nuclear-Macro-Je}
\def\langnames@fams@wals@pti{Pama-Nyungan}
\def\langnames@fams@wals@ptn{Austronesian}
\def\langnames@fams@wals@pto{Tupian}
\def\langnames@fams@wals@ptp{Austronesian}
\def\langnames@fams@wals@ptq{Dravidian}
\def\langnames@fams@wals@ptr{Austronesian}
\def\langnames@fams@wals@ptt{Austronesian}
\def\langnames@fams@wals@ptu{Austronesian}
\def\langnames@fams@wals@ptv{Austronesian}
\def\langnames@fams@wals@ptw{Salishan}
\def\langnames@fams@wals@pty{Dravidian}
\def\langnames@fams@wals@pua{Tarascan}
\def\langnames@fams@wals@pub{Sino-Tibetan}
\def\langnames@fams@wals@pud{Austronesian}
\def\langnames@fams@wals@pue{Isolate}
\def\langnames@fams@wals@puf{Austronesian}
\def\langnames@fams@wals@pug{Atlantic-Congo}
\def\langnames@fams@wals@pui{Isolate}
\def\langnames@fams@wals@puj{Austronesian}
\def\langnames@fams@wals@pum{Sino-Tibetan}
\def\langnames@fams@wals@puo{Austroasiatic}
\def\langnames@fams@wals@pup{Nuclear Trans New Guinea}
\def\langnames@fams@wals@puq{Isolate}
\def\langnames@fams@wals@pur{Tupian}
\def\langnames@fams@wals@puu{Atlantic-Congo}
\def\langnames@fams@wals@puw{Austronesian}
\def\langnames@fams@wals@pux{Sko}
\def\langnames@fams@wals@puy{Chumashan}
\def\langnames@fams@wals@pwa{Isolate}
\def\langnames@fams@wals@pwb{Atlantic-Congo}
\def\langnames@fams@wals@pwg{Austronesian}
\def\langnames@fams@wals@pwi{Wintuan}
\def\langnames@fams@wals@pwm{Austronesian}
\def\langnames@fams@wals@pwn{Austronesian}
\def\langnames@fams@wals@pwo{Sino-Tibetan}
\def\langnames@fams@wals@pwr{Indo-European}
\def\langnames@fams@wals@pww{Sino-Tibetan}
\def\langnames@fams@wals@pye{Kru}
\def\langnames@fams@wals@pym{Atlantic-Congo}
\def\langnames@fams@wals@pyn{Pano-Tacanan}
\def\langnames@fams@wals@pys{Sign Language}
\def\langnames@fams@wals@pyu{Austronesian}
\def\langnames@fams@wals@pyx{Sino-Tibetan}
\def\langnames@fams@wals@pyy{Sino-Tibetan}
\def\langnames@fams@wals@pzh{Austronesian}
\def\langnames@fams@wals@pzn{Sino-Tibetan}
\def\langnames@fams@wals@qbb{Indo-European}
\def\langnames@fams@wals@qcs{Mixe-Zoque}
\def\langnames@fams@wals@qer{Indo-European}
\def\langnames@fams@wals@qgu{Pama-Nyungan}
\def\langnames@fams@wals@qhr{Indo-European}
\def\langnames@fams@wals@qkn{Dravidian}
\def\langnames@fams@wals@qlm{Indo-European}
\def\langnames@fams@wals@qok{Austroasiatic}
\def\langnames@fams@wals@qpp{Indo-European}
\def\langnames@fams@wals@qua{Siouan}
\def\langnames@fams@wals@qub{Quechuan}
\def\langnames@fams@wals@quc{Mayan}
\def\langnames@fams@wals@qud{Quechuan}
\def\langnames@fams@wals@quf{Quechuan}
\def\langnames@fams@wals@qug{Quechuan}
\def\langnames@fams@wals@quh{Quechuan}
\def\langnames@fams@wals@qui{Chimakuan}
\def\langnames@fams@wals@quk{Quechuan}
\def\langnames@fams@wals@qul{Quechuan}
\def\langnames@fams@wals@qum{Mayan}
\def\langnames@fams@wals@qun{Salishan}
\def\langnames@fams@wals@qup{Quechuan}
\def\langnames@fams@wals@quq{Unclassifiable}
\def\langnames@fams@wals@qur{Quechuan}
\def\langnames@fams@wals@qus{Quechuan}
\def\langnames@fams@wals@quv{Mayan}
\def\langnames@fams@wals@quw{Quechuan}
\def\langnames@fams@wals@qux{Quechuan}
\def\langnames@fams@wals@quy{Quechuan}
\def\langnames@fams@wals@quz{Quechuan}
\def\langnames@fams@wals@qva{Quechuan}
\def\langnames@fams@wals@qvc{Quechuan}
\def\langnames@fams@wals@qve{Quechuan}
\def\langnames@fams@wals@qvh{Quechuan}
\def\langnames@fams@wals@qvi{Quechuan}
\def\langnames@fams@wals@qvj{Quechuan}
\def\langnames@fams@wals@qvl{Quechuan}
\def\langnames@fams@wals@qvm{Quechuan}
\def\langnames@fams@wals@qvn{Quechuan}
\def\langnames@fams@wals@qvo{Quechuan}
\def\langnames@fams@wals@qvp{Quechuan}
\def\langnames@fams@wals@qvs{Quechuan}
\def\langnames@fams@wals@qvw{Quechuan}
\def\langnames@fams@wals@qvy{Sino-Tibetan}
\def\langnames@fams@wals@qvz{Quechuan}
\def\langnames@fams@wals@qwa{Quechuan}
\def\langnames@fams@wals@qwc{Quechuan}
\def\langnames@fams@wals@qwh{Quechuan}
\def\langnames@fams@wals@qws{Quechuan}
\def\langnames@fams@wals@qwt{Athabaskan-Eyak-Tlingit}
\def\langnames@fams@wals@qxa{Quechuan}
\def\langnames@fams@wals@qxc{Quechuan}
\def\langnames@fams@wals@qxh{Quechuan}
\def\langnames@fams@wals@qxl{Quechuan}
\def\langnames@fams@wals@qxn{Quechuan}
\def\langnames@fams@wals@qxo{Quechuan}
\def\langnames@fams@wals@qxp{Quechuan}
\def\langnames@fams@wals@qxq{Turkic}
\def\langnames@fams@wals@qxr{Quechuan}
\def\langnames@fams@wals@qxs{Sino-Tibetan}
\def\langnames@fams@wals@qxu{Quechuan}
\def\langnames@fams@wals@qxw{Quechuan}
\def\langnames@fams@wals@qya{Artificial Language}
\def\langnames@fams@wals@qyp{Algic}
\def\langnames@fams@wals@raa{Sino-Tibetan}
\def\langnames@fams@wals@rab{Sino-Tibetan}
\def\langnames@fams@wals@rac{Lakes Plain}
\def\langnames@fams@wals@rad{Austronesian}
\def\langnames@fams@wals@raf{Sino-Tibetan}
\def\langnames@fams@wals@rag{Atlantic-Congo}
\def\langnames@fams@wals@rah{Sino-Tibetan}
\def\langnames@fams@wals@rai{Austronesian}
\def\langnames@fams@wals@rak{Austronesian}
\def\langnames@fams@wals@ral{Sino-Tibetan}
\def\langnames@fams@wals@ram{Nuclear-Macro-Je}
\def\langnames@fams@wals@ran{Kolopom}
\def\langnames@fams@wals@rao{Lower Sepik-Ramu}
\def\langnames@fams@wals@rap{Austronesian}
\def\langnames@fams@wals@raq{Sino-Tibetan}
\def\langnames@fams@wals@rar{Austronesian}
\def\langnames@fams@wals@ras{Rashad}
\def\langnames@fams@wals@rat{Indo-European}
\def\langnames@fams@wals@rau{Sino-Tibetan}
\def\langnames@fams@wals@rav{Sino-Tibetan}
\def\langnames@fams@wals@raw{Sino-Tibetan}
\def\langnames@fams@wals@rax{Atlantic-Congo}
\def\langnames@fams@wals@ray{Austronesian}
\def\langnames@fams@wals@raz{Austronesian}
\def\langnames@fams@wals@rbb{Austroasiatic}
\def\langnames@fams@wals@rcf{Indo-European}
\def\langnames@fams@wals@rdb{Indo-European}
\def\langnames@fams@wals@rea{Nuclear Trans New Guinea}
\def\langnames@fams@wals@reb{Austronesian}
\def\langnames@fams@wals@ree{Austronesian}
\def\langnames@fams@wals@reg{Atlantic-Congo}
\def\langnames@fams@wals@rei{Indo-European}
\def\langnames@fams@wals@rej{Austronesian}
\def\langnames@fams@wals@rel{Afro-Asiatic}
\def\langnames@fams@wals@rem{Pano-Tacanan}
\def\langnames@fams@wals@ren{Austroasiatic}
\def\langnames@fams@wals@rer{Unattested}
\def\langnames@fams@wals@res{Atlantic-Congo}
\def\langnames@fams@wals@ret{Timor-Alor-Pantar}
\def\langnames@fams@wals@rey{Pano-Tacanan}
\def\langnames@fams@wals@rga{Austronesian}
\def\langnames@fams@wals@rge{Indo-European}
\def\langnames@fams@wals@rgk{Sino-Tibetan}
\def\langnames@fams@wals@rgn{Indo-European}
\def\langnames@fams@wals@rgr{Arawakan}
\def\langnames@fams@wals@rgs{Austronesian}
\def\langnames@fams@wals@rgu{Austronesian}
\def\langnames@fams@wals@rhg{Indo-European}
\def\langnames@fams@wals@rhp{Nuclear Torricelli}
\def\langnames@fams@wals@ria{Sino-Tibetan}
\def\langnames@fams@wals@rib{Sign Language}
\def\langnames@fams@wals@rif{Afro-Asiatic}
\def\langnames@fams@wals@ril{Austroasiatic}
\def\langnames@fams@wals@rim{Atlantic-Congo}
\def\langnames@fams@wals@rin{Atlantic-Congo}
\def\langnames@fams@wals@rir{Austronesian}
\def\langnames@fams@wals@rit{Pama-Nyungan}
\def\langnames@fams@wals@riu{Austronesian}
\def\langnames@fams@wals@rjg{Austronesian}
\def\langnames@fams@wals@rji{Sino-Tibetan}
\def\langnames@fams@wals@rjs{Indo-European}
\def\langnames@fams@wals@rka{Austroasiatic}
\def\langnames@fams@wals@rkb{Nuclear-Macro-Je}
\def\langnames@fams@wals@rkh{Austronesian}
\def\langnames@fams@wals@rki{Sino-Tibetan}
\def\langnames@fams@wals@rkm{Mande}
\def\langnames@fams@wals@rkt{Indo-European}
\def\langnames@fams@wals@rma{Chibchan}
\def\langnames@fams@wals@rmb{Gunwinyguan}
\def\langnames@fams@wals@rmc{Indo-European}
\def\langnames@fams@wals@rmd{Speech Register}
\def\langnames@fams@wals@rme{Indo-European}
\def\langnames@fams@wals@rmf{Indo-European}
\def\langnames@fams@wals@rmg{Speech Register}
\def\langnames@fams@wals@rmh{Lepki-Murkim-Kembra}
\def\langnames@fams@wals@rmi{Speech Register}
\def\langnames@fams@wals@rmk{Lower Sepik-Ramu}
\def\langnames@fams@wals@rml{Indo-European}
\def\langnames@fams@wals@rmm{Austronesian}
\def\langnames@fams@wals@rmn{Indo-European}
\def\langnames@fams@wals@rmo{Indo-European}
\def\langnames@fams@wals@rmp{Nuclear Trans New Guinea}
\def\langnames@fams@wals@rmq{Indo-European}
\def\langnames@fams@wals@rms{Sign Language}
\def\langnames@fams@wals@rmt{Indo-European}
\def\langnames@fams@wals@rmu{Speech Register}
\def\langnames@fams@wals@rmv{Artificial Language}
\def\langnames@fams@wals@rmw{Indo-European}
\def\langnames@fams@wals@rmx{Austroasiatic}
\def\langnames@fams@wals@rmy{Indo-European}
\def\langnames@fams@wals@rmz{Sino-Tibetan}
\def\langnames@fams@wals@rna{Unattested}
\def\langnames@fams@wals@rnb{Sign Language}
\def\langnames@fams@wals@rnd{Atlantic-Congo}
\def\langnames@fams@wals@rng{Atlantic-Congo}
\def\langnames@fams@wals@rnl{Sino-Tibetan}
\def\langnames@fams@wals@rnn{Austronesian}
\def\langnames@fams@wals@rnp{Sino-Tibetan}
\def\langnames@fams@wals@rnw{Atlantic-Congo}
\def\langnames@fams@wals@rob{Austronesian}
\def\langnames@fams@wals@roc{Austronesian}
\def\langnames@fams@wals@rod{Atlantic-Congo}
\def\langnames@fams@wals@roe{Austronesian}
\def\langnames@fams@wals@rof{Atlantic-Congo}
\def\langnames@fams@wals@rog{Austronesian}
\def\langnames@fams@wals@roh{Indo-European}
\def\langnames@fams@wals@rol{Austronesian}
\def\langnames@fams@wals@ron{Indo-European}
\def\langnames@fams@wals@roo{North Bougainville}
\def\langnames@fams@wals@rop{Indo-European}
\def\langnames@fams@wals@ror{Austronesian}
\def\langnames@fams@wals@rou{Maban}
\def\langnames@fams@wals@row{Austronesian}
\def\langnames@fams@wals@rpt{Nuclear Trans New Guinea}
\def\langnames@fams@wals@rri{Austronesian}
\def\langnames@fams@wals@rro{Austronesian}
\def\langnames@fams@wals@rsi{Artificial Language}
\def\langnames@fams@wals@rsl{Sign Language}
\def\langnames@fams@wals@rsm{Sign Language}
\def\langnames@fams@wals@rsn{Sign Language}
\def\langnames@fams@wals@rth{Austronesian}
\def\langnames@fams@wals@rtm{Austronesian}
\def\langnames@fams@wals@rtw{Indo-European}
\def\langnames@fams@wals@rub{Atlantic-Congo}
\def\langnames@fams@wals@ruc{Atlantic-Congo}
\def\langnames@fams@wals@rue{Indo-European}
\def\langnames@fams@wals@ruf{Atlantic-Congo}
\def\langnames@fams@wals@rug{Austronesian}
\def\langnames@fams@wals@ruh{Sino-Tibetan}
\def\langnames@fams@wals@ruk{Atlantic-Congo}
\def\langnames@fams@wals@run{Atlantic-Congo}
\def\langnames@fams@wals@ruo{Indo-European}
\def\langnames@fams@wals@rup{Indo-European}
\def\langnames@fams@wals@ruq{Indo-European}
\def\langnames@fams@wals@rus{Indo-European}
\def\langnames@fams@wals@rut{Nakh-Daghestanian}
\def\langnames@fams@wals@ruu{Austronesian}
\def\langnames@fams@wals@ruy{Unattested}
\def\langnames@fams@wals@ruz{Unattested}
\def\langnames@fams@wals@rwa{Sko}
\def\langnames@fams@wals@rwk{Atlantic-Congo}
\def\langnames@fams@wals@rwm{Atlantic-Congo}
\def\langnames@fams@wals@rwo{Nuclear Trans New Guinea}
\def\langnames@fams@wals@rwr{Indo-European}
\def\langnames@fams@wals@rxd{Pama-Nyungan}
\def\langnames@fams@wals@rxw{Pama-Nyungan}
\def\langnames@fams@wals@ryn{Japonic}
\def\langnames@fams@wals@rys{Japonic}
\def\langnames@fams@wals@ryu{Japonic}
\def\langnames@fams@wals@rzh{Afro-Asiatic}
\def\langnames@fams@wals@saa{Afro-Asiatic}
\def\langnames@fams@wals@sab{Chibchan}
\def\langnames@fams@wals@sac{Algic}
\def\langnames@fams@wals@sad{Isolate}
\def\langnames@fams@wals@sae{Nambiquaran}
\def\langnames@fams@wals@saf{Atlantic-Congo}
\def\langnames@fams@wals@sag{Atlantic-Congo}
\def\langnames@fams@wals@sah{Turkic}
\def\langnames@fams@wals@saj{North Halmahera}
\def\langnames@fams@wals@sak{Atlantic-Congo}
\def\langnames@fams@wals@san{Indo-European}
\def\langnames@fams@wals@sao{Isolate}
\def\langnames@fams@wals@saq{Nilotic}
\def\langnames@fams@wals@sar{Arawakan}
\def\langnames@fams@wals@sas{Austronesian}
\def\langnames@fams@wals@sat{Austroasiatic}
\def\langnames@fams@wals@sau{Austronesian}
\def\langnames@fams@wals@sav{Atlantic-Congo}
\def\langnames@fams@wals@saw{Nuclear Trans New Guinea}
\def\langnames@fams@wals@sax{Austronesian}
\def\langnames@fams@wals@say{Afro-Asiatic}
\def\langnames@fams@wals@saz{Indo-European}
\def\langnames@fams@wals@sba{Central Sudanic}
\def\langnames@fams@wals@sbb{Austronesian}
\def\langnames@fams@wals@sbc{Austronesian}
\def\langnames@fams@wals@sbd{Mande}
\def\langnames@fams@wals@sbe{Austronesian}
\def\langnames@fams@wals@sbf{Isolate}
\def\langnames@fams@wals@sbg{West Bird's Head}
\def\langnames@fams@wals@sbh{Austronesian}
\def\langnames@fams@wals@sbi{Nuclear Torricelli}
\def\langnames@fams@wals@sbj{Maban}
\def\langnames@fams@wals@sbk{Atlantic-Congo}
\def\langnames@fams@wals@sbl{Austronesian}
\def\langnames@fams@wals@sbm{Atlantic-Congo}
\def\langnames@fams@wals@sbn{Indo-European}
\def\langnames@fams@wals@sbo{Austroasiatic}
\def\langnames@fams@wals@sbp{Atlantic-Congo}
\def\langnames@fams@wals@sbq{Nuclear Trans New Guinea}
\def\langnames@fams@wals@sbr{Austronesian}
\def\langnames@fams@wals@sbs{Atlantic-Congo}
\def\langnames@fams@wals@sbt{Isolate}
\def\langnames@fams@wals@sbu{Sino-Tibetan}
\def\langnames@fams@wals@sbw{Atlantic-Congo}
\def\langnames@fams@wals@sbx{Austronesian}
\def\langnames@fams@wals@sby{Atlantic-Congo}
\def\langnames@fams@wals@sbz{Central Sudanic}
\def\langnames@fams@wals@scb{Austroasiatic}
\def\langnames@fams@wals@sce{Mongolic-Khitan}
\def\langnames@fams@wals@scg{Austronesian}
\def\langnames@fams@wals@sch{Sino-Tibetan}
\def\langnames@fams@wals@sci{Austronesian}
\def\langnames@fams@wals@sck{Indo-European}
\def\langnames@fams@wals@scl{Indo-European}
\def\langnames@fams@wals@scn{Indo-European}
\def\langnames@fams@wals@sco{Indo-European}
\def\langnames@fams@wals@scp{Sino-Tibetan}
\def\langnames@fams@wals@scq{Austroasiatic}
\def\langnames@fams@wals@scs{Athabaskan-Eyak-Tlingit}
\def\langnames@fams@wals@sct{Austroasiatic}
\def\langnames@fams@wals@scu{Sino-Tibetan}
\def\langnames@fams@wals@scv{Atlantic-Congo}
\def\langnames@fams@wals@scw{Afro-Asiatic}
\def\langnames@fams@wals@scx{Unclassifiable}
\def\langnames@fams@wals@sda{Austronesian}
\def\langnames@fams@wals@sdb{Indo-European}
\def\langnames@fams@wals@sdc{Indo-European}
\def\langnames@fams@wals@sde{Atlantic-Congo}
\def\langnames@fams@wals@sdg{Indo-European}
\def\langnames@fams@wals@sdh{Indo-European}
\def\langnames@fams@wals@sdj{Atlantic-Congo}
\def\langnames@fams@wals@sdk{Ndu}
\def\langnames@fams@wals@sdl{Sign Language}
\def\langnames@fams@wals@sdm{Austronesian}
\def\langnames@fams@wals@sdn{Indo-European}
\def\langnames@fams@wals@sdo{Austronesian}
\def\langnames@fams@wals@sdp{Sino-Tibetan}
\def\langnames@fams@wals@sdr{Indo-European}
\def\langnames@fams@wals@sds{Afro-Asiatic}
\def\langnames@fams@wals@sdu{Austronesian}
\def\langnames@fams@wals@sdx{Austronesian}
\def\langnames@fams@wals@sea{Austroasiatic}
\def\langnames@fams@wals@sec{Salishan}
\def\langnames@fams@wals@sed{Austroasiatic}
\def\langnames@fams@wals@see{Iroquoian}
\def\langnames@fams@wals@sef{Atlantic-Congo}
\def\langnames@fams@wals@seg{Atlantic-Congo}
\def\langnames@fams@wals@seh{Atlantic-Congo}
\def\langnames@fams@wals@sei{Isolate}
\def\langnames@fams@wals@sej{Nuclear Trans New Guinea}
\def\langnames@fams@wals@sek{Athabaskan-Eyak-Tlingit}
\def\langnames@fams@wals@sel{Uralic}
\def\langnames@fams@wals@sen{Atlantic-Congo}
\def\langnames@fams@wals@seo{Isolate}
\def\langnames@fams@wals@sep{Atlantic-Congo}
\def\langnames@fams@wals@seq{Atlantic-Congo}
\def\langnames@fams@wals@ser{Uto-Aztecan}
\def\langnames@fams@wals@ses{Songhay}
\def\langnames@fams@wals@set{Sentanic}
\def\langnames@fams@wals@seu{Austronesian}
\def\langnames@fams@wals@sev{Atlantic-Congo}
\def\langnames@fams@wals@sew{Austronesian}
\def\langnames@fams@wals@sey{Tucanoan}
\def\langnames@fams@wals@sez{Sino-Tibetan}
\def\langnames@fams@wals@sfb{Sign Language}
\def\langnames@fams@wals@sfe{Austronesian}
\def\langnames@fams@wals@sfm{Hmong-Mien}
\def\langnames@fams@wals@sfs{Sign Language}
\def\langnames@fams@wals@sfw{Atlantic-Congo}
\def\langnames@fams@wals@sga{Indo-European}
\def\langnames@fams@wals@sgb{Austronesian}
\def\langnames@fams@wals@sgc{Nilotic}
\def\langnames@fams@wals@sgd{Austronesian}
\def\langnames@fams@wals@sge{Austronesian}
\def\langnames@fams@wals@sgg{Sign Language}
\def\langnames@fams@wals@sgh{Indo-European}
\def\langnames@fams@wals@sgi{Atlantic-Congo}
\def\langnames@fams@wals@sgk{Sino-Tibetan}
\def\langnames@fams@wals@sgm{Atlantic-Congo}
\def\langnames@fams@wals@sgp{Sino-Tibetan}
\def\langnames@fams@wals@sgr{Indo-European}
\def\langnames@fams@wals@sgt{Sino-Tibetan}
\def\langnames@fams@wals@sgu{Austronesian}
\def\langnames@fams@wals@sgw{Afro-Asiatic}
\def\langnames@fams@wals@sgx{Sign Language}
\def\langnames@fams@wals@sgy{Indo-European}
\def\langnames@fams@wals@sgz{Austronesian}
\def\langnames@fams@wals@sha{Atlantic-Congo}
\def\langnames@fams@wals@shb{Yanomamic}
\def\langnames@fams@wals@shc{Atlantic-Congo}
\def\langnames@fams@wals@shd{Indo-European}
\def\langnames@fams@wals@she{Dizoid}
\def\langnames@fams@wals@shg{Khoe-Kwadi}
\def\langnames@fams@wals@shh{Uto-Aztecan}
\def\langnames@fams@wals@shi{Afro-Asiatic}
\def\langnames@fams@wals@shj{Dajuic}
\def\langnames@fams@wals@shk{Nilotic}
\def\langnames@fams@wals@shl{Sino-Tibetan}
\def\langnames@fams@wals@shm{Indo-European}
\def\langnames@fams@wals@shn{Tai-Kadai}
\def\langnames@fams@wals@sho{Mande}
\def\langnames@fams@wals@shp{Pano-Tacanan}
\def\langnames@fams@wals@shq{Atlantic-Congo}
\def\langnames@fams@wals@shr{Atlantic-Congo}
\def\langnames@fams@wals@shs{Salishan}
\def\langnames@fams@wals@sht{Shastan}
\def\langnames@fams@wals@shu{Afro-Asiatic}
\def\langnames@fams@wals@shv{Afro-Asiatic}
\def\langnames@fams@wals@shw{Heibanic}
\def\langnames@fams@wals@shx{Hmong-Mien}
\def\langnames@fams@wals@shy{Afro-Asiatic}
\def\langnames@fams@wals@sia{Uralic}
\def\langnames@fams@wals@sib{Austronesian}
\def\langnames@fams@wals@sid{Afro-Asiatic}
\def\langnames@fams@wals@sie{Atlantic-Congo}
\def\langnames@fams@wals@sif{Isolate}
\def\langnames@fams@wals@sig{Atlantic-Congo}
\def\langnames@fams@wals@sih{Austronesian}
\def\langnames@fams@wals@sii{Isolate}
\def\langnames@fams@wals@sij{Austronesian}
\def\langnames@fams@wals@sil{Atlantic-Congo}
\def\langnames@fams@wals@sim{Sepik}
\def\langnames@fams@wals@sin{Indo-European}
\def\langnames@fams@wals@sip{Sino-Tibetan}
\def\langnames@fams@wals@siq{Bosavi}
\def\langnames@fams@wals@sir{Afro-Asiatic}
\def\langnames@fams@wals@sis{Isolate}
\def\langnames@fams@wals@siu{Nuclear Torricelli}
\def\langnames@fams@wals@siv{Sepik}
\def\langnames@fams@wals@siw{South Bougainville}
\def\langnames@fams@wals@six{Nuclear Trans New Guinea}
\def\langnames@fams@wals@siy{Indo-European}
\def\langnames@fams@wals@siz{Afro-Asiatic}
\def\langnames@fams@wals@sja{Chocoan}
\def\langnames@fams@wals@sjb{Austronesian}
\def\langnames@fams@wals@sjd{Uralic}
\def\langnames@fams@wals@sje{Uralic}
\def\langnames@fams@wals@sjg{Tamaic}
\def\langnames@fams@wals@sjk{Uralic}
\def\langnames@fams@wals@sjl{Sino-Tibetan}
\def\langnames@fams@wals@sjm{Austronesian}
\def\langnames@fams@wals@sjn{Artificial Language}
\def\langnames@fams@wals@sjo{Tungusic}
\def\langnames@fams@wals@sjp{Indo-European}
\def\langnames@fams@wals@sjr{Austronesian}
\def\langnames@fams@wals@sjs{Afro-Asiatic}
\def\langnames@fams@wals@sjt{Uralic}
\def\langnames@fams@wals@sju{Uralic}
\def\langnames@fams@wals@sjw{Algic}
\def\langnames@fams@wals@skb{Tai-Kadai}
\def\langnames@fams@wals@skc{Nuclear Trans New Guinea}
\def\langnames@fams@wals@skd{Miwok-Costanoan}
\def\langnames@fams@wals@ske{Austronesian}
\def\langnames@fams@wals@skf{Tupian}
\def\langnames@fams@wals@skg{Austronesian}
\def\langnames@fams@wals@skh{Austronesian}
\def\langnames@fams@wals@ski{Austronesian}
\def\langnames@fams@wals@skj{Sino-Tibetan}
\def\langnames@fams@wals@skm{Nuclear Trans New Guinea}
\def\langnames@fams@wals@skn{Austronesian}
\def\langnames@fams@wals@sko{Austronesian}
\def\langnames@fams@wals@skp{Austronesian}
\def\langnames@fams@wals@skq{Mande}
\def\langnames@fams@wals@skr{Indo-European}
\def\langnames@fams@wals@sks{Nuclear Trans New Guinea}
\def\langnames@fams@wals@skt{Atlantic-Congo}
\def\langnames@fams@wals@sku{Austronesian}
\def\langnames@fams@wals@skv{Sko}
\def\langnames@fams@wals@skw{Indo-European}
\def\langnames@fams@wals@skx{Austronesian}
\def\langnames@fams@wals@sky{Austronesian}
\def\langnames@fams@wals@skz{Austronesian}
\def\langnames@fams@wals@slc{Saliban}
\def\langnames@fams@wals@sld{Atlantic-Congo}
\def\langnames@fams@wals@sle{Dravidian}
\def\langnames@fams@wals@slf{Sign Language}
\def\langnames@fams@wals@slg{Austronesian}
\def\langnames@fams@wals@slh{Salishan}
\def\langnames@fams@wals@sli{Indo-European}
\def\langnames@fams@wals@slk{Indo-European}
\def\langnames@fams@wals@sll{Nuclear Trans New Guinea}
\def\langnames@fams@wals@slm{Austronesian}
\def\langnames@fams@wals@sln{Isolate}
\def\langnames@fams@wals@slp{Austronesian}
\def\langnames@fams@wals@slq{Turkic}
\def\langnames@fams@wals@slr{Turkic}
\def\langnames@fams@wals@slt{Sino-Tibetan}
\def\langnames@fams@wals@slu{Austronesian}
\def\langnames@fams@wals@slv{Indo-European}
\def\langnames@fams@wals@slw{Nuclear Trans New Guinea}
\def\langnames@fams@wals@slx{Atlantic-Congo}
\def\langnames@fams@wals@sly{Austronesian}
\def\langnames@fams@wals@slz{Austronesian}
\def\langnames@fams@wals@sma{Uralic}
\def\langnames@fams@wals@smb{Angan}
\def\langnames@fams@wals@smc{Nuclear Trans New Guinea}
\def\langnames@fams@wals@sme{Uralic}
\def\langnames@fams@wals@smf{Border}
\def\langnames@fams@wals@smg{Baining}
\def\langnames@fams@wals@smh{Sino-Tibetan}
\def\langnames@fams@wals@smj{Uralic}
\def\langnames@fams@wals@smk{Austronesian}
\def\langnames@fams@wals@sml{Austronesian}
\def\langnames@fams@wals@smm{Indo-European}
\def\langnames@fams@wals@smn{Uralic}
\def\langnames@fams@wals@smo{Austronesian}
\def\langnames@fams@wals@smp{Afro-Asiatic}
\def\langnames@fams@wals@smq{East Strickland}
\def\langnames@fams@wals@smr{Austronesian}
\def\langnames@fams@wals@sms{Uralic}
\def\langnames@fams@wals@smt{Sino-Tibetan}
\def\langnames@fams@wals@smu{Austroasiatic}
\def\langnames@fams@wals@smv{Indo-European}
\def\langnames@fams@wals@smw{Austronesian}
\def\langnames@fams@wals@smx{Atlantic-Congo}
\def\langnames@fams@wals@smy{Indo-European}
\def\langnames@fams@wals@smz{South Bougainville}
\def\langnames@fams@wals@sna{Atlantic-Congo}
\def\langnames@fams@wals@snc{Austronesian}
\def\langnames@fams@wals@snd{Indo-European}
\def\langnames@fams@wals@sne{Austronesian}
\def\langnames@fams@wals@snf{Atlantic-Congo}
\def\langnames@fams@wals@sng{Atlantic-Congo}
\def\langnames@fams@wals@snh{Unattested}
\def\langnames@fams@wals@sni{Pano-Tacanan}
\def\langnames@fams@wals@snj{Atlantic-Congo}
\def\langnames@fams@wals@snk{Mande}
\def\langnames@fams@wals@snl{Austronesian}
\def\langnames@fams@wals@snm{Central Sudanic}
\def\langnames@fams@wals@snn{Tucanoan}
\def\langnames@fams@wals@snp{Nuclear Trans New Guinea}
\def\langnames@fams@wals@snq{Atlantic-Congo}
\def\langnames@fams@wals@snr{Nuclear Trans New Guinea}
\def\langnames@fams@wals@sns{Austronesian}
\def\langnames@fams@wals@snu{Border}
\def\langnames@fams@wals@snv{Austronesian}
\def\langnames@fams@wals@snw{Atlantic-Congo}
\def\langnames@fams@wals@snx{Nuclear Trans New Guinea}
\def\langnames@fams@wals@sny{Sepik}
\def\langnames@fams@wals@snz{Nuclear Trans New Guinea}
\def\langnames@fams@wals@soa{Tai-Kadai}
\def\langnames@fams@wals@sob{Austronesian}
\def\langnames@fams@wals@soc{Atlantic-Congo}
\def\langnames@fams@wals@sod{Atlantic-Congo}
\def\langnames@fams@wals@soe{Atlantic-Congo}
\def\langnames@fams@wals@sog{Indo-European}
\def\langnames@fams@wals@soh{Eastern Jebel}
\def\langnames@fams@wals@soi{Indo-European}
\def\langnames@fams@wals@soj{Indo-European}
\def\langnames@fams@wals@sok{Afro-Asiatic}
\def\langnames@fams@wals@sol{Austronesian}
\def\langnames@fams@wals@som{Afro-Asiatic}
\def\langnames@fams@wals@soo{Atlantic-Congo}
\def\langnames@fams@wals@sop{Atlantic-Congo}
\def\langnames@fams@wals@soq{Dagan}
\def\langnames@fams@wals@sor{Afro-Asiatic}
\def\langnames@fams@wals@sos{Mande}
\def\langnames@fams@wals@sot{Atlantic-Congo}
\def\langnames@fams@wals@sou{Tai-Kadai}
\def\langnames@fams@wals@sov{Austronesian}
\def\langnames@fams@wals@sow{Border}
\def\langnames@fams@wals@sox{Atlantic-Congo}
\def\langnames@fams@wals@soy{Atlantic-Congo}
\def\langnames@fams@wals@soz{Atlantic-Congo}
\def\langnames@fams@wals@spa{Indo-European}
\def\langnames@fams@wals@spb{Austronesian}
\def\langnames@fams@wals@spc{Isolate}
\def\langnames@fams@wals@spd{Nuclear Trans New Guinea}
\def\langnames@fams@wals@spe{Austronesian}
\def\langnames@fams@wals@spg{Austronesian}
\def\langnames@fams@wals@spi{Lakes Plain}
\def\langnames@fams@wals@spk{Ndu}
\def\langnames@fams@wals@spl{Nuclear Trans New Guinea}
\def\langnames@fams@wals@spm{Lower Sepik-Ramu}
\def\langnames@fams@wals@spn{Lengua-Mascoy}
\def\langnames@fams@wals@spo{Salishan}
\def\langnames@fams@wals@spp{Atlantic-Congo}
\def\langnames@fams@wals@spq{Indo-European}
\def\langnames@fams@wals@spr{Austronesian}
\def\langnames@fams@wals@sps{Austronesian}
\def\langnames@fams@wals@spt{Sino-Tibetan}
\def\langnames@fams@wals@spu{Austroasiatic}
\def\langnames@fams@wals@spv{Indo-European}
\def\langnames@fams@wals@spy{Nilotic}
\def\langnames@fams@wals@sqa{Atlantic-Congo}
\def\langnames@fams@wals@sqh{Atlantic-Congo}
\def\langnames@fams@wals@sqk{Sign Language}
\def\langnames@fams@wals@sqm{Atlantic-Congo}
\def\langnames@fams@wals@sqn{Iroquoian}
\def\langnames@fams@wals@sqo{Indo-European}
\def\langnames@fams@wals@sqq{Austroasiatic}
\def\langnames@fams@wals@sqs{Sign Language}
\def\langnames@fams@wals@sqt{Afro-Asiatic}
\def\langnames@fams@wals@squ{Salishan}
\def\langnames@fams@wals@sqx{Sign Language}
\def\langnames@fams@wals@sra{Nuclear Trans New Guinea}
\def\langnames@fams@wals@srb{Austroasiatic}
\def\langnames@fams@wals@src{Indo-European}
\def\langnames@fams@wals@sre{Austronesian}
\def\langnames@fams@wals@srf{Austronesian}
\def\langnames@fams@wals@srg{Austronesian}
\def\langnames@fams@wals@srh{Indo-European}
\def\langnames@fams@wals@sri{Tucanoan}
\def\langnames@fams@wals@srk{Austronesian}
\def\langnames@fams@wals@srl{Greater Kwerba}
\def\langnames@fams@wals@srm{Indo-European}
\def\langnames@fams@wals@srn{Indo-European}
\def\langnames@fams@wals@sro{Indo-European}
\def\langnames@fams@wals@srq{Tupian}
\def\langnames@fams@wals@srr{Atlantic-Congo}
\def\langnames@fams@wals@srs{Athabaskan-Eyak-Tlingit}
\def\langnames@fams@wals@srt{Geelvink Bay}
\def\langnames@fams@wals@sru{Tupian}
\def\langnames@fams@wals@srv{Austronesian}
\def\langnames@fams@wals@srw{Austronesian}
\def\langnames@fams@wals@srx{Indo-European}
\def\langnames@fams@wals@sry{Austronesian}
\def\langnames@fams@wals@srz{Indo-European}
\def\langnames@fams@wals@ssb{Austronesian}
\def\langnames@fams@wals@ssc{Atlantic-Congo}
\def\langnames@fams@wals@ssd{Nuclear Trans New Guinea}
\def\langnames@fams@wals@sse{Austronesian}
\def\langnames@fams@wals@ssf{Austronesian}
\def\langnames@fams@wals@ssg{Austronesian}
\def\langnames@fams@wals@ssh{Afro-Asiatic}
\def\langnames@fams@wals@ssi{Indo-European}
\def\langnames@fams@wals@ssj{Nuclear Trans New Guinea}
\def\langnames@fams@wals@ssk{Sino-Tibetan}
\def\langnames@fams@wals@ssl{Atlantic-Congo}
\def\langnames@fams@wals@ssm{Austroasiatic}
\def\langnames@fams@wals@ssn{Afro-Asiatic}
\def\langnames@fams@wals@sso{Austronesian}
\def\langnames@fams@wals@ssp{Sign Language}
\def\langnames@fams@wals@ssr{Sign Language}
\def\langnames@fams@wals@sss{Austroasiatic}
\def\langnames@fams@wals@sst{Nuclear Trans New Guinea}
\def\langnames@fams@wals@ssu{Angan}
\def\langnames@fams@wals@ssv{Austronesian}
\def\langnames@fams@wals@ssw{Atlantic-Congo}
\def\langnames@fams@wals@ssx{Nuclear Trans New Guinea}
\def\langnames@fams@wals@ssy{Afro-Asiatic}
\def\langnames@fams@wals@ssz{Austronesian}
\def\langnames@fams@wals@sta{Pidgin}
\def\langnames@fams@wals@stb{Austronesian}
\def\langnames@fams@wals@std{Unattested}
\def\langnames@fams@wals@ste{Austronesian}
\def\langnames@fams@wals@stf{Nuclear Torricelli}
\def\langnames@fams@wals@stg{Austroasiatic}
\def\langnames@fams@wals@sth{Speech Register}
\def\langnames@fams@wals@sti{Austroasiatic}
\def\langnames@fams@wals@stj{Mande}
\def\langnames@fams@wals@stk{Yam}
\def\langnames@fams@wals@stm{Nuclear Trans New Guinea}
\def\langnames@fams@wals@stn{Austronesian}
\def\langnames@fams@wals@sto{Siouan}
\def\langnames@fams@wals@stp{Uto-Aztecan}
\def\langnames@fams@wals@stq{Indo-European}
\def\langnames@fams@wals@str{Salishan}
\def\langnames@fams@wals@sts{Indo-European}
\def\langnames@fams@wals@stt{Austroasiatic}
\def\langnames@fams@wals@stu{Austroasiatic}
\def\langnames@fams@wals@stv{Afro-Asiatic}
\def\langnames@fams@wals@stw{Austronesian}
\def\langnames@fams@wals@sty{Turkic}
\def\langnames@fams@wals@sua{Isolate}
\def\langnames@fams@wals@sub{Atlantic-Congo}
\def\langnames@fams@wals@suc{Austronesian}
\def\langnames@fams@wals@sue{Nuclear Trans New Guinea}
\def\langnames@fams@wals@sug{Nuclear Trans New Guinea}
\def\langnames@fams@wals@sui{Suki-Gogodala}
\def\langnames@fams@wals@suj{Atlantic-Congo}
\def\langnames@fams@wals@suk{Atlantic-Congo}
\def\langnames@fams@wals@sun{Austronesian}
\def\langnames@fams@wals@suo{Sko}
\def\langnames@fams@wals@suq{Surmic}
\def\langnames@fams@wals@sur{Afro-Asiatic}
\def\langnames@fams@wals@sus{Mande}
\def\langnames@fams@wals@sut{Otomanguean}
\def\langnames@fams@wals@suv{Sino-Tibetan}
\def\langnames@fams@wals@suw{Atlantic-Congo}
\def\langnames@fams@wals@sux{Isolate}
\def\langnames@fams@wals@suy{Nuclear-Macro-Je}
\def\langnames@fams@wals@suz{Sino-Tibetan}
\def\langnames@fams@wals@sva{Kartvelian}
\def\langnames@fams@wals@svb{Austronesian}
\def\langnames@fams@wals@svc{Indo-European}
\def\langnames@fams@wals@sve{Austronesian}
\def\langnames@fams@wals@svk{Sign Language}
\def\langnames@fams@wals@svm{Indo-European}
\def\langnames@fams@wals@svs{Isolate}
\def\langnames@fams@wals@swb{Atlantic-Congo}
\def\langnames@fams@wals@swc{Atlantic-Congo}
\def\langnames@fams@wals@swe{Indo-European}
\def\langnames@fams@wals@swf{Atlantic-Congo}
\def\langnames@fams@wals@swg{Indo-European}
\def\langnames@fams@wals@swh{Atlantic-Congo}
\def\langnames@fams@wals@swi{Tai-Kadai}
\def\langnames@fams@wals@swj{Atlantic-Congo}
\def\langnames@fams@wals@swk{Atlantic-Congo}
\def\langnames@fams@wals@swl{Sign Language}
\def\langnames@fams@wals@swm{Nuclear Trans New Guinea}
\def\langnames@fams@wals@swn{Afro-Asiatic}
\def\langnames@fams@wals@swo{Pano-Tacanan}
\def\langnames@fams@wals@swp{Austronesian}
\def\langnames@fams@wals@swq{Afro-Asiatic}
\def\langnames@fams@wals@swr{Yawa-Saweru}
\def\langnames@fams@wals@sws{Austronesian}
\def\langnames@fams@wals@swt{Timor-Alor-Pantar}
\def\langnames@fams@wals@swu{Austronesian}
\def\langnames@fams@wals@swv{Indo-European}
\def\langnames@fams@wals@sww{Austronesian}
\def\langnames@fams@wals@swx{Arawan}
\def\langnames@fams@wals@swy{Afro-Asiatic}
\def\langnames@fams@wals@sxb{Atlantic-Congo}
\def\langnames@fams@wals@sxc{Unclassifiable}
\def\langnames@fams@wals@sxe{Atlantic-Congo}
\def\langnames@fams@wals@sxg{Sino-Tibetan}
\def\langnames@fams@wals@sxk{Kalapuyan}
\def\langnames@fams@wals@sxn{Austronesian}
\def\langnames@fams@wals@sxr{Austronesian}
\def\langnames@fams@wals@sxs{Atlantic-Congo}
\def\langnames@fams@wals@sxu{Indo-European}
\def\langnames@fams@wals@sxw{Atlantic-Congo}
\def\langnames@fams@wals@sya{Austronesian}
\def\langnames@fams@wals@syb{Austronesian}
\def\langnames@fams@wals@syc{Afro-Asiatic}
\def\langnames@fams@wals@syi{Atlantic-Congo}
\def\langnames@fams@wals@syk{Afro-Asiatic}
\def\langnames@fams@wals@syl{Indo-European}
\def\langnames@fams@wals@sym{Mande}
\def\langnames@fams@wals@syo{Austroasiatic}
\def\langnames@fams@wals@sys{Central Sudanic}
\def\langnames@fams@wals@syw{Sino-Tibetan}
\def\langnames@fams@wals@syx{Atlantic-Congo}
\def\langnames@fams@wals@syy{Sign Language}
\def\langnames@fams@wals@sza{Austroasiatic}
\def\langnames@fams@wals@szb{Nuclear Trans New Guinea}
\def\langnames@fams@wals@szc{Austroasiatic}
\def\langnames@fams@wals@sze{Blue Nile Mao}
\def\langnames@fams@wals@szg{Atlantic-Congo}
\def\langnames@fams@wals@szl{Indo-European}
\def\langnames@fams@wals@szn{Austronesian}
\def\langnames@fams@wals@szp{Inanwatan}
\def\langnames@fams@wals@szs{Sign Language}
\def\langnames@fams@wals@szv{Atlantic-Congo}
\def\langnames@fams@wals@szw{Austronesian}
\def\langnames@fams@wals@szy{Austronesian}
\def\langnames@fams@wals@taa{Athabaskan-Eyak-Tlingit}
\def\langnames@fams@wals@tab{Nakh-Daghestanian}
\def\langnames@fams@wals@tac{Uto-Aztecan}
\def\langnames@fams@wals@tad{Lakes Plain}
\def\langnames@fams@wals@tae{Arawakan}
\def\langnames@fams@wals@taf{Tupian}
\def\langnames@fams@wals@tag{Rashad}
\def\langnames@fams@wals@tah{Austronesian}
\def\langnames@fams@wals@taj{Sino-Tibetan}
\def\langnames@fams@wals@tak{Afro-Asiatic}
\def\langnames@fams@wals@tal{Afro-Asiatic}
\def\langnames@fams@wals@tam{Dravidian}
\def\langnames@fams@wals@tan{Afro-Asiatic}
\def\langnames@fams@wals@tao{Austronesian}
\def\langnames@fams@wals@tap{Atlantic-Congo}
\def\langnames@fams@wals@taq{Afro-Asiatic}
\def\langnames@fams@wals@tar{Uto-Aztecan}
\def\langnames@fams@wals@tas{Pidgin}
\def\langnames@fams@wals@tat{Turkic}
\def\langnames@fams@wals@tau{Athabaskan-Eyak-Tlingit}
\def\langnames@fams@wals@tav{Tucanoan}
\def\langnames@fams@wals@taw{Nuclear Trans New Guinea}
\def\langnames@fams@wals@tax{Afro-Asiatic}
\def\langnames@fams@wals@tay{Austronesian}
\def\langnames@fams@wals@taz{Narrow Talodi}
\def\langnames@fams@wals@tba{Isolate}
\def\langnames@fams@wals@tbc{Austronesian}
\def\langnames@fams@wals@tbd{Isolate}
\def\langnames@fams@wals@tbe{Austronesian}
\def\langnames@fams@wals@tbf{Austronesian}
\def\langnames@fams@wals@tbg{Nuclear Trans New Guinea}
\def\langnames@fams@wals@tbh{Pama-Nyungan}
\def\langnames@fams@wals@tbi{Eastern Jebel}
\def\langnames@fams@wals@tbj{Austronesian}
\def\langnames@fams@wals@tbk{Austronesian}
\def\langnames@fams@wals@tbl{Austronesian}
\def\langnames@fams@wals@tbm{Atlantic-Congo}
\def\langnames@fams@wals@tbn{Chibchan}
\def\langnames@fams@wals@tbo{Austronesian}
\def\langnames@fams@wals@tbp{Lakes Plain}
\def\langnames@fams@wals@tbr{Kadugli-Krongo}
\def\langnames@fams@wals@tbs{Lower Sepik-Ramu}
\def\langnames@fams@wals@tbt{Atlantic-Congo}
\def\langnames@fams@wals@tbu{Uto-Aztecan}
\def\langnames@fams@wals@tbw{Austronesian}
\def\langnames@fams@wals@tbx{Austronesian}
\def\langnames@fams@wals@tby{North Halmahera}
\def\langnames@fams@wals@tbz{Atlantic-Congo}
\def\langnames@fams@wals@tca{Ticuna-Yuri}
\def\langnames@fams@wals@tcb{Athabaskan-Eyak-Tlingit}
\def\langnames@fams@wals@tcc{Nilotic}
\def\langnames@fams@wals@tcd{Atlantic-Congo}
\def\langnames@fams@wals@tce{Athabaskan-Eyak-Tlingit}
\def\langnames@fams@wals@tcf{Otomanguean}
\def\langnames@fams@wals@tcg{Kayagaric}
\def\langnames@fams@wals@tch{Indo-European}
\def\langnames@fams@wals@tci{Yam}
\def\langnames@fams@wals@tck{Atlantic-Congo}
\def\langnames@fams@wals@tcl{Sino-Tibetan}
\def\langnames@fams@wals@tcm{Isolate}
\def\langnames@fams@wals@tcn{Sino-Tibetan}
\def\langnames@fams@wals@tco{Sino-Tibetan}
\def\langnames@fams@wals@tcp{Sino-Tibetan}
\def\langnames@fams@wals@tcq{Lakes Plain}
\def\langnames@fams@wals@tcs{Indo-European}
\def\langnames@fams@wals@tct{Tai-Kadai}
\def\langnames@fams@wals@tcu{Uto-Aztecan}
\def\langnames@fams@wals@tcw{Totonacan}
\def\langnames@fams@wals@tcx{Dravidian}
\def\langnames@fams@wals@tcy{Dravidian}
\def\langnames@fams@wals@tcz{Sino-Tibetan}
\def\langnames@fams@wals@tda{Songhay}
\def\langnames@fams@wals@tdb{Indo-European}
\def\langnames@fams@wals@tdc{Chocoan}
\def\langnames@fams@wals@tdd{Tai-Kadai}
\def\langnames@fams@wals@tde{Dogon}
\def\langnames@fams@wals@tdf{Austroasiatic}
\def\langnames@fams@wals@tdg{Sino-Tibetan}
\def\langnames@fams@wals@tdh{Sino-Tibetan}
\def\langnames@fams@wals@tdi{Austronesian}
\def\langnames@fams@wals@tdj{Austronesian}
\def\langnames@fams@wals@tdk{Afro-Asiatic}
\def\langnames@fams@wals@tdl{Atlantic-Congo}
\def\langnames@fams@wals@tdm{Isolate}
\def\langnames@fams@wals@tdn{Austronesian}
\def\langnames@fams@wals@tdo{Atlantic-Congo}
\def\langnames@fams@wals@tdq{Atlantic-Congo}
\def\langnames@fams@wals@tdr{Austroasiatic}
\def\langnames@fams@wals@tds{Lakes Plain}
\def\langnames@fams@wals@tdt{Austronesian}
\def\langnames@fams@wals@tdv{Atlantic-Congo}
\def\langnames@fams@wals@tdx{Austronesian}
\def\langnames@fams@wals@tdy{Austronesian}
\def\langnames@fams@wals@tea{Austroasiatic}
\def\langnames@fams@wals@tec{Nilotic}
\def\langnames@fams@wals@ted{Kru}
\def\langnames@fams@wals@tee{Totonacan}
\def\langnames@fams@wals@tef{Austroasiatic}
\def\langnames@fams@wals@teg{Atlantic-Congo}
\def\langnames@fams@wals@teh{Chonan}
\def\langnames@fams@wals@tei{Nuclear Torricelli}
\def\langnames@fams@wals@tek{Atlantic-Congo}
\def\langnames@fams@wals@tel{Dravidian}
\def\langnames@fams@wals@tem{Atlantic-Congo}
\def\langnames@fams@wals@ten{Tucanoan}
\def\langnames@fams@wals@teo{Nilotic}
\def\langnames@fams@wals@tep{Uto-Aztecan}
\def\langnames@fams@wals@teq{Temeinic}
\def\langnames@fams@wals@ter{Arawakan}
\def\langnames@fams@wals@tes{Austronesian}
\def\langnames@fams@wals@tet{Austronesian}
\def\langnames@fams@wals@teu{Kuliak}
\def\langnames@fams@wals@tev{Austronesian}
\def\langnames@fams@wals@tew{Kiowa-Tanoan}
\def\langnames@fams@wals@tex{Surmic}
\def\langnames@fams@wals@tey{Kadugli-Krongo}
\def\langnames@fams@wals@tez{Afro-Asiatic}
\def\langnames@fams@wals@tfi{Atlantic-Congo}
\def\langnames@fams@wals@tfn{Athabaskan-Eyak-Tlingit}
\def\langnames@fams@wals@tfo{Geelvink Bay}
\def\langnames@fams@wals@tfr{Chibchan}
\def\langnames@fams@wals@tft{North Halmahera}
\def\langnames@fams@wals@tga{Atlantic-Congo}
\def\langnames@fams@wals@tgb{Austronesian}
\def\langnames@fams@wals@tgc{Austronesian}
\def\langnames@fams@wals@tgd{Afro-Asiatic}
\def\langnames@fams@wals@tge{Sino-Tibetan}
\def\langnames@fams@wals@tgf{Sino-Tibetan}
\def\langnames@fams@wals@tgg{Austronesian}
\def\langnames@fams@wals@tgh{Indo-European}
\def\langnames@fams@wals@tgi{Austronesian}
\def\langnames@fams@wals@tgj{Sino-Tibetan}
\def\langnames@fams@wals@tgk{Indo-European}
\def\langnames@fams@wals@tgl{Austronesian}
\def\langnames@fams@wals@tgn{Austronesian}
\def\langnames@fams@wals@tgo{Austronesian}
\def\langnames@fams@wals@tgp{Austronesian}
\def\langnames@fams@wals@tgq{Austronesian}
\def\langnames@fams@wals@tgs{Austronesian}
\def\langnames@fams@wals@tgt{Austronesian}
\def\langnames@fams@wals@tgu{Lower Sepik-Ramu}
\def\langnames@fams@wals@tgw{Atlantic-Congo}
\def\langnames@fams@wals@tgx{Athabaskan-Eyak-Tlingit}
\def\langnames@fams@wals@tgy{Atlantic-Congo}
\def\langnames@fams@wals@tgz{Pama-Nyungan}
\def\langnames@fams@wals@tha{Tai-Kadai}
\def\langnames@fams@wals@thd{Pama-Nyungan}
\def\langnames@fams@wals@the{Indo-European}
\def\langnames@fams@wals@thf{Sino-Tibetan}
\def\langnames@fams@wals@thh{Uto-Aztecan}
\def\langnames@fams@wals@thi{Tai-Kadai}
\def\langnames@fams@wals@thk{Atlantic-Congo}
\def\langnames@fams@wals@thl{Indo-European}
\def\langnames@fams@wals@thm{Austroasiatic}
\def\langnames@fams@wals@thn{Dravidian}
\def\langnames@fams@wals@thp{Salishan}
\def\langnames@fams@wals@thq{Indo-European}
\def\langnames@fams@wals@thr{Indo-European}
\def\langnames@fams@wals@ths{Sino-Tibetan}
\def\langnames@fams@wals@tht{Athabaskan-Eyak-Tlingit}
\def\langnames@fams@wals@thu{Nilotic}
\def\langnames@fams@wals@thv{Afro-Asiatic}
\def\langnames@fams@wals@thy{Atlantic-Congo}
\def\langnames@fams@wals@thz{Afro-Asiatic}
\def\langnames@fams@wals@tia{Afro-Asiatic}
\def\langnames@fams@wals@tic{Heibanic}
\def\langnames@fams@wals@tif{Nuclear Trans New Guinea}
\def\langnames@fams@wals@tig{Afro-Asiatic}
\def\langnames@fams@wals@tih{Austronesian}
\def\langnames@fams@wals@tii{Atlantic-Congo}
\def\langnames@fams@wals@tij{Sino-Tibetan}
\def\langnames@fams@wals@tik{Atlantic-Congo}
\def\langnames@fams@wals@til{Salishan}
\def\langnames@fams@wals@tim{Nuclear Trans New Guinea}
\def\langnames@fams@wals@tin{Nakh-Daghestanian}
\def\langnames@fams@wals@tio{Austronesian}
\def\langnames@fams@wals@tip{Greater Kwerba}
\def\langnames@fams@wals@tiq{Atlantic-Congo}
\def\langnames@fams@wals@tir{Afro-Asiatic}
\def\langnames@fams@wals@tis{Austronesian}
\def\langnames@fams@wals@tit{Isolate}
\def\langnames@fams@wals@tiu{Austronesian}
\def\langnames@fams@wals@tiv{Atlantic-Congo}
\def\langnames@fams@wals@tiw{Isolate}
\def\langnames@fams@wals@tix{Kiowa-Tanoan}
\def\langnames@fams@wals@tiy{Austronesian}
\def\langnames@fams@wals@tiz{Tai-Kadai}
\def\langnames@fams@wals@tja{Kru}
\def\langnames@fams@wals@tjg{Austronesian}
\def\langnames@fams@wals@tji{Sino-Tibetan}
\def\langnames@fams@wals@tjj{Pama-Nyungan}
\def\langnames@fams@wals@tjl{Tai-Kadai}
\def\langnames@fams@wals@tjm{Isolate}
\def\langnames@fams@wals@tjn{Mande}
\def\langnames@fams@wals@tjo{Afro-Asiatic}
\def\langnames@fams@wals@tjp{Pama-Nyungan}
\def\langnames@fams@wals@tjs{Sino-Tibetan}
\def\langnames@fams@wals@tju{Pama-Nyungan}
\def\langnames@fams@wals@tka{Unattested}
\def\langnames@fams@wals@tkb{Indo-European}
\def\langnames@fams@wals@tkd{Austronesian}
\def\langnames@fams@wals@tke{Atlantic-Congo}
\def\langnames@fams@wals@tkf{Unattested}
\def\langnames@fams@wals@tkg{Austronesian}
\def\langnames@fams@wals@tkl{Austronesian}
\def\langnames@fams@wals@tkm{Isolate}
\def\langnames@fams@wals@tkn{Japonic}
\def\langnames@fams@wals@tkp{Austronesian}
\def\langnames@fams@wals@tkq{Atlantic-Congo}
\def\langnames@fams@wals@tkr{Nakh-Daghestanian}
\def\langnames@fams@wals@tks{Indo-European}
\def\langnames@fams@wals@tkt{Indo-European}
\def\langnames@fams@wals@tku{Totonacan}
\def\langnames@fams@wals@tkv{Austronesian}
\def\langnames@fams@wals@tkw{Austronesian}
\def\langnames@fams@wals@tkx{Nuclear Trans New Guinea}
\def\langnames@fams@wals@tkz{Austroasiatic}
\def\langnames@fams@wals@tla{Uto-Aztecan}
\def\langnames@fams@wals@tlb{North Halmahera}
\def\langnames@fams@wals@tlc{Totonacan}
\def\langnames@fams@wals@tld{Austronesian}
\def\langnames@fams@wals@tlf{Nuclear Trans New Guinea}
\def\langnames@fams@wals@tlg{Namla-Tofanma}
\def\langnames@fams@wals@tlh{Artificial Language}
\def\langnames@fams@wals@tli{Athabaskan-Eyak-Tlingit}
\def\langnames@fams@wals@tlj{Atlantic-Congo}
\def\langnames@fams@wals@tlk{Austronesian}
\def\langnames@fams@wals@tll{Atlantic-Congo}
\def\langnames@fams@wals@tlm{Austronesian}
\def\langnames@fams@wals@tln{Austronesian}
\def\langnames@fams@wals@tlo{Narrow Talodi}
\def\langnames@fams@wals@tlp{Totonacan}
\def\langnames@fams@wals@tlq{Austroasiatic}
\def\langnames@fams@wals@tlr{Austronesian}
\def\langnames@fams@wals@tls{Austronesian}
\def\langnames@fams@wals@tlt{Austronesian}
\def\langnames@fams@wals@tlu{Austronesian}
\def\langnames@fams@wals@tlv{Austronesian}
\def\langnames@fams@wals@tlx{Austronesian}
\def\langnames@fams@wals@tly{Indo-European}
\def\langnames@fams@wals@tma{Tamaic}
\def\langnames@fams@wals@tmb{Austronesian}
\def\langnames@fams@wals@tmc{Afro-Asiatic}
\def\langnames@fams@wals@tmd{Piawi}
\def\langnames@fams@wals@tme{Unattested}
\def\langnames@fams@wals@tmf{Lengua-Mascoy}
\def\langnames@fams@wals@tmg{Indo-European}
\def\langnames@fams@wals@tmi{Austronesian}
\def\langnames@fams@wals@tmj{Greater Kwerba}
\def\langnames@fams@wals@tml{Nuclear Trans New Guinea}
\def\langnames@fams@wals@tmm{Tai-Kadai}
\def\langnames@fams@wals@tmn{Austronesian}
\def\langnames@fams@wals@tmo{Austroasiatic}
\def\langnames@fams@wals@tmq{Austronesian}
\def\langnames@fams@wals@tmr{Afro-Asiatic}
\def\langnames@fams@wals@tms{Katla-Tima}
\def\langnames@fams@wals@tmt{Austronesian}
\def\langnames@fams@wals@tmu{Lakes Plain}
\def\langnames@fams@wals@tmv{Atlantic-Congo}
\def\langnames@fams@wals@tmw{Austronesian}
\def\langnames@fams@wals@tmy{Austronesian}
\def\langnames@fams@wals@tmz{Cariban}
\def\langnames@fams@wals@tna{Pano-Tacanan}
\def\langnames@fams@wals@tnb{Chibchan}
\def\langnames@fams@wals@tnc{Tucanoan}
\def\langnames@fams@wals@tnd{Chibchan}
\def\langnames@fams@wals@tng{Afro-Asiatic}
\def\langnames@fams@wals@tnh{Nuclear Trans New Guinea}
\def\langnames@fams@wals@tni{Austronesian}
\def\langnames@fams@wals@tnk{Austronesian}
\def\langnames@fams@wals@tnl{Austronesian}
\def\langnames@fams@wals@tnm{Sentanic}
\def\langnames@fams@wals@tnn{Austronesian}
\def\langnames@fams@wals@tno{Pano-Tacanan}
\def\langnames@fams@wals@tnp{Austronesian}
\def\langnames@fams@wals@tnq{Arawakan}
\def\langnames@fams@wals@tnr{Atlantic-Congo}
\def\langnames@fams@wals@tns{Austronesian}
\def\langnames@fams@wals@tnt{Austronesian}
\def\langnames@fams@wals@tnu{Tai-Kadai}
\def\langnames@fams@wals@tnv{Indo-European}
\def\langnames@fams@wals@tnw{Austronesian}
\def\langnames@fams@wals@tnx{Austronesian}
\def\langnames@fams@wals@tny{Atlantic-Congo}
\def\langnames@fams@wals@tnz{Austroasiatic}
\def\langnames@fams@wals@tob{Guaicuruan}
\def\langnames@fams@wals@toc{Totonacan}
\def\langnames@fams@wals@tod{Mande}
\def\langnames@fams@wals@tof{Eastern Trans-Fly}
\def\langnames@fams@wals@tog{Atlantic-Congo}
\def\langnames@fams@wals@toh{Atlantic-Congo}
\def\langnames@fams@wals@toi{Atlantic-Congo}
\def\langnames@fams@wals@toj{Mayan}
\def\langnames@fams@wals@tok{Artificial Language}
\def\langnames@fams@wals@tol{Athabaskan-Eyak-Tlingit}
\def\langnames@fams@wals@tom{Austronesian}
\def\langnames@fams@wals@ton{Austronesian}
\def\langnames@fams@wals@too{Totonacan}
\def\langnames@fams@wals@top{Totonacan}
\def\langnames@fams@wals@toq{Nilotic}
\def\langnames@fams@wals@tor{Atlantic-Congo}
\def\langnames@fams@wals@tos{Totonacan}
\def\langnames@fams@wals@tou{Austroasiatic}
\def\langnames@fams@wals@tov{Indo-European}
\def\langnames@fams@wals@tow{Kiowa-Tanoan}
\def\langnames@fams@wals@tox{Austronesian}
\def\langnames@fams@wals@toy{Austronesian}
\def\langnames@fams@wals@toz{Atlantic-Congo}
\def\langnames@fams@wals@tpa{Austronesian}
\def\langnames@fams@wals@tpc{Otomanguean}
\def\langnames@fams@wals@tpe{Sino-Tibetan}
\def\langnames@fams@wals@tpf{Austronesian}
\def\langnames@fams@wals@tpg{Timor-Alor-Pantar}
\def\langnames@fams@wals@tpi{Indo-European}
\def\langnames@fams@wals@tpj{Tupian}
\def\langnames@fams@wals@tpl{Otomanguean}
\def\langnames@fams@wals@tpm{Atlantic-Congo}
\def\langnames@fams@wals@tpn{Tupian}
\def\langnames@fams@wals@tpo{Tai-Kadai}
\def\langnames@fams@wals@tpp{Totonacan}
\def\langnames@fams@wals@tpr{Tupian}
\def\langnames@fams@wals@tpt{Totonacan}
\def\langnames@fams@wals@tpu{Austroasiatic}
\def\langnames@fams@wals@tpv{Austronesian}
\def\langnames@fams@wals@tpw{Tupian}
\def\langnames@fams@wals@tpx{Otomanguean}
\def\langnames@fams@wals@tpy{Isolate}
\def\langnames@fams@wals@tpz{Austronesian}
\def\langnames@fams@wals@tqb{Tupian}
\def\langnames@fams@wals@tql{Austronesian}
\def\langnames@fams@wals@tqm{Doso-Turumsa}
\def\langnames@fams@wals@tqn{Sahaptian}
\def\langnames@fams@wals@tqo{Eleman}
\def\langnames@fams@wals@tqp{Austronesian}
\def\langnames@fams@wals@tqq{Afro-Asiatic}
\def\langnames@fams@wals@tqr{Narrow Talodi}
\def\langnames@fams@wals@tqt{Totonacan}
\def\langnames@fams@wals@tqu{Isolate}
\def\langnames@fams@wals@tqw{Isolate}
\def\langnames@fams@wals@tra{Indo-European}
\def\langnames@fams@wals@trb{Austronesian}
\def\langnames@fams@wals@trc{Otomanguean}
\def\langnames@fams@wals@trd{Austroasiatic}
\def\langnames@fams@wals@tre{Austronesian}
\def\langnames@fams@wals@trf{Indo-European}
\def\langnames@fams@wals@trg{Afro-Asiatic}
\def\langnames@fams@wals@trh{Dagan}
\def\langnames@fams@wals@tri{Cariban}
\def\langnames@fams@wals@trj{Afro-Asiatic}
\def\langnames@fams@wals@trl{Unclassifiable}
\def\langnames@fams@wals@trm{Indo-European}
\def\langnames@fams@wals@trn{Arawakan}
\def\langnames@fams@wals@tro{Sino-Tibetan}
\def\langnames@fams@wals@trp{Sino-Tibetan}
\def\langnames@fams@wals@trq{Otomanguean}
\def\langnames@fams@wals@trr{Isolate}
\def\langnames@fams@wals@trs{Otomanguean}
\def\langnames@fams@wals@trt{Geelvink Bay}
\def\langnames@fams@wals@tru{Afro-Asiatic}
\def\langnames@fams@wals@trv{Austronesian}
\def\langnames@fams@wals@trw{Indo-European}
\def\langnames@fams@wals@trx{Austronesian}
\def\langnames@fams@wals@try{Tai-Kadai}
\def\langnames@fams@wals@trz{Chapacuran}
\def\langnames@fams@wals@tsa{Atlantic-Congo}
\def\langnames@fams@wals@tsb{Afro-Asiatic}
\def\langnames@fams@wals@tsc{Atlantic-Congo}
\def\langnames@fams@wals@tsd{Indo-European}
\def\langnames@fams@wals@tse{Sign Language}
\def\langnames@fams@wals@tsg{Austronesian}
\def\langnames@fams@wals@tsh{Afro-Asiatic}
\def\langnames@fams@wals@tsi{Tsimshian}
\def\langnames@fams@wals@tsj{Sino-Tibetan}
\def\langnames@fams@wals@tsk{Sino-Tibetan}
\def\langnames@fams@wals@tsl{Tai-Kadai}
\def\langnames@fams@wals@tsm{Sign Language}
\def\langnames@fams@wals@tsn{Atlantic-Congo}
\def\langnames@fams@wals@tso{Atlantic-Congo}
\def\langnames@fams@wals@tsp{Atlantic-Congo}
\def\langnames@fams@wals@tsq{Sign Language}
\def\langnames@fams@wals@tsr{Austronesian}
\def\langnames@fams@wals@tss{Sign Language}
\def\langnames@fams@wals@tst{Songhay}
\def\langnames@fams@wals@tsu{Austronesian}
\def\langnames@fams@wals@tsv{Atlantic-Congo}
\def\langnames@fams@wals@tsw{Atlantic-Congo}
\def\langnames@fams@wals@tsx{Anim}
\def\langnames@fams@wals@tsy{Sign Language}
\def\langnames@fams@wals@tsz{Tarascan}
\def\langnames@fams@wals@tta{Siouan}
\def\langnames@fams@wals@ttb{Atlantic-Congo}
\def\langnames@fams@wals@ttc{Mayan}
\def\langnames@fams@wals@ttd{Goilalan}
\def\langnames@fams@wals@tte{Austronesian}
\def\langnames@fams@wals@ttf{Atlantic-Congo}
\def\langnames@fams@wals@ttg{Austronesian}
\def\langnames@fams@wals@tth{Austroasiatic}
\def\langnames@fams@wals@tti{Austronesian}
\def\langnames@fams@wals@ttj{Atlantic-Congo}
\def\langnames@fams@wals@ttk{Barbacoan}
\def\langnames@fams@wals@ttl{Atlantic-Congo}
\def\langnames@fams@wals@ttm{Athabaskan-Eyak-Tlingit}
\def\langnames@fams@wals@ttn{Pauwasi}
\def\langnames@fams@wals@tto{Austroasiatic}
\def\langnames@fams@wals@ttp{Austronesian}
\def\langnames@fams@wals@ttq{Afro-Asiatic}
\def\langnames@fams@wals@ttr{Afro-Asiatic}
\def\langnames@fams@wals@tts{Tai-Kadai}
\def\langnames@fams@wals@ttt{Indo-European}
\def\langnames@fams@wals@ttu{Austronesian}
\def\langnames@fams@wals@ttv{Austronesian}
\def\langnames@fams@wals@ttw{Austronesian}
\def\langnames@fams@wals@tty{Lakes Plain}
\def\langnames@fams@wals@ttz{Sino-Tibetan}
\def\langnames@fams@wals@tua{Nuclear Torricelli}
\def\langnames@fams@wals@tub{Uto-Aztecan}
\def\langnames@fams@wals@tuc{Austronesian}
\def\langnames@fams@wals@tud{Isolate}
\def\langnames@fams@wals@tue{Tucanoan}
\def\langnames@fams@wals@tuf{Chibchan}
\def\langnames@fams@wals@tug{Atlantic-Congo}
\def\langnames@fams@wals@tuh{Taulil-Butam}
\def\langnames@fams@wals@tui{Atlantic-Congo}
\def\langnames@fams@wals@tuj{North Halmahera}
\def\langnames@fams@wals@tuk{Turkic}
\def\langnames@fams@wals@tul{Atlantic-Congo}
\def\langnames@fams@wals@tum{Atlantic-Congo}
\def\langnames@fams@wals@tun{Isolate}
\def\langnames@fams@wals@tuo{Tucanoan}
\def\langnames@fams@wals@tuq{Saharan}
\def\langnames@fams@wals@tur{Turkic}
\def\langnames@fams@wals@tus{Iroquoian}
\def\langnames@fams@wals@tuu{Athabaskan-Eyak-Tlingit}
\def\langnames@fams@wals@tuv{Nilotic}
\def\langnames@fams@wals@tux{Pano-Tacanan}
\def\langnames@fams@wals@tuy{Nilotic}
\def\langnames@fams@wals@tuz{Atlantic-Congo}
\def\langnames@fams@wals@tva{Austronesian}
\def\langnames@fams@wals@tvd{Atlantic-Congo}
\def\langnames@fams@wals@tve{Austronesian}
\def\langnames@fams@wals@tvk{Austronesian}
\def\langnames@fams@wals@tvl{Austronesian}
\def\langnames@fams@wals@tvm{Austronesian}
\def\langnames@fams@wals@tvn{Sino-Tibetan}
\def\langnames@fams@wals@tvo{North Halmahera}
\def\langnames@fams@wals@tvs{Atlantic-Congo}
\def\langnames@fams@wals@tvt{Sino-Tibetan}
\def\langnames@fams@wals@tvu{Atlantic-Congo}
\def\langnames@fams@wals@tvw{Austronesian}
\def\langnames@fams@wals@tvy{Indo-European}
\def\langnames@fams@wals@twa{Salishan}
\def\langnames@fams@wals@twb{Austronesian}
\def\langnames@fams@wals@twc{Afro-Asiatic}
\def\langnames@fams@wals@twe{Timor-Alor-Pantar}
\def\langnames@fams@wals@twf{Kiowa-Tanoan}
\def\langnames@fams@wals@twg{Timor-Alor-Pantar}
\def\langnames@fams@wals@twh{Tai-Kadai}
\def\langnames@fams@wals@twl{Atlantic-Congo}
\def\langnames@fams@wals@twn{Atlantic-Congo}
\def\langnames@fams@wals@two{Atlantic-Congo}
\def\langnames@fams@wals@twp{Austronesian}
\def\langnames@fams@wals@twq{Songhay}
\def\langnames@fams@wals@twr{Uto-Aztecan}
\def\langnames@fams@wals@twt{Tupian}
\def\langnames@fams@wals@twu{Austronesian}
\def\langnames@fams@wals@tww{Walioic}
\def\langnames@fams@wals@twx{Atlantic-Congo}
\def\langnames@fams@wals@twy{Austronesian}
\def\langnames@fams@wals@txa{Austronesian}
\def\langnames@fams@wals@txb{Indo-European}
\def\langnames@fams@wals@txc{Athabaskan-Eyak-Tlingit}
\def\langnames@fams@wals@txe{Austronesian}
\def\langnames@fams@wals@txg{Sino-Tibetan}
\def\langnames@fams@wals@txh{Indo-European}
\def\langnames@fams@wals@txi{Cariban}
\def\langnames@fams@wals@txj{Saharan}
\def\langnames@fams@wals@txm{Austronesian}
\def\langnames@fams@wals@txn{Austronesian}
\def\langnames@fams@wals@txo{Sino-Tibetan}
\def\langnames@fams@wals@txq{Austronesian}
\def\langnames@fams@wals@txr{Unclassifiable}
\def\langnames@fams@wals@txs{Austronesian}
\def\langnames@fams@wals@txt{Nuclear Trans New Guinea}
\def\langnames@fams@wals@txu{Nuclear-Macro-Je}
\def\langnames@fams@wals@txx{Austronesian}
\def\langnames@fams@wals@txy{Austronesian}
\def\langnames@fams@wals@tya{Nuclear Trans New Guinea}
\def\langnames@fams@wals@tye{Mande}
\def\langnames@fams@wals@tyh{Austroasiatic}
\def\langnames@fams@wals@tyi{Atlantic-Congo}
\def\langnames@fams@wals@tyj{Tai-Kadai}
\def\langnames@fams@wals@tyn{Nuclear Trans New Guinea}
\def\langnames@fams@wals@typ{Pama-Nyungan}
\def\langnames@fams@wals@tyr{Tai-Kadai}
\def\langnames@fams@wals@tys{Tai-Kadai}
\def\langnames@fams@wals@tyt{Tai-Kadai}
\def\langnames@fams@wals@tyu{Khoe-Kwadi}
\def\langnames@fams@wals@tyv{Turkic}
\def\langnames@fams@wals@tyx{Atlantic-Congo}
\def\langnames@fams@wals@tyy{Atlantic-Congo}
\def\langnames@fams@wals@tyz{Tai-Kadai}
\def\langnames@fams@wals@tza{Sign Language}
\def\langnames@fams@wals@tzh{Mayan}
\def\langnames@fams@wals@tzj{Mayan}
\def\langnames@fams@wals@tzl{Artificial Language}
\def\langnames@fams@wals@tzm{Afro-Asiatic}
\def\langnames@fams@wals@tzn{Austronesian}
\def\langnames@fams@wals@tzo{Mayan}
\def\langnames@fams@wals@tzx{Lower Sepik-Ramu}
\def\langnames@fams@wals@uam{Unclassifiable}
\def\langnames@fams@wals@uan{Tai-Kadai}
\def\langnames@fams@wals@uar{Eleman}
\def\langnames@fams@wals@uba{Atlantic-Congo}
\def\langnames@fams@wals@ubi{Afro-Asiatic}
\def\langnames@fams@wals@ubr{Austronesian}
\def\langnames@fams@wals@ubu{Nuclear Trans New Guinea}
\def\langnames@fams@wals@uby{Abkhaz-Adyge}
\def\langnames@fams@wals@uda{Atlantic-Congo}
\def\langnames@fams@wals@ude{Tungusic}
\def\langnames@fams@wals@udg{Dravidian}
\def\langnames@fams@wals@udi{Nakh-Daghestanian}
\def\langnames@fams@wals@udj{Austronesian}
\def\langnames@fams@wals@udl{Afro-Asiatic}
\def\langnames@fams@wals@udm{Uralic}
\def\langnames@fams@wals@udu{Koman}
\def\langnames@fams@wals@ues{Austronesian}
\def\langnames@fams@wals@ufi{Nuclear Trans New Guinea}
\def\langnames@fams@wals@uga{Afro-Asiatic}
\def\langnames@fams@wals@ugb{Pama-Nyungan}
\def\langnames@fams@wals@uge{Austronesian}
\def\langnames@fams@wals@ugh{Nakh-Daghestanian}
\def\langnames@fams@wals@ugn{Sign Language}
\def\langnames@fams@wals@ugo{Sino-Tibetan}
\def\langnames@fams@wals@ugy{Sign Language}
\def\langnames@fams@wals@uha{Atlantic-Congo}
\def\langnames@fams@wals@uhn{Isolate}
\def\langnames@fams@wals@uig{Turkic}
\def\langnames@fams@wals@uis{South Bougainville}
\def\langnames@fams@wals@uiv{Atlantic-Congo}
\def\langnames@fams@wals@uji{Atlantic-Congo}
\def\langnames@fams@wals@uka{South Bird's Head Family}
\def\langnames@fams@wals@ukg{Nuclear Trans New Guinea}
\def\langnames@fams@wals@ukh{Atlantic-Congo}
\def\langnames@fams@wals@ukl{Sign Language}
\def\langnames@fams@wals@ukp{Atlantic-Congo}
\def\langnames@fams@wals@ukq{Atlantic-Congo}
\def\langnames@fams@wals@ukr{Indo-European}
\def\langnames@fams@wals@uks{Sign Language}
\def\langnames@fams@wals@uku{Atlantic-Congo}
\def\langnames@fams@wals@ukv{Nilotic}
\def\langnames@fams@wals@ukw{Atlantic-Congo}
\def\langnames@fams@wals@uky{Pama-Nyungan}
\def\langnames@fams@wals@ula{Atlantic-Congo}
\def\langnames@fams@wals@ulb{Atlantic-Congo}
\def\langnames@fams@wals@ulc{Tungusic}
\def\langnames@fams@wals@ule{Isolate}
\def\langnames@fams@wals@ulf{Isolate}
\def\langnames@fams@wals@uli{Austronesian}
\def\langnames@fams@wals@ulk{Eastern Trans-Fly}
\def\langnames@fams@wals@ull{Dravidian}
\def\langnames@fams@wals@ulm{Austronesian}
\def\langnames@fams@wals@uln{Indo-European}
\def\langnames@fams@wals@ulu{Austronesian}
\def\langnames@fams@wals@ulw{Misumalpan}
\def\langnames@fams@wals@uma{Sahaptian}
\def\langnames@fams@wals@umb{Atlantic-Congo}
\def\langnames@fams@wals@umd{Pama-Nyungan}
\def\langnames@fams@wals@umg{Pama-Nyungan}
\def\langnames@fams@wals@umi{Austronesian}
\def\langnames@fams@wals@umm{Atlantic-Congo}
\def\langnames@fams@wals@umn{Sino-Tibetan}
\def\langnames@fams@wals@umo{Bororoan}
\def\langnames@fams@wals@ump{Pama-Nyungan}
\def\langnames@fams@wals@umr{Isolate}
\def\langnames@fams@wals@ums{Austronesian}
\def\langnames@fams@wals@umu{Algic}
\def\langnames@fams@wals@una{Austronesian}
\def\langnames@fams@wals@une{Atlantic-Congo}
\def\langnames@fams@wals@ung{Worrorran}
\def\langnames@fams@wals@uni{Sko}
\def\langnames@fams@wals@unk{Arawakan}
\def\langnames@fams@wals@unm{Algic}
\def\langnames@fams@wals@unn{Pama-Nyungan}
\def\langnames@fams@wals@unr{Austroasiatic}
\def\langnames@fams@wals@unu{Austronesian}
\def\langnames@fams@wals@unz{Austronesian}
\def\langnames@fams@wals@uon{Austronesian}
\def\langnames@fams@wals@upi{Border}
\def\langnames@fams@wals@upv{Austronesian}
\def\langnames@fams@wals@ura{Isolate}
\def\langnames@fams@wals@urb{Tupian}
\def\langnames@fams@wals@urc{Giimbiyu}
\def\langnames@fams@wals@urd{Indo-European}
\def\langnames@fams@wals@ure{Uru-Chipaya}
\def\langnames@fams@wals@urg{Nuclear Trans New Guinea}
\def\langnames@fams@wals@urh{Atlantic-Congo}
\def\langnames@fams@wals@uri{Nuclear Torricelli}
\def\langnames@fams@wals@urk{Austronesian}
\def\langnames@fams@wals@url{Dravidian}
\def\langnames@fams@wals@urm{Nuclear Trans New Guinea}
\def\langnames@fams@wals@urn{Austronesian}
\def\langnames@fams@wals@uro{Baining}
\def\langnames@fams@wals@urp{Unclassifiable}
\def\langnames@fams@wals@urr{Austronesian}
\def\langnames@fams@wals@urt{Nuclear Torricelli}
\def\langnames@fams@wals@uru{Tupian}
\def\langnames@fams@wals@urv{Austronesian}
\def\langnames@fams@wals@urw{Nuclear Trans New Guinea}
\def\langnames@fams@wals@urx{Nuclear Torricelli}
\def\langnames@fams@wals@ury{Tor-Orya}
\def\langnames@fams@wals@urz{Tupian}
\def\langnames@fams@wals@usa{Nuclear Trans New Guinea}
\def\langnames@fams@wals@ush{Indo-European}
\def\langnames@fams@wals@usi{Sino-Tibetan}
\def\langnames@fams@wals@usk{Atlantic-Congo}
\def\langnames@fams@wals@usp{Mayan}
\def\langnames@fams@wals@usu{Nuclear Trans New Guinea}
\def\langnames@fams@wals@uta{Atlantic-Congo}
\def\langnames@fams@wals@ute{Uto-Aztecan}
\def\langnames@fams@wals@utp{Austronesian}
\def\langnames@fams@wals@utr{Atlantic-Congo}
\def\langnames@fams@wals@utu{Nuclear Trans New Guinea}
\def\langnames@fams@wals@uum{Turkic}
\def\langnames@fams@wals@uur{Austronesian}
\def\langnames@fams@wals@uuu{Austroasiatic}
\def\langnames@fams@wals@uve{Austronesian}
\def\langnames@fams@wals@uvh{Nuclear Trans New Guinea}
\def\langnames@fams@wals@uvl{Austronesian}
\def\langnames@fams@wals@uwa{Pama-Nyungan}
\def\langnames@fams@wals@uya{Atlantic-Congo}
\def\langnames@fams@wals@uzn{Turkic}
\def\langnames@fams@wals@uzs{Turkic}
\def\langnames@fams@wals@vaa{Indo-European}
\def\langnames@fams@wals@vae{Central Sudanic}
\def\langnames@fams@wals@vaf{Indo-European}
\def\langnames@fams@wals@vag{Atlantic-Congo}
\def\langnames@fams@wals@vah{Indo-European}
\def\langnames@fams@wals@vai{Mande}
\def\langnames@fams@wals@vaj{Kxa}
\def\langnames@fams@wals@val{Austronesian}
\def\langnames@fams@wals@vam{Sko}
\def\langnames@fams@wals@van{Nuclear Torricelli}
\def\langnames@fams@wals@vao{Austronesian}
\def\langnames@fams@wals@vap{Sino-Tibetan}
\def\langnames@fams@wals@var{Uto-Aztecan}
\def\langnames@fams@wals@vas{Indo-European}
\def\langnames@fams@wals@vau{Atlantic-Congo}
\def\langnames@fams@wals@vav{Indo-European}
\def\langnames@fams@wals@vay{Sino-Tibetan}
\def\langnames@fams@wals@vbb{Austronesian}
\def\langnames@fams@wals@vec{Indo-European}
\def\langnames@fams@wals@ved{Indo-European}
\def\langnames@fams@wals@vem{Afro-Asiatic}
\def\langnames@fams@wals@ven{Atlantic-Congo}
\def\langnames@fams@wals@veo{Chumashan}
\def\langnames@fams@wals@vep{Uralic}
\def\langnames@fams@wals@ver{Atlantic-Congo}
\def\langnames@fams@wals@vgr{Indo-European}
\def\langnames@fams@wals@vgt{Sign Language}
\def\langnames@fams@wals@vic{Indo-European}
\def\langnames@fams@wals@vid{Atlantic-Congo}
\def\langnames@fams@wals@vie{Austroasiatic}
\def\langnames@fams@wals@vif{Atlantic-Congo}
\def\langnames@fams@wals@vig{Atlantic-Congo}
\def\langnames@fams@wals@vil{Isolate}
\def\langnames@fams@wals@vin{Atlantic-Congo}
\def\langnames@fams@wals@vis{Dravidian}
\def\langnames@fams@wals@vit{Atlantic-Congo}
\def\langnames@fams@wals@viv{Austronesian}
\def\langnames@fams@wals@vka{Pama-Nyungan}
\def\langnames@fams@wals@vkj{Isolate}
\def\langnames@fams@wals@vkk{Austronesian}
\def\langnames@fams@wals@vkl{Austronesian}
\def\langnames@fams@wals@vkm{Kamakanan}
\def\langnames@fams@wals@vkn{Atlantic-Congo}
\def\langnames@fams@wals@vko{Austronesian}
\def\langnames@fams@wals@vkp{Indo-European}
\def\langnames@fams@wals@vkt{Austronesian}
\def\langnames@fams@wals@vku{Pama-Nyungan}
\def\langnames@fams@wals@vkz{Atlantic-Congo}
\def\langnames@fams@wals@vlp{Austronesian}
\def\langnames@fams@wals@vls{Indo-European}
\def\langnames@fams@wals@vma{Pama-Nyungan}
\def\langnames@fams@wals@vmb{Pama-Nyungan}
\def\langnames@fams@wals@vmc{Otomanguean}
\def\langnames@fams@wals@vmd{Dravidian}
\def\langnames@fams@wals@vme{Austronesian}
\def\langnames@fams@wals@vmf{Indo-European}
\def\langnames@fams@wals@vmg{Austronesian}
\def\langnames@fams@wals@vmh{Indo-European}
\def\langnames@fams@wals@vmi{Worrorran}
\def\langnames@fams@wals@vmj{Otomanguean}
\def\langnames@fams@wals@vmk{Atlantic-Congo}
\def\langnames@fams@wals@vml{Pama-Nyungan}
\def\langnames@fams@wals@vmm{Otomanguean}
\def\langnames@fams@wals@vmp{Otomanguean}
\def\langnames@fams@wals@vmq{Otomanguean}
\def\langnames@fams@wals@vmr{Atlantic-Congo}
\def\langnames@fams@wals@vms{Unattested}
\def\langnames@fams@wals@vmu{Pama-Nyungan}
\def\langnames@fams@wals@vmv{Maiduan}
\def\langnames@fams@wals@vmw{Atlantic-Congo}
\def\langnames@fams@wals@vmx{Otomanguean}
\def\langnames@fams@wals@vmy{Otomanguean}
\def\langnames@fams@wals@vmz{Otomanguean}
\def\langnames@fams@wals@vnk{Austronesian}
\def\langnames@fams@wals@vnm{Austronesian}
\def\langnames@fams@wals@vnp{Austronesian}
\def\langnames@fams@wals@vol{Artificial Language}
\def\langnames@fams@wals@vor{Atlantic-Congo}
\def\langnames@fams@wals@vot{Uralic}
\def\langnames@fams@wals@vra{Austronesian}
\def\langnames@fams@wals@vro{Uralic}
\def\langnames@fams@wals@vrs{Austronesian}
\def\langnames@fams@wals@vrt{Austronesian}
\def\langnames@fams@wals@vsi{Sign Language}
\def\langnames@fams@wals@vsl{Sign Language}
\def\langnames@fams@wals@vsv{Sign Language}
\def\langnames@fams@wals@vto{Tor-Orya}
\def\langnames@fams@wals@vum{Atlantic-Congo}
\def\langnames@fams@wals@vun{Atlantic-Congo}
\def\langnames@fams@wals@vut{Atlantic-Congo}
\def\langnames@fams@wals@vwa{Austroasiatic}
\def\langnames@fams@wals@waa{Sahaptian}
\def\langnames@fams@wals@wab{Austronesian}
\def\langnames@fams@wals@wac{Chinookan}
\def\langnames@fams@wals@wad{Austronesian}
\def\langnames@fams@wals@wae{Indo-European}
\def\langnames@fams@wals@waf{Unattested}
\def\langnames@fams@wals@wag{Austronesian}
\def\langnames@fams@wals@wah{Austronesian}
\def\langnames@fams@wals@wai{Unattested}
\def\langnames@fams@wals@waj{Nuclear Trans New Guinea}
\def\langnames@fams@wals@wal{Ta-Ne-Omotic}
\def\langnames@fams@wals@wam{Algic}
\def\langnames@fams@wals@wan{Mande}
\def\langnames@fams@wals@wao{Yuki-Wappo}
\def\langnames@fams@wals@wap{Arawakan}
\def\langnames@fams@wals@waq{Isolate}
\def\langnames@fams@wals@war{Austronesian}
\def\langnames@fams@wals@was{Isolate}
\def\langnames@fams@wals@wat{Austronesian}
\def\langnames@fams@wals@wau{Arawakan}
\def\langnames@fams@wals@wav{Atlantic-Congo}
\def\langnames@fams@wals@waw{Cariban}
\def\langnames@fams@wals@wax{Lower Sepik-Ramu}
\def\langnames@fams@wals@way{Cariban}
\def\langnames@fams@wals@waz{Austronesian}
\def\langnames@fams@wals@wba{Isolate}
\def\langnames@fams@wals@wbb{Austronesian}
\def\langnames@fams@wals@wbe{Lakes Plain}
\def\langnames@fams@wals@wbf{Atlantic-Congo}
\def\langnames@fams@wals@wbh{Atlantic-Congo}
\def\langnames@fams@wals@wbi{Atlantic-Congo}
\def\langnames@fams@wals@wbj{Afro-Asiatic}
\def\langnames@fams@wals@wbk{Indo-European}
\def\langnames@fams@wals@wbl{Indo-European}
\def\langnames@fams@wals@wbm{Austroasiatic}
\def\langnames@fams@wals@wbp{Pama-Nyungan}
\def\langnames@fams@wals@wbq{Dravidian}
\def\langnames@fams@wals@wbr{Indo-European}
\def\langnames@fams@wals@wbt{Pama-Nyungan}
\def\langnames@fams@wals@wbv{Pama-Nyungan}
\def\langnames@fams@wals@wbw{Austronesian}
\def\langnames@fams@wals@wca{Yanomamic}
\def\langnames@fams@wals@wci{Atlantic-Congo}
\def\langnames@fams@wals@wdd{Atlantic-Congo}
\def\langnames@fams@wals@wdg{Nuclear Trans New Guinea}
\def\langnames@fams@wals@wdj{Isolate}
\def\langnames@fams@wals@wdu{Pama-Nyungan}
\def\langnames@fams@wals@wea{Sino-Tibetan}
\def\langnames@fams@wals@wec{Kru}
\def\langnames@fams@wals@wed{Austronesian}
\def\langnames@fams@wals@weh{Atlantic-Congo}
\def\langnames@fams@wals@wei{Anim}
\def\langnames@fams@wals@wem{Atlantic-Congo}
\def\langnames@fams@wals@weo{Austronesian}
\def\langnames@fams@wals@wep{Indo-European}
\def\langnames@fams@wals@wer{Goilalan}
\def\langnames@fams@wals@wes{Indo-European}
\def\langnames@fams@wals@wet{Austronesian}
\def\langnames@fams@wals@wew{Austronesian}
\def\langnames@fams@wals@wfg{Pauwasi}
\def\langnames@fams@wals@wga{Pama-Nyungan}
\def\langnames@fams@wals@wgb{Austronesian}
\def\langnames@fams@wals@wgg{Pama-Nyungan}
\def\langnames@fams@wals@wgi{Nuclear Trans New Guinea}
\def\langnames@fams@wals@wgo{Austronesian}
\def\langnames@fams@wals@wgu{Pama-Nyungan}
\def\langnames@fams@wals@wgy{Pama-Nyungan}
\def\langnames@fams@wals@wha{Austronesian}
\def\langnames@fams@wals@whg{Nuclear Trans New Guinea}
\def\langnames@fams@wals@whk{Austronesian}
\def\langnames@fams@wals@wib{Atlantic-Congo}
\def\langnames@fams@wals@wic{Caddoan}
\def\langnames@fams@wals@wie{Pama-Nyungan}
\def\langnames@fams@wals@wif{Unattested}
\def\langnames@fams@wals@wig{Pama-Nyungan}
\def\langnames@fams@wals@wih{Pama-Nyungan}
\def\langnames@fams@wals@wii{Nuclear Torricelli}
\def\langnames@fams@wals@wij{Pama-Nyungan}
\def\langnames@fams@wals@wik{Pama-Nyungan}
\def\langnames@fams@wals@wil{Worrorran}
\def\langnames@fams@wals@wim{Pama-Nyungan}
\def\langnames@fams@wals@win{Siouan}
\def\langnames@fams@wals@wir{Tupian}
\def\langnames@fams@wals@wit{Wintuan}
\def\langnames@fams@wals@wiu{Isolate}
\def\langnames@fams@wals@wiv{Austronesian}
\def\langnames@fams@wals@wiy{Algic}
\def\langnames@fams@wals@wja{Atlantic-Congo}
\def\langnames@fams@wals@wji{Afro-Asiatic}
\def\langnames@fams@wals@wka{Afro-Asiatic}
\def\langnames@fams@wals@wkd{Austronesian}
\def\langnames@fams@wals@wkl{Dravidian}
\def\langnames@fams@wals@wku{Dravidian}
\def\langnames@fams@wals@wkw{Pama-Nyungan}
\def\langnames@fams@wals@wla{Walioic}
\def\langnames@fams@wals@wlc{Atlantic-Congo}
\def\langnames@fams@wals@wle{Afro-Asiatic}
\def\langnames@fams@wals@wlg{Gunwinyguan}
\def\langnames@fams@wals@wlh{Austronesian}
\def\langnames@fams@wals@wli{North Halmahera}
\def\langnames@fams@wals@wlk{Athabaskan-Eyak-Tlingit}
\def\langnames@fams@wals@wll{Nubian}
\def\langnames@fams@wals@wln{Indo-European}
\def\langnames@fams@wals@wlo{Austronesian}
\def\langnames@fams@wals@wlr{Austronesian}
\def\langnames@fams@wals@wls{Austronesian}
\def\langnames@fams@wals@wlu{Pama-Nyungan}
\def\langnames@fams@wals@wlv{Matacoan}
\def\langnames@fams@wals@wlw{Nuclear Trans New Guinea}
\def\langnames@fams@wals@wlx{Atlantic-Congo}
\def\langnames@fams@wals@wly{Sino-Tibetan}
\def\langnames@fams@wals@wma{Unattested}
\def\langnames@fams@wals@wmb{Mirndi}
\def\langnames@fams@wals@wmc{Nuclear Trans New Guinea}
\def\langnames@fams@wals@wmd{Nambiquaran}
\def\langnames@fams@wals@wme{Sino-Tibetan}
\def\langnames@fams@wals@wmg{Sino-Tibetan}
\def\langnames@fams@wals@wmh{Austronesian}
\def\langnames@fams@wals@wmi{Pama-Nyungan}
\def\langnames@fams@wals@wmm{Austronesian}
\def\langnames@fams@wals@wmn{Austronesian}
\def\langnames@fams@wals@wmo{Nuclear Torricelli}
\def\langnames@fams@wals@wms{Nuclear Trans New Guinea}
\def\langnames@fams@wals@wmt{Pama-Nyungan}
\def\langnames@fams@wals@wmw{Atlantic-Congo}
\def\langnames@fams@wals@wmx{Sko}
\def\langnames@fams@wals@wnb{Nuclear Trans New Guinea}
\def\langnames@fams@wals@wnc{Nuclear Trans New Guinea}
\def\langnames@fams@wals@wnd{Mangarrayi-Maran}
\def\langnames@fams@wals@wne{Indo-European}
\def\langnames@fams@wals@wng{Nuclear Trans New Guinea}
\def\langnames@fams@wals@wni{Atlantic-Congo}
\def\langnames@fams@wals@wnk{Austronesian}
\def\langnames@fams@wals@wnm{Pama-Nyungan}
\def\langnames@fams@wals@wno{Nuclear Trans New Guinea}
\def\langnames@fams@wals@wnp{Nuclear Torricelli}
\def\langnames@fams@wals@wnu{Nuclear Trans New Guinea}
\def\langnames@fams@wals@wny{Garrwan}
\def\langnames@fams@wals@woa{Northern Daly}
\def\langnames@fams@wals@wob{Kru}
\def\langnames@fams@wals@woc{Austronesian}
\def\langnames@fams@wals@wod{Nuclear Trans New Guinea}
\def\langnames@fams@wals@woe{Austronesian}
\def\langnames@fams@wals@wof{Atlantic-Congo}
\def\langnames@fams@wals@wog{Sepik}
\def\langnames@fams@wals@woi{Timor-Alor-Pantar}
\def\langnames@fams@wals@wok{Atlantic-Congo}
\def\langnames@fams@wals@wol{Atlantic-Congo}
\def\langnames@fams@wals@wom{Atlantic-Congo}
\def\langnames@fams@wals@won{Atlantic-Congo}
\def\langnames@fams@wals@woo{Austronesian}
\def\langnames@fams@wals@wor{Geelvink Bay}
\def\langnames@fams@wals@wos{Ndu}
\def\langnames@fams@wals@wow{Austronesian}
\def\langnames@fams@wals@woy{Unattested}
\def\langnames@fams@wals@wpc{Saliban}
\def\langnames@fams@wals@wrb{Pama-Nyungan}
\def\langnames@fams@wals@wre{Unattested}
\def\langnames@fams@wals@wrg{Pama-Nyungan}
\def\langnames@fams@wals@wrh{Pama-Nyungan}
\def\langnames@fams@wals@wri{Pama-Nyungan}
\def\langnames@fams@wals@wrk{Garrwan}
\def\langnames@fams@wals@wrl{Pama-Nyungan}
\def\langnames@fams@wals@wrm{Pama-Nyungan}
\def\langnames@fams@wals@wrn{Heibanic}
\def\langnames@fams@wals@wro{Worrorran}
\def\langnames@fams@wals@wrp{Austronesian}
\def\langnames@fams@wals@wrr{Yangmanic}
\def\langnames@fams@wals@wrs{Border}
\def\langnames@fams@wals@wru{Austronesian}
\def\langnames@fams@wals@wrv{Suki-Gogodala}
\def\langnames@fams@wals@wrw{Pama-Nyungan}
\def\langnames@fams@wals@wrx{Austronesian}
\def\langnames@fams@wals@wry{Indo-European}
\def\langnames@fams@wals@wrz{Gunwinyguan}
\def\langnames@fams@wals@wsa{Austronesian}
\def\langnames@fams@wals@wsg{Dravidian}
\def\langnames@fams@wals@wsi{Austronesian}
\def\langnames@fams@wals@wsk{Nuclear Trans New Guinea}
\def\langnames@fams@wals@wsr{Nuclear Trans New Guinea}
\def\langnames@fams@wals@wss{Atlantic-Congo}
\def\langnames@fams@wals@wsu{Unattested}
\def\langnames@fams@wals@wsv{Indo-European}
\def\langnames@fams@wals@wtf{Nuclear Trans New Guinea}
\def\langnames@fams@wals@wth{Pama-Nyungan}
\def\langnames@fams@wals@wti{Isolate}
\def\langnames@fams@wals@wtk{Sepik}
\def\langnames@fams@wals@wtm{Indo-European}
\def\langnames@fams@wals@wtw{Austronesian}
\def\langnames@fams@wals@wua{Pama-Nyungan}
\def\langnames@fams@wals@wub{Worrorran}
\def\langnames@fams@wals@wud{Atlantic-Congo}
\def\langnames@fams@wals@wuh{Sino-Tibetan}
\def\langnames@fams@wals@wul{Nuclear Trans New Guinea}
\def\langnames@fams@wals@wum{Atlantic-Congo}
\def\langnames@fams@wals@wun{Atlantic-Congo}
\def\langnames@fams@wals@wur{Marrku-Wurrugu}
\def\langnames@fams@wals@wut{Sko}
\def\langnames@fams@wals@wuu{Sino-Tibetan}
\def\langnames@fams@wals@wuv{Austronesian}
\def\langnames@fams@wals@wux{Limilngan-Wulna}
\def\langnames@fams@wals@wuy{Austronesian}
\def\langnames@fams@wals@wwa{Atlantic-Congo}
\def\langnames@fams@wals@wwb{Unclassifiable}
\def\langnames@fams@wals@wwo{Austronesian}
\def\langnames@fams@wals@wwr{Nyulnyulan}
\def\langnames@fams@wals@www{Atlantic-Congo}
\def\langnames@fams@wals@wxa{Sino-Tibetan}
\def\langnames@fams@wals@wya{Iroquoian}
\def\langnames@fams@wals@wyb{Pama-Nyungan}
\def\langnames@fams@wals@wyi{Pama-Nyungan}
\def\langnames@fams@wals@wym{Indo-European}
\def\langnames@fams@wals@wyr{Tupian}
\def\langnames@fams@wals@wyy{Austronesian}
\def\langnames@fams@wals@xaa{Afro-Asiatic}
\def\langnames@fams@wals@xab{Atlantic-Congo}
\def\langnames@fams@wals@xac{Sino-Tibetan}
\def\langnames@fams@wals@xad{Isolate}
\def\langnames@fams@wals@xag{Nakh-Daghestanian}
\def\langnames@fams@wals@xai{Unclassifiable}
\def\langnames@fams@wals@xak{Isolate}
\def\langnames@fams@wals@xal{Mongolic-Khitan}
\def\langnames@fams@wals@xam{Tuu}
\def\langnames@fams@wals@xan{Afro-Asiatic}
\def\langnames@fams@wals@xap{Muskogean}
\def\langnames@fams@wals@xar{Isolate}
\def\langnames@fams@wals@xas{Uralic}
\def\langnames@fams@wals@xat{Katukinan}
\def\langnames@fams@wals@xau{Greater Kwerba}
\def\langnames@fams@wals@xav{Nuclear-Macro-Je}
\def\langnames@fams@wals@xaw{Uto-Aztecan}
\def\langnames@fams@wals@xay{Austronesian}
\def\langnames@fams@wals@xbc{Indo-European}
\def\langnames@fams@wals@xbe{Pama-Nyungan}
\def\langnames@fams@wals@xbg{Pama-Nyungan}
\def\langnames@fams@wals@xbi{Nuclear Torricelli}
\def\langnames@fams@wals@xbn{Isolate}
\def\langnames@fams@wals@xbo{Turkic}
\def\langnames@fams@wals@xbr{Austronesian}
\def\langnames@fams@wals@xbw{Unclassifiable}
\def\langnames@fams@wals@xcc{Unclassifiable}
\def\langnames@fams@wals@xce{Indo-European}
\def\langnames@fams@wals@xcg{Indo-European}
\def\langnames@fams@wals@xch{Chimakuan}
\def\langnames@fams@wals@xcl{Indo-European}
\def\langnames@fams@wals@xcm{Isolate}
\def\langnames@fams@wals@xcn{Isolate}
\def\langnames@fams@wals@xco{Indo-European}
\def\langnames@fams@wals@xcr{Indo-European}
\def\langnames@fams@wals@xct{Sino-Tibetan}
\def\langnames@fams@wals@xcv{Yukaghir}
\def\langnames@fams@wals@xcw{Isolate}
\def\langnames@fams@wals@xcy{Isolate}
\def\langnames@fams@wals@xda{Pama-Nyungan}
\def\langnames@fams@wals@xdc{Indo-European}
\def\langnames@fams@wals@xdk{Pama-Nyungan}
\def\langnames@fams@wals@xdo{Atlantic-Congo}
\def\langnames@fams@wals@xdq{Nakh-Daghestanian}
\def\langnames@fams@wals@xdy{Austronesian}
\def\langnames@fams@wals@xeb{Afro-Asiatic}
\def\langnames@fams@wals@xed{Afro-Asiatic}
\def\langnames@fams@wals@xeg{Tuu}
\def\langnames@fams@wals@xel{Eastern Jebel}
\def\langnames@fams@wals@xem{Austronesian}
\def\langnames@fams@wals@xer{Nuclear-Macro-Je}
\def\langnames@fams@wals@xes{Nuclear Trans New Guinea}
\def\langnames@fams@wals@xet{Tupian}
\def\langnames@fams@wals@xeu{Eleman}
\def\langnames@fams@wals@xfa{Indo-European}
\def\langnames@fams@wals@xga{Indo-European}
\def\langnames@fams@wals@xgb{Mande}
\def\langnames@fams@wals@xgd{Pama-Nyungan}
\def\langnames@fams@wals@xgf{Uto-Aztecan}
\def\langnames@fams@wals@xgm{Pama-Nyungan}
\def\langnames@fams@wals@xgu{Worrorran}
\def\langnames@fams@wals@xgw{Pama-Nyungan}
\def\langnames@fams@wals@xhd{Afro-Asiatic}
\def\langnames@fams@wals@xhe{Indo-European}
\def\langnames@fams@wals@xho{Atlantic-Congo}
\def\langnames@fams@wals@xht{Isolate}
\def\langnames@fams@wals@xhu{Hurro-Urartian}
\def\langnames@fams@wals@xib{Isolate}
\def\langnames@fams@wals@xii{Khoe-Kwadi}
\def\langnames@fams@wals@xil{Unclassifiable}
\def\langnames@fams@wals@xip{Unattested}
\def\langnames@fams@wals@xir{Arawakan}
\def\langnames@fams@wals@xiv{Unattested}
\def\langnames@fams@wals@xiy{Tupian}
\def\langnames@fams@wals@xjb{Pama-Nyungan}
\def\langnames@fams@wals@xka{Indo-European}
\def\langnames@fams@wals@xkb{Atlantic-Congo}
\def\langnames@fams@wals@xkc{Indo-European}
\def\langnames@fams@wals@xkd{Austronesian}
\def\langnames@fams@wals@xke{Austronesian}
\def\langnames@fams@wals@xkf{Sino-Tibetan}
\def\langnames@fams@wals@xkg{Mande}
\def\langnames@fams@wals@xkh{Unattested}
\def\langnames@fams@wals@xki{Sign Language}
\def\langnames@fams@wals@xkj{Indo-European}
\def\langnames@fams@wals@xkk{Austroasiatic}
\def\langnames@fams@wals@xkl{Austronesian}
\def\langnames@fams@wals@xkn{Austronesian}
\def\langnames@fams@wals@xkp{Indo-European}
\def\langnames@fams@wals@xkq{Austronesian}
\def\langnames@fams@wals@xkr{Nuclear-Macro-Je}
\def\langnames@fams@wals@xks{Austronesian}
\def\langnames@fams@wals@xkt{Atlantic-Congo}
\def\langnames@fams@wals@xku{Atlantic-Congo}
\def\langnames@fams@wals@xkv{Atlantic-Congo}
\def\langnames@fams@wals@xkw{Lepki-Murkim-Kembra}
\def\langnames@fams@wals@xkx{Austronesian}
\def\langnames@fams@wals@xky{Austronesian}
\def\langnames@fams@wals@xkz{Sino-Tibetan}
\def\langnames@fams@wals@xla{Kamula-Elevala}
\def\langnames@fams@wals@xlc{Indo-European}
\def\langnames@fams@wals@xld{Indo-European}
\def\langnames@fams@wals@xle{Unclassifiable}
\def\langnames@fams@wals@xlg{Unclassifiable}
\def\langnames@fams@wals@xlo{Algic}
\def\langnames@fams@wals@xlp{Indo-European}
\def\langnames@fams@wals@xls{Indo-European}
\def\langnames@fams@wals@xlu{Indo-European}
\def\langnames@fams@wals@xly{Unclassifiable}
\def\langnames@fams@wals@xmb{Atlantic-Congo}
\def\langnames@fams@wals@xmc{Atlantic-Congo}
\def\langnames@fams@wals@xmd{Afro-Asiatic}
\def\langnames@fams@wals@xmf{Kartvelian}
\def\langnames@fams@wals@xmg{Atlantic-Congo}
\def\langnames@fams@wals@xmh{Pama-Nyungan}
\def\langnames@fams@wals@xmi{Unattested}
\def\langnames@fams@wals@xmj{Afro-Asiatic}
\def\langnames@fams@wals@xml{Sign Language}
\def\langnames@fams@wals@xmm{Austronesian}
\def\langnames@fams@wals@xmo{Unattested}
\def\langnames@fams@wals@xmp{Pama-Nyungan}
\def\langnames@fams@wals@xmr{Isolate}
\def\langnames@fams@wals@xms{Sign Language}
\def\langnames@fams@wals@xmt{Austronesian}
\def\langnames@fams@wals@xmu{Eastern Daly}
\def\langnames@fams@wals@xmv{Austronesian}
\def\langnames@fams@wals@xmw{Austronesian}
\def\langnames@fams@wals@xmx{Austronesian}
\def\langnames@fams@wals@xmy{Pama-Nyungan}
\def\langnames@fams@wals@xmz{Austronesian}
\def\langnames@fams@wals@xna{Afro-Asiatic}
\def\langnames@fams@wals@xnb{Austronesian}
\def\langnames@fams@wals@xng{Mongolic-Khitan}
\def\langnames@fams@wals@xnj{Atlantic-Congo}
\def\langnames@fams@wals@xnm{Nyulnyulan}
\def\langnames@fams@wals@xnn{Austronesian}
\def\langnames@fams@wals@xnq{Atlantic-Congo}
\def\langnames@fams@wals@xnr{Indo-European}
\def\langnames@fams@wals@xns{Sino-Tibetan}
\def\langnames@fams@wals@xny{Pama-Nyungan}
\def\langnames@fams@wals@xod{South Bird's Head Family}
\def\langnames@fams@wals@xog{Atlantic-Congo}
\def\langnames@fams@wals@xoi{Lower Sepik-Ramu}
\def\langnames@fams@wals@xok{Nuclear-Macro-Je}
\def\langnames@fams@wals@xom{Koman}
\def\langnames@fams@wals@xon{Atlantic-Congo}
\def\langnames@fams@wals@xoo{Isolate}
\def\langnames@fams@wals@xop{Lower Sepik-Ramu}
\def\langnames@fams@wals@xor{Pano-Tacanan}
\def\langnames@fams@wals@xow{Nuclear Trans New Guinea}
\def\langnames@fams@wals@xpa{Pama-Nyungan}
\def\langnames@fams@wals@xpc{Turkic}
\def\langnames@fams@wals@xpe{Mande}
\def\langnames@fams@wals@xpg{Indo-European}
\def\langnames@fams@wals@xpi{Unclassifiable}
\def\langnames@fams@wals@xpk{Pano-Tacanan}
\def\langnames@fams@wals@xpm{Yeniseian}
\def\langnames@fams@wals@xpn{Unclassifiable}
\def\langnames@fams@wals@xpo{Uto-Aztecan}
\def\langnames@fams@wals@xpr{Indo-European}
\def\langnames@fams@wals@xps{Indo-European}
\def\langnames@fams@wals@xpu{Afro-Asiatic}
\def\langnames@fams@wals@xqt{Afro-Asiatic}
\def\langnames@fams@wals@xrb{Atlantic-Congo}
\def\langnames@fams@wals@xre{Nuclear-Macro-Je}
\def\langnames@fams@wals@xrn{Yeniseian}
\def\langnames@fams@wals@xrr{Unclassifiable}
\def\langnames@fams@wals@xrt{Unclassifiable}
\def\langnames@fams@wals@xru{Western Daly}
\def\langnames@fams@wals@xrw{Sepik}
\def\langnames@fams@wals@xsa{Afro-Asiatic}
\def\langnames@fams@wals@xsb{Austronesian}
\def\langnames@fams@wals@xsd{Indo-European}
\def\langnames@fams@wals@xse{Nuclear Trans New Guinea}
\def\langnames@fams@wals@xsh{Atlantic-Congo}
\def\langnames@fams@wals@xsi{Austronesian}
\def\langnames@fams@wals@xsl{Athabaskan-Eyak-Tlingit}
\def\langnames@fams@wals@xsm{Atlantic-Congo}
\def\langnames@fams@wals@xsn{Atlantic-Congo}
\def\langnames@fams@wals@xso{Unclassifiable}
\def\langnames@fams@wals@xsp{Nuclear Trans New Guinea}
\def\langnames@fams@wals@xsq{Atlantic-Congo}
\def\langnames@fams@wals@xsr{Sino-Tibetan}
\def\langnames@fams@wals@xsu{Yanomamic}
\def\langnames@fams@wals@xsy{Austronesian}
\def\langnames@fams@wals@xta{Otomanguean}
\def\langnames@fams@wals@xtb{Otomanguean}
\def\langnames@fams@wals@xtc{Kadugli-Krongo}
\def\langnames@fams@wals@xtd{Otomanguean}
\def\langnames@fams@wals@xte{Nuclear Trans New Guinea}
\def\langnames@fams@wals@xtg{Indo-European}
\def\langnames@fams@wals@xti{Otomanguean}
\def\langnames@fams@wals@xtj{Otomanguean}
\def\langnames@fams@wals@xtl{Otomanguean}
\def\langnames@fams@wals@xtm{Otomanguean}
\def\langnames@fams@wals@xtn{Otomanguean}
\def\langnames@fams@wals@xto{Indo-European}
\def\langnames@fams@wals@xtp{Otomanguean}
\def\langnames@fams@wals@xtq{Indo-European}
\def\langnames@fams@wals@xts{Otomanguean}
\def\langnames@fams@wals@xtt{Otomanguean}
\def\langnames@fams@wals@xtu{Otomanguean}
\def\langnames@fams@wals@xtv{Pama-Nyungan}
\def\langnames@fams@wals@xtw{Nambiquaran}
\def\langnames@fams@wals@xty{Otomanguean}
\def\langnames@fams@wals@xua{Dravidian}
\def\langnames@fams@wals@xub{Dravidian}
\def\langnames@fams@wals@xug{Japonic}
\def\langnames@fams@wals@xuj{Dravidian}
\def\langnames@fams@wals@xum{Indo-European}
\def\langnames@fams@wals@xuo{Atlantic-Congo}
\def\langnames@fams@wals@xup{Athabaskan-Eyak-Tlingit}
\def\langnames@fams@wals@xur{Hurro-Urartian}
\def\langnames@fams@wals@xut{Pama-Nyungan}
\def\langnames@fams@wals@xuu{Khoe-Kwadi}
\def\langnames@fams@wals@xve{Indo-European}
\def\langnames@fams@wals@xwa{Isolate}
\def\langnames@fams@wals@xwc{Siouan}
\def\langnames@fams@wals@xwe{Atlantic-Congo}
\def\langnames@fams@wals@xwg{Surmic}
\def\langnames@fams@wals@xwl{Atlantic-Congo}
\def\langnames@fams@wals@xwr{Greater Kwerba}
\def\langnames@fams@wals@xxb{Atlantic-Congo}
\def\langnames@fams@wals@xxk{Austronesian}
\def\langnames@fams@wals@xxm{Isolate}
\def\langnames@fams@wals@xxr{Nuclear-Macro-Je}
\def\langnames@fams@wals@xxt{Isolate}
\def\langnames@fams@wals@xya{Pama-Nyungan}
\def\langnames@fams@wals@xyb{Pama-Nyungan}
\def\langnames@fams@wals@xyl{Unattested}
\def\langnames@fams@wals@xyy{Pama-Nyungan}
\def\langnames@fams@wals@xzh{Sino-Tibetan}
\def\langnames@fams@wals@yaa{Pano-Tacanan}
\def\langnames@fams@wals@yab{Naduhup}
\def\langnames@fams@wals@yac{Nuclear Trans New Guinea}
\def\langnames@fams@wals@yad{Peba-Yagua}
\def\langnames@fams@wals@yae{Isolate}
\def\langnames@fams@wals@yaf{Atlantic-Congo}
\def\langnames@fams@wals@yag{Isolate}
\def\langnames@fams@wals@yah{Indo-European}
\def\langnames@fams@wals@yai{Indo-European}
\def\langnames@fams@wals@yaj{Atlantic-Congo}
\def\langnames@fams@wals@yak{Sahaptian}
\def\langnames@fams@wals@yal{Mande}
\def\langnames@fams@wals@yam{Atlantic-Congo}
\def\langnames@fams@wals@yan{Misumalpan}
\def\langnames@fams@wals@yao{Atlantic-Congo}
\def\langnames@fams@wals@yap{Austronesian}
\def\langnames@fams@wals@yaq{Uto-Aztecan}
\def\langnames@fams@wals@yar{Cariban}
\def\langnames@fams@wals@yas{Atlantic-Congo}
\def\langnames@fams@wals@yat{Atlantic-Congo}
\def\langnames@fams@wals@yau{Isolate}
\def\langnames@fams@wals@yav{Atlantic-Congo}
\def\langnames@fams@wals@yaw{Arawakan}
\def\langnames@fams@wals@yay{Atlantic-Congo}
\def\langnames@fams@wals@yaz{Atlantic-Congo}
\def\langnames@fams@wals@yba{Atlantic-Congo}
\def\langnames@fams@wals@ybb{Atlantic-Congo}
\def\langnames@fams@wals@ybe{Turkic}
\def\langnames@fams@wals@ybh{Sino-Tibetan}
\def\langnames@fams@wals@ybi{Sino-Tibetan}
\def\langnames@fams@wals@ybj{Atlantic-Congo}
\def\langnames@fams@wals@ybk{Sino-Tibetan}
\def\langnames@fams@wals@ybl{Atlantic-Congo}
\def\langnames@fams@wals@ybm{Nuclear Trans New Guinea}
\def\langnames@fams@wals@ybn{Arawakan}
\def\langnames@fams@wals@ybo{Nuclear Trans New Guinea}
\def\langnames@fams@wals@ybx{Walioic}
\def\langnames@fams@wals@yby{Nuclear Trans New Guinea}
\def\langnames@fams@wals@ych{Sino-Tibetan}
\def\langnames@fams@wals@ycl{Sino-Tibetan}
\def\langnames@fams@wals@ycn{Arawakan}
\def\langnames@fams@wals@ycp{Sino-Tibetan}
\def\langnames@fams@wals@yda{Pama-Nyungan}
\def\langnames@fams@wals@ydd{Indo-European}
\def\langnames@fams@wals@yde{Nuclear Torricelli}
\def\langnames@fams@wals@ydg{Indo-European}
\def\langnames@fams@wals@ydk{Nuclear Trans New Guinea}
\def\langnames@fams@wals@yea{Dravidian}
\def\langnames@fams@wals@yec{Mixed Language}
\def\langnames@fams@wals@yee{Lower Sepik-Ramu}
\def\langnames@fams@wals@yei{Atlantic-Congo}
\def\langnames@fams@wals@yej{Indo-European}
\def\langnames@fams@wals@yel{Atlantic-Congo}
\def\langnames@fams@wals@yer{Atlantic-Congo}
\def\langnames@fams@wals@yes{Atlantic-Congo}
\def\langnames@fams@wals@yet{Isolate}
\def\langnames@fams@wals@yeu{Dravidian}
\def\langnames@fams@wals@yev{Nuclear Torricelli}
\def\langnames@fams@wals@yey{Atlantic-Congo}
\def\langnames@fams@wals@ygl{Nuclear Torricelli}
\def\langnames@fams@wals@ygm{Nuclear Trans New Guinea}
\def\langnames@fams@wals@ygp{Sino-Tibetan}
\def\langnames@fams@wals@ygr{Nuclear Trans New Guinea}
\def\langnames@fams@wals@ygs{Sign Language}
\def\langnames@fams@wals@ygu{Unattested}
\def\langnames@fams@wals@ygw{Angan}
\def\langnames@fams@wals@yha{Tai-Kadai}
\def\langnames@fams@wals@yhd{Afro-Asiatic}
\def\langnames@fams@wals@yhl{Sino-Tibetan}
\def\langnames@fams@wals@yia{Pama-Nyungan}
\def\langnames@fams@wals@yif{Sino-Tibetan}
\def\langnames@fams@wals@yig{Sino-Tibetan}
\def\langnames@fams@wals@yih{Indo-European}
\def\langnames@fams@wals@yii{Pama-Nyungan}
\def\langnames@fams@wals@yij{Pama-Nyungan}
\def\langnames@fams@wals@yik{Sino-Tibetan}
\def\langnames@fams@wals@yil{Pama-Nyungan}
\def\langnames@fams@wals@yim{Sino-Tibetan}
\def\langnames@fams@wals@yin{Austroasiatic}
\def\langnames@fams@wals@yip{Sino-Tibetan}
\def\langnames@fams@wals@yiq{Sino-Tibetan}
\def\langnames@fams@wals@yir{Nuclear Trans New Guinea}
\def\langnames@fams@wals@yis{Nuclear Torricelli}
\def\langnames@fams@wals@yit{Sino-Tibetan}
\def\langnames@fams@wals@yiu{Sino-Tibetan}
\def\langnames@fams@wals@yiv{Sino-Tibetan}
\def\langnames@fams@wals@yix{Sino-Tibetan}
\def\langnames@fams@wals@yiy{Pama-Nyungan}
\def\langnames@fams@wals@yiz{Sino-Tibetan}
\def\langnames@fams@wals@yka{Austronesian}
\def\langnames@fams@wals@ykg{Yukaghir}
\def\langnames@fams@wals@yki{Austronesian}
\def\langnames@fams@wals@ykk{Austronesian}
\def\langnames@fams@wals@ykl{Sino-Tibetan}
\def\langnames@fams@wals@ykm{Austronesian}
\def\langnames@fams@wals@ykn{Sino-Tibetan}
\def\langnames@fams@wals@yko{Atlantic-Congo}
\def\langnames@fams@wals@ykr{Nuclear Trans New Guinea}
\def\langnames@fams@wals@ykt{Sino-Tibetan}
\def\langnames@fams@wals@yku{Sino-Tibetan}
\def\langnames@fams@wals@yky{Atlantic-Congo}
\def\langnames@fams@wals@yla{Keram}
\def\langnames@fams@wals@yle{Isolate}
\def\langnames@fams@wals@ylg{Ndu}
\def\langnames@fams@wals@yli{Nuclear Trans New Guinea}
\def\langnames@fams@wals@yll{Nuclear Torricelli}
\def\langnames@fams@wals@ylm{Sino-Tibetan}
\def\langnames@fams@wals@yln{Tai-Kadai}
\def\langnames@fams@wals@ylo{Sino-Tibetan}
\def\langnames@fams@wals@ylr{Pama-Nyungan}
\def\langnames@fams@wals@ylu{Austronesian}
\def\langnames@fams@wals@yly{Austronesian}
\def\langnames@fams@wals@ymb{Nuclear Torricelli}
\def\langnames@fams@wals@ymc{Sino-Tibetan}
\def\langnames@fams@wals@ymd{Sino-Tibetan}
\def\langnames@fams@wals@yme{Peba-Yagua}
\def\langnames@fams@wals@ymh{Sino-Tibetan}
\def\langnames@fams@wals@ymi{Sino-Tibetan}
\def\langnames@fams@wals@ymk{Atlantic-Congo}
\def\langnames@fams@wals@yml{Austronesian}
\def\langnames@fams@wals@ymm{Afro-Asiatic}
\def\langnames@fams@wals@ymn{Austronesian}
\def\langnames@fams@wals@ymo{Nuclear Torricelli}
\def\langnames@fams@wals@ymp{Austronesian}
\def\langnames@fams@wals@ymq{Sino-Tibetan}
\def\langnames@fams@wals@ymr{Dravidian}
\def\langnames@fams@wals@ymx{Sino-Tibetan}
\def\langnames@fams@wals@ymz{Sino-Tibetan}
\def\langnames@fams@wals@yna{Sino-Tibetan}
\def\langnames@fams@wals@ynd{Pama-Nyungan}
\def\langnames@fams@wals@yng{Atlantic-Congo}
\def\langnames@fams@wals@ynk{Eskimo-Aleut}
\def\langnames@fams@wals@ynl{Nuclear Trans New Guinea}
\def\langnames@fams@wals@ynn{Isolate}
\def\langnames@fams@wals@yno{Tai-Kadai}
\def\langnames@fams@wals@ynq{Atlantic-Congo}
\def\langnames@fams@wals@yns{Atlantic-Congo}
\def\langnames@fams@wals@ynu{Tucanoan}
\def\langnames@fams@wals@yob{Austronesian}
\def\langnames@fams@wals@yog{Austronesian}
\def\langnames@fams@wals@yoi{Japonic}
\def\langnames@fams@wals@yok{Yokutsan}
\def\langnames@fams@wals@yol{Indo-European}
\def\langnames@fams@wals@yom{Atlantic-Congo}
\def\langnames@fams@wals@yon{Nuclear Trans New Guinea}
\def\langnames@fams@wals@yor{Atlantic-Congo}
\def\langnames@fams@wals@yot{Atlantic-Congo}
\def\langnames@fams@wals@yox{Japonic}
\def\langnames@fams@wals@yoy{Tai-Kadai}
\def\langnames@fams@wals@ypa{Sino-Tibetan}
\def\langnames@fams@wals@ypb{Sino-Tibetan}
\def\langnames@fams@wals@ypg{Sino-Tibetan}
\def\langnames@fams@wals@yph{Sino-Tibetan}
\def\langnames@fams@wals@ypm{Sino-Tibetan}
\def\langnames@fams@wals@ypn{Sino-Tibetan}
\def\langnames@fams@wals@ypo{Sino-Tibetan}
\def\langnames@fams@wals@ypp{Sino-Tibetan}
\def\langnames@fams@wals@ypz{Sino-Tibetan}
\def\langnames@fams@wals@yra{Isolate}
\def\langnames@fams@wals@yrb{Yareban}
\def\langnames@fams@wals@yre{Mande}
\def\langnames@fams@wals@yrk{Uralic}
\def\langnames@fams@wals@yrl{Tupian}
\def\langnames@fams@wals@yrn{Tai-Kadai}
\def\langnames@fams@wals@yro{Yanomamic}
\def\langnames@fams@wals@yrw{Nuclear Trans New Guinea}
\def\langnames@fams@wals@ysd{Sino-Tibetan}
\def\langnames@fams@wals@ysg{Sino-Tibetan}
\def\langnames@fams@wals@ysl{Sign Language}
\def\langnames@fams@wals@ysm{Sign Language}
\def\langnames@fams@wals@ysn{Sino-Tibetan}
\def\langnames@fams@wals@yso{Sino-Tibetan}
\def\langnames@fams@wals@ysr{Eskimo-Aleut}
\def\langnames@fams@wals@yss{Sepik}
\def\langnames@fams@wals@ysy{Sino-Tibetan}
\def\langnames@fams@wals@yta{Sino-Tibetan}
\def\langnames@fams@wals@ytl{Sino-Tibetan}
\def\langnames@fams@wals@ytp{Sino-Tibetan}
\def\langnames@fams@wals@ytw{Nuclear Trans New Guinea}
\def\langnames@fams@wals@yua{Mayan}
\def\langnames@fams@wals@yub{Pama-Nyungan}
\def\langnames@fams@wals@yuc{Isolate}
\def\langnames@fams@wals@yud{Afro-Asiatic}
\def\langnames@fams@wals@yue{Sino-Tibetan}
\def\langnames@fams@wals@yuf{Cochimi-Yuman}
\def\langnames@fams@wals@yug{Yeniseian}
\def\langnames@fams@wals@yui{Tucanoan}
\def\langnames@fams@wals@yuj{Pauwasi}
\def\langnames@fams@wals@yuk{Yuki-Wappo}
\def\langnames@fams@wals@yul{Central Sudanic}
\def\langnames@fams@wals@yum{Cochimi-Yuman}
\def\langnames@fams@wals@yun{Atlantic-Congo}
\def\langnames@fams@wals@yup{Cariban}
\def\langnames@fams@wals@yuq{Tupian}
\def\langnames@fams@wals@yur{Algic}
\def\langnames@fams@wals@yut{Nuclear Trans New Guinea}
\def\langnames@fams@wals@yuw{Nuclear Trans New Guinea}
\def\langnames@fams@wals@yux{Yukaghir}
\def\langnames@fams@wals@yuy{Mongolic-Khitan}
\def\langnames@fams@wals@yuz{Isolate}
\def\langnames@fams@wals@yva{Yawa-Saweru}
\def\langnames@fams@wals@yvt{Arawakan}
\def\langnames@fams@wals@ywa{Sepik}
\def\langnames@fams@wals@ywg{Pama-Nyungan}
\def\langnames@fams@wals@ywl{Sino-Tibetan}
\def\langnames@fams@wals@ywn{Pano-Tacanan}
\def\langnames@fams@wals@ywq{Sino-Tibetan}
\def\langnames@fams@wals@ywr{Nyulnyulan}
\def\langnames@fams@wals@ywt{Sino-Tibetan}
\def\langnames@fams@wals@ywu{Sino-Tibetan}
\def\langnames@fams@wals@yww{Pama-Nyungan}
\def\langnames@fams@wals@yxm{Pama-Nyungan}
\def\langnames@fams@wals@yyu{Nuclear Torricelli}
\def\langnames@fams@wals@yyz{Sino-Tibetan}
\def\langnames@fams@wals@yzg{Tai-Kadai}
\def\langnames@fams@wals@yzk{Sino-Tibetan}
\def\langnames@fams@wals@zaa{Otomanguean}
\def\langnames@fams@wals@zab{Otomanguean}
\def\langnames@fams@wals@zac{Otomanguean}
\def\langnames@fams@wals@zad{Otomanguean}
\def\langnames@fams@wals@zae{Otomanguean}
\def\langnames@fams@wals@zaf{Otomanguean}
\def\langnames@fams@wals@zag{Saharan}
\def\langnames@fams@wals@zah{Afro-Asiatic}
\def\langnames@fams@wals@zai{Otomanguean}
\def\langnames@fams@wals@zaj{Atlantic-Congo}
\def\langnames@fams@wals@zak{Atlantic-Congo}
\def\langnames@fams@wals@zal{Sino-Tibetan}
\def\langnames@fams@wals@zam{Otomanguean}
\def\langnames@fams@wals@zao{Otomanguean}
\def\langnames@fams@wals@zaq{Otomanguean}
\def\langnames@fams@wals@zar{Otomanguean}
\def\langnames@fams@wals@zas{Otomanguean}
\def\langnames@fams@wals@zat{Otomanguean}
\def\langnames@fams@wals@zau{Sino-Tibetan}
\def\langnames@fams@wals@zav{Otomanguean}
\def\langnames@fams@wals@zaw{Otomanguean}
\def\langnames@fams@wals@zax{Otomanguean}
\def\langnames@fams@wals@zay{Ta-Ne-Omotic}
\def\langnames@fams@wals@zaz{Afro-Asiatic}
\def\langnames@fams@wals@zbc{Austronesian}
\def\langnames@fams@wals@zbe{Austronesian}
\def\langnames@fams@wals@zbl{Artificial Language}
\def\langnames@fams@wals@zbt{Austronesian}
\def\langnames@fams@wals@zbu{Afro-Asiatic}
\def\langnames@fams@wals@zbw{Austronesian}
\def\langnames@fams@wals@zca{Otomanguean}
\def\langnames@fams@wals@zch{Tai-Kadai}
\def\langnames@fams@wals@zdj{Atlantic-Congo}
\def\langnames@fams@wals@zea{Indo-European}
\def\langnames@fams@wals@zeg{Austronesian}
\def\langnames@fams@wals@zeh{Tai-Kadai}
\def\langnames@fams@wals@zen{Afro-Asiatic}
\def\langnames@fams@wals@zga{Atlantic-Congo}
\def\langnames@fams@wals@zgb{Tai-Kadai}
\def\langnames@fams@wals@zgm{Tai-Kadai}
\def\langnames@fams@wals@zgn{Tai-Kadai}
\def\langnames@fams@wals@zgr{Austronesian}
\def\langnames@fams@wals@zhb{Sino-Tibetan}
\def\langnames@fams@wals@zhd{Tai-Kadai}
\def\langnames@fams@wals@zhi{Atlantic-Congo}
\def\langnames@fams@wals@zhn{Tai-Kadai}
\def\langnames@fams@wals@zhw{Atlantic-Congo}
\def\langnames@fams@wals@zia{Nuclear Trans New Guinea}
\def\langnames@fams@wals@zib{Sign Language}
\def\langnames@fams@wals@zik{Anim}
\def\langnames@fams@wals@zil{Mande}
\def\langnames@fams@wals@zim{Afro-Asiatic}
\def\langnames@fams@wals@zin{Atlantic-Congo}
\def\langnames@fams@wals@ziw{Atlantic-Congo}
\def\langnames@fams@wals@ziz{Afro-Asiatic}
\def\langnames@fams@wals@zka{Austronesian}
\def\langnames@fams@wals@zkg{Unclassifiable}
\def\langnames@fams@wals@zkk{Isolate}
\def\langnames@fams@wals@zko{Yeniseian}
\def\langnames@fams@wals@zkp{Nuclear-Macro-Je}
\def\langnames@fams@wals@zkr{Sino-Tibetan}
\def\langnames@fams@wals@zkt{Mongolic-Khitan}
\def\langnames@fams@wals@zku{Pama-Nyungan}
\def\langnames@fams@wals@zla{Atlantic-Congo}
\def\langnames@fams@wals@zlj{Tai-Kadai}
\def\langnames@fams@wals@zlm{Austronesian}
\def\langnames@fams@wals@zln{Tai-Kadai}
\def\langnames@fams@wals@zlq{Tai-Kadai}
\def\langnames@fams@wals@zma{Western Daly}
\def\langnames@fams@wals@zmb{Atlantic-Congo}
\def\langnames@fams@wals@zmc{Pama-Nyungan}
\def\langnames@fams@wals@zmd{Western Daly}
\def\langnames@fams@wals@zme{Giimbiyu}
\def\langnames@fams@wals@zmf{Atlantic-Congo}
\def\langnames@fams@wals@zmg{Western Daly}
\def\langnames@fams@wals@zmh{Baining}
\def\langnames@fams@wals@zmi{Austronesian}
\def\langnames@fams@wals@zmj{Western Daly}
\def\langnames@fams@wals@zmk{Pama-Nyungan}
\def\langnames@fams@wals@zml{Eastern Daly}
\def\langnames@fams@wals@zmm{Western Daly}
\def\langnames@fams@wals@zmn{Atlantic-Congo}
\def\langnames@fams@wals@zmo{Eastern Jebel}
\def\langnames@fams@wals@zmp{Atlantic-Congo}
\def\langnames@fams@wals@zmq{Atlantic-Congo}
\def\langnames@fams@wals@zmr{Western Daly}
\def\langnames@fams@wals@zms{Atlantic-Congo}
\def\langnames@fams@wals@zmt{Western Daly}
\def\langnames@fams@wals@zmu{Pama-Nyungan}
\def\langnames@fams@wals@zmv{Pama-Nyungan}
\def\langnames@fams@wals@zmw{Atlantic-Congo}
\def\langnames@fams@wals@zmx{Atlantic-Congo}
\def\langnames@fams@wals@zmy{Western Daly}
\def\langnames@fams@wals@zmz{Atlantic-Congo}
\def\langnames@fams@wals@zna{Atlantic-Congo}
\def\langnames@fams@wals@zne{Atlantic-Congo}
\def\langnames@fams@wals@zng{Austroasiatic}
\def\langnames@fams@wals@znk{Unattested}
\def\langnames@fams@wals@zns{Afro-Asiatic}
\def\langnames@fams@wals@zoc{Mixe-Zoque}
\def\langnames@fams@wals@zoh{Mixe-Zoque}
\def\langnames@fams@wals@zom{Sino-Tibetan}
\def\langnames@fams@wals@zoo{Otomanguean}
\def\langnames@fams@wals@zoq{Mixe-Zoque}
\def\langnames@fams@wals@zor{Mixe-Zoque}
\def\langnames@fams@wals@zos{Mixe-Zoque}
\def\langnames@fams@wals@zpa{Otomanguean}
\def\langnames@fams@wals@zpb{Otomanguean}
\def\langnames@fams@wals@zpc{Otomanguean}
\def\langnames@fams@wals@zpd{Otomanguean}
\def\langnames@fams@wals@zpe{Otomanguean}
\def\langnames@fams@wals@zpf{Otomanguean}
\def\langnames@fams@wals@zpg{Otomanguean}
\def\langnames@fams@wals@zph{Otomanguean}
\def\langnames@fams@wals@zpi{Otomanguean}
\def\langnames@fams@wals@zpj{Otomanguean}
\def\langnames@fams@wals@zpk{Otomanguean}
\def\langnames@fams@wals@zpl{Otomanguean}
\def\langnames@fams@wals@zpm{Otomanguean}
\def\langnames@fams@wals@zpn{Otomanguean}
\def\langnames@fams@wals@zpo{Otomanguean}
\def\langnames@fams@wals@zpp{Otomanguean}
\def\langnames@fams@wals@zpq{Otomanguean}
\def\langnames@fams@wals@zpr{Otomanguean}
\def\langnames@fams@wals@zps{Otomanguean}
\def\langnames@fams@wals@zpt{Otomanguean}
\def\langnames@fams@wals@zpu{Otomanguean}
\def\langnames@fams@wals@zpv{Otomanguean}
\def\langnames@fams@wals@zpw{Otomanguean}
\def\langnames@fams@wals@zpx{Otomanguean}
\def\langnames@fams@wals@zpy{Otomanguean}
\def\langnames@fams@wals@zpz{Otomanguean}
\def\langnames@fams@wals@zqe{Tai-Kadai}
\def\langnames@fams@wals@zrn{Afro-Asiatic}
\def\langnames@fams@wals@zro{Zaparoan}
\def\langnames@fams@wals@zrs{Mairasic}
\def\langnames@fams@wals@zsa{Austronesian}
\def\langnames@fams@wals@zsl{Sign Language}
\def\langnames@fams@wals@zsm{Austronesian}
\def\langnames@fams@wals@zsu{Austronesian}
\def\langnames@fams@wals@zte{Otomanguean}
\def\langnames@fams@wals@ztg{Otomanguean}
\def\langnames@fams@wals@ztl{Otomanguean}
\def\langnames@fams@wals@ztm{Otomanguean}
\def\langnames@fams@wals@ztn{Otomanguean}
\def\langnames@fams@wals@ztp{Otomanguean}
\def\langnames@fams@wals@ztq{Otomanguean}
\def\langnames@fams@wals@zts{Otomanguean}
\def\langnames@fams@wals@ztt{Otomanguean}
\def\langnames@fams@wals@ztu{Otomanguean}
\def\langnames@fams@wals@ztx{Otomanguean}
\def\langnames@fams@wals@zty{Otomanguean}
\def\langnames@fams@wals@zua{Afro-Asiatic}
\def\langnames@fams@wals@zuh{Nuclear Trans New Guinea}
\def\langnames@fams@wals@zul{Atlantic-Congo}
\def\langnames@fams@wals@zum{Indo-European}
\def\langnames@fams@wals@zun{Isolate}
\def\langnames@fams@wals@zuy{Afro-Asiatic}
\def\langnames@fams@wals@zwa{Afro-Asiatic}
\def\langnames@fams@wals@zyb{Tai-Kadai}
\def\langnames@fams@wals@zyg{Tai-Kadai}
\def\langnames@fams@wals@zyj{Tai-Kadai}
\def\langnames@fams@wals@zyn{Tai-Kadai}
\def\langnames@fams@wals@zyp{Sino-Tibetan}
\def\langnames@fams@wals@zzj{Tai-Kadai}
%
}
\DeclareOption{none}{%
  \def\langnames@cs@prefix{none}%
}
\DeclareOption{native}{%
  \def\langnames@langs@native@wals@deu{Deutsch}
\def\langnames@langs@native@wals@jpn{日本語}
\def\langnames@langs@native@wals@mar{मराठी}
%
  \def\langnames@langs@native@glottolog@mar{मराठी}
\def\langnames@langs@native@glottolog@jpn{日本語}
\def\langnames@langs@native@glottolog@deu{Deutsch}%
}
%    \end{macrocode}
% This line of code simply tells the package to process the options specified above.
%    \begin{macrocode}
\ProcessOptions\relax
%    \end{macrocode}
% \subsection{Macro definitions}
% \begin{macro}{\lname}
% This macro takes the value specified in its mandatory argument to call its corresponding macro from the |names| set, and prints it. This is achieved through the use of the |\csname| and |\endcsname| macros.
%    \begin{macrocode}
\newcommand*{\lname}[1]{%
  {%
    \csname langnames@langs@\langnames@cs@prefix @#1\endcsname
  }%
}
%    \end{macrocode}
% \end{macro}
% \begin{macro}{\liso}
%   This macro takes, like |\lname|, the value from the |names| set from the argument input, and prints the name as well as the ISO 639-3 code (which is the argument verbatim) between parenthesis.
%    \begin{macrocode}
\newcommand*{\liso}[1]{%
  {%
    {\csname langnames@langs@\langnames@cs@prefix @#1\endcsname}
    (ISO 639-3: #1)%
  }%
}
%    \end{macrocode}
% \end{macro}
% \begin{macro}{\lfam}
%   This macro, like |\lname| and |\liso|, calls the macro from the |names| set corresponding to the input of the mandatory argument, plus the macro from the |fams| set which gives it its genetic affiliation, which is printed between parenthesis.
%    \begin{macrocode}
\newcommand*{\lfam}[1]{%
  {%
    {\csname langnames@langs@\langnames@cs@prefix @#1\endcsname}
    (\csname langnames@fams@\langnames@cs@prefix @#1\endcsname)%
  }%
}
%    \end{macrocode}
% \end{macro}
% \begin{macro}{\newlang}
%   This macro defines new macros for a language from three mandatory arguments. The first argument of |\newlang|\marg{code} defines the code which serves as identifier (the ISO code in the case of pre-defined key-value pairs). The second argument \marg{name} defines the printed name of the language. The third argument \marg{family} defines the family to which the language belongs.
%    \begin{macrocode}
\newcommand*{\newlang}[3]{%
  \expandafter\def\csname langnames@langs@none@#1\endcsname{#2}%
  \expandafter\def\csname langnames@fams@none@#1\endcsname{#3}%
}
%    \end{macrocode}
% \end{macro}
% \begin{macro}{\renewlang}
%   The following code is used to develop the \cs{renewlang} command.
%    \begin{macrocode}
\newcommand*{\renewlang}[4]{%
  \expandafter\def\csname langnames@langs@#1@#2\endcsname{#3}%
  \expandafter\def\csname langnames@fams@#1@#2\endcsname{#4}%
}
%    \end{macrocode}
% \end{macro}
% \begin{macro}{\newlangnative}
%   The following code is used to develop the \cs{newlangnative} command.
%    \begin{macrocode}
\newcommand*{\newlangnative}[4]{%
  \expandafter\def\csname langnames@langs@native@#1@#2\endcsname{#3}%
}
%    \end{macrocode}
% \end{macro}
% \begin{macro}{\langnative}
%   The following code is used to develop the \cs{langnative} command.
%    \begin{macrocode}
\newcommand*{\langnative}[1]{%
  {%
    \csname langnames@langs@native@\langnames@cs@prefix @#1\endcsname
  }%
}
%    \end{macrocode}
% \end{macro}
% \begin{macro}{\changetonone,\changetowals,\changetoglottolog}
%   With the following code these three additional macros are defined which change the macro-set locally.
%    \begin{macrocode}
\newcommand*{\changetonone}{%
  \def\langnames@cs@prefix{none}%
}
\newcommand*{\changetowals}{%
  \def\langnames@cs@prefix{wals}%
}
\newcommand*{\changetoglottolog}{%
  \def\langnames@cs@prefix{glottolog}%
}
%    \end{macrocode}
% \end{macro}
% This package demands the user to select one package option from the available ones compulsorily. The mechanism of the package might fail if a user does not pass any option. Hence the package checks whether it is passed or not just before the beginning of the document with the following code. If no valid option is passed, an error is issued and the package defaults to the |none| set.
%    \begin{macrocode}
\def\ssp{\space\space\space\space\space\space}
\AddToHook{begindocument/before}{%
  \ifdefined\langnames@cs@prefix
  \else
    \PackageError{langnames}{%
      You haven't passed any option to `langnames'. Can't\MessageBreak
      proceed. Please pass one from the list given below.\MessageBreak
      ---------------------------------------------------\MessageBreak
      1. glottolog: Glottolog\MessageBreak
      2. wals:\ssp World Atlas of Languages\MessageBreak
      3. none:\ssp Your own list.\MessageBreak
      ---------------------------------------------------\MessageBreak
      Refer to the documentation for more details.\MessageBreak
      At the moment I will default to option `none'%
    }%
  \fi
}
%    \end{macrocode}
%
% \section{License}
%
% Copyright (C) 2022, 2023 Alejandro García Matarredona, {\mrtxt निरंजन}
% 
% This file may be distributed and/or modified under the conditions of the LaTeX\ Project Public License, either version 1.3 of this license or (at your option) any later version. The latest version of this license is in:\par
%
% \quad http://www.latex-project.org/lppl.txt\par
%
% and version 1.3 or later is part of all distributions of \LaTeX\ version 2005/12/01 or later.
% 
% \Finale
\endinput